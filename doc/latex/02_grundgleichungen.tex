\section{Weber-Gravitation als Alternative zur ART}
Die allgemeine Relativitätstheorie (ART) gilt als der Goldstandard der modernen Astrophysik, allerdings werden bestimmte Aspekte dieser Theorie
nicht objektiv betrachtet. Die ART überzeugt durch die Fähigkeit die Merkur-Periheldrehung vorhersagen zu können, aber auch durch die Vorhersage
der Gravitationswellen. Das sind große Leistungen dieser Gravitationstheorie.

Auf der anderen Seite liefert sie unphysikalische Ergebnisse für schwarze Löcher und für galaktische Skalen. Schwarze Löcher werden als Singularitäten
dargestellt, wobei davon ausgegangen werden muss, dass die gravitativen Verhältnisse in der Nähe dieser Singularitäten ebenfalls ungenau sein müssen. Die
Rotationskurven von Galaxien werden nicht korrekt Vorhergesagt, weswegen die ART \enquote{dunkle Materie} benötigt.

Eine genauere Betrachtung der Periheldrehung des Merkurs zeigt, dass auch hier die ART nicht wirklich exakt ist. Die Vorhersage der ART liefert 42.98", wobei
der tatsächliche Messwert kleiner ist.

\section{Die Überlegenheit der Weber-Gravitation}

\subsection{Periheldrehung des Merkurs}
Die \textbf{beobachtete} Periheldrehung von 574,10''/Jh. setzt sich zusammen aus:
\begin{itemize}
\item Newton'schen Störungen: 532,13''
\item Relativistischem Anteil: $\sim$42,8''
\end{itemize}

\textbf{Vorhersagen:}
\begin{itemize}
\item ART: 42,98'' (0,28'' Überschätzung)
\item WG (Simulation): $\sim$42,7''
\end{itemize}

\textbf{Die WG liegt näher am Messwert, weil:}
\begin{itemize}
\item Die $v^2/c^2$-Terme Geschwindigkeitseffekte exakter erfassen
\item Keine Singularitätsnähe-Approximation wie in der ART
\end{itemize}

\subsection{Abweichungen in Planetenbahnen}
\textbf{Numerische Simulation zeigt:}
\begin{itemize}
\item Alle Planeten umlaufen $\sim$0,3\% schneller als ART-Vorhersagen
\item Stärkster Effekt bei inneren Planeten ($\propto 1/r$)
\item Analog zum galaktischen Rotationskurven-Problem
\end{itemize}

Gleichung \textbf{\ref{eq:weber_gravitationskraft}} (Physikalischer Ursprung) führt zu:
\begin{itemize}
\item Zusätzlicher anziehender Komponente
\item Kürzeren Umlaufzeiten
\end{itemize}

\subsection{Konsequenzen}
Die WG erklärt konsistent:
\begin{itemize}
\item Merkur-Periheldrehung (42,7'' vs 42,98'')
\item Planetenbahnabweichungen (+0,3\%)
\item Galaktische Rotationskurven
\end{itemize}

ohne benötigte Zusatzannahmen wie:
\begin{itemize}
\item Raumzeitkrümmung (ART)
\item Dunkle Materie ($\Lambda$CDM)
\end{itemize}

\subsection{Experimentelle Verifikation}
Testbare Vorhersagen:
\begin{itemize}
\item Präzisionsmessung innerer Planetenbahnen
\item Asteroiden mit hoher Exzentrizität ($e \approx 0,5-0,9$)
\item Detektion von Geschwindigkeitsabhängigkeiten
\end{itemize}

\section{Grundgleichungen der Weber-Gravitation}
\subsection*{Weber-Gravitations Gleichung}
\begin{equation}\label{eq:weber_gravitationskraft}
\mathbf{F} = -\frac{GMm}{r^2}\left(1 - \frac{\dot{r}^2}{c^2} + \frac{r\ddot{r}}{2c^2}\right)\mathbf{\hat{r}}
\end{equation}

\subsection{Vorteile der Weber-Gravitation}
\begin{itemize}
\item \textbf{Keine Singularitäten} – Kollaps stoppt bei $r \approx L_p$
\item \textbf{Keine dunkle Materie} – Geschwindigkeitsabhängigkeit erklärt Rotationskurven
\item \textbf{Vereinheitlichung} – Elektromagnetismus und Gravitation nutzen dieselbe Kraftstruktur
\end{itemize}

\subsection*{Bewegungsgleichung in Polarkoordinaten}
\begin{equation}\label{eq:weber_bewegungsgleichung}
\mathbf{a} = \left(\ddot{r} - r\dot{\varphi}^2\right)\mathbf{\hat{r}} + \left(r\ddot{\varphi} + 2\dot{r}\dot{\varphi}\right)\mathbf{\hat{\varphi}} = -\frac{GM}{r^2}\left(1 - \frac{\dot{r}^2}{c^2} + \frac{r\ddot{r}}{2c^2}\right)\mathbf{\hat{r}}
\end{equation}

\subsection*{Variablenbeschreibung}
\begin{itemize}[leftmargin=*,noitemsep]
    \item $\mathbf{F}$: Gravitationskraftvektor (Weber-Kraft) [N]
    \item $\mathbf{a}$: Beschleunigungsvektor [m/s²]
    \item $G$: Gravitationskonstante [m³/kg/s²]
    \item $M$, $m$: Massen der wechselwirkenden Körper [kg]
    \item $r$: Abstand zwischen den Massenschwerpunkten [m]
    \item $\dot{r} = \frac{dr}{dt}$: Radiale Relativgeschwindigkeit [m/s]
    \item $\ddot{r} = \frac{d^2r}{dt^2}$: Radiale Relativbeschleunigung [m/s²]
    \item $c$: Lichtgeschwindigkeit [m/s]
    \item $\varphi$: Azimutwinkel [rad]
    \item $\dot{\varphi} = \frac{d\varphi}{dt}$: Winkelgeschwindigkeit [rad/s]
    \item $\ddot{\varphi} = \frac{d^2\varphi}{dt^2}$: Winkelbeschleunigung [rad/s²]
    \item $\mathbf{\hat{r}}$: Radialer Einheitsvektor (zeigt von $M$ zu $m$)
    \item $\mathbf{\hat{\varphi}}$: Azimutaler Einheitsvektor (senkrecht zu $\mathbf{\hat{r}}$)
\end{itemize}

\subsection*{Physikalische Interpretation}
\begin{itemize}[leftmargin=*,noitemsep]
    \item Der Term $-\frac{GMm}{r^2}$ entspricht der klassischen Newton'schen Gravitation
    \item $\frac{\dot{r}^2}{c^2}$: Relativistische Korrektur für radiale Bewegung
    \item $\frac{r\ddot{r}}{2c^2}$: Korrektur für radiale Beschleunigung
    \item $r\dot{\varphi}^2$: Zentripetalbeschleunigung
    \item $2\dot{r}\dot{\varphi}$: Coriolis-Term
\end{itemize}
