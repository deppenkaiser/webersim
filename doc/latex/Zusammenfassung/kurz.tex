\documentclass[11pt, a4paper]{article}
\usepackage[utf8]{inputenc}
\usepackage[T1]{fontenc}
\usepackage[ngerman]{babel}
\usepackage{amsmath, amssymb, amsthm}
\usepackage{geometry}
\geometry{left=2.5cm, right=2.5cm, top=2.5cm, bottom=2.5cm}
\usepackage{graphicx}
\usepackage{booktabs}
\usepackage{hyperref}
\hypersetup{
    colorlinks=true,
    linkcolor=blue,
    filecolor=magenta,      
    urlcolor=cyan,
}

\title{Weber-De Broglie-Bohm-Theorie (WDBT) \\ Eine vollständige und vereinheitlichte Darstellung}
\author{Zusammengestellt aus den Arbeiten von Michael Czybor}
\date{18. August 2025}

\begin{document}

\maketitle

\begin{abstract}
Diese Arbeit stellt eine vollständige und systematische Vereinheitlichung der \textbf{Weber-De Broglie-Bohm-Theorie (WDBT)} dar. Die WDBT vereint die \textbf{Weber-Elektrodynamik (WED)}, eine erweiterte \textbf{Weber-Gravitation (WG)} und die \textbf{De-Broglie-Bohm-Theorie (DBT)} zu einer feldlosen, nicht-lokalen und deterministischen Ur-Theorie der Physik. Ziel ist die \textbf{Emergenz} der gesamten etablierten Physik – von der Speziellen Relativitätstheorie über die Maxwell-Gleichungen und Quantenmechanik bis hin zur Allgemeinen Relativitätstheorie und Quantenelektrodynamik – aus wenigen grundlegenden Prinzipien direkter Teilchenwechselwirkungen und eines universellen Quantenpotentials. Die Theorie vermeidet konzeptionelle Probleme der Standardphysik (Singularitäten, Dunkle Materie, Messproblem) und macht eine Reihe testbarer Vorhersagen.
\end{abstract}

\tableofcontents

\section{Einleitung und Überblick}
\label{sec:einleitung}

Die \textbf{Weber-De Broglie-Bohm-Theorie (WDBT)} stellt ein radikal alternatives Paradigma zur etablierten Physik dar. Ihr Kernanliegen ist die Ableitung aller physikalischen Gesetze aus einer fundamentalen, feldlosen Theorie direkter Wechselwirkungen zwischen Teilchen, ergänzt um das Konzept der Führungswelle bzw. des Quantenpotentials aus der Bohm'schen Mechanik. Die WDBT integriert drei historische Ansätze zu einem kohärenten Rahmen:

\begin{enumerate}
    \item \textbf{WED:} Beschreibt elektromagnetische Wechselwirkungen durch direkte, geschwindigkeits- und beschleunigungsabhängige Kräfte zwischen Ladungen, ohne das Konzept elektromagnetischer Felder zu benötigen.
    \item \textbf{WG:} Erweitert das Newtonsche Gravitationsgesetz um geschwindigkeits- und beschleunigungsabhängige Terme, ähnlich der WED. Sie dient als Basis für die Emergenz der Relativitätstheorie.
    \item \textbf{DBT:} Bietet eine deterministische Interpretation der Quantenmechanik, in der Teilchen wohldefinierte Trajektorien besitzen, die durch eine Führungswelle (\glqq pilot wave\grqq) geleitet werden, deren Effekt durch ein \textbf{Quantenpotential} $Q$ beschrieben wird.
\end{enumerate}

Die WDBT postuliert, dass die beobachtete Physik (Relativität, Felder, Quantenphänomene) aus dieser fundamentalen Ebene \textbf{emergent} hervorgeht. Raum und Zeit selbst werden als emergent aus einer tieferen, möglicherweise diskreten oder fraktalen Struktur mit einer Dimension von $D \approx 2.71$ betrachtet.

\section{Grundlegende Gleichungen und Konzepte}
\label{sec:grundlagen}

\subsection{Weber-Elektrodynamik (WED)}
\label{subsec:wed}

Die fundamentale Wechselwirkung in der WED ist die \textbf{Weber-Kraft} zwischen zwei Ladungen $q_1$ und $q_2$. Sie ist abhängig von deren Relativgeschwindigkeit $\dot{r}$ und Relativbeschleunigung $\ddot{r}$.

\paragraph{Skalare Form (Zentralkraftnäherung):}
\begin{equation}
\label{eq:weber_skalar}
F_{12} = \frac{q_1 q_2}{4\pi\epsilon_0 r^2} \left[ 1 - \frac{\dot{r}^2}{c^2} + \beta \frac{r \ddot{r}}{c^2} \right], \quad \text{mit } \beta = 2
\end{equation}

\paragraph{Vektorielle Form (allgemeiner Kraftvektor):}
\begin{equation}
\label{eq:weber_vektor}
\vec{F}_{12} = \frac{q_1 q_2}{4\pi\epsilon_0 r^2} \left\{ \left[ 1 - \frac{v^2}{c^2} + \frac{2r (\hat{r} \cdot \vec{a})}{c^2} \right] \hat{r} + \frac{2 (\hat{r} \cdot \vec{v})}{c^2} \vec{v} \right\}
\end{equation}
wobei $\vec{v}$ und $\vec{a}$ die Relativgeschwindigkeit und -beschleunigung bezeichnen und $\hat{r}$ der Einheitsvektor in Richtung von $q_1$ nach $q_2$ ist.

\subsection{Weber-Gravitation (WG)}
\label{subsec:wg}

Analog zur WED wird das Newtonsche Gravitationsgesetz um geschwindigkeits- und beschleunigungsabhängige Terme erweitert. Der Parameter $\beta$ unterscheidet sich je nach wechselwirkenden Objekten.

\paragraph{Gravitationskraft:}
\begin{equation}
\label{eq:weber_grav}
\vec{F}_{\text{WG}} = -\frac{G M m}{r^2} \left( 1 - \frac{\dot{r}^2}{c^2} + \beta \frac{r \ddot{r}}{c^2} \right) \hat{\vec{r}}
\end{equation}
\begin{align*}
\text{mit } \beta &= 0.5 \quad \text{für Massen} \\
\beta &= 1.0 \quad \text{für Photonen}
\end{align*}

\subsection{De-Broglie-Bohm-Theorie (DBT) und Quantenpotential}
\label{subsec:dbt}

In der DBT wird die Wellenfunktion $\Psi(\vec{x}, t)$ in Polardarstellung geschrieben:
\[
\Psi(\vec{x}, t) = R(\vec{x}, t) \, e^{i S(\vec{x}, t) / \hbar}
\]
wobei $R$ die Amplitude und $S$ die Phase ist. Die Wahrscheinlichkeitsdichte ist $\rho = R^2$.

\paragraph{Quantenpotential:}
\begin{equation}
\label{eq:quantenpotential}
Q = -\frac{\hbar^2}{2m} \frac{\nabla^2 R}{R} = -\frac{\hbar^2}{2m} \frac{\nabla^2 \sqrt{\rho}}{\sqrt{\rho}}
\end{equation}

\paragraph{Führungsgleichung (Guidance Equation):}
Die Geschwindigkeit eines Teilchens wird durch den Gradienten der Phase $S$ bestimmt:
\begin{equation}
\label{eq:fuehrungsgleichung}
\vec{v} = \frac{1}{m} \vec{\nabla} S = \frac{\hbar}{m} \, \operatorname{Im} \left( \frac{\vec{\nabla} \Psi}{\Psi} \right)
\end{equation}

\paragraph{Fundamentale Bewegungsgleichung:}
Die Newtonsche Bewegungsgleichung wird um das Quantenpotential erweitert:
\begin{equation}
\label{eq:bewegung_q}
m \frac{d^2 \vec{x}}{dt^2} = -\vec{\nabla} V - \vec{\nabla} Q
\end{equation}
wobei $V$ das klassische Potential (z.B. aus WED oder WG) darstellt.

\subsection{Kraftdichte in kontinuierlichen Medien}
\label{subsec:kraftdichte}

Für die Beschreibung von Plasmen oder anderen kontinuierlichen Medien muss die Weber-Kraft gemittelt werden. Die Kraftdichte ergibt sich aus dem Integral über die Paarkräfte, gewichtet mit der Paarkorrelationsfunktion $g(\vec{r})$:
\begin{equation}
\label{eq:kraftdichte}
\vec{f}_{\text{Weber}} = n_e n_i \int d^3r \, \vec{F}_{12}(\vec{r}) \, g(\vec{r})
\end{equation}
wobei $n_e$ und $n_i$ die Dichten der wechselwirkenden Teilchen (z.B. Elektronen und Ionen) sind.

\subsection{Fraktale Raumdimension}
\label{subsec:fraktal}

Die WDBT postuliert eine fundamentale fraktale Struktur des Raums mit der Dimension:
\begin{equation}
\label{eq:fraktal_dim}
D = \frac{\ln 20}{\ln(2 + \phi)} \approx 2.71, \quad \text{mit } \phi = \frac{1 + \sqrt{5}}{2} \quad \text{(goldener Schnitt)}
\end{equation}
Diese Dimension erscheint in Skalierungsgesetzen, z.B. für die Stromdichte in Plasmen:
\begin{equation}
\label{eq:fraktal_strom}
j(r) \propto r^{D-3} \approx r^{-0.29}
\end{equation}

\section{Emergenz der etablierten Physik}
\label{sec:emergenz}

Ein zentrales Anliegen der WDBT ist zu zeigen, wie die bekannten physikalischen Theorien aus ihren fundamentalen Gleichungen hervorgehen.

\subsection{Emergenz der Speziellen Relativitätstheorie (SRT)}
\label{subsec:emergenz_srt}

\paragraph{Grundpostulat:} Die relativistische Physik emergiert aus der geschwindigkeitsabhängigen WG.

\paragraph{Systematische Herleitung:}
\begin{enumerate}
    \item \textbf{Ausgangspunkt:} Weber-Gravitationskraft (Gl. \ref{eq:weber_grav}) mit $\beta_m = 0.5$.
    \item \textbf{Verallgemeinertes Potential:} Über ein verallgemeinertes Potential $U_{\text{WG}}(r, \dot{r})$ und Mittelung über den kosmischen Hintergrund wird die relativistische Energie-Impuls-Beziehung hergeleitet:
    \begin{equation}
    \label{eq:energie_impuls}
    E^2 = (p c)^2 + (m c^2)^2
    \end{equation}
    \item \textbf{Definition relativistischer Größen:} Aus Gl. (\ref{eq:energie_impuls}) werden die bekannten Größen \textit{definiert}:
    \begin{align}
    E &= \gamma m c^2 \\
    \vec{p} &= \gamma m \vec{v} \\
    \gamma &= \frac{1}{\sqrt{1 - v^2/c^2}}
    \end{align}
    \item \textbf{Emergenz relativistischer Effekte:} Aus diesen Definitionen folgen:
    \begin{itemize}
        \item Zeitdilatation: $\Delta t = \gamma \Delta t_0$
        \item Längenkontraktion: $L = L_0 / \gamma$
        \item Relativistische Dynamik: $\vec{F} = \frac{d}{dt}(\gamma m \vec{v})$
    \end{itemize}
    \item \textbf{Schlussfolgerung:} Die Lorentz-Invarianz ist keine fundamentale Eigenschaft der Raumzeit, sondern ein emergentes Phänomen. Die Lichtgeschwindigkeit $c$ ist die Grenzgeschwindigkeit der Weber-Wechselwirkung.
\end{enumerate}

\subsection{Emergenz der Maxwell-Gleichungen}
\label{subsec:emergenz_maxwell}

\paragraph{Grundpostulat:} Die klassische Elektrodynamik emergiert aus der WED durch Kontinuumslimes und Mittelung.

\paragraph{Systematische Herleitung:}
\begin{enumerate}
    \item \textbf{Ausgangspunkt:} Vektorielle Weber-Kraft (Gl. \ref{eq:weber_vektor}) zwischen Ladungen.
    \item \textbf{Superposition:} Die Gesamtkraft auf eine Testladung $q$ im Feld vieler Quellenladungen $q_i$ ist die Summe aller Einzelkräfte:
    \[
    \vec{F}_{\text{ges}} = q \sum_i \frac{q_i}{4\pi\epsilon_0 r_i^2} \left\{ \left[ 1 - \frac{v_i^2}{c^2} + \frac{2r_i (\hat{r}_i \cdot \vec{a}_i)}{c^2} \right] \hat{r}_i + \frac{2 (\hat{r}_i \cdot \vec{v}_i)}{c^2} \vec{v}_i \right\}
    \]
    \item \textbf{Definition effektiver Felder:} Durch Koeffizientenvergleich mit der Lorentz-Kraft $\vec{F} = q(\vec{E} + \vec{v} \times \vec{B})$ werden die Felder $\vec{E}$ und $\vec{B}$ \textit{definiert}:
    \begin{align}
    \vec{E} &= \sum_i \frac{q_i}{4\pi\epsilon_0 r_i^2} \left[ 1 - \frac{v_i^2}{c^2} + \frac{2r_i (\hat{r}_i \cdot \vec{a}_i)}{c^2} \right] \hat{r}_i \\
    \vec{B} &= \sum_i \frac{q_i}{4\pi\epsilon_0 r_i^2 c^2} \cdot 2 (\hat{r}_i \cdot \vec{v}_i) \vec{v}_i \quad \text{(Umformung führt zum Biot-Savart-Gesetz)}
    \end{align}
    \item \textbf{Kontinuumslimes:} Übergang von diskreten Summen zu Integralen mit Ladungsdichte $\rho$ und Stromdichte $\vec{j}$:
    \begin{align}
    \sum_i q_i &\rightarrow \int d^3r' \, \rho(\vec{r}') \\
    \sum_i q_i \vec{v}_i &\rightarrow \int d^3r' \, \vec{j}(\vec{r}')
    \end{align}
    \item \textbf{Emergenz der Feldgleichungen:} Im Kontinuumslimes ergeben sich aus den Definitionen von $\vec{E}$ und $\vec{B}$ die Maxwell-Gleichungen:
    \begin{subequations}
    \begin{align}
    \nabla \cdot \vec{E} &= \frac{\rho}{\epsilon_0} \quad &\text{(Gaußsches Gesetz)} \\
    \nabla \cdot \vec{B} &= 0 \quad &\text{(Gaußsches Gesetz für den Magnetismus)} \\
    \nabla \times \vec{E} &= -\frac{\partial \vec{B}}{\partial t} \quad &\text{(Faradaysches Induktionsgesetz)} \\
    \nabla \times \vec{B} &= \mu_0 \vec{j} + \mu_0 \epsilon_0 \frac{\partial \vec{E}}{\partial t} \quad &\text{(Maxwellsches Durchflutungsgesetz)}
    \end{align}
    \end{subequations}
    \item \textbf{Modifikation in der vollen WDBT:} In der vollständigen Theorie wird die Kraftgleichung um das Quantenpotential $Q$ erweitert (Gl. \ref{eq:bewegung_q}), was zu modifizierten Maxwell-Gleichungen führt, die zusätzliche Terme enthalten.
\end{enumerate}

\subsection{Emergenz der Quantenmechanik (Schrödinger-Gleichung)}
\label{subsec:emergenz_qm}

\paragraph{Grundpostulat:} Die Schrödinger-Gleichung emergiert als effektive Beschreibung für die Dynamik eines Ensembles von Teilchen, deren Trajektorien deterministisch durch die WDBT-Gleichung geführt werden.

\paragraph{Systematische Herleitung (Madelung-Transformation):}
\begin{enumerate}
    \item \textbf{Ausgangspunkt:} Fundamentale Bewegungsgleichung der WDBT (Gl. \ref{eq:bewegung_q}): $m \frac{d^2 \vec{x}}{dt^2} = -\vec{\nabla} V - \vec{\nabla} Q$.
    \item \textbf{Madelung-Transformation:} Einführung der Wellenfunktion $\Psi(\vec{x}, t) = R(\vec{x}, t) e^{i S(\vec{x}, t)/\hbar}$ mit $\rho = R^2$ und $\vec{v} = \frac{1}{m} \vec{\nabla} S$.
    \item \textbf{Herleitung zweier reeller Gleichungen:}
    \begin{itemize}
        \item Einsetzen in die Kontinuitätsgleichung (Teilchenzahlerhaltung) liefert:
        \begin{equation}
        \label{eq:kontinuitaet}
        \frac{\partial \rho}{\partial t} + \vec{\nabla} \cdot (\rho \vec{v}) = 0
        \end{equation}
        \item Einsetzen in die Newtonsche Gleichung (Gl. \ref{eq:bewegung_q}) liefert eine modifizierte Hamilton-Jacobi-Gleichung:
        \begin{equation}
        \label{eq:hamilton_jacobi_mod}
        \frac{\partial S}{\partial t} + \frac{1}{2m} |\nabla S|^2 + V + Q = 0
        \end{equation}
        wobei $Q$ das Quantenpotential (Gl. \ref{eq:quantenpotential}) ist.
    \end{itemize}
    \item \textbf{Synthese zur Schrödinger-Gleichung:} Zeigt man, dass die beiden reellen Gleichungen (\ref{eq:kontinuitaet}) und (\ref{eq:hamilton_jacobi_mod}) äquivalent sind zur komplexen Schrödinger-Gleichung, erhält man:
    \begin{equation}
    \label{eq:schroedinger}
    i\hbar \frac{\partial \Psi}{\partial t} = \left( -\frac{\hbar^2}{2m} \nabla^2 + V \right) \Psi
    \end{equation}
    Die Schrödinger-Gleichung ist eine effektive Beschreibung für die Ensemble-Dynamik.
\end{enumerate}

\subsection{Konvergente Emergenz der Allgemeinen Relativitätstheorie (ART)}
\label{subsec:emergenz_art}

\paragraph{Kernargument:} Die ART wird durch Einführung des Quantenpotentials $Q$ vervollständigt (ART+). ART+ und WDBT konvergieren konzeptionell in zentralen Eigenschaften (Singularitätenfreiheit, Determinismus), bleiben aber experimentell unterscheidbar.

\paragraph{Systematischer Vergleich:}
\begin{enumerate}
    \item \textbf{Problem der ART:} Singularitäten in Schwarzen Löchern und beim Urknall.
    \item \textbf{Schritt 1:} Einführung des Quantenpotentials in die Einstein-Gleichungen:
    \begin{equation}
    \label{eq:einstein_mod}
    G_{\mu\nu} = 8\pi G (T_{\mu\nu} + Q_{\mu\nu})
    \end{equation}
    wobei $Q_{\mu\nu}$ der Energie-Impuls-Tensor des Quantenpotentials ist. Dies führt zu Singularitätenfreiheit (Big Bounce statt Big Bang), Determinismus und Verstärkung der Nicht-Lokalität.
    \item \textbf{Schritt 2:} Berücksichtigung instantaner (avancierter) Lösungen der Einstein-Gleichungen neben den retardierten Wellen.
    \item \textbf{Konvergenz:} ART+ und WDBT sind sich in puncto Singularitätenfreiheit und Determinismus sehr ähnlich.
    \item \textbf{Experimenteller Unterschied:}
    \begin{itemize}
        \item \textbf{ART+:} Lichtablenkung ist frequenzunabhängig (rein geometrisch).
        \item \textbf{WDBT:} Lichtablenkung ist frequenzabhängig $\Delta\phi(f)$ (dynamisch durch WG mit $\beta_\gamma=1$).
    \end{itemize}
\end{enumerate}

\subsection{Emergenz der Quantenelektrodynamik (QED)}
\label{subsec:emergenz_qed}

\paragraph{Kernargument:} Die Konzepte der QED (Feldquantisierung, Feynman-Diagramme, Renormierung) emergieren aus der WDBT als effektive Beschreibungen.

\paragraph{Systematische Emergenzbeziehungen:}
\begin{itemize}
    \item \textbf{Feldquantisierung:} Photonen emergieren als Anregungen des Quanten-Vakuums, beschrieben durch eine Vakuum-Wellenfunktion $\psi_{\text{Vak}}$.
    \item \textbf{Feynman-Diagramme:} Emergieren aus der Mittelung über alle nicht-lokalen Weber-Wechselwirkungspfade zwischen Ladungen (Pfadintegral-Formulierung).
    \item \textbf{Renormierung:} Divergenzen in Loop-Integralen werden durch das Quantenpotential $Q$ regularisiert, da die Führungswelle endliche Ausdehnung hat und so als natürlicher Cut-off wirkt.
    \item \textbf{Vorhersagen:} Die WDBT reproduziert QED-Ergebnisse (z.B. den anomalen g-Faktor des Elektrons) und sagt modifizierte Vorhersagen voraus, z.B. für die Lamb-Shift:
    \begin{equation}
    \label{eq:lamb_shift}
    \Delta E^{\text{WDBT}}_{\text{Lamb}} = \Delta E_{\text{QED}} + \frac{e^2 \hbar}{4\pi\epsilon_0 m_e^2 c^3} \langle r \rangle
    \end{equation}
    wobei $\langle r \rangle$ der Erwartungswert des Abstands im Atom ist. Der Zusatzterm rührt von der Kopplung an das Quantenpotential des Vakuums $Q_{\text{vak}}$ her.
\end{itemize}

\section{Anwendungen in der Plasmaphysik und Astrophysik}
\label{sec:anwendungen}

\subsection{Modifizierte Plasmatheorie}
\label{subsec:plasma}

Die WDBT führt zu Modifikationen der klassischen Magnetohydrodynamik (MHD).

\paragraph{Dispersionsrelation im Plasma:}
\begin{equation}
\label{eq:dispersion}
\omega^2 = \omega_p^2 \left( 1 + \frac{\hbar^2 k^2}{4m_e^2 \omega_p^2} \right)
\end{equation}

\paragraph{Modifizierte Ampère-Gleichung:}
Durch die nicht-lokalen Weber-Wechselwirkungen ergibt sich ein zusätzlicher Term:
\begin{equation}
\label{eq:ampere_mod}
\nabla \times \vec{B} = \mu_0 \vec{j} + \frac{\mu_0 e^2 n_e \lambda_e^2}{\epsilon_0} \frac{\partial \vec{j}}{\partial t}
\end{equation}
wobei $\lambda_e$ eine charakteristische Länge (z.B. Elektronenwellenlänge) ist.

\subsection{Anwendungen in der Fusionsforschung}
\label{subsec:fusion}
\begin{itemize}
    \item Selbstorganisierte Plasmastabilisierung durch das Quantenpotential $Q$.
    \item Erklärung anomaler Transportphänomene ohne ad-hoc Turbulenzmodelle.
    \item Fraktale Birkeland-Ströme könnten den Bau kompakterer Fusionsreaktoren ermöglichen.
\end{itemize}

\subsection{Anwendungen in der Astrophysik}
\label{subsec:astro}
\begin{itemize}
    \item Erklärung von Galaxienfilamenten und Anisotropien im kosmischen Mikrowellenhintergrund (CMB) durch fraktale Dichteverteilung ($D \approx 2.71$).
    \item Alternative zur Dunklen Materie: Die flachen Rotationskurven von Galaxien werden durch die Kombination aus Weber-Gravitation (Gl. \ref{eq:weber_grav}) und dem Effekt des Quantenpotentials $Q$ erklärt.
    \item Der Sonnenwind wird als emergentes Quantenphänomen gedeutet.
\end{itemize}

\subsection{Raumfahrtantrieb}
\label{subsec:antrieb}
\begin{itemize}
    \item Konzepte für Hybrid-Plasmaantriebe, die Coulomb-Explosion und Quantenresonanz ausnutzen.
    \item Höhere Effizienz wird durch nicht-lokale Beschleunigungseffekte der Weber-Kraft erwartet.
\end{itemize}

\section{Kernaussagen, Kritik und philosophische Implikationen}
\label{sec:kern_kritik}

\subsection{Zentrale Kernaussagen der WDBT}
\label{subsec:kernaussagen}
\begin{itemize}
    \item \textbf{Keine Felder:} Wechselwirkungen sind direkt zwischen Teilchen (\textbf{Aktion auf Distanz}).
    \item \textbf{Keine Singularitäten:} Das Quantenpotential $Q$ verhindert unendliche Dichten (z.B. in Schwarzen Löchern und beim Urknall).
    \item \textbf{Keine Dunkle Materie:} Galaxienrotationen werden durch die fraktale Raumstruktur und $Q$ erklärt.
    \item \textbf{Statisches Universum:} Die Hubble-Konstante und Rotverschiebung werden durch kumulative Gravitationseffekte der Weber-Kraft erklärt, nicht durch Expansion.
    \item \textbf{Fraktale Raumdimension:} $D \approx 2.71$ ist eine fundamentale Eigenschaft.
    \item \textbf{Emergenz:} SRT, ART, Maxwell-Theorie und QM emergieren aus der WDBT.
\end{itemize}

\subsection{Kritik an etablierten Theorien}
\label{subsec:kritik}
\begin{itemize}
    \item \textbf{Allgemeine Relativitätstheorie (ART):} Kritik an Singularitäten, der Notwendigkeit Dunkler Materie und der \glqq metaphysischen\grqq{} Krümmung der Raumzeit.
    \item \textbf{Maxwell-Theorie:} Kritik an unendlichen Selbstenergien punktförmiger Ladungen, Strahlungsparadoxa (z.B. Abraham-Lorentz-Kraft) und der Erklärung des Aharonov-Bohm-Effekts.
    \item \textbf{Quantenmechanik (QM):} Kritik am Kollaps der Wellenfunktion, dem Messproblem und der nicht-lokalen Verschränkung, die in der WDBT durch deterministische Führung erklärt wird.
\end{itemize}

\subsection{Philosophische Implikationen}
\label{subsec:philosophie}
\begin{itemize}
    \item Physik sollte sich auf beobachtbare Phänomene (\textbf{empirische Adäquatheit}) konzentrieren, nicht auf mathematische Eleganz oder \glqq metaphysische\grqq{} Konzepte wie Felder und gekrümmte Raumzeit.
    \item Die Wahl zwischen konkurrierenden Paradigmen (WDBT vs. Standardmodell) ist oft nicht rein empirisch, sondern auch von weltanschaulichen und konzeptionellen Präferenzen geleitet.
    \item Raum und Zeit könnten emergente Phänomene aus einer tieferen, diskreten oder fraktalen Struktur sein.
    \item Die Theorie vertritt einen \textbf{ontologischen Determinismus}.
\end{itemize}

\section{Experimentelle Vorhersagen}
\label{sec:vorhersagen}

Die WDBT sagt mehrere Abweichungen von den Vorhersagen der Standardtheorien voraus:
\begin{itemize}
    \item \textbf{Wellenlängenabhängige Lichtablenkung:} $\Delta\phi \propto \lambda^2$ bei Gravitationslinsen.
    \item \textbf{Frequenzabhängige Shapiro-Laufzeitverzögerung} von Signalen.
    \item \textbf{Abweichungen vom linearen Hubble-Gesetz} auf sehr großen Skalen.
    \item \textbf{Modifizierte Dispersionsrelation} in ultrakalten Quantengasen.
    \item \textbf{Stabilere Plasmawellen} bei sehr hohen Dichten ($n_e > 10^{20} \, \text{m}^{-3}$).
    \item \textbf{Modifizierter Lamb-Shift:} Siehe Gl. (\ref{eq:lamb_shift}).
    \item \textbf{Fraktale Strukturen} ($D \approx 2.71$) in astrophysikalischen und Laborplasmen.
    \item \textbf{Perihelberechnung} (z.B. Merkur) mit Genauigkeit der ART, aber auf anderer Grundlage.
\end{itemize}

\section{Zusammenfassung und Ausblick}
\label{sec:zusammenfassung}

Die Weber-De Broglie-Bohm-Theorie (WDBT) bietet eine kohärente, feldlose, nicht-lokale und deterministische Alternative zum Standardmodell der Physik. Sie vereint Plasmaphysik, Quantenmechanik, Elektrodynamik und Gravitation in einem konsistenten Rahmen, der konzeptionelle Probleme der etablierten Theorien (Singularitäten, unendliche Größen, Messproblem) vermeidet. Durch die Postulierung einer fundamentalen Ebene direkter Wechselwirkungen und eines Quantenpotentials wird die Emergenz der gesamten bekannten Physik beansprucht. Die Theorie macht eine Reihe spezifischer, testbarer Vorhersagen, die eine experimentelle Überprüfung ermöglichen. Sie fordert einen Paradigmenwechsel weg von Feldtheorien und Raumzeitkrümmung hin zu direkten Wechselwirkungen und nicht-lokaler Ganzheit.

\appendix
\section*{Anhang}
\addcontentsline{toc}{section}{Anhang}

\subsection*{Anhang A: Herleitung des modifizierten Lamb-Shifts}
\label{anh:a}

Die Modifikation der Lamb-Shift in Gl. (\ref{eq:lamb_shift}) ergibt sich aus der Kopplung des atomaren Elektrons an das Quantenpotential des Vakuums $Q_{\text{vak}}$, das in der WDBT eine reale physikalische Größe darstellt. Diese Kopplung addiert einen energieverschiebenden Term zum bekannten QED-Resultat.

\subsection*{Anhang B: Herleitung von Impuls und Energie aus der Weber-Wechselwirkung}
\label{anh:b}

Aus einer Lagrangian-Formulierung der Weber-Wechselwirkung (sowohl für WED als auch WG) können die relativistischen Ausdrücke für Impuls und Energie hergeleitet werden:
\begin{align*}
\vec{p} &= \gamma m \vec{v} \\
E &= \gamma m c^2
\end{align*}
Dies wird als ultimative Umsetzung des Mach'schen Prinzips interpretiert, da die Trägheit eines Körpers durch seine Wechselwirkung mit allen anderen Massen im Universum entsteht.

\subsection*{Anhang C: Berechnung der Gesamtmasse des Universums}
\label{anh:c}

Basierend auf dem Mach'schen Prinzip und der Emergenz der Trägheit aus der Weber-Gravitation kann die Gesamtmasse $M$ des Universums abgeschätzt werden. Aus der Bedingung, dass die Trägheitskraft gleich der gravitativen Wechselwirkung mit dem gesamten Universum sein soll, ergibt sich:
\[
c^2 = k G \int \frac{\rho(\vec{r})}{r} dV
\]
Unter der Annahme eines homogenen Universums mit Radius $R$ und Dichte $\rho$ folgt daraus:
\[
M = \frac{2 c^2 R}{3 k G}
\]
wobei $k$ eine Konstante ist. Dies stellt eine bemerkenswerte Verbindung zwischen kosmologischen Parametern und fundamentalen Konstanten her.

\end{document}
