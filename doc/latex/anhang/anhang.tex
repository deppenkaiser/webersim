\chapter{Anhang}
\section{Der Aharonov-Bohm-Effekt}
\label{sec:aharonov-bohm}

Der \textbf{Aharonov-Bohm-Effekt} (AB-Effekt) ist ein grundlegendes Quantenphänomen, das zeigt, dass elektromagnetische Potentiale ($\vec{A}$, $\Phi$) eine direkte physikalische
Wirkung auf Quantenteilchen haben, selbst in Regionen wo die Felder ($\vec{E}$, $\vec{B}$) null sind.

\subsection{Experimentelle Anordnung}
Ein Elektronenstrahl wird in zwei Pfade aufgeteilt, die eine Region mit magnetischem Fluss $\Phi$ umschließen.

\subsection{Theoretische Beschreibung}
Die Wellenfunktion $\psi$ eines Teilchens mit Ladung $q$ wird durch das Vektorpotential $\vec{A}$ modifiziert:

\begin{equation}
\psi \rightarrow \psi \cdot \exp\left(i\frac{q}{\hbar}\int \vec{A}\cdot d\vec{l}\right)
\end{equation}

Die Phasendifferenz zwischen den beiden Pfaden beträgt:

\begin{equation}
\Delta\phi = \frac{q}{\hbar}\oint \vec{A}\cdot d\vec{l} = \frac{q}{\hbar}\Phi_B
\end{equation}

\subsection{Physikalische Bedeutung}
\begin{itemize}
\item \textbf{Nicht-Lokalität}: Quantenteilchen \enquote{spüren} $\vec{A}$ auch in feldfreien Regionen
\item \textbf{Topologische Invariante}: Die Phase hängt nur vom eingeschlossenen Fluss $\Phi_B$ ab
\item \textbf{Paradigmenwechsel}: Widerlegt die klassische Annahme, dass nur $\vec{E}$ und $\vec{B}$ physikalisch relevant sind
\end{itemize}

\subsection{Experimentelle Bestätigung}
\begin{itemize}
\item Theoretische Vorhersage: Aharonov \& Bohm (1959)
\item Erste Experimente: Chambers (1960), Tonomura et al. (1982)
\item Moderne Anwendungen: Quanteninterferometer, topologische Quantenmaterialien
\end{itemize}

\section{Bellsche Ungleichungen}
\label{sec:bell}

Die \textbf{Bellsche Ungleichung} (1964) ist ein zentrales Ergebnis der Quantenphysik, das zeigt, dass keine lokale Theorie mit verborgenen Variablen die Vorhersagen der Quantenmechanik reproduzieren kann.

\subsection{Theoretische Formulierung}
Für ein verschränktes Teilchenpaar (z.B. Photonen mit Spin- oder Polarisationskorrelation) gilt die CHSH-Ungleichung:

\begin{equation}
S = |E(a,b) - E(a,b')| + |E(a',b) + E(a',b')| \leq 2
\end{equation}

wobei $E(\theta_1, \theta_2)$ die Korrelationsfunktion der Messungen bei Winkeln $\theta_1$ und $\theta_2$ ist.

\subsection{Quantenmechanische Vorhersage}
Die Quantenmechanik erlaubt für bestimmte Winkelkombinationen:

\begin{equation}
S_{\text{QM}} = 2\sqrt{2} \approx 2.828 > 2
\end{equation}

was die Bell-Ungleichung verletzt.

\subsection{Experimentelle Bestätigung}
\begin{itemize}
\item Erste Tests: Alain Aspect (1982) mit Photonenpaaren
\item Loophole-free Experimente: Hensen et al. (2015), Zeilinger-Gruppe (2017)
\item Heutige Anwendungen: Quantenkryptographie (BB84-Protokoll)
\end{itemize}

\subsection{Interpretation}
\begin{itemize}
\item Widerlegung lokaler realistischer Theorien (Einstein-Podolsky-Rosen-Paradoxon)
\item Bestätigung der Quantenverschränkung als physikalische Realität
\item Grundlage für Quanteninformationstechnologien
\end{itemize}
