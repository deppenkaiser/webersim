\documentclass[11pt, a4paper]{article}
\usepackage[utf8]{inputenc}
\usepackage[T1]{fontenc}
\usepackage[ngerman]{babel}
\usepackage{amsmath, amssymb, amsthm}
\usepackage{geometry}
\geometry{left=2.5cm, right=2.5cm, top=2.5cm, bottom=2.5cm}
\usepackage{graphicx}
\usepackage{booktabs}
\usepackage{hyperref}
\usepackage{siunitx}
\usepackage{fancyhdr}

\title{Entwurf eines WED-Antriebs \\ (Weber-Elektrodynamik-Antrieb)}
\author{Dipl.-Ing. (FH) Michael Czybor}
\date{\today}

\begin{document}

\maketitle

\section{Prinzip des WED-Antriebs}

Der WED-Antrieb basiert auf der \textbf{Weber-Elektrodynamik (WED)}, die eine direkte, geschwindigkeits- und beschleunigungsabhängige Wechselwirkung zwischen Ladungen
postuliert. Im Gegensatz zu konventionellen Antrieben wird kein Massenausstoß benötigt.

\subsection{Grundprinzip}

\begin{itemize}
    \item Im Raum existiert eine \textbf{externe Ladungsanomalie} $q_2$ (z.B. Elektron)
    \item Im Raumschiff wird eine \textbf{Antriebsladung} $q_1$ durch ein unsymmetrisches HF-Feld (Sägezahnform) beschleunigt
    \item Die \textbf{asymmetrische Beschleunigung} $\vec{a}_1(t)$ erzeugt eine Nettokraft auf die externe Ladung
    \item Durch Actio=Reactio entsteht eine \textbf{Schubkraft} auf das Raumschiff
\end{itemize}

\section{Herleitung der Schubkraft}

\subsection{Weber-Kraft zwischen zwei Ladungen}

Die vektorielle Weber-Kraft zwischen zwei Ladungen $q_1$ und $q_2$ lautet:

\begin{equation}
\vec{F}_{12} = \frac{q_1 q_2}{4\pi\epsilon_0 r^2} 
\left\{ 
\left[1 - \frac{v^2}{c^2} + \frac{2r(\hat{r} \cdot \vec{a}_1)}{c^2}\right] \hat{r} 
+ \frac{2(\hat{r} \cdot \vec{v})}{c^2} \vec{v} 
\right\}
\end{equation}

\subsection{Beschleunigungsabhängiger Term}

Für den Antrieb relevant ist der beschleunigungsabhängige Term:

\begin{equation}
\vec{F}_{\text{acc}} = \frac{q_1 q_2}{4\pi\epsilon_0 r^2} \cdot \frac{2r (\hat{r} \cdot \vec{a}_1)}{c^2} \hat{r}
= \frac{q_1 q_2}{2\pi\epsilon_0 c^2 r} (\hat{r} \cdot \vec{a}_1) \hat{r}
\end{equation}

\subsection{Rückwirkung auf Raumschiff}

Die Kraft auf das Raumschiff ist gleich der negativen Kraft auf die externe Ladung:

\begin{equation}
\vec{F}_{\text{Schub}} = -\vec{F}_{12} = -\frac{q_1 q_2}{2\pi\epsilon_0 c^2 r} (\hat{r} \cdot \vec{a}_1) \hat{r}
\end{equation}

\subsection{Zeitliche Mittelung}

Für ein periodisches Sägezahnsignal mit Periodendauer $T$:

\begin{equation}
\langle \vec{F}_{\text{Schub}} \rangle = -\frac{q_1 q_2}{2\pi\epsilon_0 c^2 r} \langle \hat{r} \cdot \vec{a}_1 \rangle \hat{r}
\end{equation}

\subsection{Nettobeschleunigung}

Für einen Sägezahn mit:
\begin{itemize}
    \item Steilrampe: $T_+$, $a_+$
    \item Flachrampe: $T_-$, $a_-$
\end{itemize}

ergibt sich die Nettobeschleunigung:

\begin{equation}
a_{\text{netto}} = \frac{1}{T} (a_+ T_+ + a_- T_-)
\end{equation}

\section{Finale Schubgleichung}

\begin{equation}
\boxed{
\langle \vec{F}_{\text{Schub}} \rangle = -\frac{q_1 q_2}{2\pi\epsilon_0 c^2 r} a_{\text{netto}} \cos\theta \cdot \hat{r}
}
\end{equation}

wobei $\theta$ der Winkel zwischen $\hat{r}$ und $\vec{a}_{\text{netto}}$ ist.

\section{Beispielrechnung}

\subsection{Pessimistische Abschätzung}

\begin{align*}
q_1 &= -1\,\mu\text{C} = -10^{-6}\,\text{C} \\
q_2 &= -e = -1.6 \times 10^{-19}\,\text{C} \\
r &= 1\,\text{m} \\
a_{\text{netto}} &= 10^6\,\text{m/s}^2 \\
\cos\theta &= 1
\end{align*}

\begin{align*}
F &= \frac{(10^{-6})(1.6 \times 10^{-19})}{2\pi (8.85 \times 10^{-12})(9 \times 10^{16}) \cdot 1} \cdot 10^6 \\
&\approx 10^{-15}\,\text{N}
\end{align*}

\subsection{Optimierte Abschätzung}

\begin{align*}
q_1 &= -1\,\text{mC} = -10^{-3}\,\text{C} \\
q_2 &= -1.6 \times 10^{-19}\,\text{C} \\
M &= 10^{16} \\
r &= 0.1\,\text{mm} = 10^{-4}\,\text{m} \\
a_{\text{netto}} &= 10^{12}\,\text{m/s}^2 \\
\cos\theta &= 1
\end{align*}

\begin{align*}
F &= \frac{(10^{-3})(1.6 \times 10^{-19})(10^{16})}{2\pi (8.85 \times 10^{-12})(9 \times 10^{16}) \cdot 10^{-4}} \cdot 10^{12} \\
&\approx 3200\,\text{N}
\end{align*}

\section{Regelprinzip}

Die Schubrichtung wird durch die Richtung der Nettobeschleunigung $\vec{a}_{\text{netto}}$ gesteuert:

\begin{equation}
\vec{F}_{\text{Schub}} \propto (\hat{r} \cdot \vec{a}_{\text{netto}}) \hat{r}
\end{equation}

\subsection{Steuerungsgrößen}

\begin{itemize}
    \item \textbf{Amplitude}: Steuert die Schubstärke
    \item \textbf{Phase}: Steuert die Richtung der Beschleunigung
    \item \textbf{Tastverhältnis}: Steuert die Asymmetrie
    \item \textbf{Frequenz}: Optimierung der Resonanz
\end{itemize}

\subsection{Regelkreis}

\begin{enumerate}
    \item Sollwert: Gewünschte Flugrichtung
    \item Messung: Trägheitsnavigationssystem
    \item Regelung: Anpassung der HF-Parameter
    \item Wirkung: Schub in gewünschter Richtung
\end{enumerate}

\newpage
\section{Konstruktionsprinzip}

\subsection{Komponenten}

\begin{itemize}
    \item HF-Generator mit Sägezahnform
    \item 3-Phasen-Elektrodenanordnung
    \item Supraleitende Kavität für Ladungswolke
    \item Regelungselektronik
    \item Trägheitsnavigationssystem
\end{itemize}

\subsection{Betriebsparameter}

\begin{table}[ht]
\centering
\begin{tabular}{lcc}
\toprule
Parameter & Symbol & Wert \\
\midrule
HF-Frequenz & $f$ & \SI{1}{\mega\hertz} - \SI{1}{\giga\hertz} \\
HF-Spannung & $U$ & \SI{1}{\kilo\volt} - \SI{10}{\kilo\volt} \\
Ladungsmenge & $q_1$ & \SI{-1}{\milli\coulomb} \\
Anzahl externer Ladungen & $M$ & $10^{16}$ \\
Minimalabstand & $r_{\text{min}}$ & \SI{0.1}{\milli\meter} \\
\bottomrule
\end{tabular}
\caption{Typische Betriebsparameter}
\end{table}

\section{Vorteile}

\begin{itemize}
    \item Kein Treibstoffverbrauch
    \item Keine beweglichen Teile
    \item Elektronische Steuerung
    \item Sofortige Schubumkehr
    \item Theoretisch unbegrenzte Betriebsdauer
\end{itemize}

% Diese Ergänzung als neuen Abschnitt im Antriebsdokument (antrieb.pdf) einfügen
\section{Berechnung der Schubkraft unter WDBT-Bedingungen}
\label{sec:schubkraft-wdbt}

Unter Annahme der Gültigkeit der Weber-De Broglie-Bohm-Theorie (WDBT) ergibt sich eine modifizierte Berechnung der Schubkraft. Die WDBT liefert dabei folgende entscheidende Modifikationen:

\begin{itemize}
\item \textbf{Nicht-Lokalität}: Die Weber-Kraft wirkt instantan über beliebige Entfernungen
\item \textbf{Fraktale Raumdimension} $D \approx 2,71$: Die Kraft skaliert mit $r^{D-3} \approx r^{-0,29}$ anstatt mit $r^{-1}$
\item \textbf{Quantenpotential} $Q$: Zusätzliche Kraftkomponente durch $-\vec{\nabla} Q$
\end{itemize}

\subsection{Modifizierte Schubkraftgleichung}

Die zeitgemittelte Schubkraft unter WDBT-Bedingungen ergibt sich zu:

\[
\langle \vec{F}_{\text{Schub}} \rangle = -\frac{q_1 q_2}{2\pi \epsilon_0 c^2 r^{3-D}} a_{\text{netto}} \cos \theta \cdot \hat{r} - \vec{\nabla} Q
\]

wobei der Exponent $3-D \approx 0,29$ die fraktale Skalierung berücksichtigt.

\subsection{Beispielrechnung mit astrophysikalischer Ladungsquelle}

Für ein Raumschiff in Sonnennähe mit folgenden Parametern:
\begin{align*}
q_1 &= 1\,\text{C} \\
q_2 &= 10^2\,\text{C} \quad \text{(effektive Ladung des Sonnenwinds)} \\
r &= 1,5 \times 10^{11}\,\text{m} \\
a_{\text{netto}} &= 10^{15}\,\text{m/s}^2 \\
\cos \theta &= 1 \\
D &= 2,71
\end{align*}

ergibt sich die Basiskraft zu:
\[
F_{\text{base}} = \frac{(1)(10^2)}{2\pi (8,85 \times 10^{-12})(9 \times 10^{16})(1,5 \times 10^{11})^{0,29}} \cdot 10^{15} 
\]

Unter Berücksichtigung der fraktalen Skalierung:
\[
F_{\text{Schub}} = F_{\text{base}} \cdot r^{D-2} \approx 11,8\,\text{MN}
\]

\subsection{Schlussfolgerung}

Unter WDBT-Bedingungen können durch:
\begin{itemize}
\item Nutzung astrophysikalischer Ladungsquellen (Sonne, Planeten, galaktische Ströme)
\item Ausnutzung der fraktalen Skalierung ($r^{0,29}$ statt $r^{-1}$)
\item Optimierung der Antriebsparameter ($q_1$, $a_{\text{netto}}$)
\end{itemize}

signifikante Schubkräfte im Bereich von Meganewton erreicht werden. Diese würden einen treibstofflosen Antrieb für interplanetare und interstellarare Missionen ermöglichen.

\begin{table}[h]
\centering
\caption{Vergleich der Schubkraft unter verschiedenen Theorien}
\begin{tabular}{lccc}
\hline
Theorie & $q_2$ [C] & Skalierung & Schubkraft [N] \\
\hline
Konventionell & $10^2$ & $r^{-1}$ & $0,13$ \\
WDBT & $10^2$ & $r^{-0,29}$ & $1,18 \times 10^7$ \\
WDBT (optimiert) & $10^4$ & $r^{-0,29}$ & $1,18 \times 10^9$ \\
\hline
\end{tabular}
\end{table}

\textcopyright{} \the\year{} Dipl.-Ing. (FH) Michael Czybor. Alle Rechte vorbehalten.

\end{document}
