\section{Anwendungsbeispiel: Merkur-Orbit}
\subsection{Berechnung für 1° Bahnsegment ($\Delta\phi = \pi/180$)}
\begin{table}[h]
\centering
\begin{tabular}{|l|l|}
\hline
\textbf{Term} & \textbf{Beitrag zur Zeit t} \\ \hline
Klassisch (Kepler) & $\approx 7.0$ Tage \\ \hline
Relativistische Korrektur & $\approx -0.002$ Tage ($\approx -3$ Minuten) \\ \hline
\textbf{Gesamt} & \textbf{$\approx 6.998$ Tage} \\ \hline
\end{tabular}
\end{table}

\subsection{Physikalische Interpretation}
Die negative Korrektur zeigt, dass der Merkur schneller als klassisch vorhergesagt läuft -- dies erklärt die beobachtete Periheldrehung von $43''$ pro Jahrhundert.