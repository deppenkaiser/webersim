\section{Beispiel: Sonne-Jupiter-System}

Mit $ M_\odot = 1.989 \times 10^{30} \text{ kg} $, $ m_J = 1.898 \times 10^{27} \text{ kg} $:
\begin{align}
\vec{R}_\odot &= -\frac{m_J}{M_\odot + m_J} \vec{r}_J \approx -7.425 \times 10^8 \text{ m} \\
\vec{V}_\odot &= -\frac{m_J}{M_\odot + m_J} \vec{v}_J \approx -12.46 \text{ m/s}
\end{align}

\begin{table}[H]
    \centering
    \begin{tabular}{lll}
        \toprule
        Größe & Heliozentrisch & Baryzentrisch \\
        \midrule
        Sonnenposition & $\vec{0}$ & $\approx -742,500 \text{ km}$ \\
        Jupiterposition & $778.5 \times 10^6 \text{ km}$ & $\approx 777.8 \times 10^6 \text{ km}$ \\
        \bottomrule
    \end{tabular}
    \caption{Vergleich der Koordinatensysteme}
\end{table}