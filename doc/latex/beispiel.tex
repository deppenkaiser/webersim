\section{Beispiel: Sonne-Jupiter-System}
Mit $M_\odot = 1.989 \times 10^{30} \text{ kg}$, $m_J = 1.898 \times 10^{27} \text{ kg}$:
\[
\vec{R}_\odot = -\frac{m_J}{M_\odot + m_J} \vec{r}_J \approx -7.425 \times 10^8 \text{ m}
\]
\[
\vec{V}_\odot = -\frac{m_J}{M_\odot + m_J} \vec{v}_J \approx -12.46 \text{ m/s}
\]

\begin{table}[h]
\centering
\begin{tabular}{|l|l|l|}
\hline
\textbf{Größe} & \textbf{Heliozentrisch} & \textbf{Baryzentrisch} \\ \hline
Sonnenposition & $\vec{0}$ & $\approx -742,500 \text{ km}$ \\ \hline
Jupiterposition & $778.5 \times 10^6 \text{ km}$ & $\approx 777.8 \times 10^6 \text{ km}$ \\ \hline
\end{tabular}
\end{table}