\section{Beispiel: Merkur im Perihel ($\varphi_0 = 0$)}
\begin{table}[h]
\centering
\begin{tabular}{|l|l|}
\hline
\textbf{Parameter} & \textbf{Wert} \\ \hline
Große Halbachse $a$ & $5.79 \times 10^{10}$ m \\ \hline
Exzentrizität $e$ & 0.2056 \\ \hline
Radius im Perihel $r(0)$ & $4.60 \times 10^{10}$ m \\ \hline
\end{tabular}
\end{table}

\subsection{Berechnung}
Kepler-Term:
\[
\frac{h}{r^2(0)} \approx 1.236 \times 10^{-6}\,\text{rad/s}
\]
Weber-Korrektur:
\[
\frac{3GMh}{c^2 r^3(0)} \approx 1.02 \times 10^{-13}\,\text{rad/s}
\]

\subsection{$\Delta\phi$ nach 1 Sekunde}
\[
\Delta \phi \approx 1.236 \times 10^{-6}\,\text{rad} + 1.02 \times 10^{-13}\,\text{rad}
\]
Die Weber-Korrektur ist winzig, aber kumuliert über 415 Umläufe (100 Jahre) ergibt sich die beobachtete Periheldrehung von $43''$.