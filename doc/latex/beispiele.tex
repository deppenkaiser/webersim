\section{Beispielberechnungen}
\subsection{Perihel ($\phi = 0$)}
\[
\vec{r}(0) = \begin{pmatrix} a(1-e) \\ 0 \end{pmatrix} \approx \begin{pmatrix} 4.6 \times 10^{10} \\ 0 \end{pmatrix} \text{m}
\]
\[
\vec{v}(0) = \begin{pmatrix} 0 \\ \sqrt{\frac{GM}{a(1-e^2)}}(1+e) \end{pmatrix} \approx \begin{pmatrix} 0 \\ 59 \times 10^3 \end{pmatrix} \text{m/s}
\]

\subsection{Physikalische Interpretation}
\begin{table}[h]
\centering
\begin{tabular}{|l|l|l|}
\hline
\textbf{Effekt} & \textbf{Mathematische Ursache} & \textbf{Konsequenz} \\ \hline
Periheldrehung & $\kappa \neq 1$ & Bahn schließt sich nicht nach $2\pi$ \\ \hline
Geschwindigkeitsmodulation & Terme mit $1/c^2$ in $\vec{v}(\phi)$ & Variation der Bahngeschwindigkeit \\ \hline
Energieerhaltung & Spezifische Form der Weber-Kraft & Modifiziertes Potential \\ \hline
\end{tabular}
\end{table}