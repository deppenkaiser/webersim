\newpage
\section{Bewegungsgleichung in Polarkoordinaten (1. Ordnung)}
Die Bahngleichung \(r(\phi)\) in der Weber-Gravitation bis zur Ordnung \(\mathcal{O}(c^{-2})\) lautet:

\begin{equation}
r(\phi) = \frac{a(1 - e^2)}{1 + e \cos\left(\kappa\phi\right)}
\end{equation}

\noindent mit den Definitionen:
\begin{align*}
h &= \sqrt{GMa(1 - e^2)}, \\
\kappa &= \sqrt{1 - \frac{6GM}{c^2a(1 - e^2)}}.
\end{align*}

\subsection*{Physikalische Interpretation}
\begin{itemize}
    \item \textbf{Struktur}: 
        \begin{itemize}
            \item Der Nenner \(1 + e \cos(\kappa\phi)\) beschreibt eine Ellipse mit relativistisch\\modifizierter Winkelabhängigkeit.
            \item Die Wurzel \(\kappa\) quantifiziert die Abweichung von der Kepler-Bahn (\(\kappa = 1\) im Newton-Fall).
        \end{itemize}
    \item \textbf{Periheldrehung}:
        Die Periheldrehung pro Umlauf ergibt sich aus der Nicht-Ganzzahligkeit von \(\kappa\):
        \[
        \Delta\phi = 2\pi\left(\frac{1}{\kappa} - 1\right)
        \]
    \item \textbf{Grenzfälle}:
        \begin{itemize}
            \item Newton (\(c \to \infty\)): \(\kappa = 1\) \(\Rightarrow\) \(r_N(\phi) = \frac{a(1 - e^2)}{1 + e \cos\phi}\).
            \item Kreisbahn (\(e = 0\)): \(r(\phi) = a\) (keine Winkelabhängigkeit).
        \end{itemize}
\end{itemize}

\subsection*{Mathematische Herleitung}
Die Gleichung folgt aus der Lösung der Bewegungsgleichung:
\begin{equation}
\frac{d^2u}{d\phi^2} + u = \frac{GM}{h^2} + \frac{6GM}{c^2} u^2 \quad \left(u = \frac{1}{r}\right),
\end{equation}

wobei der Term \(\frac{6GM}{c^2} u^2\) die Weber-spezifische Korrektur 1. Ordnung darstellt. Der Ansatz \(u(\phi) = \frac{1 + e \cos(\kappa\phi)}{a(1 - e^2)}\) führt auf die angegebene Lösung.

\section{Bewegungsgleichungen}
Die Weber-Kraft in Polarkoordinaten:
\begin{equation}
\mathbf{F} = -\frac{GMm}{r^2}\left(1 - \frac{\dot{r}^2}{c^2} + \frac{r\ddot{r}}{2c^2}\right)\mathbf{\hat{r}}
\end{equation}

Mit $u=1/r$ und Drehimpuls $\mathbf{h}=r^2\dot{\phi}$:
\begin{equation}
\frac{d^2u}{d\phi^2} + u = \frac{GM}{h^2} + \frac{6GM}{c^2}u^2 + \frac{GM}{2c^2}\left(u\frac{d^2u}{d\phi^2} + \left(\frac{du}{d\phi}\right)^2\right)
\end{equation}

\subsection{Bahngleichung 2. Ordnung}
Bahngleichung:
\begin{equation}
\boxed
{
    r(\phi) = \frac{a(1-e^2)}{1 + e\cos\left(\kappa\phi + \frac{\alpha\phi^2}{c^4}\right)}
}
\end{equation}
mit:
\begin{align}
\kappa &= \sqrt{1 - \frac{6GM}{c^2a(1-e^2)} + \frac{27G^2M^2}{2c^4a^2(1-e^2)^2}}\\
\alpha &= \frac{3G^2M^2e}{8h^4}
\end{align}

\section{Herleitung der Periheldrehung $\Delta\phi$}

Die Periheldrehung in der Weber-Gravitation ergibt sich aus der modifizierten Bahngleichung und lässt sich wie folgt herleiten:

\subsection{Bahngleichung}
Die Bahn eines Planeten in der Weber-Gravitation wird beschrieben durch:
\begin{equation}
r(\phi) = \frac{a(1 - e^2)}{1 + e \cos(\kappa\phi)},
\end{equation}
wobei:
\begin{itemize}
\item $a$ die große Halbachse,
\item $e$ die Exzentrizität der Bahn,
\item $\kappa$ eine Konstante ist, die relativistische Korrekturen enthält:
\begin{equation}
\kappa = \sqrt{1 - \frac{6GM}{c^2 a(1 - e^2)}}.
\end{equation}
\end{itemize}

\subsection{Perihelbedingung}
Das Perihel (sonnennächster Punkt) tritt auf, wenn der Nenner maximal wird, d.h. wenn:
\begin{equation}
\cos(\kappa\phi) = 1.
\end{equation}
Die Lösungen dieser Bedingung sind:
\begin{equation}
\kappa\phi = 2\pi n \quad \text{(für $n \in \mathbb{Z}$)}.
\end{equation}
Somit ergeben sich die Winkel für aufeinanderfolgende Periheldurchgänge zu:
\begin{equation}
\phi_n = \frac{2\pi n}{\kappa}.
\end{equation}

\subsection{Periheldrehung pro Umlauf}
Die Periheldrehung $\Delta\phi$ ist die Differenz zwischen dem Winkel für einen vollständigen Umlauf ($n = 1$) und dem Newton'schen Fall ($\kappa = 1$):
\begin{equation}
\Delta\phi = \phi_1 - 2\pi = \frac{2\pi}{\kappa} - 2\pi.
\end{equation}
Daraus folgt die gesuchte Gleichung:
\begin{equation}
\boxed{\Delta\phi = 2\pi\left(\frac{1}{\kappa} - 1\right)}.
\end{equation}

\subsection{Interpretation}
\begin{itemize}
\item Im Newton'schen Grenzfall ($\kappa = 1$) verschwindet die Periheldrehung ($\Delta\phi = 0$).
\item Für $\kappa < 1$ (Weber-Gravitation) ergibt sich eine positive Periheldrehung, die mit Beobachtungen (z.B. Merkurperihel) übereinstimmt.
\end{itemize}

\section{Periheldrehung in 2. Ordnung}

\subsection{Erweiterte Bahngleichung}
In 2. Ordnung lautet die Bahngleichung:
\begin{equation}
r(\phi) = \frac{a(1 - e^2)}{1 + e \cos\left(\kappa\phi + \alpha\phi^2\right)},
\end{equation}
mit den erweiterten Koeffizienten:
\begin{align}
\kappa &= \sqrt{1 - \frac{6GM}{c^2 a(1 - e^2)} + \frac{27G^2 M^2}{2c^4 a^2 (1 - e^2)^2}}, \\
\alpha &= \frac{3G^2 M^2 e}{8c^4 h^4}, \quad h = \sqrt{GMa(1 - e^2)}.
\end{align}

\subsection{Entwicklung von $\kappa$}
Eine Taylor-Entwicklung von $\kappa$ bis zur 2. Ordnung liefert:
\begin{equation}
\kappa \approx 1 - \frac{3GM}{c^2 a(1 - e^2)} + \frac{27G^2 M^2}{4c^4 a^2 (1 - e^2)^2} + \mathcal{O}(c^{-6}).
\end{equation}

\subsection{Perihelbedingung (2. Ordnung)}
Das Perihel tritt auf bei:
\begin{equation}
\cos\left(\kappa\phi + \alpha\phi^2\right) = 1 \quad \Rightarrow \quad \kappa\phi + \alpha\phi^2 = 2\pi n.
\end{equation}

\subsection{Lösung für $\Delta\phi$}
Für $n=1$ (ein Umlauf) ergibt sich die quadratische Gleichung:
\begin{equation}
\alpha\phi^2 + \kappa\phi - 2\pi = 0.
\end{equation}
Die Lösung lautet:
\begin{equation}
\phi = \frac{-\kappa + \sqrt{\kappa^2 + 8\pi\alpha}}{2\alpha}.
\end{equation}

\subsection{Näherung für kleine Korrekturen}
Da $\alpha \sim c^{-4}$ klein ist, entwickeln wir die Wurzel:
\begin{equation}
\phi \approx \frac{2\pi}{\kappa} - \frac{4\pi^2\alpha}{\kappa^3} + \mathcal{O}(\alpha^2).
\end{equation}
Die Periheldrehung pro Umlauf wird damit:
\begin{equation}
\Delta\phi = \phi - 2\pi \approx 2\pi\left(\frac{1}{\kappa} - 1\right) - \frac{4\pi^2\alpha}{\kappa^3}.
\end{equation}

\subsection{Endgültige Formel}
Einsetzen von $\kappa \approx 1$ im Korrekturterm liefert:
\begin{equation}
\boxed{
\Delta\phi \approx 2\pi\left(\frac{1}{\kappa} - 1\right) - 4\pi^2\alpha
},
\end{equation}
wobei:
\begin{itemize}
\item Der erste Term die 1. Ordnung (wie zuvor) beschreibt
\item Der zweite Term ($-4\pi^2\alpha$) die 2. Ordnungskorrektur darstellt
\item Für Merkur beträgt der 2. Ordnungsterm $\sim 10^{-7}$ Bogensekunden/Jh. und ist damit vernachlässigbar
\end{itemize}

\subsection{Vollständige Koeffizienten}
Explizit ausgedrückt:
\begin{align*}
\Delta\phi^{(2)} &= \frac{6\pi GM}{c^2 a(1 - e^2)} \left[1 + \frac{9GM}{4c^2 a(1 - e^2)}\right] - \frac{3\pi^2 G^2 M^2 e}{2c^4 h^4} \\
&= \Delta\phi^{(1)} + \frac{27\pi G^2 M^2}{2c^4 a^2 (1 - e^2)^2} - \frac{3\pi^2 G^2 M^2 e}{2c^4 [GMa(1 - e^2)]^2}
\end{align*}
