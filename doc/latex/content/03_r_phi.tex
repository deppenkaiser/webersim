\section{Bewegungsgleichungen}
Die Weber-Kraft in Polarkoordinaten:
\begin{equation}
\mathbf{F} = -\frac{GMm}{r^2}\left(1 - \frac{\dot{r}^2}{c^2} + \frac{r\ddot{r}}{2c^2}\right)\mathbf{\hat{r}}
\end{equation}

Mit $u=1/r$ und Drehimpuls $h=r^2\dot{\phi}$:
\begin{equation}
\frac{d^2u}{d\phi^2} + u = \frac{GM}{h^2} + \frac{6GM}{c^2}u^2 + \frac{GM}{2c^2}\left(u\frac{d^2u}{d\phi^2} + \left(\frac{du}{d\phi}\right)^2\right)
\end{equation}

\section{Störungsrechnung}
Ansatz:
\begin{equation}
u(\phi) = \sum_{k=0}^2 \frac{u_k(\phi)}{c^{2k}} + \mathcal{O}(c^{-6})
\end{equation}

\subsection{Ordnungen}
\begin{itemize}
\item[0.] Kepler-Lösung:
\begin{equation}
u_0 = \frac{GM}{h^2}(1 + e\cos\phi)
\end{equation}

\item[1.] ART-Äquivalent:
\begin{equation}
u_1 = \frac{3G^2M^2e}{h^4}\phi\sin\phi,\quad \Delta\phi_1 = \frac{6\pi GM}{c^2a(1-e^2)}
\end{equation}

\item[2.] Korrekturterm:
\begin{equation}
u_2 = \frac{G^3M^3e}{h^6}\left(\frac{27}{4}\phi\sin\phi + \frac{3e}{8}\phi^2\cos\phi\right)
\end{equation}
\end{itemize}

\section{Resultate}
Bahngleichung:
\begin{equation}
r(\phi) = \frac{a(1-e^2)}{1 + e\cos\left(\kappa\phi + \frac{\alpha\phi^2}{c^4}\right)}
\end{equation}
mit:
\begin{align}
\kappa &= \sqrt{1 - \frac{6GM}{c^2a(1-e^2)} + \frac{27G^2M^2}{2c^4a^2(1-e^2)^2}}\\
\alpha &= \frac{3G^2M^2e}{8h^4}
\end{align}

Periheldrehung:
\begin{equation}
\Delta\phi = \frac{6\pi GM}{c^2a(1-e^2)}\left(1 \underbrace{- \frac{3GM}{4c^2a(1-e^2)}}_{\text{Korrektur}}\right)
\end{equation}

\section{Interpretation}
\begin{itemize}
\item ART-Wert ($43''$/Jh.) wird systematisch überschätzt
\item WG zeigt konsistente Reduktion durch $c^{-4}$-Terme
\item Physikalische Ursache: Nichtlineare Rückkopplung der Beschleunigungsterme
\end{itemize}
