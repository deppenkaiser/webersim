\newpage
\section{Winkelgeschwindigkeit 1. Ordnung}
Die Winkelgeschwindigkeit \(\omega(\phi)\) in der Weber-Gravitation bis zur Ordnung \(\mathcal{O}(c^{-2})\) lautet:

\begin{equation}
\omega(\phi) = \frac{h}{a^2(1 - e^2)^2} \left[1 + e \cos\left(\kappa\phi\right)\right]^2
\end{equation}

wobei:
\begin{itemize}
    \item \(h = \sqrt{GMa(1 - e^2)}\) der spezifische Drehimpuls ist,
    \item \(\kappa = \sqrt{1 - \frac{6GM}{c^2a(1 - e^2)}}\),
    \item Terme der Ordnung \(\mathcal{O}(c^{-4})\) (z. B. \(\alpha\phi^2\)) werden vernachlässigt.
\end{itemize}

\subsection*{Bedeutung der Terme}
\begin{itemize}
    \item Die Wurzel \(\kappa\) beschreibt die Periheldrehung 1. Ordnung ohne Näherung.
    \item Für \(c \to \infty\) wird \(\kappa = 1\), und die Gleichung reduziert sich auf die Newton’sche Form:
    \[
    \omega_N(\phi) = \frac{h(1 + e \cos\phi)^2}{a^2(1 - e^2)^2}.
    \]
\end{itemize}

\section{Winkelgeschwindigkeit 2. Ordnung}

\subsection{Entwicklung von \(\kappa\)}
Die Konstante \(\kappa\) muss bis zur 2. Ordnung präzise sein:
\begin{equation}
\kappa = \sqrt{1 - \frac{6GM}{c^2a(1-e^2)} + \frac{27G^2M^2}{2c^4a^2(1-e^2)^2}}
\end{equation}

\subsection{Winkelgeschwindigkeit}
Mit dem exakten \(\kappa\) und \(\alpha = \frac{3G^2M^2e}{8h^4c^4}\):
\begin{equation}
\boxed
{
    \omega(\phi) = \frac{h[1 + e\cos(\kappa\phi + \alpha\phi^2)]^2}{a^2(1-e^2)^2}
}
\end{equation}
