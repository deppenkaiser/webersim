\section{Exakte Formulierung von \(\omega(\phi)\)}

\subsection{Korrekte Entwicklung von \(\kappa\)}
Die Konstante \(\kappa\) muss bis zur 2. Ordnung präzise sein:
\begin{equation}
\kappa = \sqrt{1 - \frac{6GM}{c^2a(1-e^2)} + \frac{27G^2M^2}{2c^4a^2(1-e^2)^2}}
\end{equation}

\subsection{Winkelgeschwindigkeit}
Mit dem exakten \(\kappa\) und \(\alpha = \frac{3G^2M^2e}{8h^4c^4}\):
\begin{equation}
\boxed
{
    \omega(\phi) = \frac{h[1 + e\cos(\kappa\phi + \alpha\phi^2)]^2}{a^2(1-e^2)^2}
}
\end{equation}

\subsection{Konsistenzcheck}
\begin{itemize}
\item \textbf{Newton-Limit} (\(c \to \infty\)):
\begin{equation}
\kappa \to 1,\quad \alpha \to 0 \quad \Rightarrow \quad \omega_N = \frac{h(1+e\cos\phi)^2}{a^2(1-e^2)^2}
\end{equation}

\item \textbf{1. Ordnung}:
\begin{equation}
\omega_{1} = \frac{h}{a^2(1-e^2)^2}\left[1 + 2e\cos\left(\phi - \frac{3GM\phi}{c^2a(1-e^2)}\right)\right]
\end{equation}
\end{itemize}

\subsection{Bedeutung der Terme}
\begin{tabular}{|l|l|}
\hline
Term & Physikalische Wirkung \\ \hline
\(\frac{6GM}{c^2a(1-e^2)}\) & Periheldrehung (43''/Jh für Merkur) \\ \hline
\(\frac{27G^2M^2}{2c^4a^2(1-e^2)^2}\) & Korrektur zur ART-Vorhersage \\ \hline
\(\alpha\phi^2\) & Nichtperiodische Bahnstörung \\ \hline
\end{tabular}
