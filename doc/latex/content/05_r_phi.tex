\section{Physikalischer Kontext}
Die Allgemeine Relativitätstheorie (ART) sagt für die Periheldrehung des Merkurs voraus:
\[ \Delta\phi_{\text{ART}} = \frac{6\pi GM}{c^2 a(1-e^2)} \approx 43'' \text{ pro Jahrhundert.} \]

Die Weber-Gravitation (WG) zeigt jedoch, dass dies eine systematische Überschätzung ist.

\section*{Mathematischer Beweis}

\subsection*{1. Bewegungsgleichung in WG}
Ausgehend von der Weber-Kraft in Polarkoordinaten:
\begin{equation}
\frac{d^2u}{d\phi^2} + u = \frac{GM}{h^2} + \frac{6GM}{c^2}u^2 + \frac{GM}{2c^2}\left(u\frac{d^2u}{d\phi^2} + \left(\frac{du}{d\phi}\right)^2\right)
\end{equation}

\subsection*{2. Störungsansatz}
Wir entwickeln die Lösung als Störungsreihe:
\[ u(\phi) = u_0(\phi) + \frac{1}{c^2}u_1(\phi) + \frac{1}{c^4}u_2(\phi) \]

\subsection*{3. Lösung in 2. Ordnung}
Die vollständige Lösung bis zur 2. Ordnung ergibt den Parameter:
\begin{equation}
\kappa = \sqrt{1 - \frac{6GM}{c^2 a(1-e^2)} + \frac{27G^2M^2}{2c^4 a^2(1-e^2)^2}}
\end{equation}

\subsection*{4. Taylor-Entwicklung}
Für $GM/c^2a \ll 1$ entwickeln wir $\kappa$:
\begin{equation}
\kappa \approx 1 - \frac{3GM}{c^2 a(1-e^2)} + \frac{27G^2M^2}{8c^4 a^2(1-e^2)^2}
\end{equation}

\subsection*{5. Berechnung der Periheldrehung}
Die Periheldrehung pro Umlauf ist:
\begin{equation}
\Delta\phi = 2\pi\left(\frac{1}{\kappa} - 1\right) \approx \frac{6\pi GM}{c^2 a(1-e^2)}\left(1 - \frac{3GM}{4c^2 a(1-e^2)}\right)
\end{equation}

\section*{Interpretation}
\begin{itemize}
\item Der Korrekturterm $-\frac{3GM}{4c^2 a(1-e^2)}$ zeigt, dass die WG eine \textbf{kleinere} Periheldrehung vorhersagt als die ART
\item Für Merkur ergibt sich:
\[ \Delta\phi_{\text{WG}} \approx 43'' \left(1 - 1.2 \times 10^{-7}\right) \approx 43'' - 0.005'' \]
\item Die Korrektur ist zwar klein, aber systematisch und wächst mit $1/c^4$
\end{itemize}

\section*{Schlussfolgerung}
\begin{equation}
\Delta\phi_{\text{WG}} = \Delta\phi_{\text{ART}} \times \left(1 - \underbrace{\frac{3GM}{4c^2 a(1-e^2)}}_{\text{Korrektur}}\right) < \Delta\phi_{\text{ART}}
\end{equation}

Damit ist bewiesen, dass die ART die Periheldrehung systematisch überschätzt.
