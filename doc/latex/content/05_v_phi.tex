\section{Bahngeschwindigkeit in Weber-Gravitation}
\subsection*{Definition}
Die Bahngeschwindigkeit $v(\phi)$ ergibt sich aus Winkelgeschwindigkeit $\omega(\phi)$ und Radialabstand $r(\phi)$:
\begin{equation}
v(\phi) = \omega(\phi) \cdot r(\phi) = \frac{h}{r(\phi)}
\end{equation}

\subsection{Explizite Formel}
Mit der Bahngleichung:
\begin{equation}
r(\phi) = \frac{a(1-e^2)}{1 + e\cos\left(\kappa\phi + \alpha\phi^2/c^4\right)}
\end{equation}
folgt:
\begin{equation}\boxed{
v(\phi) = \frac{h}{a(1-e^2)} \left(1 + e\cos\left[\left(1 - \frac{3GM}{c^2a(1-e^2)} + \frac{9G^2M^2}{8c^4a^2(1-e^2)^2}\right)\phi + \frac{3G^2M^2e}{8c^4h^4}\phi^2\right]\right)
}\end{equation}

\subsection*{Physikalische Terme}
\begin{itemize}[leftmargin=*,noitemsep]
    \item $h$: Drehimpuls pro Masse ($h = r^2\dot{\phi}$)
    \item $\kappa\phi$: Periheldrehung (1. Ordnung in $c^{-2}$)
    \item $\alpha\phi^2/c^4$: Nichtlineare Bahnstörung (2. Ordnung)
\end{itemize}

\subsection*{Grenzfälle}
\begin{align*}
    \text{Newton: } & c \to \infty \Rightarrow v_N = \frac{h(1+e\cos\phi)}{a(1-e^2)} \\
    \text{Perihel: } & \phi=0 \Rightarrow v(0) = \frac{h(1+e)}{a(1-e^2)} \\
    \text{Aphel: } & \phi=\pi \Rightarrow v(\pi) = \frac{h(1-e)}{a(1-e^2)}
\end{align*}