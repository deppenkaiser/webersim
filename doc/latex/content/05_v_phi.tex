\newpage
\section{Bahngeschwindigkeit in 1. Ordnung}
Die Bahngeschwindigkeit \(v(\phi)\) in der Weber-Gravitation bis zur Ordnung \(\mathcal{O}(c^{-2})\) lautet:

\begin{equation}
v(\phi) = \frac{h}{a(1 - e^2)} \left(1 + e \cos\left(\kappa\phi\right)\right)
\end{equation}

\noindent mit den Definitionen:
\begin{align*}
h &= \sqrt{GMa(1 - e^2)}, \\
\kappa &= \sqrt{1 - \frac{6GM}{c^2a(1 - e^2)}}.
\end{align*}

\subsection*{Physikalische Interpretation}
\begin{itemize}
    \item \textbf{Struktur}: Die Geschwindigkeit folgt aus \(v(\phi) = h/r(\phi)\) mit der Bahngleichung \(r(\phi) = \frac{a(1 - e^2)}{1 + e \cos(\kappa\phi)}\).
    \item \textbf{Relativistische Korrektur}: Die Wurzel \(\kappa\) modifiziert die Periheldrehung gegenüber Newton (\(\kappa = 1\)).
    \item \textbf{Grenzfälle}:
        \begin{itemize}
            \item Perihel (\(\phi = 0\)): \(v(0) = \frac{h(1 + e)}{a(1 - e^2)}\),
            \item Aphel (\(\phi = \pi\)): \(v(\pi) = \frac{h(1 - e)}{a(1 - e^2)}\),
            \item Newton (\(c \to \infty\)): \(v_N(\phi) = \frac{h(1 + e \cos\phi)}{a(1 - e^2)}\).
        \end{itemize}
\end{itemize}

\section{Bahngeschwindigkeit in 2. Ordnung}
Die Bahngeschwindigkeit $v(\phi)$ ergibt sich aus Winkelgeschwindigkeit $\omega(\phi)$ und Radialabstand $r(\phi)$:
\begin{equation}
v(\phi) = \omega(\phi) \cdot r(\phi) = \frac{h}{r(\phi)}
\end{equation}

Mit der Bahngleichung und Winkelgeschwindigkeit:
\begin{align}
r(\phi) &= \frac{a(1-e^2)}{1 + e\cos\left(\kappa\phi + \alpha\phi^2\right)}\\
\omega(\phi) &= \frac{h[1 + e\cos(\kappa\phi + \alpha\phi^2)]^2}{a^2(1-e^2)^2}
\end{align}

ergibt sich:
\begin{equation}
v(\phi) = \frac{h \left(1 + e\cos(\kappa\phi + \alpha\phi^2)\right)}{a(1 - e^2)}.
\end{equation}
