\section{Klassische Weber-Kraft (Elektrodynamik)}
\begin{equation}
    \boxed
    {
        \bm{F}_{\text{Weber}}^{\text{EM}} = \frac{Qq}{4\pi\epsilon_0 r^2}\left(1 - \frac{\dot{r}^2}{c^2} + \frac{2r\ddot{r}}{c^2}\right)\bm{\hat{r}}
    }
\end{equation}

\subsection*{Symbolbeschreibung}
\begin{itemize}[leftmargin=*,noitemsep]
    \item $\bm{F}_{\text{Weber}}^{\text{EM}}$: Weber-Kraft zwischen Ladungen
    \item $Q, q$: Elektrische Ladungen
    \item $\epsilon_0$: Elektrische Feldkonstante
    \item $r$: Ladungsabstand
    \item $\dot{r} = \frac{dr}{dt}$: Relative Radialgeschwindigkeit
    \item $\ddot{r} = \frac{d^2r}{dt^2}$: Relative Radialbeschleunigung
    \item $c$: Lichtgeschwindigkeit
    \item $\bm{\hat{r}}$: Radialer Einheitsvektor
\end{itemize}

\subsection*{Beziehung zur Coulomb-Kraft}
\begin{itemize}[leftmargin=*,noitemsep]
    \item Erster Term entspricht Coulomb-Kraft: $\frac{Qq}{4\pi\epsilon_0 r^2}$
    \item Zusatzterme $\left(-\frac{\dot{r}^2}{c^2} + \frac{2r\ddot{r}}{c^2}\right)$ beschreiben Bewegungsabhängige Korrekturen
    \item Reduktion auf Coulomb-Kraft im statischen Fall ($\dot{r} = \ddot{r} = 0$)
\end{itemize}

\subsection*{Vergleich mit Maxwell-Theorie}
\begin{itemize}[leftmargin=*,noitemsep]
    \item Alternative Beschreibung elektromagnetischer Phänomene \cite{weber1846}
    \item Fernwirkungsansatz (direkte Ladungswechselwirkung)
    \item Implizite Retardierung durch Geschwindigkeits-/Beschleunigungsterme
    \item Keine Vorhersage von EM-Wellen im Vakuum
\end{itemize}

\subsection{Ansatz zur Weber-Gravitation (WG)}
\begin{itemize}[leftmargin=*,noitemsep]
    \item Kein vordefiniertes Raummodell benötigt
    \item Natürliche Diskretisierung durch Punktteilchen
    \item Gravitative Erweiterung möglich:
\end{itemize}

\begin{equation}
\bm{F}_{\text{Weber}}^{G} = G\frac{mM}{r^2}\left(1 - \frac{\dot{r}^2}{c^2} + \frac{2r\ddot{r}}{c^2}\right)\bm{\hat{r}}
\end{equation}

\section{Weber-Kraft und Gravitation}
\subsection*{Tisserands Ansatz}
Die Übertragung der elektrodynamischen Weber-Kraft \cite{tisserand1894} auf die Gravitation scheiterte an der Erklärung der Periheldrehung des Merkurs.

\subsection*{Hinweis}
Die korrekte gravitative Formulierung wird separat vorgestellt und erfordert eine Modifikation der Original-Weberschen Formel.
