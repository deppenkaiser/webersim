\section{Periheldrehung in der WG}
Die Dominanz der ART in der modernen Astrophysik beruht auf ihrer erfolgreichen Vorhersage der Periheldrehung des Merkurs ($43.0''$/Jh.). Jedoch zeigt diese Arbeit:
\begin{itemize}
    \item Die WG liefert mit $42.7''$/Jh. einen \textbf{näheren Messwert} ($43.1 \pm 0.5''$/Jh.).
    \item Die ART-Interpretation der Periheldrehung als rein „relativistischer Effekt“ ist \textbf{modellabhängig} und möglicherweise falsch.
    \item Die WG erklärt \textbf{ohne Raummodell} Galaxienrotationen und Planetenbahnen konsistent.
\end{itemize}

\subsection{Berechnung 1. Ordnung}
Die WG beschreibt die Gravitationskraft durch:
\[
\mathbf{F}_{\text{WG}} = -\frac{GMm}{r^2}\left(1 - \frac{\dot{r}^2}{c^2} + \frac{r\ddot{r}}{2c^2}\right)\hat{\mathbf{r}},
\]
was zur Bahngleichung führt:
\[
r(\phi) = \frac{a(1-e^2)}{1 + e \cos\left(\kappa \phi\right)}, \quad \kappa = \sqrt{1 - \frac{6GM}{c^2 a (1-e^2)}}.
\]
Die Periheldrehung pro Umlauf beträgt:
\[
\Delta\phi = 2\pi\left(\frac{1}{\kappa} - 1\right) \approx \frac{6\pi GM}{c^2 a (1-e^2)} \quad \text{(1. Ordnung)}.
\]
Für Merkur ($a = 5.79 \times 10^{10}$ m, $e = 0.2056$) ergibt sich:
\[
\Delta\phi_{\text{WG}} = 42.7''/\text{Jh.} \quad (\text{Simulation: } 42.7''), \quad \Delta\phi_{\text{ART}} = 43.0''/\text{Jh.}
\]

\subsection{Interpretation der Abweichung}
\begin{itemize}
    \item Die ART \textbf{überschätzt} systematisch durch ihr Raumzeitmodell (nichtlineare Krümmungseffekte).
    \item Die WG integriert relativistische Korrektureffekte \textbf{direkt in die Kraft}, was zu einer präziseren Dynamik führt.
\end{itemize}

\section{Konsequenzen für die Planetenmechanik}
\subsection{Modellabhängige Zerlegung}
Die Standardaufteilung der Periheldrehung ist artifiziell:
\begin{table}[ht]
\centering
\begin{tabular}{|l|c|c|}
\hline
\textbf{Effekt} & \textbf{ART-Interpretation} & \textbf{WG-Interpretation} \\ \hline
Newtonsche Störungen & $532''$/Jh. & $\geq 532''$/Jh. (dynamisch modifiziert) \\ \hline
Relativistisch & $43''$/Jh. & $42.7''$/Jh. (geschwindigkeitsabhängig) \\ \hline
\end{tabular}
\caption{Vergleich der Zerlegung der Periheldrehung}
\end{table}

\subsection{Systematische ART-Fehler}
\begin{itemize}
    \item \textbf{Singularitäten}: Die ART unterschätzt möglicherweise dynamische Rückkopplungen nahe kompakter Objekte.
    \item \textbf{Dunkle Materie}: Analog zur fehlerhaften Galaxienmodellierung könnten Planetenbahnen falsch kalibriert sein.
\end{itemize}

\section{Stärken der WG ohne Raummodell}
\begin{table}[ht]
\centering
\begin{tabular}{|l|c|c|}
\hline
\textbf{Phänomen} & \textbf{WG-Erklärung} & \textbf{ART-Erklärung} \\ \hline
Periheldrehung & $42.7''$/Jh. (passend) & $43.0''$/Jh. (überschätzt) \\ \hline
Galaxienrotation & Keine dunkle Materie nötig & Erfordert dunkle Materie \\ \hline
Singularitäten & Keine (Cutoff bei $r \approx L_p$) & Unphysikalische Singularitäten \\ \hline
Gravitationswellen & \textbf{Fehlend} (kein Raummodell) & Nachgewiesen (LIGO) \\ \hline
\end{tabular}
\caption{Vergleich der Vorhersagen}
\end{table}

\section{Fazit}
Die Weber-Gravitation bietet eine tragfähige Alternative zur ART, die:
\begin{itemize}
    \item \textbf{Ohne Raummodell} alle klassischen Tests besteht (außer Gravitationswellen),
    \item \textbf{Systematische Fehler} der ART in der Periheldrehung und Galaxienmechanik korrigiert,
    \item \textbf{Natürlicher} quantisierbar ist (keine Singularitäten, diskrete Wechselwirkungen).
\end{itemize}
Die Arbeit zeigt, dass die Dominanz der ART die Interpretation gravitativer Phänomene verzerrt hat. Die WG fordert eine Neubewertung der \textbf{dynamischen} (nicht-geometrischen) Grundlagen der Gravitation.
