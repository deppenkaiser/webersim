\section{Verallgemeinerung der Kraftgleichung}

Die gegebene skalare Gleichung lautet:
\begin{equation}
F = -F_N \left(1 - \frac{\dot{r}^2}{c^2} + \frac{r\ddot{r}}{2c^2}\right)
\end{equation}

\subsection{Schritt 1: Vektorielle Darstellung der Ableitungen}

Für einen beliebigen Ortsvektor $\mathbf{r}(t)$ mit $r = |\mathbf{r}|$ gelten folgende Beziehungen:

\begin{align}
\dot{r} &= \frac{d}{dt}\sqrt{\mathbf{r}\cdot\mathbf{r}} = \frac{\mathbf{r}\cdot\dot{\mathbf{r}}}{r} = \frac{\mathbf{r}\cdot\mathbf{v}}{r} \\
\ddot{r} &= \frac{d}{dt}\left(\frac{\mathbf{r}\cdot\mathbf{v}}{r}\right) = \frac{v^2 + \mathbf{r}\cdot\mathbf{a}}{r} - \frac{(\mathbf{r}\cdot\mathbf{v})^2}{r^3}
\end{align}

\subsection{Schritt 2: Einsetzen in die Originalgleichung}

Wir setzen die Ausdrücke für $\dot{r}$ und $\ddot{r}$ ein:

\begin{align}
F &= -F_N \left[1 - \frac{1}{c^2}\left(\frac{\mathbf{r}\cdot\mathbf{v}}{r}\right)^2 + \frac{r}{2c^2}\left(\frac{v^2 + \mathbf{r}\cdot\mathbf{a}}{r} - \frac{(\mathbf{r}\cdot\mathbf{v})^2}{r^3}\right)\right] \\
&= -F_N \left[1 - \frac{(\mathbf{r}\cdot\mathbf{v})^2}{c^2 r^2} + \frac{v^2 + \mathbf{r}\cdot\mathbf{a}}{2c^2} - \frac{(\mathbf{r}\cdot\mathbf{v})^2}{2c^2 r^2}\right]
\end{align}

\subsection{Schritt 3: Vereinfachung und Endergebnis}

Zusammenfassen der Terme ergibt die verallgemeinerte \textbf{\enquote{vektorielle Form}}:

\begin{equation}
    \boxed
    {
        F = -F_N \left(1 + \frac{v^2}{2c^2} + \frac{\mathbf{r}\cdot\mathbf{a}}{2c^2} - \frac{3(\mathbf{r}\cdot\mathbf{v})^2}{2c^2 r^2}\right)
    }
\end{equation}

\subsection{Physikalische Interpretation}

\begin{itemize}
\item $\dfrac{v^2}{2c^2}$: Relativistische Korrektur zur kinetischen Energie
\item $\dfrac{\mathbf{r}\cdot\mathbf{a}}{2c^2}$: Beschleunigungsabhängiger Term
\item $\dfrac{3(\mathbf{r}\cdot\mathbf{v})^2}{2c^2 r^2}$: Richtungsabhängige Korrektur für nicht-radiale Bewegung
\end{itemize}

\subsection{Spezialfall: Radiale Bewegung}

Für $\mathbf{v} \parallel \mathbf{r}$ ($\mathbf{r}\cdot\mathbf{v} = rv$) und $\mathbf{a} = 0$:

\begin{equation}
F = -F_N \left(1 + \frac{v^2}{2c^2} - \frac{3v^2}{2c^2}\right) = -F_N \left(1 - \frac{v^2}{c^2}\right)
\end{equation}

\section{Bahngleichungen}
\subsection{Bahngleichung 1. Ordnung}
Die Bahngleichung \(r(\phi)\) in der Weber-Gravitation bis zur Ordnung \(\mathcal{O}(c^{-2})\) lautet:

\begin{equation}
\label{eq:bahngleichung_1_ordnung}
\boxed
{
    r(\phi) = \frac{a(1 - e^2)}{1 + e \cos\left(\kappa\phi\right)}
}
\end{equation}

\noindent mit der Definition:
\begin{equation}
\label{eq:kappa_1_ordnung}
\boxed
{
    \kappa = \sqrt{1 - \frac{6GM}{c^2a(1 - e^2)}}    
}
\end{equation}

\subsection{Bahngleichung 2. Ordnung}
Bahngleichung:
\begin{equation}
\label{eq:bahngleichung_2_ordnung}
    \boxed
    {
        r(\phi) = \frac{a(1-e^2)}{1 + e\cos\left(\kappa\phi + \alpha\phi^2\right)}
    }
\end{equation}

mit:
$h$ aus Gleichung (\ref{eq:spezifischer_drehimpuls_h})
\begin{equation}
\label{eq:kappa_2_ordnung}
\boxed
{
    \kappa = \sqrt{1 - \frac{6GM}{c^2a(1-e^2)} + \frac{27G^2M^2}{2c^4a^2(1-e^2)^2}}
}
\end{equation}
\begin{equation}
\label{eq:alpha}
\boxed
{
    \alpha = \frac{3G^2M^2e}{8h^4c^4}    
}
\end{equation}

\section{Periheldrehung}
\subsection{Periheldrehung 1. Ordnung}
Die Periheldrehung $\Delta\phi$ in der Weber-Gravitation ergibt sich aus der modifizierten Bahngleichung und lässt sich wie folgt herleiten:

\subsection*{Perihelbedingung}
Das Perihel (sonnennächster Punkt) tritt auf, wenn der Nenner maximal wird, d.h. wenn:\\
\[\cos(\kappa\phi) = 1\]
Die Lösungen dieser Bedingung sind: $\kappa\phi = 2\pi n \quad \text{(für $n \in \mathbb{Z}$)}$.\\

Somit ergeben sich die Winkel für aufeinanderfolgende Periheldurchgänge zu:
\[
    \phi_n = \frac{2\pi n}{\kappa}.
\]

\subsection*{Periheldrehung pro Umlauf}
Die Periheldrehung $\Delta\phi$ ist die Differenz zwischen dem Winkel für einen vollständigen Umlauf ($n = 1$) und dem Newton'schen Fall ($\kappa = 1$):
\[
    \Delta\phi = \phi_1 - 2\pi = \frac{2\pi}{\kappa} - 2\pi.
\]
Daraus folgt die gesuchte Gleichung:
\begin{equation}
\boxed
{
    \Delta\phi = 2\pi\left(\frac{1}{\kappa} - 1\right)
}.
\end{equation}

\subsection*{Interpretation}
\begin{itemize}
\item Im Newton'schen Grenzfall ($\kappa = 1$) verschwindet die Periheldrehung ($\Delta\phi = 0$).
\item Für $\kappa < 1$ (Weber-Gravitation) ergibt sich eine positive Periheldrehung, die mit Beobachtungen (z.B. Merkurperihel) übereinstimmt.
\end{itemize}

\subsection{Periheldrehung 2. Ordnung}
\subsection*{Entwicklung von $\kappa$}
Eine Taylor-Entwicklung von $\kappa$ bis zur 2. Ordnung liefert:
\[
    \kappa \approx 1 - \frac{3GM}{c^2 a(1 - e^2)} + \frac{27G^2 M^2}{4c^4 a^2 (1 - e^2)^2} + \mathcal{O}(c^{-6}).
\]

\subsection*{Perihelbedingung}
Das Perihel tritt auf bei:
\[
    \cos\left(\kappa\phi + \alpha\phi^2\right) = 1 \quad \Rightarrow \quad \kappa\phi + \alpha\phi^2 = 2\pi n.
\]

\subsection*{Lösung für $\Delta\phi$}
Für $n=1$ (ein Umlauf) ergibt sich die quadratische Gleichung:
\[
\alpha\phi^2 + \kappa\phi - 2\pi = 0.
\]
Die Lösung lautet:
\begin{equation}
\phi = \frac{-\kappa + \sqrt{\kappa^2 + 8\pi\alpha}}{2\alpha}.
\end{equation}

\subsection*{Näherung für kleine Korrekturen}
Da $\alpha \sim c^{-4}$ klein ist, entwickeln wir die Wurzel:
\[
    \phi \approx \frac{2\pi}{\kappa} - \frac{4\pi^2\alpha}{\kappa^3} + \mathcal{O}(\alpha^2)
\]
Die Periheldrehung pro Umlauf wird damit:
\begin{equation}
    \boxed
    {
        \Delta\phi = \phi - 2\pi \approx 2\pi\left(\frac{1}{\kappa} - 1\right) - \frac{4\pi^2\alpha}{\kappa^3}
    }
\end{equation}
