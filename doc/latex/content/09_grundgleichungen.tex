\newpage
\section{Weber-Gravitation als Alternative zur ART}
Die allgemeine Relativitätstheorie (ART) gilt als der Goldstandard der modernen Astrophysik, allerdings werden bestimmte Aspekte dieser Theorie
nicht objektiv betrachtet. Die ART überzeugt durch die Fähigkeit die Merkur-Periheldrehung vorhersagen zu können, aber auch durch die Vorhersage
der Gravitationswellen. Das sind große Leistungen dieser Gravitationstheorie.

Auf der anderen Seite liefert sie unphysikalische Ergebnisse für schwarze Löcher und für galaktische Skalen. Schwarze Löcher werden als Singularitäten
dargestellt, wobei davon ausgegangen werden muss, dass die gravitativen Verhältnisse in der Nähe dieser Singularitäten ebenfalls ungenau sein müssen. Die
Rotationskurven von Galaxien werden nicht korrekt Vorhergesagt, weswegen die ART \enquote{dunkle Materie} benötigt.

\subsection{Grundgleichungen der Weber-Gravitation}
\subsection*{Weber-Gravitations Gleichung}
\begin{equation}
\mathbf{F} = -\frac{GMm}{r^2}\left(1 - \frac{\dot{r}^2}{c^2} + \frac{r\ddot{r}}{2c^2}\right)\mathbf{\hat{r}}
\end{equation}

\subsection*{Spezifischer Drehimpuls}
Der Drehimpuls pro Masseneinheit $h$ ist definiert als:
\begin{equation}
\label{eq:spezifischer_drehimpuls_h}
h = r^2\dot{\varphi} = \sqrt{GMa(1-e^2)}
\end{equation}
wobei $a$ die große Halbachse und $e$ die Exzentrizität der Bahn ist.

\subsection{Bewegungsgleichung in Polarkoordinaten}
\begin{equation}
\mathbf{a} = \left(\ddot{r} - r\dot{\varphi}^2\right)\mathbf{\hat{r}} + \left(r\ddot{\varphi} + 2\dot{r}\dot{\varphi}\right)\mathbf{\hat{\varphi}} = -\frac{GM}{r^2}\left(1 - \frac{\dot{r}^2}{c^2} + \frac{r\ddot{r}}{2c^2}\right)\mathbf{\hat{r}}
\end{equation}

\subsection*{Variablenbeschreibung}
\begin{itemize}[leftmargin=*,noitemsep]
    \item $\mathbf{F}$: Gravitationskraftvektor (Weber-Kraft) [N]
    \item $\mathbf{a}$: Beschleunigungsvektor [m/s²]
    \item $G$: Gravitationskonstante [m³/kg/s²]
    \item $M$, $m$: Massen der wechselwirkenden Körper [kg]
    \item $r$: Abstand zwischen den Massenschwerpunkten [m]
    \item $\dot{r} = \frac{dr}{dt}$: Radiale Relativgeschwindigkeit [m/s]
    \item $\ddot{r} = \frac{d^2r}{dt^2}$: Radiale Relativbeschleunigung [m/s²]
    \item $c$: Lichtgeschwindigkeit [m/s]
    \item $\varphi$: Azimutwinkel [rad]
    \item $\dot{\varphi} = \frac{d\varphi}{dt}$: Winkelgeschwindigkeit [rad/s]
    \item $\ddot{\varphi} = \frac{d^2\varphi}{dt^2}$: Winkelbeschleunigung [rad/s²]
    \item $h$: Spezifischer Drehimpuls [m²/s]
    \item $\mathbf{\hat{r}}$: Radialer Einheitsvektor (zeigt von $M$ zu $m$)
    \item $\mathbf{\hat{\varphi}}$: Azimutaler Einheitsvektor (senkrecht zu $\mathbf{\hat{r}}$)
\end{itemize}

\subsection*{Physikalische Interpretation}
\begin{itemize}[leftmargin=*,noitemsep]
    \item Der Term $-\frac{GMm}{r^2}$ entspricht der klassischen Newton'schen Gravitation
    \item $\frac{\dot{r}^2}{c^2}$: Relativistische Korrektur für radiale Bewegung
    \item $\frac{r\ddot{r}}{2c^2}$: Korrektur für radiale Beschleunigung
    \item $r\dot{\varphi}^2$: Zentripetalbeschleunigung
    \item $2\dot{r}\dot{\varphi}$: Coriolis-Term
    \item $h$: Erhaltungsgröße für Planetenbahnen
\end{itemize}
