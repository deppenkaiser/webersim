\newpage
\section{Winkelgeschwindigkeit}
\subsection{Winkelgeschwindigkeit 1. Ordnung}
Die Winkelgeschwindigkeit \(\omega(\phi)\) in der Weber-Gravitation bis zur Ordnung \(\mathcal{O}(c^{-2})\) lautet:

\begin{equation}
\label{eq:winkelgeschwindigkeit_1_ordnung}
\boxed
{
    \omega(\phi) = \frac{h}{a^2(1 - e^2)^2} \left[1 + e \cos\left(\kappa\phi\right)\right]^2   
}
\end{equation}

wobei:
$h$ aus Gleichung (\ref{eq:spezifischer_drehimpuls_h}), $\kappa$ aus Gleichung (\ref{eq:kappa_1_ordnung}) stammt.

\subsection*{Bedeutung der Terme}
\begin{itemize}
    \item \(\kappa\) beschreibt die Periheldrehung 1. Ordnung ohne Näherung.
    \item Für \(c \to \infty\) wird \(\kappa = 1\), und die Gleichung reduziert sich auf die Newton’sche Form:
    \[
    \omega_N(\phi) = \frac{h(1 + e \cos\phi)^2}{a^2(1 - e^2)^2}.
    \]
\end{itemize}

\subsection{Winkelgeschwindigkeit 2. Ordnung}

\subsection*{Winkelgeschwindigkeit}
Mit $h$ aus Gleichung (\ref{eq:spezifischer_drehimpuls_h}), $\kappa$ aus Gleichung (\ref{eq:kappa_2_ordnung}) und $\alpha$ aus Gleichung (\ref{eq:alpha}):
\begin{equation}
\label{eq:winkelgeschwindigkeit_2_ordnung}
\boxed
{
    \omega(\phi) = \frac{h[1 + e\cos(\kappa\phi + \alpha\phi^2)]^2}{a^2(1-e^2)^2}
}
\end{equation}
