\section{Bahngeschwindigkeit}
\subsection{Bahngeschwindigkeit in 1. Ordnung}
Die Bahngeschwindigkeit \(v(\phi)\) in der Weber-Gravitation bis zur Ordnung \(\mathcal{O}(c^{-2})\) lautet:

\begin{equation}
    \boxed
    {
        v(\phi) = \frac{h}{a(1 - e^2)} \left(1 + e \cos\left(\kappa\phi\right)\right)   
    }
\end{equation}

\noindent mit den Definitionen:
$h$ aus Gleichung (\ref{eq:spezifischer_drehimpuls_h}), $\kappa$ aus Gleichung (\ref{eq:kappa_1_ordnung})

\subsection*{Physikalische Interpretation}
\textbf{Grenzfälle}:
\begin{itemize}
    \item Perihel (\(\phi = 0\)): \(v(0) = \frac{h(1 + e)}{a(1 - e^2)}\),
    \item Aphel (\(\phi = \pi\)): \(v(\pi) = \frac{h(1 - e)}{a(1 - e^2)}\),
    \item Newton (\(c \to \infty\)): \(v_N(\phi) = \frac{h(1 + e \cos\phi)}{a(1 - e^2)}\).
\end{itemize}

\subsection{Bahngeschwindigkeit in 2. Ordnung}
Die Bahngeschwindigkeit $v(\phi)$ ergibt sich aus Winkelgeschwindigkeit $\omega(\phi)$ und Radialabstand $r(\phi)$:
\[
v(\phi) = \omega(\phi) \cdot r(\phi) = \frac{h}{r(\phi)}
\]

Mit der Bahngleichung (\ref{eq:bahngleichung_2_ordnung}), der Winkelgeschwindigkeit (\ref{eq:winkelgeschwindigkeit_2_ordnung}) und $h$ Gl. (\ref{eq:spezifischer_drehimpuls_h}) ergibt sich:
\begin{equation}
    \boxed
    {
        v(\phi) = \frac{h \left(1 + e\cos(\kappa\phi + \alpha\phi^2)\right)}{a(1 - e^2)}.   
    }
\end{equation}
