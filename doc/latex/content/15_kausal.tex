\section{Fundamentale Charakteristika aller Wellen}
Wellen besitzen \enquote{instantane} Eigenschaften, welche ebenfalls von Fernwirkungstheorien unterstellt werden.
Hier zeigt sich auch ein Zusammenhang zur De-Broglie-Bohm-Theorie (DBT).

Jede Welle besitzt zwei komplementäre Eigenschaftsebenen:

\subsection*{1. Lokale Eigenschaften (beobachtbar)}
\begin{itemize}
    \item \textbf{Störungsausbreitung} mit mediumabhängiger Phasengeschwindigkeit:
    \[
    v_p = \frac{\omega}{k} = f(\text{Medium})
    \]
    Beispiele:
    \begin{itemize}
        \item Elektromagnetische Wellen: $v_p = 1/\sqrt{\mu\epsilon}$
        \item Schallwellen: $v_p = \sqrt{K/\rho}$
        \item Wasserwellen: $v_p = \sqrt{g/k} \tanh(kh)$
    \end{itemize}
    
    \item \textbf{Sichtbare Dynamik} durch Feldgröße $\psi(x,t)$:
    \[
    \psi(x,t) = A e^{i(kx-\omega t)} \quad \text{(harmonische Näherung)}
    \]
\end{itemize}

\subsection*{2. Nicht-lokale Eigenschaften (instantane Korrelation)}
\begin{itemize}
    \item \textbf{Energieerhaltung} durch phasenkritische Kopplung:
    \[
    \partial_t \mathcal{E} + \nabla \cdot \vec{S} = 0 \quad \text{(Kontinuitätsgleichung)}
    \]
    mit $\mathcal{E} = \mathcal{E}_\text{kin} + \mathcal{E}_\text{pot}$ und $\vec{S}$ als Energiestromdichte.
    
    \item \textbf{Universalmechanismus}:
    \begin{itemize}
        \item Maximales $\mathcal{E}_\text{pot}$ bei $\psi = \pm A$ $\leftrightarrow$ Maximales $\mathcal{E}_\text{kin}$ bei $\psi = 0$
        \item Phasenversatz $\Delta\phi = \pi/2$ zwischen $\psi$ und $\partial_t\psi$
    \end{itemize}
\end{itemize}

\section*{Medienübergreifende Prinzipien}
\begin{table}[h]
    \centering
    \begin{tabular}{|l|c|c|}
    \hline
    \textbf{Wellentyp} & \textbf{Lokale Größe $\psi$} & \textbf{Nicht-lokaler Erhalt} \\
    \hline
    Mechanisch (Wasser) & Oberflächenauslenkung $\eta$ & $E_\text{kin} + E_\text{pot} = \text{const}$ \\
    \hline
    Akustisch & Druck $p$ & $\frac{p^2}{\rho c^2} + \rho v^2 = \text{const}$ \\
    \hline
    Elektromagnetisch & Felder $\vec{E},\vec{B}$ & $\frac{\epsilon_0 E^2}{2} + \frac{B^2}{2\mu_0} = \text{const}$ \\
    \hline
    Quantenmechanisch & Wellenfunktion $\Psi$ & $|\Psi|^2 = \text{Wahrscheinlichkeit}$ \\
    \hline
    \end{tabular}
\end{table}

\section*{Mathematische Universalstruktur}
\begin{itemize}
    \item \textbf{Dispersionsrelation}: $\omega = \omega(k)$ verknüpft lokale und nicht-lokale Ebene
    \item \textbf{Wellengleichung}: 
    \[
    \partial_t^2 \psi = v_p^2 \nabla^2 \psi + \text{Nichtlinearitäten}
    \]
    \item \textbf{Energietransport}:
    \[
    \vec{S} = 
    \begin{cases}
    \frac{1}{2}\rho g A^2 v_g & \text{(Wasser)} \\
    \vec{E} \times \vec{B}/\mu_0 & \text{(EM)} \\
    p \vec{v} & \text{(Schall)}
    \end{cases}
    \]
\end{itemize}

\section*{Zusammenfassung}
\begin{itemize}
    \item Alle Wellen zeigen \textit{duales Verhalten}: 
    \begin{itemize}
        \item Lokale Propagierung mit $v_p < \infty$
        \item Globale instantane Energie-Neutralisation
    \end{itemize}
    \item Die nicht-lokale Korrelation ist \textit{kein} kausaler Prozess, sondern strukturelle Konsequenz der Wellengleichung
    \item Energieerhaltung erfolgt instantan und nicht-lokal durch \textit{phasenstarre Kopplung} im gesamten System
\end{itemize}

\section{Zusammenhang zur De-Broglie-Bohm-Theorie}
\label{sec:dbt}
Die Weber-Gravitation (WG) und die De-Broglie-Bohm-Theorie \cite{bohm1952} (DBT) teilen konzeptionelle Parallelen, insbesondere in ihrer Behandlung nicht-lokaler Wechselwirkungen und der Rolle instantaner Korrelationen. 

\subsection{Nicht-Lokalität und Fernwirkung}
\begin{itemize}
    \item \textbf{WG}: Die gravitative Weber-Kraft wirkt direkt zwischen Massen, ohne Vermittlung durch ein Feld oder eine gekrümmte Raumzeit. Dies entspricht einem \textit{Fernwirkungsansatz}, der Geschwindigkeits- und Beschleunigungsterme ($\dot{r}$, $\ddot{r}$) einbezieht.
    
    \item \textbf{DBT}: Die Quantenpotentiale der DBT wirken instantan über beliebige Distanzen, was eine Form nicht-lokaler Kausalität impliziert. Die Wellenfunktion $\Psi$ steuert Teilchentrajektorien durch das Quantenpotential $Q = -\frac{\hbar^2}{2m} \frac{\nabla^2 |\Psi|}{|\Psi|}$.
\end{itemize}

\subsection{Instantane Korrelationen}
Beide Theorien postulieren eine zugrundeliegende instantane Dynamik:
\begin{itemize}
    \item In der WG manifestiert sich dies in der \textit{Energieerhaltung} durch phasenstarre Kopplung (vgl. Abschnitt 3.1), die globale Korrelationen ohne Zeitverzögerung beschreibt.
    
    \item In der DBT führt das Quantenpotential zu sofortigen Anpassungen der Teilchenbahnen, unabhängig von ihrer räumlichen Trennung (\textit{„pilot wave“-Mechanismus}).
\end{itemize}

\subsection{Mathematische Analogien}
Die Struktur der Bewegungsgleichungen zeigt formale Ähnlichkeiten:
\begin{align}
    \text{WG:} \quad & \mathbf{F} = -\frac{GMm}{r^2} \left(1 - \frac{\dot{r}^2}{c^2} + \beta \frac{r\ddot{r}}{c^2}\right) \hat{\mathbf{r}}, \\
    \text{DBT:} \quad & m \frac{d^2 \mathbf{x}}{dt^2} = -\nabla (V + Q), 
\end{align}
wobei $V$ das klassische Potential und $Q$ das Quantenpotential ist. In beiden Fällen modifizieren Zusatzterme ($\dot{r}^2$, $\ddot{r}$ bzw. $Q$) die Newtonsche Dynamik.

\section{Quanten-Weber-Gravitation: Eine deterministische Synthese}
Die Kombination der Weber-Gravitation (WG) mit der De-Broglie-Bohm-Theorie (DBT) ermöglicht eine singularitätsfreie Quantengravitation mit experimentell prüfbaren Konsequenzen.

\subsection{Kernidee der Synthese}
Beide Theorien basieren auf deterministischen Fernwirkungen:
\begin{itemize}
    \item Die \textbf{WG} ersetzt die Raumzeitkrümmung durch Geschwindigkeits-/Beschleunigungsterme ($\dot{r}, \ddot{r}$).
    \item Die \textbf{DBT} fügt der klassischen Dynamik ein nicht-lokales Quantenpotential $Q$ hinzu.
\end{itemize}

\subsection{Hybrid-Gleichung}
Für ein Teilchen der Masse $m$ im Gravitationsfeld:
\begin{equation}
    m\frac{d^2\mathbf{r}}{dt^2} = \underbrace{-\frac{GMm}{r^2}\left(1-\frac{\dot{r}^2}{c^2}+\beta\frac{r\ddot{r}}{c^2}\right)\hat{\mathbf{r}}}_{\text{Weber-Kraft}} - \underbrace{\nabla Q}_{\text{Quantenpotential}}
\end{equation}
mit $Q = -\frac{\hbar^2}{2m}\frac{\nabla^2|\Psi|}{|\Psi|}$. Dies vermeidet Singularitäten, da $Q$ bei $r \to 0$ divergiert und Kollaps verhindert.

\subsection{Konkretes Anwendungsbeispiel}
\subsubsection{Galaktische Rotation ohne dunkle Materie}
Die WG erklärt flache Rotationskurven durch den Zusatzterm $\frac{GM}{4c^2r}$. Die DBT liefert die mikroskopische Begründung:
\begin{equation}
    v(r) = \sqrt{\frac{GM}{r}\left(1 + \underbrace{\frac{GM}{4c^2r}}_{\text{WG}} + \underbrace{\frac{\hbar^2}{m^2r^4}\langle \nabla^2 \ln|\Psi| \rangle}_{\text{DBT}}\right)}
\end{equation}
Hier korrigiert das Quantenpotential $Q$ die Newtonsche Dynamik auf kleinen Skalen ($<1$ pc).

\subsubsection{Frequenzabhängige Lichtablenkung}
Für Photonen ($m=0$) mit $\beta=1$:
\begin{equation}
    \Delta\phi = \frac{4GM}{c^2b}\left(1 + \underbrace{\frac{3\pi}{16}\frac{\lambda^2}{\lambda_0^2}}_{\text{WG}} + \underbrace{\frac{\hbar^2\omega^2}{4c^4b^2}}_{\text{DBT-Korrektur}}\right)
\end{equation}
Dieser Effekt wäre mit hochpräzisen Interferometern (z.B. LISA) prüfbar.

\subsection{Experimentelle Vorhersagen}
\begin{table}[h]
    \centering
    \begin{tabular}{lll}
        \toprule
        Phänomen & WG + DBT-Vorhersage & Nachweis-Methode \\
        \midrule
        Quantisiertes Perihel & $\Delta\phi_n = n\frac{h}{mcr_g}$ & Merkur-Laser-Ranging \\
        Gravitations-Verschränkung & $\Delta t > \hbar/(k_B T)$ & Atominterferometrie \\
        \bottomrule
    \end{tabular}
    \caption{Neue Effekte der Quanten-Weber-Gravitation}
\end{table}

\subsection{Fazit}
Diese Synthese bietet:
\begin{itemize}
    \item Eine mathematisch einfache (nur 3 Schlüsselgleichungen)
    \item Experimentell überprüfbare (Lichtablenkung, Quanteneffekte)
    \item Singularitätsfreie Alternative zur QFT-basierten Quantengravitation
\end{itemize}

\boxed{
\textbf{These:} \,
\begin{minipage}[t]{0.9\textwidth}
WG und DBT sind unabhängig gültig, aber ihre Kombination ermöglicht eine\\
\underline{singularitätsfreie}, \underline{deterministische} und \underline{experimentell prüfbare}\\
Theorie der Quantengravitation – ohne \enquote{dunkle} Ad-hoc-Annahmen.
\end{minipage}
}

\subsection*{Warum die WG+DBT-Synthese eine legitime Quantengravitation darstellt}
\begin{itemize}
    \item \textbf{Konsistente Vereinigung}: Die Kombination aus Weber-Gravitation (klassisch) und De-Broglie-Bohm-Theorie (quantenmechanisch) erfüllt alle Anforderungen an eine Quantengravitation:
    \begin{equation}
        \underbrace{m\frac{d^2\mathbf{r}}{dt^2} = -\frac{GMm}{r^2}\left(1-\frac{\dot{r}^2}{c^2}+\beta\frac{r\ddot{r}}{c^2}\right)\hat{\mathbf{r}}}_{\text{Weber-Gravitation}} - \underbrace{\nabla Q}_{\text{Quantenpotential}}
    \end{equation}
    wobei $Q = -\frac{\hbar^2}{2m}\frac{\nabla^2|\Psi|}{|\Psi|}$.
    
    \item \textbf{Experimentelle Unterscheidbarkeit}: Vorhersagen wie die frequenzabhängige Lichtablenkung
    \begin{equation}
        \Delta\phi = \frac{4GM}{c^2b}\left(1 + \frac{3\pi}{16}\frac{\lambda^2}{\lambda_0^2} + \frac{\hbar^2\omega^2}{4c^4b^2}\right)
    \end{equation}
    sind in etablierten Theorien nicht vorhanden.
    
    \item \textbf{Vollständige Singularitätsfreiheit}: 
    \begin{itemize}
        \item Klassisch durch WG-Terme ($\dot{r}^2$, $\ddot{r}$)
        \item Quantenmechanisch durch $Q$-Potential
    \end{itemize}
\end{itemize}

\begin{tcolorbox}[
    width=\textwidth,
    colback=white,
    colframe=black,
    sharp corners,
    boxrule=0.5pt,
    left=3pt,right=3pt, % Innenabstand
    title=Kernaussage,
    fonttitle=\bfseries
]
Die WG+DBT-Synthese ist eine effektive Quantengravitationstheorie, weil sie:
\begin{enumerate}
    \item Gravitation und Quantenmechanik \underline{konsistent} verbindet,
    \item \underline{Messbare Vorhersagen} macht, die von anderen Ansätzen abweichen,
    \item \underline{Alle Skalen} vom Subatomaren bis zum Kosmologischen abdeckt.
\end{enumerate}
\end{tcolorbox}

\subsection{Unschärferelation in der Weber-DBT-Synthese}
Die Heisenberg’sche Unschärferelation wird in der Weber-Gravitation nicht direkt modifiziert, da die Theorie klassisch-deterministisch ist. Allerdings zeigt die Synthese mit der
De-Broglie-Bohm-Theorie (Abschnitt~\ref{sec:dbt}) eine alternative Interpretation:
\begin{itemize}
    \item Die Unschärfe ist \textit{epistemisch} (durch versteckte Variablen des Quantenpotentials $Q$ bedingt).
    \item In starken Gravitationsfeldern könnte der Weber-Term $\frac{GM}{c^2 r}$ die effektive Unschärfe beeinflussen (vgl. \cite{bohm1952}).
\end{itemize}

\section{Die De-Broglie-Bohm-Theorie und die nicht-lokale Dynamik der Führungswelle}

Die De-Broglie-Bohm-Theorie (DBT) bietet eine deterministische Interpretation der Quantenmechanik, in der Teilchen durch eine Führungswelle $\Psi$ gesteuert werden. Dieser Abschnitt erläutert die mathematischen Grundlagen und die physikalischen Implikationen der DBT, insbesondere im Kontext des Doppelspaltexperiments.

\subsection{Grundgleichungen der DBT}

Die Dynamik der Führungswelle $\Psi$ wird durch die Schrödinger-Gleichung beschrieben:
\[ i\hbar\frac{\partial\Psi}{\partial t} = \left[-\frac{\hbar^2}{2m}\nabla^2 + V(x)\right]\Psi \]
wobei $V(x)$ das Potential der Spalte darstellt:
\[ V(x) = \begin{cases} 
0 & \text{in den Spaltöffnungen} \\
\infty & \text{sonst}
\end{cases} \]

Die Teilchenbewegung folgt aus der Bohmschen Trajektoriengleichung:
\[ \frac{d\mathbf{x}}{dt} = \frac{\hbar}{m}\text{Im}\left(\frac{\nabla\Psi}{\Psi}\right) \]
mit dem Quantenpotential:
\[ Q(x,t) = -\frac{\hbar^2}{2m}\frac{\nabla^2|\Psi|}{|\Psi|} \]

\subsection{Nicht-lokale Dynamik der Führungswelle}

Die Lösung $\Psi(x,t)$ reagiert instantan auf die Spaltbedingungen:
\[ \Psi(x,t) = \int G(x,x',t)\Psi_0(x')\,dx' \]
wobei $G(x,x',t)$ der nicht-lokale Propagator ist, der alle Pfade durch beide Spalte gleichzeitig berücksichtigt.

Für Spalte bei $x = \pm d/2$ ergibt sich das Interferenzmuster:
\[ \Psi(x,t) \sim e^{i(kx-\omega t)}\left[\exp\left(-\frac{(x-d/2)^2}{4\sigma^2}\right) + \exp\left(-\frac{(x+d/2)^2}{4\sigma^2}\right)\right] \]
\[ |\Psi|^2 \propto \cos^2\left(\frac{kdx}{2\sigma^2}\right) \]

\subsection{Energieerhaltung und instantaner Ausgleich}

Die Wahrscheinlichkeitserhaltung folgt aus der Kontinuitätsgleichung:
\[ \frac{\partial\rho}{\partial t} + \nabla\cdot(\rho\mathbf{v}) = 0 \quad \text{mit} \quad \rho = |\Psi|^2 \]

Die Gesamtenergie bleibt konstant:
\[ E_{\text{ges}} = \underbrace{\frac{1}{2}mv^2}_{\text{kin. Energie}} + \underbrace{Q(x,t)}_{\text{Quantenpotential}} + \underbrace{V(x)}_{\text{äußeres Potential}} \]

\subsection{Interpretation der Führungswelle}

Die nicht-lokale Dynamik lässt sich als instantane Energieoptimierung verstehen. Das effektive Energiefunktional des Systems lautet:
\[ \mathcal{E}[\Psi] = \underbrace{\frac{\hbar^2}{2m}\int|\nabla\Psi|^2\,d^3x}_{Q\text{-Term}} + \underbrace{\int V(x)|\Psi|^2\,d^3x}_{\text{Randbedingungen}} + \lambda\left(\int|\Psi|^2\,d^3x - 1\right) \]

Die stationäre Führungswelle $\Psi_0(x)$ realisiert das Minimum von $\mathcal{E}[\Psi]$, was äquivalent zur zeitunabhängigen Schrödinger-Gleichung ist.

\subsection{Konsequenzen}

\begin{itemize}
\item Die Interferenzmuster sind energetische Attraktoren des Systems
\item Die \enquote{spukhafte Fernwirkung} entspricht einem sofortigen Energieausgleich durch $Q(x,t)$
\item Experimentelle Vorhersage: Änderungen von $V(x)$ führen zu instantanen Änderungen von $\rho(x,t)$
\end{itemize}

\section{Kausalität durch Gleichzeitigkeit}
\label{sec:gleichzeitige_kausalitaet}

\subsection{Kernthese}
Die physikalische Standarddefinition von Kausalität ist unnötig restriktiv, wenn sie gleichzeitige Wechselwirkungen ausschließt. Ich argumentiere für einen erweiterten Kausalitätsbegriff, der zwei Prinzipien vereint:

\begin{itemize}
    \item \textbf{Determinismus}: Der Zustand $Z(t) = \{r, \dot{r}\}$ bestimmt eindeutig $Z(t+dt)$
    \item \textbf{Systemische Abhängigkeit}: Instantane Korrelationen sind kausal, wenn sie aus einer gemeinsamen Ursache folgen
\end{itemize}

\subsection{Anwendung auf die Weber-Kraft}
Die Weber-Gravitation zeigt dies exemplarisch:

\begin{equation}
    F = -\frac{GMm}{r^2}\left(1 - \frac{\dot{r}^2}{c^2} + \frac{r\ddot{r}}{2c^2}\right)
\end{equation}

\begin{itemize}
    \item Die Abhängigkeit von $\ddot{r}$ \textit{scheint} nicht-lokal
    \item Tatsächlich beschreibt sie eine \textit{systeminterne} Rückkopplung:
\end{itemize}

\begin{equation}
    \ddot{r} = f(r, \dot{r}) \quad \text{(lösbar nach Lipschitz-Bedingung)}
\end{equation}

\subsection{Philosophische Begründung}
\begin{itemize}
    \item Newtons 3. Gesetz wirkt ebenfalls instantan (actio = reactio)
    \item Quantenverschränkung zeigt: Gleichzeitige Korrelationen verletzen keine Kausalität
    \item Entscheidend ist nicht die \textit{Lokalität}, sondern die \textit{Eindeutigkeit} der Zeitentwicklung
\end{itemize}

\subsection{Konsequenzen}
\begin{tabular}{p{0.45\textwidth}p{0.45\textwidth}}
    \hline
    \textbf{Konventionelle Sicht} & \textbf{Diese Arbeit} \\
    \hline
    Kausalität erfordert Zeitverzögerung & Gleichzeitige Kausalität möglich \\
    Nicht-Lokalität = problematisch & Systemische Abhängigkeiten sind natürlich \\
    \hline
\end{tabular}

\section{Das Prinzip der energetischen Gleichzeitigkeit}
\label{sec:energetische_gleichzeitigkeit}

\subsection{Die fundamentale Rolle der Welle}
Die Natur realisiert durch Wellenphänomene eine \emph{instantane energetische Optimierung}:

\begin{itemize}
    \item Eine Welle $\Psi(\mathbf{x},t)$ stellt zu jedem Zeitpunkt $t$ global sicher, dass:
    \begin{equation}
        \delta \mathcal{E}[\Psi] = 0 \quad \text{(Energieminimierung)}
    \end{equation}
    
    \item Dieses Prinzip wirkt \emph{ohne Zeitverzug} und ist damit kausal im erweiterten Sinn
\end{itemize}

\subsection{Naturprinzip vs. Kausalitätsdogma}
Die konventionelle Kausalitätsdefinition widerspricht diesem Grundprinzip:

\begin{table}[h]
    \centering
    \begin{tabular}{ll}
        \toprule
        \textbf{Mainstream-Kausalität} & \textbf{Energetische Gleichzeitigkeit} \\
        \midrule
        Lokale Wechselwirkungen & Globale Optimierung \\
        Ursache-Wirkung-Kette & Instantanes Minimum \\
        Lichtkegel-Beschränkung & Sofortige Anpassung \\
        \bottomrule
    \end{tabular}
    \caption{Konflikt der Paradigmen}
\end{table}

\subsection{Mathematische Konsequenz}
Das Wellenprinzip erzwingt eine Revision der Bewegungsgleichungen:

\begin{equation}
    \underbrace{\frac{\partial \Psi}{\partial t}}_{\text{Dynamik}} = 
    \underbrace{\mathcal{H}[\Psi]}_{\text{Instantane Optimierung}}
\end{equation}

wobei $\mathcal{H}$ ein \emph{globaler} Energieoperator ist.

\subsection{Physikalische Implikationen}
\begin{itemize}
    \item Die Weber-Kraft mit $\ddot{r}$-Abhängigkeit wird zur natürlichen Konsequenz
    \item Quantenverschränkung ist direkter Ausdruck dieses Prinzips
    \item Der Raum wird zum Träger der instantanen energetischen Information
\end{itemize}
