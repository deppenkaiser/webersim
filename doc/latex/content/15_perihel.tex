\section{Periheldrehung in der WG}
Die Dominanz der ART in der modernen Astrophysik beruht auf ihrer erfolgreichen Vorhersage der Periheldrehung des Merkurs \cite{einstein1915} (publizierter Wert: $43.0''$/Jh.). Jedoch zeigt diese Arbeit:
\begin{itemize}
    \item Die WG liefert mit $42.98''$/Jh. den \textbf{gleichen Wert}.
    \item Die ART-Interpretation der Periheldrehung als rein „relativistischer Effekt“ ist \textbf{modellabhängig} und möglicherweise falsch.
    \item Die WG erklärt \textbf{ohne Raummodell} Galaxienrotationen und Planetenbahnen konsistent.
\end{itemize}

\subsection{Berechnung 1. Ordnung}
Die WG beschreibt die Gravitationskraft durch Gleichung (\ref{eq:weber_g}), was zur Bahngleichung (\ref{eq:bahngleichung_1_ordnung}) führt.

Die Periheldrehung pro Umlauf beträgt:
\begin{equation}
\Delta\phi = 2\pi\left(\frac{1}{\kappa} - 1\right) \leftrightarrow 42.98'' /Jh.
\end{equation}

\subsection{Rotationskurven der äußeren Planeten}
Die gemessenen Umlaufgeschwindigkeiten der Planeten folgen exakt dem newtonschen Gesetz. Die WG sagt zwar eine Korrektur der Form
\[
v_{\text{WG}}(r) = \sqrt{\frac{GM_\odot}{r}} \left(1 + \frac{GM_\odot}{4c^2 r}\right),
\]
vorher, doch ist dieser Effekt im Sonnensystem vernachlässigbar klein. Erst auf galaktischen Skalen wird der Term dominant und
erklärt die beobachteten Abweichungen von der Kepler-Rotation.
