\section{Periheldrehung in der WG}
Die Dominanz der ART in der modernen Astrophysik beruht auf ihrer erfolgreichen Vorhersage der Periheldrehung des Merkurs (publizierter Wert: $43.0''$/Jh.). Jedoch zeigt diese Arbeit:
\begin{itemize}
    \item Die WG liefert mit $42.98''$/Jh. den \textbf{gleichen Wert}.
    \item Die ART-Interpretation der Periheldrehung als rein „relativistischer Effekt“ ist \textbf{modellabhängig} und möglicherweise falsch.
    \item Die WG erklärt \textbf{ohne Raummodell} Galaxienrotationen und Planetenbahnen konsistent.
\end{itemize}

\subsection{Berechnung 1. Ordnung}
Die WG beschreibt die Gravitationskraft durch:
\begin{equation}
\mathbf{F}_{\text{WG}} = -\frac{GMm}{r^2}\left(1 - \frac{\dot{r}^2}{c^2} + \frac{r\ddot{r}}{2c^2}\right)\hat{\mathbf{r}},
\end{equation}

was zur Bahngleichung führt:
\begin{equation}
r(\phi) = \frac{a(1-e^2)}{1 + e \cos\left(\kappa \phi\right)}, \quad \kappa = \sqrt{1 - \frac{6GM}{c^2 a (1-e^2)}}.
\end{equation}

Die Periheldrehung pro Umlauf beträgt:
\begin{equation}
\Delta\phi = 2\pi\left(\frac{1}{\kappa} - 1\right) \leftrightarrow 42.98'' /Jh.
\end{equation}
