\newpage
\section{Umlaufperiode in der Weber-Gravitation}
Die Umlaufperiode $T$ wird durch Integration über den Winkel $\phi$ bestimmt, bei dem das Argument $\theta(\phi)$ der Bahngleichung um $2\pi$ fortschreitet:

\begin{equation}
T = \int_{0}^{\phi_1} \frac{d\phi}{\dot{\phi}},
\quad \text{mit} \quad \dot{\phi} = \frac{h}{r^2(\phi)}
\end{equation}

\subsubsection{1. Ordnung}
Für die Bahngleichung 1. Ordnung:
\begin{equation}
\theta(\phi) = \kappa\phi
\end{equation}
Die Integrationsgrenze $\phi_1$ folgt aus:
\begin{equation}
\kappa\phi_1 = 2\pi \quad \Rightarrow \quad \phi_1 = \frac{2\pi}{\kappa}
\end{equation}
Damit ergibt sich:
\begin{equation}
T_1 = \frac{2\pi a^{3/2}}{\sqrt{GM}} \left(1 + \frac{3GM}{c^2a(1-e^2)}\right)
\end{equation}

\subsubsection{2. Ordnung}
Für die Bahngleichung 2. Ordnung:
\begin{equation}
\theta(\phi) = \kappa\phi + \alpha\phi^2
\end{equation}
Die Integrationsgrenze $\phi_1$ ist Lösung von:
\begin{equation}
\kappa\phi_1 + \alpha\phi_1^2 = 2\pi
\end{equation}
Mit der Näherung für kleine $\alpha$:
\begin{equation}
\phi_1 \approx \frac{2\pi}{\kappa} - \frac{4\pi^2\alpha}{\kappa^3}
\end{equation}
Die Umlaufperiode in 2. Ordnung:
\begin{equation}
T_2 \approx \frac{2\pi a^{3/2}}{\sqrt{GM}} \left[1 + \frac{3GM}{2c^2a(1-e^2)} + \frac{45G^2M^2}{8c^4a^2(1-e^2)^2}\left(1-\frac{e^2}{3}\right)\right]
\end{equation}

\subsection{Geschlossenheit der Rosettenbahn}
Die Bahn in der Weber-Gravitation ist \textbf{nicht exakt geschlossen}, da das Argument
\begin{equation}
\theta(\phi) = \kappa\phi + \alpha \phi^2
\end{equation}
\textbf{keine Periodizität} besitzt. Für Planeten ist die Abweichung jedoch praktisch irrelevant:
\begin{itemize}
\item Für Merkur: $\alpha \sim 10^{-30}$ $\Rightarrow$ keine messbare Abweichung über das Alter des Sonnensystems.
\item Die Bahn erscheint quasi-geschlossen („Rosettenbahn“).
\end{itemize}

\subsection{Dynamik der Periheldrehung}
Die Periheldrehung $\Delta\phi$ in der Weber-Gravitation wird durch $\alpha \phi^2$ \textbf{leicht variabel}:
\begin{equation}
\Delta\phi_N \approx \underbrace{2\pi\left(\frac{1}{\kappa} - 1\right)}_{\text{ART-Term}} - \underbrace{\frac{4\pi^2 \alpha}{\kappa^3}(1 + 2N)}_{\text{WG-Korrektur}}.
\end{equation}
Hierbei ist $N$ die Anzahl der Umläufe. Die Korrektur wächst linear mit $N$, führt aber zu \textbf{keiner Schwingung oder Instabilität}.

\subsection{Implikationen für Entropie und Zeit}
Die aperiodische Bahn der Weber-Gravitation zeigt \textbf{Aspekte irreversibler Dynamik}:
\begin{itemize}
\item \textbf{Keine exakte Rekurrenz}: $\theta(\phi) = \kappa\phi + \alpha \phi^2$ verletzt die Poincaré-Wiederkehr.
\item \textbf{Abweichung von der Maßerhaltung}: Die Phasenraumvolumina entwickeln sich nicht streng periodisch.
\item \textbf{Emergenz gerichteter Zeit}: Die kumulative Wirkung von $\alpha \phi^2$ definiert eine \textit{Präferenzrichtung}.
\end{itemize}
Dennoch ist dies \textbf{keine echte Entropie}, da das System deterministisch bleibt.
