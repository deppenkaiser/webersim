\newpage
\subsection{Umlaufperiode $T$ in 1. Ordnung}
Die Umlaufperiode $T$ eines Planeten in der Weber-Gravitation (WG) ergibt sich aus der modifizierten Bahngleichung (\ref{eq:bahngleichung_1_ordnung}) und dem spezifischen Drehimpuls $h$ (\ref{eq:spezifischer_drehimpuls_h}).

\subsubsection*{Ausgangsgleichungen}
\begin{enumerate}
    \item $\kappa$ aus Gl. (\ref{eq:kappa_1_ordnung})
    \item Winkelgeschwindigkeit Gl. (\ref{eq:winkelgeschwindigkeit_1_ordnung})
\end{enumerate}

\subsubsection*{Schritt 2: Integration über einen Umlauf}
Die Periode $T$ ist die Zeit für $\phi = 0 \to 2\pi/\kappa$ (WG-Korrektur durch $\kappa$):
\begin{align}
    T &= \int_0^{2\pi/\kappa} \frac{d\phi}{\dot{\phi}} 
       = \frac{a^2(1-e^2)^2}{h} \int_0^{2\pi/\kappa} \frac{d\phi}{(1 + e \cos(\kappa \phi))^2}.
\end{align}

\subsubsection*{Schritt 3: Lösung des Integrals}
Mit der Substitution $\psi = \kappa \phi$ und $\cos^2$-Identität:
\begin{align}
\label{eq:integral_t}
    T &= \frac{a^2(1-e^2)^2}{h \kappa} \int_0^{2\pi} \frac{d\psi}{(1 + e \cos \psi)^2} 
       = \frac{2\pi a^2(1-e^2)^2}{h \kappa (1-e^2)^{3/2}} 
       = \frac{2\pi a^{3/2}}{\sqrt{GM} \kappa}.
\end{align}

\subsubsection*{Schritt 4: Entwicklung von $\kappa$}
Für kleine relativistische Korrekturen ($c^{-2}$-Ordnung) gilt:
\begin{equation}
    \kappa \approx 1 - \frac{3GM}{c^2 a(1-e^2)} + \mathcal{O}(c^{-4}).
\end{equation}
Einsetzen in Gl.~\eqref{eq:integral_t} liefert die Periode in 1. Ordnung:
\begin{equation}
    \boxed{
    T \approx \frac{2\pi a^{3/2}}{\sqrt{GM}} \left(1 + \frac{3GM}{c^2 a(1-e^2)}\right).
    }
\end{equation}

\newpage
\section{Umlaufperiode \( T \) 2. Ordnung}

\subsubsection*{Ausgangsgleichungen}
\begin{enumerate}
    \item Bahngleichung in Polarkoordinaten (\ref{eq:bahngleichung_2_ordnung})
    \item $\kappa$ aus Gl. (\ref{eq:kappa_2_ordnung})
    \item $\alpha$ aus Gl. (\ref{eq:alpha})
    \item $h$ aus Gl. (\ref{eq:spezifischer_drehimpuls_h})
\end{enumerate}

\subsection*{Schritt 1: Entwicklung von \(\kappa\)}
\begin{equation}
\kappa \approx 1 - \frac{3GM}{c^2a(1-e^2)} + \frac{27G^2M^2}{4c^4a^2(1-e^2)^2} - \frac{81G^3M^3}{8c^6a^3(1-e^2)^3} + \mathcal{O}(c^{-8}) 
\end{equation}

\subsection*{Schritt 2: Vollständige Integration}
Die Umlaufperiode \( T \) ist:
\begin{equation}
T = \frac{1}{h} \int_0^{2\pi} r^2(\phi) \, d\phi = \frac{a^2(1-e^2)^2}{h} \int_0^{2\pi} \frac{d\phi}{\left[1 + e\cos\left(\kappa\phi + \alpha\phi^2\right)\right]^2}
\end{equation}

\subsection*{Schritt 3: Behandlung des Integrals}
Mit Substitution \(\psi = \kappa\phi + \alpha\phi^2\) und Entwicklung bis \(\mathcal{O}(c^{-4})\):
\begin{align}
T &= \frac{a^2(1-e^2)^2}{h} \left[ \int_0^{2\pi} \frac{d\phi}{(1 + e\cos\psi)^2} + \mathcal{O}(c^{-6}) \right] \\
  &= \frac{2\pi a^{3/2}}{\sqrt{GM}} \left[1 + \frac{3GM}{2c^2a(1-e^2)} + \frac{45G^2M^2}{8c^4a^2(1-e^2)^2}\left(1 - \frac{e^2}{3}\right)\right]
\end{align}
