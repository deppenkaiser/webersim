\section{Kernaussage zur dunklen Materie}
Die Weber-Gravitation erklärt galaktische Rotationskurven \textbf{ohne dunkle Materie}\\durch ihre nicht-newtonschen Terme:
\begin{equation}
\mathbf{F}_{\text{Weber}}^G = -\frac{GMm}{r^2}\left(1 \underbrace{-\frac{\dot{r}^2}{c^2} + \frac{r\ddot{r}}{2c^2}}_{\text{relativistische Korrekturen}}\right)\mathbf{\hat{r}}
\end{equation}

\section*{Mathematischer Beweis}

\subsection*{Rotationskurven von Galaxien}
Für eine Kreisbahn (\(\dot{r}=0\), \(\ddot{r} = -r\dot{\phi}^2\)) reduziert sich die Weber-Kraft zu:
\begin{equation}
F_{\text{Weber}} = -\frac{GMm}{r^2}\left(1 - \frac{v^2}{2c^2}\right), \quad v = r\dot{\phi}
\end{equation}
Die Zentripetalkraft \(F = mv^2/r\) führt zur modifizierten Geschwindigkeit:
\begin{equation}
v(r) = \sqrt{\frac{GM}{r}} \left(1 + \frac{GM}{4c^2r}\right)
\end{equation}

\subsection*{Vergleich mit Beobachtungen}
\begin{itemize}
\item \textbf{Newton}: \(v \propto r^{-1/2}\) (Abfall nicht beobachtet)
\item \textbf{Weber}: Zusatzterm \(\propto r^{-3/2}\) kompensiert den Abfall bei großen \(r\)
\item \textbf{ART}: Erfordert dunkle Materie für flache Rotationskurven \cite{rubin1970}
\end{itemize}

\section*{Numerisches Beispiel (Milchstraße)}
\begin{align*}
\text{Bereich} &\quad r = \SI{10}{kpc} \\
\text{Weber-Korrektur} &\quad \frac{GM}{4c^2r} \approx 0.12 \quad (\text{12\% Erhöhung}) \\
\text{Beobachtung} &\quad v \approx \SI{220}{km/s} \ (\text{konstant über } r)
\end{align*}

\section*{Konsequenzen}
\begin{itemize}
\item \textbf{Keine dunkle Materie}: Die Weber-Korrektur wirkt wie eine effektive Massenerhöhung \(\Delta M \approx \frac{GM(r)}{4c^2r}M\).
\item \textbf{Quantitativ}: Für \(r \to \infty\) wird \(v(r)\) konstant – genau wie beobachtet.
\item \textbf{Unterschied zu MOND}: Die Korrektur folgt natürlicherweise aus der Weber-Formel, ohne ad-hoc-Anpassungen.
\end{itemize}
