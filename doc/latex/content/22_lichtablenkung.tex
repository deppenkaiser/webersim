\newpage
\section{Lichtablenkung mit Frequenzabhängigkeit}

Die modifizierte Weber-Kraft für Photonen ($m=0$, $E=h\nu$) mit $\beta=1$ lautet:

\begin{equation}
    \boxed
    {
        F = -\frac{GM}{r^2}\frac{E}{c^2}\left(1 - \frac{\dot{r}^2}{c^2} + \frac{r\ddot{r}}{c^2}\right)
    }
\end{equation}

\subsection{Bahngleichung}
Mit Drehimpulserhaltung $h=r^2\dot{\phi}$ und $u=1/r$ ergibt sich:

\[
    \frac{d^2u}{d\phi^2} + u = \frac{GM}{c^2}\left(3u^2 + \frac{E^2}{c^2h^2}u^3\right)
\]

\subsection{Lösung für kleine Ablenkungen}
Entwicklung um $u_0=b^{-1}\cos\phi$ ($b$=Stoßparameter):

\[
\Delta\phi = \underbrace{\frac{4GM}{c^2b}}_{\text{ART-Term}} + \underbrace{\frac{3\pi GM}{4c^2b^2}\left(\frac{h}{E}\right)^2}_{\text{Frequenzterm}}
\]

\subsection{Frequenzabhängigkeit}
Mit $\lambda = c/\nu$ und $E=h\nu$:

\begin{equation}
    \boxed
    {
        \Delta\phi = \frac{4GM}{c^2b}\left(1 + \frac{3\pi}{16}\frac{\lambda^2}{\lambda_0^2}\right), \quad \lambda_0=\frac{hc}{E}    
    }
\end{equation}

\begin{table}[h]
\centering
\caption{Vorhersagen für verschiedene Wellenlängen}
\begin{tabular}{lcc}
\hline
Bereich & $\lambda$ [m] & $\Delta\phi/\Delta\phi_\text{ART}$ \\
\hline
Radio & $1$ & $1 + 2.4\times10^{-24}$ \\
Optisch & $5\times10^{-7}$ & $1 + 9.6\times10^{-18}$ \\
Röntgen & $1\times10^{-10}$ & $1 + 2.4\times10^{-10}$ \\
\hline
\end{tabular}
\end{table}

\section{Stoßdynamik der Lichtablenkung}

\subsection{Effektives Potential für Photonen}
Die Weber-Kraft erzeugt ein effektives Potential für Photonen im Gravitationsfeld:

\begin{equation}
    \boxed
    {
        V_{\text{eff}}(r) = -\frac{GM}{r}\frac{E}{c^2}\left(1 + \frac{h^2}{c^2r^2}\right)
    }
\end{equation}

wobei $h = b\cdot c$ der spezifische Drehimpuls ist ($b$=Stoßparameter). Der zweite Term entspricht einer relativistischen Korrektur.

\subsection{Energie- und Impulsübertrag}
Während des Vorbeiflugs erfährt das Photon:

\begin{itemize}
\item \textbf{Radialer Impulsübertrag}:
  \[
  \Delta p_r = \int_{-\infty}^\infty F_r\, dt = \frac{2GME}{c^3b^2}
  \]
  
\item \textbf{Energieänderung} (Rotverschiebung):
\begin{equation}
    \boxed
    {
        \frac{\Delta E}{E} = -\frac{GM}{c^2b} + \mathcal{O}\left(\frac{v^2}{c^2}\right)
    }
\end{equation}
\end{itemize}

\subsection{Nichtlinearer Stoßprozess}
Die Ablenkung entsteht durch:

\begin{enumerate}
\item \textbf{Anziehende Komponente}: Der $1/r^2$-Term der Weber-Kraft krümmt die Bahn
\item \textbf{Geschwindigkeitsabhängige Terme}: 
  \[
  -\frac{\dot{r}^2}{c^2} + \frac{r\ddot{r}}{c^2}
  \]
  führen zur Frequenzabhängigkeit
\item \textbf{Drehimpulserhaltung}: Erzwingt die hyperbolische Trajektorie
\end{enumerate}

\subsection{Parameterabhängigkeit}
\begin{table}[h]
\centering
\caption{Einfluss der Stoßparameter}
\begin{tabular}{lc}
\hline
Parameter & Effekt auf $\Delta\phi$ \\
\hline
$b \downarrow$ & $\propto b^{-1}$ (stärkere Ablenkung) \\
$E \uparrow$ & $\propto E^{-2}$ (schwächere Frequenzabhängigkeit) \\
$M \uparrow$ & linearer Anstieg \\
\hline
\end{tabular}
\end{table}

\subsection{Vergleich zur klassischen Streuung}
\begin{equation}
\frac{d\sigma}{d\Omega} \approx \left(\frac{4GM}{c^2\theta^2}\right)^2 \left(1 + \frac{3\pi h\nu}{16Mc^2}\right)
\end{equation}
wobei der zweite Term die Weber-spezifische Modifikation darstellt.
