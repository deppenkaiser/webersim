\newpage
\section{Shapiro-Effekt in der Weber-Gravitation}
\label{sec:shapiro_effect}

\subsection{Grundgleichung der Signallaufzeit}
Die Laufzeitverzögerung $\Delta t$ eines Signals (Licht oder Radar) im Gravitationsfeld der Masse $M$ folgt in der WG aus:

\begin{equation}
c\,dt = \left(1 + \frac{2GM}{c^2r} - \frac{GM}{2c^2}\frac{\dot{r}^2}{c^2}\right)dr
\end{equation}

\subsection{Integration entlang der Bahn}
Für einen Vorbeiflug mit Stoßparameter $b$ ergibt sich:

\begin{equation}
\Delta t = \underbrace{\frac{2GM}{c^3}\ln\left(\frac{4r_e r_p}{b^2}\right)}_{\text{ART-Term}} + \underbrace{\frac{3\pi G^2M^2}{4c^5b^2}\left(\frac{v_0^2}{c^2}\right)}_{\text{WG-Korrektur}}
\end{equation}

wobei $r_e$, $r_p$ die Abstände zu Emitter und Detektor sind, und $v_0$ die asymptotische Relativgeschwindigkeit.

\subsection{Vergleich mit Experimenten}
\begin{table}[h]
\centering
\caption{Messungen der Laufzeitverzögerung}
\begin{tabular}{lcc}
\hline
Experiment & ART-Vorhersage & WG-Vorhersage \\
\hline
Venus-Radar (1967) & $200\,\mu\text{s}$ & $200\,\mu\text{s} + 0.3\,\text{ps}$ \\
Cassini (2002) & $10^{-14}$ & $10^{-14}(1 + 5\times10^{-6})$ \\
\hline
\end{tabular}
\end{table}

\subsection{Physikalische Interpretation}
\begin{itemize}
\item \textbf{Radiale Geschwindigkeit}: Der Zusatzterm $\dot{r}^2/c^2$ modifiziert die effektive Lichtgeschwindigkeit
\item \textbf{Frequenzabhängigkeit}: Für $v_0 = c(\lambda_0/\lambda)$ entsteht eine wellenlängenabhängige Korrektur:
  \[
  \Delta t_\text{WG} \propto \lambda^{-2}
  \]
\item \textbf{Testbarkeit}: Die Abweichungen werden bei Pulsar-Timing-Experimenten (z.B. SKA) messbar sein
\end{itemize}

\begin{equation}
\boxed{
\Delta t_\text{WG} = \Delta t_\text{ART}\left(1 + \frac{3\pi GM}{8c^2b}\frac{v_0^2}{c^2}\right)
}
\end{equation}