\newpage
\section{Mikroskopische Herleitung des Gitterabstands $d$}
\label{sec:gitterabstand}

\subsection{Grundannahmen}

Ausgehend von den ersten Prinzipien der quantisierten Dodekaeder-Gravitation fordern wir:

\begin{enumerate}
\item Die Raumzeit besitzt eine diskrete $H_3$-Symmetrie
\item Der Kommutator der Koordinatenoperatoren gehorcht:
\begin{equation}
[X_i,X_j] = i\hbar\theta_0\sum_{k=1}^5\epsilon_{ijk}\omega_k
\end{equation}
\item Die Feinstrukturkonstante $\alpha$ bestimmt die Skalierung
\end{enumerate}

\subsection{Herleitungsschritte}

\subsubsection{Schritt 1: Quantisierungsbedingung}
Aus der Ikosaedersymmetrie folgt für den minimalen Abstand:
\begin{equation}
d^2 = \frac{5\hbar}{2}\theta_0\left(1 + \frac{1}{\phi^2}\right)
\end{equation}
wobei $\phi = (1+\sqrt{5})/2$ der Goldene Schnitt ist.

\subsubsection{Schritt 2: Kopplung an $\alpha$}
Die elektromagnetische Kopplung fixiert $\theta_0$:
\begin{equation}
\theta_0 = \frac{\alpha^{-1}}{12\pi^2}\ell_{\text{Planck}}^2
\end{equation}

\subsubsection{Schritt 3: Finaler Ausdruck}
Einsetzen liefert den Gitterabstand:
\begin{equation}
d = \left(\frac{5\hbar\alpha^{-1}}{24\pi^2}\left(1+\frac{1}{\phi^2}\right)\right)^{1/2}\ell_{\text{Planck}}
\end{equation}

\subsection{Numerische Abschätzung}

Mit $\alpha \approx 1/137$ und $\ell_{\text{Planck}} \approx 1.6\times10^{-35}$m:
\begin{equation}
d \approx 1.1\times10^3\,\text{Mpc}
\end{equation}

\subsection{Konsistenzcheck}

Die Beziehung zum CMB-Peak bei $\ell=36$:
\begin{equation}
d = \frac{2\pi}{\ell}\chi_{\text{CMB}} \approx 1.05\times10^3\,\text{Mpc}
\end{equation}
wobei $\chi_{\text{CMB}}$ die comoving Distanz zur letzten Streuoberfläche ist.
