\section{Shapiro-Effekt in der Weber-DBT-Gravitation}
\label{sec:shapiro_effect}

Der Shapiro-Effekt beschreibt die gravitative Zeitverzögerung von Lichtsignalen. In der Weber-DBT-Theorie ergibt sich eine modifizierte Version dieses Effekts durch die Kombination aus Weber-Gravitation und Quantenpotential.

\subsection{Laufzeitverzögerung}
Für ein Lichtsignal, das an einer Masse $M$ mit minimalem Abstand $b$ vorbeiläuft, beträgt die zusätzliche Laufzeit:

\begin{equation}
\Delta t = \frac{2GM}{c^3}\left[\ln\left(\frac{4r_e r_p}{b^2}\right) + \frac{\beta GM}{c^2b}\right] + \frac{\hbar^2}{4E^2c^3b^2}(r_e + r_p)
\end{equation}

wobei:
\begin{itemize}
\item $r_e$ und $r_p$ die Abstände von Masse zu Emitter bzw. Detektor sind
\item $\beta = 0.5$ für die Weber-Gravitation
\item $E = h\nu$ die Photonenenergie
\end{itemize}

\subsection{Herleitung}
Aus der WG-DBT-Metrik für schwache Felder:

\begin{equation}
ds^2 = -\left(1 - \frac{2GM}{c^2r} + \frac{Q}{E}\right)c^2dt^2 + \left(1 + \frac{2GM}{c^2r}\right)dr^2
\end{equation}

Die Lichtlaufzeit folgt aus:

\begin{equation}
\Delta t = 2\int_{b}^{r_e} \frac{1}{c}\left[\left(1 + \frac{2GM}{c^2r} - \frac{Q}{E}\right)^{-1} - 1\right] dr
\end{equation}

Mit dem Quantenpotential $Q \approx -\hbar^2/(2Eb^2)$ für $r \approx b$:

\begin{equation}
\Delta t \approx \frac{2GM}{c^3}\ln\left(\frac{4r_e r_p}{b^2}\right) + \frac{\beta G^2M^2}{c^5b} + \frac{\hbar^2(r_e + r_p)}{4E^2c^3b^2}
\end{equation}

\subsection{Physikalische Interpretation}
\begin{itemize}
\item Der erste Term entspricht der klassischen ART-Vorhersage
\item Der Weber-Term ($\beta$) führt zu einer zusätzlichen $1/b$-Abhängigkeit
\item Der DBT-Term zeigt charakteristische Frequenzabhängigkeit ($\propto \nu^{-2}$)
\end{itemize}

Für Radarsignale ($\nu \sim 10^{10}$ Hz) im Sonnensystem:
\begin{equation}
\Delta t_{\text{WG-DBT}} \approx 240\,\mu\text{s} - 10^{-36}\,\mu\text{s} + 10^{-72}\,\mu\text{s}
\end{equation}

Die Quantenkorrektur ist vernachlässigbar, aber prinzipiell vorhanden.
