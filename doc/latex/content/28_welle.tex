\newpage
\section{Fundamentale Charakteristika aller Wellen}
Diese Diskussion soll zeigen, dass Wellen \enquote{instantane} Eigenschaften besitzen, welche ebenfalls von Fernwirkungstheorien unterstellt werden.
Hier zeigt sich auch ein Zusammenhang zur De-Broglie-Bohm-Theorie (DBT).

Jede Welle besitzt zwei komplementäre Eigenschaftsebenen:

\subsection*{1. Lokale Eigenschaften (beobachtbar)}
\begin{itemize}
    \item \textbf{Störungsausbreitung} mit mediumabhängiger Phasengeschwindigkeit:
    \[
    v_p = \frac{\omega}{k} = f(\text{Medium})
    \]
    Beispiele:
    \begin{itemize}
        \item Elektromagnetische Wellen: $v_p = 1/\sqrt{\mu\epsilon}$
        \item Schallwellen: $v_p = \sqrt{K/\rho}$
        \item Wasserwellen: $v_p = \sqrt{g/k} \tanh(kh)$
    \end{itemize}
    
    \item \textbf{Sichtbare Dynamik} durch Feldgröße $\psi(x,t)$:
    \[
    \psi(x,t) = A e^{i(kx-\omega t)} \quad \text{(harmonische Näherung)}
    \]
\end{itemize}

\subsection*{2. Nicht-lokale Eigenschaften (instantane Korrelation)}
\begin{itemize}
    \item \textbf{Energieerhaltung} durch phasenkritische Kopplung:
    \[
    \partial_t \mathcal{E} + \nabla \cdot \vec{S} = 0 \quad \text{(Kontinuitätsgleichung)}
    \]
    mit $\mathcal{E} = \mathcal{E}_\text{kin} + \mathcal{E}_\text{pot}$ und $\vec{S}$ als Energiestromdichte.
    
    \item \textbf{Universalmechanismus}:
    \begin{itemize}
        \item Maximales $\mathcal{E}_\text{pot}$ bei $\psi = \pm A$ $\leftrightarrow$ Maximales $\mathcal{E}_\text{kin}$ bei $\psi = 0$
        \item Phasenversatz $\Delta\phi = \pi/2$ zwischen $\psi$ und $\partial_t\psi$
    \end{itemize}
\end{itemize}

\section*{Medienübergreifende Prinzipien}
\begin{table}[h]
    \centering
    \begin{tabular}{|l|c|c|}
    \hline
    \textbf{Wellentyp} & \textbf{Lokale Größe $\psi$} & \textbf{Nicht-lokaler Erhalt} \\
    \hline
    Mechanisch (Wasser) & Oberflächenauslenkung $\eta$ & $E_\text{kin} + E_\text{pot} = \text{const}$ \\
    \hline
    Akustisch & Druck $p$ & $\frac{p^2}{\rho c^2} + \rho v^2 = \text{const}$ \\
    \hline
    Elektromagnetisch & Felder $\vec{E},\vec{B}$ & $\frac{\epsilon_0 E^2}{2} + \frac{B^2}{2\mu_0} = \text{const}$ \\
    \hline
    Quantenmechanisch & Wellenfunktion $\Psi$ & $|\Psi|^2 = \text{Wahrscheinlichkeit}$ \\
    \hline
    \end{tabular}
\end{table}

\section*{Mathematische Universalstruktur}
\begin{itemize}
    \item \textbf{Dispersionsrelation}: $\omega = \omega(k)$ verknüpft lokale und nicht-lokale Ebene
    \item \textbf{Wellengleichung}: 
    \[
    \partial_t^2 \psi = v_p^2 \nabla^2 \psi + \text{Nichtlinearitäten}
    \]
    \item \textbf{Energietransport}:
    \[
    \vec{S} = 
    \begin{cases}
    \frac{1}{2}\rho g A^2 v_g & \text{(Wasser)} \\
    \vec{E} \times \vec{B}/\mu_0 & \text{(EM)} \\
    p \vec{v} & \text{(Schall)}
    \end{cases}
    \]
\end{itemize}

\section*{Zusammenfassung}
\begin{itemize}
    \item Alle Wellen zeigen \textit{duales Verhalten}: 
    \begin{itemize}
        \item Lokale Propagierung mit $v_p < \infty$
        \item Globale instantane Energie-Neutralisation
    \end{itemize}
    \item Die nicht-lokale Korrelation ist \textit{kein} kausaler Prozess, sondern strukturelle Konsequenz der Wellengleichung
    \item Energieerhaltung erfolgt instantan und nicht-lokal durch \textit{phasenstarre Kopplung} im gesamten System
\end{itemize}
