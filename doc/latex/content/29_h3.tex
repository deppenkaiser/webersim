\newpage
\section{Die Ikosaedergruppe $H_3$ als fundamentale Symmetrie}
\label{subsec:h3_symmetry}

\subsubsection{Mathematische Struktur}
Die Ikosaedergruppe $H_3$ ist die diskrete Drehgruppe des regelmäßigen Dodekaeders mit folgenden Eigenschaften:

\begin{itemize}
    \item \textbf{Ordnung}: 120 Elemente (inkl. Inversion)
    \item \textbf{Erzeugende}: Drei Spiegelungen $s_1, s_2, s_3$ mit $(s_is_j)^{m_{ij}} = 1$
    \item \textbf{Coxeter-Matrix}:
    \[
    (m_{ij}) = \begin{pmatrix}
    1 & 3 & 2 \\
    3 & 1 & 5 \\
    2 & 5 & 1
    \end{pmatrix}
    \]
    \item \textbf{Charaktertafel}:
    
    \begin{tabular}{c|ccccccccc}
    & $\chi_1$ & $\chi_2$ & $\chi_3$ & $\chi_4$ & $\chi_5$ & $\chi_6$ & $\chi_7$ & $\chi_8$ & $\chi_9$ \\
    \hline
    1 & 1 & 3 & 3 & 4 & 4 & 5 & 5 & 6 & 9 \\
    $15C_2$ & 1 & -1 & -1 & 0 & 0 & 1 & 1 & -2 & 1 \\
    $20C_3$ & 1 & 0 & 0 & 1 & 1 & -1 & -1 & 0 & 0 \\
    $12C_5$ & 1 & $\phi$ & $-\phi^{-1}$ & -1 & -1 & 0 & 0 & 1 & -1 \\
    $12C_5^2$ & 1 & $-\phi^{-1}$ & $\phi$ & -1 & -1 & 0 & 0 & 1 & -1 \\
    \end{tabular}
    mit $\phi = \frac{1+\sqrt{5}}{2}$
\end{itemize}

\subsubsection{Physikalische Realisierung}
Die $H_3$-Symmetrie manifestiert sich in:

\begin{enumerate}
    \item \textbf{Raumquantisierung}:
    \[
    \mathcal{H}_{\text{space}} = \bigoplus_{k=1}^{5} \mathbb{C}^3 \otimes D_k
    \]
    wobei $D_k$ irreduzible Darstellungen sind
    
    \item \textbf{Weber-Kraft mit $H_3$-Symmetrie}:
    \[
    F_{H_3} = -\frac{GMm}{r^2}\sum_{k=0}^{4}\omega_5^k f_k(r,\dot{r},\ddot{r})
    \]
    mit $\omega_5 = e^{2\pi i/5}$
    
    \item \textbf{Kopplung an Standardmodell}:
    \begin{align*}
    SU(3)_C &\hookrightarrow H_3 \text{ via } 3 \times \bar{3} \text{ Darstellung} \\
    SU(2)_L &\hookrightarrow A_5 \subset H_3 \\
    U(1)_Y &\subset Z_{10} \subset H_3
    \end{align*}
\end{enumerate}

\begin{theorem}[Goldene Quantenbedingung]
Für stabile Konfigurationen gilt:
\[
\frac{\langle \psi | H | \psi \rangle}{E_0} = \phi = \frac{1+\sqrt{5}}{2}
\]
wobei $E_0$ die Grundzustandsenergie im $H_3$-Potential ist.
\end{theorem}

\subsubsection{Experimentelle Konsequenzen}
\begin{itemize}
    \item \textbf{Multiplettstruktur}:
    \[
    m_n = m_0 \left(1 + \frac{n}{5\phi}\right), \quad n=0,...,4
    \]
    
    \item \textbf{Verbotene Übergänge}:
    Auswahlregeln führen zu Unterdrückung von:
    \[
    \Gamma_{5^-} \to \Gamma_{3^+} \text{ mit } \Delta J = \phi\text{-quantisiert}
    \]
\end{itemize}

