\section{Bahngleichung in der Weber-DBT-Gravitation}
\label{sec:bahngleichung}

\subsection{Bewegungsgleichung in Polarkoordinaten}
Die Bewegungsgleichung für ein Teilchen der Masse \( m \) im kombinierten Weber-DBT-Feld lautet:

\begin{equation}
m \frac{d}{dt} (\gamma_{\text{WG}} \mathbf{v}) = -\frac{GMm}{r^2} \left(1 - \frac{\dot{r}^2}{c^2} + \beta \frac{r \ddot{r}}{c^2}\right) \hat{\mathbf{r}} - \nabla Q,
\label{eq:bewegungsgleichung}
\end{equation}

mit dem Weber-Lorentz-Faktor
\begin{equation}
\gamma_{\text{WG}} = \left(1 - \frac{v^2}{c^2} + \beta \frac{\mathbf{r} \cdot \mathbf{a}}{c^2}\right)^{-1/2}
\end{equation}
und dem Quantenpotential
\begin{equation}
Q = -\frac{\hbar^2}{2m} \frac{\nabla^2 |\Psi|}{|\Psi|}.
\end{equation}

\subsection{Polarkoordinatenformulierung}
In Polarkoordinaten \((r, \phi)\) werden Geschwindigkeit und Beschleunigung zu:

\begin{align}
\mathbf{v} &= \dot{r} \hat{\mathbf{r}} + r \dot{\phi} \hat{\boldsymbol{\phi}}, \\
\mathbf{a} &= (\ddot{r} - r \dot{\phi}^2) \hat{\mathbf{r}} + (r \ddot{\phi} + 2 \dot{r} \dot{\phi}) \hat{\boldsymbol{\phi}}.
\end{align}

Der spezifische Drehimpuls \( h = r^2 \dot{\phi} \) ist eine Erhaltungsgröße.

\subsection{Radiale Komponente}
Die radiale Komponente von Gl.~\eqref{eq:bewegungsgleichung} lautet:

\begin{equation}
m \frac{d}{dt} (\gamma_{\text{WG}} \dot{r}) - m \gamma_{\text{WG}} r \dot{\phi}^2 = -\frac{GMm}{r^2} \left(1 - \frac{\dot{r}^2}{c^2} + \beta \frac{r \ddot{r}}{c^2}\right) - \frac{\partial Q}{\partial r}.
\label{eq:radial}
\end{equation}

\subsection{Substitution und Umformung}
Mit der Substitution \( u = 1/r \) und \( h = r^2 \dot{\phi} \) erhalten wir:

\begin{align}
\dot{r} &= -h \frac{du}{d\phi}, \\
\ddot{r} &= -h^2 u^2 \frac{d^2 u}{d\phi^2}.
\end{align}

Einsetzen in Gl.~\eqref{eq:radial} führt auf:

\begin{equation}
\frac{d^2 u}{d\phi^2} + u = \frac{GM}{h^2} \left(1 - \frac{h^2}{c^2} \left(\frac{du}{d\phi}\right)^2 + \beta \frac{h^2}{c^2} u \frac{d^2 u}{d\phi^2}\right) + \frac{1}{m h^2 u^2} \frac{\partial Q}{\partial u}.
\label{eq:hauptgleichung}
\end{equation}

\subsection{Quantenpotential in Polarkoordinaten}
Für eine radialsymmetrische Wellenfunktion \( |\Psi(r)| \) wird das Quantenpotential:

\begin{equation}
Q = -\frac{\hbar^2}{2m} \frac{1}{|\Psi|} \left(\frac{d^2 |\Psi|}{dr^2} + \frac{1}{r} \frac{d|\Psi|}{dr}\right).
\end{equation}

\subsection{Vollständige Bahngleichung}
Die exakte Bahngleichung der Weber-DBT-Gravitation lautet somit:

\begin{equation}
\frac{d^2 u}{d\phi^2} \left(1 - \beta \frac{GM}{c^2} u\right) + u = \frac{GM}{h^2} \left(1 - \frac{h^2}{c^2} \left(\frac{du}{d\phi}\right)^2\right) - \frac{\hbar^2}{2m h^2 u^2} \frac{d}{du} \left(\frac{\nabla^2 |\Psi|}{|\Psi|}\right).
\label{eq:final}
\end{equation}

\subsection{Bemerkungen}
\begin{itemize}
\item Für \( Q = 0 \) reduziert sich Gl.~\eqref{eq:final} auf die reine Weber-Gravitation
\item Die Nichtlinearitäten erfordern i.A. numerische Lösungsmethoden
\item Der Parameter \( \beta = 0.5 \) reproduziert die beobachtete Periheldrehung des Merkur
\end{itemize}
