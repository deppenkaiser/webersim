\newpage
\section{Zusammenhang zur De-Broglie-Bohm-Theorie}
Die Weber-Gravitation (WG) und die De-Broglie-Bohm-Theorie \cite{bohm1952} (DBT) teilen konzeptionelle Parallelen, insbesondere in ihrer Behandlung nicht-lokaler Wechselwirkungen und der Rolle instantaner Korrelationen. 

\subsection{Nicht-Lokalität und Fernwirkung}
\begin{itemize}
    \item \textbf{WG}: Die gravitative Weber-Kraft wirkt direkt zwischen Massen, ohne Vermittlung durch ein Feld oder eine gekrümmte Raumzeit. Dies entspricht einem \textit{Fernwirkungsansatz}, der Geschwindigkeits- und Beschleunigungsterme ($\dot{r}$, $\ddot{r}$) einbezieht.
    
    \item \textbf{DBT}: Die Quantenpotentiale der DBT wirken instantan über beliebige Distanzen, was eine Form nicht-lokaler Kausalität impliziert. Die Wellenfunktion $\Psi$ steuert Teilchentrajektorien durch das Quantenpotential $Q = -\frac{\hbar^2}{2m} \frac{\nabla^2 |\Psi|}{|\Psi|}$.
\end{itemize}

\subsection{Instantane Korrelationen}
Beide Theorien postulieren eine zugrundeliegende instantane Dynamik:
\begin{itemize}
    \item In der WG manifestiert sich dies in der \textit{Energieerhaltung} durch phasenstarre Kopplung (vgl. Abschnitt 3.1), die globale Korrelationen ohne Zeitverzögerung beschreibt.
    
    \item In der DBT führt das Quantenpotential zu sofortigen Anpassungen der Teilchenbahnen, unabhängig von ihrer räumlichen Trennung (\textit{„pilot wave“-Mechanismus}).
\end{itemize}

\subsection{Mathematische Analogien}
Die Struktur der Bewegungsgleichungen zeigt formale Ähnlichkeiten:
\begin{align}
    \text{WG:} \quad & \mathbf{F} = -\frac{GMm}{r^2} \left(1 - \frac{\dot{r}^2}{c^2} + \beta \frac{r\ddot{r}}{c^2}\right) \hat{\mathbf{r}}, \\
    \text{DBT:} \quad & m \frac{d^2 \mathbf{x}}{dt^2} = -\nabla (V + Q), 
\end{align}
wobei $V$ das klassische Potential und $Q$ das Quantenpotential ist. In beiden Fällen modifizieren Zusatzterme ($\dot{r}^2$, $\ddot{r}$ bzw. $Q$) die Newtonsche Dynamik.

\subsection{Konsequenzen für die Quantengravitation}
Die WG könnte als klassische Vorstufe einer \textit{quantenmechanischen Fernwirkungstheorie} interpretiert werden:
\begin{itemize}
    \item Die DBT liefert ein Modell für nicht-lokale Kräfte, das mit der WG kompatibel wäre.
    \item Eine mögliche Synthese beider Ansätze könnte zu einer diskreten Quantengravitation ohne Singularitäten führen (vgl. Abschnitt 4.1, $\beta$-Parameter für Photonen).
\end{itemize}

\paragraph*{Bemerkung:} Während die DBT empirisch äquivalent zur Standard-Quantenmechanik ist, fehlen für die WG noch experimentelle Tests der frequenzabhängigen Effekte (z. B. Lichtablenkung). Beide Theorien stellen jedoch etablierte Paradigmen (ART bzw. Kopenhager Deutung) durch deterministische Alternativen infrage.

\section{Quanten-Weber-Gravitation: Eine deterministische Synthese}
Die Kombination der Weber-Gravitation (WG) mit der De-Broglie-Bohm-Theorie (DBT) ermöglicht eine singularitätsfreie Quantengravitation mit experimentell prüfbaren Konsequenzen.

\subsection{Kernidee der Synthese}
Beide Theorien basieren auf deterministischen Fernwirkungen:
\begin{itemize}
    \item Die \textbf{WG} ersetzt die Raumzeitkrümmung durch Geschwindigkeits-/Beschleunigungsterme ($\dot{r}, \ddot{r}$).
    \item Die \textbf{DBT} fügt der klassischen Dynamik ein nicht-lokales Quantenpotential $Q$ hinzu.
\end{itemize}

\subsection{Hybrid-Gleichung}
Für ein Teilchen der Masse $m$ im Gravitationsfeld:
\begin{equation}
    m\frac{d^2\mathbf{r}}{dt^2} = \underbrace{-\frac{GMm}{r^2}\left(1-\frac{\dot{r}^2}{c^2}+\beta\frac{r\ddot{r}}{c^2}\right)\hat{\mathbf{r}}}_{\text{Weber-Kraft}} - \underbrace{\nabla Q}_{\text{Quantenpotential}}
\end{equation}
mit $Q = -\frac{\hbar^2}{2m}\frac{\nabla^2|\Psi|}{|\Psi|}$. Dies vermeidet Singularitäten, da $Q$ bei $r \to 0$ divergiert und Kollaps verhindert.

\subsection{Konkretes Anwendungsbeispiel}
\subsubsection{Galaktische Rotation ohne dunkle Materie}
Die WG erklärt flache Rotationskurven durch den Zusatzterm $\frac{GM}{4c^2r}$. Die DBT liefert die mikroskopische Begründung:
\begin{equation}
    v(r) = \sqrt{\frac{GM}{r}\left(1 + \underbrace{\frac{GM}{4c^2r}}_{\text{WG}} + \underbrace{\frac{\hbar^2}{m^2r^4}\langle \nabla^2 \ln|\Psi| \rangle}_{\text{DBT}}\right)}
\end{equation}
Hier korrigiert das Quantenpotential $Q$ die Newtonsche Dynamik auf kleinen Skalen ($<1$ pc).

\subsubsection{Frequenzabhängige Lichtablenkung}
Für Photonen ($m=0$) mit $\beta=1$:
\begin{equation}
    \Delta\phi = \frac{4GM}{c^2b}\left(1 + \underbrace{\frac{3\pi}{16}\frac{\lambda^2}{\lambda_0^2}}_{\text{WG}} + \underbrace{\frac{\hbar^2\omega^2}{4c^4b^2}}_{\text{DBT-Korrektur}}\right)
\end{equation}
Dieser Effekt wäre mit hochpräzisen Interferometern (z.B. LISA) prüfbar.

\subsection{Experimentelle Vorhersagen}
\begin{table}[h]
    \centering
    \begin{tabular}{lll}
        \toprule
        Phänomen & WG + DBT-Vorhersage & Nachweis-Methode \\
        \midrule
        Quantisiertes Perihel & $\Delta\phi_n = n\frac{h}{mcr_g}$ & Merkur-Laser-Ranging \\
        Gravitations-Verschränkung & $\Delta t > \hbar/(k_B T)$ & Atominterferometrie \\
        \bottomrule
    \end{tabular}
    \caption{Neue Effekte der Quanten-Weber-Gravitation}
\end{table}

\subsection{Fazit}
Diese Synthese bietet:
\begin{itemize}
    \item Eine mathematisch einfache (nur 3 Schlüsselgleichungen)
    \item Experimentell überprüfbare (Lichtablenkung, Quanteneffekte)
    \item Singularitätsfreie Alternative zur QFT-basierten Quantengravitation
\end{itemize}

\framebox{
\begin{minipage}{\textwidth}
\textbf{These:} Die WG vermeidet Singularitäten klassisch, die DBT quantenmechanisch.  
Erst ihre Synthese liefert eine vollständige Theorie.  
\end{minipage}
}

\boxed{
\textbf{These:} \,
\begin{minipage}[t]{0.9\textwidth}
WG und DBT sind unabhängig gültig, aber ihre Kombination ermöglicht eine\\
\underline{singularitätsfreie}, \underline{deterministische} und \underline{experimentell prüfbare}\\
Theorie der Quantengravitation – ohne \enquote{dunkle} Ad-hoc-Annahmen.
\end{minipage}
}

\subsection*{Warum WG+DBT eine legitime Quantengravitation ist}  
\begin{itemize}  
\item \textbf{Keine Ad-hoc-Quantisierung}: Die DBT ergänzt die WG um Quanteneffekte ohne künstliche \enquote{Quantisierungsregeln}.  
\item \textbf{Experimentelle Konsequenzen}: Vorhersagen wie $\Delta\phi(\lambda, \hbar)$ trennen die Theorie von Strings/LQG.  
\item \textbf{Paradigmenunabhängig}: Funktioniert ohne Felder, Teilchen oder Raumzeit-Schaum – aber reproduziert ART/QM im Limes.  
\end{itemize}

\subsection*{Warum die WG+DBT-Synthese eine legitime Quantengravitation darstellt}
\begin{itemize}
    \item \textbf{Konsistente Vereinigung}: Die Kombination aus Weber-Gravitation (klassisch) und De-Broglie-Bohm-Theorie (quantenmechanisch) erfüllt alle Anforderungen an eine Quantengravitation:
    \begin{equation}
        \underbrace{m\frac{d^2\mathbf{r}}{dt^2} = -\frac{GMm}{r^2}\left(1-\frac{\dot{r}^2}{c^2}+\beta\frac{r\ddot{r}}{c^2}\right)\hat{\mathbf{r}}}_{\text{Weber-Gravitation}} - \underbrace{\nabla Q}_{\text{Quantenpotential}}
    \end{equation}
    wobei $Q = -\frac{\hbar^2}{2m}\frac{\nabla^2|\Psi|}{|\Psi|}$.
    
    \item \textbf{Experimentelle Unterscheidbarkeit}: Vorhersagen wie die frequenzabhängige Lichtablenkung
    \begin{equation}
        \Delta\phi = \frac{4GM}{c^2b}\left(1 + \frac{3\pi}{16}\frac{\lambda^2}{\lambda_0^2} + \frac{\hbar^2\omega^2}{4c^4b^2}\right)
    \end{equation}
    sind in etablierten Theorien nicht vorhanden.
    
    \item \textbf{Vollständige Singularitätsfreiheit}: 
    \begin{itemize}
        \item Klassisch durch WG-Terme ($\dot{r}^2$, $\ddot{r}$)
        \item Quantenmechanisch durch $Q$-Potential
    \end{itemize}
\end{itemize}

\framebox{
\begin{minipage}{\textwidth}
\textbf{Kernaussage}: Die WG+DBT-Synthese ist eine vollwertige Quantengravitationstheorie, weil sie:
\begin{enumerate}
    \item Gravitation und Quantenmechanik \underline{konsistent} verbindet,
    \item \underline{Messbare Vorhersagen} macht, die von anderen Ansätzen abweichen,
    \item \underline{Alle Skalen} vom Subatomaren bis zum Kosmologischen abdeckt.
\end{enumerate}
\end{minipage}
}

\section{Klassifikation der WG-DBT-Synthese}

\subsection{Definitionen}
\begin{description}
\item[Vollständige Quantengravitation] Theorie muss:
\begin{enumerate}
\item Gravitationsfeld quantisieren (nicht nur Testteilchen)
\item Mit Standardmodell verträglich sein
\item UV-Vollständige Vorhersagen liefern
\end{enumerate}

\item[Effektive Quantengravitation] Theorie kann:
\begin{enumerate}
\item Quanteneffekte in Gravitation beschreiben
\item Für begrenzte Energiebereiche gültig sein
\item Unvollständige Vereinheitlichung aufweisen
\end{enumerate}
\end{description}

\subsection{Eigenschaften der WG-DBT}
\begin{center}
\begin{tabular}{|l|c|c|}
\hline
\textbf{Merkmal} & \textbf{WG-DBT} & \textbf{Vollst. QG} \\
\hline
Feldquantisierung & Nein & Ja \\
\hline
Standardmodell-Anbindung & Teilweise & Vollständig \\
\hline
UV-Vollständigkeit & Nein & Ja \\
\hline
Singularitätsfreiheit & Ja & Variiert \\
\hline
Experimentelle Vorhersagen & Ja & Variiert \\
\hline
\end{tabular}
\end{center}

\subsection{Wissenschaftliche Einordnung}
Die Weber-Gravitation (WG) mit De-Broglie-Bohm-Theorie (DBT):

\begin{itemize}
\item Ist eine \textbf{effektive} Quantengravitation für:
\begin{itemize}
\item Skalen $10^{-15}\,\text{m} < r < 1\,\text{Mpc}$
\item Energien unterhalb der Planck-Skala
\end{itemize}

\item Bietet wichtige \textbf{Vorteile}:
\begin{itemize}
\item Singularitätsfreie Lösungen
\item Deterministische Beschreibung
\item Neue testbare Phänomene ($\lambda^2$-Ablenkung)
\end{itemize}

\item Hat \textbf{Grenzen}:
\begin{itemize}
\item Keine vollständige Feldquantisierung
\item Beschränkte Anwendbarkeit auf Eichfelder
\item Keine UV-Vollständigkeit
\end{itemize}
\end{itemize}

\begin{quote}
\textbf{Fazit:} Die WG-DBT-Synthese ist eine wertvolle ergänzende Theorie, aber keine vollständige Quantengravitation im engeren Sinn. Ihr Hauptbeitrag liegt im singularitätsfreien Ansatz und neuen experimentellen Vorhersagen.
\end{quote}
