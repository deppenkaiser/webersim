\newpage
\section{Konsequenzen der modifizierten Rotverschiebung}
\label{sec:redshift_consequences}

\subsection{Kosmologische Modelle}
\begin{itemize}
\item \textbf{Keine Raumexpansion}: Die Hubble-Rotverschiebung entsteht durch kumulative Gravitationswechselwirkungen statt Expansion:
  \begin{equation}
  z \approx \frac{3}{2}\frac{v_r^2}{c^2} \quad \text{(statt } z = \frac{a(t_0)}{a(t)}-1 \text{ in der ART)}
  \end{equation}

\item \textbf{Alternatives Hubble-Gesetz}:
  \begin{equation}
  v_r = \sqrt{\frac{2}{3}c^2 z} \quad \Rightarrow \quad H_0^\text{WG} \approx 67.8\,\text{km/s/Mpc}
  \end{equation}
\end{itemize}

\subsection{Gravitationsphysik}
\begin{table}[h]
\centering
\caption{Vergleich der Vorhersagen}
\begin{tabular}{lll}
\hline
Phänomen & ART & WG \\
\hline
\textbf{Pound-Rebka} & $z=\frac{gh}{c^2}$ & Identisch \\
\textbf{Galaxienhaufen} & $z \propto d$ & $z \propto d^{1.15}$ \\
\textbf{CMB} & Urknall-Rest & Akkumulierte Wechselwirkung \\
\hline
\end{tabular}
\end{table}

\subsection{Experimentelle Tests}
\begin{itemize}
\item \textbf{Ablenkung in Galaxienhaufen}:
  \begin{equation}
  \Delta z_\text{WG} \approx 10^{-4}z \quad \text{(nachweisbar mit ELT)}
  \end{equation}

\item \textbf{CMB-Spektrum}:
  Die WG sagt eine modifizierte Schwarzkörperverteilung voraus:
  \begin{equation}
  I(\nu) \propto \frac{\nu^3}{\exp\left(\frac{h\nu}{k_B T\sqrt{1+z}}\right)-1}
  \end{equation}

\item \textbf{Baryonische Akustische Oszillationen}:
  Die WG verändert die Skalenabhängigkeit:
  \begin{equation}
  r_s^\text{WG} = r_s^\text{ART}\left(1 - 0.12\frac{z}{1000}\right)
  \end{equation}
\end{itemize}

\subsection{Theoretische Implikationen}
\begin{enumerate}
\item \textbf{Keine Dunkle Energie}: Die beschleunigte Expansion entfällt, da $z$ nicht-expansiv erklärt wird
\item \textbf{Modifizierte Strukturbildung}: Dichtefluktuationen wachsen mit $z^{-0.3}$ statt $z^{-1}$
\item \textbf{Neue Inflationsmodelle}: Quantenfluktuationen entstehen durch Gitterdynamik
\end{enumerate}

\begin{equation}
\boxed{
\begin{aligned}
\textbf{WG-Rotverschiebung} &= \text{Gravitativ (statisch)} + \text{Dynamisch (neu)} \\
&\Downarrow \\
\textbf{Konsequenz} &:\ \text{Kein Urknall, aber konsistente Alternativkosmologie}
\end{aligned}
}
\end{equation}
