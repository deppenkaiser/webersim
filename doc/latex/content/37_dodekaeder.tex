\newpage
\section{Dodekaedrisches Raummodell für die Weber-Gravitation}
\subsection{Mathematische Grundlagen}
Inspiriert durch quasikristalline Strukturen schlagen wir ein \textbf{Dodekaeder-Netzwerk} als fundamentale Raumstruktur vor. Sei $\mathcal{D} = (V,E)$ ein ungerichteter Graph mit:

\begin{itemize}
\item Knotenmenge $V$ als diskreten Raumzeit-Ereignissen
\item Kanten $E$ mit Längen $r_{ij}(t)$, die der Weber-Dynamik gehorchen
\end{itemize}

Die Wirkung des Systems lautet:
\begin{equation}
\mathcal{S}_{\mathcal{D}} = \sum_{(i,j)\in E} \int \left[ 
\frac{\mathcal{K}}{2} \left( \dot{r}_{ij}^2 - \frac{\dot{r}_{ij}^4}{4c^4} + \frac{r_{ij}\ddot{r}_{ij}}{2c^2} \right) 
+ \lambda \hbar \, \mathcal{I}_{ij} |\Psi_i - \Psi_j|^2 
\right] dt
\end{equation}

\subsection{Emergenz der Metrik}
Im Kontinuumslimes definieren wir die effektive Metrik:
\begin{equation}
g_{\mu\nu}(x) = \lim_{\epsilon \to 0} \frac{1}{N_\epsilon} \sum_{i,j \in B_\epsilon(x)} \frac{\Delta x_{ij}^\mu \Delta x_{ij}^\nu}{r_{ij}^2}
\end{equation}
wobei $B_\epsilon(x)$ eine $\epsilon$-Kugel um $x$ und $N_\epsilon$ die Normierung ist.

\subsection{Weber-Kraft aus Gitterdynamik}
Störungstheorie liefert die effektive Kraft:
\begin{equation}
F_{ij} = -\frac{\partial \mathcal{V}_{\text{eff}}}{\partial r_{ij}} = -\frac{G m_i m_j}{r_{ij}^2} \left(1 - \frac{\dot{r}_{ij}^2}{c^2} + \frac{r_{ij}\ddot{r}_{ij}}{2c^2}\right)
\end{equation}

\subsection{Quantenmechanische Erweiterung}
Die Führungswelle $\Psi$ entsteht als Eigenlösung des Netzwerkoperators:
\begin{equation}
\hat{H} \Psi = \left[ -\frac{\hbar^2}{2m} \sum_{\langle ijk \rangle} \nabla_{ijk}^2 + \mathcal{V}_{\text{WG}} \right] \Psi = E \Psi
\end{equation}
wobei $\nabla_{ijk}^2$ der diskrete Laplace-Operator auf Dodekaeder-Flächen ist.

\section{Vollständige Quantengravitation durch emergente Phänomene}
\subsection{Stufen der Emergenz}

\begin{enumerate}
    \item \textbf{Mikroskopische Ebene:} Das Dodekaeder-Netzwerk mit Weber-Dynamik
    \begin{equation}
        \mathcal{L}_{\text{mikro}} = \sum_{k=1}^{12} \left( \frac{1}{2}\dot{\phi}_k^2 - V(\phi_k) \right) + \mathcal{K} \sum_{\langle kl \rangle} \ddot{r}_{kl}^2
    \end{equation}
    
    \item \textbf{Mesoskopische Ebene:} Emergenz der effektiven Metrik
    \begin{equation}
        g_{\mu\nu}(x) = \langle \psi|\hat{g}_{\mu\nu}(x)|\psi \rangle, \quad \hat{g}_{\mu\nu} \sim \sum_p e^{ipx}\hat{a}_p + h.c.
    \end{equation}
    
    \item \textbf{Makroskopische Ebene:} Reproduktion der Einstein-Gleichungen
    \begin{equation}
        \delta \mathcal{S}_{\text{eff}} = 0 \Rightarrow G_{\mu\nu} + \Lambda g_{\mu\nu} = \frac{8\pi G}{c^4} T_{\mu\nu}
    \end{equation}
\end{enumerate}

\subsection{Emergenz der Materiefelder}
Die Teilchenphysik entsteht als Anregungsmoden:

\begin{equation}
    \mathcal{L}_{\text{Materie}} = \sum_{n=1}^{3} \bar{\psi}_n (i\gamma^\mu D_\mu - m_n)\psi_n + \frac{1}{4} F_{\mu\nu}F^{\mu\nu}
\end{equation}

wobei die Fermionen $\psi_n$ als topologische Defekte im Netzwerk auftreten:

\begin{equation}
    \psi(x) \sim \prod_{k \in \text{Umweg}} e^{i\int A_k dx^k} \phi_k
\end{equation}

\subsection{Quantengravitative Korrekturen}
Die vollständige Wirkung inkl. Quantenfluktuationen:

\begin{equation}
    \mathcal{S}_{\text{QG}} = \int \mathcal{D}g\mathcal{D}\phi \exp\left[ i\left( \mathcal{S}_{\text{EH}} + \mathcal{S}_{\text{WG}} + \hbar \Delta \mathcal{S}_{\text{top}} \right) \right]
\end{equation}

mit den charakteristischen Skalen:
\begin{itemize}
    \item Netzwerkkonstante: $\ell_p \sim 10^{-35}$ m
    \item Emergenzskala: $\ell_e \sim 10^{-18}$ m
    \item Beobachtungsskala: $\ell_o \sim 1$ m
\end{itemize}

\subsection{Experimentelle Signaturen}
\begin{table}[h]
    \centering
    \begin{tabular}{lll}
        \toprule
        Effekt & Skala & Nachweismethode \\
        \midrule
        Raumzeit-\enquote{Körnigkeit} & $10^{-18}$ m & Gamma-Astronomie \\
        Modified Dispersion & $10^{-3}$ eV & Atominterferometrie \\
        Quantisierte Flächen & $10^{-10}$ m & Neutronenstreuung \\
        \bottomrule
    \end{tabular}
    \caption{Vorhersagen des Modells}
\end{table}

\section{Renormierungsgruppenanalyse des Dodekaeder-Modells}
\subsection{Skalenabhängige Kopplungskonstanten}

Die Renormierungsgruppenflüsse der fundamentalen Parameter werden durch folgende Differentialgleichungen beschrieben:

\begin{equation}
\frac{dg_i}{d\ln\mu} = \beta_i(g_j), \quad i,j \in \{\mathcal{K},\lambda,G,\hbar\}
\end{equation}

Mit den beta-Funktionen:

\begin{equation}
\begin{aligned}
\beta_\mathcal{K} &= \frac{5-\sqrt{5}}{4\pi^2}\mathcal{K}^2 - \frac{\mathcal{K}^3}{12c^4} \\
\beta_\lambda &= 3\lambda^2 - \frac{\lambda}{2\pi}\left(\frac{\mathcal{K}}{c^2}-\frac{4G\mu^2}{\hbar c^3}\right) \\
\beta_G &= \frac{G^2\mu^2}{\hbar c^5}(2-\omega) \\
\beta_\hbar &= -\frac{\hbar}{2\pi}\left(\frac{\lambda^2}{4} + \frac{G\mu^2}{c^5}\right)
\end{aligned}
\end{equation}

\subsection{Fixpunkte und Phasenübergänge}

Die kritischen Punkte ergeben sich aus $\beta_i(g_j^*) = 0$:

\begin{table}[h]
\centering
\begin{tabular}{lccl}
\toprule
Fixpunkt & $\mathcal{K}^*$ & $\lambda^*$ & Stabilität \\
\midrule
Gaußscher FP & 0 & 0 & UV-stabil \\
Weber-FP & $\frac{5\pi c^2}{3}$ & 0 & IR-attraktiv \\
Quanten-FP & $\frac{2G\mu^2}{\hbar c}$ & $\frac{1}{\sqrt{3}}$ & kritisch \\
\bottomrule
\end{tabular}
\caption{Fixpunkte der Renormierungsgruppe}
\end{table}

\subsection{Anomalie-Dimensionen}

Die Skalendimensionen der Felder am Quanten-Fixpunkt:

\begin{equation}
\gamma_\phi = \frac{5-\sqrt{5}}{16\pi^2}\mathcal{K}^*, \quad \gamma_\Psi = \frac{\lambda^*}{12\pi^2}
\end{equation}

\subsection{Phasendiagramm}

\begin{equation}
\mathcal{F} = -T \ln Z = \int d^4x \left[ \frac{1}{2}(\nabla \phi)^2 + \frac{m^2}{2}\phi^2 + \frac{\lambda}{4!}\phi^4 \right]
\end{equation}

Mit den Übergangstemperaturen:

\begin{equation}
T_c = \begin{cases}
\frac{\hbar c}{k_B}\sqrt{\frac{3\mathcal{K}}{2G}} & \text{(Geometrie-Phase)} \\
\frac{\hbar}{k_B}\left(\frac{\lambda^3c^7}{G^2}\right)^{1/5} & \text{(Materie-Phase)}
\end{cases}
\end{equation}

\subsection{Asymptotische Freiheit}

Die effektive Kopplung zeigt das charakteristische Verhalten:

\begin{equation}
\alpha_{\text{WG}}(\mu) = \frac{\mathcal{K}(\mu)}{4\pi c^2} \approx \frac{1}{11\ln(\mu^2/\Lambda^2_{\text{QG}})}
\end{equation}

wobei $\Lambda_{\text{QG}} \sim 10^{19}$ GeV die Quantengravitationsskala ist.

\section{Emergenz der Allgemeinen Relativitätstheorie}
\label{sec:emergenz_art}

\subsection{Kontinuumslimes der Netzwerkdynamik}
Aus der mikroskopischen Wirkung des Dodekaeder-Netzwerks
\begin{equation}
\mathcal{S}_{\text{net}} = \sum_{\langle ij \rangle} \int \left[ \frac{\mathcal{K}}{2} \left( \dot{r}_{ij}^2 - \frac{\dot{r}_{ij}^4}{4c^4} + \frac{r_{ij}\ddot{r}_{ij}}{2c^2} \right) + \lambda \hbar \mathcal{I}_{ij} |\Psi_i - \Psi_j|^2 \right] dt
\end{equation}
emergiert im makroskopischen Limes ($N \to \infty$, $\ell_p \to 0$) die effektive Raumzeit-Metrik
\begin{equation}
g_{\mu\nu}(x) = \lim_{\epsilon \to 0} \frac{1}{N_\epsilon} \sum_{i,j \in B_\epsilon(x)} \frac{\Delta x_{ij}^\mu \Delta x_{ij}^\nu}{r_{ij}^2}
\end{equation}

\subsection{Herleitung der Einstein-Hilbert-Wirkung}
Die Entwicklung der Netzwerkfluktuationen $\delta r_{ij} = r_{ij} - \bar{r}$ führt auf
\begin{equation}
\mathcal{S}_{\text{eff}} = \int d^4x \sqrt{-g} \left( \frac{R}{16\pi G} - \Lambda + \mathcal{L}_{\text{m}} \right) + \mathcal{O}(\ell_p^2)
\end{equation}
mit:
\begin{itemize}
\item $G = \frac{3\sqrt{5}}{8} \frac{c^4\ell_p^2}{\mathcal{K}\hbar}$ (emergente Gravitationskonstante)
\item $\Lambda = \frac{5(1-\phi)}{2\ell_p^2}$ ($\phi$ = Goldberg-Coxeter-Parameter)
\item $\mathcal{L}_{\text{m}} = \sum_f \bar{\psi}_f (i\gamma^\mu D_\mu - m_f)\psi_f$ (emergente Materiefelder)
\end{itemize}

\subsection{Einstein-Gleichungen als effektive Dynamik}
Die Variation $\delta\mathcal{S}_{\text{eff}}/\delta g_{\mu\nu} = 0$ liefert
\begin{equation}
G_{\mu\nu} + \Lambda g_{\mu\nu} = 8\pi G T_{\mu\nu}
\end{equation}
wobei der Energie-Impuls-Tensor aus den Materiefluktuationen entsteht:
\begin{equation}
T_{\mu\nu} = \frac{2}{\sqrt{-g}} \frac{\delta (\sqrt{-g}\mathcal{L}_{\text{m}})}{\delta g^{\mu\nu}}
\end{equation}

\subsection{Konsistenzbedingungen}
Für die exakte Emergenz müssen gelten:
\begin{enumerate}
\item \textbf{Skalierungsrelation}:
\begin{equation}
\frac{\langle r_{ij}^2 \rangle}{\ell_p^2} = 5(3-\sqrt{5}) \approx 6.91
\end{equation}

\item \textbf{Lorentz-Symmetrierestoration}:
\begin{equation}
\text{SO}(4,1)_{\text{mikro}} \to \text{SO}(3,1)_{\text{macro}} \quad \text{für} \quad \frac{E}{E_p} < 10^{-5}
\end{equation}

\item \textbf{Störungstheoretische Konvergenz}:
\begin{equation}
\sum_{n=0}^\infty \left( \frac{\ell_p}{L} \right)^n \mathcal{O}_n \to 0 \quad \text{für} \quad L \gg \ell_p
\end{equation}
\end{enumerate}

\section{Autonome Quanten-Weber-Gravitation}
Die Weber-Gravitation (WG) formuliert eine eigenständige Quantentheorie der Gravitation,\\die ohne Rückgriff auf ART oder SRT auskommt. Ihre fundamentalen Prinzipien sind:

\subsection{Raumzeit-Konzept}
\begin{itemize}
    \item \textbf{Diskrete Basis}: Die Raumzeit wird durch ein dynamisches Dodekaeder-Netzwerk $\mathcal{D}=(V,E)$ beschrieben, wobei Knoten $v\in V$ Quantenzustände $|j,m\rangle$ und Kanten $e_{ij}\in E$ Operatoren $\hat{r}_{ij}$ mit Planck-Länge $\ell_p$ tragen.
    
    \item \textbf{Emergente Metrik}: Die klassische Metrik entsteht als Mittelwert:
    \begin{equation}
        g_{\mu\nu}(x) = \lim_{\epsilon\to0} \frac{1}{N_\epsilon} \sum_{\substack{i,j\in B_\epsilon(x)}} \frac{\Delta x_{ij}^\mu \Delta x_{ij}^\nu}{\langle \hat{r}_{ij}^2 \rangle}
    \end{equation}
\end{itemize}

\subsection{Quanten-Dynamik}
Die Zeitentwicklung folgt aus dem Weber-Hamiltonoperator:
\begin{equation}
    \hat{H}_{\text{WG}} = \sum_{\langle ij \rangle} \left( \frac{\hat{p}_{ij}^2}{2m_{ij}} - \frac{Gm_im_j}{\hat{r}_{ij}} \left[1 - \frac{\langle \hat{\dot{r}}_{ij}^2 \rangle}{c^2} + \beta \frac{\hat{r}_{ij}\hat{\ddot{r}}_{ij}}{c^2}\right] \right) + \hat{Q}
\end{equation}
wobei $\hat{Q} = -\frac{\hbar^2}{2m}\frac{\nabla^2|\Psi|}{|\Psi|}$ das nicht-lokale Quantenpotential ist.

\subsection{Schlüsselunterschiede zu ART/SRT}
\begin{tabular}{ll}
    \textbf{WG} & \textbf{ART/SRT} \\
    \hline
    Fernwirkung via $\hat{r}_{ij}$, $\hat{\ddot{r}}_{ij}$ & Lokale Krümmung $R_{\mu\nu}$ \\
    Frequenzabhängige Lichtablenkung & Metrik-basierte Ablenkung \\
    Statisches Universum mit $z\sim v_r^2$ & Expandierende Raumzeit \\
    Quantisiertes $r_{ij}\in n\ell_p$ & Kontinuierliche Koordinaten \\
\end{tabular}

\subsection{Konsequenzen}
\begin{itemize}
    \item Singularitätsfreiheit durch $\hat{Q}$-Divergenz bei $r_{ij}\to0$
    \item Lorentz-Invarianz nur als Niedrigenergie-Näherung
    \item Nachweisbare Abweichungen bei $\lambda\lesssim\ell_p$ (z.B. $\Delta\phi(\lambda)$)
\end{itemize}

\section{Fehlende Komponenten zur vollständigen Quantengravitation}

Die Quanten-Weber-Gravitation (QWG) benötigt folgende Erweiterungen, um eine vollständige Theorie der Quantengravitation zu werden:

\subsection{Quantenfeldtheorie der Gravitation}
\begin{itemize}
    \item Einführung eines gravitativen Feldoperators:
    \begin{equation}
        \hat{\mathcal{G}}_{\mu\nu}(x) = -\frac{\hbar^2}{\ell_p^2} \frac{\delta}{\delta g^{\mu\nu}(x)} \ln|\Psi[g]|
    \end{equation}
    \item Entwicklung einer Störungstheorie für $\hat{\mathcal{G}}_{\mu\nu}$
\end{itemize}

\subsection{Dynamische Quantisierung der Raumzeit}
\begin{itemize}
    \item Pfadintegral-Formulierung des Netzwerks:
    \begin{equation}
        Z = \int \mathcal{D}[r_{ij}] e^{i\mathcal{S}_{\text{net}}[r_{ij}]/\hbar}, \quad \mathcal{S}_{\text{net}} = \sum_{\langle ij \rangle} \mathcal{K}\left(\dot{r}_{ij}^2 + \beta r_{ij}\ddot{r}_{ij}\right)
    \end{equation}
    \item Spin-Netzwerk-Quantisierung der Zellflächen:
    \begin{equation}
        \hat{A}_f|j\rangle = 8\pi\gamma\ell_p^2\sqrt{j(j+1)}|j\rangle
    \end{equation}
\end{itemize}

\subsection{Kopplung an das Standardmodell}
\begin{itemize}
    \item Konstruktion chiraler Fermionen:
    \begin{equation}
        \hat{\psi}_L(x) = \prod_{\gamma_L} \hat{h}_{ij}, \quad \hat{\psi}_R(x) = \prod_{\gamma_R} \hat{h}_{ij}
    \end{equation}
    \item Generierung von Eichfeldern:
    \begin{equation}
        \hat{A}_\mu(x) = \sum_{\langle ij \rangle} \epsilon_\mu^{ij}(\hat{h}_{ij} - \hat{h}_{ji})
    \end{equation}
\end{itemize}

\subsection{Renormierung und UV-Vollständigkeit}
\begin{itemize}
    \item Erweiterte Renormierungsgruppengleichungen:
    \begin{equation}
        \beta_G = \frac{G^2}{\hbar c^5}(2 - \omega + a_1G^2 + a_2\lambda^2)
    \end{equation}
    \item Suche nach nicht-trivialen UV-Fixpunkten $(G^*,\lambda^*)$
\end{itemize}

\subsection{Kausalitätsstruktur}
\begin{itemize}
    \item Quanten-kausale Weber-Kraft:
    \begin{equation}
        \hat{F}_{ij}(t) = \int_{-\infty}^t K(t-t';\ell_p/c)\hat{F}_{ij}^{\text{inst}}(t')\,dt'
    \end{equation}
    \item Kausalkern mit Planck-Zeit-Cutoff:
    \begin{equation}
        K(\tau) \sim e^{-\tau^2c^2/2\ell_p^2}
    \end{equation}
\end{itemize}

\begin{table}[h]
\centering
\caption{Zusammenfassung der fehlenden Komponenten}
\begin{tabular}{ll}
\toprule
\textbf{Komponente} & \textbf{Erforderliche Erweiterung} \\
\midrule
Quantenfeldtheorie & Feldoperator $\hat{\mathcal{G}}_{\mu\nu}$ \\
Raumzeit-Quantisierung & Pfadintegral über Netzwerkkonfigurationen \\
Standardmodell & Chirale Fermionen und Eichfelder \\
UV-Vollständigkeit & Nicht-trivialer RG-Fixpunkt \\
Kausalität & Quanten-kausaler Propagator \\
\bottomrule
\end{tabular}
\end{table}

