\section{Kausalität durch Gleichzeitigkeit}
\label{sec:gleichzeitige_kausalitaet}

\subsection{Kernthese}
Die physikalische Standarddefinition von Kausalität ist unnötig restriktiv, wenn sie gleichzeitige Wechselwirkungen ausschließt. Ich argumentiere für einen erweiterten Kausalitätsbegriff, der zwei Prinzipien vereint:

\begin{itemize}
    \item \textbf{Determinismus}: Der Zustand $Z(t) = \{r, \dot{r}\}$ bestimmt eindeutig $Z(t+dt)$
    \item \textbf{Systemische Abhängigkeit}: Instantane Korrelationen sind kausal, wenn sie aus einer gemeinsamen Ursache folgen
\end{itemize}

\subsection{Anwendung auf die Weber-Kraft}
Die Weber-Gravitation zeigt dies exemplarisch:

\begin{equation}
    F = -\frac{GMm}{r^2}\left(1 - \frac{\dot{r}^2}{c^2} + \frac{r\ddot{r}}{2c^2}\right)
\end{equation}

\begin{itemize}
    \item Die Abhängigkeit von $\ddot{r}$ \textit{scheint} nicht-lokal
    \item Tatsächlich beschreibt sie eine \textit{systeminterne} Rückkopplung:
\end{itemize}

\begin{equation}
    \ddot{r} = f(r, \dot{r}) \quad \text{(lösbar nach Lipschitz-Bedingung)}
\end{equation}

\subsection{Philosophische Begründung}
\begin{itemize}
    \item Newtons 3. Gesetz wirkt ebenfalls instantan (actio = reactio)
    \item Quantenverschränkung zeigt: Gleichzeitige Korrelationen verletzen keine Kausalität
    \item Entscheidend ist nicht die \textit{Lokalität}, sondern die \textit{Eindeutigkeit} der Zeitentwicklung
\end{itemize}

\subsection{Konsequenzen}
\begin{tabular}{p{0.45\textwidth}p{0.45\textwidth}}
    \hline
    \textbf{Konventionelle Sicht} & \textbf{Diese Arbeit} \\
    \hline
    Kausalität erfordert Zeitverzögerung & Gleichzeitige Kausalität möglich \\
    Nicht-Lokalität = problematisch & Systemische Abhängigkeiten sind natürlich \\
    \hline
\end{tabular}
