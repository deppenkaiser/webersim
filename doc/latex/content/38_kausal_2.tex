\section{Konkrete Revision der Bewegungsgleichungen}
\label{sec:revision_bewegungsgleichungen}

\subsection{Das traditionelle Paradigma}
Klassische Bewegungsgleichungen folgen dem Ursache-Wirkung-Schema:

\begin{equation}
    m\ddot{\mathbf{r}} = \mathbf{F}(\mathbf{r},\dot{\mathbf{r}},t) \quad \text{(Newtons 2. Gesetz)}
\end{equation}

\subsection{Das Wellenprinzip revolutioniert dies in drei Aspekten}

\subsubsection{1. Globale statt lokale Abhängigkeit}
Die Dynamik wird durch instantane Energieoptimierung bestimmt:

\begin{equation}
    \frac{\delta \mathcal{E}[\Psi]}{\delta \Psi^*} = 0 \quad \text{mit} \quad 
    \mathcal{E} = \underbrace{\int \frac{\hbar^2}{2m}|\nabla\Psi|^2 dV}_{\text{kin. Anteil}} + \underbrace{\int V|\Psi|^2 dV}_{\text{pot. Anteil}}
\end{equation}

\subsubsection{2. Höhere Ableitungen werden essenziell}
Die Weber-Kraft zeigt dies explizit:

\begin{equation}
    \mathbf{F}_{\text{Weber}} \sim \ddot{r} \quad \text{(Beschleunigungsabhängigkeit)}
\end{equation}

\subsubsection{3. Nicht-Linearität als Normalfall}
Die Bewegungsgleichung wird intrinsisch nichtlinear:

\begin{equation}
    i\hbar\partial_t \Psi = -\frac{\hbar^2}{2m}\nabla^2 \Psi + \underbrace{g|\Psi|^2\Psi}_{\text{selbstwirkender Term}}
\end{equation}

\subsection{Konkretes Beispiel: Revisierte Gravitation}
Für die Weber-Gravitation bedeutet dies:

\begin{equation}
    \nabla^2 \phi = 4\pi G\rho \quad \rightarrow \quad \nabla^2 \phi = 4\pi G\left(\rho + \frac{1}{4c^2}\partial_t^2\rho\right)
\end{equation}

\subsection{Konsequenzen für die Theoriebildung}
\begin{tabular}{p{0.45\textwidth}p{0.45\textwidth}}
    \hline
    \textbf{Alte Gleichungen} & \textbf{Neue Form} \\
    \hline
    Lokale Kräfte & Globale Energieoptimierung \\
    Trennung von Feldern und Teilchen & Unified Wave Description \\
    Linearisierbare Systeme & Essentielle Nichtlinearität \\
    \hline
\end{tabular}
