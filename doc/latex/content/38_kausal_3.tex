\section{Interpretation der hybriden Lichtablenkungsformel}
\label{sec:hybrid_light_deflection}

\subsection{Gleichung (4.4.5) im Kontext}
Die kombinierte Vorhersage der Quanten-Weber-Gravitation für Lichtablenkung lautet:

\begin{equation}
\Delta\phi = \frac{4GM}{c^2b}\left(1 + \underbrace{\frac{3\pi}{16}\frac{\lambda^2}{\lambda_0^2}}_{\text{Weber-Term}} + \underbrace{\frac{\hbar^2\omega^2}{4c^4b^2}}_{\text{DBT-Korrektur}}\right)
\tag{4.4.5}
\end{equation}

\subsection{Term-für-Term-Analyse}

\subsubsection{1. ART-Grundterm}
\begin{itemize}
\item $\frac{4GM}{c^2b}$ ist die klassische Lichtablenkung der Allgemeinen Relativitätstheorie
\item Dominanter Beitrag für makroskopische Systeme ($b \gg \lambda$)
\end{itemize}

\subsubsection{2. Weber-Korrektur}
\begin{itemize}
\item $\frac{3\pi}{16}\frac{\lambda^2}{\lambda_0^2}$ ist die wellenlängenabhängige Modifikation
\item Physikalischer Ursprung: Kopplung der Photonenenergie $E=\hbar\omega$ an das Gravitationspotential
\item Experimentell nachweisbar durch Vergleich verschiedener Spektralbereiche
\end{itemize}

\subsubsection{3. Quantenmechanische Korrektur}
\begin{itemize}
\item $\frac{\hbar^2\omega^2}{4c^4b^2}$ stammt aus dem Quantenpotential der De-Broglie-Bohm-Theorie
\item Wird relevant bei Stoßparametern in Compton-Wellenlängennähe ($b \sim \hbar/mc$)
\item Manifestiert sich als \enquote{Quantenspreizung} der Lichtablenkung
\end{itemize}

\subsection{Experimentelle Konsequenzen}
Für verschiedene Wellenlängen $\lambda$ ergeben sich charakteristische Abweichungen:

\begin{table}[h]
\centering
\begin{tabular}{lcc}
\toprule
Wellenlänge & Weber-Term & DBT-Term \\
\midrule
Radiowellen ($1$ m) & $\sim 10^{-24}$ & vernachlässigbar \\
Sichtbares Licht ($500$ nm) & $\sim 10^{-18}$ & $\sim 10^{-34}$ \\
Gammastrahlung ($1$ pm) & $\sim 10^{-6}$ & $\sim 10^{-22}$ \\
\bottomrule
\end{tabular}
\caption{Vorhergesagte Abweichungen von der ART}
\end{table}

\subsection{Theoretische Bedeutung}
Gleichung (4.4.5) zeigt:
\begin{itemize}
\item Die Vereinheitlichung von Weber-Gravitation und Quantenphysik
\item Eine klare experimentelle Signatur zur Unterscheidung von der ART
\item Die Notwendigkeit neuer Messungen im Gammastrahlenbereich
\end{itemize}
