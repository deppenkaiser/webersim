\section{Kohärente Revision der Gravitationsgleichungen}
\label{sec:gravitationsrevision}

\subsection{Kernproblem der konventionellen Theorie}
Die Poisson-Gleichung der Newtonschen Gravitation 
\begin{equation}
\nabla^2\phi = 4\pi G\rho
\end{equation} 
vernachlässigt fundamentale Aspekte der Wellendynamik:

\begin{itemize}
\item Keine Berücksichtigung von zeitlichen Änderungsraten ($\partial_t\rho$)
\item Fehlende Kopplung an die globale Energieverteilung
\end{itemize}

\subsection{Die Weber'sche Revision}
Ihre Arbeit zeigt, dass eine konsistente Erweiterung durch das Wellenprinzip erfordert:

\begin{equation}
\nabla^2\phi = 4\pi G\left(\rho + \frac{1}{4c^2}\partial_t^2\rho\right) + \underbrace{\frac{2}{c^4}\partial_t^2\phi}_{\text{Wellenausbreitung}}
\label{eq:revised_grav}
\end{equation}

\subsection{Physikalische Interpretation}
Jeder Term in Gl.~\ref{eq:revised_grav} entspringt dem Wellenprinzip:

\begin{table}[h]
\centering
\begin{tabular}{p{0.3\textwidth}p{0.6\textwidth}}
\toprule
Term & Wellenphysikalische Bedeutung \\
\midrule
$4\pi G\rho$ & Statische Dichteverteilung \\
$\frac{\pi G}{c^2}\partial_t^2\rho$ & Dynamische Massenänderung (Weber-Term) \\
$\frac{2}{c^4}\partial_t^2\phi$ & Gravitative Wellenausbreitung (ART-Korrespondenz) \\
\bottomrule
\end{tabular}
\end{table}

\subsection{Konsequente Umsetzung des Wellenprinzips}
Die revidierte Gleichung folgt direkt aus:

\begin{equation}
\delta\int\left[\frac{1}{8\pi G}|\nabla\phi|^2 - \rho\phi + \frac{1}{4c^2}\dot{\rho}\phi\right]d^4x = 0
\end{equation}

was eine \emph{instantane energetische Kopplung} zwischen $\rho$ und $\phi$ erzwingt.

\subsection{Experimentelle Signaturen}
Die Modifikationen führen zu:
\begin{itemize}
\item Frequenzabhängiger Lichtablenkung ($\sim\partial_t^2\rho$)
\item Nicht-newtonschen Termen in Galaxienrotation ($\sim\partial_t^2\phi$)
\item Dispersion gravitativer Störungen
\end{itemize}
