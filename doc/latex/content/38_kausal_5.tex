\section{Rückwirkende Systematisierung durch das Wellenprinzip}
\label{sec:wellenprinzip_systematik}

\subsection{Das universelle Wirkprinzip}
Alle vorherigen Gleichungen lassen sich aus einem einheitlichen Variationsprinzip ableiten:

\begin{equation}
\delta \int \mathcal{L}[\Psi,\partial_\mu\Psi] d^4x = 0 \quad \text{mit} \quad 
\mathcal{L} = \underbrace{\frac{1}{2}\partial_\mu\Psi\partial^\mu\Psi}_{\text{Kinetik}} - \underbrace{V(\Psi)}_{\text{Selbstwirkung}}
\end{equation}

\subsection{Konsequente Revision der Grundgleichungen}

\subsubsection{1. Weber-Gravitation}
Die ursprüngliche Bewegungsgleichung
\begin{equation}
\mathbf{F} = -\frac{GMm}{r^2}\left(1-\frac{\dot{r}^2}{c^2}+\frac{r\ddot{r}}{2c^2}\right)
\end{equation}
wird zur Feldgleichung:
\begin{equation}
\Box \phi = 4\pi G\left(\rho + \frac{1}{c^2}\partial_t^2\rho\right)
\end{equation}

\subsubsection{2. Quantenpotential}
Das Bohm'sche Quantenpotential
\begin{equation}
Q = -\frac{\hbar^2}{2m}\frac{\nabla^2|\Psi|}{|\Psi|}
\end{equation}
erscheint als natürlicher Bestandteil des Wellenlagrangians:
\begin{equation}
\mathcal{L}_Q = \frac{\hbar^2}{2m}|\nabla\Psi|^2
\end{equation}

\subsubsection{3. Lichtablenkung}
Die hybriden Terme in
\begin{equation}
\Delta\phi = \frac{4GM}{c^2b}\left(1 + \frac{3\pi}{16}\frac{\lambda^2}{\lambda_0^2} + \frac{\hbar^2\omega^2}{4c^4b^2}\right)
\end{equation}
folgen aus der Störungstheorie des Wellenoperators:
\begin{equation}
\Box \Psi = \left(\frac{1}{c^2}\partial_t^2 - \nabla^2\right)\Psi = \frac{4\pi G}{c^4}T_{\mu\nu}\partial^\mu\partial^\nu\Psi
\end{equation}

\subsection{Systematische Klassifikation}
\begin{table}[h]
\centering
\begin{tabular}{p{0.25\textwidth}p{0.25\textwidth}p{0.25\textwidth}}
\toprule
\textbf{Originalgleichung} & \textbf{Wellenformulierung} & \textbf{Physikalische Konsequenz} \\
\midrule
Newton-Kraft & $\delta\mathcal{E}[\phi]=0$ & Weber-Terme \\
Schrödinger-Gleichung & $\Psi=e^{iS/\hbar}$ & Quantenpotential \\
Geodätengleichung & $\Box h_{\mu\nu}=0$ & Metrik-Wellen-Kopplung \\
\bottomrule
\end{tabular}
\caption{Systematische Vereinheitlichung durch das Wellenprinzip}
\end{table}

\subsection{Fundamentale Einsichten}
\begin{itemize}
\item Alle Wechselwirkungen sind Manifestationen \textit{einer} Grundgleichung
\item Der scheinbare Widerspruch zwischen Lokalität und Gleichzeitigkeit löst sich auf
\item Die \enquote{Kraft}-Begriffe werden durch energetische Optimierung ersetzt
\end{itemize}
