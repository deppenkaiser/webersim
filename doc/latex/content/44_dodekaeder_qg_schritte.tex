\section{Fehlende Komponenten zur vollständigen Quantengravitation}

Die Quanten-Weber-Gravitation (QWG) benötigt folgende Erweiterungen, um eine vollständige Theorie der Quantengravitation zu werden:

\subsection{Quantenfeldtheorie der Gravitation}
\begin{itemize}
    \item Einführung eines gravitativen Feldoperators:
    \begin{equation}
        \hat{\mathcal{G}}_{\mu\nu}(x) = -\frac{\hbar^2}{\ell_p^2} \frac{\delta}{\delta g^{\mu\nu}(x)} \ln|\Psi[g]|
    \end{equation}
    \item Entwicklung einer Störungstheorie für $\hat{\mathcal{G}}_{\mu\nu}$
\end{itemize}

\subsection{Dynamische Quantisierung der Raumzeit}
\begin{itemize}
    \item Pfadintegral-Formulierung des Netzwerks:
    \begin{equation}
        Z = \int \mathcal{D}[r_{ij}] e^{i\mathcal{S}_{\text{net}}[r_{ij}]/\hbar}, \quad \mathcal{S}_{\text{net}} = \sum_{\langle ij \rangle} \mathcal{K}\left(\dot{r}_{ij}^2 + \beta r_{ij}\ddot{r}_{ij}\right)
    \end{equation}
    \item Spin-Netzwerk-Quantisierung der Zellflächen:
    \begin{equation}
        \hat{A}_f|j\rangle = 8\pi\gamma\ell_p^2\sqrt{j(j+1)}|j\rangle
    \end{equation}
\end{itemize}

\subsection{Kopplung an das Standardmodell}
\begin{itemize}
    \item Konstruktion chiraler Fermionen:
    \begin{equation}
        \hat{\psi}_L(x) = \prod_{\gamma_L} \hat{h}_{ij}, \quad \hat{\psi}_R(x) = \prod_{\gamma_R} \hat{h}_{ij}
    \end{equation}
    \item Generierung von Eichfeldern:
    \begin{equation}
        \hat{A}_\mu(x) = \sum_{\langle ij \rangle} \epsilon_\mu^{ij}(\hat{h}_{ij} - \hat{h}_{ji})
    \end{equation}
\end{itemize}

\subsection{Renormierung und UV-Vollständigkeit}
\begin{itemize}
    \item Erweiterte Renormierungsgruppengleichungen:
    \begin{equation}
        \beta_G = \frac{G^2}{\hbar c^5}(2 - \omega + a_1G^2 + a_2\lambda^2)
    \end{equation}
    \item Suche nach nicht-trivialen UV-Fixpunkten $(G^*,\lambda^*)$
\end{itemize}

\subsection{Kausalitätsstruktur}
\begin{itemize}
    \item Quanten-kausale Weber-Kraft:
    \begin{equation}
        \hat{F}_{ij}(t) = \int_{-\infty}^t K(t-t';\ell_p/c)\hat{F}_{ij}^{\text{inst}}(t')\,dt'
    \end{equation}
    \item Kausalkern mit Planck-Zeit-Cutoff:
    \begin{equation}
        K(\tau) \sim e^{-\tau^2c^2/2\ell_p^2}
    \end{equation}
\end{itemize}

\begin{table}[h]
\centering
\caption{Zusammenfassung der fehlenden Komponenten}
\begin{tabular}{ll}
\toprule
\textbf{Komponente} & \textbf{Erforderliche Erweiterung} \\
\midrule
Quantenfeldtheorie & Feldoperator $\hat{\mathcal{G}}_{\mu\nu}$ \\
Raumzeit-Quantisierung & Pfadintegral über Netzwerkkonfigurationen \\
Standardmodell & Chirale Fermionen und Eichfelder \\
UV-Vollständigkeit & Nicht-trivialer RG-Fixpunkt \\
Kausalität & Quanten-kausaler Propagator \\
\bottomrule
\end{tabular}
\end{table}
