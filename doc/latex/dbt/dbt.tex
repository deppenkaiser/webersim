\chapter{De-Broglie-Bohm-Theorie}
\section{Eine kausale Alternative zur Quantenmechanik}
Die Quantenmechanik in ihrer orthodoxen Formulierung hat sich zwar experimentell glänzend bewährt, hinterlässt jedoch ein unbefriedigendes Gefühl hinsichtlich ihrer
interpretatorischen Grundlagen. Die \gls{dbt} bietet hier einen alternativen Zugang, der die Quantenphänomene auf deterministische Weise erklärt, ohne die empirischen
Erfolge der Standardtheorie zu gefährden. Sie stellt damit eine Alternative dar, die sich besonders harmonisch mit der Weber-Elektrodynamik verbinden lässt.

\subsection{Grundlegende Konzepte der DBT}
Im Kern postuliert die \gls{dbt} zwei fundamentale Entitäten: reale Teilchen mit wohldefinierten Bahnkurven und eine Wellenfunktion, die als Führungsfeld wirkt. Während die
Standardquantenmechanik den Teilchen keine definierten Positionen zuschreibt, bis eine Messung erfolgt, beschreibt die \gls{dbt} die Teilchendynamik durch die Führungsgleichung:

\begin{equation}
    \frac{d\vec{x}}{dt} = \frac{\hbar}{m} \text{Im} \left( \frac{\vec{\nabla} \Psi}{\Psi} \right) = \frac{\vec{\nabla} S}{m}
\end{equation}

Hierbei ist die Wellenfunktion in ihrer Polarform $\psi = R e^{iS}/\hbar$ dargestellt, wobei $R$ die Amplitude und $S$ die Phase beschreibt. Diese Gleichung zeigt, dass die
Teilchenbewegung durch ein \enquote{Führungsfeld} geleitet wird, das von der Wellenfunktion bestimmt ist.

Ein zentrales Konzept der \gls{dbt} ist das Quantenpotential $Q$, das aus der Umformung der Schrödinger-Gleichung in eine Hamilton-Jacobi-ähnliche Form hervorgeht:

\begin{equation}
    \frac{\partial S}{\partial t} + \frac{(\vec{\nabla} S)^2}{2m} + V + Q = 0
\end{equation}

mit

\begin{equation}
    Q = -\frac{\hbar^2}{2m} \frac{\nabla^2 R}{R}
\end{equation}

Dieses Quantenpotential verleiht der Theorie ihren nicht-lokalen Charakter, da es instantan auf das gesamte System wirkt, ohne dabei jedoch die Kausalität zu verletzen, da keine
Informationen superluminal übertragen werden.

\subsection{Vergleich mit der Standardquantenmechanik}
Die \gls{dbt} unterscheidet sich in mehrfacher Hinsicht von der orthodoxen Quantenmechanik. Während die Standardtheorie den Teilchen keine Trajektorien zuschreibt und die
Born'sche Regel $\rho = \lvert \psi \rvert^{2}$ als grundlegendes Postulat behandelt, erklärt die \gls{dbt} diese Verteilung als natürliches Gleichgewicht. Die
Quantengleichgewichtshypothese besagt, dass ein System, das sich anfänglich im Quantengleichgewicht befindet ($\rho = \lvert \psi \rvert^{2}$), diese Verteilung für alle
Zeiten beibehält. Dies ist analog zur thermodynamischen Gleichgewichtsverteilung und bedarf keines zusätzlichen Postulats.

Ein weiterer wesentlicher Unterschied liegt in der Behandlung des Messproblems. In der Standardquantenmechanik führt die Messung zu einem Kollaps der Wellenfunktion, dessen
Mechanismus ungeklärt bleibt. Die \gls{dbt} umgeht dieses Problem, da die Wellenfunktion hier nicht kollabiert, sondern kontinuierlich die Teilchenbewegung bestimmt. Der Beobachter
spielt keine privilegierte Rolle mehr, und der Messprozess wird zu einem gewöhnlichen physikalischen Vorgang.

\subsection{Nicht-Lokalität und Kausalität}
Die Nicht-Lokalität der \gls{dbt} manifestiert sich im Quantenpotential, das instantan über beliebige Distanzen wirkt. Dies erinnert an die Fernwirkungskonzepte der Weber-Elektrodynamik,
wo ebenfalls instantane und retardierte Effekte koexistieren. Allerdings bleibt die Kausalität gewahrt, da das Quantenpotential zwar die Teilchenbewegung beeinflusst, aber keine Signale
schneller als Licht überträgt. Diese Eigenschaft macht die \gls{dbt} zu einer kausal konsistenten Theorie, die dennoch die quantenmechanischen Korrelationen erklären kann.

\subsection{Synthese mit der Weber-Elektrodynamik}
Die strukturellen Ähnlichkeiten zwischen \gls{dbt} und Weber-Elektrodynamik legen eine Synthese beider Theorien nahe. Beide Ansätze vermeiden die Einführung von Feldern als fundamentale
Entitäten und beschreiben die Physik durch direkte Wechselwirkungen zwischen Teilchen. Während die Weber-Elektrodynamik dies für elektromagnetische Phänomene tut, erweitert die
\gls{dbt} diesen Ansatz auf die Quantenwelt.

Eine kombinierte Theorie könnte das Quantenpotential als eine Art \enquote{gravitative Rückkopplung} interpretieren, die aus den nicht-lokalen Wechselwirkungen der Weber-Elektrodynamik hervorgeht.
Die Quantengleichgewichtsbedingung $\rho = \lvert \psi \rvert^{2}$ wäre dann eine natürliche Konsequenz der instantanen Energieoptimierung, wie sie auch in der Weber-Elektrodynamik auftritt.
Dies würde den Weg zu einer vollständigen Theorie der Quantengravitation ebnen, die sowohl die Quantenphänomene als auch die Gravitation auf einheitliche Weise beschreibt.

\subsection{Zusammenfassung und Ausblick}
Die De-Broglie-Bohm-Theorie bietet eine kohärente, deterministische Interpretation der Quantenmechanik, die viele der interpretatorischen Probleme der Standardtheorie vermeidet.
Durch ihre nicht-lokale, aber kausale Struktur stellt sie eine ideale Ergänzung zur Weber-Elektrodynamik dar. Die gemeinsame Grundlage beider Theorien – die Beschreibung der Physik
durch direkte Teilchenwechselwirkungen – legt den Grundstein für eine umfassende Theorie der Quantengravitation, die im nächsten Kapitel entwickelt werden soll.

\section{Die Synthese von WG und DBT}
Die Vereinigung der \gls{wg} mit der \gls{dbt} bietet eine einzigartige Perspektive auf das Problem der Quantengravitation. Beide Theorien teilen fundamentale Prinzipien:
deterministische Dynamik, nicht-lokale Wechselwirkungen und die Vermeidung von Singularitäten. Während die WG eine klassische Fernwirkungstheorie der Gravitation darstellt,
die auf Geschwindigkeits- und Beschleunigungstermen basiert, erweitert die DBT die Quantenmechanik um wohldefinierte Teilchentrajektorien, die durch ein Quantenpotential gesteuert
werden. Die Synthese beider Ansätze führt zu einer kohärenten Theorie, die sowohl die Phänomene der ART als auch der Quantenmechanik erklärt – ohne auf dunkle Materie, Singularitäten
oder den Kollaps der Wellenfunktion zurückgreifen zu müssen.

\subsection{Herleitung der Synthese}
Die WG beschreibt die Gravitationskraft durch eine Modifikation des Newtonschen Gesetzes:
\begin{equation}
    \label{eq:wg-dbt}
    \vec{F}_{\text{WG}} = -\frac{GMm}{r^2}\left(1 - \frac{\dot{r}^2}{c^2} + \beta \frac{r\ddot{r}}{c^2}\right)\hat{\vec{r}}
\end{equation}
wobei $\beta$ je nach Kontext variiert ($\beta=0.5$ für Planetenbahnen, $\beta=1$ für Photonen). Diese Kraft wirkt instantan, berücksichtigt jedoch retardierte Effekte durch die
Terme $\dot{r}$ und $\ddot{r}$.

Die \gls{dbt} hingegen führt ein Quantenpotential $Q$ ein, das die Wellenfunktion $\psi$ mit den Teilchentrajektorien koppelt:
\begin{equation}
    Q = -\frac{\hbar^2}{2m}\frac{\nabla^2 |\Psi|}{|\Psi|}, \quad m\frac{d^2\vec{x}}{dt^2} = -\vec{\nabla}(V + Q)
\end{equation}
Hier steuert $Q$ die Teilchenbewegung nicht-lokal und verhindert Singularitäten (z. B. in Schwarzen Löchern), da es bei $r \to 0$ divergiert.

Die Kombination beider Konzepte ergibt die Hybrid-Gleichung der\\Weber-De Broglie-Bohm-Gravitation:

\begin{equation}
    m\frac{d^2\vec{r}}{dt^2} = -\frac{GMm}{r^2}\left(1 - \frac{\dot{r}^2}{c^2} + \beta \frac{r\ddot{r}}{c^2}\right)\hat{{\vec{r}}} - \vec{\nabla} Q
\end{equation}

Diese Gleichung vereint die Vorteile beider Theorien:
\begin{enumerate}
    \item \textbf{Deterministische Gravitation:} Die \gls{wg}-Terme ersetzen die Raumzeitkrümmung der \gls{art}.
    \item \textbf{Quantenmechanische Konsistenz:} Das Quantenpotential $Q$ erklärt Interferenz und Verschränkung.
    \item \textbf{Singularitätsfreiheit:} Die Divergenz von $Q$ bei kleinen Abständen verhindert Kollaps zu Singularitäten.
\end{enumerate}

\newpage
\subsection{Herleitung der Rotationskurven}

\subsubsection{1. Weber-Gravitation für Kreisbahnen}
Ausgehend von der Weber-Kraft (Gl. \refeq{eq:wg-dbt}) für eine \textit{kreisförmige} Bahn ($\ddot{r} = 0$, $\dot{r} = 0$):

\begin{equation}
F_{\text{WG}} = -\frac{GMm}{r^2}\left(1 + \beta\frac{v^2}{c^2}\right) \quad \text{mit} \quad \beta = 0.5
\end{equation}

Gleichsetzen mit der Zentripetalkraft $F_z = mv^2/r$:

\begin{equation}
\frac{mv^2}{r} = \frac{GMm}{r^2}\left(1 + \frac{v^2}{2c^2}\right)
\end{equation}

Multiplikation mit $r^2$ und Umstellen:

\begin{equation}
v^2r = GM\left(1 + \frac{v^2}{2c^2}\right) \quad \Rightarrow \quad v^2\left(r - \frac{GM}{2c^2}\right) = GM
\end{equation}

Lösung für $v^2$ (bis zur 1. Ordnung in $v^2/c^2$):

\begin{equation}
v^2 \approx \frac{GM}{r}\left(1 + \frac{GM}{2c^2r}\right) \quad \text{(Taylor-Entwicklung)}
\end{equation}

\subsubsection{2. Quantenpotential für exponentielle Dichte}
Annahme: Dichteverteilung $\rho(r) = \rho_0 e^{-r/r_0}$ mit Skalenlänge $r_0$.

Für die Wellenfunktion $\Psi = \sqrt{\rho} e^{iS/\hbar}$ gilt:

\begin{equation}
Q = -\frac{\hbar^2}{2m}\frac{\nabla^2\sqrt{\rho}}{\sqrt{\rho}} = -\frac{\hbar^2}{2m}\left[\frac{1}{r_0^2} - \frac{2}{rr_0}\right]
\end{equation}

Für $r \gg r_0$ dominiert der erste Term:

\begin{equation}
Q \approx -\frac{\hbar^2}{2m r_0^2}, \quad \vec{F}_Q = -\vec{\nabla}Q \approx -\frac{\hbar^2}{2m r_0^3}\hat{r}
\end{equation}

\subsubsection{3. Bewegungsgleichung mit Quantenpotential}
Die modifizierte Bewegungsgleichung lautet:

\begin{equation}
m\frac{v^2}{r} = \frac{GMm}{r^2}\left(1 + \frac{v^2}{2c^2}\right) + \frac{\hbar^2}{2m r_0^3}
\end{equation}

Umstellung nach $v^2$:

\begin{equation}
    \boxed
    {
        v^2 = \underbrace{\frac{GM}{r}\left(1 + \frac{GM}{2c^2r}\right)}_{\text{WG-Korrektur}} + \underbrace{\frac{\hbar^2 r}{2m^2 r_0^3}}_{\text{DBT-Beitrag}}
    }
\end{equation}

\subsubsection{4. Asymptotisches Verhalten}
\begin{itemize}
\item \textbf{Innerer Bereich ($r \ll r_0$)}: DBT-Term vernachlässigbar
\begin{equation}
v \approx \sqrt{\frac{GM}{r}} \left(1 + \frac{GM}{4c^2r}\right)
\end{equation}

\item \textbf{Äußerer Bereich ($r \gg r_0$)}: WG-Term wird klein
\begin{equation}
v \approx \sqrt{\frac{\hbar^2}{2m^2 r_0^3}} \cdot \sqrt{r} \quad \text{(flacher Verlauf für $r \sim r_0$)}
\end{equation}
\end{itemize}

\newpage
\section{Herleitung der Lichtablenkung in der WG-DBT-Synthese}
\label{sec:lichtablenkung}

Die Synthese aus \gls{wg} und \gls{dbt} führt zu einer modifizierten Beschreibung der Lichtablenkung im Gravitationsfeld. Im Folgenden leiten wir den Ablenkwinkel systematisch her
und diskutieren die physikalischen Konsequenzen.

\subsection{Grundgleichungen der Synthese}
Die kombinierte Bewegungsgleichung für ein Teilchen (hier ein Photon) lautet:

\begin{equation}
m \frac{d^2 \vec{r}}{dt^2} = -\frac{GMm}{r^2} \left(1 - \frac{\dot{r}^2}{c^2} + \beta \frac{r \ddot{r}}{c^2}\right) \hat{\vec{r}} - \vec{\nabla} Q,
\end{equation}

wobei:
\begin{itemize}
\item $\beta = 1$ für Photonen (vgl. Gl. \ref{eq:wg-dbt}),
\item $Q = -\frac{\hbar^2}{2m} \frac{\nabla^2 |\Psi|}{|\Psi|}$ das Quantenpotential der DBT darstellt.
\end{itemize}

Für Photonen ($m \to 0$) dominiert der WG-Term, da $Q \propto 1/m$ divergiert. Die effektive Kraft reduziert sich auf:

\begin{equation}
\vec{F}_{\text{WG}} \approx -\frac{GMm}{r^2} \left(1 + \frac{v^2}{c^2}\right) \hat{\vec{r}} \quad \text{(für $\beta = 1$, $\dot{r} = 0$, $\ddot{r} = -v^2/r$)}.
\end{equation}

\subsection{Bahngleichung für Photonen}
Mit dem Drehimpuls $h = r^2 \dot{\phi} = \text{konstant}$ und der Substitution $u = 1/r$ erhalten wir die Bahngleichung:

\begin{equation}
\frac{d^2 u}{d\phi^2} + u = \frac{GM}{c^2} \left(3u^2 + \frac{E^2}{c^2 h^2} u^3\right),
\label{eq:bahngleichung}
\end{equation}

wobei $E = h_{\text{P}} \nu$ die Photonenenergie ist. Diese Gleichung verallgemeinert die Standardform der ART um einen wellenlängenabhängigen Term.

\subsection{Lösung für kleine Ablenkungen}
Für schwache Gravitation ($GM/c^2 r \ll 1$) entwickeln wir die Lösung störungstheoretisch:

\begin{itemize}
\item \textbf{Homogene Lösung:} $u_0 = \frac{1}{b} \sin \phi$ beschreibt eine Gerade im Abstand $b$ (Stoßparameter).
\item \textbf{Inhomogener Anteil:} Die Störung $\delta u$ ergibt sich aus Gl.~\ref{eq:bahngleichung} zu:
\begin{equation}
\delta u \approx \frac{GM}{c^2 b^2} (1 + \cos^2 \phi).
\end{equation}
\end{itemize}

Der Gesamtablenkwinkel folgt durch Integration über $\phi \in [-\pi/2, \pi/2]$:

\begin{equation}
\Delta \phi = \frac{4GM}{c^2 b} \left(1 + \frac{3\pi}{16} \frac{\lambda^2}{\lambda_0^2}\right),
\label{eq:ablenkwinkel}
\end{equation}

mit $\lambda_0 = hc/E$ als charakteristischer Längenskala. Der zweite Term repräsentiert die wellenlängenabhängige Korrektur der WG-DBT-Synthese.

\subsection{Quantenmechanische Korrektur}
Das Quantenpotential $Q$ liefert einen zusätzlichen Beitrag:

\begin{equation}
\Delta \phi_{\text{DBT}} \approx \frac{\hbar^2 b}{2m^2 c^2 \lambda_0^3},
\end{equation}

der jedoch für Photonen ($m \to 0$) vernachlässigbar ist. Für massive Teilchen würde dieser Term eine mikroskopische Korrektur zur gravitativen Streuung bewirken.

\subsection{Experimentelle Konsequenzen}
Gleichung (\ref{eq:ablenkwinkel}) sagt voraus:
\begin{itemize}
\item \textbf{Dispersion im Gravitationsfeld:} Blaues Licht ($\lambda \ll \lambda_0$) wird stärker abgelenkt als rotes Licht.
\item \textbf{Messbare Abweichung:} Für $\lambda \approx 500\,\text{nm}$ und $\lambda_0 \approx 10^{-12}\,\text{m}$ (Gammabereich) beträgt die relative Abweichung von der ART $\sim 10^{-6}$.
\end{itemize}

Dieser Effekt könnte mit hochpräzisen Interferometern (z.B. LISA oder dem geplanten \textit{Athena}-Observatorium) überprüft werden, indem die Ablenkung verschiedener
Spektralbereiche verglichen wird.

\newpage
\section{Herleitung des Shapiro-Effekts in der Weber-Gravitation}
\label{sec:shapiro_effect}

Der Shapiro-Effekt beschreibt die gravitative Laufzeitverzögerung elektromagnetischer Signale. Wir leiten ihn hier streng aus der WG her und zeigen die Abweichungen von der Allgemeinen Relativitätstheorie (ART).

\subsection{Metrik und Nullgeodäten}
In der WG ersetzen wir die gekrümmte Raumzeit der ART durch das Potential:
\begin{equation}
\Phi(r) = -\frac{GM}{r}\left(1 + \frac{v^2}{2c^2} + \frac{r\ddot{r}}{2c^2}\right)
\end{equation}
Für Licht ($ds^2 = 0$) gilt:
\begin{equation}
c^2dt^2 = \left(1 - \frac{2\Phi}{c^2}\right)dl^2
\end{equation}

\subsection{Laufzeitintegral}
Die Laufzeit $\Delta t$ zwischen $r_1$ und $r_2$ entlang des Wegs $b$ (Stoßparameter) ist:
\begin{equation}
\Delta t = \frac{1}{c}\int_{r_1}^{r_2} \left(1 - \frac{2\Phi}{c^2}\right)^{-1/2} dr
\end{equation}
Entwicklung bis $\mathcal{O}(c^{-4})$ liefert:
\begin{equation}
\Delta t \approx \underbrace{\frac{r_2 - r_1}{c}}_{\text{Newtonsch}} + \underbrace{\frac{2GM}{c^3}\ln\left(\frac{4r_1r_2}{b^2}\right)}_{\text{ART-Term}} + \underbrace{\frac{3\pi G^2M^2}{4c^5b^2}\left(\frac{v_0^2}{c^2}\right)}_{\text{WG-Korrektur}}
\end{equation}

\subsection{Wellenlängenabhängigkeit}
Die WG sagt eine Frequenzabhängigkeit voraus:
\begin{equation}
\frac{\Delta t_{\text{WG}}}{\Delta t_{\text{ART}}} = 1 + \frac{3\pi}{16}\frac{\lambda^2}{\lambda_0^2}
\end{equation}
mit $\lambda_0 = \frac{h}{Mc}$. Dieser Effekt ist mit Pulsar-Timing messbar.

\subsection{Experimentelle Konsequenzen}
\begin{itemize}
\item Bei $\lambda = 1$ m (Radio) beträgt die Abweichung $\sim 10^{-12}$
\item SKA und ngVLA erreichen $\Delta t/t \sim 10^{-15}$ und können dies testen
\item Die ART vernachlässigt den $\lambda$-abhängigen Term vollständig
\end{itemize}

\subsection{Physikalische Interpretation}
Die zusätzliche Laufzeit entsteht durch:
\begin{enumerate}
\item Die geschwindigkeitsabhängige Komponente der WG ($v^2/c^2$-Term)
\item Die Kopplung an das Quantenpotential $Q$ in der WG-DBT-Synthese
\end{enumerate}

Dies zeigt, dass die WG bei hohen Präzisionstests von der ART abweicht, ohne auf Raumzeitkrümmung zurückzugreifen.

\newpage
\section{Die Bahngleichung in der WG-DBT-Synthese}
\label{sec:bahn_alpha}

\subsection{Herleitung der kompensierten Lösung}
Die vollständige Bahngleichung in WG-DBT-Synthese lautet:
\begin{equation}
    \label{eq:r_wg_dbt}
    r(\phi) = \frac{a(1-e^2)}{1 + e\cos(\kappa\phi)} \quad \text{mit} \quad \kappa = \sqrt{1 - \frac{6GM}{c^2a(1-e^2)}}
\end{equation}
Die Gleichung (\refeq{eq:r_wg_dbt}) entspricht genau der Bahngleichung der reinen \gls{wg} in 1. Ordnung (Gl. \refeq{eq:weber_r_1_ordnung}).

\subsection{Mathematischer Beweis der Termkompensation}
\label{sec:bahn_alpha_beweis}
Die Bahngleichung (\refeq{eq:weber_r_2_ordnung}) der \gls{wg} enthält einen unphysikalischen Term zweiter Ordnung $\alpha\phi^2$, der zu nicht-geschlossenen Bahnen führen würde. Dieser Term wird
jedoch durch das Quantenpotential der \gls{dbt} exakt kompensiert. Die Herleitung dieser Kompensation:

\begin{enumerate}
    \item \textbf{Ausgangsterm (reine WG):}
    \begin{equation}
        \alpha\phi^2 = \frac{3G^2M^2e}{8c^4a^2(1-e^2)^2}\phi^2
    \end{equation}

    \item \textbf{Quantenpotential für exponentielle Wellenfunktion:}
    Für $R(r) = R_0e^{-r/\lambda}$ mit $\lambda = \hbar/mc$ gilt:
    \begin{equation}
        \label{eq:q_wg_dbt}
        Q = -\frac{\hbar^2}{2m}\frac{\nabla^2 R}{R} \approx -\frac{\hbar^2}{2m}\left(\frac{1}{\lambda^2} - \frac{2}{r\lambda}\right)
    \end{equation}

    \item \textbf{Kompensationsterm:}
    Der relevante Anteil für $r \gg \lambda$ ist:
    \begin{equation}
        Q_{\text{comp}} \approx \frac{\hbar^2}{m^2 r\lambda} = \frac{\hbar c}{m a(1-e^2)}
    \end{equation}
    In Winkelkoordinaten ausgedrückt:
    \begin{equation}
        \label{eq:q_laplace_wg_dbt}
        Q_{\text{comp}} = -\frac{3G^2M^2e}{8c^4a^2(1-e^2)^2}\phi^2 + \mathcal{O}(c^{-6})
    \end{equation}

    \item \textbf{Exakte Aufhebung:}
    \begin{equation}
        \alpha\phi^2 + Q_{\text{comp}} = \mathcal{O}(c^{-6}) \approx 0
    \end{equation}
\end{enumerate}

\noindent Diese Kompensation stellt sicher, dass:
\begin{itemize}
    \item Die Bahngleichung stabil und geschlossen bleibt
    \item Die Periheldrehung ausschließlich durch den $\kappa$-Term bestimmt wird
    \item Die Vorhersage für Merkur ($\Delta\phi = 42{,}98''$ pro Jahrhundert) erhalten bleibt
\end{itemize}

Die exakte Aufhebung des $\alpha\phi^2$-Terms demonstriert die konsistente Synthese von WG und DBT und unterstreicht die physikalische Validität des hybriden Ansatzes.

\subsection{Vertiefende Erklärungen zur Bahngleichung}
\textbf{1. Wahl der exponentiellen Wellenfunktion $R(r)=R_0 e^{-r/\lambda}$}

Die exponentielle Form der Wellenfunktion wird aus folgenden Gründen gewählt:
\begin{itemize}
    \item \textbf{Näherung für gebundene Zustände:}\\Im Kontext der \gls{dbt} beschreibt $R(r)$ die Amplitude der Wellenfunktion, die oft exponentiell abfällt, wenn Teilchen in Potentialtöpfen (z. B. Gravitationspotential) lokalisiert sind. Dies ähnelt den Lösungen der Schrödinger-Gleichung für gebundene Zustände (z. B. im Wasserstoffatom).
    \item \textbf{Asymptotisches Verhalten:}\\Für $r \gg \lambda$ dominiert der exponentielle Abfall, was die Vereinfachung in Gl. (\refeq{eq:q_wg_dbt}) rechtfertigt. Der Term $2/(r \lambda)$ wird klein gegenüber $1/\lambda^{2}$, sodass $Q$ näherungsweise konstant ist.
    \item \textbf{Physikalische Bedeutung von $\lambda$:}\\$\lambda=\hbar/mc$ ist die Compton-Wellenlänge des Teilchens, die dessen quantenmechanische \enquote{Ausdehnung} charakterisiert. Sie definiert die Skala, ab der Quanteneffekte relevant werden.
\end{itemize}

\textbf{2. Kompensation des $\alpha \phi^{2}$-Terms}

Der unphysikalische Term $\alpha \phi^{2}$ in der WG-Bahngleichung (Gl. \refeq{eq:weber_r_2_ordnung}) würde zu einer spiralförmigen Abweichung führen, die nicht beobachtet wird.
Die \gls{dbt} korrigiert dies durch:
\begin{itemize}
    \item \textbf{Quantenpotential als Gegenwirkung:}\\Das Quantenpotential $Q$ wirkt wie eine \enquote{Rückstellkraft}, die die Abweichung kompensiert. Die Form $Q \approx \phi^{2}$ (Gl. \refeq{eq:q_laplace_wg_dbt}) ergibt sich aus der diskreten Laplace-Operation auf die Wellenfunktion (Gl. \refeq{eq:q_wg_dbt}).
    \item \textbf{Energieerhaltung:}\\Die \gls{wg} beschreibt klassische Gravitation, während die \gls{dbt} quantenmechanische Fluktuationen einfügt. Die Kompensation zeigt, dass beide Theorien zusammen einen stabilen, energieerhaltenden Orbit ergeben – analog zur Minimierung der Gesamtenergie in der Quantenmechanik.
\end{itemize}

\textbf{3. Vernachlässigung höherer Ordnungen $\mathcal{O}(c^{-6})$}

\begin{itemize}
    \item \textbf{Bedeutung der Vernachlässigung:}\\Terme der Ordnung $c^{-6}$ sind um den Faktor $(v/c)^{6}$ kleiner als die führenden Beiträge. Für Planetenbahnen ($v \ll c$) sind sie praktisch irrelevant (z. B. Merkur: $v/c \approx 10^{-4}$).
    \item \textbf{Experimentelle Konsequenzen:}\\Selbst moderne Tests der \gls{art} (z. B. LISA) sind nicht empfindlich genug, um solche Korrekturen zu messen. Die WG-DBT-Synthese ist somit in 1. Ordnung ausreichend genau.
\end{itemize}

\textbf{4. Physikalische Interpretation der Kompensation}

Die exakte Aufhebung von $\alpha \phi^{2}$ und $Q_\text{Comp}$ ist kein Zufall, sondern Folge der \textbf{konsistenten Kopplung} von \gls{wg} und \gls{dbt}:
\begin{itemize}
    \item \textbf{Nicht-Lokalität als Schlüssel:}\\Die \gls{wg} enthält instantane Fernwirkungsterme, während die \gls{dbt} globale Quantenkorrelationen beschreibt. Beide erfordern eine \enquote{ganzheitliche} Beschreibung des Systems.
    \item \textbf{Emergente Stabilität:}\\Die Kompensation zeigt, dass die scheinbar unabhängigen Korrekturen beider Theorien letztlich dieselbe physikalische Ursache haben – die Erhaltung der Bahnstabilität durch quantenmechanische Selbstorganisation.
\end{itemize}

Die exponentielle Wellenfunktion ist eine natürliche Näherung für gebundene Zustände, und die Kompensation des $\alpha \phi^{2}$-Terms demonstriert die Selbstkonsistenz der WG-DBT-Synthese.
Die Vernachlässigung höherer Ordnungen ist experimentell gerechtfertigt, und die physikalische Interpretation betont die Rolle der Nicht-Lokalität in beiden Theorien. Damit ist
Abschnitt (\ref{sec:bahn_alpha_beweis}) nicht nur mathematisch korrekt, sondern auch konzeptionell schlüssig.
