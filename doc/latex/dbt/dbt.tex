\chapter{De-Broglie-Bohm-Theorie}
\section{Eine kausale Alternative zur Quantenmechanik}
Die Quantenmechanik in ihrer orthodoxen Formulierung hat sich zwar experimentell glänzend bewährt, hinterlässt jedoch ein unbefriedigendes Gefühl hinsichtlich ihrer
interpretatorischen Grundlagen. Die \gls{dbt} bietet hier einen alternativen Zugang, der die Quantenphänomene auf deterministische Weise erklärt, ohne die empirischen
Erfolge der Standardtheorie zu gefährden. Sie stellt damit eine Alternative dar, die sich besonders harmonisch mit der Weber-Elektrodynamik verbinden lässt.

\subsection{Grundlegende Konzepte der DBT}
Im Kern postuliert die \gls{dbt} zwei fundamentale Entitäten: reale Teilchen mit wohldefinierten Bahnkurven und eine Wellenfunktion, die als Führungsfeld wirkt. Während die
Standardquantenmechanik den Teilchen keine definierten Positionen zuschreibt, bis eine Messung erfolgt, beschreibt die \gls{dbt} die Teilchendynamik durch die Führungsgleichung:

\begin{equation}
    \frac{d\vec{x}}{dt} = \frac{\hbar}{m} \text{Im} \left( \frac{\nabla \Psi}{\Psi} \right) = \frac{\nabla S}{m}
\end{equation}

Hierbei ist die Wellenfunktion in ihrer Polarform $\psi = R e^{iS}/\hbar$ dargestellt, wobei $R$ die Amplitude und $S$ die Phase beschreibt. Diese Gleichung zeigt, dass die
Teilchenbewegung durch ein \enquote{Führungsfeld} geleitet wird, das von der Wellenfunktion bestimmt ist.

Ein zentrales Konzept der \gls{dbt} ist das Quantenpotential $Q$, das aus der Umformung der Schrödinger-Gleichung in eine Hamilton-Jacobi-ähnliche Form hervorgeht:

\begin{equation}
    \frac{\partial S}{\partial t} + \frac{(\nabla S)^2}{2m} + V + Q = 0
\end{equation}

mit

\begin{equation}
    Q = -\frac{\hbar^2}{2m} \frac{\nabla^2 R}{R}
\end{equation}

Dieses Quantenpotential verleiht der Theorie ihren nicht-lokalen Charakter, da es instantan auf das gesamte System wirkt, ohne dabei jedoch die Kausalität zu verletzen, da keine
Informationen superluminal übertragen werden.

\subsection{Vergleich mit der Standardquantenmechanik}
Die \gls{dbt} unterscheidet sich in mehrfacher Hinsicht von der orthodoxen Quantenmechanik. Während die Standardtheorie den Teilchen keine Trajektorien zuschreibt und die
Born'sche Regel $\rho = \lvert \psi \rvert^{2}$ als grundlegendes Postulat behandelt, erklärt die \gls{dbt} diese Verteilung als natürliches Gleichgewicht. Die
Quantengleichgewichtshypothese besagt, dass ein System, das sich anfänglich im Quantengleichgewicht befindet ($\rho = \lvert \psi \rvert^{2}$), diese Verteilung für alle
Zeiten beibehält. Dies ist analog zur thermodynamischen Gleichgewichtsverteilung und bedarf keines zusätzlichen Postulats.

Ein weiterer wesentlicher Unterschied liegt in der Behandlung des Messproblems. In der Standardquantenmechanik führt die Messung zu einem Kollaps der Wellenfunktion, dessen
Mechanismus ungeklärt bleibt. Die \gls{dbt} umgeht dieses Problem, da die Wellenfunktion hier nicht kollabiert, sondern kontinuierlich die Teilchenbewegung bestimmt. Der Beobachter
spielt keine privilegierte Rolle mehr, und der Messprozess wird zu einem gewöhnlichen physikalischen Vorgang.

\subsection{Nicht-Lokalität und Kausalität}
Die Nicht-Lokalität der \gls{dbt} manifestiert sich im Quantenpotential, das instantan über beliebige Distanzen wirkt. Dies erinnert an die Fernwirkungskonzepte der Weber-Elektrodynamik,
wo ebenfalls instantane und retardierte Effekte koexistieren. Allerdings bleibt die Kausalität gewahrt, da das Quantenpotential zwar die Teilchenbewegung beeinflusst, aber keine Signale
schneller als Licht überträgt. Diese Eigenschaft macht die \gls{dbt} zu einer kausal konsistenten Theorie, die dennoch die quantenmechanischen Korrelationen erklären kann.

\subsection{Synthese mit der Weber-Elektrodynamik}
Die strukturellen Ähnlichkeiten zwischen \gls{dbt} und Weber-Elektrodynamik legen eine Synthese beider Theorien nahe. Beide Ansätze vermeiden die Einführung von Feldern als fundamentale
Entitäten und beschreiben die Physik durch direkte Wechselwirkungen zwischen Teilchen. Während die Weber-Elektrodynamik dies für elektromagnetische Phänomene tut, erweitert die
\gls{dbt} diesen Ansatz auf die Quantenwelt.

Eine kombinierte Theorie könnte das Quantenpotential als eine Art \enquote{gravitative Rückkopplung} interpretieren, die aus den nicht-lokalen Wechselwirkungen der Weber-Elektrodynamik hervorgeht.
Die Quantengleichgewichtsbedingung $\rho = \lvert \psi \rvert^{2}$ wäre dann eine natürliche Konsequenz der instantanen Energieoptimierung, wie sie auch in der Weber-Elektrodynamik auftritt.
Dies würde den Weg zu einer vollständigen Theorie der Quantengravitation ebnen, die sowohl die Quantenphänomene als auch die Gravitation auf einheitliche Weise beschreibt.

\subsection{Zusammenfassung und Ausblick}
Die De-Broglie-Bohm-Theorie bietet eine kohärente, deterministische Interpretation der Quantenmechanik, die viele der interpretatorischen Probleme der Standardtheorie vermeidet.
Durch ihre nicht-lokale, aber kausale Struktur stellt sie eine ideale Ergänzung zur Weber-Elektrodynamik dar. Die gemeinsame Grundlage beider Theorien – die Beschreibung der Physik
durch direkte Teilchenwechselwirkungen – legt den Grundstein für eine umfassende Theorie der Quantengravitation, die im nächsten Kapitel entwickelt werden soll.

