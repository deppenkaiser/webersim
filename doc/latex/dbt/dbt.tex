\chapter{De-Broglie-Bohm-Theorie}
\section{Eine kausale Alternative zur Quantenmechanik}
Die Quantenmechanik in ihrer orthodoxen Formulierung hat sich zwar experimentell glänzend bewährt, hinterlässt jedoch ein unbefriedigendes Gefühl hinsichtlich ihrer
interpretatorischen Grundlagen. Die \gls{dbt} bietet hier einen alternativen Zugang, der die Quantenphänomene auf deterministische Weise erklärt, ohne die empirischen
Erfolge der Standardtheorie zu gefährden. Sie stellt damit eine Alternative dar, die sich besonders harmonisch mit der Weber-Elektrodynamik verbinden lässt.

\subsection{Grundlegende Konzepte der DBT}
Im Kern postuliert die \gls{dbt} zwei fundamentale Entitäten: reale Teilchen mit wohldefinierten Bahnkurven und eine Wellenfunktion, die als Führungsfeld wirkt. Während die
Standardquantenmechanik den Teilchen keine definierten Positionen zuschreibt, bis eine Messung erfolgt, beschreibt die \gls{dbt} die Teilchendynamik durch die Führungsgleichung:

\begin{equation}
    \frac{d\vec{x}}{dt} = \frac{\hbar}{m} \text{Im} \left( \frac{\vec{\nabla} \Psi}{\Psi} \right) = \frac{\vec{\nabla} S}{m}
\end{equation}

Hierbei ist die Wellenfunktion in ihrer Polarform $\psi = R e^{iS}/\hbar$ dargestellt, wobei $R$ die Amplitude und $S$ die Phase beschreibt. Diese Gleichung zeigt, dass die
Teilchenbewegung durch ein \enquote{Führungsfeld} geleitet wird, das von der Wellenfunktion bestimmt ist.

Ein zentrales Konzept der \gls{dbt} ist das Quantenpotential $Q$, das aus der Umformung der Schrödinger-Gleichung in eine Hamilton-Jacobi-ähnliche Form hervorgeht:

\begin{equation}
    \frac{\partial S}{\partial t} + \frac{(\vec{\nabla} S)^2}{2m} + V + Q = 0
\end{equation}

mit

\begin{equation}
    Q = -\frac{\hbar^2}{2m} \frac{\nabla^2 R}{R}
\end{equation}

Dieses Quantenpotential verleiht der Theorie ihren nicht-lokalen Charakter, da es instantan auf das gesamte System wirkt, ohne dabei jedoch die Kausalität zu verletzen, da keine
Informationen superluminal übertragen werden.

\subsection{Vergleich mit der Standardquantenmechanik}
Die \gls{dbt} unterscheidet sich in mehrfacher Hinsicht von der orthodoxen Quantenmechanik. Während die Standardtheorie den Teilchen keine Trajektorien zuschreibt und die
Born'sche Regel $\rho = \lvert \psi \rvert^{2}$ als grundlegendes Postulat behandelt, erklärt die \gls{dbt} diese Verteilung als natürliches Gleichgewicht. Die
Quantengleichgewichtshypothese besagt, dass ein System, das sich anfänglich im Quantengleichgewicht befindet ($\rho = \lvert \psi \rvert^{2}$), diese Verteilung für alle
Zeiten beibehält. Dies ist analog zur thermodynamischen Gleichgewichtsverteilung und bedarf keines zusätzlichen Postulats.

Ein weiterer wesentlicher Unterschied liegt in der Behandlung des Messproblems. In der Standardquantenmechanik führt die Messung zu einem Kollaps der Wellenfunktion, dessen
Mechanismus ungeklärt bleibt. Die \gls{dbt} umgeht dieses Problem, da die Wellenfunktion hier nicht kollabiert, sondern kontinuierlich die Teilchenbewegung bestimmt. Der Beobachter
spielt keine privilegierte Rolle mehr, und der Messprozess wird zu einem gewöhnlichen physikalischen Vorgang.

\subsection{Nicht-Lokalität und Kausalität}
Die Nicht-Lokalität der \gls{dbt} manifestiert sich im Quantenpotential, das instantan über beliebige Distanzen wirkt. Dies erinnert an die Fernwirkungskonzepte der Weber-Elektrodynamik,
wo ebenfalls instantane und retardierte Effekte koexistieren. Allerdings bleibt die Kausalität gewahrt, da das Quantenpotential zwar die Teilchenbewegung beeinflusst, aber keine Signale
schneller als Licht überträgt. Diese Eigenschaft macht die \gls{dbt} zu einer kausal konsistenten Theorie, die dennoch die quantenmechanischen Korrelationen erklären kann.

\subsection{Synthese mit der Weber-Elektrodynamik}
Die strukturellen Ähnlichkeiten zwischen \gls{dbt} und Weber-Elektrodynamik legen eine Synthese beider Theorien nahe. Beide Ansätze vermeiden die Einführung von Feldern als fundamentale
Entitäten und beschreiben die Physik durch direkte Wechselwirkungen zwischen Teilchen. Während die Weber-Elektrodynamik dies für elektromagnetische Phänomene tut, erweitert die
\gls{dbt} diesen Ansatz auf die Quantenwelt.

Eine kombinierte Theorie könnte das Quantenpotential als eine Art \enquote{gravitative Rückkopplung} interpretieren, die aus den nicht-lokalen Wechselwirkungen der Weber-Elektrodynamik hervorgeht.
Die Quantengleichgewichtsbedingung $\rho = \lvert \psi \rvert^{2}$ wäre dann eine natürliche Konsequenz der instantanen Energieoptimierung, wie sie auch in der Weber-Elektrodynamik auftritt.
Dies würde den Weg zu einer vollständigen Theorie der Quantengravitation ebnen, die sowohl die Quantenphänomene als auch die Gravitation auf einheitliche Weise beschreibt.

\subsection{Zusammenfassung und Ausblick}
Die De-Broglie-Bohm-Theorie bietet eine kohärente, deterministische Interpretation der Quantenmechanik, die viele der interpretatorischen Probleme der Standardtheorie vermeidet.
Durch ihre nicht-lokale, aber kausale Struktur stellt sie eine ideale Ergänzung zur Weber-Elektrodynamik dar. Die gemeinsame Grundlage beider Theorien – die Beschreibung der Physik
durch direkte Teilchenwechselwirkungen – legt den Grundstein für eine umfassende Theorie der Quantengravitation, die im nächsten Kapitel entwickelt werden soll.

\section{Die Synthese von WG und DBT}
Die Vereinigung der \gls{wg} mit der \gls{dbt} bietet eine einzigartige Perspektive auf das Problem der Quantengravitation. Beide Theorien teilen fundamentale Prinzipien:
deterministische Dynamik, nicht-lokale Wechselwirkungen und die Vermeidung von Singularitäten. Während die WG eine klassische Fernwirkungstheorie der Gravitation darstellt,
die auf Geschwindigkeits- und Beschleunigungstermen basiert, erweitert die DBT die Quantenmechanik um wohldefinierte Teilchentrajektorien, die durch ein Quantenpotential gesteuert
werden. Die Synthese beider Ansätze führt zu einer kohärenten Theorie, die sowohl die Phänomene der ART als auch der Quantenmechanik erklärt – ohne auf dunkle Materie, Singularitäten
oder den Kollaps der Wellenfunktion zurückgreifen zu müssen.

\subsection{Herleitung der Synthese}
Die WG beschreibt die Gravitationskraft durch eine Modifikation des Newtonschen Gesetzes:
\[
    \vec{F}_{\text{WG}} = -\frac{GMm}{r^2}\left(1 - \frac{\dot{r}^2}{c^2} + \beta \frac{r\ddot{r}}{c^2}\right)\hat{\vec{r}}
\]
wobei $\beta$ je nach Kontext variiert ($\beta=0.5$ für Planetenbahnen, $\beta=1$ für Photonen). Diese Kraft wirkt instantan, berücksichtigt jedoch retardierte Effekte durch die
Terme $\dot{r}$ und $\ddot{r}$.

Die \gls{dbt} hingegen führt ein Quantenpotential $Q$ ein, das die Wellenfunktion $\psi$ mit den Teilchentrajektorien koppelt:
\begin{equation}
    Q = -\frac{\hbar^2}{2m}\frac{\nabla^2 |\Psi|}{|\Psi|}, \quad m\frac{d^2\vec{x}}{dt^2} = -\vec{\nabla}(V + Q)
\end{equation}
Hier steuert $Q$ die Teilchenbewegung nicht-lokal und verhindert Singularitäten (z. B. in Schwarzen Löchern), da es bei $r \to 0$ divergiert.

Die Kombination beider Konzepte ergibt die Hybrid-Gleichung der\\Weber-De Broglie-Bohm-Gravitation:

\begin{equation}
    m\frac{d^2\vec{r}}{dt^2} = -\frac{GMm}{r^2}\left(1 - \frac{\dot{r}^2}{c^2} + \beta \frac{r\ddot{r}}{c^2}\right)\hat{{\vec{r}}} - \vec{\nabla} Q
\end{equation}

Diese Gleichung vereint die Vorteile beider Theorien:
\begin{enumerate}
    \item \textbf{Deterministische Gravitation:} Die \gls{wg}-Terme ersetzen die Raumzeitkrümmung der \gls{art}.
    \item \textbf{Quantenmechanische Konsistenz:} Das Quantenpotential $Q$ erklärt Interferenz und Verschränkung.
    \item \textbf{Singularitätsfreiheit:} Die Divergenz von $Q$ bei kleinen Abständen verhindert Kollaps zu Singularitäten.
\end{enumerate}

\subsection{Physikalische Konsequenzen}
\textbf{1. Galaktische Rotation ohne dunkle Materie}\\
Die \gls{wg} erklärt flache Rotationskurven durch den Zusatzterm $\frac{GM}{4c^2r}$. Die \gls{dbt} liefert die mikroskopische Begründung:
\begin{equation}
    v(r) = \sqrt{\frac{GM}{r}} \left(1 + \frac{GM}{4c^2 r} + \frac{\hbar^2}{m^2 r^3} \nabla^2 \ln|\Psi|\right)^{1/2}
\end{equation}
Hier korrigiert $Q$ die Dynamik auf kleinen Skalen ($<1 pc$), während die \gls{wg}-Korrektur auf galaktischen Skalen wirkt.

\textbf{2. Frequenzabhängige Lichtablenkung}\\
Für Photonen ($m=0, \beta=1$) sagt die Synthese eine wellenlängenabhängige Ablenkung voraus:
\begin{equation}
    \Delta\phi = \frac{4GM}{c^2 b}\left(1 + \frac{3\pi}{16}\frac{\lambda^2}{\lambda_0^2} + \frac{\hbar^2 \omega^2}{4c^4 b^2}\right)
\end{equation}
Dieser Effekt wäre mit hochpräzisen Interferometern (z. B. LISA) nachweisbar und unterscheidet die Theorie von der \gls{art}.

\textbf{3. Quantisierte Periheldrehung}\\
In der Synthese zeigt die Periheldrehung diskrete Korrekturen durch $Q$:
\begin{equation}
    \Delta\phi_n = \Delta\phi_{\text{WG}} + n\frac{\hbar}{mc^2 g} \quad (n \in \mathbb{Z})    
\end{equation}
wobei $g$ die Gravitationsbeschleunigung ist. Dies könnte durch Laser-Ranging-Experimente am Merkur überprüft werden.

Die Synthese aus \gls{wg} und \gls{dbt} bietet eine minimalistische, aber leistungsfähige Alternative zu konventionellen Quantengravitationsmodellen. Sie vermeidet nicht nur
die Probleme der \gls{art} (Singularitäten, dunkle Materie) und der Quantenmechanik (Messproblem), sondern liefert auch testbare Vorhersagen. Während sie keine vollständige
Feldquantisierung der Gravitation leistet, ist sie eine effektive Theorie für Skalen unterhalb der Planck-Länge und oberhalb der Quantenfluktuationen. Ihr größter Vorteil
liegt in der kausalen Interpretierbarkeit aller Phänomene – von der kosmologischen Rotverschiebung bis zur Quantenverschränkung.
