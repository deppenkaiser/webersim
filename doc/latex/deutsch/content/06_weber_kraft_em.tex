\section{Klassische Weber-Kraft (Elektrodynamik)}
Die Ausgangsgleichung der Weber-Elektrodynamik soll in der hier angegebenen Form als \textbf{\enquote{skalare Form}} bezeichnet werden.

\begin{equation}
    \boxed
    {
        \bm{F}_{\text{Weber}}^{\text{EM}} = \frac{Qq}{4\pi\epsilon_0 r^2}\left(1 - \frac{\dot{r}^2}{c^2} + \frac{2r\ddot{r}}{c^2}\right)\bm{\hat{r}}
    }
\end{equation}

\subsection*{Symbolbeschreibung}
\begin{itemize}[leftmargin=*,noitemsep]
    \item $\bm{F}_{\text{Weber}}^{\text{EM}}$: Weber-Kraft zwischen Ladungen
    \item $Q, q$: Elektrische Ladungen
    \item $\epsilon_0$: Elektrische Feldkonstante
    \item $r$: Ladungsabstand
    \item $\dot{r} = \frac{dr}{dt}$: Relative Radialgeschwindigkeit
    \item $\ddot{r} = \frac{d^2r}{dt^2}$: Relative Radialbeschleunigung
    \item $c$: Lichtgeschwindigkeit
    \item $\bm{\hat{r}}$: Radialer Einheitsvektor
\end{itemize}

\subsection*{Beziehung zur Coulomb-Kraft}
\begin{itemize}[leftmargin=*,noitemsep]
    \item Erster Term entspricht Coulomb-Kraft: $\frac{Qq}{4\pi\epsilon_0 r^2}$
    \item Zusatzterme $\left(-\frac{\dot{r}^2}{c^2} + \frac{2r\ddot{r}}{c^2}\right)$ beschreiben Bewegungsabhängige Korrekturen
    \item Reduktion auf Coulomb-Kraft im statischen Fall ($\dot{r} = \ddot{r} = 0$)
\end{itemize}

\newpage
\section{Verallgemeinerung der Weber-Kraft}

Die ursprüngliche skalare Form der Weber-Kraft lautet:

\begin{equation}
F = F_Q \left(1 - \frac{\dot{r}^2}{c^2} + \frac{2r\ddot{r}}{c^2}\right)
\end{equation}

wobei $F_Q = \dfrac{q_1 q_2}{4\pi\epsilon_0 r^2}$ die Coulomb-Kraft ist.

\subsection{Schritt 1: Ersetzung der zeitlichen Ableitungen}

Für einen beliebigen Ortsvektor $\mathbf{r}(t)$ mit $r = |\mathbf{r}|$ gilt:

\begin{align}
\dot{r} &= \frac{d}{dt}\sqrt{\mathbf{r}\cdot\mathbf{r}} = \frac{\mathbf{r}\cdot\dot{\mathbf{r}}}{r} = \frac{\mathbf{r}\cdot\mathbf{v}}{r} \\
\ddot{r} &= \frac{d}{dt}\left(\frac{\mathbf{r}\cdot\mathbf{v}}{r}\right) = \frac{\dot{\mathbf{r}}\cdot\mathbf{v} + \mathbf{r}\cdot\dot{\mathbf{v}}}{r} - \frac{(\mathbf{r}\cdot\mathbf{v})^2}{r^3} \\
&= \frac{v^2 + \mathbf{r}\cdot\mathbf{a}}{r} - \frac{(\mathbf{r}\cdot\mathbf{v})^2}{r^3}
\end{align}

\subsection{Schritt 2: Einsetzen in die ursprüngliche Gleichung}

Setzen wir diese Ausdrücke in die Weber-Kraft ein:

\begin{align}
F &= F_Q \left[1 - \frac{(\mathbf{r}\cdot\mathbf{v})^2}{c^2 r^2} + \frac{2r}{c^2}\left(\frac{v^2 + \mathbf{r}\cdot\mathbf{a}}{r} - \frac{(\mathbf{r}\cdot\mathbf{v})^2}{r^3}\right)\right] \\
&= F_Q \left[1 - \frac{(\mathbf{r}\cdot\mathbf{v})^2}{c^2 r^2} + \frac{2(v^2 + \mathbf{r}\cdot\mathbf{a})}{c^2} - \frac{2(\mathbf{r}\cdot\mathbf{v})^2}{c^2 r^2}\right]
\end{align}

\subsection{Schritt 3: Vereinfachung und Endergebnis}

Zusammenfassen der Terme ergibt die verallgemeinerte vektorielle Form, diese Form soll als \textbf{\enquote{vektorielle Form}} bezeichnet werden:

\begin{equation}
    \boxed
    {
        F = F_Q \left(1 + \frac{2v^2}{c^2} + \frac{2\mathbf{r}\cdot\mathbf{a}}{c^2} - \frac{3(\mathbf{r}\cdot\mathbf{v})^2}{c^2 r^2}\right)
    }
\end{equation}

\subsection{Interpretation der Terme}

\begin{itemize}
\item $\dfrac{2v^2}{c^2}$: Relativistische Korrektur der kinetischen Energie
\item $\dfrac{2\mathbf{r}\cdot\mathbf{a}}{c^2}$: Beschleunigungsabhängiger Term
\item $\dfrac{3(\mathbf{r}\cdot\mathbf{v})^2}{c^2 r^2}$: Richtungsabhängige Korrektur für nicht-radiale Bewegung
\end{itemize}

\subsection{Spezialfall: Radiale Bewegung}

Für $\mathbf{v} \parallel \mathbf{r}$ ($\mathbf{r}\cdot\mathbf{v} = rv$) und $\mathbf{a} = 0$ vereinfacht sich dies zu:

\begin{equation}
F = F_Q \left(1 + \frac{2v^2}{c^2} - \frac{3v^2}{c^2}\right) = F_Q \left(1 - \frac{v^2}{c^2}\right)
\end{equation}

was exakt mit der entsprechenden Lösung der Maxwell-Theorie übereinstimmt.

\section{Weber-Kraft und Gravitation}
\subsection*{Tisserands Ansatz}
Die Übertragung der elektrodynamischen Weber-Kraft \cite{tisserand1894} auf die Gravitation scheiterte an der Erklärung der Periheldrehung des Merkurs.

\subsection*{Hinweis}
Die korrekte gravitative Formulierung wird separat vorgestellt und erfordert eine Modifikation der Original-Weberschen Formel.
