\section{Die Rolle des $\beta$-Parameters}

Der $\beta$-Parameter in der Weber-Kraft

\begin{equation}
F = -\frac{GMm}{r^2}\left(1 - \frac{\dot{r}^2}{c^2} + \beta\frac{r\ddot{r}}{c^2}\right)\hat{r}
\end{equation}

bestimmt das Verhältnis von Beschleunigungs- zu Geschwindigkeitstermen und variiert je nach Wechselwirkungstyp:

\subsection{Elektrodynamik (Original-Weber)}
Für elektromagnetische Wechselwirkungen gilt $\beta=2$:
\begin{itemize}
\item Führt zur korrekten Beschreibung beschleunigter Ladungen
\item Reproduziert die magnetische Komponente der Lorentz-Kraft
\item Keine Lichtablenkung ($m=0$ liefert $F=0$)
\end{itemize}

\subsection{Gravitation (Massen)}
Für massive Körper im Gravitationsfeld:
\begin{itemize}
\item $\beta=0.5$ erklärt die Periheldrehung des Merkur
\item Führt zur ART-konformen Lichtablenkung für makroskopische Körper
\item Universelle Formel: $\beta = 1 - \frac{mc^2}{2E}$
\end{itemize}

\subsection{Photonen (Lichtablenkung)}
Für masselose Teilchen ($m=0$, $E=h\nu$):
\begin{itemize}
\item $\beta=1$ erzwingt die Frequenzabhängigkeit
\item Beschleunigungsterm dominiert: $\frac{r\ddot{r}}{c^2} \approx \frac{h^2}{c^2r^4}$
\item Liefert den Zusatzterm $\propto \lambda^{-2}$
\end{itemize}

\begin{table}[h]
\centering
\caption{$\beta$-Werte im Vergleich}
\begin{tabular}{lcc}
\hline
Anwendung & $\beta$ & Physikalische Konsequenz \\
\hline
Elektrodynamik & 2 & Magnetische Wechselwirkungen \\
Gravitation (Massen) & 0.5 & Periheldrehung des Merkur \\
Photonen & 1 & Frequenzabhängige Lichtablenkung \\
\hline
\end{tabular}
\end{table}