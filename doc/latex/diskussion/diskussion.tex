\chapter{Diskussion}
\section{Eine quantisierte De-Broglie-Bohm-Theorie – Konsequenzen und Perspektiven}
Die Idee einer raumzeitlich quantisierten \gls{dbt} stellt einen radikalen, aber folgerichtigen Schritt in der Entwicklung einer physikalisch konsistenten Quantengravitation dar.
Wenn wir annehmen, dass sowohl Raum als Zeit nicht kontinuierlich, sondern aus diskreten Einheiten bestehen, ergeben sich tiefgreifende Konsequenzen für die Struktur der \gls{dbt} – und
möglicherweise Lösungen für einige ihrer offenen Fragen.

\subsection{Grundannahmen des Modells}
In dieser modifizierten \gls{dbt} wird die klassische Raumzeit durch ein diskretes Gitter ersetzt:
\begin{itemize}
    \item \textbf{Raum} ist ein Vielfaches einer fundamentalen Länge $l_0$ (z. B. Planck-Länge oder Compton-Wellenlänge eines Elementarteilchens).
    \item \textbf{Zeit} verläuft in ganzzahligen Schritten $t_n = n\tau_0$ wobei $\tau_0$ eine elementare Zeiteinheit darstellt.
    \item Die Wellenfunktion $\psi$ wird nicht mehr über einen kontinuierlichen Raum, sondern über diskrete Gitterpunkte definiert.
\end{itemize}
Diese Annahmen führen zu einer digitalen Physik, in der alle messbaren Größen – Positionen, Impulse, Energien – als ganzzahlige Vielfache elementarer Einheiten auftreten.

\subsection{Konsequenzen für die Dynamik der DBT}
\textbf{(a) Das Quantenpotential wird diskret}\\
In der Standard-\gls{dbt} steuert das Quantenpotential (Gl. \refeq{eq:bohm_potenzial}) die Teilchenbewegung. In der quantisierten Version müssen Ableitungen durch Finite
Differenzen ersetzt werden:
\begin{equation}
    \nabla^{2} \psi \to \sum_\text{Nachbarn j} \left( \psi_j - \psi_i \right),
\end{equation}
wobei die Summe über benachbarte Gitterpunkte läuft. Das Quantenpotential erhält damit eine lokal begrenzte Wirkung, was die Nicht-Lokalität der DBT mildert, ohne sie ganz aufzuheben.

\textbf{(b) Teilchentrajektorien werden schrittweise}\\
Die Bahnen von Teilchen sind nicht mehr glatte Kurven, sondern Sprünge zwischen Gitterpunkten, getaktet durch die diskrete Zeit. Dies erinnert an Pfadintegral-Formulierungen der
Quantenmechanik, bei denen Teilchen alle möglichen Pfade \enquote{abtasten} – nur dass hier die Pfade auf das Gitter beschränkt sind.

\textbf{(c) Natürliche Regularisierung der Vakuumenergie}\\
Ein Hauptproblem der Quantenfeldtheorie – die divergente Vakuumenergie – entfällt, da das Modell eine kürzestmögliche Wellenlänge $\lambda_\text{min} = 2l_0$ vorsieht. Hochfrequente Fluktuationen,
die in kontinuierlichen Theorien zu Unendlichkeiten führen, werden automatisch abgeschnitten.

\subsection{Experimentelle Konsequenzen}
Falls Raum und Zeit tatsächlich quantisiert sind, müssten sich in Präzisionsexperimenten Abweichungen von der Standard-\gls{dbt} zeigen:

\begin{itemize}
    \item \textbf{Energieniveaus in Atomen:} Die diskrete Raumzeit würde zu minimalen Verschiebungen in Spektrallinien führen, insbesondere bei schweren Atomen.
    \item \textbf{Quanteninterferenz:} Doppelspaltexperimente mit sehr kurzen Wellenlängen könnten \enquote{Pixelierungs-Effekte} offenbaren.
\end{itemize}

\subsection{Philosophische Implikationen}
Diese Theorie würde die ontologische Frage nach der Natur der Realität neu stellen:
\begin{itemize}
    \item Ist die Wellenfunktion nur ein mathematisches Hilfsmittel – oder bildet sie eine fundamentale, diskrete Struktur ab?
    \item Wenn Raum und Zeit zählbar sind, könnte das Universum letztlich ein algorithmischer Prozess sein, bei dem $\psi$ die \enquote{Programmierung} und $Q$ die \enquote{Ausführungsregeln} darstellt.
    \item Die Nicht-Lokalität der Quantenmechanik würde zu einer geometrischen Eigenschaft des Gitters – ähnlich wie Verschränkung in Tensor-Netzwerk-Modellen.
\end{itemize}

\subsection{Die quantisierte De-Broglie-Bohm-Theorie}
\label{sec:discrete-dbb}

\subsubsection{Grundgleichungen}
Die Wellenfunktion lebt auf einem diskreten Gitter mit Abstand $\ell_0$ und Zeitschritten $\tau_0$:

\begin{equation}
\Psi(\vec{r}, t) \rightarrow \Psi_{i,j,k}^n \quad \text{mit} \quad 
\begin{cases}
\vec{r} = (i\ell_0, j\ell_0, k\ell_0) & i,j,k \in \mathbb{Z} \\
t = n \tau_0 & n \in \mathbb{N}
\end{cases}
\end{equation}

Das Quantenpotential wird diskretisiert:

\begin{equation}
Q_{i,j,k}^n = -\frac{\hbar^2}{2m\ell_0^2} \left( \frac{\Delta^2 R}{R} \right)_{i,j,k}^n
\end{equation}

wobei der diskrete Laplace-Operator:

\begin{equation}
(\Delta^2 R)_{i,j,k} = R_{i+1,j,k} + R_{i-1,j,k} + \text{(zyklisch)} - 6R_{i,j,k}
\end{equation}

\subsubsection{Bewegungsgleichung}
Die Teilchentrajektorie $\vec{r}(t)$ wird zu einer Folge von Gittersprüngen:

\begin{equation}
\vec{r}^{~n+1} = \vec{r}^{~n} + \tau_0 \left. \frac{\nabla S}{m} \right|_{\vec{r}^{~n}}^n
\end{equation}

mit der diskreten Phase $S_{i,j,k}^n = \hbar \arg(\Psi_{i,j,k}^n)$.

Eine quantisierte \gls{dbt} bietet eine brückenschlagende Perspektive zwischen der deterministischen Führung der Bohm'schen Mechanik und den diskreten Strukturen der
Quantengravitation. Während sie experimentell noch nicht überprüft ist, liefert sie ein faszinierendes Gedankenmodell, das zeigt:
\begin{itemize}
    \item Die Raumzeit könnte emergenter sein als angenommen.
    \item Die Wellenfunktion könnte eine tiefere, algorithmische Bedeutung haben.
    \item Die DBT ist anpassungsfähiger, als ihre traditionelle Form vermuten lässt.
\end{itemize}
Diese Überlegungen werfen mehr Fragen auf, als sie beantworten – aber genau das macht sie zu einem lohnenden Thema für die zukünftige physikalische Grundlagenforschung.
