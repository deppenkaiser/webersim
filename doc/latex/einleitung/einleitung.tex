\chapter{Einleitung}
\section{Motivation}
Viele Schüler und Studierende erleben den Physikunterricht als frustrierend und unverständlich. Besonders die moderne Physik – mit der \gls{art}
und der \gls{srt} – wirkt oft unphysikalisch und voller logischer Widersprüche. Energie scheint unter bestimmten Bedingungen unendlich zu werden,
Überlichtgeschwindigkeit wird in manchen Fällen postuliert, obwohl sie eigentlich unmöglich sein soll, und Begriffe wie \enquote{dunkle Energie} oder \enquote{dunkle Materie} wirken wie
Platzhalter für unser Unverständnis.

Ein grundlegendes Problem liegt in den Widersprüchen zwischen \gls{art} und \gls{srt}. Die \gls{srt} baut auf Inertialsystemen auf, also Bezugssystemen, die sich gleichförmig und unbeschleunigt
bewegen. Laut \gls{art} gibt es keine perfekten Inertialsysteme, da jede Masse die Raumzeit krümmt und damit Beschleunigungen erzeugt. Schon allein dieser Widerspruch wirft
Fragen auf: Wenn Inertialsysteme streng genommen punktförmig sein müssten, um frei von jeder Krümmung zu sein, bräuchte man unendlich viele davon – und damit auch unendlich
viele verschiedene Lichtgeschwindigkeiten, da diese vom Bezugssystem abhängt.
\begin{quote}
    \enquote{Einstein's postulates contain inherent contradictions when applied to real gravitational systems, challenging the universality of special relativity.} \cite{Rubcic1998}
\end{quote}
Hinzu kommt, dass viele Konzepte der modernen Physik unserer Intuition widersprechen. Die Quantenmechanik verlangt, dass Teilchen gleichzeitig Wellen sind und erst durch
Beobachtung einen definierten Zustand annehmen.
\begin{quote}
    \enquote{The observer-dependent collapse of the wavefunction is not a fundamental feature of nature but a limitation of the standard interpretation.} \cite{bohm1952}
\end{quote}
Die \gls{art} beschreibt eine gekrümmte Raumzeit, die sich kaum jemand wirklich vorstellen kann, und die \gls{srt} führt zu scheinbar
paradoxen Zeitdehnungen und Längenkontraktionen. Selbst der Urknall als Anfangspunkt des Universums wirft Fragen auf: Wie kann etwas aus dem Nichts entstehen? Warum gibt es
überhaupt eine Singularität, wenn doch unsere physikalischen Gesetze dort versagen?

All diese Punkte zeigen, dass die moderne Physik noch lange nicht abgeschlossen ist. Statt blind akzeptierte Theorien als absolute Wahrheit zu betrachten, sollten wir die
Widersprüche hinterfragen und nach konsistenteren Erklärungen suchen.

\subsection{Dogmatismus und blinde Flecken der modernen Physik}
Die Probleme gehen noch tiefer. Die heutige Physik leidet unter einem gewissen Dogmatismus – Theorien wie die \gls{art} oder die Quantenfeldtheorie werden oft unhinterfragt
als absolute Wahrheit akzeptiert, obwohl sie fundamentale Schwächen aufweisen. Der Peer-Review-Prozess, der eigentlich Qualität sichern soll, fungiert oft als Filtermechanismus,
der unorthodoxe, aber möglicherweise richtige Ansätze aussortiert, während etablierte, aber fragwürdige Modelle weiterhin dominieren.
\begin{quote}
    \enquote{Theoretical physics has become stuck in a paradigm that values mathematical elegance over empirical testability, leading to a stagnation of genuine progress.} \cite{Smolin2006}
\end{quote}
Ein Beispiel dafür sind die Singularitäten in der \gls{art}. Wenn die Gleichungen an einem Punkt völlig versagen, warum sollte die Theorie in ihrer unmittelbaren Umgebung noch gültig sein?
Trotzdem werden Schwarze Löcher und Urknall als Tatsachen behandelt, obwohl die zugrundeliegende Mathematik dort zusammenbricht. Ähnlich verhält es sich mit den Standardmodellen der
Teilchenphysik und Kosmologie: Sie funktionieren in begrenzten Rahmen, doch sobald man sie extrapoliert, ergeben sich unsinnige Konsequenzen – unendliche Energien, kausalitätsverletzende
Wurmlöcher oder ein unendliches Multiversum mit parallelen Realitäten.

\subsection{Spekulation statt Fortschritt}
Besonders auffällig ist der Stillstand der theoretischen Physik seit etwa 100 Jahren. Nach den revolutionären Durchbrüchen der Quantenmechanik und Relativitätstheorie zu Beginn des
20. Jahrhunderts gibt es kaum noch echte Neuerungen. Stattdessen dominieren hochspekulative Ideen wie Zeitreisen, Wurmlöcher oder höhere Dimensionen – Konzepte, die mehr mit
Science-Fiction als mit empirischer Wissenschaft zu tun haben.

Es ist an der Zeit, die Grundlagen kritisch zu hinterfragen. Anstatt immer kompliziertere Modelle auf wackeligen Annahmen aufzubauen, sollte die Physik zurück zu einer streng logischen,
nachvollziehbaren Methodik finden. Nur so kann sie wieder zu einer echten Wissenschaft werden, die die Natur erklärt – statt sie mit mathematischen Abstraktionen zu verschleiern.

\subsection{Alternative Theorien}
Ein zentrales Problem der modernen Physik liegt in ihrem übermäßigen Vertrauen in die Mathematik. Nur weil etwas mathematisch formulierbar ist,
muss es noch lange nicht der physikalischen Realität entsprechen. Doch statt diese Grenzen anzuerkennen, werden grundlegende Prinzipien der klassischen Physik – wie Energieerhaltung
oder die Gesetze der Thermodynamik – zugunsten abstrakter Gleichungen aufgegeben. Die \gls{art} beispielsweise postuliert eine dynamische Raumzeit, die scheinbar Energie aus dem Nichts
erzeugen oder vernichten kann. Wo bleibt da die strenge Bilanz der Physik?

Konkrete Widersprüche zeigen sich in der Praxis: Nach der \gls{art} müssten Planeten durch die Abstrahlung von Gravitationswellen Energie verlieren – doch warum sind Planetenbahnen dann über
Milliarden Jahre stabil? Wenn die Raumzeit als elastisches Gebilde beschrieben wird, das sich verformen und bewegen lässt: Welche Kraft verrichtet hier Arbeit, und woher kommt die Energie
dafür? Die Standarderklärungen bleiben vage oder weichen auf mathematische Tricks aus.

Auch die vermeintlichen Beweise für den Urknall sind keineswegs so eindeutig, wie oft behauptet wird. Die kosmische \gls{cmb} wird automatisch als Echo des Urknalls
interpretiert – doch es gibt alternative Erklärungen, etwa thermische Gleichgewichtsprozesse oder Streuphänomene.
\begin{quote}
    \enquote{The interpretation of cosmic microwave background as proof of the Big Bang ignores alternative explanations, such as intrinsic redshifts in plasma cosmology.} \cite{Arp1998}
\end{quote}
Ebenso könnte die Rotverschiebung von Galaxien nicht nur durch Expansion, sondern auch durch andere Mechanismen verursacht werden. Selbst Phänomene wie die Lichtablenkung oder der
Shapiro-Effekt lassen sich ohne \gls{art} erklären, wenn man alternative Gravitationsmodelle zulässt.
\begin{quote}
    \enquote{Weber's formulation of electrodynamics provides a consistent framework for gravitational phenomena without invoking curved spacetime.} \cite{WeberElectrodynamics}
\end{quote}
In diesem Buch sollen solche alternativen Erklärungen aufgezeigt werden. Die Physik darf nicht bei mathematischen Dogmen stehen bleiben – sie muss sich wieder auf Logik, Experiment und
echte Kausalität besinnen.

\section{Abweichende Perspektiven in der Physik: Licht, Relativität und alternative Modelle}
\subsection{Feynmans Teilchenmodell des Lichts}
Richard Feynman vertrat die Auffassung, dass alle Lichtphänomene – einschließlich Interferenz und Beugung – ausschließlich durch Photonen als Teilchen erklärt werden können.
Dies wirft die Frage auf: Brauchen wir überhaupt ein Welle-Teilchen-Dualismus, oder ist die Wellennatur nur ein statistischer Effekt der Quantenmechanik?

\subsection{Widersprüche in der QED: Überlichtschnelle Photonen und Pfadintegrale}
In der \gls{qed} werden Photonen im Pfadintegral-Formalismus über alle möglichen Wege summiert – einschließlich solcher, die scheinbar mit Überlichtgeschwindigkeit verlaufen.
Mathematisch mittelt sich dies zwar zu korrekten Ergebnissen, doch physikalisch stellt sich die Frage:

\begin{itemize}
    \item Wenn Photonen virtuell schneller als Licht sein können, widerspricht dies nicht der \gls{srt}?
    \item Ist die Lichtgeschwindigkeit wirklich eine absolute Grenze, oder nur ein makroskopischer Effekt?
\end{itemize}

\subsection{Energieabhängige Lichtgeschwindigkeit? Experimentelle Hinweise}
Einige alternative Theorien (z. B. Schleifenquantengravitation oder VSL-Modelle) schlagen vor, dass die Lichtgeschwindigkeit von der Photonenenergie abhängen könnte.
Mögliche Indizien:

\begin{itemize}
    \item Gammablitze mit extrem hohen Energien zeigen minimale Laufzeitunterschiede (z. B. Fermi-Teleskop-Daten).
    \item Quantengravitationseffekte könnten bei hohen Energien zu Dispersion führen.
\end{itemize}

\begin{quote}
    \enquote{The constancy of the speed of light is not an immutable law but a parameter that may vary under extreme conditions, offering solutions to cosmological puzzles.} \cite{Magueijo2003}
\end{quote}

\section{Die Entwicklung des Wellenkonzepts in der Physik}
Das Verständnis von Wellen in der Physik hat sich im Laufe der Zeit radikal gewandelt. Während klassische Wellen wie Schall oder Wasserwellen als Störungen eines materiellen Mediums
beschrieben werden konnten, führten elektromagnetische Wellen und Quantenphänomene zu grundlegenden Umbrüchen. Maxwell zeigte 1865, dass Licht sich als elektromagnetische Welle
auch ohne Äther ausbreitet – was die Frage aufwarf, wie Energie ohne Trägermedium transportiert wird. Die \gls{srt} etablierte die Lichtgeschwindigkeit
als absolute Grenze, während die \gls{art} sie als lokal variabel beschreibt – ein scheinbarer Widerspruch, den alternative Theorien wie die Weber-Elektrodynamik
zu lösen versuchen.

Die Quantenphysik revolutionierte das Wellenkonzept weiter: De Broglie verband Teilchen- und Welleneigenschaften, und die \gls{qed} beschreibt Photonen als Felder mit überlichtschnellen
Pfadintegral-Komponenten. Doch diese mathematische Eleganz wirft physikalische Deutungsprobleme auf – etwa die Rolle des Beobachters beim Kollaps der Wellenfunktion oder die nicht-lokale
Natur der Quantenverschränkung. Auch Gravitationswellen in der \gls{art} bleiben rätselhaft: Wenn Raumzeit als schwingendes Medium gilt, woher stammt die Energie für ihre Verformung?

Diese Widersprüche zeigen, dass die etablierten Theorien möglicherweise nur Annäherungen an eine tiefere Wahrheit sind. Die Physik steht vor grundlegenden Fragen: Ist die Lichtgeschwindigkeit
wirklich konstant? Wie vereint man Quantenphysik und Relativität? Gibt es eine objektive Realität jenseits des Beobachters? Die Suche nach Antworten könnte eine neue wissenschaftliche
Revolution auslösen – eine, die unser Verständnis von Wellen und der fundamentalen Natur der Wirklichkeit grundlegend verändert.

\section{Wellenphänomene: Die Dualität von instantaner Ganzheit und lokaler Ausbreitung}
Wellen besitzen eine einzigartige Doppelnatur, die sich durch die gesamte Physik zieht. Einerseits zeigen sie lokale Ausbreitungsphänomene, andererseits weisen sie instantane globale
Eigenschaften auf, die sich jeder klassischen Kausalität entziehen. Diese Dualität wird besonders deutlich, wenn wir die fundamentalen Wechselwirkungen betrachten.

Die newtonsche Mechanik postuliert mit \enquote{actio = reactio} eine instantane Fernwirkung - eine Kraft wirkt unverzüglich zwischen zwei Körpern. Mathematisch ausgedrückt:
\[
    \vec{F}{12} = -\vec{F}{21}
\]
Diese Gleichung beschreibt eine momentane Wechselwirkung ohne zeitliche Verzögerung. Ähnlich verhält es sich im Coulombschen Gesetz:
\[
    \vec{F} = \frac{1}{4\pi\epsilon_0}\frac{q_1q_2}{r^2}\hat{r}
\]
Diese Fernwirkungstheorien funktionieren in ihrem Gültigkeitsbereich bemerkenswert gut, wie die erfolgreiche Beschreibung planetarer Bewegungen zeigt. Sie bleiben aber unvollständig,
da sie den Energie- und Impulstransport zwischen den wechselwirkenden Körpern nicht erklären können.

Interferenzphänomene offenbaren eine weitere tiefgreifende Eigenschaft von Wellen. Betrachten wir das Doppelspaltexperiment: Die Wahrscheinlichkeitsdichte an einem Punkt x auf dem
Schirm ergibt sich aus der Überlagerung der Teilwellen:
\[
    |\Psi(x)|^2 = |\psi_1(x) + \psi_2(x)|^2
\]
Dieses Muster erfüllt einen energetischen Zweck - es minimiert die Gesamtenergie des Systems. Die Welle \enquote{weiß} instantan, wie sie sich verteilen muss, um dieses Minimum zu erreichen,
ohne dass eine lokale Wechselwirkung dies erklären könnte.

Die Weber-Elektrodynamik bietet hier einen interessanten Brückenschlag. Sie erweitert das Coulombsche Gesetz um geschwindigkeits- und beschleunigungsabhängige Terme:
\begin{equation}
    \label{eq:weber_em_skalar}
    \vec{F} = \frac{q_1q_2}{4\pi\epsilon_0r^2}\left[1 - \frac{\dot{r}^2}{c^2} + \frac{2r\ddot{r}}{c^2}\right]\hat{r}
\end{equation}
Diese Gleichung beschreibt:
\begin{enumerate}
    \item Den statischen Coulomb-Term (instantane Fernwirkung)
    \item Einen geschwindigkeitsabhängigen Term (magnetische Effekte)
    \item Einen beschleunigungsabhängigen Term (Strahlungswiderstand)
\end{enumerate}
Die instantane Komponente bleibt erhalten, wird aber durch retardierte Effekte ergänzt. Dies zeigt, wie eine Theorie sowohl momentane Wechselwirkungen als auch Ausbreitungseffekte vereinen kann.
Diese Form soll als \textbf{\enquote{skalare Form}} bezeichnet werden.

Die energetische Betrachtung offenbart den tieferen Sinn dieser Doppelnatur. Eine Welle im Gleichgewichtszustand minimiert stets die Gesamtenergie:
\[
    \delta \int \left[\frac{\hbar^2}{2m}|\nabla\Psi|^2 + V|\Psi|^2\right] d^3x = 0
\]
Diese Bedingung wird global instantan erfüllt, während sich lokale Störungen mit endlicher Geschwindigkeit ausbreiten. Die Weber-Elektrodynamik zeigt, dass ähnliche Prinzipien auch in der
klassischen Physik wirksam sind - die instantane Komponente sorgt für die Energieerhaltung, während die retardierten Terme den Energietransport beschreiben.

Die Konsequenzen dieser Betrachtung sind weitreichend:
\begin{enumerate}
    \item Instantane Effekte sind nicht unbedingt unphysikalisch, sondern können energetische Zwangsbedingungen darstellen
    \item Die Ausbreitungsgeschwindigkeit beschreibt nur den Energietransport, nicht die globale Struktur
    \item Fernwirkungstheorien enthalten einen wahren Kern, der in modernen Theorien oft vernachlässigt wird
\end{enumerate}
Diese Erkenntnisse ebnen den Weg für ein neues Verständnis von Wellenphänomenen, das die scheinbaren Widersprüche zwischen instantaner Ganzheit und lokaler Kausalität auflösen könnte.

\section{Das erweiterte Kausalitätskonzept: Eine synthetische Betrachtung von instantaner Ganzheit und lokaler Dynamik}
Die herkömmliche Vorstellung von Kausalität als linearer Ursache-Wirkungs-Kette mit strenger Lokalität und endlicher Ausbreitungsgeschwindigkeit erweist sich bei genauer Betrachtung
der Wellenphänomene als zu eng gefasst. Die Physik steht vor dem Paradoxon, dass einerseits die Relativitätstheorie eine maximale Signalgeschwindigkeit postuliert, während andererseits
Quantenphänomene wie Verschränkung und das \gls{epr} \cite{EPR1935} nahelegen, dass bestimmte Korrelationen instantan über beliebige Distanzen hinweg bestehen können. Dieses Spannungsfeld verlangt
nach einem neuen, umfassenderen Kausalitätsbegriff.

Der Schlüssel zum Verständnis liegt in der Anerkennung zweier komplementärer, aber gleichberechtigter Kausalitätsebenen, die gemeinsam die Dynamik physikalischer Systeme bestimmen.
Auf der einen Seite steht die lokale Kausalität, wie sie durch die Maxwellschen Gleichungen oder die relativistische Feldtheorie beschrieben wird. Diese Ebene regelt den Energietransport
und die Ausbreitung von Störungen im Raum mit endlicher Geschwindigkeit. Die bekannte Lichtkegel-Struktur der Raumzeit mit ihrer strikten Trennung von zeitartigen, lichtartigen und
raumartigen Abständen gehört in diesen Bereich.

Parallel dazu existiert eine systemische Kausalitätsebene, die für die globale Organisation des Wellenfeldes verantwortlich ist. Diese manifestiert sich in Phänomenen wie der
spontanen Symmetriebrechung, dem Aharonov-Bohm-Effekt oder den bereits erwähnten verschränkten Quantenzuständen. Während die lokale Kausalität durch Differentialgleichungen mit
Randbedingungen beschrieben wird, folgt die systemische Kausalität einem Variationsprinzip, das das gesamte System simultan optimiert.
Das Quantenpotential \cite{bohm1952}
\begin{equation}
    \label{eq:bohm_potenzial}
    Q(\mathbf{r},t) = -\frac{\hbar^2}{2m} \frac{\nabla^2 \sqrt{\rho(\mathbf{r},t)}}{\sqrt{\rho(\mathbf{r},t)}}
\end{equation}
ist hierfür ein ausgezeichnetes Beispiel - es wirkt nicht durch lokale Wechselwirkungen, sondern durch die instantane Anpassung der gesamten
Wellenfunktion an die globalen Randbedingungen.

Die Weber-Elektrodynamik mit ihrer charakteristischen Kraftgleichung (\refeq{eq:weber_em_skalar}) zeigt exemplarisch, wie beide Kausalitätsebenen in einer konsistenten Theorie vereint werden können.
Der erste Term repräsentiert die systemische Komponente - eine instantane Coulomb-Wechselwirkung, die für die grundlegende Struktur des Feldes sorgt. Die zusätzlichen Geschwindigkeits- und
Beschleunigungsterme hingegen beschreiben die lokale Dynamik des Energietransports, einschließlich retardierter Effekte und Strahlungsphänomene.

Diese duale Struktur der Kausalität löst zahlreiche konzeptionelle Probleme der modernen Physik. So erklärt sie beispielsweise, warum im Doppelspaltexperiment das Interferenzmuster
auch dann entsteht, wenn die Teilchen nacheinander das Experiment durchlaufen - die systemische Kausalität \enquote{kennt} die Gesamtanordnung und organisiert die Wahrscheinlichkeitsverteilung
entsprechend. Gleichzeitig bleibt der Energietransport durch die lokale Kausalität begrenzt, was Relativitätsprinzipien nicht verletzt.

Die Konsequenzen dieses erweiterten Kausalitätsverständnisses sind tiefgreifend. Es ermöglicht eine physikalische Interpretation der Quantenmechanik, die ohne den problematischen
\enquote{Kollaps} der Wellenfunktion auskommt. Messprozesse erscheinen nicht mehr als mysteriöse Eingriffe in das System, sondern als besondere Fälle der systemischen Selbstorganisation.
Die scheinbare Beobachterabhängigkeit quantenmechanischer Phänomene erweist sich als Spezialfall der allgemeinen systemischen Kausalität, die immer dann besonders augenfällig wird, wenn
ein Teilsystem (der \enquote{Beobachter}) mit einem anderen (dem \enquote{beobachteten System}) korreliert.

Letztlich führt dieser Ansatz zu einer natürlichen Synthese von klassischer und Quantenphysik, von Relativitätstheorie und Wellenmechanik. Anstatt die paradoxen Aspekte der Quantentheorie
als grundlegende Prinzipien zu akzeptieren, erklärt sie sie als Konsequenz des Zusammenspiels zweier komplementärer Kausalitätsebenen - einer systemischen Ganzheit, die instantan wirkt,
und einer lokalen Dynamik, die den Energie- und Impulstransport mit endlicher Geschwindigkeit vermittelt.
