\chapter{Einleitung}
\section{Motivation}
Viele Schüler und Studierende erleben den Physikunterricht als frustrierend und unverständlich. Besonders die moderne Physik – mit der \gls{art}
und der \gls{srt} – wirkt oft unphysikalisch und voller logischer Widersprüche. Energie scheint unter bestimmten Bedingungen unendlich zu werden,
Überlichtgeschwindigkeit wird in manchen Fällen postuliert, obwohl sie eigentlich unmöglich sein soll, und Begriffe wie \enquote{dunkle Energie} oder \enquote{dunkle Materie} wirken wie
Platzhalter für unser Unverständnis.

Ein grundlegendes Problem liegt in den Widersprüchen zwischen \gls{art} und \gls{srt}. Die \gls{srt} baut auf Inertialsystemen auf, also Bezugssystemen, die sich gleichförmig und unbeschleunigt
bewegen. Doch laut \gls{art} gibt es keine perfekten Inertialsysteme, da jede Masse die Raumzeit krümmt und damit Beschleunigungen erzeugt. Schon allein dieser Widerspruch wirft
Fragen auf: Wenn Inertialsysteme streng genommen punktförmig sein müssten, um frei von jeder Krümmung zu sein, bräuchte man unendlich viele davon – und damit auch unendlich
viele verschiedene Lichtgeschwindigkeiten, da diese vom Bezugssystem abhängt.

Hinzu kommt, dass viele Konzepte der modernen Physik unserer Intuition widersprechen. Die Quantenmechanik verlangt, dass Teilchen gleichzeitig Wellen sind und erst durch
Beobachtung einen definierten Zustand annehmen. Die \gls{art} beschreibt eine gekrümmte Raumzeit, die sich kaum jemand wirklich vorstellen kann, und die \gls{srt} führt zu scheinbar
paradoxen Zeitdehnungen und Längenkontraktionen. Selbst der Urknall als Anfangspunkt des Universums wirft Fragen auf: Wie kann etwas aus dem Nichts entstehen? Warum gibt es
überhaupt eine Singularität, wenn doch unsere physikalischen Gesetze dort versagen?

All diese Punkte zeigen, dass die moderne Physik noch lange nicht abgeschlossen ist. Statt blind akzeptierte Theorien als absolute Wahrheit zu betrachten, sollten wir die
Widersprüche hinterfragen und nach konsistenteren Erklärungen suchen.

\subsection{Dogmatismus und blinde Flecken der modernen Physik}
Doch die Probleme gehen noch tiefer. Die heutige Physik leidet unter einem gewissen Dogmatismus – Theorien wie die \gls{art} oder die Quantenfeldtheorie werden oft unhinterfragt
als absolute Wahrheit akzeptiert, obwohl sie fundamentale Schwächen aufweisen. Der Peer-Review-Prozess, der eigentlich Qualität sichern soll, fungiert oft als Filtermechanismus,
der unorthodoxe, aber möglicherweise richtige Ansätze aussortiert, während etablierte, aber fragwürdige Modelle weiterhin dominieren.

Ein Beispiel dafür sind die Singularitäten in der \gls{art}. Wenn die Gleichungen an einem Punkt völlig versagen, warum sollte die Theorie in ihrer unmittelbaren Umgebung noch gültig sein?
Trotzdem werden Schwarze Löcher und Urknall als Tatsachen behandelt, obwohl die zugrundeliegende Mathematik dort zusammenbricht. Ähnlich verhält es sich mit den Standardmodellen der
Teilchenphysik und Kosmologie: Sie funktionieren in begrenzten Rahmen, doch sobald man sie extrapoliert, ergeben sich unsinnige Konsequenzen – unendliche Energien, kausalitätsverletzende
Wurmlöcher oder ein unendliches Multiversum mit parallelen Realitäten.

\subsection{Spekulation statt Fortschritt}
Besonders auffällig ist der Stillstand der theoretischen Physik seit etwa 100 Jahren. Nach den revolutionären Durchbrüchen der Quantenmechanik und Relativitätstheorie zu Beginn des
20. Jahrhunderts gibt es kaum noch echte Neuerungen. Stattdessen dominieren hochspekulative Ideen wie Zeitreisen, Wurmlöcher oder höhere Dimensionen – Konzepte, die mehr mit
Science-Fiction als mit empirischer Wissenschaft zu tun haben.

Es ist an der Zeit, die Grundlagen kritisch zu hinterfragen. Anstatt immer kompliziertere Modelle auf wackeligen Annahmen aufzubauen, sollte die Physik zurück zu einer streng logischen,
nachvollziehbaren Methodik finden. Nur so kann sie wieder zu einer echten Wissenschaft werden, die die Natur erklärt – statt sie mit mathematischen Abstraktionen zu verschleiern.

\subsection{Alternative Theorien}
Ein zentrales Problem der modernen Physik \cite{Smolin2006} liegt in ihrem übermäßigen Vertrauen in die Mathematik \cite{Hossenfelder2018}. Nur weil etwas mathematisch formulierbar ist,
muss es noch lange nicht der physikalischen Realität entsprechen. Doch statt diese Grenzen anzuerkennen, werden grundlegende Prinzipien der klassischen Physik – wie Energieerhaltung
oder die Gesetze der Thermodynamik – zugunsten abstrakter Gleichungen aufgegeben. Die \gls{art} beispielsweise postuliert eine dynamische Raumzeit, die scheinbar Energie aus dem Nichts
erzeugen oder vernichten kann. Wo bleibt da die strenge Bilanz der Physik?

Konkrete Widersprüche zeigen sich in der Praxis: Nach der \gls{art} müssten Planeten durch die Abstrahlung von Gravitationswellen Energie verlieren – doch warum sind Planetenbahnen dann über
Milliarden Jahre stabil? Wenn die Raumzeit als elastisches Gebilde beschrieben wird, das sich verformen und bewegen lässt: Welche Kraft verrichtet hier Arbeit, und woher kommt die Energie
dafür? Die Standarderklärungen bleiben vage oder weichen auf mathematische Tricks aus.

Auch die vermeintlichen Beweise für den Urknall sind keineswegs so eindeutig, wie oft behauptet wird. Die kosmische \gls{cmb} wird automatisch als Echo des Urknalls
interpretiert – doch es gibt alternative Erklärungen, etwa thermische Gleichgewichtsprozesse oder Streuphänomene. Ebenso könnte die Rotverschiebung von Galaxien nicht nur durch Expansion,
sondern auch durch andere Mechanismen verursacht werden \cite{Arp1998, TiredLight}. Selbst Phänomene wie die Lichtablenkung oder der Shapiro-Effekt lassen sich ohne \gls{art} erklären, wenn man alternative
Gravitationsmodelle zulässt.

In diesem Buch sollen solche alternativen Erklärungen aufgezeigt werden. Die Physik darf nicht bei mathematischen Dogmen stehen bleiben – sie muss sich wieder auf Logik, Experiment und
echte Kausalität besinnen.

\newpage
\section{Abweichende Perspektiven in der Physik: Licht, Relativität und alternative Modelle}
\subsection{Feynmans Teilchenmodell des Lichts}
Richard Feynman vertrat die Auffassung, dass alle Lichtphänomene – einschließlich Interferenz und Beugung – ausschließlich durch Photonen als Teilchen erklärt werden können.
In \cite{Feynman1963} argumentiert er:

\enquote{Ich möchte betonen, dass Licht aus Teilchen besteht – zumindest genauso sehr, wie es aus Wellen besteht. […] Selbst Interferenzmuster lassen sich durch die Wahrscheinlichkeitsverteilung
einzelner Photonen beschreiben.}

Dies wirft die Frage auf: Brauchen wir überhaupt ein Welle-Teilchen-Dualismus, oder ist die Wellennatur nur ein statistischer Effekt der Quantenmechanik?

\subsection{Widersprüche in der QED: Überlichtschnelle Photonen und Pfadintegrale}
In der \gls{qed} werden Photonen im Pfadintegral-Formalismus über alle möglichen Wege summiert – einschließlich solcher, die scheinbar mit Überlichtgeschwindigkeit verlaufen \cite{FeynmanQED}.
Mathematisch mittelt sich dies zwar zu korrekten Ergebnissen, doch physikalisch stellt sich die Frage \cite{Cramer1986}:

\begin{itemize}
    \item Wenn Photonen virtuell schneller als Licht sein können, widerspricht dies nicht der \gls{srt}?
    \item Ist die Lichtgeschwindigkeit wirklich eine absolute Grenze, oder nur ein makroskopischer Effekt?
\end{itemize}

\subsection{Widersprüche zwischen ART und SRT: Variable vs. absolute Lichtgeschwindigkeit}
Die \gls{art} verwendet eine effektive Lichtgeschwindigkeit \cite{MisnerThorneWheeler}, die in gekrümmter Raumzeit lokal variieren kann (z. B. in der Nähe von Schwarzen Löchern).
Die \gls{srt} hingegen postuliert eine strikt konstante Lichtgeschwindigkeit – doch selbst hier gibt es Probleme:

\begin{itemize}
    \item Wenn Inertialsysteme punktförmig sein müssen \cite{Einstein1905} (um keine Krümmung zu spüren), dann gibt es unendlich viele Bezugssysteme mit jeweils leicht unterschiedlicher
    Lichtgeschwindigkeit.
    \item Das Zwillingsparadoxon zeigt, dass Zeitdilatation reell ist – aber wenn die Lichtgeschwindigkeit absolut ist, warum hängt sie dann vom Bewegungszustand des
    Beobachters ab?
\end{itemize}

\subsection{Energieabhängige Lichtgeschwindigkeit? Experimentelle Hinweise}
Einige alternative Theorien (z. B. Schleifenquantengravitation oder VSL-Modelle \cite{AmelinoCamelia1998}) schlagen vor, dass die Lichtgeschwindigkeit von der Photonenenergie abhängen könnte.
Mögliche Indizien:

\begin{itemize}
    \item Gammablitze mit extrem hohen Energien zeigen minimale Laufzeitunterschiede \cite{Fermi2015} (z. B. Fermi-Teleskop-Daten).
    \item Quantengravitationseffekte könnten bei hohen Energien zu Dispersion führen.
\end{itemize}

\subsection{Brauchen wir eine neue Lichttheorie?}
Die etablierte Physik beharrt auf der konstanten Lichtgeschwindigkeit und dem Welle-Teilchen-Dualismus – doch die Widersprüche in \gls{qed}, \gls{art} und \gls{srt} legen nahe,
dass eine fundamentalere Beschreibung möglich ist. In diesem Buch werden alternative Ansätze untersucht, die ohne dogmatische Annahmen auskommen.

\section{Wellen in der Physik – Vom klassischen Modell zur quantenmechanischen Revolution}
Wellenphänomene durchziehen die gesamte Physik wie ein roter Faden, doch ihr Verständnis hat sich im Laufe der Zeit grundlegend gewandelt. Während klassische Wellen wie Schall
oder Wasserwellen als lokale Störungen eines materiellen Trägermediums beschrieben werden können, stellen uns elektromagnetische Wellen und Quantenphänomene vor völlig neue
Herausforderungen. James Clerk Maxwell zeigte 1865 \cite{Maxwell1865}, dass Licht als elektromagnetische Welle beschrieben werden kann, die sich auch ohne Medium mit konstanter
Geschwindigkeit ausbreitet. Diese Erkenntnis warf fundamentale Fragen auf: Wie wird Energie transportiert? Was ist das Wesen des Raums, wenn kein Äther mehr als Trägermedium benötigt wird?

Die \gls{srt} machte die Lichtgeschwindigkeit zur absoluten Obergrenze für Kausalität \cite{Einstein1905_SRT}, während die \gls{art} eine flexible Raumzeit einführte, in der die
Lichtgeschwindigkeit lokal variieren kann \cite{MisnerThorneWheeler}. Hier zeigen sich bereits Widersprüche - wie kann die Lichtgeschwindigkeit gleichzeitig absolut und variabel sein?
Alternative Ansätze wie die Weber-Elektrodynamik \cite{Weber1846, WeberElectrodynamics} bieten hier mögliche Auswege, indem sie verzögerte Fernwirkungen ohne die Notwendigkeit eines
Feldes postulieren.

Noch radikalere Umwälzungen brachte die Quantenphysik. Louis de Broglie verband 1924 Teilchen- und Welleneigenschaften \cite{deBroglie1924}, indem er jedem Objekt eine Materiewelle zuordnete.
Die \gls{qed} beschreibt Photonen als Quantenfelder, deren Pfadintegrale sogar überlichtschnelle Komponenten enthalten \cite{FeynmanQED_Pfadintegrale} - ein mathematisch notwendiges,
aber physikalisch schwer zu deutendes Konzept. Während die Kopenhagener Deutung hier nur statistische Aussagen macht, bietet die De-Broglie-Bohm-Theorie \cite{Bohm1952} eine deterministische
Alternative mit Führungswellen, die allerdings nicht-lokale Effekte benötigt.

Gravitationswellen \cite{LIGO2016} in der \gls{art} zeigen ähnliche Paradoxien: Wenn die Raumzeit als elastisches Medium betrachtet wird, das Energie in Form von Wellen transportieren kann, woher kommt
dann die Energie für ihre Verformung? Warum verlieren Planetenbahnen keine messbare Energie durch Gravitationsstrahlung? Alternative Raummodelle versuchen diese Widersprüche aufzulösen,
etwa durch Äther-Konzepte oder skalare Tensor-Theorien.

Die offenen Fragen sind grundlegend: Wie lassen sich Quantenphänomene mit der Relativitätstheorie vereinbaren? Ist die Lichtgeschwindigkeit wirklich universell konstant oder gibt es
energieabhängige Effekte? Wie kann der Welle-Teilchen-Dualismus auf fundamentaler Ebene verstanden werden? Diese Probleme zeigen, dass unser Verständnis von Wellenphänomenen noch lange
nicht abgeschlossen ist. Vielmehr deuten die bestehenden Widersprüche darauf hin, dass die etablierten Theorien möglicherweise nur Annäherungen an eine noch zu entdeckende, tiefere
Wahrheit darstellen.
