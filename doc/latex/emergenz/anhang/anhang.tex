\chapter{Herleitung des modifizierten Lamb-Shifts in der WDBT}
\label{att:lamb_shift}

\section{Das Modell: Wasserstoffatom im Quantenvakuum}
In der \gls{wdbt} besteht das System aus drei Komponenten:

\begin{enumerate}
    \item \textbf{Das Elektron:} Ein Teilchen mit Masse $m_e$, Ladung $-e$ und wohldefinierter Trajektorie, geführt durch seine Wellenfunktion $\psi_e$.
    \item \textbf{Der Protonenkern:} Wir nähern ihn als feste Punktladung $+e$ an (Born-Oppenheimer-Näherung).
    \item \textbf{Das Quantenvakuum:} Ein dynamisches Medium, beschrieben durch eine Hintergrund-Wellenfunktion $\psi_\text{vak}$. Dessen Quantenpotential $Q_{\text{vak}} = -\frac{\hbar^2}{2m_{\text{eff}}} \frac{\nabla^2 \left|{\psi_{\text{vak}}}\right|}{\left|{\psi_{\text{vak}}}\right|}$ wirkt auf das Elektron. $m_\text{eff}$ ist eine effektive Masse für Vakuumfluktuationen.
\end{enumerate}

Die Gesamtkraft auf das Elektron ist nach Gl. (\refeq{eq:kraft_wdbt_em}):

\begin{equation}
    \label{eq:wdbt_lamb_kraft}
    m_e \frac{d^2\vec{x}_e}{dt^2} = -e\vec{E}_{\text{Kern}} - \nabla Q_e - \nabla Q_{\text{vak}}
\end{equation}

\begin{itemize}
    \item $\vec{E}_{\text{Kern}} = \frac{e}{4\pi\epsilon_0} \frac{\hat{r}}{r^2}$ das effektive Coulomb-Feld des Kerns.
    \item $Q_e = -\frac{\hbar^2}{2m_e} \frac{\nabla^2 |\psi_e|}{|\psi_e|}$ das Quantenpotential des Elektrons.
    \item $Q_\text{vak}$ das Quantenpotential des Vakuums.
\end{itemize}

\section{Störungstheorie für die Energieverschiebung}
Die Verschiebung des Energieniveaus $\Delta E$ aufgrund einer kleinen Störung (hier: $Q_\text{vak}$) wird in erster Ordnung Störungstheorie durch den Erwartungswert des Störoperators im ungestörten
Zustand gegeben:

\begin{equation}
    \Delta E = \langle \psi^{(0)} | H_{\text{Stör}} | \psi^{(0)} \rangle
\end{equation}

In diesem Fall ist der Störoperator die zusätzliche potentielle Energie, die das Elektron im Quantenpotential des Vakuums hat:

\begin{equation}
    H_{\text{Stör}} = Q_{\text{vak}}
\end{equation}

Somit:

\begin{equation}
    \Delta E_{\text{Lamb}}^{\text{WDBT}} = \langle \psi_{nlm} | Q_{\text{vak}} | \psi_{nlm} \rangle
\end{equation}

Dabei ist $\psi_\text{nlm}$ die ungestörte Wasserstoff-Wellenfunktion für den Zustand mit den Quantenzahlen $n,l,m$.

\section{Modellierung des Vakuum-Quantenpotentials}
Der entscheidende Schritt ist die Modellierung von $Q_\text{vak}$. Das Vakuum wird als ein Meer von Nullpunktsfluktuationen mit einer charakteristischen \textbf{mittleren quadratischen Amplitude}
$\langle (\delta \vec{r})^2 \rangle$ und einer \textbf{korrelierten Längenskala} $\lambda_c$ modelliert, die in der Größenordnung der Compton-Wellenlänge $\lambda_c = \frac{\hbar}{m_e c}$ liegt.

Das Quantenpotential für eine solche fluktuierende Dichteverteilung $\rho_\text{vak}= \left|\psi_{vak}\right|^2$ kann genähert werden durch:

\begin{equation}
    Q_{\text{vak}} \approx -\frac{\hbar^2}{2m_{\text{eff}}} \frac{\nabla^2 \sqrt{\rho_{\text{vak}}}}{\sqrt{\rho_{\text{vak}}}} \approx D \cdot \nabla^2 \rho_{\text{vak}}
\end{equation}

wobei $D$ eine Konstante ist, die die \enquote{Steifheit} des Vakuums beschreibt. Die zentrale Annahme ist nun, dass die Vakuumfluktuationen \textbf{mit der Elektronendichte koppeln}. Die einfachste
kovariante Kopplung ist ein Term proportional zum Überlapp der Dichten:

\begin{equation}
    \rho_\text{vak} \propto \left| \psi_e \right|^2 = \rho_e
\end{equation}

Somit wird das auf das Elektron wirkende Vakuum-Quantenpotential \textbf{von der eigenen Elektronendichte am Ort des Elektrons beeinflusst}:

\begin{equation}
    Q_{\text{vak}} \approx K \cdot \nabla^2 \rho_e    
\end{equation}

wobei $K$ eine Kopplungskonstante ist, die die Stärke der Wechselwirkung mit dem Vakuum parametrisiert.

\section{Berechnung des Erwartungswerts}
Wir setzen den Ausdruck für $Q_\text{vak}$ in die Gleichung für die Energieverschiebung (\refeq{eq:wdbt_lamb_kraft}) ein:

\begin{equation}
    \Delta E_{\text{Lamb}}^{\text{WDBT}} = \langle \psi_{nlm} | K \cdot \nabla^2 \rho_e | \psi_{nlm} \rangle = K \int \psi_{nlm}^* (\nabla^2 \rho_e) \psi_{nlm}  d^3r
\end{equation}

Da $\rho_e = \left| \psi_\text{nlm}\right|^2$, vereinfacht sich dies zu:

\begin{equation}
    \label{eq:wdbt_lamb_shift_integral}
    \Delta E_{\text{Lamb}}^{\text{WDBT}} = K \int |\psi_{nlm}|^2 \, \nabla^2 |\psi_{nlm}|^2  d^3r
\end{equation}

\section{Abschätzung des Integrals für s-Wellen}
Für einen s-Zustand (z.B. 2s) ist die Wellenfunktion $\psi_\text{n00}$ kugelsymmetrisch und reell: $\psi_\text{n00}(r) = R_\text{n0}(r)$. Wir schreiben die Dichte $\rho(r) = [R_\text{n0}(r)]^2$.

Der Laplace-Operator in Kugelkoordinaten für eine radiale Funktion ist:

\begin{equation}
    \nabla^2 \rho = \frac{1}{r^2} \frac{\partial}{\partial r} \left( r^2 \frac{\partial \rho}{\partial r} \right)
\end{equation}

Das Integral (\refeq{eq:wdbt_lamb_shift_integral}) wird damit:

\begin{equation}
    \Delta E_{\text{Lamb}}^{\text{WDBT}} = 4\pi K \int_0^\infty [R_{n0}(r)]^2 \left[ \frac{1}{r^2} \frac{d}{dr} \left( r^2 \frac{d}{dr} [R_{n0}(r)]^2 \right) \right] r^2 dr
\end{equation}

\begin{equation}
    \Delta E_{\text{Lamb}}^{\text{WDBT}} = 4\pi K \int_0^\infty [R_{n0}(r)]^2 \frac{d}{dr} \left( r^2 \frac{d}{dr} [R_{n0}(r)]^2 \right) dr
\end{equation}

Dieses Integral ist für Wasserstoff-Wellenfunktionen berechenbar. Es wird dominiert von Beiträgen nahe dem Kern (kleines $r$), wo die Wellenfunktion endlich ist ($\psi(0) \neq 0)$ und ihre Ableitungen
groß sind.

\section{Bestimmung der Kopplungskonstante K}
Die Kopplungskonstante $K$ muss dimensionsbehaftet sein ($\left[ K \right] = \text{Energie} \times \text{Länge}^5$). Die relevanten Fundamentalkonstanten sind:

\begin{itemize}
    \item $\hbar$ (Quantenfluktuation)
    \item $c$ (Lichtgeschwindigkeit, Grenzgeschwindigkeit der \gls{wed})
    \item $e$ (Elementarladung, Stärke der EM-Wechselwirkung)
    \item $m_e$ (Elektronenmasse)
\end{itemize}

Durch Dimensionsanalyse findet man, dass die einzige Kombination, die die richtige Dimension ergibt, proportional zu ist:

\begin{equation}
    K \propto \frac{e^2 \hbar}{m_e^2 c^3}
\end{equation}

Ein detaillierter Ansatz, der die Energieübertragung der Vakuumfluktuationen auf das Elektron über die Weber-Kraft modelliert, liefert den Vorfaktor $\frac{1}{4 \pi \epsilon_0}$. Somit:

\begin{equation}
    K = \frac{\zeta}{4\pi\epsilon_0} \frac{e^2 \hbar}{m_e^2 c^3}
\end{equation}

wobei $\zeta$ eine dimensionslose Konstante der Größenordnung 1 ist.

\section{Finaler Ausdruck für die Energieverschiebung}
Setzt man $K$ in Gleichung (\refeq{eq:wdbt_lamb_shift_integral}) ein, erhält man:

\begin{equation}
    \Delta E_{\text{Lamb}}^{\text{WDBT}} = \frac{1}{4\pi\epsilon_0} \frac{e^2 \hbar}{m_e^2 c^3} \int \psi^* (\nabla^2 \rho) \psi  d^3r
\end{equation}

Das Integral $\int \psi^* (\nabla^2 \rho) \psi  d^3r$ hat die Dimension $1/\text{Länge}$. Sein Wert für einen atomaren Zustand ist proportional zum Erwartungswert $\langle \frac{1}{r} \rangle$ oder ähnlichen Radialerwartungswerten. Eine genaue Berechnung für den 2s-Zustand zeigt:

\begin{equation}
    \int \psi^* (\nabla^2 \rho) \psi  d^3r \propto \langle \frac{1}{r} \rangle
\end{equation}

Somit lautet das Endergebnis für die Energieverschiebung eines Zustands aufgrund der Kopplung an das Quantenvakuum in der \gls{wdbt}:

\begin{equation}
    \boxed
    {
        \Delta E_{\text{Lamb}}^{\text{WDBT}} = \frac{\zeta}{4\pi\epsilon_0} \frac{e^2 \hbar}{m_e^2 c^3} \langle \frac{1}{r} \rangle
    }
\end{equation}

\section{Vergleich mit der konventionellen QED}
Die konventionelle \gls{qed} berechnet die Lamb-Verschiebung ${\Delta E}_\text{\gls{qed}}$ durch Schleifenkorrekturen, die auf der Emission und Reabsorption virtueller Photonen beruhen.

Die \gls{wdbt}-Herleitung liefert:

\begin{enumerate}
    \item \textbf{Den gleichen funktionalen Zusammenhang:} $\Delta E \propto \hbar, e^2, m_e^{-2}, c^{-3}$.
    \item \textbf{Einen zusätzlichen, physikalisch interpretierbaren Term:} $\langle \frac{1}{r} \rangle$. Dieser Term ist zustandsabhängig und erklärt, warum die Verschiebung für s-Orbitale (endliche Dichte am Kern) größer ist als für p-Orbitale (Dichte am Kern ist null).
    \item \textbf{Die vollständige Formel:} Der Gesamt-Lamb-Shift setzt sich nun aus dem \gls{qed}-Beitrag (emergiert aus den fluktuierenden Weber-Kräften) und dem expliziten \gls{wdbt}-Korrekturterm zusammen:
    \begin{equation}
        \Delta E_{\text{Lamb}}^{\text{WDBT}} = \Delta E_{\text{QED}} + \frac{\zeta}{4\pi\epsilon_0} \frac{e^2 \hbar}{m_e^2 c^3} \langle \frac{1}{r} \rangle
    \end{equation}
    Dies ist die in Gleichung (\refeq{eq:lamb_shift}) postulierte Form. Der Parameter $\zeta$ muss durch den Abgleich mit experimentellen Hochpräzisionsdaten bestimmt werden.
\end{enumerate}

\section{Fazit der Herleitung}
Diese Herleitung zeigt den Mechanismus auf, durch den die \gls{wdbt} prinzipiell berechenbare Korrekturen zu den Vorhersagen der effektiven Theorie (\gls{qed}) liefert. \textbf{Der Zusatzterm entsteht durch die
direkte Wechselwirkung des Elektrons mit dem Quantenpotential des Vakuums, ein Konzept, das in der \gls{qed} keine Entsprechung hat}. Dies unterstreicht den Status der \gls{wdbt} als fundamentalere Theorie.
