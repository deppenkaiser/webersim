\chapter{Herleitung des modifizierten Lamb-Shifts in der WDBT}
\label{app:lamb_shift}

\section{Das Modell: Wasserstoffatom im Quantenvakuum}
In der \gls{wdbt} besteht das System aus drei Komponenten:

\begin{enumerate}
    \item \textbf{Das Elektron:} Ein Teilchen mit Masse $m_e$, Ladung $-e$ und wohldefinierter Trajektorie, geführt durch seine Wellenfunktion $\psi_e$.
    \item \textbf{Der Protonenkern:} Wir nähern ihn als feste Punktladung $+e$ an (Born-Oppenheimer-Näherung).
    \item \textbf{Das Quantenvakuum:} Ein dynamisches Medium, beschrieben durch eine Hintergrund-Wellenfunktion $\psi_\text{vak}$. Dessen Quantenpotential $Q_{\text{vak}} = -\frac{\hbar^2}{2m_{\text{eff}}} \frac{\nabla^2 \left|{\psi_{\text{vak}}}\right|}{\left|{\psi_{\text{vak}}}\right|}$ wirkt auf das Elektron. $m_\text{eff}$ ist eine effektive Masse für Vakuumfluktuationen.
\end{enumerate}

Die Gesamtkraft auf das Elektron ist nach Gl. (5.1):
