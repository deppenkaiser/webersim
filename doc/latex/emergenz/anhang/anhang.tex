\chapter{Herleitung des modifizierten Lamb-Shifts in der WDBT}
\label{att:lamb_shift}

\section{Das Modell: Wasserstoffatom im Quantenvakuum}
In der \gls{wdbt} besteht das System aus drei Komponenten:

\begin{enumerate}
    \item \textbf{Das Elektron:} Ein Teilchen mit Masse $m_e$, Ladung $-e$ und wohldefinierter Trajektorie, geführt durch seine Wellenfunktion $\psi_e$.
    \item \textbf{Der Protonenkern:} Wir nähern ihn als feste Punktladung $+e$ an (Born-Oppenheimer-Näherung).
    \item \textbf{Das Quantenvakuum:} Ein dynamisches Medium, beschrieben durch eine Hintergrund-Wellenfunktion $\psi_\text{vak}$. Dessen Quantenpotential $Q_{\text{vak}} = -\frac{\hbar^2}{2m_{\text{eff}}} \frac{\nabla^2 \left|{\psi_{\text{vak}}}\right|}{\left|{\psi_{\text{vak}}}\right|}$ wirkt auf das Elektron. $m_\text{eff}$ ist eine effektive Masse für Vakuumfluktuationen.
\end{enumerate}

Die Gesamtkraft auf das Elektron ist nach Gl. (\refeq{eq:kraft_wdbt_em}):

\begin{equation}
    m_e \frac{d^2\vec{x}_e}{dt^2} = -e\vec{E}_{\text{Kern}} - \nabla Q_e - \nabla Q_{\text{vak}}
\end{equation}

\begin{itemize}
    \item $\vec{E}_{\text{Kern}} = \frac{e}{4\pi\epsilon_0} \frac{\hat{r}}{r^2}$ das effektive Coulomb-Feld des Kerns.
    \item $Q_e = -\frac{\hbar^2}{2m_e} \frac{\nabla^2 |\psi_e|}{|\psi_e|}$ das Quantenpotential des Elektrons.
    \item $Q_\text{vak}$ das Quantenpotential des Vakuums.
\end{itemize}

\section{Störungstheorie für die Energieverschiebung}
Die Verschiebung des Energieniveaus $\Delta E$ aufgrund einer kleinen Störung (hier: $Q_\text{vak}$) wird in erster Ordnung Störungstheorie durch den Erwartungswert des Störoperators im ungestörten
Zustand gegeben:

\begin{equation}
    \Delta E = \langle \psi^{(0)} | H_{\text{Stör}} | \psi^{(0)} \rangle
\end{equation}

In diesem Fall ist der Störoperator die zusätzliche potentielle Energie, die das Elektron im Quantenpotential des Vakuums hat:

\begin{equation}
    H_{\text{Stör}} = Q_{\text{vak}}
\end{equation}

Somit:

\begin{equation}
    \Delta E_{\text{Lamb}}^{\text{WDBT}} = \langle \psi_{nlm} | Q_{\text{vak}} | \psi_{nlm} \rangle
\end{equation}

Dabei ist $\psi_\text{nlm}$ die ungestörte Wasserstoff-Wellenfunktion für den Zustand mit den Quantenzahlen $n,l,m$.

\section{Modellierung des Vakuum-Quantenpotentials}
Der entscheidende Schritt ist die Modellierung von $Q_\text{vak}$. Das Vakuum wird als ein Meer von Nullpunktsfluktuationen mit einer charakteristischen \textbf{mittleren quadratischen Amplitude}
$\langle (\delta \vec{r})^2 \rangle$ und einer \textbf{korrelierten Längenskala} $\lambda_c$ modelliert, die in der Größenordnung der Compton-Wellenlänge $\lambda_c = \frac{\hbar}{m_e c}$ liegt.

Das Quantenpotential für eine solche fluktuierende Dichteverteilung $\rho_\text{vak}= \left|\psi_{vak}\right|^2$ kann genähert werden durch:

\begin{equation}
    Q_{\text{vak}} \approx -\frac{\hbar^2}{2m_{\text{eff}}} \frac{\nabla^2 \sqrt{\rho_{\text{vak}}}}{\sqrt{\rho_{\text{vak}}}} \approx D \cdot \nabla^2 \rho_{\text{vak}}
\end{equation}

wobei $D$ eine Konstante ist, die die \enquote{Steifheit} des Vakuums beschreibt. Die zentrale Annahme ist nun, dass die Vakuumfluktuationen \textbf{mit der Elektronendichte koppeln}. Die einfachste
kovariante Kopplung ist ein Term proportional zum Überlapp der Dichten:

\begin{equation}
    \rho_\text{vak} \propto \left| \psi_e \right|^2 = \rho_e
\end{equation}

Somit wird das auf das Elektron wirkende Vakuum-Quantenpotential \textbf{von der eigenen Elektronendichte am Ort des Elektrons beeinflusst}:

\begin{equation}
    Q_{\text{vak}} \approx K \cdot \nabla^2 \rho_e    
\end{equation}

wobei $K$ eine Kopplungskonstante ist, die die Stärke der Wechselwirkung mit dem Vakuum parametrisiert.

\section{Berechnung des Erwartungswerts}
