\chapter{Die konvergente Emergenz der ART}
 Wie die \gls{art} zur \gls{wdbt} strebt.

\section{Die unvollständige ART: Das Singularitäten-Problem}

\begin{itemize}
    \item \textbf{Standard-\gls{art}:} $G_{\mu\nu} = 8\pi G T_{\mu\nu}$
    \item Diese Gleichung führt unter generischen Bedingungen zu Singularitäten (Schwarze Löcher, Urknall).
    \item \textbf{Interpretation:} Dies ist kein physikalisches, sondern ein theoretisches Versagen. Die \gls{art} ist an ihren Grenzen unvollständig.
\end{itemize}

\section{Schritt 1 der Emergenz: Vervollständigung der ART durch die DBT}

\begin{itemize}
    \item Die naheliegendste Erweiterung zur Vermeidung von Singularitäten ist die Einführung des \textbf{Bohm'schen Quantenpotentials $Q$}.
    \item Die vervollständigte Einstein-Gleichung lauten nun:
    \begin{equation}
        G_{\mu\nu} = 8\pi G (T_{\mu\nu} + Q_{\mu\nu})
    \end{equation}
    \item \textbf{Konsequenzen dieser Erweiterung:}
    \begin{enumerate}
        \item \textbf{Singularitätenfreiheit:} $Q$ wirkt repulsiv und verhindert die Bildung von Punkt-Singularitären. Resultat: Big Bounce statt Big Bang.
        \item \textbf{Nicht-Lokalität:} Das Quantenpotential $Q$ ist fundamental nicht-lokal. Diese Eigenschaft wird nun in die Gravitation selbst eingebracht.
        \item \textbf{Angleich an die WDBT:} Die erweiterte \gls{art} gewinnt zentrale Eigenschaften der \gls{wdbt}: Deterministische Trajektorien, Singularitätenfreiheit und Nicht-Lokalität.
    \end{enumerate}
\end{itemize}

\section{Schritt 2 der Emergenz: Vervollständigung durch Berücksichtigung der Nicht-Lokalität}

\begin{itemize}
    \item Die \gls{art} (selbst die erweiterte Version) ist eine lokale Feldtheorie. Die ursprüngliche \gls{wg} der \gls{wdbt} ist jedoch instantan und nicht-lokal.
    \item Auch diese Eigenschaft kann in die \gls{art} \enquote{importiert} werden, indem man die \textbf{Lösungen der Einstein-Gleichungen} betrachtet. Die \textbf{Gravitationswelle} (Ausbreitung mit $c$) ist nur eine spezielle Lösung.
    \item Die \textbf{instanteane Krümmung} (die die Bewegung der Planeten determiniert) ist eine andere. In einer vollständigen Behandlung muss die \gls{art} beide Beschreibungen gleichberechtigt zulassen können – die retardierten und die avancierten Lösungen (Wheeler-Feynman-Ansatz).
    \item \textbf{Resultat:} Die so vervollständigte \gls{art} wird ebenfalls \textbf{nicht-lokal und lokal} gleichzeitig, genau wie die \gls{wdbt}. Die Retardierung der Wellen ist ein Spezialfall, die Instantaneität der Felder die Regel.
\end{itemize}

\section{Die konvergenten Theorien: ART+ vs. WDBT}
Durch diese beiden Schritte der Vervollständigung nähern sich die erweiterte \gls{art} (\gls{art}+) und die \gls{wdbt} konzeptionell stark an:

\begin{table}[h]
\centering
\begin{tabular}{|p{0.25\textwidth}|p{0.3\textwidth}|p{0.3\textwidth}|}
\hline
\textbf{Eigenschaft} & \textbf{\gls{art}+ (Vervollständigt)} & \textbf{\gls{wdbt} (Fundamental)} \\
\hline
\textbf{Singularitäten} & Keine (Big Bounce) & Keine (Big Bounce) \\
\hline
\textbf{Nicht-Lokalität} & Ja (durch $Q_{\mu\nu}$ \& Feldlösungen) & Ja (fundamental durch \gls{wg}) \\
\hline
\textbf{Determinismus} & Ja (durch $Q$) & Ja (fundamental) \\
\hline
\textbf{Urknall} & Nein & Nein \\
\hline
\textbf{Lichtablenkung} & Frequenzunabhängig & Frequenzabhängig ($\Delta \phi(f)$) \\
\hline
\textbf{Grundlage} & Geometrische Beschreibung & Dyn. Wechselwirkung \\
\hline
\end{tabular}
\caption{Vergleich der vervollständigten ART (ART+) mit der fundamentalen WDBT}
\end{table}

\textbf{Die beiden Theorien scheinen zu konvergieren!}

\section{Der experimentelle Entscheid: Die frequenzabhängige Lichtablenkung}
Trotz der konzeptionellen Konvergenz bleibt ein \textbf{entscheidender, experimentell überprüfbarer Unterschied}:

\begin{itemize}
    \item \textbf{\gls{art}+:} Basiert letztlich auf einer \textbf{geometrischen} Beschreibung. Die Lichtablenkung erfolgt rein geometrisch und ist daher \textbf{frequenzunabhängig}.
    \item \textbf{\gls{wdbt}:} Basiert auf einer dynamischen Wechselwirkung (Weber-Kraft). Die Lichtablenkung ist eine echte Kraftwirkung und daher frequenzabhängig ($\Delta \Phi(f)$).
\end{itemize}

\textbf{Diese Abweichung ist der Lackmustest.} Welche der beiden konvergenten Beschreibungen ist die fundamentalere?

\begin{itemize}
    \item Misst man \textbf{keine} Frequenzabhängigkeit, so ist die geometrische Beschreibung der \gls{art}+ ausreichend.
    \item Misst man \textbf{eine} Frequenzabhängigkeit, ist dies der schlüssige Beweis für die Richtigkeit der dynamischen Grundlage der \gls{wdbt}.
\end{itemize}

\section{Fazit: WDBT als fundamentale Ur-Theorie}
Die Suche nach einer konsistenten Erweiterung der \gls{art} führt also in eine Richtung, die der \gls{wdbt} erstaunlich ähnlich sieht. Dies ist kein Zufall, sondern ein Indiz dafür, dass die \gls{wdbt}
die korrekte fundamentale Ur-Theorie ist.

Die \gls{wdbt} liefert nicht nur die konsistenteste Beschreibung, sondern auch die scharfste, überprüfbare Vorhersage ($\Delta \Phi(f)$), um sich endgültig von allen abgeleiteten effektiven
Theorien (wie der \gls{art}) zu unterscheiden.
