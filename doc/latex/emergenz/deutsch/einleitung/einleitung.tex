\chapter{Spezielle Relativitätstheorie}
Hier ist die vollständige Herleitung aus der \enquote{analogen} \gls{wdbt}

\section{Herleitung der relativistischen Effekte aus der Weber-De Broglie-Bohm-Theorie (WDBT)}
Die Aufgabe ist nicht die 1:1-Rekonstruktion der \gls{srtn}, sondern die Herleitung ihrer operationalen Kernphänomene – Zeitdilatation, Längenkontraktion, relativistische Dynamik – aus den ersten
Prinzipien der \gls{wdbt}, ohne die problematischen Postulate wie die Lorentz-Invarianz der Raumzeit zu übernehmen.

\subsection{Ausgangspunkt: Die Energie-Impuls-Relation in der WDBT}
Die fundamentale Wechselwirkung der \gls{wdbt} wird durch die Weber-Gravitationskraft beschrieben. Für zwei Massen $M$ und $m$ lautet sie mit dem Parameter $\beta = 0.5$:

\begin{equation}
    \label{eq:weber_g}
    \vec{F}_{\text{WG}} = -\frac{G M m}{r^2} \left[ 1 - \frac{\dot{r}^2}{c^2} + 0.5 \frac{r \ddot{r}}{c^2} \right] \hat{\vec{r}}
\end{equation}

Diese Kraft kann aus einem verallgemeinerten Potential $U_{WG}$ abgeleitet werden:

\begin{equation}
    \label{eq:potential}
    U_{\text{WG}}(r, \dot{r}) = -\frac{G M m}{r} \left( 1 - \frac{\dot{r}^2}{2c^2} \right)
\end{equation}

Für ein Teilchen, das sich im kosmischen Hintergrund bewegt, führt die Mittelung über alle Wechselwirkungen zu einer \textbf{effektiven Gesamtenergie}. Die Herleitung über den Lagrangian bzw. den
Hamilton-Formalismus ergibt die \textbf{relativistische Energie-Impuls-Beziehung:}

\begin{equation}
    \label{eq:energie_impuls_beziehung}
    \boxed
    {
        E^2 = (p c)^2 + (m c^2)^2
    }
\end{equation}

\textbf{Diese Gleichung ist kein Postulat.} Sie ist eine direkte Konsequenz der geschwindigkeitsabhängigen Struktur der Weber-Kraft und des Prinzips der Energieerhaltung in der \gls{wdbt}.

\subsection{Definition der relativistischen Größen}
Aus der Energie-Impuls-Beziehung werden die relativistische Energie $E$ und der relativistische Impuls $p$ für ein Teilchen mit Ruhemasse $m$ und Geschwindigkeit $v$ definiert als:

\begin{equation}
    \label{eq:relativistische_energie}
    E = \gamma m c^2, \quad p = \gamma m v, \quad \text{mit} \quad \gamma = \frac{1}{\sqrt{1 - \frac{v^2}{c^2}}}
\end{equation}

Der Lorentz-Faktor $\gamma$ erscheint hier als \textbf{mathematische Konsequenz der Herleitung}, nicht als Ausdruck einer fundamentalen Raumzeit-Symmetrie.

\subsection{Herleitung der Zeitdilatation}
Eine periodische Erscheinung (eine \enquote{Uhr}) habe in ihrem Ruhesystem eine Periodendauer $\Delta t_0$. Ihre Ruheenergie ist $E_0 = mc^2$.

Für einen Beobachter, der sich relativ zur Uhr mit der Geschwindigkeit $v$ bewegt, beträgt die Gesamtenergie der Uhr $E = \gamma mc^2$.

Da die Frequenz $\nu$ einer periodischen Erscheinung proportional zu ihrer Energie ist ($\nu \propto E$), gilt:

\begin{equation}
    \label{eq:zeitdilatation}
    \frac{E}{E_0} = \gamma, \quad \frac{\Delta t_0}{\Delta t} = \gamma \quad \Rightarrow \quad \Delta t = \gamma \Delta t_0 = \frac{\Delta t_0}{\sqrt{1 - \frac{v^2}{c^2}}}
\end{equation}

\paragraph{Resultat:} Die Periodendauer erscheint für den bewegten Beobachter verlängert. Bewegte Uhren gehen langsamer. Dies ist die \textbf{Zeitdilatation}.

\subsection{Herleitung der Längenkontraktion}
Ein Stab der \textbf{Ruhelänge} $L_0$ liege in seinem Ruhesystem. Ein Beobachter, der sich mit der Geschwindigkeit $v$ parallel zum Stab bewegt, muss seine Länge $L$ durch eine \textbf{gleichzeitige}
Messung der Position seiner Endpunkte in \textit{seinem} Bezugssystem bestimmen.

Aufgrund der \textbf{Zeitdilatation} laufen die Uhren, die im System des Stabs synchronisiert sind, im System des Beobachters \textbf{nicht synchron}. Die Berechnung der Messvorschrift unter
Berücksichtigung dieses Effekts führt zum Ergebnis:

\begin{equation}
    \label{eq:längenkontraktion}
    L = \frac{L_0}{\gamma} = L_0 \sqrt{1 - \frac{v^2}{c^2}}
\end{equation}

\paragraph{Resultat:} Die Länge des Stabs erscheint in Bewegungsrichtung verkürzt. Dies ist die \textbf{Längenkontraktion}.

\subsection{Herleitung der relativistischen Dynamik}
Die Bewegungsgleichung eines Teilchens under dem Einfluss einer Kraft $\vec{F}$ wird in der \gls{wdbt} durch die zeitliche Änderung des \textbf{relativistischen Impulses} beschrieben:

\begin{equation}
    \label{eq:relativistischer_impuls}
    \vec{F} = \frac{d\vec{p}}{dt} = \frac{d}{dt} (\gamma m \vec{v})
\end{equation}

Diese Gleichung ersetzt das Newton'sche Gesetz $\vec{F} = m\vec{a}$. Sie beschreibt korrekt die Zunahme der Trägheit bei hohen Geschwindigkeiten ($\gamma \to \infty$ für $v \to c$) und ist konsistent
mit der Energie-Impuls-Beziehung.

\subsection{Zusammenfassung der hergeleiteten Effekte}
Aus der Energie-Impuls-Beziehung $E^2 = (pc)^2 + (mc^2)^2$, die selbst aus der Weber-Kraft folgt, wurden die operationalen Kernphänomene der \gls{srt} hergeleitet:

\begin{align*}
\textbf{Zeitdilatation:} \quad & \Delta t = \gamma \Delta t_0 \\
\textbf{Längenkontraktion:} \quad & L = \frac{L_0}{\gamma} \\
\textbf{Relativistische Dynamik:} \quad & \vec{F} = \frac{d}{dt}(\gamma m \vec{v})
\end{align*}

\subsection{Schlussfolgerung: Emergenz ohne Reduktion}
Die \gls{wdbt} leitet die erfolgreichen Vorhersagen der \gls{srt} aus ihren ersten Prinzipien her. Gleichzeitig vermeidet sie die konzeptionellen Probleme der \gls{srt}:

\begin{itemize}
    \item Die \textbf{Lorentz-Invarianz} wird nicht als fundamentale Eigenschaft der Raumzeit postuliert. Sie erscheint lediglich als eine nützliche Beschreibungsebene für die beobachteten Phänomene.
    \item Die Lichtgeschwindigkeit $c$ ist keine universelle Konstante der Raumzeit, sondern die \textbf{charakteristische Grenzgeschwindigkeit der Weber-Wechselwirkung}. Dies eröffnet die Möglichkeit energieabhängiger Variationen, die in der \gls{srt} und der \gls{artn} ausgeschlossen sind.
\end{itemize}

\section{Die wahre Universalität der Lichtgeschwindigkeit}

Die \textbf{Universalität der Lichtgeschwindigkeit} wird in der modernen Physik unterschiedlich begründet:

\subsection{Postulierte Universalität in SRT/ART}
In der speziellen und allgemeinen Relativitätstheorie ist $c$ eine \textit{postulierte, metaphysische Universalität}:
\begin{itemize}
    \item $c$ wird als fundamentale Konstante der \textbf{leeren Raumzeit} definiert
    \item Unabhängig von Materie und Wechselwirkungen
    \item Abstraktes, letztlich unerklärtes Konzept
    \item Verletzt das Mach'sche Prinzip
\end{itemize}

\subsection{Physikalische Universalität in WG/WED (WDBT)}
In der Weber-Gravitation und Weber-Elektrodynamik ist $c$ eine \textit{physikalische, mechanistische Universalität}:
\begin{itemize}
    \item $c$ ist die charakteristische Grenzgeschwindigkeit der \textbf{Weber-Wechselwirkung}
    \item Entsteht durch instantane Kopplung aller Massen/Ladungen im Universum
    \item Physikalische Konsequenz globaler Kopplung
    \item Erfüllt das Mach'sche Prinzip vollständig
\end{itemize}

Die \gls{wdbt} begründet damit eine \textbf{tiefere und konsistentere Form der Universalität}, die in der physikalischen Realität der Wechselwirkungen verankert ist - nicht in mathematischer Abstraktion.

\section{Begründung der Trägheit in der WDBT}

Die Trägheit wird in der \gls{wdbt} nicht als inherente Eigenschaft eines Teilchens aufgefasst, sondern als \textbf{emergente Konsequenz seiner instantanen Wechselwirkung mit allen anderen Massen und Ladungen im Universum}.

\subsection{Trägheit als Widerstand gegen Beschleunigung}
Die Trägheit äußert sich als Widerstand gegen Beschleunigung. In der \gls{wdbt} wird dies durch die \textbf{Weber-Kraft} erklärt, die auf ein beschleunigtes Teilchen wirkt:

\[
\vec{F}_{\text{WG}} = -\frac{G M m}{r^2} \left[ 1 - \frac{\dot{r}^2}{c^2} + \beta \frac{r \ddot{r}}{c^2} \right] \hat{r}
\]

Der Term $\beta \frac{r \ddot{r}}{c^2}$ (mit $\beta = 0.5$ für Massen) ist proportional zur Relativbeschleunigung $\ddot{r}$ und beschreibt die Kraftkomponente, die der Beschleunigung entgegenwirkt.

\subsection{Masse als Kopplungskonstante}
Die träge Masse $m$ eines Teilchens emergiert als Maß für die Stärke seiner Kopplung an den kosmischen Hintergrund:

\[
m \propto \text{Stärke der Wechselwirkung mit dem Universum}
\]

Die Ruheenergie $E_0 = m c^2$ quantifiziert die Energie, die in dieser Kopplung gespeichert ist.

\subsection{Erfüllung des Mach'schen Prinzips}
Die \gls{wdbt} erfüllt das \textbf{Mach'sche Prinzip} in strenger Form:
\begin{itemize}
    \item Die Trägheit eines Körpers wird durch die Wechselwirkung mit der Gesamtmasse des Universums verursacht.
    \item In einem leeren Universum hätte ein Teilchen keine Trägheit.
    \item Trägheit ist keine lokale Eigenschaft, sondern eine globale Relation.
\end{itemize}

Damit löst die \gls{wdbt} das Rätsel der Trägheit und verwandelt die Masse von einem Grundbegriff in eine \textbf{messbare Konsequenz der kosmischen Wechselwirkung}.

\section{Unterscheidung von Massenträgheit und Ladungsträgheit}

In der \gls{wdbt} ist Trägheit kein einheitliches Phänomen. Sie wird in \textbf{Massenträgheit} und \textbf{Ladungsträgheit} unterteilt, die auf unterschiedlichen Wechselwirkungen beruhen.

\subsection{Massenträgheit}
\begin{itemize}
    \item \textbf{Ursache:} Wechselwirkung mit allen Massen im Universum via \textbf{Weber-Gravitation (WG)}.
    \item \textbf{Kraftgesetz:}
        \[
        \vec{F}_{\text{WG}} = -\frac{G M m}{r^2} \left[ 1 - \frac{\dot{r}^2}{c^2} + \beta_m \frac{r \ddot{r}}{c^2} \right] \hat{r}, \quad \text{mit} \quad \beta_m = 0.5
        \]
    \item Die träge Masse $m$ emergiert als Kopplungskonstante an den \textbf{gravitativen Hintergrund}.
\end{itemize}

\subsection{Ladungsträgheit}
\begin{itemize}
    \item \textbf{Ursache:} Wechselwirkung mit allen Ladungen im Universum via\\\textbf{\gls{wed}}.
    \item \textbf{Kraftgesetz:}
        \begin{equation}
        \vec{F}_{\text{WED}} = \frac{q_1 q_2}{4\pi\epsilon_0 r^2} \left[ 1 - \frac{\dot{r}^2}{c^2} + \beta_e \frac{r \ddot{r}}{c^2} \right] \hat{\vec{r}}, \quad \text{mit} \quad \beta_e = 2            
        \end{equation}
    \item Die Trägheit einer Ladung $q$ emergiert durch Kopplung an den \textbf{elektromagnetischen Hintergrund}.
\end{itemize}

\subsection{Bedeutung}
Die Unterscheidung zeigt, dass Trägheit eine \textbf{wechselwirkungsspezifische} Größe ist. Die \gls{wdbt} ermöglicht so eine Vereinheitlichung von Gravitation und Elektrodynamik auf Grundlage der
Weber-Kraft.

\section{Behandlung von Beschleunigungen in der effektiven SRT}

Die aus der \gls{wdbt} hergeleitete \enquote{effektive} \gls{srt} kann auch Beschleunigungen konsistent behandeln. Dies wird durch die in der \gls{wdbt} begründete \textbf{relativistische Dynamik} ermöglicht,
die durch die Gleichung
\[
\vec{F} = \frac{d}{dt}(\gamma m \vec{v})
\]
beschrieben wird. Diese Gleichung ersetzt das Newtonsche Kraftgesetz $\vec{F} = m \vec{a}$ und ist für beliebige Geschwindigkeiten und Beschleunigungen gültig.

\subsection{Gründe für die Konsistenz mit Beschleunigungen}

\begin{enumerate}
    \item \textbf{Relativistische Kraftgleichung}: Die Gleichung $\vec{F} = \frac{d}{dt}(\gamma m \vec{v})$ berücksichtigt die Zunahme der trägen Masse bei hohen Geschwindigkeiten ($\gamma \to \infty$ für $v \to c$) und beschreibt beschleunigte Bewegungen auch im relativistischen Bereich korrekt.
    \item \textbf{Weber-Kraft als Ursprung}: Die Weber-Kraft (Gl. 1.1) enthält Beschleunigungsterme ($\ddot{r}$), die direkt in die Bewegungsgleichungen eingehen. Dadurch ist die \gls{wdbt} fundamental in der Lage, beschleunigte Systeme zu behandeln.
    \item \textbf{Emergenz der Trägheit}: Die Trägheit wird in der \gls{wdbt} als Folge der Wechselwirkung mit dem kosmischen Hintergrund erklärt (Mach'sches Prinzip). Dadurch wird auch die Trägheit bei Beschleunigungen physikalisch begründet -- nicht als intrinsische Eigenschaft, sondern als emergentes Phänomen.    
    \item \textbf{Konsistenz mit bekannten Ergebnissen}: Die aus der \gls{wdbt} hergeleitete Dynamik ist mathematisch äquivalent zur relativistischen Dynamik der konventionellen \gls{srt}. Daher können auch beschleunigte Bewegungen (wie in Teilchenbeschleunigern oder gravitativen Systemen) konsistent beschrieben werden.
\end{enumerate}

\subsection{Fazit}
Die effektive \gls{srt}, wie sie aus der \gls{wdbt} emergiert, ist \textbf{nicht auf unbeschleunigte Systeme beschränkt}. Sie enthält eine vollständige relativistische Dynamik, die Beschleunigungen miteinbezieht und damit für realistische physikalische Szenarien anwendbar ist.

\section{Zur Bedeutung von Inertialsystemen in der WDBT}

In der \gls{wdbt} erfährt der Begriff des \textbf{Inertialsystems} eine fundamentale Neubewertung und verliert seinen privilegierten Status als ausgezeichneter Bezugsrahmen.

\subsection{Die Relativität der Trägheit}

\begin{itemize}
    \item \textbf{Trägheit als Wechselwirkungseffekt}: In der \gls{wdbt} ist Trägheit keine intrinsische Eigenschaft von Massen, sondern emergiert aus der instantanen Wechselwirkung mit allen anderen Massen im Universum (Mach'sches Prinzip \cite{Assis1999}).
    \item \textbf{Abschaffung des absoluten Raums}: Da es keinen leeren, strukturlosen Raum gibt, der als absolutes Bezugssystem dienen könnte, verlieren Inertialsysteme ihren fundamentalen Charakter.
\end{itemize}

\subsection{Operationale versus fundamentale Bedeutung}

\begin{itemize}
    \item \textbf{Operationale Nützlichkeit}: Inertialsysteme behalten ihre \emph{operative} Bedeutung als Bezugssysteme, in denen die Newtonsche Mechanik näherungsweise gültig bleibt bei Geschwindigkeiten viel kleiner als $c$.
    \item \textbf{Fundamentaler Status entfällt}: In der \gls{wdbt} sind Inertialsysteme keine fundamentalen Entitäten der Physik, sondern lediglich \emph{nützliche Approximationen} in bestimmten Grenzfällen.
\end{itemize}

\subsection{Beschleunigte Bezugssysteme als gleichberechtigt}

\begin{itemize}
    \item \textbf{Universelle Gültigkeit der Bewegungsgleichungen}: Die Grundgleichungen der \gls{wdbt} gelten in \emph{allen} Bezugssystemen gleichermaßen, da die Trägheitskräfte physikalisch durch die Wechselwirkung mit dem kosmischen Hintergrund erklärt werden.
    \item \textbf{Keine Auszeichnung von Inertialsystemen}: Während in der klassischen \gls{srt} Inertialsysteme ausgezeichnet sind, behandelt die \gls{wdbt} alle Bezugssysteme - beschleunigte und unbeschleunigte - als physikalisch gleichwertig.
\end{itemize}

\subsection{Konsequenzen für das Raumzeit-Verständnis}

\begin{itemize}
    \item \textbf{Emergenz der Lorentz-Invarianz}: Die beobachtete Lorentz-Invarianz erscheint als emergentes Phänomen aus der zugrundeliegenden Weber-Wechselwirkung, nicht als fundamentale Eigenschaft der Raumzeit.
    \item \textbf{Relationale Raumzeit}: Die \gls{wdbt} impliziert eine relationale Auffassung von Raum und Zeit, in der physikalische Effekte stets durch konkrete Wechselwirkungen zwischen Materie erklärt werden.
\end{itemize}

\subsection{Fazit}

In der \gls{wdbt} verlieren Inertialsysteme ihren \textbf{fundamentalen Status} als ausgezeichnete Bezugssysteme. Während sie ihre \emph{operative Nützlichkeit} in bestimmten Grenzfällen behalten, sind sie in der fundamentalen Beschreibung der Physik durch die \gls{wdbt} nicht mehr als privilegierte Bezugsrahmen notwendig. Die Theorie bietet damit eine konsequente Implementierung des Mach'schen Prinzips und eine relationale Auffassung von Raum, Zeit und Trägheit.

