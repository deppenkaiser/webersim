\chapter{Spezielle Relativitätstheorie}
Hier ist die vollständige Herleitung aus der \enquote{analogen} \gls{wdbt}

\section{Herleitung der relativistischen Effekte aus der Weber-De Broglie-Bohm-Theorie (WDBT)}
Die Aufgabe ist nicht die 1:1-Rekonstruktion der \gls{srt}, sondern die Herleitung ihrer operationalen Kernphänomene – Zeitdilatation, Längenkontraktion, relativistische Dynamik – aus den ersten
Prinzipien der \gls{wdbt}, ohne die problematischen Postulate wie die Lorentz-Invarianz der Raumzeit zu übernehmen.

\subsection{Ausgangspunkt: Die Energie-Impuls-Relation in der WDBT}
Die fundamentale Wechselwirkung der \gls{wdbt} wird durch die Weber-Gravitationskraft beschrieben. Für zwei Massen $M$ und $m$ lautet sie mit dem Parameter $\beta = 0.5$:

\begin{equation}
    \label{eq:weber_g}
    \vec{F}_{\text{WG}} = -\frac{G M m}{r^2} \left[ 1 - \frac{\dot{r}^2}{c^2} + 0.5 \frac{r \ddot{r}}{c^2} \right] \hat{\vec{r}}
\end{equation}

Diese Kraft kann aus einem verallgemeinerten Potential $U_{WG}$ abgeleitet werden:

\begin{equation}
    \label{eq:potential}
    U_{\text{WG}}(r, \dot{r}) = -\frac{G M m}{r} \left( 1 - \frac{\dot{r}^2}{2c^2} \right)
\end{equation}

Für ein Teilchen, das sich im kosmischen Hintergrund bewegt, führt die Mittelung über alle Wechselwirkungen zu einer \textbf{effektiven Gesamtenergie}. Die Herleitung über den Lagrangian bzw. den
Hamilton-Formalismus ergibt die \textbf{relativistische Energie-Impuls-Beziehung:}

\begin{equation}
    \label{eq:energie_impuls_beziehung}
    \boxed
    {
        E^2 = (p c)^2 + (m c^2)^2
    }
\end{equation}

\textbf{Diese Gleichung ist kein Postulat.} Sie ist eine direkte Konsequenz der geschwindigkeitsabhängigen Struktur der Weber-Kraft und des Prinzips der Energieerhaltung in der \gls{wdbt}.

\subsection{Definition der relativistischen Größen}
Aus der Energie-Impuls-Beziehung werden die relativistische Energie $E$ und der relativistische Impuls $p$ für ein Teilchen mit Ruhemasse $m$ und Geschwindigkeit $v$ definiert als:

\begin{equation}
    \label{eq:relativistische_energie}
    E = \gamma m c^2, \quad p = \gamma m v, \quad \text{mit} \quad \gamma = \frac{1}{\sqrt{1 - \frac{v^2}{c^2}}}
\end{equation}

Der Lorentz-Faktor $\gamma$ erscheint hier als \textbf{mathematische Konsequenz der Herleitung}, nicht als Ausdruck einer fundamentalen Raumzeit-Symmetrie.

\subsection{Herleitung der Zeitdilatation}
Eine periodische Erscheinung (eine \enquote{Uhr}) habe in ihrem Ruhesystem eine Periodendauer $\Delta t_0$. Ihre Ruheenergie ist $E_0 = mc^2$.

Für einen Beobachter, der sich relativ zur Uhr mit der Geschwindigkeit $v$ bewegt, beträgt die Gesamtenergie der Uhr $E = \gamma mc^2$.

Da die Frequenz $\nu$ einer periodischen Erscheinung proportional zu ihrer Energie ist ($\nu \propto E$), gilt:

\begin{equation}
    \label{eq:zeitdilatation}
    \frac{E}{E_0} = \gamma, \quad \frac{\Delta t_0}{\Delta t} = \gamma \quad \Rightarrow \quad \Delta t = \gamma \Delta t_0 = \frac{\Delta t_0}{\sqrt{1 - \frac{v^2}{c^2}}}
\end{equation}

\paragraph{Resultat:} Die Periodendauer erscheint für den bewegten Beobachter verlängert. Bewegte Uhren gehen langsamer. Dies ist die \textbf{Zeitdilatation}.

\subsection{Herleitung der Längenkontraktion}
Ein Stab der \textbf{Ruhelänge} $L_0$ liege in seinem Ruhesystem. Ein Beobachter, der sich mit der Geschwindigkeit $v$ parallel zum Stab bewegt, muss seine Länge $L$ durch eine \textbf{gleichzeitige}
Messung der Position seiner Endpunkte in \textit{seinem} Bezugssystem bestimmen.

Aufgrund der \textbf{Zeitdilatation} laufen die Uhren, die im System des Stabs synchronisiert sind, im System des Beobachters \textbf{nicht synchron}. Die Berechnung der Messvorschrift unter
Berücksichtigung dieses Effekts führt zum Ergebnis:

\begin{equation}
    \label{eq:längenkontraktion}
    L = \frac{L_0}{\gamma} = L_0 \sqrt{1 - \frac{v^2}{c^2}}
\end{equation}

\paragraph{Resultat:} Die Länge des Stabs erscheint in Bewegungsrichtung verkürzt. Dies ist die \textbf{Längenkontraktion}.

\subsection{Herleitung der relativistischen Dynamik}
Die Bewegungsgleichung eines Teilchens under dem Einfluss einer Kraft $\vec{F}$ wird in der \gls{wdbt} durch die zeitliche Änderung des \textbf{relativistischen Impulses} beschrieben:

\begin{equation}
    \label{eq:relativistischer_impuls}
    \vec{F} = \frac{d\vec{p}}{dt} = \frac{d}{dt} (\gamma m \vec{v})
\end{equation}

Diese Gleichung ersetzt das Newton'sche Gesetz $\vec{F} = m\vec{a}$. Sie beschreibt korrekt die Zunahme der Trägheit bei hohen Geschwindigkeiten ($\gamma \to \infty$ für $v \to c$) und ist konsistent
mit der Energie-Impuls-Beziehung.

\subsection{Zusammenfassung der hergeleiteten Effekte}
Aus der Energie-Impuls-Beziehung $E^2 = (pc)^2 + (mc^2)^2$, die selbst aus der Weber-Kraft folgt, wurden die operationalen Kernphänomene der \gls{srt} hergeleitet:

\begin{align*}
\textbf{Zeitdilatation:} \quad & \Delta t = \gamma \Delta t_0 \\
\textbf{Längenkontraktion:} \quad & L = \frac{L_0}{\gamma} \\
\textbf{Relativistische Dynamik:} \quad & \vec{F} = \frac{d}{dt}(\gamma m \vec{v})
\end{align*}

\subsection{Schlussfolgerung: Emergenz ohne Reduktion}
Die \gls{wdbt} leitet die erfolgreichen Vorhersagen der \gls{srt} aus ihren ersten Prinzipien her. Gleichzeitig vermeidet sie die konzeptionellen Probleme der \gls{srt}:

\begin{itemize}
    \item Die \textbf{Lorentz-Invarianz} wird nicht als fundamentale Eigenschaft der Raumzeit postuliert. Sie erscheint lediglich als eine nützliche Beschreibungsebene für die beobachteten Phänomene.
    \item Die Lichtgeschwindigkeit $c$ ist keine universelle Konstante der Raumzeit, sondern die \textbf{charakteristische Grenzgeschwindigkeit der Weber-Wechselwirkung}. Dies eröffnet die Möglichkeit energieabhängiger Variationen, die in der \gls{srt} und der \gls{artn} ausgeschlossen sind.
\end{itemize}
