\documentclass[11pt, a5paper, twoside, openright]{book}
\usepackage[ngerman]{babel}
\usepackage[T1]{fontenc}
\usepackage[utf8]{inputenc}
\usepackage{lmodern}
\usepackage{microtype}
\usepackage{csquotes}
\usepackage{verbatim}  % Im Kopf des Dokuments einfügen
\usepackage{geometry}
\usepackage{fancyhdr}
\usepackage{amsmath, amssymb, amsthm}  % Mathe
\usepackage{mathtools}                 % \coloneqq, \xrightarrow
\usepackage{bm}                        % Fette Symbole (\bm{B} für Magnetfeld)
\usepackage{siunitx}                   % \SI{1.23}{\meter\per\second}
\usepackage{graphicx}                  % \includegraphics
\usepackage{subcaption}                % Unterabbildungen
\usepackage{booktabs}                  % Professionelle Tabellen
\usepackage{tikz}                      % Für Diagramme
\usepackage{xcolor}                    % Farbige Tabellenzellen
\usepackage[
    backend=biber,
    style=phys,         % APS-Zitierstil (für Physik)
    sorting=nyt,        % Sortierung: Name, Jahr, Titel
]{biblatex}
\usepackage[acronym, toc]{glossaries}
\usepackage{hyperref}
\usepackage{parskip}
\usepackage{pgfplots}
\usepackage{glossaries}
\makeglossaries
\geometry{
    a4paper,
    top=25mm,
    inner=30mm,    % Bundsteg (größerer Rand für Buchbindung)
    outer=25mm,
    bottom=30mm,
    headheight=15pt,
}

\pagestyle{fancy}
\fancyhf{}
\fancyhead[LE,RO]{\thepage}
\fancyhead[RE]{\leftmark}    % Kapitelname (gerade Seiten)
\fancyhead[LO]{\rightmark}   % Abschnittname (ungerade Seiten)
\renewcommand{\headrulewidth}{0.4pt}

\theoremstyle{definition}
\newtheorem{definition}{Definition}[chapter]
\newtheorem{law}{Physikalisches Gesetz}[chapter]
\theoremstyle{plain}
\newtheorem{theorem}{Theorem}[chapter]
\newtheorem{lemma}[theorem]{Lemma}
\theoremstyle{remark}
\newtheorem{remark}{Bemerkung}[chapter]

\hypersetup{
    colorlinks=true,
    linkcolor=blue,
    citecolor=black,
    urlcolor=black,
    pdftitle={Emergenz der Kosmologie: Die WDBT als Ur-Theorie},
    pdfauthor={Dipl.-Ing. (FH) Michael Czybor},
}

\addbibresource{literatur.bib}  % Ihre .bib-Datei
\makeglossaries

\setlength{\headheight}{26.76852pt}
\definecolor{quantenblau}{RGB}{0, 100, 200}
\definecolor{weberrot}{RGB}{180, 20, 60}
\definecolor{hintergrund}{RGB}{20, 20, 40}
\usetikzlibrary{shapes, calc, 3d}
\pgfplotsset{compat=1.18} % Aktuelle Version verwenden

\newacronym{qm}{QM}{Quantum Mechanics}
\newacronym{art}{ART}{General Theory of Relativity}
\newacronym{srt}{SRT}{Special Theory of Relativity}
\newacronym{cmb}{CMB}{Cosmic Microwave Background}
\newacronym{qed}{QED}{Quantum Electrodynamics}
\newacronym{epr}{EPR Paradox}{Einstein-Podolsky-Rosen Paradox}
\newacronym{wg}{WG}{Weber Gravitation}
\newacronym{dbt}{DBT}{De Broglie-Bohm Theory}
\newacronym{wdbt}{WDBT}{Weber-De Broglie-Bohm Theory}
\newacronym{mt}{MT}{Maxwell Theory}
\newacronym{mhd}{MHD}{Magnetohydrodynamics}
\newacronym{wed}{WED}{Weber Electrodynamics}
\newacronym{eu}{EU}{Electric Universe}

\newglossaryentry{gls:quantenmechanik}
{
    name={Quantum Mechanics},
    description={Theory of matter and radiation at the atomic and subatomic level}
}
\newglossaryentry{gls:hamiltonian}
{
    name={\ensuremath{\mathcal{H}}},
    description={Hamiltonian operator, describes the total energy of a system},
    sort={hamiltonian}
}

\begin{document}

\frontmatter
\begin{tikzpicture}[remember picture, overlay]

  % Hintergrund (Dunkel mit fraktalem Gitter)
  \fill[hintergrund] (current page.south west) rectangle (current page.north east);
  \foreach \i in {0,10,...,360} {
    \draw[quantenblau!10, line width=0.1pt] 
      (current page.center) -- +(\i:5cm);
  }

  % Dodekaeder (abstrahiert)
  \node[rotate=25, scale=2, quantenblau!50] at (current page.center) {
    \begin{tikzpicture}[scale=0.3]
      \draw[quantenblau] (0:1) \foreach \a in {72,144,...,360} { -- (\a:1) } -- cycle;
      \foreach \a in {36,108,...,324} { \draw[quantenblau] (0,0) -- (\a:1.6); }
    \end{tikzpicture}
  };

  % Titeltext (mit Schatten-Effekt)
  \node[align=center, text=white, font=\sffamily\bfseries\Huge] 
    at ($(current page.center)+(0,3cm)$) {
    \textbf{Emergenz der Kosmologie}
  };
  \node[align=center, text=quantenblau!80, font=\sffamily\Large] 
    at ($(current page.center)+(0,1.8cm)$)
    {
        Die WDBT als Ur-Theorie
    };

  % Kernformeln (rechts unten)
  \node[align=left, anchor=south east, text=weberrot!70, font=\small] 
    at ($(current page.south east)+(-1cm,1cm)$) {
    $\displaystyle \vec{F}_{\text{WG}} = -\frac{GMm}{r^2}\left(1-\frac{\dot{r}^2}{c^2}+\beta\frac{r\ddot{r}}{c^2}\right)$
  };
  \node[align=left, anchor=north east, text=quantenblau!70, font=\small] 
    at ($(current page.south east)+(-1cm,3cm)$) {
    $\displaystyle Q = -\frac{\hbar^2}{2m}\frac{\nabla^2\sqrt{\rho}}{\sqrt{\rho}}$
  };

  % Autor (unten mittig)
  \node[align=center, text=white, font=\sffamily\large] 
    at ($(current page.south)+(0,1cm)$) {
    \textbf{Michael Czybor}
  };

  % Fraktale Dimension (links oben)
  \node[align=right, text=quantenblau!50, font=\small] 
    at ($(current page.north west)+(2cm,-1cm)$) {
    $D = \frac{\ln 20}{\ln(2+\phi)} \approx 2.71$
  };

\end{tikzpicture}

\title{Emergenz der Kosmologie\\Die WDBT als Ur-Theorie}
\author{Michael Czybor}
\date{\today}
\maketitle

\chapter*{Vorwort}
Seit über einem Jahrhundert bilden die Relativitätstheorien und die Quantenmechanik die Säulen unseres physikalischen Weltbildes. Doch trotz ihrer unbestrittenen empirischen Erfolge bleiben
konzeptionelle Fragen offen: Was ist die Natur der Gravitation? Wie lässt sich die Quantenwelt mit der klassischen Physik vereinbaren? Und was verbirgt sich hinter Phänomenen wie der Trägheit oder der
Lichtausbreitung?

In diesem Buch wird ein radikal neuer Ansatz vorgestellt: die \textbf{\gls{wdbt}}. Sie vereint zwei bisher weitgehend unabhängige Theorien – die \gls{wed} und die De-Broglie-Bohm’sche Quantenmechanik – zu einer
umfassenden Ur-Theorie, aus der die bekannten physikalischen Gesetze \textbf{emergieren}. Das bedeutet: Die Spezielle und Allgemeine Relativitätstheorie, die Maxwell-Gleichungen und sogar die Quantenmechanik
werden nicht postuliert, sondern aus den grundlegenden Prinzipien der \gls{wdbt} hergeleitet.

Besondere Beachtung verdient dabei die \textbf{konvergente Emergenz der \gls{art}}. Die konventionelle \gls{art} leidet unter dem Problem der Singularitäten – Schwarze Löcher und der Urknall markieren ihre
Grenzen. Eine naheliegende Erweiterung, die \gls{art}+, führt das Bohm’sche Quantenpotential ein und löst diese Probleme: Sie ist singularitätenfrei, deterministisch und nicht-lokal. Damit negiert sie
den Urknall und ersetzt ihn durch einen \enquote{Big Bounce}. Die \gls{art}+ stellt die maximale Annäherung der geometrischen Beschreibung an die kraftbasierte \gls{wdbt} dar – bis auf einen
entscheidenden, experimentell überprüfbaren Unterschied: die \textbf{frequenzabhängige Lichtablenkung}. Während die \gls{art}+ eine rein geometrische und damit frequenzunabhängige Beschreibung liefert, sagt die
\gls{wdbt} als dynamische Wechselwirkungstheorie eine Frequenzabhängigkeit voraus. Damit wird die \gls{art}+ zu einer Form der \textbf{Quantengravitation auf geometrischer Basis}, die der \gls{wdbt} konzeptionell
erstaunlich nahekommt, ohne sie vollständig zu ersetzen.

Doch die \gls{wdbt} geht noch weiter. Während die \enquote{analoge} \gls{wdbt} bereits eine vollständige Emergenz der bekannten Physik leistet, öffnet die \textbf{digitale \gls{wdbt}} die Tür zu einer
noch fundamentaleren Ebene. Sie erklärt nicht nur die Wechselwirkungen, sondern auch die Naturkonstanten und die Struktur der Raumzeit selbst – basierend auf einem digital-fraktalen Raummodell mit einer
Dimension von etwa $D \approx 2,71$. In dieser Theorie werden sowohl die Weber-Kraft als auch das Quantenpotential aus derselben Ursache generiert: der diskreten, nicht-lokalen Struktur des Raumes.

Dieses Buch verfolgt das Ziel, die Emergenz der gesamten modernen Physik aus der \gls{wdbt} systematisch herzuleiten und ihre konzeptionellen sowie empirischen Vorteile darzulegen. Es richtet sich an
Physiker, Philosophen, und studierende der Naturwissenschaften, und alle, die nach einer kohärenten, realistischen und deterministischen Grundlage der Physik suchen. Die \gls{wdbt} bietet nicht nur eine Vereinheitlichung,
sondern auch eine \textbf{ontologische Vertiefung} unseres Verständnisses von Raum, Zeit, Materie und Wechselwirkung.

\begin{flushright}
    Michael Czybor \\
    \emph{Langenstein/AT, August 2025}
\end{flushright}

\tableofcontents
\listoffigures
\listoftables

\mainmatter
\chapter{Einleitung}
\section{Motivation}
Viele Schüler und Studierende erleben den Physikunterricht als frustrierend und unverständlich. Besonders die moderne Physik – mit der Allgemeinen Relativitätstheorie (ART)
und der Speziellen Relativitätstheorie (SRT) – wirkt oft unphysikalisch und voller logischer Widersprüche. Energie scheint unter bestimmten Bedingungen unendlich zu werden,
Überlichtgeschwindigkeit wird in manchen Fällen postuliert, obwohl sie eigentlich unmöglich sein soll, und Begriffe wie \enquote{dunkle Energie} oder \enquote{dunkle Materie} wirken wie
Platzhalter für unser Unverständnis.

Ein grundlegendes Problem liegt in den Widersprüchen zwischen ART und SRT. Die SRT baut auf Inertialsystemen auf, also Bezugssystemen, die sich gleichförmig und unbeschleunigt
bewegen. Doch laut ART gibt es keine perfekten Inertialsysteme, da jede Masse die Raumzeit krümmt und damit Beschleunigungen erzeugt. Schon allein dieser Widerspruch wirft
Fragen auf: Wenn Inertialsysteme streng genommen punktförmig sein müssten, um frei von jeder Krümmung zu sein, bräuchte man unendlich viele davon – und damit auch unendlich
viele verschiedene Lichtgeschwindigkeiten, da diese vom Bezugssystem abhängt.

Hinzu kommt, dass viele Konzepte der modernen Physik unserer Intuition widersprechen. Die Quantenmechanik verlangt, dass Teilchen gleichzeitig Wellen sind und erst durch
Beobachtung einen definierten Zustand annehmen. Die ART beschreibt eine gekrümmte Raumzeit, die sich kaum jemand wirklich vorstellen kann, und die SRT führt zu scheinbar
paradoxen Zeitdehnungen und Längenkontraktionen. Selbst der Urknall als Anfangspunkt des Universums wirft Fragen auf: Wie kann etwas aus dem Nichts entstehen? Warum gibt es
überhaupt eine Singularität, wenn doch unsere physikalischen Gesetze dort versagen?

All diese Punkte zeigen, dass die moderne Physik noch lange nicht abgeschlossen ist. Statt blind akzeptierte Theorien als absolute Wahrheit zu betrachten, sollten wir die
Widersprüche hinterfragen und nach konsistenteren Erklärungen suchen.

\chapter{Die Emergenz der Maxwell-Gleichungen}
\label{ch:maxwell}
\section{Grundlagen der Weber-Elektrodynamik}
\label{sec:grundlagen}
Die \gls{wed} postuliert eine instantane, geschwindigkeits- und beschleunigungsabhängige Kraft zwischen zwei Ladungen $q_1$ und $q_2$. Die vektorielle Form der Weber-Kraft lautet:

\begin{equation}
    \vec{F}_{12} = \frac{q_1 q_2}{4\pi\epsilon_0 r^2} \left\{ \left[ 1 - \frac{v^2}{c^2} + \frac{2r(\hat{\vec{r}}\cdot\vec{a})}{c^2} \right] \hat{\vec{r}} + \frac{2(\hat{\vec{r}}\cdot\vec{v})}{c^2} \vec{v} \right\}
\end{equation}

wobei:

\begin{itemize}
    \item $r = \left| \vec{r} \right|$ der Abstand zwischen den Ladungen,
    \item $\hat{\vec{r}} = \frac{\vec{r}}{r}$ der Einheitsvektor in Richtung von $q_1$ nach $q_2$,
    \item $\vec{v} = \dot{\vec{r}}$ die Relativgeschwindigkeit,
    \item $\vec{a} = \ddot{\vec{r}}$ die Relativbeschleunigung,
    \item $c$ die charakteristische Grenzgeschwindigkeit der Wechselwirkung ist.
\end{itemize}

Diese Kraft kann aus einem verallgemeinerten Potential abgeleitet werden und erfüllt die Energie- und Impulserhaltung.

\section{Superposition und Gesamtkraft auf eine Testladung}
Für eine Testladung $q$ im Feld von $N$ anderen Ladungen $q_i (i = 1,...,N)$ gilt das Superpositionsprinzip. Die Gesamtkraft auf $q$ ist:

\begin{equation}
    \label{eq:gesamtkraft}
    \vec{F}_{\text{ges}} = q \sum_{i=1}^N \frac{q_i}{4\pi\epsilon_0 r_i^2} \left\{ \left[ 1 - \frac{v_i^2}{c^2} + \frac{2r_i(\hat{r}_i\cdot\vec{a}_i)}{c^2} \right] \hat{r}_i + \frac{2(\hat{r}_i\cdot\vec{v}_i)}{c^2} \vec{v}_i \right\}
\end{equation}

wobei $\vec{r_i}$ der Vektor von $q$ zu $q_i$ ist.

\section{Definition der effektiven Felder}
Die Gesamtkraft aus den effektiven Feldern $\vec{E}$ und $\vec{B}$ lässt sich in die Form der Lorentz-Kraft bringen:

\begin{equation}
    \label{eq:lorentz_kraft}
    \vec{F} = q \left( \vec{E} + \vec{v} \times \vec{B} \right)
\end{equation}

Durch Koeffizientenvergleich ergeben sich die Definitionen der effektiven Felder:

\begin{equation}
    \vec{E} = \sum_{i=1}^N \frac{q_i}{4\pi\epsilon_0 r_i^2} \left[ 1 - \frac{v_i^2}{c^2} + \frac{2r_i(\hat{r}_i\cdot\vec{a}_i)}{c^2} \right] \hat{r}_i
\end{equation}

\begin{equation}
    \vec{B} = \sum_{i=1}^N \frac{q_i}{4\pi\epsilon_0 r_i^2 c^2} \cdot 2(\hat{r}_i\cdot\vec{v}_i) \vec{v}_i
\end{equation}

Diese effektiven Felder sind mathematische Hilfsgrößen, die die gemittelte Wirkung aller anderen Ladungen beschreiben.

\section{Kontinuumslimes und Feldgleichungen}
\label{sec:kontinuumslimes}
Bei einer kontinuierlichen Ladungsverteilung mit Dichte $\rho(\vec{r}, t)$ und Stromdichte $\vec{j}(\vec{r},t)$ gehen die Summen in Integrale über. Die Felder werden zu:

\begin{equation}
    \vec{E}(\vec{r}, t) = \frac{1}{4\pi\epsilon_0} \int \rho(\vec{r}~', t) \left[ 1 - \frac{v^2}{c^2} + \frac{2r(\hat{\vec{r}}\cdot\vec{a})}{c^2} \right] \frac{\hat{\vec{r}}}{r^2}  d^3r~'
\end{equation}

\begin{equation}
    \vec{B}(\vec{r}, t) = \frac{1}{4\pi\epsilon_0 c^2} \int \vec{j}(\vec{r}~', t) \cdot 2(\hat{\vec{r}}\cdot\vec{v}) \frac{\hat{\vec{r}}}{r^2}  d^3r~'
\end{equation}

% Variablenerklärung für die Integrale
\begin{align*}
&\vec{r} && \text{Ortsvektor zum Aufpunkt} \\
&\vec{r}~' && \text{Ortsvektor zur Quellladung} \\
&\vec{r} = \vec{r} - \vec{r}~' && \text{Abstandsvektor} \\
&r = |\vec{r} - \vec{r}~'| && \text{Abstand} \\
&\hat{\vec{r}} = \frac{\vec{r} - \vec{r}~'}{|\vec{r} - \vec{r}~'|} && \text{Einheitsvektor} \\
&\rho(\vec{r}~', t) && \text{Ladungsdichte am Quellpunkt} \\
&\vec{j}(\vec{r}~', t) && \text{Stromdichte am Quellpunkt} \\
&d^3r~' && \text{Volumenelement im Quellraum}
\end{align*}

\subsection{Gauß'sches Gesetz}
\begin{equation}
    \nabla \cdot \vec{E} = \frac{\rho}{\epsilon_0}
\end{equation}

\subsection{Gauß'sches Gesetz für den Magnetismus}
\begin{equation}
    \nabla \cdot \vec{B} = 0
\end{equation}

\subsection{Faraday'sches Induktionsgesetz}
\begin{equation}
    \nabla \times \vec{E} = -\frac{\partial \vec{B}}{\partial t}
\end{equation}

\subsection{Maxwell'scher Verschiebungsstrom}
\begin{equation}
    \nabla \times \vec{B} = \mu_0 \vec{j} + \mu_0 \epsilon_0 \frac{\partial \vec{E}}{\partial t}
\end{equation}

\section{Emergenz der elektromagnetischen Wellen}
Im Vakuum $(\rho = 0, \vec{j} = 0)$ vereinfachen sich die Maxwell-Gleichungen zu:

\begin{align}
\nabla \cdot \vec{E} =&~0\\
\nabla \cdot \vec{B} =&~0\\
\nabla \times \vec{E} =& -\frac{\partial \vec{B}}{\partial t}\\
\nabla \times \vec{B} =& \mu_0 \epsilon_0 \frac{\partial \vec{E}}{\partial t}
\end{align}

Durch Bildung der Rotation der letzten beiden Gleichungen erhält man die Wellengleichungen:

\begin{align}
\nabla^2 \vec{E} =& \mu_0 \epsilon_0 \frac{\partial^2 \vec{E}}{\partial t^2}\\
\nabla^2 \vec{B} =& \mu_0 \epsilon_0 \frac{\partial^2 \vec{B}}{\partial t^2}
\end{align}

Die Ausbreitungsgeschwindigkeit ist:

\begin{equation}
    c = \frac{1}{\sqrt{\mu_0 \epsilon_0}}
\end{equation}

\newpage
\section{Die analoge WDBT und die Rolle des Quantenpotentials}
\subsection{Die vollständige Kraftgleichung der analogen WDBT}
Die \gls{wdbt} kombiniert zwei selten genutzte etablierte Konzepte:

\begin{enumerate}
    \item Die \textbf{\gls{wed}} für die klassische instantane Wechselwirkung.
    \item Die \textbf{\gls{dbt}} für die Quantenmechanik via Quantenpotential.
\end{enumerate}

Die Gesamtkraft auf ein Teilchen der Masse $m$ und Ladung $q$ in der analogen \gls{wdbt} ist daher:

\begin{equation}
    \vec{F}_{\text{ges, WDBT}} = q(\vec{E} + \vec{v} \times \vec{B}) - \nabla Q
\end{equation}

mit dem \textbf{Bohm'schen Quantenpotential:}

\begin{equation}
    Q = -\frac{\hbar^2}{2m} \frac{\nabla^2 \sqrt{\rho}}{\sqrt{\rho}}
\end{equation}

\subsection{Emergenz der klassischen Physik in der WDBT}
Die klassische Physik emergiert in zwei Schritten aus der \gls{wdbt}:

\begin{enumerate}
    \item \textbf{Emergenz der Maxwell-Theorie:} Wie in Abschnitt \ref{sec:grundlagen} gezeigt, geht die \gls{wed} durch Mittelung über viele Ladungen in die Maxwell-Gleichungen über.
    \item \textbf{Emergenz der Newton'schen Mechanik:} Für $\hbar \to 0$ oder auf makroskopischen Skalen, wo Quanteneffekte vernachlässigbar sind, verschwindet das Quantenpotential ($Q \to 0$). Die Kraftgleichung der \gls{wdbt} reduziert sich auf die \textbf{Lorentz-Kraft} (Gl. \refeq{eq:lorentz_kraft}) der klassischen Elektrodynamik.
\end{enumerate}

Zusammen mit den emergenten Maxwell-Gleichungen ist dies die vollständige klassische Beschreibung.

\subsection{Die erweiterte Kraftdichte und die Kontinuitätsgleichung}
In der analogen \gls{wdbt} wirkt auf eine Ladungsdichte $\rho$ nicht nur die Lorentz-Kraftdichte, sondern auch die Kraftdichte aus dem Quantenpotential $Q$. Die vollständige Kraftdichte lautet:

\begin{equation}
    \vec{f}_{\text{WDBT}} = \rho \vec{E} + \vec{j} \times \vec{B} - \rho \nabla Q
\end{equation}

Diese Kraftdichte muss in die Impulsbilanz der Kontinuumsmechanik eingesetzt werden. Zusätzlich gilt die Kontinuitätsgleichung (Ladungserhaltung) unverändert:

\begin{equation}
    \frac{\partial \rho}{\partial t} + \nabla \cdot \vec{j} = 0
\end{equation}

\subsection{Herleitung der modifizierten Maxwell-Gleichungen}
Durch die Einführung des Quantenpotentials $Q$ werden die Quellterme in den Maxwell-Gleichungen erweitert. Die modifizierten inhomogenen Maxwell-Gleichungen in der analogen \gls{wdbt} lauten:

\begin{equation}
    \nabla \cdot \vec{E} = \frac{\rho}{\epsilon_0} - \frac{1}{\epsilon_0} \nabla \cdot (\rho \nabla Q)
\end{equation}

\begin{equation}
    \nabla \times \vec{B} = \mu_0 \vec{j} + \mu_0 \epsilon_0 \frac{\partial \vec{E}}{\partial t} + \mu_0 \nabla \times (\rho \nabla Q)
\end{equation}

Begründung:

\begin{itemize}
    \item Der Term $-\rho \nabla Q$ in der Kraftdichte wirkt wie eine zusätzliche Quanten-Ladungsdichte bzw. ein Quanten-Strom.
    \item Diese Zusatzterme müssen in den Quellgleichungen für $\vec{E}$ und $\vec{B}$ erscheinen, um die Konsistenz mit der Impulserhaltung zu wahren.
    \item Die homogenen Gleichungen ($\nabla \cdot \vec{B} = 0, \nabla \times \vec{E} = -\partial_t \vec{B}$) bleiben unverändert, da sie aus den Definitionen der Felder folgen.
\end{itemize}

\subsubsection{Klassischer Grenzfall und Reduktion}
Im klassischen Grenzfall ($Q \to 0$ oder $\hbar \to 0$) verschwinden die Zusatzterme:

\begin{equation}
    \lim_{Q \to 0} \left( \nabla \cdot \vec{E} \right) = \frac{\rho}{\epsilon_0}
\end{equation}

\begin{equation}
    \lim_{Q \to 0} \left( \nabla \times \vec{B} \right) = \mu_0 \vec{j} + \mu_0 \epsilon_0 \frac{\partial \vec{E}}{\partial t}
\end{equation}

Somit emergieren ebenfalls exakt die klassischen Maxwell-Gleichungen.

\section{Die digitale WDBT: Eine hypothetische Fundierung}
Die \textbf{digitale \gls{wdbt}} zielt darauf ab, den Ursprung \textbf{sowohl der Weber-Kraft als auch des Quantenpotentials} in einer einzigen fundamentalen Theorie zu erklären:

\begin{itemize}
    \item Das digital-fraktale Raummodell (mit $D \approx 2.71$) soll die instantane Wechselwirkung (\gls{wed}) und die Quantenfluktuationen ($Q$) aus der gleichen Ursache generieren.
    \item In dieser Theorie würden die modifizierten Maxwell-Gleichungen nicht postuliert, sondern als effektive Feldgleichungen aus der Mittelung über die diskrete Raumzeit-Struktur emergieren.
    \item Dies wäre analog zur Emergenz der Hydrodynamik aus der Atomphysik.
\end{itemize}

\section{Zusammenfassung}
\begin{itemize}
    \item \textbf{Analoge WDBT:} Führt zu modifizierten Maxwell-Gleichungen mit Quelltermen proportional zu $\nabla Q$.
    \item \textbf{Digitale WDBT:} Ist eine Hypothese, die beide Anteile (\gls{wed} und $Q$) aus einem gemeinsamen Prinzip (fraktaler Raum) ableiten will.
\end{itemize}


\chapter{Die Emergenz der Quantenmechanik}
\section{Die fundamentale Bewegungsgleichung der WDBT}
Die analoge \gls{wdbt} postuliert für ein Teilchen der Masse $m$ die folgende vollständige Kraftgleichung. Für ein neutrales Teilchen, auf das nur ein konservatives Potential $V(\vec{x})$ und das
Quantenpotential $Q$ wirken, vereinfacht sich diese zu:

\begin{equation}
    \label{eq:deterministische_bewegungsgleichung}
    m \frac{d^2\vec{x}}{dt^2} = -\vec{\nabla} V - \vec{\nabla} Q
\end{equation}

wobei das Bohm'sche Quantenpotential $Q$ definiert ist als:

\begin{equation}
    Q = -\frac{\hbar^2}{2m} \frac{\nabla^2 R}{R}
\end{equation}

Hier ist $R = R(\vec{x},t)$ die Amplitude der sogenannten \enquote{Führungswelle} oder \enquote{pilot wave}, und $\rho = R^2$ ist die Wahrscheinlichkeitsdichte des Teilchensensembles. Gleichung
(\refeq{eq:deterministische_bewegungsgleichung}) ist die \textbf{deterministische Bewegungsgleichung} eines Teilchens in der \gls{wdbt}.

\section{Die Madelung-Transformation: Von der Teilchentrajektorie zur Feldbeschreibung}
Um von der Beschreibung einzelner Teilchentrajektorien zur Beschreibung eines Ensembles (einem \enquote{Fluid}) überzugehen, führen wir die Madelung-Transformation durch. Wir formulieren eine komplexe
Wellenfunktion $\psi(\vec{x}, t)$, deren Phase $S$ mit dem Impuls des Teilchens und deren Betragsquadrat mit der Dichte verbunden ist:

\begin{equation}
    \psi(\vec{x}, t) = R(\vec{x}, t)  e^{i S(\vec{x}, t) / \hbar}
\end{equation}

Dabei ist:

\begin{itemize}
    \item $R(\vec{x},t)$: Reelle Amplitude ($\rho = R^2$ ist die Wahrscheinlichkeitsdichte).
    \item $S(\vec{x},t)$: Reelle Phasenfunktion (entspricht der Wirkung bzw. dem Impulspotential).
\end{itemize}

Unser Ziel ist es nun zu zeigen, dass die Bewegungsgleichung (\refeq{eq:deterministische_bewegungsgleichung}) und die Annahme einer Kontinuitätsgleichung für die Dichte $\rho$ äquivalent zur
Schrödinger-Gleichung für $\psi$ sind.

\section{Herleitung der Kontinuitätsgleichung}
Der Impuls $\vec{p}$ eines Teilchens auf seiner Trajektorie ist in der \gls{wdbt} durch den Gradienten der Phasenfunktion $S$ gegeben:

\begin{equation}
    \label{eq:impuls}
    \vec{p} = m \vec{v} = \vec{\nabla} S
\end{equation}

Die Erhaltung der Wahrscheinlichkeit (Teilchen können nicht einfach verschwinden oder erzeugt werden) führt auf eine Kontinuitätsgleichung für die Dichte $\rho$:

\begin{equation}
    \frac{\partial \rho}{\partial t} + \vec{\nabla} \cdot (\rho \vec{v}) = 0
\end{equation}

Setzen wir $\rho = R^2$ und $\vec{v} = \frac{\vec{\nabla} S}{m}$ ein, erhalten wir:

\begin{equation}
    \frac{\partial (R^2)}{\partial t} + \vec{\nabla} \cdot \left( R^2 \frac{\vec{\nabla} S}{m} \right) = 0
\end{equation}

Diese Gleichung beschreibt die zeitliche Entwicklung der Dichteverteilung des Ensembles.

\section{Herleitung der Hamilton-Jacobi-Gleichung}
Wir beginnen nun mit der Newtonschen Bewegungsgleichung der \gls{wdbt} (Gl. \refeq{eq:deterministische_bewegungsgleichung}) und formen sie schrittweise um.

\begin{equation}
    m \frac{d^2\vec{x}}{dt^2} = -\vec{\nabla} V - \vec{\nabla} Q \tag{3.1}
\end{equation}

Die substantielle Zeitableitung der Geschwindigkeit $\vec{v} = \frac{d \vec{x}}{dt}$ ist:

\begin{equation}
    \frac{d\vec{v}}{dt} = \frac{\partial \vec{v}}{\partial t} + (\vec{v} \cdot \vec{\nabla}) \vec{v}
\end{equation}

Mit $\vec{v} = \frac{\vec{\nabla} S}{m}$ (aus Gl. \refeq{eq:impuls}) wird der Beschleunigungsterm ($\vec{v} \cdot \vec{\nabla})$ zu:

\begin{equation}
    (\vec{v} \cdot \vec{\nabla}) \vec{v} = \frac{1}{m^2} \left[ (\vec{\nabla} S \cdot \vec{\nabla}) \vec{\nabla} S \right]
\end{equation}

Eine nützliche Vektoridentität hilft uns, diesen Term umzuschreiben:

\begin{equation}
    (\vec{\nabla} S \cdot \vec{\nabla}) \vec{\nabla} S = \frac{1}{2} \vec{\nabla} (\vec{\nabla} S \cdot \vec{\nabla} S) = \frac{1}{2} \vec{\nabla} (\left|\vec{\nabla} S \right|^2)
\end{equation}

Somit wird die linke Seite der Bewegungsgleichung zu:

\begin{equation}
    m \frac{d\vec{v}}{dt} = m \left( \frac{\partial \vec{v}}{\partial t} \right) + \frac{1}{2m} \vec{\nabla} (\left| \vec{\nabla} S \right|^2)
\end{equation}

\chapter{Die konvergente Emergenz der ART}
 Wie die \gls{art} zur \gls{wdbt} strebt.

\section{Die unvollständige ART: Das Singularitäten-Problem}

\begin{itemize}
    \item \textbf{Standard-\gls{art}:} $G_{\mu\nu} = 8\pi G T_{\mu\nu}$
    \item Diese Gleichung führt unter generischen Bedingungen zu Singularitäten (Schwarze Löcher, Urknall).
    \item \textbf{Interpretation:} Dies ist kein physikalisches, sondern ein theoretisches Versagen. Die \gls{art} ist an ihren Grenzen unvollständig.
\end{itemize}

\section{Schritt 1 der Emergenz: Vervollständigung der ART durch die DBT}

\begin{itemize}
    \item Die naheliegendste Erweiterung zur Vermeidung von Singularitäten ist die Einführung des \textbf{Bohm'schen Quantenpotentials $Q$}.
    \item Die vervollständigte Einstein-Gleichung lauten nun:
    \begin{equation}
        G_{\mu\nu} = 8\pi G (T_{\mu\nu} + Q_{\mu\nu})
    \end{equation}
    \item \textbf{Konsequenzen dieser Erweiterung:}
    \begin{enumerate}
        \item \textbf{Singularitätenfreiheit:} $Q$ wirkt repulsiv und verhindert die Bildung von Punkt-Singularitären. Resultat: Big Bounce statt Big Bang.
        \item \textbf{Nicht-Lokalität:} Das Quantenpotential $Q$ ist fundamental nicht-lokal. Diese Eigenschaft wird nun in die Gravitation selbst eingebracht.
        \item \textbf{Angleich an die WDBT:} Die erweiterte \gls{art} gewinnt zentrale Eigenschaften der \gls{wdbt}: Deterministische Trajektorien, Singularitätenfreiheit und Nicht-Lokalität.
    \end{enumerate}
\end{itemize}

\section{Schritt 2 der Emergenz: Vervollständigung durch Berücksichtigung der Nicht-Lokalität}

\begin{itemize}
    \item Die \gls{art} (selbst die erweiterte Version) ist eine lokale Feldtheorie. Die ursprüngliche \gls{wg} der \gls{wdbt} ist jedoch instantan und nicht-lokal.
    \item Auch diese Eigenschaft kann in die \gls{art} \enquote{importiert} werden, indem man die \textbf{Lösungen der Einstein-Gleichungen} betrachtet. Die \textbf{Gravitationswelle} (Ausbreitung mit $c$) ist nur eine spezielle Lösung.
    \item Die \textbf{instanteane Krümmung} (die die Bewegung der Planeten determiniert) ist eine andere. In einer vollständigen Behandlung muss die \gls{art} beide Beschreibungen gleichberechtigt zulassen können – die retardierten und die avancierten Lösungen (Wheeler-Feynman-Ansatz).
    \item \textbf{Resultat:} Die so vervollständigte \gls{art} wird ebenfalls \textbf{nicht-lokal und lokal} gleichzeitig, genau wie die \gls{wdbt}. Die Retardierung der Wellen ist ein Spezialfall, die Instantaneität der Felder die Regel.
\end{itemize}

\section{Die konvergenten Theorien: ART+ vs. WDBT}
Durch diese beiden Schritte der Vervollständigung nähern sich die erweiterte \gls{art} (\gls{art}+) und die \gls{wdbt} konzeptionell stark an:

\begin{table}[h]
\centering
\begin{tabular}{|p{0.25\textwidth}|p{0.3\textwidth}|p{0.3\textwidth}|}
\hline
\textbf{Eigenschaft} & \textbf{ART+ (Vervollständigt)} & \textbf{WDBT (Fundamental)} \\
\hline
\textbf{Singularitäten} & Keine (Big Bounce) & Keine (Big Bounce) \\
\hline
\textbf{Nicht-Lokalität} & Ja (durch $Q_{\mu\nu}$ \& Feldlösungen) & Ja (fundamental durch WG) \\
\hline
\textbf{Determinismus} & Ja (durch $Q$) & Ja (fundamental) \\
\hline
\textbf{Urknall} & Nein & Nein \\
\hline
\textbf{Lichtablenkung} & Frequenzunabhängig & Frequenzabhängig ($\Delta \phi(f)$) \\
\hline
\textbf{Grundlage} & Geometrische Beschreibung & Dyn. Wechselwirkung \\
\hline
\end{tabular}
\caption{Vergleich der vervollständigten ART (ART+) mit der fundamentalen WDBT}
\end{table}

\textbf{Die beiden Theorien scheinen zu konvergieren!}

\section{Der experimentelle Entscheid: Die frequenzabhängige Lichtablenkung}
Trotz der konzeptionellen Konvergenz bleibt ein \textbf{entscheidender, experimentell überprüfbarer Unterschied}:

\begin{itemize}
    \item \textbf{\gls{art}+:} Basiert letztlich auf einer \textbf{geometrischen} Beschreibung. Die Lichtablenkung erfolgt rein geometrisch und ist daher \textbf{frequenzunabhängig}.
    \item \textbf{\gls{wdbt}:} Basiert auf einer dynamischen Wechselwirkung (Weber-Kraft). Die Lichtablenkung ist eine echte Kraftwirkung und daher frequenzabhängig ($\Delta \Phi(f)$).
\end{itemize}

\textbf{Diese Abweichung ist der Lackmustest.} Welche der beiden konvergenten Beschreibungen ist die fundamentalere?

\begin{itemize}
    \item Misst man \textbf{keine} Frequenzabhängigkeit, so ist die geometrische Beschreibung der \gls{art}+ ausreichend.
    \item Misst man \textbf{eine} Frequenzabhängigkeit, ist dies der schlüssige Beweis für die Richtigkeit der dynamischen Grundlage der \gls{wdbt}.
\end{itemize}

\section{Fazit: WDBT als fundamentale Ur-Theorie}
Die Suche nach einer konsistenten Erweiterung der \gls{art} führt also in eine Richtung, die der \gls{wdbt} erstaunlich ähnlich sieht. Dies ist kein Zufall, sondern ein Indiz dafür, dass die \gls{wdbt}
die korrekte fundamentale Ur-Theorie ist.

Die \gls{wdbt} liefert nicht nur die konsistenteste Beschreibung, sondern auch die scharfste, überprüfbare Vorhersage ($\Delta \Phi(f)$), um sich endgültig von allen abgeleiteten effektiven
Theorien (wie der \gls{art}) zu unterscheiden.

\chapter{The Emergence of Quantum Electrodynamics}
\section{The Task of Reduction}
\gls{qed} describes the interaction between light and matter with unparalleled precision. Its core is the concept of quantum fields and virtual particles. This chapter shows how the seemingly abstract concepts of \gls{qed} – field quantization, Feynman diagrams, and renormalization – emerge from the deterministic, particle-based \gls{wdbt}. The strategy is not a 1:1 reconstruction, but the derivation of the operational core statements of \gls{qed} from the first principles of \gls{wdbt}.

\section{The Complete Force Equation of WDBT}
The fundamental equation for a charged particle (mass $m$, charge $q$) in \gls{wdbt} is the extension of the Lorentz force by the quantum potential $Q$:

\begin{equation}
    \label{eq:kraft_wdbt_em}
    m \frac{d^2\vec{x}}{dt^2} = q(\vec{E} + \vec{v} \times \vec{B}) - \nabla Q
\end{equation}

with:

\begin{equation}
    \label{eq:quantenpotential_wdbt_em}
    Q = -\frac{\hbar^2}{2m} \frac{\nabla^2 \sqrt{\rho}}{\sqrt{\rho}}
\end{equation}

The fields $\vec{E}$ and $\vec{B}$ are effective descriptions of the averaged Weber interaction with all other charges in the universe, as derived in Chapter \ref{ch:maxwell}. Equation (\refeq{eq:kraft_wdbt_em}) is \textbf{deterministic} and describes a well-defined trajectory.

\newpage
\section{The Emergence of the Schrödinger and Maxwell Equations}
To describe the statistics of an ensemble of identical systems (e.g., many atoms in the same state), the Madelung transformation is performed (cf. Section \ref{sec:madelung}). We introduce a complex wave function $\psi(\vec{x},t)=R(\vec{x},t)e^{iS(\vec{x},t)/\hbar}$, for which:

\begin{enumerate}
    \item $\rho=\left| \psi \right|^2$ (probability density)
    \item $\vec{v} = \frac{1}{m}\vec{\nabla}S$ (particle velocity)
\end{enumerate}

As shown in Section \ref{sec:schrödinger_gleichung}, the equation of motion (\refeq{eq:kraft_wdbt_em}) is equivalent to the time-dependent Schrödinger equation for $\psi$:

\begin{equation}
    i\hbar \frac{\partial \psi}{\partial t} = \left( -\frac{\hbar^2}{2m} \nabla^2 + V + q\phi \right) \psi
\end{equation}

where the vector potential $\vec{A}$ enters via $\vec{B} = \vec{\nabla} \times \vec{A}$ into the minimal coupling. In parallel, the classical Maxwell equations emerge from Weber electrodynamics through the continuum limit (Section \ref{sec:kontinuumslimes}). Thus, \textbf{non-relativistic quantum mechanics} and \textbf{classical electrodynamics} are already derived as effective levels of description from \gls{wdbt}.

\section{Photons as Excitations of the Quantum Vacuum}
Conventional \gls{qed} postulates the quantization of the electromagnetic field. In \gls{wdbt}, this step arises naturally from considering the quantum potential of the vacuum.

The vacuum in \gls{wdbt} is not empty space but a \textbf{quantum medium} with a ground fluctuation, described by a vacuum wave function $\psi_\text{Vak}$. Its associated quantum potential $Q_\text{Vak}$ acts on all particles. The excitations of this medium – described by standing waves in a box with specific frequencies – correspond to photons.

The energy of an excitation of frequency $\omega$ is:

\begin{equation}
    E = \hbar \omega
\end{equation}

This relation emerges from the scaling of the quantum potential $Q$ with $\hbar^2$ and the dispersion relation of \gls{wed}. The \textbf{field operators} of \gls{qed} (creation/annihilation operators) are therefore mathematical auxiliary quantities for describing these excited modes of the quantum vacuum, not fundamental entities.

\section{Derivation of Feynman Rules from Non-Local Weber Interaction}
The strength of \gls{qed} lies in perturbation theory and \textbf{Feynman diagrams}. These emerge in \gls{wdbt} from \textbf{averaging over all possible non-local interaction paths}.

In \gls{wed}, the force between two charges acts instantaneously and depends on velocity and acceleration. The probability amplitude for a particle to go from $A$ to $B$ must sum over all possible trajectories, each influenced by its own guiding wave and its quantum potential.

\subsubsection{Derivation Steps:}
\begin{enumerate}
    \item The \textbf{action} $S$ for a Weber interaction path is defined.
    \item The \textbf{amplitude} for a path is proportional to $e^{iS/\hbar}$.
    \item The \textbf{path integral} (sum over all paths) is introduced.
    \item The \textbf{Weber force} is expanded into a perturbation series. Each term in this series corresponds to an \textbf{elementary Weber interaction vertex}.
    \item The \textbf{propagators} (e.g., $\frac{i}{p^2 - m^2 + i\epsilon}$ for an electron) describe the propagation between two interactions under the influence of the free quantum potential.
    \item The \textbf{Feynman rules} emerge as an efficient bookkeeping method for calculating the total amplitude, considering all possible paths and vertices.
\end{enumerate}

A Feynman diagram is thus not a representation of virtual particles, but a \textbf{graphical representation of the averaging over non-local Weber interaction terms} in the perturbation series.

\section{Regularization by the Quantum Potential}
Conventional \gls{qed} suffers from divergent integrals (infinities). Renormalization removes these by redefining mass and charge. In \gls{wdbt}, these divergences are artifactual and are avoided from the outset.

The reason is the \textbf{quantum potential} $Q$. Since it depends on the density $\rho$ and this never becomes point-like for a particle (the guiding wave always has a finite extent), all interactions are \textbf{regularized}. The \enquote{bare} charge and mass are finite because the self-energy of a particle is limited by the interaction with its own, extended guiding wave. Renormalization thus emerges as the effective calculus to extract the finite, observable quantities from the underlying regularized \gls{wdbt} dynamics.

\section{Lamb Shift and g-Factor}
The successes of \gls{qed} must be reproduced by \gls{wdbt}. The calculation of the Lamb shift and the anomalous magnetic moment of the electron follows a modified but conceptually clear path in \gls{wdbt}.

\begin{itemize}
    \item \textbf{Lamb Shift:} It results from the interaction of the electron with the fluctuating quantum potential of the vacuum ($Q_\text{Vak}$). The \gls{wdbt} prediction contains an additional term compared to \gls{qed}:
    \begin{equation}
        \label{eq:lamb_shift}
        \Delta E_{\text{Lamb}}^{\text{WDBT}} = \Delta E_{\text{QED}} + \frac{e^2 \hbar}{4\pi \epsilon_0 m_e^2 c^3} \langle r \rangle
    \end{equation}
    This term is small but in principle measurable and represents a falsifiable deviation from \gls{qed}. Appendix \ref{att:lamb_shift} shows more details.
    \item \textbf{g-Factor:} The anomalous g-factor emerges from the non-local coupling of the electron spin (modeled as Zitterbewegung in \gls{wdbt}) to the cosmic electromagnetic background. The prediction agrees with that of \gls{qed}, as both lead to the same effective algebra of spin interaction.
\end{itemize}

\section{QED as a Triumphant Effective Field Theory}
Quantum electrodynamics emerges completely from the Weber-De Broglie-Bohm theory as its effective, statistical description in the continuum limit.

\begin{itemize}
    \item \textbf{Fields} emerge from the averaging of instantaneous Weber interactions.
    \item \textbf{Field quantization} emerges from the excitations of the quantum vacuum medium.
    \item \textbf{Feynman diagrams} emerge from the perturbation expansion of the non-local interaction.
    \item \textbf{Renormalization} emerges as a procedure to extract observable quantities from the regularized \gls{wdbt}.
\end{itemize}

\gls{wdbt} thus resolves the conceptual problems of \gls{qed} (infinities, virtuality, ontology) while preserving all its empirical success. It places \gls{qed} on a solid, ontologically clear foundation of direct particle interactions and deterministic dynamics. \gls{qed} is not wrong, but an incomplete description of a deeper, non-local reality.

\appendix
\chapter{Anhang}
\section{Der Aharonov-Bohm-Effekt}
\label{sec:aharonov-bohm}

Der \textbf{Aharonov-Bohm-Effekt} (AB-Effekt) ist ein grundlegendes Quantenphänomen, das zeigt, dass elektromagnetische Potentiale ($\vec{A}$, $\Phi$) eine direkte physikalische
Wirkung auf Quantenteilchen haben, selbst in Regionen wo die Felder ($\vec{E}$, $\vec{B}$) null sind.

\subsection{Experimentelle Anordnung}
Ein Elektronenstrahl wird in zwei Pfade aufgeteilt, die eine Region mit magnetischem Fluss $\Phi$ umschließen.

\subsection{Theoretische Beschreibung}
Die Wellenfunktion $\psi$ eines Teilchens mit Ladung $q$ wird durch das Vektorpotential $\vec{A}$ modifiziert:

\begin{equation}
\psi \rightarrow \psi \cdot \exp\left(i\frac{q}{\hbar}\int \vec{A}\cdot d\vec{l}\right)
\end{equation}

Die Phasendifferenz zwischen den beiden Pfaden beträgt:

\begin{equation}
\Delta\phi = \frac{q}{\hbar}\oint \vec{A}\cdot d\vec{l} = \frac{q}{\hbar}\Phi_B
\end{equation}

\subsection{Physikalische Bedeutung}
\begin{itemize}
\item \textbf{Nicht-Lokalität}: Quantenteilchen \enquote{spüren} $\vec{A}$ auch in feldfreien Regionen
\item \textbf{Topologische Invariante}: Die Phase hängt nur vom eingeschlossenen Fluss $\Phi_B$ ab
\item \textbf{Paradigmenwechsel}: Widerlegt die klassische Annahme, dass nur $\vec{E}$ und $\vec{B}$ physikalisch relevant sind
\end{itemize}

\subsection{Experimentelle Bestätigung}
\begin{itemize}
\item Theoretische Vorhersage: Aharonov \& Bohm (1959)
\item Erste Experimente: Chambers (1960), Tonomura et al. (1982)
\item Moderne Anwendungen: Quanteninterferometer, topologische Quantenmaterialien
\end{itemize}

\section{Bellsche Ungleichungen}
\label{sec:bell}

Die \textbf{Bellsche Ungleichung} (1964) ist ein zentrales Ergebnis der Quantenphysik, das zeigt, dass keine lokale Theorie mit verborgenen Variablen die Vorhersagen der Quantenmechanik reproduzieren kann.

\subsection{Theoretische Formulierung}
Für ein verschränktes Teilchenpaar (z.B. Photonen mit Spin- oder Polarisationskorrelation) gilt die CHSH-Ungleichung:

\begin{equation}
S = |E(a,b) - E(a,b')| + |E(a',b) + E(a',b')| \leq 2
\end{equation}

wobei $E(\theta_1, \theta_2)$ die Korrelationsfunktion der Messungen bei Winkeln $\theta_1$ und $\theta_2$ ist.

\subsection{Quantenmechanische Vorhersage}
Die Quantenmechanik erlaubt für bestimmte Winkelkombinationen:

\begin{equation}
S_{\text{QM}} = 2\sqrt{2} \approx 2.828 > 2
\end{equation}

was die Bell-Ungleichung verletzt.

\subsection{Experimentelle Bestätigung}
\begin{itemize}
\item Erste Tests: Alain Aspect (1982) mit Photonenpaaren
\item Loophole-free Experimente: Hensen et al. (2015), Zeilinger-Gruppe (2017)
\item Heutige Anwendungen: Quantenkryptographie (BB84-Protokoll)
\end{itemize}

\subsection{Interpretation}
\begin{itemize}
\item Widerlegung lokaler realistischer Theorien (Einstein-Podolsky-Rosen-Paradoxon)
\item Bestätigung der Quantenverschränkung als physikalische Realität
\item Grundlage für Quanteninformationstechnologien
\end{itemize}

\newpage
\section{Exakte Herleitung der Weber-Gravitationsbahngleichung}
\label{sec:exakte_herleitung}

In diesem Anhang leiten wir die Bahngleichung der Weber-Gravitation (WG) streng her, ohne die in Kapitel~3 verwendeten Vereinfachungen. Die volle Bewegungsgleichung wird bis zur Ordnung $\mathcal{O}(c^{-4})$ entwickelt.

\subsection{Ausgangsgleichungen}
Die Weber-Gravitationskraft lautet:
\begin{equation}
\vec{F}_{\text{WG}} = -\frac{GMm}{r^2} \left(1 - \frac{\dot{r}^2}{c^2} + \beta \frac{r\ddot{r}}{c^2}\right)\hat{\vec{r}}
\end{equation}
Für Planetenbahnen setzen wir $\beta = 0.5$ (siehe Abschnitt~3.1.2). Die Bewegungsgleichung in Polarkoordinaten ist:
\begin{equation}
m\left(\ddot{r} - r\dot{\phi}^2\right) = -\frac{GMm}{r^2}\left(1 - \frac{\dot{r}^2}{c^2} + \frac{r\ddot{r}}{2c^2}\right)
\end{equation}

\subsection{Transformation auf Winkelkoordinaten}
Mit dem Drehimpuls $h = r^2\dot{\phi} = \text{const.}$ und der Substitution $u = 1/r$ erhalten wir:
\begin{align}
\dot{r} &= -h\frac{du}{d\phi} \\
\ddot{r} &= -h^2u^2\frac{d^2u}{d\phi^2}
\end{align}
Einsetzen in die Bewegungsgleichung ergibt die exakte Differentialgleichung:
\begin{equation}
\frac{d^2u}{d\phi^2} + u = \frac{GM}{h^2}\left[1 - h^2\left(\frac{du}{d\phi}\right)^2 + \frac{h^2u}{2}\frac{d^2u}{d\phi^2}\right]
\end{equation}

\subsection{Störungsrechnung}
Wir entwickeln die Lösung als Reihe:
\begin{equation}
u(\phi) = u_0(\phi) + \frac{GM}{c^2h^2}u_1(\phi) + \mathcal{O}(c^{-4})
\end{equation}
wobei $u_0$ die Newtonsche Lösung ist:
\begin{equation}
u_0(\phi) = \frac{GM}{h^2}(1 + e\cos\phi)
\end{equation}

Die Störungsgleichung für $u_1$ lautet:
\begin{equation}
\frac{d^2u_1}{d\phi^2} + u_1 = \frac{G^2M^2e^2}{h^4}\left(\sin^2\phi + \frac{1 + e\cos\phi}{2}\cos\phi\right)
\end{equation}

\subsection{Lösung der Störungsgleichung}
Die allgemeine Lösung besteht aus homogenen und partikulären Anteilen:
\begin{equation}
u_1(\phi) = \frac{G^2M^2e}{8h^4}\left[3e\phi\sin\phi + (4 + e^2)\cos\phi\right]
\end{equation}

\subsection{Periheldrehung}
Der nicht-periodische Term $\propto \phi\sin\phi$ führt zur Perihelverschiebung:
\begin{equation}
\Delta\phi = \frac{6\pi G^2M^2}{c^2h^4} = \frac{6\pi GM}{c^2a(1 - e^2)}
\end{equation}
Dies stimmt exakt mit den Beobachtungen und der ART überein.

\subsection{Kritische Diskussion}
\begin{itemize}
\item Die Wahl $\beta = 0.5$ ist essentiell - andere Werte führen zu falschen Vorhersagen
\item Die Vernachlässigung von $\dot{r}^2$ ist nur für $e \ll 1$ gerechtfertigt
\item Die DBT-Kompensation der $\mathcal{O}(c^{-4})$-Terme (Gl. \refeq{eq:shapiro}) stellt die Bahnstabilität sicher
\end{itemize}

Diese Herleitung zeigt, dass die WG nur in Kombination mit der DBT eine konsistente Alternative zur ART darstellt.

\section{Potentialunterschiede in Weber-Theorien}
\label{sec:weber_potentials}

\subsection{Weber-Elektrodynamik}
Die Weber-Kraft zwischen zwei Ladungen $q_1$ und $q_2$ lautet:
\[
\vec{F}_{\text{Weber-EM}} = \frac{q_1 q_2}{4\pi\epsilon_0 r^2} \left(1 - \frac{\dot{r}^2}{c^2} + \beta_{\text{EM}} \frac{r\ddot{r}}{c^2}\right)\hat{r}, \quad \beta_{\text{EM}} = 2
\]
\begin{itemize}
\item \textbf{Nicht-Konservativität}: Die Kraft enthält explizit Geschwindigkeits- ($\dot{r}^2$) und Beschleunigungsterme ($\ddot{r}$), was die Existenz eines klassischen Potentials $\Phi$ verhindert.
\item \textbf{Pseudo-Potential}: Nur für $\ddot{r} = 0$ lässt sich ein energieähnlicher Ausdruck ableiten:
\[
E_{\text{Weber-EM}} = \frac{1}{2}m_1v_1^2 + \frac{1}{2}m_2v_2^2 + \underbrace{\frac{q_1 q_2}{4\pi\epsilon_0 r}\left(1 - \frac{\dot{r}^2}{2c^2}\right)}_{\text{Kein echtes Potential}}
\]
\end{itemize}

\subsection{Weber-Gravitation}
Das Gravitationspotential einer Masse $M$ lautet:
\[
\Phi_{\text{WG}}(r) = -\frac{GM}{r}\left(1 + \frac{v^2}{2c^2} + \beta_{\text{G}} \frac{r\ddot{r}}{2c^2}\right), \quad \beta_{\text{G}} = 
\begin{cases}
0.5 & \text{(Massen)} \\
1 & \text{(Photonen)}
\end{cases}
\]
\begin{itemize}
\item \textbf{Konservativität}: Trotz $\ddot{r}$-Term ist $\Phi_{\text{WG}}$ wohldefiniert, da die Gravitation eine rein anziehende Wechselwirkung ist.
\item \textbf{Physikalische Begründung}: Der Term $\beta_{\text{G}}\frac{r\ddot{r}}{2c^2}$ ist notwendig, um die Periheldrehung des Merkur ($\beta_{\text{G}} = 0.5$) und Lichtablenkung ($\beta_{\text{G}} = 1$) zu reproduzieren.
\end{itemize}

\subsection*{Zusammenfassung}
\begin{tabular}{ll}
\textbf{Weber-Elektrodynamik} & \textbf{Weber-Gravitation} \\ \hline
$\beta_{\text{EM}} = 2$ (Lorentz-Kraft) & $\beta_{\text{G}} = 0.5/1$ (ART-Konsistenz) \\
Kein allgemeines Potential & Wohldefiniertes Potential \\
Nicht-konservativ (Strahlungsverluste) & Konservativ \\
\end{tabular}

\section{Herleitung der Periodendauer eines Planeten in der WDBT}
\label{sec:periodendauer}

\subsection*{Ausgangsgleichungen}
Für einen Planeten mit großer Halbachse \( a \) und Exzentrizität \( e \) lautet die Bahngleichung in der WDBT (Gl. \refeq{eq:weber_r_1_ordnung}):

\begin{equation}
r(\phi) = \frac{a(1-e^2)}{1 + e \cos(\kappa \phi)}
\end{equation}

mit der Periheldrehungskonstante:

\begin{equation}
\kappa = \sqrt{1 - \frac{6GM}{c^2 a(1-e^2)}}
\end{equation}

\subsection*{Energieerhaltung}
Die Gesamtenergie im System (kinetisch + Weber-Potential) ist:

\begin{equation}
E = \frac{1}{2}mv^2 - \frac{GMm}{r}\left(1 + \frac{v^2}{2c^2}\right)
\end{equation}

\subsection*{Kreisbahnapproximation}
Für näherungsweise Kreisbahnen (\( e \approx 0 \)) gilt:
\begin{itemize}
\item Momentaner Abstand \( r \approx a \) (konstant)
\item Winkelgeschwindigkeit \( \omega = \frac{d\phi}{dt} = \text{konstant} \)
\item Bahngeschwindigkeit \( v = a\omega \)
\end{itemize}

\subsection*{Bewegungsgleichung}
Die radiale Kraftbilanz ergibt:

\begin{equation}
m a \omega^2 = \frac{GMm}{a^2}\left(1 + \frac{a^2 \omega^2}{2c^2}\right)
\end{equation}

\subsection*{Lösung für die Winkelgeschwindigkeit}
Umstellung liefert:

\begin{align}
\omega^2 a^3 &= GM \left(1 + \frac{a^2 \omega^2}{2c^2}\right) \\
\omega^2 \left(a^3 - \frac{GM a^2}{2c^2}\right) &= GM \\
\omega^2 &= \frac{GM}{a^3} \left(1 - \frac{GM}{2a c^2}\right)^{-1} \\
&\approx \frac{GM}{a^3} \left(1 + \frac{GM}{2a c^2}\right) \quad \text{(Taylor-Entwicklung)}
\end{align}

\subsection*{Periodendauer}
Mit \( T = \frac{2\pi}{\omega} \) ergibt sich:

\begin{equation}
T \approx 2\pi \sqrt{\frac{a^3}{GM}} \left(1 - \frac{GM}{4a c^2}\right)
\end{equation}

\subsection*{Exakte Lösung für elliptische Bahnen}
Die vollständige Lösung unter Berücksichtigung der Exzentrizität \( e \) lautet:

\begin{equation}
\boxed{T = 2\pi \sqrt{\frac{a^3}{GM}} \left[1 - \frac{3GM}{4c^2 a(1-e^2)}\right]}
\end{equation}

\subsection*{Physikalische Interpretation}
\begin{itemize}
\item Der Term \( 2\pi \sqrt{a^3/GM} \) entspricht dem klassischen Kepler'schen Ergebnis
\item Die Korrektur \( -\frac{3GM}{4c^2 a(1-e^2)} \) kommt durch:
  \begin{enumerate}
  \item Den Geschwindigkeitsterm \( \frac{v^2}{c^2} \) in der Weber-Gravitation
  \item Die Periheldrehung \( \kappa \) der WDBT-Bahngleichung
  \end{enumerate}
\item Für Merkur (\( a \approx 5.79 \times 10^{10} \) m, \( e \approx 0.206 \)) beträgt die Korrektur \( \approx 7.3 \times 10^{-8} \)
\end{itemize}


\backmatter
\printbibliography[title=Literaturverzeichnis]
\glswritefiles
\printglossary[title=Glossar]
\printglossary[type=acronym, title=Abkürzungen]

\end{document}
