\documentclass[11pt, a5paper, twoside, openright]{book}
\usepackage[ngerman]{babel}
\usepackage[T1]{fontenc}
\usepackage[utf8]{inputenc}
\usepackage{lmodern}
\usepackage{microtype}
\usepackage{csquotes}
\usepackage{verbatim}  % Im Kopf des Dokuments einfügen
\usepackage{geometry}
\usepackage{fancyhdr}
\usepackage{amsmath, amssymb, amsthm}  % Mathe
\usepackage{mathtools}                 % \coloneqq, \xrightarrow
\usepackage{bm}                        % Fette Symbole (\bm{B} für Magnetfeld)
\usepackage{siunitx}                   % \SI{1.23}{\meter\per\second}
\usepackage{graphicx}                  % \includegraphics
\usepackage{subcaption}                % Unterabbildungen
\usepackage{booktabs}                  % Professionelle Tabellen
\usepackage{tikz}                      % Für Diagramme
\usepackage{xcolor}                    % Farbige Tabellenzellen
\usepackage[
    backend=biber,
    style=phys,         % APS-Zitierstil (für Physik)
    sorting=nyt,        % Sortierung: Name, Jahr, Titel
]{biblatex}
\usepackage[acronym, toc]{glossaries}
\usepackage{hyperref}
\usepackage{parskip}
\usepackage{pgfplots}
\usepackage{glossaries}
\makeglossaries
\geometry{
    a4paper,
    top=25mm,
    inner=30mm,    % Bundsteg (größerer Rand für Buchbindung)
    outer=25mm,
    bottom=30mm,
    headheight=15pt,
}

\pagestyle{fancy}
\fancyhf{}
\fancyhead[LE,RO]{\thepage}
\fancyhead[RE]{\leftmark}    % Kapitelname (gerade Seiten)
\fancyhead[LO]{\rightmark}   % Abschnittname (ungerade Seiten)
\renewcommand{\headrulewidth}{0.4pt}

\theoremstyle{definition}
\newtheorem{definition}{Definition}[chapter]
\newtheorem{law}{Physikalisches Gesetz}[chapter]
\theoremstyle{plain}
\newtheorem{theorem}{Theorem}[chapter]
\newtheorem{lemma}[theorem]{Lemma}
\theoremstyle{remark}
\newtheorem{remark}{Bemerkung}[chapter]

\hypersetup{
    colorlinks=true,
    linkcolor=blue,
    citecolor=black,
    urlcolor=black,
    pdftitle={Emergenz der Kosmologie: Die WDBT als Ur-Theorie},
    pdfauthor={Dipl.-Ing. (FH) Michael Czybor},
}

\addbibresource{literatur.bib}  % Ihre .bib-Datei
\makeglossaries

\setlength{\headheight}{26.76852pt}
\definecolor{quantenblau}{RGB}{0, 100, 200}
\definecolor{weberrot}{RGB}{180, 20, 60}
\definecolor{hintergrund}{RGB}{20, 20, 40}
\usetikzlibrary{shapes, calc, 3d}
\pgfplotsset{compat=1.18} % Aktuelle Version verwenden

\newacronym{qm}{QM}{Quantum Mechanics}
\newacronym{art}{ART}{General Theory of Relativity}
\newacronym{srt}{SRT}{Special Theory of Relativity}
\newacronym{cmb}{CMB}{Cosmic Microwave Background}
\newacronym{qed}{QED}{Quantum Electrodynamics}
\newacronym{epr}{EPR Paradox}{Einstein-Podolsky-Rosen Paradox}
\newacronym{wg}{WG}{Weber Gravitation}
\newacronym{dbt}{DBT}{De Broglie-Bohm Theory}
\newacronym{wdbt}{WDBT}{Weber-De Broglie-Bohm Theory}
\newacronym{mt}{MT}{Maxwell Theory}
\newacronym{mhd}{MHD}{Magnetohydrodynamics}
\newacronym{wed}{WED}{Weber Electrodynamics}
\newacronym{eu}{EU}{Electric Universe}

\newglossaryentry{gls:quantenmechanik}
{
    name={Quantum Mechanics},
    description={Theory of matter and radiation at the atomic and subatomic level}
}
\newglossaryentry{gls:hamiltonian}
{
    name={\ensuremath{\mathcal{H}}},
    description={Hamiltonian operator, describes the total energy of a system},
    sort={hamiltonian}
}

\begin{document}

\frontmatter
\begin{tikzpicture}[remember picture, overlay]

  % Hintergrund (Dunkel mit fraktalem Gitter)
  \fill[hintergrund] (current page.south west) rectangle (current page.north east);
  \foreach \i in {0,10,...,360} {
    \draw[quantenblau!10, line width=0.1pt] 
      (current page.center) -- +(\i:5cm);
  }

  % Dodekaeder (abstrahiert)
  \node[rotate=25, scale=2, quantenblau!50] at (current page.center) {
    \begin{tikzpicture}[scale=0.3]
      \draw[quantenblau] (0:1) \foreach \a in {72,144,...,360} { -- (\a:1) } -- cycle;
      \foreach \a in {36,108,...,324} { \draw[quantenblau] (0,0) -- (\a:1.6); }
    \end{tikzpicture}
  };

  % Titeltext (mit Schatten-Effekt)
  \node[align=center, text=white, font=\sffamily\bfseries\Huge] 
    at ($(current page.center)+(0,3cm)$) {
    \textbf{Emergenz der Kosmologie}
  };
  \node[align=center, text=quantenblau!80, font=\sffamily\Large] 
    at ($(current page.center)+(0,1.8cm)$)
    {
        Die WDBT als Ur-Theorie
    };

  % Kernformeln (rechts unten)
  \node[align=left, anchor=south east, text=weberrot!70, font=\small] 
    at ($(current page.south east)+(-1cm,1cm)$) {
    $\displaystyle \vec{F}_{\text{WG}} = -\frac{GMm}{r^2}\left(1-\frac{\dot{r}^2}{c^2}+\beta\frac{r\ddot{r}}{c^2}\right)$
  };
  \node[align=left, anchor=north east, text=quantenblau!70, font=\small] 
    at ($(current page.south east)+(-1cm,3cm)$) {
    $\displaystyle Q = -\frac{\hbar^2}{2m}\frac{\nabla^2\sqrt{\rho}}{\sqrt{\rho}}$
  };

  % Autor (unten mittig)
  \node[align=center, text=white, font=\sffamily\large] 
    at ($(current page.south)+(0,1cm)$) {
    \textbf{Michael Czybor}
  };

  % Fraktale Dimension (links oben)
  \node[align=right, text=quantenblau!50, font=\small] 
    at ($(current page.north west)+(2cm,-1cm)$) {
    $D = \frac{\ln 20}{\ln(2+\phi)} \approx 2.71$
  };

\end{tikzpicture}

\title{Emergenz der Kosmologie\\Die WDBT als Ur-Theorie}
\author{Michael Czybor}
\date{\today}
\maketitle

\chapter*{Vorwort}

\begin{flushright}
    Michael Czybor \\
    \emph{Langenstein/AT, August 2025}
\end{flushright}

\tableofcontents
\listoffigures
\listoftables

\mainmatter
\chapter{Einleitung}
\section{Motivation}
Viele Schüler und Studierende erleben den Physikunterricht als frustrierend und unverständlich. Besonders die moderne Physik – mit der Allgemeinen Relativitätstheorie (ART)
und der Speziellen Relativitätstheorie (SRT) – wirkt oft unphysikalisch und voller logischer Widersprüche. Energie scheint unter bestimmten Bedingungen unendlich zu werden,
Überlichtgeschwindigkeit wird in manchen Fällen postuliert, obwohl sie eigentlich unmöglich sein soll, und Begriffe wie \enquote{dunkle Energie} oder \enquote{dunkle Materie} wirken wie
Platzhalter für unser Unverständnis.

Ein grundlegendes Problem liegt in den Widersprüchen zwischen ART und SRT. Die SRT baut auf Inertialsystemen auf, also Bezugssystemen, die sich gleichförmig und unbeschleunigt
bewegen. Doch laut ART gibt es keine perfekten Inertialsysteme, da jede Masse die Raumzeit krümmt und damit Beschleunigungen erzeugt. Schon allein dieser Widerspruch wirft
Fragen auf: Wenn Inertialsysteme streng genommen punktförmig sein müssten, um frei von jeder Krümmung zu sein, bräuchte man unendlich viele davon – und damit auch unendlich
viele verschiedene Lichtgeschwindigkeiten, da diese vom Bezugssystem abhängt.

Hinzu kommt, dass viele Konzepte der modernen Physik unserer Intuition widersprechen. Die Quantenmechanik verlangt, dass Teilchen gleichzeitig Wellen sind und erst durch
Beobachtung einen definierten Zustand annehmen. Die ART beschreibt eine gekrümmte Raumzeit, die sich kaum jemand wirklich vorstellen kann, und die SRT führt zu scheinbar
paradoxen Zeitdehnungen und Längenkontraktionen. Selbst der Urknall als Anfangspunkt des Universums wirft Fragen auf: Wie kann etwas aus dem Nichts entstehen? Warum gibt es
überhaupt eine Singularität, wenn doch unsere physikalischen Gesetze dort versagen?

All diese Punkte zeigen, dass die moderne Physik noch lange nicht abgeschlossen ist. Statt blind akzeptierte Theorien als absolute Wahrheit zu betrachten, sollten wir die
Widersprüche hinterfragen und nach konsistenteren Erklärungen suchen.


\appendix
\chapter{Anhang}
\section{Der Aharonov-Bohm-Effekt}
\label{sec:aharonov-bohm}

Der \textbf{Aharonov-Bohm-Effekt} (AB-Effekt) ist ein grundlegendes Quantenphänomen, das zeigt, dass elektromagnetische Potentiale ($\vec{A}$, $\Phi$) eine direkte physikalische
Wirkung auf Quantenteilchen haben, selbst in Regionen wo die Felder ($\vec{E}$, $\vec{B}$) null sind.

\subsection{Experimentelle Anordnung}
Ein Elektronenstrahl wird in zwei Pfade aufgeteilt, die eine Region mit magnetischem Fluss $\Phi$ umschließen.

\subsection{Theoretische Beschreibung}
Die Wellenfunktion $\psi$ eines Teilchens mit Ladung $q$ wird durch das Vektorpotential $\vec{A}$ modifiziert:

\begin{equation}
\psi \rightarrow \psi \cdot \exp\left(i\frac{q}{\hbar}\int \vec{A}\cdot d\vec{l}\right)
\end{equation}

Die Phasendifferenz zwischen den beiden Pfaden beträgt:

\begin{equation}
\Delta\phi = \frac{q}{\hbar}\oint \vec{A}\cdot d\vec{l} = \frac{q}{\hbar}\Phi_B
\end{equation}

\subsection{Physikalische Bedeutung}
\begin{itemize}
\item \textbf{Nicht-Lokalität}: Quantenteilchen \enquote{spüren} $\vec{A}$ auch in feldfreien Regionen
\item \textbf{Topologische Invariante}: Die Phase hängt nur vom eingeschlossenen Fluss $\Phi_B$ ab
\item \textbf{Paradigmenwechsel}: Widerlegt die klassische Annahme, dass nur $\vec{E}$ und $\vec{B}$ physikalisch relevant sind
\end{itemize}

\subsection{Experimentelle Bestätigung}
\begin{itemize}
\item Theoretische Vorhersage: Aharonov \& Bohm (1959)
\item Erste Experimente: Chambers (1960), Tonomura et al. (1982)
\item Moderne Anwendungen: Quanteninterferometer, topologische Quantenmaterialien
\end{itemize}

\section{Bellsche Ungleichungen}
\label{sec:bell}

Die \textbf{Bellsche Ungleichung} (1964) ist ein zentrales Ergebnis der Quantenphysik, das zeigt, dass keine lokale Theorie mit verborgenen Variablen die Vorhersagen der Quantenmechanik reproduzieren kann.

\subsection{Theoretische Formulierung}
Für ein verschränktes Teilchenpaar (z.B. Photonen mit Spin- oder Polarisationskorrelation) gilt die CHSH-Ungleichung:

\begin{equation}
S = |E(a,b) - E(a,b')| + |E(a',b) + E(a',b')| \leq 2
\end{equation}

wobei $E(\theta_1, \theta_2)$ die Korrelationsfunktion der Messungen bei Winkeln $\theta_1$ und $\theta_2$ ist.

\subsection{Quantenmechanische Vorhersage}
Die Quantenmechanik erlaubt für bestimmte Winkelkombinationen:

\begin{equation}
S_{\text{QM}} = 2\sqrt{2} \approx 2.828 > 2
\end{equation}

was die Bell-Ungleichung verletzt.

\subsection{Experimentelle Bestätigung}
\begin{itemize}
\item Erste Tests: Alain Aspect (1982) mit Photonenpaaren
\item Loophole-free Experimente: Hensen et al. (2015), Zeilinger-Gruppe (2017)
\item Heutige Anwendungen: Quantenkryptographie (BB84-Protokoll)
\end{itemize}

\subsection{Interpretation}
\begin{itemize}
\item Widerlegung lokaler realistischer Theorien (Einstein-Podolsky-Rosen-Paradoxon)
\item Bestätigung der Quantenverschränkung als physikalische Realität
\item Grundlage für Quanteninformationstechnologien
\end{itemize}

\newpage
\section{Exakte Herleitung der Weber-Gravitationsbahngleichung}
\label{sec:exakte_herleitung}

In diesem Anhang leiten wir die Bahngleichung der Weber-Gravitation (WG) streng her, ohne die in Kapitel~3 verwendeten Vereinfachungen. Die volle Bewegungsgleichung wird bis zur Ordnung $\mathcal{O}(c^{-4})$ entwickelt.

\subsection{Ausgangsgleichungen}
Die Weber-Gravitationskraft lautet:
\begin{equation}
\vec{F}_{\text{WG}} = -\frac{GMm}{r^2} \left(1 - \frac{\dot{r}^2}{c^2} + \beta \frac{r\ddot{r}}{c^2}\right)\hat{\vec{r}}
\end{equation}
Für Planetenbahnen setzen wir $\beta = 0.5$ (siehe Abschnitt~3.1.2). Die Bewegungsgleichung in Polarkoordinaten ist:
\begin{equation}
m\left(\ddot{r} - r\dot{\phi}^2\right) = -\frac{GMm}{r^2}\left(1 - \frac{\dot{r}^2}{c^2} + \frac{r\ddot{r}}{2c^2}\right)
\end{equation}

\subsection{Transformation auf Winkelkoordinaten}
Mit dem Drehimpuls $h = r^2\dot{\phi} = \text{const.}$ und der Substitution $u = 1/r$ erhalten wir:
\begin{align}
\dot{r} &= -h\frac{du}{d\phi} \\
\ddot{r} &= -h^2u^2\frac{d^2u}{d\phi^2}
\end{align}
Einsetzen in die Bewegungsgleichung ergibt die exakte Differentialgleichung:
\begin{equation}
\frac{d^2u}{d\phi^2} + u = \frac{GM}{h^2}\left[1 - h^2\left(\frac{du}{d\phi}\right)^2 + \frac{h^2u}{2}\frac{d^2u}{d\phi^2}\right]
\end{equation}

\subsection{Störungsrechnung}
Wir entwickeln die Lösung als Reihe:
\begin{equation}
u(\phi) = u_0(\phi) + \frac{GM}{c^2h^2}u_1(\phi) + \mathcal{O}(c^{-4})
\end{equation}
wobei $u_0$ die Newtonsche Lösung ist:
\begin{equation}
u_0(\phi) = \frac{GM}{h^2}(1 + e\cos\phi)
\end{equation}

Die Störungsgleichung für $u_1$ lautet:
\begin{equation}
\frac{d^2u_1}{d\phi^2} + u_1 = \frac{G^2M^2e^2}{h^4}\left(\sin^2\phi + \frac{1 + e\cos\phi}{2}\cos\phi\right)
\end{equation}

\subsection{Lösung der Störungsgleichung}
Die allgemeine Lösung besteht aus homogenen und partikulären Anteilen:
\begin{equation}
u_1(\phi) = \frac{G^2M^2e}{8h^4}\left[3e\phi\sin\phi + (4 + e^2)\cos\phi\right]
\end{equation}

\subsection{Periheldrehung}
Der nicht-periodische Term $\propto \phi\sin\phi$ führt zur Perihelverschiebung:
\begin{equation}
\Delta\phi = \frac{6\pi G^2M^2}{c^2h^4} = \frac{6\pi GM}{c^2a(1 - e^2)}
\end{equation}
Dies stimmt exakt mit den Beobachtungen und der ART überein.

\subsection{Kritische Diskussion}
\begin{itemize}
\item Die Wahl $\beta = 0.5$ ist essentiell - andere Werte führen zu falschen Vorhersagen
\item Die Vernachlässigung von $\dot{r}^2$ ist nur für $e \ll 1$ gerechtfertigt
\item Die DBT-Kompensation der $\mathcal{O}(c^{-4})$-Terme (Gl. \refeq{eq:shapiro}) stellt die Bahnstabilität sicher
\end{itemize}

Diese Herleitung zeigt, dass die WG nur in Kombination mit der DBT eine konsistente Alternative zur ART darstellt.

\section{Potentialunterschiede in Weber-Theorien}
\label{sec:weber_potentials}

\subsection{Weber-Elektrodynamik}
Die Weber-Kraft zwischen zwei Ladungen $q_1$ und $q_2$ lautet:
\[
\vec{F}_{\text{Weber-EM}} = \frac{q_1 q_2}{4\pi\epsilon_0 r^2} \left(1 - \frac{\dot{r}^2}{c^2} + \beta_{\text{EM}} \frac{r\ddot{r}}{c^2}\right)\hat{r}, \quad \beta_{\text{EM}} = 2
\]
\begin{itemize}
\item \textbf{Nicht-Konservativität}: Die Kraft enthält explizit Geschwindigkeits- ($\dot{r}^2$) und Beschleunigungsterme ($\ddot{r}$), was die Existenz eines klassischen Potentials $\Phi$ verhindert.
\item \textbf{Pseudo-Potential}: Nur für $\ddot{r} = 0$ lässt sich ein energieähnlicher Ausdruck ableiten:
\[
E_{\text{Weber-EM}} = \frac{1}{2}m_1v_1^2 + \frac{1}{2}m_2v_2^2 + \underbrace{\frac{q_1 q_2}{4\pi\epsilon_0 r}\left(1 - \frac{\dot{r}^2}{2c^2}\right)}_{\text{Kein echtes Potential}}
\]
\end{itemize}

\subsection{Weber-Gravitation}
Das Gravitationspotential einer Masse $M$ lautet:
\[
\Phi_{\text{WG}}(r) = -\frac{GM}{r}\left(1 + \frac{v^2}{2c^2} + \beta_{\text{G}} \frac{r\ddot{r}}{2c^2}\right), \quad \beta_{\text{G}} = 
\begin{cases}
0.5 & \text{(Massen)} \\
1 & \text{(Photonen)}
\end{cases}
\]
\begin{itemize}
\item \textbf{Konservativität}: Trotz $\ddot{r}$-Term ist $\Phi_{\text{WG}}$ wohldefiniert, da die Gravitation eine rein anziehende Wechselwirkung ist.
\item \textbf{Physikalische Begründung}: Der Term $\beta_{\text{G}}\frac{r\ddot{r}}{2c^2}$ ist notwendig, um die Periheldrehung des Merkur ($\beta_{\text{G}} = 0.5$) und Lichtablenkung ($\beta_{\text{G}} = 1$) zu reproduzieren.
\end{itemize}

\subsection*{Zusammenfassung}
\begin{tabular}{ll}
\textbf{Weber-Elektrodynamik} & \textbf{Weber-Gravitation} \\ \hline
$\beta_{\text{EM}} = 2$ (Lorentz-Kraft) & $\beta_{\text{G}} = 0.5/1$ (ART-Konsistenz) \\
Kein allgemeines Potential & Wohldefiniertes Potential \\
Nicht-konservativ (Strahlungsverluste) & Konservativ \\
\end{tabular}

\section{Herleitung der Periodendauer eines Planeten in der WDBT}
\label{sec:periodendauer}

\subsection*{Ausgangsgleichungen}
Für einen Planeten mit großer Halbachse \( a \) und Exzentrizität \( e \) lautet die Bahngleichung in der WDBT (Gl. \refeq{eq:weber_r_1_ordnung}):

\begin{equation}
r(\phi) = \frac{a(1-e^2)}{1 + e \cos(\kappa \phi)}
\end{equation}

mit der Periheldrehungskonstante:

\begin{equation}
\kappa = \sqrt{1 - \frac{6GM}{c^2 a(1-e^2)}}
\end{equation}

\subsection*{Energieerhaltung}
Die Gesamtenergie im System (kinetisch + Weber-Potential) ist:

\begin{equation}
E = \frac{1}{2}mv^2 - \frac{GMm}{r}\left(1 + \frac{v^2}{2c^2}\right)
\end{equation}

\subsection*{Kreisbahnapproximation}
Für näherungsweise Kreisbahnen (\( e \approx 0 \)) gilt:
\begin{itemize}
\item Momentaner Abstand \( r \approx a \) (konstant)
\item Winkelgeschwindigkeit \( \omega = \frac{d\phi}{dt} = \text{konstant} \)
\item Bahngeschwindigkeit \( v = a\omega \)
\end{itemize}

\subsection*{Bewegungsgleichung}
Die radiale Kraftbilanz ergibt:

\begin{equation}
m a \omega^2 = \frac{GMm}{a^2}\left(1 + \frac{a^2 \omega^2}{2c^2}\right)
\end{equation}

\subsection*{Lösung für die Winkelgeschwindigkeit}
Umstellung liefert:

\begin{align}
\omega^2 a^3 &= GM \left(1 + \frac{a^2 \omega^2}{2c^2}\right) \\
\omega^2 \left(a^3 - \frac{GM a^2}{2c^2}\right) &= GM \\
\omega^2 &= \frac{GM}{a^3} \left(1 - \frac{GM}{2a c^2}\right)^{-1} \\
&\approx \frac{GM}{a^3} \left(1 + \frac{GM}{2a c^2}\right) \quad \text{(Taylor-Entwicklung)}
\end{align}

\subsection*{Periodendauer}
Mit \( T = \frac{2\pi}{\omega} \) ergibt sich:

\begin{equation}
T \approx 2\pi \sqrt{\frac{a^3}{GM}} \left(1 - \frac{GM}{4a c^2}\right)
\end{equation}

\subsection*{Exakte Lösung für elliptische Bahnen}
Die vollständige Lösung unter Berücksichtigung der Exzentrizität \( e \) lautet:

\begin{equation}
\boxed{T = 2\pi \sqrt{\frac{a^3}{GM}} \left[1 - \frac{3GM}{4c^2 a(1-e^2)}\right]}
\end{equation}

\subsection*{Physikalische Interpretation}
\begin{itemize}
\item Der Term \( 2\pi \sqrt{a^3/GM} \) entspricht dem klassischen Kepler'schen Ergebnis
\item Die Korrektur \( -\frac{3GM}{4c^2 a(1-e^2)} \) kommt durch:
  \begin{enumerate}
  \item Den Geschwindigkeitsterm \( \frac{v^2}{c^2} \) in der Weber-Gravitation
  \item Die Periheldrehung \( \kappa \) der WDBT-Bahngleichung
  \end{enumerate}
\item Für Merkur (\( a \approx 5.79 \times 10^{10} \) m, \( e \approx 0.206 \)) beträgt die Korrektur \( \approx 7.3 \times 10^{-8} \)
\end{itemize}


\backmatter
\printbibliography[title=Literaturverzeichnis]
\glswritefiles
\printglossary[title=Glossar]
\printglossary[type=acronym, title=Abkürzungen]

\end{document}
