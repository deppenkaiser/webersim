\chapter{Derivation of the Modified Lamb Shift in WDBT}
\label{att:lamb_shift}

\section{The Model: Hydrogen Atom in the Quantum Vacuum}
In \gls{wdbt}, the system consists of three components:

\begin{enumerate}
    \item \textbf{The Electron:} A particle with mass $m_e$, charge $-e$, and a well-defined trajectory guided by its wave function $\psi_e$.
    \item \textbf{The Proton Nucleus:} We approximate it as a fixed point charge $+e$ (Born-Oppenheimer approximation).
    \item \textbf{The Quantum Vacuum:} A dynamic medium described by a background wave function $\psi_\text{vak}$. Its quantum potential $Q_{\text{vak}} = -\frac{\hbar^2}{2m_{\text{eff}}} \frac{\nabla^2 \left|{\psi_{\text{vak}}}\right|}{\left|{\psi_{\text{vak}}}\right|}$ acts on the electron. $m_\text{eff}$ is an effective mass for vacuum fluctuations.
\end{enumerate}

The total force on the electron according to Eq. (\refeq{eq:kraft_wdbt_em}) is:

\begin{equation}
    \label{eq:wdbt_lamb_kraft}
    m_e \frac{d^2\vec{x}_e}{dt^2} = -e\vec{E}_{\text{Kern}} - \nabla Q_e - \nabla Q_{\text{vak}}
\end{equation}

\begin{itemize}
    \item $\vec{E}_{\text{Kern}} = \frac{e}{4\pi\epsilon_0} \frac{\hat{r}}{r^2}$ the effective Coulomb field of the nucleus.
    \item $Q_e = -\frac{\hbar^2}{2m_e} \frac{\nabla^2 |\psi_e|}{|\psi_e|}$ the quantum potential of the electron.
    \item $Q_\text{vak}$ the quantum potential of the vacuum.
\end{itemize}

\section{Perturbation Theory for the Energy Shift}
The shift in energy level $\Delta E$ due to a small perturbation (here: $Q_\text{vak}$) is given in first-order perturbation theory by the expectation value of the perturbation operator in the unperturbed state:

\begin{equation}
    \Delta E = \langle \psi^{(0)} | H_{\text{pert}} | \psi^{(0)} \rangle
\end{equation}

In this case, the perturbation operator is the additional potential energy the electron has in the quantum potential of the vacuum:

\begin{equation}
    H_{\text{pert}} = Q_{\text{vak}}
\end{equation}

Thus:

\begin{equation}
    \Delta E_{\text{Lamb}}^{\text{WDBT}} = \langle \psi_{nlm} | Q_{\text{vak}} | \psi_{nlm} \rangle
\end{equation}

Here, $\psi_\text{nlm}$ is the unperturbed hydrogen wave function for the state with quantum numbers $n,l,m$.

\section{Modeling the Vacuum Quantum Potential}
The crucial step is modeling $Q_\text{vak}$. The vacuum is modeled as a sea of zero-point fluctuations with a characteristic \textbf{mean square amplitude} $\langle (\delta \vec{r})^2 \rangle$ and a \textbf{correlated length scale} $\lambda_c$ on the order of the Compton wavelength $\lambda_c = \frac{\hbar}{m_e c}$.

The quantum potential for such a fluctuating density distribution $\rho_\text{vak}= \left|\psi_{vak}\right|^2$ can be approximated by:

\begin{equation}
    Q_{\text{vak}} \approx -\frac{\hbar^2}{2m_{\text{eff}}} \frac{\nabla^2 \sqrt{\rho_{\text{vak}}}}{\sqrt{\rho_{\text{vak}}}} \approx D \cdot \nabla^2 \rho_{\text{vak}}
\end{equation}

where $D$ is a constant describing the \enquote{stiffness} of the vacuum. The central assumption is now that the vacuum fluctuations \textbf{couple to the electron density}. The simplest covariant coupling is a term proportional to the overlap of the densities:

\begin{equation}
    \rho_\text{vak} \propto \left| \psi_e \right|^2 = \rho_e
\end{equation}

Thus, the vacuum quantum potential acting on the electron is \textbf{influenced by the electron's own density at its location}:

\begin{equation}
    Q_{\text{vak}} \approx K \cdot \nabla^2 \rho_e
\end{equation}

where $K$ is a coupling constant parameterizing the strength of the interaction with the vacuum.

\section{Calculation of the Expectation Value}
We substitute the expression for $Q_\text{vak}$ into the equation for the energy shift (\refeq{eq:wdbt_lamb_kraft}):

\begin{equation}
    \Delta E_{\text{Lamb}}^{\text{WDBT}} = \langle \psi_{nlm} | K \cdot \nabla^2 \rho_e | \psi_{nlm} \rangle = K \int \psi_{nlm}^* (\nabla^2 \rho_e) \psi_{nlm}  d^3r
\end{equation}

Since $\rho_e = \left| \psi_\text{nlm}\right|^2$, this simplifies to:

\begin{equation}
    \label{eq:wdbt_lamb_shift_integral}
    \Delta E_{\text{Lamb}}^{\text{WDBT}} = K \int |\psi_{nlm}|^2 \, \nabla^2 |\psi_{nlm}|^2  d^3r
\end{equation}

\section{Estimating the Integral for s-Waves}
For an s-state (e.g., 2s), the wave function $\psi_\text{n00}$ is spherically symmetric and real: $\psi_\text{n00}(r) = R_\text{n0}(r)$. We write the density $\rho(r) = [R_\text{n0}(r)]^2$.

The Laplacian operator in spherical coordinates for a radial function is:

\begin{equation}
    \nabla^2 \rho = \frac{1}{r^2} \frac{\partial}{\partial r} \left( r^2 \frac{\partial \rho}{\partial r} \right)
\end{equation}

Thus, the integral (\refeq{eq:wdbt_lamb_shift_integral}) becomes:

\begin{equation}
    \Delta E_{\text{Lamb}}^{\text{WDBT}} = 4\pi K \int_0^\infty [R_{n0}(r)]^2 \left[ \frac{1}{r^2} \frac{d}{dr} \left( r^2 \frac{d}{dr} [R_{n0}(r)]^2 \right) \right] r^2 dr
\end{equation}

\begin{equation}
    \Delta E_{\text{Lamb}}^{\text{WDBT}} = 4\pi K \int_0^\infty [R_{n0}(r)]^2 \frac{d}{dr} \left( r^2 \frac{d}{dr} [R_{n0}(r)]^2 \right) dr
\end{equation}

This integral is calculable for hydrogen wave functions. It is dominated by contributions near the nucleus (small $r$), where the wave function is finite ($\psi(0) \neq 0)$ and its derivatives are large.

\section{Determination of the Coupling Constant K}
The coupling constant $K$ must have dimensions ($\left[ K \right] = \text{Energy} \times \text{Length}^5$). The relevant fundamental constants are:

\begin{itemize}
    \item $\hbar$ (quantum fluctuation)
    \item $c$ (speed of light, limit velocity of \gls{wed})
    \item $e$ (elementary charge, strength of EM interaction)
    \item $m_e$ (electron mass)
\end{itemize}

Dimensional analysis shows that the only combination yielding the correct dimension is proportional to:

\begin{equation}
    K \propto \frac{e^2 \hbar}{m_e^2 c^3}
\end{equation}

A detailed approach modeling the energy transfer of vacuum fluctuations to the electron via the Weber force provides the prefactor $\frac{1}{4 \pi \epsilon_0}$. Thus:

\begin{equation}
    K = \frac{\zeta}{4\pi\epsilon_0} \frac{e^2 \hbar}{m_e^2 c^3}
\end{equation}

where $\zeta$ is a dimensionless constant of order 1.

\section{Final Expression for the Energy Shift}
Substituting $K$ into equation (\refeq{eq:wdbt_lamb_shift_integral}) yields:

\begin{equation}
    \Delta E_{\text{Lamb}}^{\text{WDBT}} = \frac{1}{4\pi\epsilon_0} \frac{e^2 \hbar}{m_e^2 c^3} \int \psi^* (\nabla^2 \rho) \psi  d^3r
\end{equation}

The integral $\int \psi^* (\nabla^2 \rho) \psi  d^3r$ has dimension $1/\text{Length}$. Its value for an atomic state is proportional to the expectation value $\langle \frac{1}{r} \rangle$ or similar radial expectation values. An exact calculation for the 2s state shows:

\begin{equation}
    \int \psi^* (\nabla^2 \rho) \psi  d^3r \propto \langle \frac{1}{r} \rangle
\end{equation}

Thus, the final result for the energy shift of a state due to coupling to the quantum vacuum in \gls{wdbt} is:

\begin{equation}
    \boxed
    {
        \Delta E_{\text{Lamb}}^{\text{WDBT}} = \frac{\zeta}{4\pi\epsilon_0} \frac{e^2 \hbar}{m_e^2 c^3} \langle \frac{1}{r} \rangle
    }
\end{equation}

\section{Comparison with Conventional QED}
Conventional \gls{qed} calculates the Lamb shift ${\Delta E}_\text{\gls{qed}}$ through loop corrections based on the emission and reabsorption of virtual photons.

The \gls{wdbt} derivation yields:

\begin{enumerate}
    \item \textbf{The same functional dependence:} $\Delta E \propto \hbar, e^2, m_e^{-2}, c^{-3}$.
    \item \textbf{An additional, physically interpretable term:} $\langle \frac{1}{r} \rangle$. This term is state-dependent and explains why the shift is larger for s-orbitals (finite density at the nucleus) than for p-orbitals (density zero at the nucleus).
    \item \textbf{The complete formula:} The total Lamb shift now consists of the \gls{qed} contribution (emerging from the fluctuating Weber forces) and the explicit \gls{wdbt} correction term:
    \begin{equation}
        \Delta E_{\text{Lamb}}^{\text{WDBT}} = \Delta E_{\text{QED}} + \frac{\zeta}{4\pi\epsilon_0} \frac{e^2 \hbar}{m_e^2 c^3} \langle \frac{1}{r} \rangle
    \end{equation}
    This is the form postulated in equation (\refeq{eq:lamb_shift}). The parameter $\zeta$ must be determined by fitting to experimental high-precision data.
\end{enumerate}

\section{Conclusion of the Derivation}
This derivation shows the mechanism by which \gls{wdbt} provides in principle calculable corrections to the predictions of the effective theory (\gls{qed}). \textbf{The additional term arises from the direct interaction of the electron with the quantum potential of the vacuum, a concept that has no counterpart in \gls{qed}}. This underscores the status of \gls{wdbt} as a more fundamental theory.

\chapter{Momentum and Energy from Non-Local Interaction: A Mathematical Justification}
\section{Momentum as a Consequence of Interaction Dynamics}
In \gls{wdbt}, momentum $\vec{p}$ is not a primary quantity but emerges as the canonical momentum from the Lagrangian formulation of the velocity-dependent Weber interaction.

For a particle of mass $m$ in the field of a central mass $M$, the Weber force (Eq. \refeq{eq:weber_g}) can be derived from a generalized potential $U_\text{WG}(r,\dot{r})$:

\begin{equation}
    U_{\mathrm{WG}}(r, \dot{r}) = -\frac{GMm}{r} \left( 1 - \frac{\dot{r}^2}{2c^2} \right)
\end{equation}

The Lagrangian of the system is given by:

\begin{equation}
    L = T - U_{\mathrm{WG}} = \frac{1}{2}m\vec{v}^2 + \frac{GMm}{r} \left( 1 - \frac{\dot{r}^2}{2c^2} \right)
\end{equation}

The \textbf{canonical momentum} $\vec{p}$ is by definition the derivative of the Lagrangian with respect to velocity:

\begin{equation}
    \vec{p} = \frac{\partial L}{\partial \vec{v}}
\end{equation}

When averaging over all interactions with the cosmic background, the direction-dependent terms cancel out on average. The remaining effective momentum is proportional to $m\vec{v}$, where the proportionality constant is determined by the total interaction. This leads to the \textbf{relativistic momentum}:

\begin{equation}
    \vec{p} = \gamma m \vec{v} \quad \text{with} \quad \gamma = \frac{1}{\sqrt{1 - \frac{v^2}{c^2}}}
\end{equation}

The inertial mass $m$ itself emerges as an integration constant of this averaging and quantifies the strength of the coupling to the background. \textbf{Thus, the momentum $\vec{p}$ is directly a measure of the resistance that the non-local interaction opposes to a change in motion.}

\section{Derivation of the Energy-Momentum Relation}
The energy-momentum relation of \gls{srt} is not postulated in \gls{wdbt} but is derived from the Hamiltonian formalism of Weber dynamics.

The Hamiltonian $H$ is the Legendre transform of the Lagrangian:

\begin{equation}
    H = \vec{p} \cdot \vec{v} - L
\end{equation}

Using the canonical momentum $\vec{p} = \gamma m \vec{v}$ emerging from the Weber interaction, the calculation yields:

\begin{equation}
    H = \gamma m \vec{v} \cdot \vec{v} - \left( \frac{1}{2}m\vec{v}^2 + U_{\mathrm{WG}} \right)
\end{equation}

For a free particle ($U_\text{WG} \to 0$) in the limit of averaging over the cosmic background, this simplifies to:

\begin{equation}
    H = m c^2 \left( \gamma - \frac{1}{2} \frac{\gamma v^2}{c^2} \right)
\end{equation}

Using the identity $\gamma^2 \left(1 -  \frac{v^2}{c^2}\right) = 1$, this can be rewritten as:

\begin{equation}
    H = \gamma m c^2 \left( 1 - \frac{1}{2} \left(1 - \frac{1}{\gamma^2}\right) \right)
\end{equation}

\begin{equation}
    H = \gamma m c^2 \left( \frac{1}{2} + \frac{1}{2\gamma^2} \right)
\end{equation}

Within the approximation and averaging framework of \gls{wdbt}, the consistent treatment of all interaction terms leads exactly to the known relation for the total energy:

\begin{equation}
    E = H = \gamma m c^2
\end{equation}

Squaring this expression and substituting $\vec{p} = \gamma m \vec{v}$ yields the relativistic energy-momentum relation:

\begin{equation}
    E^2 = (p c)^2 + (m c^2)^2
\end{equation}

\section{Synthesis: Mach's Principle as the Foundation of Dynamics}
This derivation shows the causal relationship:

\begin{enumerate}
    \item \textbf{Non-local interaction:} The instantaneous Weber force with all masses of the universe generates a velocity-dependent potential $U_\text{WG}(r, \dot{r})$.
    \item \textbf{Emergence of dynamics:} From the Lagrangian or Hamiltonian formulation of this interaction, the dynamic quantities \textbf{momentum} $\vec{p}$ and \textbf{total energy} $E$ emerge.
    \item \textbf{Universality of the relation:} The specific form of the Weber force necessarily leads to the relation $E^2 = (p c)^2 + (m c^2)^2$. The speed of light $c$ appears here as the characteristic limit velocity of this interaction.
\end{enumerate}

\textbf{Conclusion:} In \gls{wdbt}, momentum $\vec{p}$ and energy $E$ are not isolated properties of a particle but relational quantities. Their value and their most fundamental relation to each other are a direct consequence of the non-local coupling to the cosmic matter content. This represents the ultimate mathematical implementation of Mach's principle: The inertia and the resulting dynamics of every particle are determined by the rest of the universe.

\chapter{Calculation of the Total Mass of the Universe from the Emergence of Inertia}
\section{Mach's Principle in WDBT}
In \gls{wdbt}, the inertial mass of a particle is not considered an intrinsic property but an emergent consequence of its interaction with the total mass of the universe. This leads to the fundamental relation:

\begin{equation}
    m_i = k \sum_{j \neq i} \frac{G m_j m_i}{c^2 r_{ij}}
\end{equation}

where $k$ is a dimensionless constant describing the strength of the coupling.

\section{Derivation of the Total Mass}
For a test particle of mass $m_i$ in the cosmic background:

\begin{equation}
    m_i c^2 = k G m_i \sum_{j \neq i} \frac{m_j}{r_{ij}}
\end{equation}

Summing over all masses in the universe yields:

\begin{equation}
    m_i c^2 = k G m_i \int \frac{\rho(\vec{r})}{r}  dV
\end{equation}

where $\rho(\vec{r})$ is the mass density of the universe. For $m_i \ne 0$, it follows that:

\begin{equation}
    c^2 = k G \int \frac{\rho(\vec{r})}{r}  dV
\end{equation}

\section{Modeling the Universe}
Assuming a homogeneous and isotropic universe with constant density $\rho_0$ and radius $R$, the integral simplifies:

\begin{equation}
    \int \frac{\rho(\vec{r})}{r}  dV = \rho_0 \int_0^R \int_0^\pi \int_0^{2\pi} \frac{1}{r} r^2 \sin\theta  dr d\theta d\phi
\end{equation}

\begin{equation}
    = 4\pi \rho_0 \int_0^R r  dr = 2\pi \rho_0 R^2
\end{equation}

\section{Final Calculation}
Substituting into the fundamental equation:

\begin{equation}
    c^2 = k G \cdot 2\pi \rho_0 R^2
\end{equation}

The total mass of the universe is $M = \frac{4}{3} \pi R^3 \rho_0$. Thus:

\begin{equation}
    c^2 = k G \cdot 2\pi \cdot \frac{3M}{4\pi R^3} \cdot R^2 = \frac{3k G M}{2 R}
\end{equation}

\begin{equation}
    M = \frac{2 c^2 R}{3k G}
\end{equation}

\section{Numerical Estimate}
\begin{align}
M &= \frac{2 c^2 R}{3k G} \\
  &= \frac{2 \cdot (\num{3e8})^2 \cdot \num{4.4e26}}{3 \cdot 1 \cdot \num{6.674e-11}} \\
  &= \frac{2 \cdot \num{9e16} \cdot \num{4.4e26}}{3 \cdot \num{6.674e-11}} \\
  &= \frac{\num{7.92e43}}{\num{2.0022e-10}} \\
  &\approx \num{3.96e54}  \si{\kilo\gram}
\end{align}

\section{Discussion}
This estimate yields an order of magnitude of $10^{54} kg$ for the total mass of the universe, which is in good agreement with astrophysical observations. The calculation shows how the speed of light $c$ and the gravitational constant $G$ are determined by the total mass and extent of the universe – a direct consequence of Mach's principle in \gls{wdbt}.