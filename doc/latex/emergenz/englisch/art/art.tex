\chapter{The Convergent Emergence of GR}
 How \gls{art} tends towards \gls{wdbt}.

\section{The Incomplete GR: The Singularity Problem}

\begin{itemize}
    \item \textbf{Standard-\gls{art}:} $G_{\mu\nu} = 8\pi G T_{\mu\nu}$
    \item This equation leads to singularities under generic conditions (Black Holes, Big Bang).
    \item \textbf{Interpretation:} This is not a physical but a theoretical failure. \gls{art} is incomplete at its limits.
\end{itemize}

\section{Step 1 of Emergence: Completing GR through DBT}

\begin{itemize}
    \item The most obvious extension to avoid singularities is the introduction of the \textbf{Bohmian quantum potential $Q$}.
    \item The completed Einstein equations now read:
    \begin{equation}
        G_{\mu\nu} = 8\pi G (T_{\mu\nu} + Q_{\mu\nu})
    \end{equation}
    \item \textbf{Consequences of this extension:}
    \begin{enumerate}
        \item \textbf{Singularity-free:} $Q$ acts repulsively and prevents the formation of point singularities. Result: Big Bounce instead of Big Bang.
        \item \textbf{Non-locality:} The quantum potential $Q$ is fundamentally non-local. This property is now introduced into gravity itself.
        \item \textbf{Alignment with WDBT:} The extended \gls{art} gains central properties of \gls{wdbt}: Deterministic trajectories, singularity-free, and non-locality.
    \end{enumerate}
\end{itemize}

\section{Step 2 of Emergence: Completion by Accounting for Non-Locality}

\begin{itemize}
    \item \gls{art} (even the extended version) is a local field theory. The original \gls{wg} of \gls{wdbt}, however, is instantaneous and non-local.
    \item This property can also be \enquote{imported} into \gls{art} by considering the \textbf{solutions of the Einstein equations}. The \textbf{gravitational wave} (propagation with $c$) is only a special solution.
    \item The \textbf{instantaneous curvature} (which determines the motion of planets) is another. In a complete treatment, \gls{art} must be able to allow both descriptions equally – the retarded and the advanced solutions (Wheeler-Feynman approach).
    \item \textbf{Result:} The thus completed \gls{art} also becomes \textbf{non-local and local} simultaneously, just like \gls{wdbt}. The retardation of waves is a special case, the instantaneity of the fields is the rule.
\end{itemize}

\section{The Convergent Theories: GR+ vs. WDBT}
Through these two steps of completion, the extended \gls{art} (\gls{art}+) and \gls{wdbt} converge conceptually:

\begin{table}[h]
\centering
\begin{tabular}{|p{0.25\textwidth}|p{0.3\textwidth}|p{0.3\textwidth}|}
\hline
\textbf{Property} & \textbf{\gls{art}+ (Completed)} & \textbf{\gls{wdbt} (Fundamental)} \\
\hline
\textbf{Singularities} & None (Big Bounce) & None (Big Bounce) \\
\hline
\textbf{Non-locality} & Yes (via $Q_{\mu\nu}$ \& field solutions) & Yes (fundamental via \gls{wg}) \\
\hline
\textbf{Determinism} & Yes (via $Q$) & Yes (fundamental) \\
\hline
\textbf{Big Bang} & No & No \\
\hline
\textbf{Light deflection} & Frequency-independent & Frequency-dependent ($\Delta \phi(f)$) \\
\hline
\textbf{Basis} & Geometric description & Dyn. Interaction \\
\hline
\end{tabular}
\caption{Comparison of the completed GR (GR+) with the fundamental WDBT}
\end{table}

\textbf{The two theories appear to converge!}

\section{The Experimental Decider: Frequency-Dependent Light Deflection}
Despite the conceptual convergence, a \textbf{decisive, experimentally verifiable difference} remains:

\begin{itemize}
    \item \textbf{\gls{art}+:} Ultimately based on a \textbf{geometric} description. Light deflection is purely geometric and therefore \textbf{frequency-independent}.
    \item \textbf{\gls{wdbt}:} Based on a dynamic interaction (Weber force). Light deflection is a real force effect and therefore frequency-dependent ($\Delta \Phi(f)$).
\end{itemize}

\textbf{This deviation is the litmus test.} Which of the two convergent descriptions is the more fundamental one?

\begin{itemize}
    \item If one measures \textbf{no} frequency dependence, then the geometric description of \gls{art}+ is sufficient.
    \item If one measures \textbf{a} frequency dependence, this is the conclusive proof for the correctness of the dynamic foundation of \gls{wdbt}.
\end{itemize}

\section{Conclusion: WDBT as the Fundamental Primal Theory}
The search for a consistent extension of \gls{art} thus leads in a direction that looks remarkably similar to \gls{wdbt}. This is no coincidence but an indication that \gls{wdbt}
is the correct fundamental primal theory.

\gls{wdbt} not only provides the most consistent description but also the sharpest, testable prediction ($\Delta \Phi(f)$) to ultimately distinguish itself from all derived effective
theories (like \gls{art}).