\chapter{Special Theory of Relativity}
Here is the complete derivation from the \enquote{analogous} \gls{wdbt}

\section{Derivation of Relativistic Effects from the Weber-De Broglie-Bohm Theory (WDBT)}
The task is not the 1:1 reconstruction of \gls{srtn}, but the derivation of its operational core phenomena – time dilation, length contraction, relativistic dynamics – from the first principles of \gls{wdbt}, without adopting problematic postulates like Lorentz invariance of spacetime.

\subsection{Starting Point: The Energy-Momentum Relation in WDBT}
The fundamental interaction of \gls{wdbt} is described by the Weber gravitational force. For two masses $M$ and $m$ it reads with the parameter $\beta = 0.5$:

\begin{equation}
    \label{eq:weber_g}
    \vec{F}_{\text{WG}} = -\frac{G M m}{r^2} \left[ 1 - \frac{\dot{r}^2}{c^2} + 0.5 \frac{r \ddot{r}}{c^2} \right] \hat{\vec{r}}
\end{equation}

This force can be derived from a generalized potential $U_{WG}$:

\begin{equation}
    \label{eq:potential}
    U_{\text{WG}}(r, \dot{r}) = -\frac{G M m}{r} \left( 1 - \frac{\dot{r}^2}{2c^2} \right)
\end{equation}

For a particle moving in the cosmic background, averaging over all interactions leads to an \textbf{effective total energy}. The derivation via the Lagrangian or Hamiltonian formalism yields the \textbf{relativistic energy-momentum relation:}

\begin{equation}
    \label{eq:energie_impuls_beziehung}
    \boxed
    {
        E^2 = (p c)^2 + (m c^2)^2
    }
\end{equation}

\textbf{This equation is not a postulate.} It is a direct consequence of the velocity-dependent structure of the Weber force and the principle of energy conservation in \gls{wdbt}.

\subsection{Definition of Relativistic Quantities}
From the energy-momentum relation, the relativistic energy $E$ and relativistic momentum $p$ for a particle with rest mass $m$ and velocity $v$ are defined as:

\begin{equation}
    \label{eq:relativistische_energie}
    E = \gamma m c^2, \quad p = \gamma m v, \quad \text{with} \quad \gamma = \frac{1}{\sqrt{1 - \frac{v^2}{c^2}}}
\end{equation}

The Lorentz factor $\gamma$ appears here as a \textbf{mathematical consequence of the derivation}, not as an expression of a fundamental spacetime symmetry.

\subsection{Derivation of Time Dilation}
A periodic phenomenon (a \enquote{clock}) has a period $\Delta t_0$ in its rest frame. Its rest energy is $E_0 = mc^2$.

For an observer moving relative to the clock with velocity $v$, the total energy of the clock is $E = \gamma mc^2$.

Since the frequency $\nu$ of a periodic phenomenon is proportional to its energy ($\nu \propto E$), it follows:

\begin{equation}
    \label{eq:zeitdilatation}
    \frac{E}{E_0} = \gamma, \quad \frac{\Delta t_0}{\Delta t} = \gamma \quad \Rightarrow \quad \Delta t = \gamma \Delta t_0 = \frac{\Delta t_0}{\sqrt{1 - \frac{v^2}{c^2}}}
\end{equation}

\paragraph{Result:} The period appears lengthened for the moving observer. Moving clocks run slower. This is \textbf{time dilation}.

\subsection{Derivation of Length Contraction}
A rod of \textbf{rest length} $L_0$ lies in its rest frame. An observer moving with velocity $v$ parallel to the rod must determine its length $L$ by a \textbf{simultaneous} measurement of the positions of its endpoints in \textit{his} reference frame.

Due to \textbf{time dilation}, the clocks synchronized in the rod's frame are \textbf{not synchronized} in the observer's frame. Calculating the measurement procedure considering this effect leads to the result:

\begin{equation}
    \label{eq:längenkontraktion}
    L = \frac{L_0}{\gamma} = L_0 \sqrt{1 - \frac{v^2}{c^2}}
\end{equation}

\paragraph{Result:} The length of the rod appears shortened in the direction of motion. This is \textbf{length contraction}.

\subsection{Derivation of Relativistic Dynamics}
The equation of motion of a particle under the influence of a force $\vec{F}$ is described in \gls{wdbt} by the time derivative of the \textbf{relativistic momentum}:

\begin{equation}
    \label{eq:relativistischer_impuls}
    \vec{F} = \frac{d\vec{p}}{dt} = \frac{d}{dt} (\gamma m \vec{v})
\end{equation}

This equation replaces Newton's law $\vec{F} = m\vec{a}$. It correctly describes the increase of inertia at high velocities ($\gamma \to \infty$ for $v \to c$) and is consistent with the energy-momentum relation.

\subsection{Summary of Derived Effects}
From the energy-momentum relation $E^2 = (pc)^2 + (mc^2)^2$, which itself follows from the Weber force, the operational core phenomena of \gls{srt} were derived:

\begin{align*}
\textbf{Time dilation:} \quad & \Delta t = \gamma \Delta t_0 \\
\textbf{Length contraction:} \quad & L = \frac{L_0}{\gamma} \\
\textbf{Relativistic dynamics:} \quad & \vec{F} = \frac{d}{dt}(\gamma m \vec{v})
\end{align*}

\subsection{Conclusion: Emergence without Reduction}
\gls{wdbt} derives the successful predictions of \gls{srt} from its first principles. At the same time, it avoids the conceptual problems of \gls{srt}:

\begin{itemize}
    \item \textbf{Lorentz invariance} is not postulated as a fundamental property of spacetime. It merely appears as a useful level of description for the observed phenomena.
    \item The speed of light $c$ is not a universal constant of spacetime, but the \textbf{characteristic limit velocity of the Weber interaction}. This opens the possibility of energy-dependent variations, which are excluded in \gls{srt} and \gls{artn}.
\end{itemize}

\section{The True Universality of the Speed of Light}

The \textbf{universality of the speed of light} is justified differently in modern physics:

\subsection{Postulated Universality in SRT/GR}
In special and general relativity, $c$ is a \textit{postulated, metaphysical universality}:
\begin{itemize}
    \item $c$ is defined as a fundamental constant of \textbf{empty spacetime}
    \item Independent of matter and interactions
    \item Abstract, ultimately unexplained concept
    \item Violates Mach's principle
\end{itemize}

\subsection{Physical Universality in WG/WED (WDBT)}
In Weber Gravitation and Weber Electrodynamics, $c$ is a \textit{physical, mechanistic universality}:
\begin{itemize}
    \item $c$ is the characteristic limit velocity of the \textbf{Weber interaction}
    \item Arises from instantaneous coupling of all masses/charges in the universe
    \item Physical consequence of global coupling
    \item Fulfills Mach's principle completely
\end{itemize}

\gls{wdbt} thus establishes a \textbf{deeper and more consistent form of universality}, anchored in the physical reality of interactions - not in mathematical abstraction.

\section{Justification of Inertia in WDBT}

In \gls{wdbt}, inertia is not considered an inherent property of a particle, but as an \textbf{emergent consequence of its instantaneous interaction with all other masses and charges in the universe}.

\subsection{Inertia as Resistance to Acceleration}
Inertia manifests as resistance to acceleration. In \gls{wdbt}, this is explained by the \textbf{Weber force} acting on an accelerated particle:

\[
\vec{F}_{\text{WG}} = -\frac{G M m}{r^2} \left[ 1 - \frac{\dot{r}^2}{c^2} + \beta \frac{r \ddot{r}}{c^2} \right] \hat{r}
\]

The term $\beta \frac{r \ddot{r}}{c^2}$ (with $\beta = 0.5$ for masses) is proportional to the relative acceleration $\ddot{r}$ and describes the force component opposing acceleration.

\subsection{Mass as Coupling Constant}
The inertial mass $m$ of a particle emerges as a measure of the strength of its coupling to the cosmic background:

\[
m \propto \text{Strength of interaction with the universe}
\]

The rest energy $E_0 = m c^2$ quantifies the energy stored in this coupling.

\subsection{Fulfillment of Mach's Principle}
\gls{wdbt} fulfills \textbf{Mach's principle} in a strict form:
\begin{itemize}
    \item The inertia of a body is caused by the interaction with the total mass of the universe.
    \item In an empty universe, a particle would have no inertia.
    \item Inertia is not a local property, but a global relation.
\end{itemize}

Thus, \gls{wdbt} solves the riddle of inertia and transforms mass from a fundamental concept into a \textbf{measurable consequence of cosmic interaction}.

\section{Distinction Between Mass Inertia and Charge Inertia}

In \gls{wdbt}, inertia is not a unified phenomenon. It is divided into \textbf{mass inertia} and \textbf{charge inertia}, based on different interactions.

\subsection{Mass Inertia}
\begin{itemize}
    \item \textbf{Cause:} Interaction with all masses in the universe via \textbf{Weber Gravitation (WG)}.
    \item \textbf{Force law:}
        \[
        \vec{F}_{\text{WG}} = -\frac{G M m}{r^2} \left[ 1 - \frac{\dot{r}^2}{c^2} + \beta_m \frac{r \ddot{r}}{c^2} \right] \hat{r}, \quad \text{with} \quad \beta_m = 0.5
        \]
    \item The inertial mass $m$ emerges as a coupling constant to the \textbf{gravitational background}.
\end{itemize}

\subsection{Charge Inertia}
\begin{itemize}
    \item \textbf{Cause:} Interaction with all charges in the universe via\\\textbf{\gls{wed}}.
    \item \textbf{Force law:}
        \begin{equation}
        \vec{F}_{\text{WED}} = \frac{q_1 q_2}{4\pi\epsilon_0 r^2} \left[ 1 - \frac{\dot{r}^2}{c^2} + \beta_e \frac{r \ddot{r}}{c^2} \right] \hat{\vec{r}}, \quad \text{with} \quad \beta_e = 2            
        \end{equation}
    \item The inertia of a charge $q$ emerges through coupling to the \textbf{electromagnetic background}.
\end{itemize}

\subsection{Significance}
The distinction shows that inertia is an \textbf{interaction-specific} quantity. \gls{wdbt} thus enables a unification of gravitation and electrodynamics based on the Weber force.

\section{Treatment of Accelerations in the Effective SRT}

The \enquote{effective} \gls{srt} derived from \gls{wdbt} can also treat accelerations consistently. This is made possible by the \textbf{relativistic dynamics} founded in \gls{wdbt}, described by the equation
\[
\vec{F} = \frac{d}{dt}(\gamma m \vec{v})
\]
This equation replaces Newton's force law $\vec{F} = m \vec{a}$ and is valid for arbitrary velocities and accelerations.

\subsection{Reasons for Consistency with Accelerations}

\begin{enumerate}
    \item \textbf{Relativistic force equation}: The equation $\vec{F} = \frac{d}{dt}(\gamma m \vec{v})$ accounts for the increase of inertial mass at high velocities ($\gamma \to \infty$ for $v \to c$) and correctly describes accelerated motions even in the relativistic regime.
    \item \textbf{Weber force as origin}: The Weber force (Eq. 1.1) contains acceleration terms ($\ddot{r}$) that directly enter the equations of motion. This makes \gls{wdbt} fundamentally capable of treating accelerated systems.
    \item \textbf{Emergence of inertia}: In \gls{wdbt}, inertia is explained as a consequence of interaction with the cosmic background (Mach's principle). Thus, inertia during accelerations is also physically justified – not as an intrinsic property, but as an emergent phenomenon.    
    \item \textbf{Consistency with known results}: The dynamics derived from \gls{wdbt} is mathematically equivalent to the relativistic dynamics of conventional \gls{srt}. Therefore, accelerated motions (as in particle accelerators or gravitational systems) can also be described consistently.
\end{enumerate}

\subsection{Conclusion}
The effective \gls{srt}, as it emerges from \gls{wdbt}, is \textbf{not limited to unaccelerated systems}. It contains a complete relativistic dynamics that includes accelerations and is thus applicable to realistic physical scenarios.

\section{On the Significance of Inertial Frames in WDBT}

In \gls{wdbt}, the concept of \textbf{inertial frames} undergoes a fundamental re-evaluation and loses its privileged status as a distinguished reference frame.

\subsection{The Relativity of Inertia}

\begin{itemize}
    \item \textbf{Inertia as an interaction effect}: In \gls{wdbt}, inertia is not an intrinsic property of masses but emerges from the instantaneous interaction with all other masses in the universe (Mach's principle \cite{Assis1999}).
    \item \textbf{Abolition of absolute space}: Since there is no empty, structureless space that could serve as an absolute reference system, inertial frames lose their fundamental character.
\end{itemize}

\subsection{Operational versus Fundamental Significance}

\begin{itemize}
    \item \textbf{Operational usefulness}: Inertial frames retain their \emph{operative} significance as reference frames in which Newtonian mechanics remains approximately valid at velocities much smaller than $c$.
    \item \textbf{Fundamental status eliminated}: In \gls{wdbt}, inertial frames are not fundamental entities of physics, but merely \emph{useful approximations} in certain limiting cases.
\end{itemize}

\subsection{Accelerated Reference Frames as Equally Valid}

\begin{itemize}
    \item \textbf{Universal validity of equations of motion}: The fundamental equations of \gls{wdbt} hold equally in \emph{all} reference frames, since inertial forces are physically explained by interaction with the cosmic background.
    \item \textbf{No privileging of inertial frames}: While in classical \gls{srt} inertial frames are privileged, \gls{wdbt} treats all reference frames - accelerated and unaccelerated - as physically equivalent.
\end{itemize}

\subsection{Consequences for the Understanding of Spacetime}

\begin{itemize}
    \item \textbf{Emergence of Lorentz invariance}: The observed Lorentz invariance appears as an emergent phenomenon from the underlying Weber interaction, not as a fundamental property of spacetime.
    \item \textbf{Relational spacetime}: \gls{wdbt} implies a relational conception of space and time, in which physical effects are always explained by concrete interactions between matter.
\end{itemize}

\subsection{Conclusion}

In \gls{wdbt}, inertial frames lose their \textbf{fundamental status} as privileged reference systems. While they retain their \emph{operative usefulness} in certain limiting cases, they are no longer necessary as privileged reference frames in the fundamental description of physics by \gls{wdbt}. The theory thus offers a consistent implementation of Mach's principle and a relational conception of space, time, and inertia.