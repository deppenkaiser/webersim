\documentclass[11pt, a5paper, twoside, openright]{book}
\usepackage[english]{babel}
\usepackage[T1]{fontenc}
\usepackage[utf8]{inputenc}
\usepackage{lmodern}
\usepackage{microtype}
\usepackage{csquotes}
\usepackage{verbatim}  % Im Kopf des Dokuments einfügen
\usepackage{geometry}
\usepackage{fancyhdr}
\usepackage{amsmath, amssymb, amsthm}  % Mathe
\usepackage{mathtools}                 % \coloneqq, \xrightarrow
\usepackage{bm}                        % Fette Symbole (\bm{B} für Magnetfeld)
\usepackage{siunitx}                   % \SI{1.23}{\meter\per\second}
\usepackage{graphicx}                  % \includegraphics
\usepackage{subcaption}                % Unterabbildungen
\usepackage{booktabs}                  % Professionelle Tabellen
\usepackage{tikz}                      % Für Diagramme
\usepackage{xcolor}                    % Farbige Tabellenzellen
\usepackage[
    backend=biber,
    style=phys,         % APS-Zitierstil (für Physik)
    sorting=nyt,        % Sortierung: Name, Jahr, Titel
]{biblatex}
\usepackage[acronym, toc]{glossaries}
\usepackage{hyperref}
\usepackage{parskip}
\usepackage{pgfplots}
\usepackage{glossaries}
\makeglossaries
\geometry{
    a4paper,
    top=25mm,
    inner=30mm,    % Bundsteg (größerer Rand für Buchbindung)
    outer=25mm,
    bottom=30mm,
    headheight=15pt,
}

\pagestyle{fancy}
\fancyhf{}
\fancyhead[LE,RO]{\thepage}
\fancyhead[RE]{\leftmark}    % Kapitelname (gerade Seiten)
\fancyhead[LO]{\rightmark}   % Abschnittname (ungerade Seiten)
\renewcommand{\headrulewidth}{0.4pt}

\theoremstyle{definition}
\newtheorem{definition}{Definition}[chapter]
\newtheorem{law}{Physikalisches Gesetz}[chapter]
\theoremstyle{plain}
\newtheorem{theorem}{Theorem}[chapter]
\newtheorem{lemma}[theorem]{Lemma}
\theoremstyle{remark}
\newtheorem{remark}{Bemerkung}[chapter]

\hypersetup{
    colorlinks=true,
    linkcolor=blue,
    citecolor=black,
    urlcolor=black,
    pdftitle={WDB-Theorie - Eine effektive Quantengravitation},
    pdfauthor={Dipl.-Ing. (FH) Michael Czybor},
}

\addbibresource{literatur.bib}  % Ihre .bib-Datei
\makeglossaries

\setlength{\headheight}{26.76852pt}

\newacronym{qm}{QM}{Quantenmechanik}
\newacronym{art}{ART}{Allgemeine Relativitätstheorie}
\newacronym{srt}{SRT}{Spezielle Relativitätstheorie}
\newacronym{cmb}{CMB}{Hintergrundstrahlung}
\newacronym{qed}{QED}{Quantenelektrodynamik}
\newacronym{qft}{QFT}{Quantenfeldtheorie}
\newacronym{epr}{EPR-Paradoxon}{Einstein-Podolsky-Rosen-Paradoxon}
\newacronym{wg}{WG}{Weber-Gravitation}
\newacronym{wed}{WED}{Weber-Elektrodynamik}
\newacronym{dbt}{DBT}{De-Broglie-Bohm-Theorie}
\newacronym{wdbt}{WDBT}{Weber-De Broglie-Bohm-Theorie}
\newacronym{mt}{MT}{Maxwell-Theorie}
\newacronym{iwt}{IWT}{Informations-Weber-Theorie}
\newacronym{dstt}{DSTT}{Dynamischen Schwere-Trägheits-Theorie}

\newglossaryentry{gls:quantenmechanik}
{
    name={Quantenmechanik},
    description={Theorie der Materie und Strahlung auf atomarer und subatomarer Ebene}
}
\newglossaryentry{gls:hamiltonian}
{
    name={\ensuremath{\mathcal{H}}},
    description={Hamilton-Operator, beschreibt die Gesamtenergie eines Systems},
    sort={hamiltonian}
}


\begin{document}

\frontmatter
\begin{tikzpicture}[remember picture, overlay]

  % Hintergrund (Dunkel mit fraktalem Gitter)
  \fill[hintergrund] (current page.south west) rectangle (current page.north east);
  \foreach \i in {0,10,...,360} {
    \draw[quantenblau!10, line width=0.1pt] 
      (current page.center) -- +(\i:5cm);
  }

  % Dodekaeder (abstrahiert)
  \node[rotate=25, scale=2, quantenblau!50] at (current page.center) {
    \begin{tikzpicture}[scale=0.3]
      \draw[quantenblau] (0:1) \foreach \a in {72,144,...,360} { -- (\a:1) } -- cycle;
      \foreach \a in {36,108,...,324} { \draw[quantenblau] (0,0) -- (\a:1.6); }
    \end{tikzpicture}
  };

  % Titeltext (mit Schatten-Effekt)
  \node[align=center, text=white, font=\sffamily\bfseries\Huge] 
    at ($(current page.center)+(0,3cm)$) {
    \textbf{Emergence of Cosmology}
  };
  \node[align=center, text=quantenblau!80, font=\sffamily\Large] 
    at ($(current page.center)+(0,1.8cm)$)
    {
        The WDBT as a Primal Theory
    };

  % Kernformeln (rechts unten)
  \node[align=left, anchor=south east, text=weberrot!70, font=\small] 
    at ($(current page.south east)+(-1cm,1cm)$) {
    $\displaystyle \vec{F}_{\text{WG}} = -\frac{GMm}{r^2}\left(1-\frac{\dot{r}^2}{c^2}+\beta\frac{r\ddot{r}}{c^2}\right)$
  };
  \node[align=left, anchor=north east, text=quantenblau!70, font=\small] 
    at ($(current page.south east)+(-1cm,3cm)$) {
    $\displaystyle Q = -\frac{\hbar^2}{2m}\frac{\nabla^2\sqrt{\rho}}{\sqrt{\rho}}$
  };

  % Autor (unten mittig)
  \node[align=center, text=white, font=\sffamily\large] 
    at ($(current page.south)+(0,1cm)$) {
    \textbf{Michael Czybor}
  };

  % Fraktale Dimension (links oben)
  \node[align=right, text=quantenblau!50, font=\small] 
    at ($(current page.north west)+(2cm,-1cm)$) {
    $D = \frac{\ln 20}{\ln(2+\phi)} \approx 2.71$
  };

\end{tikzpicture}

\title{Emergence of Cosmology\\The WDBT as a Primal Theory}
\author{Michael Czybor}
\date{\today}
\maketitle

\chapter*{Preface}
For over a century, the theories of relativity and quantum mechanics have formed the pillars of our physical worldview. Yet, despite their undisputed empirical successes, conceptual questions remain open: What is the nature of gravity? How can the quantum world be reconciled with classical physics? And what lies behind phenomena such as inertia or the propagation of light?

This book presents a radically new approach: the \textbf{\gls{wdbt}}. It unifies two previously largely independent theories – \gls{wed} and De Broglie-Bohm quantum mechanics – into a comprehensive primal theory from which the known physical laws \textbf{emerge}. This means: Special and General Relativity, Maxwell's equations, and even quantum mechanics are not postulated but derived from the fundamental principles of the \gls{wdbt}.

Particular attention should be paid to the \textbf{convergent emergence of \gls{art}}. Conventional \gls{art} suffers from the problem of singularities – black holes and the Big Bang mark its limits. A natural extension, \gls{art}+, introduces the Bohmian quantum potential and solves these problems: it is singularity-free, deterministic, and non-local. Thus, it negates the Big Bang and replaces it with a "Big Bounce". \gls{art}+ represents the closest approximation of the geometric description to the force-based \gls{wdbt} – except for one crucial, experimentally verifiable difference: \textbf{frequency-dependent light deflection}. While \gls{art}+ provides a purely geometric and thus frequency-independent description, the \gls{wdbt} as a dynamic interaction theory predicts a frequency dependence. This makes \gls{art}+ a form of \textbf{geometrically based quantum gravity} that comes conceptually remarkably close to the \gls{wdbt} without fully replacing it.

But the \gls{wdbt} goes even further. While the "analog" \gls{wdbt} already achieves a complete emergence of known physics, the \textbf{digital \gls{wdbt}} opens the door to an even more fundamental level. It explains not only the interactions but also the nature constants and the structure of spacetime itself – based on a digital-fractal space model with a dimension of about $D \approx 2.71$. In this theory, both the Weber force and the quantum potential are generated from the same cause: the discrete, non-local structure of space.

This book aims to systematically derive the emergence of all modern physics from the \gls{wdbt} and present its conceptual and empirical advantages. It is aimed at physicists, philosophers, science students, and all those seeking a coherent, realistic, and deterministic foundation for physics. The \gls{wdbt} offers not only a unification but also an \textbf{ontological deepening} of our understanding of space, time, matter, and interaction.

\begin{flushright}
    Michael Czybor \\
    \emph{Langenstein/AT, August 2025}
\end{flushright}

\tableofcontents
\listoffigures
\listoftables

\mainmatter
\chapter{Einführung}
\section{Plasmen als Schlüssel zu einer neuen Physik}
Seit über einem Jahrhundert dominieren Feldtheorien das Denken – von den Maxwell-Gleichungen bis zur \gls{qed}. Doch gerade dort, wo diese Theorien an ihre Grenzen stoßen, in der
Welt der Plasmen, offenbart sich eine tiefere Wahrheit: \textbf{Die Natur kennt keine Felder}. Was wir als elektromagnetische Wechselwirkungen interpretieren, ist in Wirklichkeit ein
komplexes Geflecht direkter, nicht-lokaler Kräfte zwischen Teilchen – eine Erkenntnis, die bereits in der \gls{wed} \cite{Weber1846} angelegt ist und durch die \gls{dbt} \cite{bohm1952}
ihre volle Bedeutung erlangt.

\section{Das kosmische Plasma: Eine Herausforderung für die Standardmodelle}
Im großen Maßstab des Universums zeigt sich das Versagen der Feldtheorien besonders deutlich. Die kosmische \gls{cmb}, oft als Beweis für den Urknall gefeiert, könnte
ebenso gut das thermische Gleichgewicht eines unendlichen, statischen Plasmauniversums beschreiben. Die Rotverschiebung ferner Galaxien, die heute als Indiz für die Expansion des
Raumes gedeutet wird, lässt sich alternativ durch Energieverluste des Lichts in intergalaktischen Plasmen erklären – ein Prozess, den die \gls{wed} präziser beschreibt
als die \gls{art} \cite{einstein1915}.

Die rätselhaften Rotationskurven der Galaxien, die zur Erfindung der dunklen Materie führten, finden in der Plasma-Kosmologie eine natürliche Erklärung: Elektromagnetische Kräfte,
modifiziert durch die Geschwindigkeitsabhängigkeit der Weber-Wechselwirkung, können die beobachteten Geschwindigkeitsprofile erzeugen, ohne auf unsichtbare Teilchen zurückgreifen
zu müssen. Die filamentären Strukturen des kosmischen Netzes, die sich über Hunderte von Millionen Lichtjahren erstrecken, ähneln verblüffend den Mustern, die in
Plasmadynamik-Experimenten auf Laborskala entstehen – ein Hinweis darauf, dass das Universum in seinem Wesen ein elektrisches Phänomen ist.

\subsection{Sternentstehung und Plasmadynamik}
Auch die Geburt der Sterne wirft Fragen auf, die das Feldparadigma nicht befriedigend beantworten kann. Wie können interstellare Wolken aus diffusem Plasma unter ihrer eigenen
Gravitation kollabieren, wenn die elektromagnetischen Abstoßungskräfte um Größenordnungen stärker sind? Die \gls{wdbt} hingegen bietet eine elegante Lösung: Das Quantenpotential der \gls{dbt}
wirkt als nicht-lokale, stabilisierende Kraft, die den Kollaps trotz der elektromagnetischen Barrieren ermöglicht. Gleichzeitig erklärt die Weber-Gravitation mit ihrer geschwindigkeitsabhängigen
Komponente, warum protoplanetare Scheiben rotationsstabil bleiben, ohne dass dunkle Materie als \enquote{Klebstoff} benötigt wird. Details hierzu können dem Anhang (\ref{app:sternentstehung})
entnommen werden.

Die Herausforderung der Sternentstehung liegt im scheinbaren Widerspruch zwischen der enormen elektromagnetischen Abstoßung geladener Teilchen in interstellaren Wolken und der
vergleichsweise schwachen Gravitation, die den Kollaps einleiten soll. Während klassische Modelle auf zusätzliche Annahmen wie magnetische Stabilisierung oder Turbulenzdämpfung
zurückgreifen müssen, bietet die \gls{wdbt} eine elegante Lösung durch das Zusammenspiel des Quantenpotentials und der Weber-Gravitation.

Das Quantenpotential wirkt hier nicht nur als quantenmechanische Korrektur, sondern als entscheidender Vermittler zwischen mikroskopischen und makroskopischen Prozessen. Indem es
die Teilchen in kohärenten, geordneten Bahnen hält, verhindert es die sonst dominierende elektromagnetische Abstoßung und ermöglicht eine großräumige Verdichtung der Wolke.
Gleichzeitig stabilisiert es die Struktur gegen turbulente Fragmentierung, ohne den Kollaps selbst zu blockieren – im Gegensatz zu klassischen Modellen, die solche Effekte nur
durch externe Mechanismen erklären können.

Die Weber-Gravitation ergänzt diesen Prozess, indem ihre geschwindigkeitsabhängigen Terme eine rotationsstabile Kontraktion der Wolke bewirken. Dadurch entsteht ein
selbstorganisierter Kollaps, der weder auf hypothetische dunkle Materie noch auf ad-hoc-Annahmen angewiesen ist. Die fraktale Struktur des Plasmas, die sich natürlich aus der
\gls{wdbt} ergibt, erklärt zudem die hierarchische Anordnung von Sternentstehungsregionen in Filamenten – ein Phänomen, das in herkömmlichen Theorien nur schwer abzubilden ist.

Kurz gesagt: Die \gls{wdbt} zeigt, dass Sternentstehung kein Kampf zwischen Gravitation und elektromagnetischen Kräften ist, sondern ein koordinierter Prozess, der durch
nicht-lokale Quanteneffekte und direkte Teilchenwechselwirkungen gesteuert wird. Dieses Bild passt nicht nur besser zu Beobachtungen, sondern vermeidet auch die willkürlichen
Zusatzannahmen der etablierten Modelle.

\subsection{Kernfusion: Vom ITER zum feldlosen Plasma}
Auf der irdischen Skala zeigt sich das Potential der neuen Sichtweise vielleicht am deutlichsten in der Fusionsforschung. Seit Jahrzehnten kämpfen Projekte wie ITER mit den
Unwägbarkeiten der Plasmaturbulenz – einem Problem, das im Rahmen der \gls{mhd} unlösbar erscheint. Doch was, wenn die Turbulenz gar kein chaotisches Phänomen ist,
sondern die Manifestation einer tieferen, nicht-lokalen Ordnung?

Die \gls{wdbt} legt nahe, dass Plasmen in Fusionsreaktoren nicht durch äußere Magnetfelder kontrolliert werden müssen, sondern sich selbst organisieren können – gesteuert durch
das Quantenpotential und die direkten Teilchenwechselwirkungen der \gls{wed}. Es gibt Hinweise dafür, dass Plasmen in dieser Beschreibung stabilere Konfigurationen
einnehmen, als die Feldtheorie vorhersagt. Sollte sich dies bestätigen, könnte es den Weg zu kompakteren, effizienteren Fusionsreaktoren ebnen – eine Revolution der Energiegewinnung.

Die Kernfusion gilt seit Jahrzehnten als vielversprechende Lösung für die Energieprobleme der Menschheit, doch die technischen Herausforderungen bleiben immens. Projekte wie ITER oder
Wendelstein 7-X setzen auf die \gls{mhd}, um Plasmen bei extrem hohen Temperaturen (über 100 Millionen Grad) einzuschließen. Doch trotz enormer Fortschritte kämpfen diese Anlagen mit
unkontrollierbarer Turbulenz, anomalem Teilchentransport und instabilen Plasmarändern – Probleme, die sich mit den klassischen Modellen nur unzureichend beschreiben lassen. Hier setzt
die \gls{wdbt} an und bietet einen radikal neuen Ansatz, der die Fusion revolutionieren könnte.

\subsubsection{Die Grenzen der MHD in der Fusionsforschung}
Die \gls{mhd} beschreibt Plasmen als kontinuierliche Fluide, die durch Magnetfelder geformt werden. Doch diese Näherung vernachlässigt mikroskopische Effekte wie Teilchenkorrelationen
oder nicht-lokale Wechselwirkungen – genau jene Phänomene, die in Fusionsplasmen eine entscheidende Rolle spielen. Turbulenz und anomaler Widerstand entstehen, weil die Lorentzkraft der
\gls{mhd} die komplexe Dynamik geladener Teilchen nur unvollständig erfasst. Die Folge sind unvorhersehbare Energieverluste und instabile Plasmen, die den Betrieb von Tokamaks oder
Stellaratoren erschweren.

\subsubsection{Die WDBT als Alternative: Mikroskopische Fundierung und Selbstorganisation}
Die \gls{wdbt} löst diese Probleme, indem sie Plasmen nicht als Fluide, sondern als Systeme direkt wechselwirkender Teilchen beschreibt. Die Weber-Kraft (Gl. 2.2) berücksichtigt nicht
nur die Coulomb-Wechselwirkung, sondern auch geschwindigkeits- und beschleunigungsabhängige Terme, die in der \gls{mhd} fehlen. Dadurch erfasst sie kollektive Phänomene wie Plasmawellen oder
Turbulenz präziser. Besonders relevant ist das Bohm’sche Quantenpotential (Gl. 2.4), das nicht-lokale Korrelationen zwischen Teilchen beschreibt und in dichten Plasmen eine stabilisierende
Wirkung entfaltet. Experimente in Wendelstein 7-X zeigen bereits, dass Plasmen bei hohen Dichten ($n_e > 10^{20}m^{-3}$) stabiler sind als die \gls{mhd} vorhersagt – ein Effekt, den die \gls{wdbt}
durch den Quantenterm $Q$ natürlich erklärt.

\subsubsection{Praktische Vorteile: Kompaktere Reaktoren und effizientere Plasmen}
Die \gls{wdbt} bietet konkrete Vorteile für die Fusionsforschung:

\begin{enumerate}
    \item \textbf{Selbstorganisierte Stabilität:}\\Das Quantenpotential $Q$ wirkt wie eine intrinsische Dämpfung, die Instabilitäten wie Edge-Localized Modes (ELMs) unterdrücken kann. Dadurch könnten aufwendige Magnetfeldspulen teilweise überflüssig werden.
    \item \textbf{Reduzierter anomaler Transport:}\\Die Weber-Kraftdichte (Gl. 2.7) beschreibt den Teilchentransport durch Paarkorrelationen, nicht durch statistische Turbulenzmodelle. Dies könnte Energieverluste minimieren und die Einschlusszeiten verlängern.
    \item \textbf{Filamentäre Strukturen:}\\Die fraktale Skalierung von Birkeland-Strömen (Gl. 2.14) legt nahe, dass sich Plasmen in Fusionsreaktoren selbstorganisieren könnten – ähnlich wie in astrophysikalischen Phänomenen. Dies würde kompaktere Reaktordesigns ermöglichen.
\end{enumerate}

\subsubsection{Experimentelle Perspektiven}
Um das Potenzial der \gls{wdbt} auszuschöpfen, sind gezielte Experimente nötig:

\begin{itemize}
    \item \textbf{Quantenpotential-Effekte:}\\Hochdichte-Experimente (z. B. SPARC) könnten den Einfluss von $Q$ auf Plasmawellen direkt messen.
    \item \textbf{Nicht-lokaler Transport:}\\Präzise Messungen des anomalen Widerstands in Tokamaks könnten die Vorhersagen der \gls{wdbt} validieren.
    \item \textbf{Filamentbildung:}\\Laborexperimente mit Z-Pinch-Anordnungen sollten die fraktale Skalierung (Gl. 2.14) überprüfen.
\end{itemize}

\subsubsection{Fazit: Ein Paradigmenwechsel in der Fusionsforschung}
Die \gls{wdbt} bietet nicht nur eine theoretische Alternative zur \gls{mhd}, sondern auch praktische Lösungen für die hartnäckigsten Probleme der Fusionsforschung. Durch ihre mikroskopische Fundierung
und die Einbeziehung nicht-lokaler Quanteneffekte könnte sie den Weg zu stabileren, effizienteren Fusionsreaktoren ebnen – und damit die Vision einer sauberen, unerschöpflichen Energiequelle
Wirklichkeit werden lassen. Die experimentelle Validierung dieser Vorhersagen wird entscheiden, ob die \gls{wdbt} die Fusionsforschung tatsächlich in ein neues Zeitalter führen kann.

\subsection{Die Anwendungen: Von der Medizin zur Raumfahrt}
Die Konsequenzen dieser neuen Physik reichen weit über die Grundlagenforschung hinaus. In der Plasmamedizin, wo kalte Plasmen zur Wundheilung eingesetzt werden, könnte die
\gls{wed} erklären, warum bestimmte Plasma-Konfigurationen biologisch wirksamer sind als andere – nicht wegen der Feldstärke, sondern aufgrund der spezifischen,
nicht-lokalen Wechselwirkung mit Gewebemolekülen.

In der Raumfahrtantriebstechnik zeigen Plasmantriebe wie der VASIMR bereits heute, dass hohe spezifische Impulse möglich sind – doch ihre Effizienz bleibt hinter den theoretischen
Grenzen zurück. Die WDBT bietet hier einen neuen Ansatz: Wenn die Strahlbeschleunigung nicht durch Felder, sondern durch direkt wirkende Weber-Kräfte erfolgt, könnten völlig neue
Antriebskonzepte entstehen, die das Zeitalter der interplanetaren Raumfahrt einläuten.

\section{Hybrid-Plasmaantrieb: Thermoelektrische Resonanzexpansion}
\label{sec:hybrid_antrieb}

Die Kombination kryogener Treibstoffe mit Weber-De-Broglie-Bohm-Elektrodynamik (WDBT) führt zu einem neuartigen Antriebskonzept, das die Vorteile chemischer und elektrischer Systeme vereint.

\subsection{Physikalische Grundlagen}
\label{subsec:grundlagen}

Für ein flüssiges Ionengas mit Teilchendichte $n_e$ gilt die \textbf{erweiterte Zustandsgleichung}:

\begin{equation}
p = \underbrace{n_e k_B T_e}_{\text{thermisch}} 
+ \underbrace{\frac{e^2 n_e^{4/3}}{4\pi \epsilon_0} \left(1 + \beta \frac{v^2}{c^2}\right)}_{\text{WDBT-Korrektur}}
\label{eq:druck}
\end{equation}

mit $\beta = 2$ für die Weber-Kraft. Die \textbf{kritische Dichte} für Dominanz des Coulomb-Drucks liegt bei:

\begin{equation}
n_c = \left(\frac{4\pi \epsilon_0 k_B T_e}{e^2}\right)^3 \approx 10^{28}\,\text{m}^{-3}\quad\text{(für }T_e=10^4\,\text{K)}
\end{equation}

\subsection{Resonanzbedingungen}
\label{subsec:resonanz}

Das System verhält sich analog zu einem Helmholtz-Resonator mit\\\textbf{Plasma-Resonanzfrequenz}:

\begin{equation}
f_r = \frac{c_s}{2\pi}\sqrt{\frac{A_d}{V_c L_d}} \quad \text{mit} \quad c_s = \sqrt{\gamma \left(\frac{k_B T_e}{m_i} + \frac{\hbar^2}{4m_e m_i}\frac{\nabla^2 n_e}{n_e}\right)}
\label{eq:resonanz}
\end{equation}

\subsection{Energietransferanalyse}
\label{subsec:energie}

Die \textbf{Energiedichteskalierung} zeigt den WDBT-Vorteil:

\begin{table}[h]
\centering
\caption{Vergleich der Energiedichten}
\label{tab:energie}
\begin{tabular}{lcc}
\toprule
Treibstofftyp & $E$ [MJ/kg] & $p_{\text{max}}$ [GPa] \\
\midrule
TNT & 4.6 & 20 \\
Flüssiger Wasserstoff & 142 & 25 \\
WDBT-Plasma (LH$_2$) & 175 & 175 \\
\bottomrule
\end{tabular}
\end{table}

\subsection{Technische Umsetzung}
\label{subsec:tech}

Die \textbf{optimale Düsengeometrie} folgt der fraktalen Skalierung:

\begin{equation}
\frac{dA}{dx} = -A^{1-1/D} \quad \text{mit} \quad D = \frac{\ln 20}{\ln(2+\phi)} \approx 2.71
\label{eq:duese}
\end{equation}

Die Stabilitätsbedingung für den \textbf{Quanten-Federeffekt} lautet:

\begin{equation}
\tau_{\text{ion}} > \sqrt{\frac{m_e}{e^2 n_e^{2/3}}} \approx 10^{-11}\,\text{s}\quad\text{(für }n_e=10^{28}\,\text{m}^{-3)}
\end{equation}

\begin{remark}
Die magnetische Steuerung erfolgt durch ein \textbf{radiales $B$-Feld} mit:
\[
B > \frac{m_i v_{\text{exp}}}{e r_d} \approx 0.5\,\text{T}\quad\text{(für }r_d=1\,\text{cm)}
\]
\end{remark}

\subsection{Experimentelle Validierung}
\label{subsec:experiment}

Messgrößen zur Bestätigung der WDBT-Effekte:

\begin{itemize}
\item \textbf{Expansionsgeschwindigkeit}:
\[
\frac{\Delta v}{v_{\text{klassisch}}} = \sqrt{1 + \frac{Q}{k_B T_e}} - 1
\]

\item \textbf{Spektrale Dichtemodulation}:
\[
\left.\frac{\delta n_e}{n_e}\right|_{\text{res}} \propto \frac{\hbar}{m_e c_s^2 \tau_{\text{ion}}}
\]
\end{itemize}

\subsection*{Zusammenfassung}
Das Konzept kombiniert erstmals:
\begin{enumerate}
\item Kryogene Energiespeicherung,
\item Elektrostatische Druckverstärkung,
\item Nicht-lineare WDBT-Resonanz.
\end{enumerate}

\subsection{Das Prinzip des Hybrid-Plasmaantriebs}
Die Idee eines Antriebssystems, das die Vorteile chemischer Expansion und elektrostatischer Plasmabeschleunigung vereint, basiert auf einem tiefen Verständnis der Wechselwirkungen zwischen kryogener
Materie und Quantenpotentialen. Stellen Sie sich einen extrem komprimierten flüssigen Wasserstofftank vor, der schlagartig ionisiert wird. Durch die Ionisation entstehen zwei simultane Effekte: Erstens
die klassische thermische Expansion des nun heißen Plasmas, zweitens eine viel stärkere elektrostatische Abstoßung der Ionen untereinander. Diese Coulomb-Explosion wird in der \gls{wdbt} durch die
geschwindigkeitsabhängige Weber-Kraft noch verstärkt – ähnlich wie eine Feder, die nicht nur durch ihre Spannung, sondern zusätzlich durch resonante Schwingungen Energie freisetzt.

Der Schlüssel zur Kontrolle dieses Systems liegt in der präzisen Abstimmung der Resonanzbedingungen. Wie bei einem perfekt konstruierten Bassreflex-Lautsprecher muss das Verhältnis von Kammervolumen
zur Düsengeometrie so gewählt werden, dass die natürliche Schwingungsfrequenz des Plasmas mit der Ionisationsrate synchronisiert ist. Das Quantenpotential Q wirkt hierbei als aktiver Dämpfer, der
chaotische Turbulenzen unterdrückt und die Energie in eine kohärente Expansionswelle umlenkt. Praktisch erreicht man dies durch eine fraktale Düsenform, deren Verzweigungsmuster
(Skalierungsexponent $D \approx 2.71$) genau der nicht-lokalen Korrelationslänge des Plasmas entspricht.

Die daraus resultierende Schubkraft übertrifft konventionelle Systeme durch einen einzigartigen Mechanismus: Während chemische Triebwerke durch die Bindungsenergie von Molekülen begrenzt sind und
elektrische Antriebe durch magnetische Sättigungseffekte, nutzt dieser Hybridantrieb die kollektive Quantennatur des Plasmas selbst. Die Ionen beschleunigen nicht isoliert, sondern als kohärentes
Ganzes, dessen Dynamik durch das Bohm'sche Potential gesteuert wird. Magnetfelder dienen dabei nur noch zur Feinjustierung der Ausbreitungsrichtung, nicht mehr zur primären Energieübertragung.

Experimentell manifestiert sich dieser Effekt in charakteristischen Signalen: Eine um 20-30\% erhöhte Expansionsgeschwindigkeit gegenüber klassischen Vorhersagen, sowie typische Dichtemodulationen
im Ultraschallbereich (50-100 kHz), die direkt mit der fraktalen Dimension $D$ korrelieren. Die technische Umsetzung erfordert zwar präzise Steuerung der Ionisationsfront (Nanosekunden-Laserpulse),
ermöglicht aber kompaktere Bauformen als herkömmliche Plasmatriebwerke – bei gleichzeitig höherem spezifischem Impuls.

Diese Synergie aus kryogener Speicherung, elektrostatischer Explosion und Quantenkohärenz markiert einen Paradigmenwechsel in der Antriebstechnik, der nur durch die \gls{wdbt} vollständig erklärbar
ist. Sie zeigt, wie scheinbar getrennte physikalische Prinzipien in Wirklichkeit Aspekte einer tieferen, einheitlichen Beschreibung sind – jenseits der klassischen Feldtheorien.

\subsubsection{Der Ionisationsantrieb: Eine Alternative zur klassischen Verbrennung}
Im Gegensatz zu herkömmlichen Verbrennungsprozessen, bei denen chemische Reaktionen wie die Oxidation von Wasserstoff genutzt werden, setzt der hier beschriebene Antrieb ausschließlich auf
Ionisation – also die Umwandlung von neutralen Gasatomen oder -molekülen in geladene Teilchen (Plasma). Während eine Verbrennung Energie durch die Umwandlung von Molekülbindungen freisetzt, beruht der
Ionisationsantrieb auf elektrodynamischen und quantenmechanischen Effekten.

\textbf{Schlüsselunterschiede:}
\begin{enumerate}
    \item \textbf{Keine chemische Reaktion nötig}
        \begin{itemize}
            \item Herkömmliche Triebwerke benötigen einen Oxidator (z. B. Sauerstoff), um den Treibstoff zu verbrennen.
            \item Beim Ionisationsantrieb wird das Gas (z. B. Wasserstoff) durch elektrische oder laserinduzierte Ionisation direkt in Plasma umgewandelt – ohne Flamme oder chemische Reaktionsprodukte.
        \end{itemize}
    \item \textbf{Energiefreisetzung durch Coulomb-Explosion}
        \begin{itemize}
            \item Beim Ionisieren entstehen positiv geladene Ionen, die sich gegenseitig abstoßen.
            \item Diese elektrostatische Abstoßung erzeugt einen extrem schnellen Expansionsdruck – viel stärker als bei thermischer Verbrennung.
        \end{itemize}
    \item \textbf{Quantenmechanische Stabilisierung}
        \begin{itemize}
            \item Das Bohm’sche Quantenpotential ($Q$) verhindert, dass das Plasma instabil wird oder unkontrolliert expandiert.
            \item Dadurch lässt sich die Energie gezielt in Schub umwandeln, statt in eine ungerichtete Druckwelle.
        \end{itemize}
\end{enumerate}

\textbf{Vorteile gegenüber Verbrennung}
\begin{itemize}
    \item \textbf{Höhere Effizienz:}\\Die Coulomb-Abstoßung kann mehr Energie pro Kilogramm Treibstoff freisetzen als chemische Reaktionen.
    \item \textbf{Sauberer Betrieb:}\\Keine Verbrennungsrückstände (nur ionisierte Teilchen, die im Vakuum neutralisiert werden).
    \item \textbf{Präzise Steuerung:}\\Die Expansion kann durch Magnetfelder oder das Quantenpotential gesteuert werden.
    \item \textbf{Gewichtsreduktion:}\\Es muss kein Sauerstoff für die Verbrennung mitgeführt werden.
\end{itemize}

Es handelt sich hier nicht um eine Verbrennung, sondern um einen elektrodynamisch getriebenen Prozess, der Plasmen nutzt, um Schub zu erzeugen. Diese Methode könnte Antriebssysteme
revolutionieren – von Raumschiffen bis hin zu neuen Energieumwandlungskonzepten.

\textbf{Zusammenfassend:} \textit{Ionisation ersetzt die Flamme – und Quantenphysik sorgt für die Kontrolle.}

\section{Eine neue Ära der Physik}
Dieses Buch wird zeigen, dass die Vereinigung von \gls{wed}, \gls{dbt} und Plasmaphysik mehr ist als eine akademische Übung – es ist der Schlüssel zu
einem neuen Verständnis des Universums. Von den größten kosmischen Strukturen bis hin zur Kontrolle von Fusionsplasmen eröffnet sich eine Welt jenseits der Quantenfelder, in der
die Natur nicht durch abstrakte Feldgleichungen, sondern durch reale, messbare Wechselwirkungen beschrieben wird.

Die kommenden Kapitel werden diese Vision mit mathematischer Strenge und experimentellen Belegen untermauern. Die Reise beginnt mit den Grundlagen – einer feldlosen Beschreibung
der Plasmadynamik, die zeigt, warum die \gls{wdbt} nicht nur eine Alternative, sondern die logisch konsistentere Theorie ist.

\chapter{Die Emergenz der Maxwell-Gleichungen}
\label{ch:maxwell}
\section{Grundlagen der Weber-Elektrodynamik}
\label{sec:grundlagen}
Die \gls{wed} postuliert eine instantane, geschwindigkeits- und beschleunigungsabhängige Kraft zwischen zwei Ladungen $q_1$ und $q_2$. Die vektorielle Form der Weber-Kraft lautet:

\begin{equation}
    \vec{F}_{12} = \frac{q_1 q_2}{4\pi\epsilon_0 r^2} \left\{ \left[ 1 - \frac{v^2}{c^2} + \frac{2r(\hat{\vec{r}}\cdot\vec{a})}{c^2} \right] \hat{\vec{r}} + \frac{2(\hat{\vec{r}}\cdot\vec{v})}{c^2} \vec{v} \right\}
\end{equation}

wobei:

\begin{itemize}
    \item $r = \left| \vec{r} \right|$ der Abstand zwischen den Ladungen,
    \item $\hat{\vec{r}} = \frac{\vec{r}}{r}$ der Einheitsvektor in Richtung von $q_1$ nach $q_2$,
    \item $\vec{v} = \dot{\vec{r}}$ die Relativgeschwindigkeit,
    \item $\vec{a} = \ddot{\vec{r}}$ die Relativbeschleunigung,
    \item $c$ die charakteristische Grenzgeschwindigkeit der Wechselwirkung ist.
\end{itemize}

Diese Kraft kann aus einem verallgemeinerten Potential abgeleitet werden und erfüllt die Energie- und Impulserhaltung.

\section{Superposition und Gesamtkraft auf eine Testladung}
Für eine Testladung $q$ im Feld von $N$ anderen Ladungen $q_i (i = 1,...,N)$ gilt das Superpositionsprinzip. Die Gesamtkraft auf $q$ ist:

\begin{equation}
    \label{eq:gesamtkraft}
    \vec{F}_{\text{ges}} = q \sum_{i=1}^N \frac{q_i}{4\pi\epsilon_0 r_i^2} \left\{ \left[ 1 - \frac{v_i^2}{c^2} + \frac{2r_i(\hat{r}_i\cdot\vec{a}_i)}{c^2} \right] \hat{r}_i + \frac{2(\hat{r}_i\cdot\vec{v}_i)}{c^2} \vec{v}_i \right\}
\end{equation}

wobei $\vec{r_i}$ der Vektor von $q$ zu $q_i$ ist.

\section{Definition der effektiven Felder}
Die Gesamtkraft aus den effektiven Feldern $\vec{E}$ und $\vec{B}$ lässt sich in die Form der Lorentz-Kraft bringen:

\begin{equation}
    \label{eq:lorentz_kraft}
    \vec{F} = q \left( \vec{E} + \vec{v} \times \vec{B} \right)
\end{equation}

Durch Koeffizientenvergleich ergeben sich die Definitionen der effektiven Felder:

\begin{equation}
    \vec{E} = \sum_{i=1}^N \frac{q_i}{4\pi\epsilon_0 r_i^2} \left[ 1 - \frac{v_i^2}{c^2} + \frac{2r_i(\hat{r}_i\cdot\vec{a}_i)}{c^2} \right] \hat{r}_i
\end{equation}

\begin{equation}
    \vec{B} = \sum_{i=1}^N \frac{q_i}{4\pi\epsilon_0 r_i^2 c^2} \cdot 2(\hat{r}_i\cdot\vec{v}_i) \vec{v}_i
\end{equation}

Diese effektiven Felder sind mathematische Hilfsgrößen, die die gemittelte Wirkung aller anderen Ladungen beschreiben.

\section{Kontinuumslimes und Feldgleichungen}
\label{sec:kontinuumslimes}
Bei einer kontinuierlichen Ladungsverteilung mit Dichte $\rho(\vec{r}, t)$ und Stromdichte $\vec{j}(\vec{r},t)$ gehen die Summen in Integrale über. Die Felder werden zu:

\begin{equation}
    \vec{E}(\vec{r}, t) = \frac{1}{4\pi\epsilon_0} \int \rho(\vec{r}~', t) \left[ 1 - \frac{v^2}{c^2} + \frac{2r(\hat{\vec{r}}\cdot\vec{a})}{c^2} \right] \frac{\hat{\vec{r}}}{r^2}  d^3r~'
\end{equation}

\begin{equation}
    \vec{B}(\vec{r}, t) = \frac{1}{4\pi\epsilon_0 c^2} \int \vec{j}(\vec{r}~', t) \cdot 2(\hat{\vec{r}}\cdot\vec{v}) \frac{\hat{\vec{r}}}{r^2}  d^3r~'
\end{equation}

% Variablenerklärung für die Integrale
\begin{align*}
&\vec{r} && \text{Ortsvektor zum Aufpunkt} \\
&\vec{r}~' && \text{Ortsvektor zur Quellladung} \\
&\vec{r} = \vec{r} - \vec{r}~' && \text{Abstandsvektor} \\
&r = |\vec{r} - \vec{r}~'| && \text{Abstand} \\
&\hat{\vec{r}} = \frac{\vec{r} - \vec{r}~'}{|\vec{r} - \vec{r}~'|} && \text{Einheitsvektor} \\
&\rho(\vec{r}~', t) && \text{Ladungsdichte am Quellpunkt} \\
&\vec{j}(\vec{r}~', t) && \text{Stromdichte am Quellpunkt} \\
&d^3r~' && \text{Volumenelement im Quellraum}
\end{align*}

\subsection{Gauß'sches Gesetz}
\begin{equation}
    \nabla \cdot \vec{E} = \frac{\rho}{\epsilon_0}
\end{equation}

\subsection{Gauß'sches Gesetz für den Magnetismus}
\begin{equation}
    \nabla \cdot \vec{B} = 0
\end{equation}

\subsection{Faraday'sches Induktionsgesetz}
\begin{equation}
    \nabla \times \vec{E} = -\frac{\partial \vec{B}}{\partial t}
\end{equation}

\subsection{Maxwell'scher Verschiebungsstrom}
\begin{equation}
    \nabla \times \vec{B} = \mu_0 \vec{j} + \mu_0 \epsilon_0 \frac{\partial \vec{E}}{\partial t}
\end{equation}

\section{Emergenz der elektromagnetischen Wellen}
Im Vakuum $(\rho = 0, \vec{j} = 0)$ vereinfachen sich die Maxwell-Gleichungen zu:

\begin{align}
\nabla \cdot \vec{E} =&~0\\
\nabla \cdot \vec{B} =&~0\\
\nabla \times \vec{E} =& -\frac{\partial \vec{B}}{\partial t}\\
\nabla \times \vec{B} =& \mu_0 \epsilon_0 \frac{\partial \vec{E}}{\partial t}
\end{align}

Durch Bildung der Rotation der letzten beiden Gleichungen erhält man die Wellengleichungen:

\begin{align}
\nabla^2 \vec{E} =& \mu_0 \epsilon_0 \frac{\partial^2 \vec{E}}{\partial t^2}\\
\nabla^2 \vec{B} =& \mu_0 \epsilon_0 \frac{\partial^2 \vec{B}}{\partial t^2}
\end{align}

Die Ausbreitungsgeschwindigkeit ist:

\begin{equation}
    c = \frac{1}{\sqrt{\mu_0 \epsilon_0}}
\end{equation}

\newpage
\section{Die analoge WDBT und die Rolle des Quantenpotentials}
\subsection{Die vollständige Kraftgleichung der analogen WDBT}
Die \gls{wdbt} kombiniert zwei selten genutzte etablierte Konzepte:

\begin{enumerate}
    \item Die \textbf{\gls{wed}} für die klassische instantane Wechselwirkung.
    \item Die \textbf{\gls{dbt}} für die Quantenmechanik via Quantenpotential.
\end{enumerate}

Die Gesamtkraft auf ein Teilchen der Masse $m$ und Ladung $q$ in der analogen \gls{wdbt} ist daher:

\begin{equation}
    \vec{F}_{\text{ges, WDBT}} = q(\vec{E} + \vec{v} \times \vec{B}) - \nabla Q
\end{equation}

mit dem \textbf{Bohm'schen Quantenpotential:}

\begin{equation}
    Q = -\frac{\hbar^2}{2m} \frac{\nabla^2 \sqrt{\rho}}{\sqrt{\rho}}
\end{equation}

\subsection{Emergenz der klassischen Physik in der WDBT}
Die klassische Physik emergiert in zwei Schritten aus der \gls{wdbt}:

\begin{enumerate}
    \item \textbf{Emergenz der Maxwell-Theorie:} Wie in Abschnitt \ref{sec:grundlagen} gezeigt, geht die \gls{wed} durch Mittelung über viele Ladungen in die Maxwell-Gleichungen über.
    \item \textbf{Emergenz der Newton'schen Mechanik:} Für $\hbar \to 0$ oder auf makroskopischen Skalen, wo Quanteneffekte vernachlässigbar sind, verschwindet das Quantenpotential ($Q \to 0$). Die Kraftgleichung der \gls{wdbt} reduziert sich auf die \textbf{Lorentz-Kraft} (Gl. \refeq{eq:lorentz_kraft}) der klassischen Elektrodynamik.
\end{enumerate}

Zusammen mit den emergenten Maxwell-Gleichungen ist dies die vollständige klassische Beschreibung.

\subsection{Die erweiterte Kraftdichte und die Kontinuitätsgleichung}
In der analogen \gls{wdbt} wirkt auf eine Ladungsdichte $\rho$ nicht nur die Lorentz-Kraftdichte, sondern auch die Kraftdichte aus dem Quantenpotential $Q$. Die vollständige Kraftdichte lautet:

\begin{equation}
    \vec{f}_{\text{WDBT}} = \rho \vec{E} + \vec{j} \times \vec{B} - \rho \nabla Q
\end{equation}

Diese Kraftdichte muss in die Impulsbilanz der Kontinuumsmechanik eingesetzt werden. Zusätzlich gilt die Kontinuitätsgleichung (Ladungserhaltung) unverändert:

\begin{equation}
    \frac{\partial \rho}{\partial t} + \nabla \cdot \vec{j} = 0
\end{equation}

\subsection{Herleitung der modifizierten Maxwell-Gleichungen}
Durch die Einführung des Quantenpotentials $Q$ werden die Quellterme in den Maxwell-Gleichungen erweitert. Die modifizierten inhomogenen Maxwell-Gleichungen in der analogen \gls{wdbt} lauten:

\begin{equation}
    \nabla \cdot \vec{E} = \frac{\rho}{\epsilon_0} - \frac{1}{\epsilon_0} \nabla \cdot (\rho \nabla Q)
\end{equation}

\begin{equation}
    \nabla \times \vec{B} = \mu_0 \vec{j} + \mu_0 \epsilon_0 \frac{\partial \vec{E}}{\partial t} + \mu_0 \nabla \times (\rho \nabla Q)
\end{equation}

Begründung:

\begin{itemize}
    \item Der Term $-\rho \nabla Q$ in der Kraftdichte wirkt wie eine zusätzliche Quanten-Ladungsdichte bzw. ein Quanten-Strom.
    \item Diese Zusatzterme müssen in den Quellgleichungen für $\vec{E}$ und $\vec{B}$ erscheinen, um die Konsistenz mit der Impulserhaltung zu wahren.
    \item Die homogenen Gleichungen ($\nabla \cdot \vec{B} = 0, \nabla \times \vec{E} = -\partial_t \vec{B}$) bleiben unverändert, da sie aus den Definitionen der Felder folgen.
\end{itemize}

\subsubsection{Klassischer Grenzfall und Reduktion}
Im klassischen Grenzfall ($Q \to 0$ oder $\hbar \to 0$) verschwinden die Zusatzterme:

\begin{equation}
    \lim_{Q \to 0} \left( \nabla \cdot \vec{E} \right) = \frac{\rho}{\epsilon_0}
\end{equation}

\begin{equation}
    \lim_{Q \to 0} \left( \nabla \times \vec{B} \right) = \mu_0 \vec{j} + \mu_0 \epsilon_0 \frac{\partial \vec{E}}{\partial t}
\end{equation}

Somit emergieren ebenfalls exakt die klassischen Maxwell-Gleichungen.

\section{Die digitale WDBT: Eine hypothetische Fundierung}
Die \textbf{digitale \gls{wdbt}} zielt darauf ab, den Ursprung \textbf{sowohl der Weber-Kraft als auch des Quantenpotentials} in einer einzigen fundamentalen Theorie zu erklären:

\begin{itemize}
    \item Das digital-fraktale Raummodell (mit $D \approx 2.71$) soll die instantane Wechselwirkung (\gls{wed}) und die Quantenfluktuationen ($Q$) aus der gleichen Ursache generieren.
    \item In dieser Theorie würden die modifizierten Maxwell-Gleichungen nicht postuliert, sondern als effektive Feldgleichungen aus der Mittelung über die diskrete Raumzeit-Struktur emergieren.
    \item Dies wäre analog zur Emergenz der Hydrodynamik aus der Atomphysik.
\end{itemize}

\section{Zusammenfassung}
\begin{itemize}
    \item \textbf{Analoge WDBT:} Führt zu modifizierten Maxwell-Gleichungen mit Quelltermen proportional zu $\nabla Q$.
    \item \textbf{Digitale WDBT:} Ist eine Hypothese, die beide Anteile (\gls{wed} und $Q$) aus einem gemeinsamen Prinzip (fraktaler Raum) ableiten will.
\end{itemize}


\chapter{Die Emergenz der Quantenmechanik}
\section{Die fundamentale Bewegungsgleichung der WDBT}
Die analoge \gls{wdbt} postuliert für ein Teilchen der Masse $m$ die folgende vollständige Kraftgleichung. Für ein neutrales Teilchen, auf das nur ein konservatives Potential $V(\vec{x})$ und das
Quantenpotential $Q$ wirken, vereinfacht sich diese zu:

\begin{equation}
    \label{eq:deterministische_bewegungsgleichung}
    m \frac{d^2\vec{x}}{dt^2} = -\vec{\nabla} V - \vec{\nabla} Q
\end{equation}

wobei das Bohm'sche Quantenpotential $Q$ definiert ist als:

\begin{equation}
    Q = -\frac{\hbar^2}{2m} \frac{\nabla^2 R}{R}
\end{equation}

Hier ist $R = R(\vec{x},t)$ die Amplitude der sogenannten \enquote{Führungswelle} oder \enquote{pilot wave}, und $\rho = R^2$ ist die Wahrscheinlichkeitsdichte des Teilchensensembles. Gleichung
(\refeq{eq:deterministische_bewegungsgleichung}) ist die \textbf{deterministische Bewegungsgleichung} eines Teilchens in der \gls{wdbt}.

\section{Die Madelung-Transformation: Von der Teilchentrajektorie zur Feldbeschreibung}
Um von der Beschreibung einzelner Teilchentrajektorien zur Beschreibung eines Ensembles (einem \enquote{Fluid}) überzugehen, führen wir die Madelung-Transformation durch. Wir formulieren eine komplexe
Wellenfunktion $\psi(\vec{x}, t)$, deren Phase $S$ mit dem Impuls des Teilchens und deren Betragsquadrat mit der Dichte verbunden ist:

\begin{equation}
    \psi(\vec{x}, t) = R(\vec{x}, t)  e^{i S(\vec{x}, t) / \hbar}
\end{equation}

Dabei ist:

\begin{itemize}
    \item $R(\vec{x},t)$: Reelle Amplitude ($\rho = R^2$ ist die Wahrscheinlichkeitsdichte).
    \item $S(\vec{x},t)$: Reelle Phasenfunktion (entspricht der Wirkung bzw. dem Impulspotential).
\end{itemize}

Unser Ziel ist es nun zu zeigen, dass die Bewegungsgleichung (\refeq{eq:deterministische_bewegungsgleichung}) und die Annahme einer Kontinuitätsgleichung für die Dichte $\rho$ äquivalent zur
Schrödinger-Gleichung für $\psi$ sind.

\section{Herleitung der Kontinuitätsgleichung}
Der Impuls $\vec{p}$ eines Teilchens auf seiner Trajektorie ist in der \gls{wdbt} durch den Gradienten der Phasenfunktion $S$ gegeben:

\begin{equation}
    \label{eq:impuls}
    \vec{p} = m \vec{v} = \vec{\nabla} S
\end{equation}

Die Erhaltung der Wahrscheinlichkeit (Teilchen können nicht einfach verschwinden oder erzeugt werden) führt auf eine Kontinuitätsgleichung für die Dichte $\rho$:

\begin{equation}
    \frac{\partial \rho}{\partial t} + \vec{\nabla} \cdot (\rho \vec{v}) = 0
\end{equation}

Setzen wir $\rho = R^2$ und $\vec{v} = \frac{\vec{\nabla} S}{m}$ ein, erhalten wir:

\begin{equation}
    \frac{\partial (R^2)}{\partial t} + \vec{\nabla} \cdot \left( R^2 \frac{\vec{\nabla} S}{m} \right) = 0
\end{equation}

Diese Gleichung beschreibt die zeitliche Entwicklung der Dichteverteilung des Ensembles.

\section{Herleitung der Hamilton-Jacobi-Gleichung}
Wir beginnen nun mit der Newtonschen Bewegungsgleichung der \gls{wdbt} (Gl. \refeq{eq:deterministische_bewegungsgleichung}) und formen sie schrittweise um.

\begin{equation}
    m \frac{d^2\vec{x}}{dt^2} = -\vec{\nabla} V - \vec{\nabla} Q \tag{3.1}
\end{equation}

Die substantielle Zeitableitung der Geschwindigkeit $\vec{v} = \frac{d \vec{x}}{dt}$ ist:

\begin{equation}
    \frac{d\vec{v}}{dt} = \frac{\partial \vec{v}}{\partial t} + (\vec{v} \cdot \vec{\nabla}) \vec{v}
\end{equation}

Mit $\vec{v} = \frac{\vec{\nabla} S}{m}$ (aus Gl. \refeq{eq:impuls}) wird der Beschleunigungsterm ($\vec{v} \cdot \vec{\nabla})$ zu:

\begin{equation}
    (\vec{v} \cdot \vec{\nabla}) \vec{v} = \frac{1}{m^2} \left[ (\vec{\nabla} S \cdot \vec{\nabla}) \vec{\nabla} S \right]
\end{equation}

Eine nützliche Vektoridentität hilft uns, diesen Term umzuschreiben:

\begin{equation}
    (\vec{\nabla} S \cdot \vec{\nabla}) \vec{\nabla} S = \frac{1}{2} \vec{\nabla} (\vec{\nabla} S \cdot \vec{\nabla} S) = \frac{1}{2} \vec{\nabla} (\left|\vec{\nabla} S \right|^2)
\end{equation}

Somit wird die linke Seite der Bewegungsgleichung zu:

\begin{equation}
    m \frac{d\vec{v}}{dt} = m \left( \frac{\partial \vec{v}}{\partial t} \right) + \frac{1}{2m} \vec{\nabla} (\left| \vec{\nabla} S \right|^2)
\end{equation}

\chapter{The Convergent Emergence of GR}
 How \gls{art} tends towards \gls{wdbt}.

\section{The Incomplete GR: The Singularity Problem}

\begin{itemize}
    \item \textbf{Standard-\gls{art}:} $G_{\mu\nu} = 8\pi G T_{\mu\nu}$
    \item This equation leads to singularities under generic conditions (Black Holes, Big Bang).
    \item \textbf{Interpretation:} This is not a physical but a theoretical failure. \gls{art} is incomplete at its limits.
\end{itemize}

\section{Step 1 of Emergence: Completing GR through DBT}

\begin{itemize}
    \item The most obvious extension to avoid singularities is the introduction of the \textbf{Bohmian quantum potential $Q$}.
    \item The completed Einstein equations now read:
    \begin{equation}
        G_{\mu\nu} = 8\pi G (T_{\mu\nu} + Q_{\mu\nu})
    \end{equation}
    \item \textbf{Consequences of this extension:}
    \begin{enumerate}
        \item \textbf{Singularity-free:} $Q$ acts repulsively and prevents the formation of point singularities. Result: Big Bounce instead of Big Bang.
        \item \textbf{Non-locality:} The quantum potential $Q$ is fundamentally non-local. This property is now introduced into gravity itself.
        \item \textbf{Alignment with WDBT:} The extended \gls{art} gains central properties of \gls{wdbt}: Deterministic trajectories, singularity-free, and non-locality.
    \end{enumerate}
\end{itemize}

\section{Step 2 of Emergence: Completion by Accounting for Non-Locality}

\begin{itemize}
    \item \gls{art} (even the extended version) is a local field theory. The original \gls{wg} of \gls{wdbt}, however, is instantaneous and non-local.
    \item This property can also be \enquote{imported} into \gls{art} by considering the \textbf{solutions of the Einstein equations}. The \textbf{gravitational wave} (propagation with $c$) is only a special solution.
    \item The \textbf{instantaneous curvature} (which determines the motion of planets) is another. In a complete treatment, \gls{art} must be able to allow both descriptions equally – the retarded and the advanced solutions (Wheeler-Feynman approach).
    \item \textbf{Result:} The thus completed \gls{art} also becomes \textbf{non-local and local} simultaneously, just like \gls{wdbt}. The retardation of waves is a special case, the instantaneity of the fields is the rule.
\end{itemize}

\section{The Convergent Theories: GR+ vs. WDBT}
Through these two steps of completion, the extended \gls{art} (\gls{art}+) and \gls{wdbt} converge conceptually:

\begin{table}[h]
\centering
\begin{tabular}{|p{0.25\textwidth}|p{0.3\textwidth}|p{0.3\textwidth}|}
\hline
\textbf{Property} & \textbf{\gls{art}+ (Completed)} & \textbf{\gls{wdbt} (Fundamental)} \\
\hline
\textbf{Singularities} & None (Big Bounce) & None (Big Bounce) \\
\hline
\textbf{Non-locality} & Yes (via $Q_{\mu\nu}$ \& field solutions) & Yes (fundamental via \gls{wg}) \\
\hline
\textbf{Determinism} & Yes (via $Q$) & Yes (fundamental) \\
\hline
\textbf{Big Bang} & No & No \\
\hline
\textbf{Light deflection} & Frequency-independent & Frequency-dependent ($\Delta \phi(f)$) \\
\hline
\textbf{Basis} & Geometric description & Dyn. Interaction \\
\hline
\end{tabular}
\caption{Comparison of the completed GR (GR+) with the fundamental WDBT}
\end{table}

\textbf{The two theories appear to converge!}

\section{The Experimental Decider: Frequency-Dependent Light Deflection}
Despite the conceptual convergence, a \textbf{decisive, experimentally verifiable difference} remains:

\begin{itemize}
    \item \textbf{\gls{art}+:} Ultimately based on a \textbf{geometric} description. Light deflection is purely geometric and therefore \textbf{frequency-independent}.
    \item \textbf{\gls{wdbt}:} Based on a dynamic interaction (Weber force). Light deflection is a real force effect and therefore frequency-dependent ($\Delta \Phi(f)$).
\end{itemize}

\textbf{This deviation is the litmus test.} Which of the two convergent descriptions is the more fundamental one?

\begin{itemize}
    \item If one measures \textbf{no} frequency dependence, then the geometric description of \gls{art}+ is sufficient.
    \item If one measures \textbf{a} frequency dependence, this is the conclusive proof for the correctness of the dynamic foundation of \gls{wdbt}.
\end{itemize}

\section{Conclusion: WDBT as the Fundamental Primal Theory}
The search for a consistent extension of \gls{art} thus leads in a direction that looks remarkably similar to \gls{wdbt}. This is no coincidence but an indication that \gls{wdbt}
is the correct fundamental primal theory.

\gls{wdbt} not only provides the most consistent description but also the sharpest, testable prediction ($\Delta \Phi(f)$) to ultimately distinguish itself from all derived effective
theories (like \gls{art}).
\chapter{Die Emergenz der Quantenelektrodynamik}
\section{Die Aufgabe der Reduktion}
Die \gls{qed} beschreibt die Wechselwirkung zwischen Licht und Materie mit unvergleichlicher Präzision. Ihr Herzstück ist das Konzept der Quantenfelder und virtueller Teilchen. Dieses Kapitel zeigt,
wie die scheinbar abstrakten Konzepte der \gls{qed} – die Feldquantisierung, Feynman-Diagramme und die Renormierung – aus der deterministischen, teilchenbasierten \gls{wdbt} emergieren. Die Strategie
ist nicht die 1:1-Rekonstruktion, sondern die Herleitung der operationalen Kernaussagen der \gls{qed} aus den ersten Prinzipien der \gls{wdbt}.

\section{Die vollständige Kraftgleichung der WDBT}
Die fundamentale Gleichung für ein geladenes Teilchen (Masse $m$, Ladung $q$) in der \gls{wdbt} ist die Erweiterung der Lorentz-Kraft um das Quantenpotential $Q$:

\begin{equation}
    \label{eq:kraft_wdbt_em}
    m \frac{d^2\vec{x}}{dt^2} = q(\vec{E} + \vec{v} \times \vec{B}) - \nabla Q
\end{equation}

mit:

\begin{equation}
    \label{eq:quantenpotential_wdbt_em}
    Q = -\frac{\hbar^2}{2m} \frac{\nabla^2 \sqrt{\rho}}{\sqrt{\rho}}
\end{equation}

Die Felder $\vec{E}$ und $\vec{B}$ sind dabei effektive Beschreibungen der gemittelten Weber-Wechselwirkung mit allen anderen Ladungen im Universum, wie in Kapitel \ref{ch:maxwell} hergeleitet. Gleichung
(\refeq{eq:kraft_wdbt_em}) ist \textbf{deterministisch} und beschreibt eine wohldefinierte Trajektorie.


\appendix
\chapter{Anhang}
\section{Der Aharonov-Bohm-Effekt}
\label{sec:aharonov-bohm}

Der \textbf{Aharonov-Bohm-Effekt} (AB-Effekt) ist ein grundlegendes Quantenphänomen, das zeigt, dass elektromagnetische Potentiale ($\vec{A}$, $\Phi$) eine direkte physikalische
Wirkung auf Quantenteilchen haben, selbst in Regionen wo die Felder ($\vec{E}$, $\vec{B}$) null sind.

\subsection{Experimentelle Anordnung}
Ein Elektronenstrahl wird in zwei Pfade aufgeteilt, die eine Region mit magnetischem Fluss $\Phi$ umschließen.

\subsection{Theoretische Beschreibung}
Die Wellenfunktion $\psi$ eines Teilchens mit Ladung $q$ wird durch das Vektorpotential $\vec{A}$ modifiziert:

\begin{equation}
\psi \rightarrow \psi \cdot \exp\left(i\frac{q}{\hbar}\int \vec{A}\cdot d\vec{l}\right)
\end{equation}

Die Phasendifferenz zwischen den beiden Pfaden beträgt:

\begin{equation}
\Delta\phi = \frac{q}{\hbar}\oint \vec{A}\cdot d\vec{l} = \frac{q}{\hbar}\Phi_B
\end{equation}

\subsection{Physikalische Bedeutung}
\begin{itemize}
\item \textbf{Nicht-Lokalität}: Quantenteilchen \enquote{spüren} $\vec{A}$ auch in feldfreien Regionen
\item \textbf{Topologische Invariante}: Die Phase hängt nur vom eingeschlossenen Fluss $\Phi_B$ ab
\item \textbf{Paradigmenwechsel}: Widerlegt die klassische Annahme, dass nur $\vec{E}$ und $\vec{B}$ physikalisch relevant sind
\end{itemize}

\subsection{Experimentelle Bestätigung}
\begin{itemize}
\item Theoretische Vorhersage: Aharonov \& Bohm (1959)
\item Erste Experimente: Chambers (1960), Tonomura et al. (1982)
\item Moderne Anwendungen: Quanteninterferometer, topologische Quantenmaterialien
\end{itemize}


\backmatter
\printbibliography[title=Bibliography]
\glswritefiles
\printglossary[title=Glossary]
\printglossary[type=acronym, title=Abbreviations]

\end{document}