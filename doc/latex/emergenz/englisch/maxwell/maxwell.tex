\chapter{The Emergence of Maxwell's Equations}
\label{ch:maxwell}
\section{Foundations of Weber Electrodynamics}
\label{sec:grundlagen}
\gls{wed} postulates an instantaneous, velocity- and acceleration-dependent force between two charges $q_1$ and $q_2$. The vector form of the Weber force is:

\begin{equation}
    \vec{F}_{12} = \frac{q_1 q_2}{4\pi\epsilon_0 r^2} \left\{ \left[ 1 - \frac{v^2}{c^2} + \frac{2r(\hat{\vec{r}}\cdot\vec{a})}{c^2} \right] \hat{\vec{r}} + \frac{2(\hat{\vec{r}}\cdot\vec{v})}{c^2} \vec{v} \right\}
\end{equation}

where:

\begin{itemize}
    \item $r = \left| \vec{r} \right|$ is the distance between the charges,
    \item $\hat{\vec{r}} = \frac{\vec{r}}{r}$ is the unit vector in the direction from $q_1$ to $q_2$,
    \item $\vec{v} = \dot{\vec{r}}$ is the relative velocity,
    \item $\vec{a} = \ddot{\vec{r}}$ is the relative acceleration,
    \item $c$ is the characteristic limit velocity of the interaction.
\end{itemize}

This force can be derived from a generalized potential and satisfies energy and momentum conservation.

\section{Superposition and Total Force on a Test Charge}
For a test charge $q$ in the field of $N$ other charges $q_i (i = 1,...,N)$, the principle of superposition holds. The total force on $q$ is:

\begin{equation}
    \label{eq:gesamtkraft}
    \vec{F}_{\text{ges}} = q \sum_{i=1}^N \frac{q_i}{4\pi\epsilon_0 r_i^2} \left\{ \left[ 1 - \frac{v_i^2}{c^2} + \frac{2r_i(\hat{r}_i\cdot\vec{a}_i)}{c^2} \right] \hat{r}_i + \frac{2(\hat{r}_i\cdot\vec{v}_i)}{c^2} \vec{v}_i \right\}
\end{equation}

where $\vec{r_i}$ is the vector from $q$ to $q_i$.

\section{Definition of the Effective Fields}
The total force from the effective fields $\vec{E}$ and $\vec{B}$ can be cast into the form of the Lorentz force:

\begin{equation}
    \label{eq:lorentz_kraft}
    \vec{F} = q \left( \vec{E} + \vec{v} \times \vec{B} \right)
\end{equation}

By comparing coefficients, the definitions of the effective fields emerge:

\begin{equation}
    \vec{E} = \sum_{i=1}^N \frac{q_i}{4\pi\epsilon_0 r_i^2} \left[ 1 - \frac{v_i^2}{c^2} + \frac{2r_i(\hat{r}_i\cdot\vec{a}_i)}{c^2} \right] \hat{r}_i
\end{equation}

\begin{equation}
    \vec{B} = \sum_{i=1}^N \frac{q_i}{4\pi\epsilon_0 r_i^2 c^2} \cdot 2(\hat{r}_i\cdot\vec{v}_i) \vec{v}_i
\end{equation}

These effective fields are mathematical auxiliary quantities describing the averaged effect of all other charges.

\section{Continuum Limit and Field Equations}
\label{sec:kontinuumslimes}
For a continuous charge distribution with density $\rho(\vec{r}, t)$ and current density $\vec{j}(\vec{r},t)$, the sums turn into integrals. The fields become:

\begin{equation}
    \vec{E}(\vec{r}, t) = \frac{1}{4\pi\epsilon_0} \int \rho(\vec{r}~', t) \left[ 1 - \frac{v^2}{c^2} + \frac{2r(\hat{\vec{r}}\cdot\vec{a})}{c^2} \right] \frac{\hat{\vec{r}}}{r^2}  d^3r~'
\end{equation}

\begin{equation}
    \vec{B}(\vec{r}, t) = \frac{1}{4\pi\epsilon_0 c^2} \int \vec{j}(\vec{r}~', t) \cdot 2(\hat{\vec{r}}\cdot\vec{v}) \frac{\hat{\vec{r}}}{r^2}  d^3r~'
\end{equation}

% Variable explanation for the integrals
\begin{align*}
&\vec{r} && \text{Position vector to the field point} \\
&\vec{r}~' && \text{Position vector to the source charge} \\
&\vec{r} = \vec{r} - \vec{r}~' && \text{Separation vector} \\
&r = |\vec{r} - \vec{r}~'| && \text{Distance} \\
&\hat{\vec{r}} = \frac{\vec{r} - \vec{r}~'}{|\vec{r} - \vec{r}~'|} && \text{Unit vector} \\
&\rho(\vec{r}~', t) && \text{Charge density at the source point} \\
&\vec{j}(\vec{r}~', t) && \text{Current density at the source point} \\
&d^3r~' && \text{Volume element in source space}
\end{align*}

\subsection{Gauss's Law}
\begin{equation}
    \nabla \cdot \vec{E} = \frac{\rho}{\epsilon_0}
\end{equation}

\subsection{Gauss's Law for Magnetism}
\begin{equation}
    \nabla \cdot \vec{B} = 0
\end{equation}

\subsection{Faraday's Law of Induction}
\begin{equation}
    \nabla \times \vec{E} = -\frac{\partial \vec{B}}{\partial t}
\end{equation}

\subsection{Maxwell's Displacement Current}
\begin{equation}
    \nabla \times \vec{B} = \mu_0 \vec{j} + \mu_0 \epsilon_0 \frac{\partial \vec{E}}{\partial t}
\end{equation}

\section{Emergence of Electromagnetic Waves}
In vacuum $(\rho = 0, \vec{j} = 0)$, Maxwell's equations simplify to:

\begin{align}
\nabla \cdot \vec{E} =&~0\\
\nabla \cdot \vec{B} =&~0\\
\nabla \times \vec{E} =& -\frac{\partial \vec{B}}{\partial t}\\
\nabla \times \vec{B} =& \mu_0 \epsilon_0 \frac{\partial \vec{E}}{\partial t}
\end{align}

By taking the curl of the last two equations, one obtains the wave equations:

\begin{align}
\nabla^2 \vec{E} =& \mu_0 \epsilon_0 \frac{\partial^2 \vec{E}}{\partial t^2}\\
\nabla^2 \vec{B} =& \mu_0 \epsilon_0 \frac{\partial^2 \vec{B}}{\partial t^2}
\end{align}

The propagation velocity is:

\begin{equation}
    c = \frac{1}{\sqrt{\mu_0 \epsilon_0}}
\end{equation}

\newpage
\section{The Analogous WDBT and the Role of the Quantum Potential}
\subsection{The Complete Force Equation of the Analogous WDBT}
\gls{wdbt} combines two rarely used established concepts:

\begin{enumerate}
    \item \textbf{\gls{wed}} for the classical instantaneous interaction.
    \item \textbf{\gls{dbt}} for quantum mechanics via the quantum potential.
\end{enumerate}

The total force on a particle of mass $m$ and charge $q$ in the analogous \gls{wdbt} is therefore:

\begin{equation}
    \vec{F}_{\text{ges, WDBT}} = q(\vec{E} + \vec{v} \times \vec{B}) - \nabla Q
\end{equation}

with the \textbf{Bohmian quantum potential:}

\begin{equation}
    Q = -\frac{\hbar^2}{2m} \frac{\nabla^2 \sqrt{\rho}}{\sqrt{\rho}}
\end{equation}

\subsection{Emergence of Classical Physics in WDBT}
Classical physics emerges from \gls{wdbt} in two steps:

\begin{enumerate}
    \item \textbf{Emergence of Maxwell Theory:} As shown in Section \ref{sec:grundlagen}, \gls{wed} transitions into Maxwell's equations through averaging over many charges.
    \item \textbf{Emergence of Newtonian Mechanics:} For $\hbar \to 0$ or on macroscopic scales where quantum effects are negligible, the quantum potential vanishes ($Q \to 0$). The force equation of \gls{wdbt} reduces to the \textbf{Lorentz force} (Eq. \refeq{eq:lorentz_kraft}) of classical electrodynamics.
\end{enumerate}

Together with the emergent Maxwell's equations, this is the complete classical description.

\subsection{The Extended Force Density and the Continuity Equation}
In the analogous \gls{wdbt}, a charge density $\rho$ is acted upon not only by the Lorentz force density but also by the force density from the quantum potential $Q$. The complete force density is:

\begin{equation}
    \vec{f}_{\text{WDBT}} = \rho \vec{E} + \vec{j} \times \vec{B} - \rho \nabla Q
\end{equation}

This force density must be inserted into the momentum balance of continuum mechanics. Additionally, the continuity equation (charge conservation) holds unchanged:

\begin{equation}
    \frac{\partial \rho}{\partial t} + \nabla \cdot \vec{j} = 0
\end{equation}

\subsection{Derivation of the Modified Maxwell's Equations}
By introducing the quantum potential $Q$, the source terms in Maxwell's equations are extended. The modified inhomogeneous Maxwell's equations in the analogous \gls{wdbt} are:

\begin{equation}
    \nabla \cdot \vec{E} = \frac{\rho}{\epsilon_0} - \frac{1}{\epsilon_0} \nabla \cdot (\rho \nabla Q)
\end{equation}

\begin{equation}
    \nabla \times \vec{B} = \mu_0 \vec{j} + \mu_0 \epsilon_0 \frac{\partial \vec{E}}{\partial t} + \mu_0 \nabla \times (\rho \nabla Q)
\end{equation}

Justification:

\begin{itemize}
    \item The term $-\rho \nabla Q$ in the force density acts like an additional quantum charge density or quantum current.
    \item These additional terms must appear in the source equations for $\vec{E}$ and $\vec{B}$ to maintain consistency with momentum conservation.
    \item The homogeneous equations ($\nabla \cdot \vec{B} = 0, \nabla \times \vec{E} = -\partial_t \vec{B}$) remain unchanged, as they follow from the definitions of the fields.
\end{itemize}

\subsubsection{Classical Limit and Reduction}
In the classical limit ($Q \to 0$ or $\hbar \to 0$), the additional terms vanish:

\begin{equation}
    \lim_{Q \to 0} \left( \nabla \cdot \vec{E} \right) = \frac{\rho}{\epsilon_0}
\end{equation}

\begin{equation}
    \lim_{Q \to 0} \left( \nabla \times \vec{B} \right) = \mu_0 \vec{j} + \mu_0 \epsilon_0 \frac{\partial \vec{E}}{\partial t}
\end{equation}

Thus, the classical Maxwell's equations also emerge exactly.

\section{The Digital WDBT: A Hypothetical Foundation}
The \textbf{digital \gls{wdbt}} aims to explain the origin \textbf{of both the Weber force and the quantum potential} in a single fundamental theory:

\begin{itemize}
    \item The digital-fractal space model (with $D \approx 2.71$) is supposed to generate the instantaneous interaction (\gls{wed}) and the quantum fluctuations ($Q$) from the same cause.
    \item In this theory, the modified Maxwell's equations would not be postulated but would emerge as effective field equations from averaging over the discrete spacetime structure.
    \item This would be analogous to the emergence of hydrodynamics from atomic physics.
\end{itemize}

\section{Summary}
\begin{itemize}
    \item \textbf{Analog \gls{wdbt}:} Leads to modified Maxwell's equations with source terms proportional to $\nabla Q$.
    \item \textbf{Digital \gls{wdbt}:} Is a hypothesis that aims to derive both components (\gls{wed} and $Q$) from a common principle (fractal space).
\end{itemize}