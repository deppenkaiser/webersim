\chapter{The Emergence of Quantum Electrodynamics}
\section{The Task of Reduction}
\gls{qed} describes the interaction between light and matter with unparalleled precision. Its core is the concept of quantum fields and virtual particles. This chapter shows how the seemingly abstract concepts of \gls{qed} – field quantization, Feynman diagrams, and renormalization – emerge from the deterministic, particle-based \gls{wdbt}. The strategy is not a 1:1 reconstruction, but the derivation of the operational core statements of \gls{qed} from the first principles of \gls{wdbt}.

\section{The Complete Force Equation of WDBT}
The fundamental equation for a charged particle (mass $m$, charge $q$) in \gls{wdbt} is the extension of the Lorentz force by the quantum potential $Q$:

\begin{equation}
    \label{eq:kraft_wdbt_em}
    m \frac{d^2\vec{x}}{dt^2} = q(\vec{E} + \vec{v} \times \vec{B}) - \nabla Q
\end{equation}

with:

\begin{equation}
    \label{eq:quantenpotential_wdbt_em}
    Q = -\frac{\hbar^2}{2m} \frac{\nabla^2 \sqrt{\rho}}{\sqrt{\rho}}
\end{equation}

The fields $\vec{E}$ and $\vec{B}$ are effective descriptions of the averaged Weber interaction with all other charges in the universe, as derived in Chapter \ref{ch:maxwell}. Equation (\refeq{eq:kraft_wdbt_em}) is \textbf{deterministic} and describes a well-defined trajectory.

\newpage
\section{The Emergence of the Schrödinger and Maxwell Equations}
To describe the statistics of an ensemble of identical systems (e.g., many atoms in the same state), the Madelung transformation is performed (cf. Section \ref{sec:madelung}). We introduce a complex wave function $\psi(\vec{x},t)=R(\vec{x},t)e^{iS(\vec{x},t)/\hbar}$, for which:

\begin{enumerate}
    \item $\rho=\left| \psi \right|^2$ (probability density)
    \item $\vec{v} = \frac{1}{m}\vec{\nabla}S$ (particle velocity)
\end{enumerate}

As shown in Section \ref{sec:schrödinger_gleichung}, the equation of motion (\refeq{eq:kraft_wdbt_em}) is equivalent to the time-dependent Schrödinger equation for $\psi$:

\begin{equation}
    i\hbar \frac{\partial \psi}{\partial t} = \left( -\frac{\hbar^2}{2m} \nabla^2 + V + q\phi \right) \psi
\end{equation}

where the vector potential $\vec{A}$ enters via $\vec{B} = \vec{\nabla} \times \vec{A}$ into the minimal coupling. In parallel, the classical Maxwell equations emerge from Weber electrodynamics through the continuum limit (Section \ref{sec:kontinuumslimes}). Thus, \textbf{non-relativistic quantum mechanics} and \textbf{classical electrodynamics} are already derived as effective levels of description from \gls{wdbt}.

\section{Photons as Excitations of the Quantum Vacuum}
Conventional \gls{qed} postulates the quantization of the electromagnetic field. In \gls{wdbt}, this step arises naturally from considering the quantum potential of the vacuum.

The vacuum in \gls{wdbt} is not empty space but a \textbf{quantum medium} with a ground fluctuation, described by a vacuum wave function $\psi_\text{Vak}$. Its associated quantum potential $Q_\text{Vak}$ acts on all particles. The excitations of this medium – described by standing waves in a box with specific frequencies – correspond to photons.

The energy of an excitation of frequency $\omega$ is:

\begin{equation}
    E = \hbar \omega
\end{equation}

This relation emerges from the scaling of the quantum potential $Q$ with $\hbar^2$ and the dispersion relation of \gls{wed}. The \textbf{field operators} of \gls{qed} (creation/annihilation operators) are therefore mathematical auxiliary quantities for describing these excited modes of the quantum vacuum, not fundamental entities.

\section{Derivation of Feynman Rules from Non-Local Weber Interaction}
The strength of \gls{qed} lies in perturbation theory and \textbf{Feynman diagrams}. These emerge in \gls{wdbt} from \textbf{averaging over all possible non-local interaction paths}.

In \gls{wed}, the force between two charges acts instantaneously and depends on velocity and acceleration. The probability amplitude for a particle to go from $A$ to $B$ must sum over all possible trajectories, each influenced by its own guiding wave and its quantum potential.

\subsubsection{Derivation Steps:}
\begin{enumerate}
    \item The \textbf{action} $S$ for a Weber interaction path is defined.
    \item The \textbf{amplitude} for a path is proportional to $e^{iS/\hbar}$.
    \item The \textbf{path integral} (sum over all paths) is introduced.
    \item The \textbf{Weber force} is expanded into a perturbation series. Each term in this series corresponds to an \textbf{elementary Weber interaction vertex}.
    \item The \textbf{propagators} (e.g., $\frac{i}{p^2 - m^2 + i\epsilon}$ for an electron) describe the propagation between two interactions under the influence of the free quantum potential.
    \item The \textbf{Feynman rules} emerge as an efficient bookkeeping method for calculating the total amplitude, considering all possible paths and vertices.
\end{enumerate}

A Feynman diagram is thus not a representation of virtual particles, but a \textbf{graphical representation of the averaging over non-local Weber interaction terms} in the perturbation series.

\section{Regularization by the Quantum Potential}
Conventional \gls{qed} suffers from divergent integrals (infinities). Renormalization removes these by redefining mass and charge. In \gls{wdbt}, these divergences are artifactual and are avoided from the outset.

The reason is the \textbf{quantum potential} $Q$. Since it depends on the density $\rho$ and this never becomes point-like for a particle (the guiding wave always has a finite extent), all interactions are \textbf{regularized}. The \enquote{bare} charge and mass are finite because the self-energy of a particle is limited by the interaction with its own, extended guiding wave. Renormalization thus emerges as the effective calculus to extract the finite, observable quantities from the underlying regularized \gls{wdbt} dynamics.

\section{Lamb Shift and g-Factor}
The successes of \gls{qed} must be reproduced by \gls{wdbt}. The calculation of the Lamb shift and the anomalous magnetic moment of the electron follows a modified but conceptually clear path in \gls{wdbt}.

\begin{itemize}
    \item \textbf{Lamb Shift:} It results from the interaction of the electron with the fluctuating quantum potential of the vacuum ($Q_\text{Vak}$). The \gls{wdbt} prediction contains an additional term compared to \gls{qed}:
    \begin{equation}
        \label{eq:lamb_shift}
        \Delta E_{\text{Lamb}}^{\text{WDBT}} = \Delta E_{\text{QED}} + \frac{e^2 \hbar}{4\pi \epsilon_0 m_e^2 c^3} \langle r \rangle
    \end{equation}
    This term is small but in principle measurable and represents a falsifiable deviation from \gls{qed}. Appendix \ref{att:lamb_shift} shows more details.
    \item \textbf{g-Factor:} The anomalous g-factor emerges from the non-local coupling of the electron spin (modeled as Zitterbewegung in \gls{wdbt}) to the cosmic electromagnetic background. The prediction agrees with that of \gls{qed}, as both lead to the same effective algebra of spin interaction.
\end{itemize}

\section{QED as a Triumphant Effective Field Theory}
Quantum electrodynamics emerges completely from the Weber-De Broglie-Bohm theory as its effective, statistical description in the continuum limit.

\begin{itemize}
    \item \textbf{Fields} emerge from the averaging of instantaneous Weber interactions.
    \item \textbf{Field quantization} emerges from the excitations of the quantum vacuum medium.
    \item \textbf{Feynman diagrams} emerge from the perturbation expansion of the non-local interaction.
    \item \textbf{Renormalization} emerges as a procedure to extract observable quantities from the regularized \gls{wdbt}.
\end{itemize}

\gls{wdbt} thus resolves the conceptual problems of \gls{qed} (infinities, virtuality, ontology) while preserving all its empirical success. It places \gls{qed} on a solid, ontologically clear foundation of direct particle interactions and deterministic dynamics. \gls{qed} is not wrong, but an incomplete description of a deeper, non-local reality.