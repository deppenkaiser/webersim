\chapter{The Emergence of Quantum Mechanics}
\section{The Fundamental Equation of Motion of WDBT}
The analogous \gls{wdbt} postulates the following complete force equation for a particle of mass $m$. For a neutral particle acted upon only by a conservative potential $V(\vec{x})$ and the quantum potential $Q$, this simplifies to:

\begin{equation}
    \label{eq:deterministische_bewegungsgleichung}
    m \frac{d^2\vec{x}}{dt^2} = -\vec{\nabla} V - \vec{\nabla} Q
\end{equation}

where the Bohmian quantum potential $Q$ is defined as:

\begin{equation}
    \label{eq:quantenpotential}
    Q = -\frac{\hbar^2}{2m} \frac{\nabla^2 R}{R}
\end{equation}

Here, $R = R(\vec{x},t)$ is the amplitude of the so-called \enquote{guiding wave} or \enquote{pilot wave}, and $\rho = R^2$ is the probability density of the particle ensemble. Equation (\refeq{eq:deterministische_bewegungsgleichung}) is the \textbf{deterministic equation of motion} of a particle in \gls{wdbt}.

\section{The Madelung Transformation: From Particle Trajectory to Field Description}
\label{sec:madelung}
To transition from the description of individual particle trajectories to the description of an ensemble (a \enquote{fluid}), we perform the Madelung transformation. We formulate a complex wave function $\psi(\vec{x}, t)$, whose phase $S$ is connected to the particle's momentum and whose squared magnitude is connected to the density:

\begin{equation}
    \psi(\vec{x}, t) = R(\vec{x}, t)  e^{i S(\vec{x}, t) / \hbar}
\end{equation}

Here:

\begin{itemize}
    \item $R(\vec{x},t)$: Real amplitude ($\rho = R^2$ is the probability density).
    \item $S(\vec{x},t)$: Real phase function (corresponds to the action or momentum potential).
\end{itemize}

Our goal is now to show that the equation of motion (\refeq{eq:deterministische_bewegungsgleichung}) and the assumption of a continuity equation for the density $\rho$ are equivalent to the Schrödinger equation for $\psi$.

\section{Derivation of the Continuity Equation}
\label{sec:kontinuitätsgleichung}
The momentum $\vec{p}$ of a particle on its trajectory is given in \gls{wdbt} by the gradient of the phase function $S$:

\begin{equation}
    \label{eq:impuls}
    \vec{p} = m \vec{v} = \vec{\nabla} S
\end{equation}

The conservation of probability (particles cannot simply vanish or be created) leads to a continuity equation for the density $\rho$:

\begin{equation}
    \frac{\partial \rho}{\partial t} + \vec{\nabla} \cdot (\rho \vec{v}) = 0
\end{equation}

Substituting $\rho = R^2$ and $\vec{v} = \frac{\vec{\nabla} S}{m}$, we obtain:

\begin{equation}
    \label{eq:kontinuitätsgleichung}
    \frac{\partial (R^2)}{\partial t} + \vec{\nabla} \cdot \left( R^2 \frac{\vec{\nabla} S}{m} \right) = 0
\end{equation}

This equation describes the temporal evolution of the density distribution of the ensemble.

\section{Derivation of the Hamilton-Jacobi Equation}
\label{sec:hamilton_jakobi_gleichung}
We now start with Newton's equation of motion of \gls{wdbt} (Eq. \refeq{eq:deterministische_bewegungsgleichung}) and transform it step by step.

\begin{equation}
    m \frac{d^2\vec{x}}{dt^2} = -\vec{\nabla} V - \vec{\nabla} Q \tag{3.1}
\end{equation}

The substantial time derivative of the velocity $\vec{v} = \frac{d \vec{x}}{dt}$ is:

\begin{equation}
    \frac{d\vec{v}}{dt} = \frac{\partial \vec{v}}{\partial t} + (\vec{v} \cdot \vec{\nabla}) \vec{v}
\end{equation}

With $\vec{v} = \frac{\vec{\nabla} S}{m}$ (from Eq. \refeq{eq:impuls}), the acceleration term ($\vec{v} \cdot \vec{\nabla})$ becomes:

\begin{equation}
    (\vec{v} \cdot \vec{\nabla}) \vec{v} = \frac{1}{m^2} \left[ (\vec{\nabla} S \cdot \vec{\nabla}) \vec{\nabla} S \right]
\end{equation}

A useful vector identity helps us rewrite this term:

\begin{equation}
    (\vec{\nabla} S \cdot \vec{\nabla}) \vec{\nabla} S = \frac{1}{2} \vec{\nabla} (\vec{\nabla} S \cdot \vec{\nabla} S) = \frac{1}{2} \vec{\nabla} (\left|\vec{\nabla} S \right|^2)
\end{equation}

Thus, the left side of the equation of motion becomes:

\begin{equation}
    m \frac{d\vec{v}}{dt} = m \left( \frac{\partial \vec{v}}{\partial t} \right) + \frac{1}{2m} \vec{\nabla} (\left| \vec{\nabla} S \right|^2)
\end{equation}

Now we substitute $\vec{v} = \frac{\vec{\nabla} S}{m}$ and set this equal to the right side of (\refeq{eq:deterministische_bewegungsgleichung}):

\begin{equation}
    m \frac{\partial}{\partial t}\left( \frac{\vec{\nabla} S}{m} \right) + \frac{1}{2m} \vec{\nabla} (\left| \vec{\nabla} S \right|^2) = -\vec{\nabla} V - \vec{\nabla} Q
\end{equation}

By swapping ($\frac{\partial}{\partial t} \vec{\nabla} S = \vec{\nabla} \frac{\partial S}{\partial t}$), this simplifies to:

\begin{equation}
    \vec{\nabla} \left( \frac{\partial S}{\partial t} \right) + \frac{1}{2m} \vec{\nabla} (\left| \vec{\nabla} S \right|^2) = -\vec{\nabla} V - \vec{\nabla} Q
\end{equation}

We can now apply the gradient operator $\vec{\nabla}$ to all terms:

\begin{equation}
    \label{eq:zwischenstand}
    \vec{\nabla} \left( \frac{\partial S}{\partial t} + \frac{1}{2m} \left| \vec{\nabla} S \right|^2 + V + Q \right) = 0
\end{equation}

Equation (\refeq{eq:zwischenstand}) states that the expression in parentheses is spatially constant. This leads us to the modified Hamilton-Jacobi equation:

\begin{equation}
    \frac{\partial S}{\partial t} + \frac{1}{2m} \left| \vec{\nabla} S \right|^2 + V + Q = 0
\end{equation}

Now substituting the definition of the quantum potential (\refeq{eq:quantenpotential}), we obtain the central equation:

\begin{equation}
    \label{eq:modifizierte_hamilton_jacobi}
    \frac{\partial S}{\partial t} + \frac{1}{2m} \left| \vec{\nabla} S \right|^2 + V - \frac{\hbar^2}{2m} \frac{\nabla^2 R}{R} = 0
\end{equation}

\section{Synthesis to the Schrödinger Equation}
\label{sec:schrödinger_gleichung}
We now have the two real equations describing the dynamics of the system:

\begin{enumerate}
    \item The continuity equation (\refeq{eq:kontinuitätsgleichung})
    \item The modified Hamilton-Jacobi equation (\refeq{eq:modifizierte_hamilton_jacobi})
\end{enumerate}

The ingenious insight is that these two equations are equivalent to the single complex Schrödinger equation. To see this, we substitute the wave function $\psi = R e^{iS/\hbar}$ into the time-dependent Schrödinger equation:

\begin{equation}
    \label{eq:schrödinger_gleichung}
    i\hbar \frac{\partial \psi}{\partial t} = \left( -\frac{\hbar^2}{2m} \nabla^2 + V \right) \psi
\end{equation}

We perform the derivatives explicitly.

\textbf{Left Side:}

\begin{equation}
    \frac{\partial \psi}{\partial t} = \frac{\partial R}{\partial t} e^{iS/\hbar} + R \cdot \frac{i}{\hbar} \frac{\partial S}{\partial t} e^{iS/\hbar}
\end{equation}

\begin{equation}
    i\hbar \frac{\partial \psi}{\partial t} = i\hbar \frac{\partial R}{\partial t} e^{iS/\hbar} - R \frac{\partial S}{\partial t} e^{iS/\hbar}
\end{equation}

\textbf{Right Side:}\\
We first calculate $\nabla^2 \psi$.

\begin{equation}
    \vec{\nabla} \psi = (\vec{\nabla} R) e^{iS/\hbar} + R \frac{i}{\hbar} (\vec{\nabla} S) e^{iS/\hbar}
\end{equation}

\begin{equation}
    \nabla^2 \psi = \left[ \nabla^2 R + \frac{2i}{\hbar} (\vec{\nabla} R \cdot \vec{\nabla} S) + \frac{i}{\hbar} R \nabla^2 S - \frac{1}{\hbar^2} R \left| \vec{\nabla} S \right|^2 \right] e^{iS/\hbar}
\end{equation}

Thus, the right side becomes:

\begin{equation}
    \left( -\frac{\hbar^2}{2m} \nabla^2 + V \right) \psi = -\frac{\hbar^2}{2m} \left[ \nabla^2 R + \frac{2i}{\hbar} (\vec{\nabla} R \cdot \vec{\nabla} S) + \frac{i}{\hbar} R \nabla^2 S - \frac{1}{\hbar^2} R \left| \vec{\nabla} S \right|^2 \right] e^{iS/\hbar} + V R e^{iS/\hbar}
\end{equation}

Now setting left and right sides equal and dividing by the common factor $e^{iS/\hbar}$:

\begin{equation}
    i\hbar \frac{\partial R}{\partial t} - R \frac{\partial S}{\partial t} = -\frac{\hbar^2}{2m} \nabla^2 R - \frac{i\hbar}{m} (\vec{\nabla} R \cdot \vec{\nabla} S) - \frac{i\hbar}{2m} R \nabla^2 S + \frac{1}{2m} R \left| \vec{\nabla} S \right|^2 + V R
\end{equation}

We now separate the real and imaginary parts.

\textbf{Imaginary Part:}

\begin{equation}
    \hbar \frac{\partial R}{\partial t} = -\frac{\hbar}{m} (\vec{\nabla} R \cdot \vec{\nabla} S) - \frac{\hbar}{2m} R \nabla^2 S
\end{equation}

Multiplying both sides by $2R/\hbar$, we get:

\begin{equation}
    2R \frac{\partial R}{\partial t} = -\frac{2}{m} R (\vec{\nabla} R \cdot \vec{\nabla} S) - \frac{1}{m} R^2 \nabla^2 S
\end{equation}

\begin{equation}
    \frac{\partial (R^2)}{\partial t} + \vec{\nabla} \cdot \left( R^2 \frac{\vec{\nabla} S}{m} \right) = 0 \tag{\refeq{eq:kontinuitätsgleichung}}
\end{equation}

This is exactly the \textbf{continuity equation (\refeq{eq:kontinuitätsgleichung})} from Section (\ref{sec:kontinuitätsgleichung}).

\textbf{Real Part:}

\begin{equation}
    - R \frac{\partial S}{\partial t} = -\frac{\hbar^2}{2m} \nabla^2 R + \frac{1}{2m} R \left| \vec{\nabla} S \right|^2 + V R
\end{equation}

Dividing by $R$:

\begin{equation}
    - \frac{\partial S}{\partial t} = -\frac{\hbar^2}{2m} \frac{\nabla^2 R}{R} + \frac{1}{2m} |\nabla S|^2 + V
\end{equation}

\begin{equation}
    \frac{\partial S}{\partial t} + \frac{1}{2m} \left| \vec{\nabla} S \right|^2 + V - \frac{\hbar^2}{2m} \frac{\nabla^2 R}{R} = 0 \tag{\refeq{eq:modifizierte_hamilton_jacobi}}
\end{equation}

This is exactly the modified Hamilton-Jacobi equation (\refeq{eq:modifizierte_hamilton_jacobi}) from Section (\ref{sec:schrödinger_gleichung}).

Thus, it is shown that the two axioms of \gls{wdbt} – the equation of motion (\refeq{eq:deterministische_bewegungsgleichung}) with quantum potential and the continuity equation (\refeq{eq:kontinuitätsgleichung}) – are mathematically equivalent to the postulated Schrödinger equation (\refeq{eq:schrödinger_gleichung}).

\section{Conclusion: Quantum Mechanics as an Emergent Description}
The Schrödinger equation thus emerges not as a fundamental postulate, but as an effective field description for the dynamics of an ensemble of particles, whose individual trajectories are guided by the deterministic equation (\refeq{eq:deterministische_bewegungsgleichung}) of \gls{wdbt}. \gls{wdbt} resolves the interpretational problems of conventional quantum mechanics by providing a realistic, causal, and non-local theory from which the statistical predictions of quantum mechanics arise.

\section{WDBT as a Comprehensive Theory: Beyond Quantum Mechanics}
Emergence always means that the higher-level theory, from which the subordinate theory emerges, can make more far-reaching statements (\textbf{empirical superiority}); otherwise, it would merely be an equivalence.

The derived equivalence to the Schrödinger equation shows that \gls{wdbt} reproduces all statistical predictions of conventional quantum mechanics. But \gls{wdbt} is more than \gls{qm}. While \gls{qm} contents itself with calculating probability densities $\rho = \left| \psi \right|^2$, \gls{wdbt} provides a \textbf{complete kinematic and dynamic description} of the underlying reality in the form of well-defined particle trajectories guided by the guiding wave $\psi$.

This ontological surplus of \gls{wdbt} makes it a more comprehensive theory and enables the explanation and modeling of phenomena that in orthodox \gls{qm} are not only unexplained but \textbf{actively obscured}, as they contradict the postulate of the completeness of the wave function.

The paradigmatic example is the \textbf{double-slit experiment}:

\begin{itemize}
    \item In \textbf{conventional \gls{qm}}, a \enquote{particle} arrives as a delocalized wave at the double slit, interferes with itself, and the probability of detecting it behind the slit at a specific location is given by $\left| \psi \right|^2$. The concept of a trajectory through a specific slit is forbidden and \enquote{nonsensical}.
    \item In \textbf{\gls{wdbt}}, each particle traverses exactly one of the two slits on a well-defined trajectory. However, the guiding wave $\psi$ passes through both slits, interferes behind them, and forms the familiar interference pattern. The quantum potential $Q$, which depends on the entire guiding wave, guides the particle along its trajectory into regions of constructive interference and away from regions of destructive interference.
\end{itemize}

\textbf{The crucial extension of \gls{wdbt}:} The guiding wave is reflected not only at the slit but also at the slit barrier. This creates complex interference patterns and variations in the quantum potential in the immediate vicinity of the barrier. Particles passing very close to the barrier experience a repulsive force (\enquote{kick}) from this modified quantum potential, which slightly deflects their trajectory towards the other slit.

\textbf{Result:} \gls{wdbt} predicts that a small but non-vanishing fraction of particles passing through the left slit will, due to this interaction with the barrier, finally be registered on the right side of the detector screen, and vice versa.

This phenomenon is:

\begin{enumerate}
    \item \textbf{Invisible in orthodox \gls{qm}:} The theory makes no statement about the path, only about the final outcome.
    \item \textbf{Calculable in WDBT:} The trajectories can (at least in principle) be calculated for given initial conditions.
    \item \textbf{Experimentally verifiable:} Modern experiments with sensitive probes near the slits could potentially detect these subtle repulsion effects.
\end{enumerate}

This underscores the status of \gls{wdbt} as a fundamental theory: It not only provides a consistent interpretation of established quantum phenomena but also expands the horizon for new, testable predictions beyond the framework of conventional quantum mechanics. The Schrödinger equation thus emerges as a subset of \gls{wdbt} – namely, as the description of the dynamics of the guiding wave – while the particle dynamics described by equation (\refeq{eq:deterministische_bewegungsgleichung}) open a deeper level of description.