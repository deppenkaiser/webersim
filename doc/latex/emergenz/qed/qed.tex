\chapter{Die Emergenz der Quantenelektrodynamik}
\section{Die Aufgabe der Reduktion}
Die \gls{qed} beschreibt die Wechselwirkung zwischen Licht und Materie mit unvergleichlicher Präzision. Ihr Herzstück ist das Konzept der Quantenfelder und virtueller Teilchen. Dieses Kapitel zeigt,
wie die scheinbar abstrakten Konzepte der \gls{qed} – die Feldquantisierung, Feynman-Diagramme und die Renormierung – aus der deterministischen, teilchenbasierten \gls{wdbt} emergieren. Die Strategie
ist nicht die 1:1-Rekonstruktion, sondern die Herleitung der operationalen Kernaussagen der \gls{qed} aus den ersten Prinzipien der \gls{wdbt}.

\section{Die vollständige Kraftgleichung der WDBT}
Die fundamentale Gleichung für ein geladenes Teilchen (Masse $m$, Ladung $q$) in der \gls{wdbt} ist die Erweiterung der Lorentz-Kraft um das Quantenpotential $Q$:

\begin{equation}
    \label{eq:kraft_wdbt_em}
    m \frac{d^2\vec{x}}{dt^2} = q(\vec{E} + \vec{v} \times \vec{B}) - \nabla Q
\end{equation}

mit:

\begin{equation}
    \label{eq:quantenpotential_wdbt_em}
    Q = -\frac{\hbar^2}{2m} \frac{\nabla^2 \sqrt{\rho}}{\sqrt{\rho}}
\end{equation}

Die Felder $\vec{E}$ und $\vec{B}$ sind dabei effektive Beschreibungen der gemittelten Weber-Wechselwirkung mit allen anderen Ladungen im Universum, wie in Kapitel \ref{ch:maxwell} hergeleitet. Gleichung
(\refeq{eq:kraft_wdbt_em}) ist \textbf{deterministisch} und beschreibt eine wohldefinierte Trajektorie.

\newpage
\section{Die Emergenz der Schrödinger- und Maxwell-Gleichungen}
Um die Statistik eines Ensembles identischer Systeme (z.B. viele Atome im gleichen Zustand) zu beschreiben, wird die Madelung-Transformation durchgeführt (vgl. Abschnitt \ref{sec:madelung}). Wir führen
eine komplexe Wellenfunktion $\psi(\vec{x},t)=R(\vec{x},t)e^{iS(\vec{x},t)/\hbar}$ ein, für die gilt:

\begin{enumerate}
    \item $\rho=\left| \psi \right|^2$ (Wahrscheinlichkeitsdichte)
    \item $\vec{v} = \frac{1}{m}\vec{\nabla}S$ (Teilchengeschwindigkeit)
\end{enumerate}

Wie in Abschnitt \ref{sec:schrödinger_gleichung} gezeigt, ist die Bewegungsgleichung (\refeq{eq:kraft_wdbt_em}) äquivalent zur zeitabhängigen Schrödinger-Gleichung für $\psi$:

\begin{equation}
    i\hbar \frac{\partial \psi}{\partial t} = \left( -\frac{\hbar^2}{2m} \nabla^2 + V + q\phi \right) \psi
\end{equation}

wobei das Vektorpotential $\vec{A}$ über $\vec{B} = \vec{\nabla} \times \vec{A}$ in die minimale Kopplung eingeht. Parallel dazu emergieren aus der Weber-Elektrodynamik durch Kontinuumslimes die
klassischen Maxwell-Gleichungen (Abschnitt \ref{sec:kontinuumslimes}). Damit ist die \textbf{nicht-relativistische Quantenmechanik} und die \textbf{klassische Elektrodynamik} bereits als effektive
Beschreibungsebenen aus der \gls{wdbt} hergeleitet.

\section{Photonen als Anregungen des Quanten-Vakuums}
Die konventionelle \gls{qed} postuliert die Quantisierung des elektromagnetischen Feldes. In der \gls{wdbt} ergibt sich dieser Schritt naturalistisch aus der Berücksichtigung des Quantenpotentials
des Vakuums.

Das Vakuum ist in der \gls{wdbt} kein leerer Raum, sondern ein \textbf{Quantenmedium} mit einer Grundfluktuation, beschrieben durch eine Vakuum-Wellenfunktion $\psi_\text{Vak}$. Deren zugehöriges Quantenpotential
$Q_\text{Vak}$ wirkt auf alle Teilchen. Die Anregungen dieses Mediums – beschrieben durch stehende Wellen in einer Box mit bestimmten Frequenzen – entsprechen den Photonen.

Die Energie einer Anregung der Frequenz $\omega$ ist:

\begin{equation}
    E = \hbar \omega
\end{equation}

Diese Relation emergiert aus der Skalierung des Quantenpotentials $Q$ mit $\hbar^2$ und der Dispensionsrelation der \gls{wed}. Die \textbf{Feldoperatoren} der \gls{qed} (Erzeuger/Vernichter) sind deshalb
mathematische Hilfsgrößen zur Beschreibung dieser angeregten Moden des Quanten-Vakuums, nicht fundamentale Entitäten.

\section{Herleitung der Feynman-Regeln aus der nicht-lokalen Weber-Wechselwirkung}
Die Stärke der \gls{qed} liegt in der Störungstheorie und den \textbf{Feynman-Diagrammen}. Diese emergieren in der \gls{wdbt} aus der \textbf{Mittelung über alle möglichen nicht-lokalen Wechselwirkungspfade}.

In der \gls{wed} wirkt die Kraft zwischen zwei Ladungen instantan und hängt von Geschwindigkeit und Beschleunigung ab. Die Wahrscheinlichkeitsamplitude dafür, dass ein Teilchen von $A$ nach $B$ gelangt,
muss über alle möglichen Trajektorien summiert werden, die alle durch ihre eigene Führungswelle und ihr Quantenpotential beeinflusst werden.

\subsubsection{Herleitungsschritte:}
\begin{enumerate}
    \item Die \textbf{Wirkung} $S$ für einen Weber-Wechselwirkungspfad wird definiert.
\end{enumerate}

