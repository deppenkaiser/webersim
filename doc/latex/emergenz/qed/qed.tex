\chapter{Die Emergenz der Quantenelektrodynamik}
\section{Die Aufgabe der Reduktion}
Die \gls{qed} beschreibt die Wechselwirkung zwischen Licht und Materie mit unvergleichlicher Präzision. Ihr Herzstück ist das Konzept der Quantenfelder und virtueller Teilchen. Dieses Kapitel zeigt,
wie die scheinbar abstrakten Konzepte der \gls{qed} – die Feldquantisierung, Feynman-Diagramme und die Renormierung – aus der deterministischen, teilchenbasierten \gls{wdbt} emergieren. Die Strategie
ist nicht die 1:1-Rekonstruktion, sondern die Herleitung der operationalen Kernaussagen der \gls{qed} aus den ersten Prinzipien der \gls{wdbt}.

\section{Die vollständige Kraftgleichung der WDBT}
Die fundamentale Gleichung für ein geladenes Teilchen (Masse $m$, Ladung $q$) in der \gls{wdbt} ist die Erweiterung der Lorentz-Kraft um das Quantenpotential $Q$:

\begin{equation}
    \label{eq:kraft_wdbt_em}
    m \frac{d^2\vec{x}}{dt^2} = q(\vec{E} + \vec{v} \times \vec{B}) - \nabla Q
\end{equation}

mit:

\begin{equation}
    \label{eq:quantenpotential_wdbt_em}
    Q = -\frac{\hbar^2}{2m} \frac{\nabla^2 \sqrt{\rho}}{\sqrt{\rho}}
\end{equation}

Die Felder $\vec{E}$ und $\vec{B}$ sind dabei effektive Beschreibungen der gemittelten Weber-Wechselwirkung mit allen anderen Ladungen im Universum, wie in Kapitel \ref{ch:maxwell} hergeleitet. Gleichung
(\refeq{eq:kraft_wdbt_em}) ist \textbf{deterministisch} und beschreibt eine wohldefinierte Trajektorie.
