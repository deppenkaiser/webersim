\chapter{Die Emergenz der Quantenmechanik}
\section{Die fundamentale Bewegungsgleichung der WDBT}
Die analoge \gls{wdbt} postuliert für ein Teilchen der Masse $m$ die folgende vollständige Kraftgleichung. Für ein neutrales Teilchen, auf das nur ein konservatives Potential $V(\vec{x})$ und das
Quantenpotential $Q$ wirken, vereinfacht sich diese zu:

\begin{equation}
    \label{eq:deterministische_bewegungsgleichung}
    m \frac{d^2\vec{x}}{dt^2} = -\vec{\nabla} V - \vec{\nabla} Q
\end{equation}

wobei das Bohm'sche Quantenpotential $Q$ definiert ist als:

\begin{equation}
    \label{eq:quantenpotential}
    Q = -\frac{\hbar^2}{2m} \frac{\nabla^2 R}{R}
\end{equation}

Hier ist $R = R(\vec{x},t)$ die Amplitude der sogenannten \enquote{Führungswelle} oder \enquote{pilot wave}, und $\rho = R^2$ ist die Wahrscheinlichkeitsdichte des Teilchensensembles. Gleichung
(\refeq{eq:deterministische_bewegungsgleichung}) ist die \textbf{deterministische Bewegungsgleichung} eines Teilchens in der \gls{wdbt}.

\section{Die Madelung-Transformation: Von der Teilchentrajektorie zur Feldbeschreibung}
Um von der Beschreibung einzelner Teilchentrajektorien zur Beschreibung eines Ensembles (einem \enquote{Fluid}) überzugehen, führen wir die Madelung-Transformation durch. Wir formulieren eine komplexe
Wellenfunktion $\psi(\vec{x}, t)$, deren Phase $S$ mit dem Impuls des Teilchens und deren Betragsquadrat mit der Dichte verbunden ist:

\begin{equation}
    \psi(\vec{x}, t) = R(\vec{x}, t)  e^{i S(\vec{x}, t) / \hbar}
\end{equation}

Dabei ist:

\begin{itemize}
    \item $R(\vec{x},t)$: Reelle Amplitude ($\rho = R^2$ ist die Wahrscheinlichkeitsdichte).
    \item $S(\vec{x},t)$: Reelle Phasenfunktion (entspricht der Wirkung bzw. dem Impulspotential).
\end{itemize}

Unser Ziel ist es nun zu zeigen, dass die Bewegungsgleichung (\refeq{eq:deterministische_bewegungsgleichung}) und die Annahme einer Kontinuitätsgleichung für die Dichte $\rho$ äquivalent zur
Schrödinger-Gleichung für $\psi$ sind.

\section{Herleitung der Kontinuitätsgleichung}
\label{sec:kontinuitätsgleichung}
Der Impuls $\vec{p}$ eines Teilchens auf seiner Trajektorie ist in der \gls{wdbt} durch den Gradienten der Phasenfunktion $S$ gegeben:

\begin{equation}
    \label{eq:impuls}
    \vec{p} = m \vec{v} = \vec{\nabla} S
\end{equation}

Die Erhaltung der Wahrscheinlichkeit (Teilchen können nicht einfach verschwinden oder erzeugt werden) führt auf eine Kontinuitätsgleichung für die Dichte $\rho$:

\begin{equation}
    \frac{\partial \rho}{\partial t} + \vec{\nabla} \cdot (\rho \vec{v}) = 0
\end{equation}

Setzen wir $\rho = R^2$ und $\vec{v} = \frac{\vec{\nabla} S}{m}$ ein, erhalten wir:

\begin{equation}
    \label{eq:kontinuitätsgleichung}
    \frac{\partial (R^2)}{\partial t} + \vec{\nabla} \cdot \left( R^2 \frac{\vec{\nabla} S}{m} \right) = 0
\end{equation}

Diese Gleichung beschreibt die zeitliche Entwicklung der Dichteverteilung des Ensembles.

\section{Herleitung der Hamilton-Jacobi-Gleichung}
\label{sec:hamilton_jakobi_gleichung}
Wir beginnen nun mit der Newtonschen Bewegungsgleichung der \gls{wdbt} (Gl. \refeq{eq:deterministische_bewegungsgleichung}) und formen sie schrittweise um.

\begin{equation}
    m \frac{d^2\vec{x}}{dt^2} = -\vec{\nabla} V - \vec{\nabla} Q \tag{3.1}
\end{equation}

Die substantielle Zeitableitung der Geschwindigkeit $\vec{v} = \frac{d \vec{x}}{dt}$ ist:

\begin{equation}
    \frac{d\vec{v}}{dt} = \frac{\partial \vec{v}}{\partial t} + (\vec{v} \cdot \vec{\nabla}) \vec{v}
\end{equation}

Mit $\vec{v} = \frac{\vec{\nabla} S}{m}$ (aus Gl. \refeq{eq:impuls}) wird der Beschleunigungsterm ($\vec{v} \cdot \vec{\nabla})$ zu:

\begin{equation}
    (\vec{v} \cdot \vec{\nabla}) \vec{v} = \frac{1}{m^2} \left[ (\vec{\nabla} S \cdot \vec{\nabla}) \vec{\nabla} S \right]
\end{equation}

Eine nützliche Vektoridentität hilft uns, diesen Term umzuschreiben:

\begin{equation}
    (\vec{\nabla} S \cdot \vec{\nabla}) \vec{\nabla} S = \frac{1}{2} \vec{\nabla} (\vec{\nabla} S \cdot \vec{\nabla} S) = \frac{1}{2} \vec{\nabla} (\left|\vec{\nabla} S \right|^2)
\end{equation}

Somit wird die linke Seite der Bewegungsgleichung zu:

\begin{equation}
    m \frac{d\vec{v}}{dt} = m \left( \frac{\partial \vec{v}}{\partial t} \right) + \frac{1}{2m} \vec{\nabla} (\left| \vec{\nabla} S \right|^2)
\end{equation}

Nun setzen wir $\vec{v} = \frac{\vec{\nabla} S}{m}$ ein und setzen dies gleich der rechten Seite von (\refeq{eq:deterministische_bewegungsgleichung}):

\begin{equation}
    m \frac{\partial}{\partial t}\left( \frac{\vec{\nabla} S}{m} \right) + \frac{1}{2m} \vec{\nabla} (\left| \vec{\nabla} S \right|^2) = -\vec{\nabla} V - \vec{\nabla} Q
\end{equation}

Durch Vertauschung ($\frac{\partial}{\partial t} \vec{\nabla} S = \vec{\nabla} \frac{\partial S}{\partial t}$), vereinfacht sich dies zu:

\begin{equation}
    \vec{\nabla} \left( \frac{\partial S}{\partial t} \right) + \frac{1}{2m} \vec{\nabla} (\left| \vec{\nabla} S \right|^2) = -\vec{\nabla} V - \vec{\nabla} Q
\end{equation}

Wir können nun den Gradientenoperator $\vec{\nabla}$ auf alle Terme anwenden:

\begin{equation}
    \label{eq:zwischenstand}
    \vec{\nabla} \left( \frac{\partial S}{\partial t} + \frac{1}{2m} \left| \vec{\nabla} S \right|^2 + V + Q \right) = 0
\end{equation}

Gleichung (\refeq{eq:zwischenstand}) besagt, dass der Ausdruck in der Klammer räumlich konstant ist. Dies führt uns zur modifizierten Hamilton-Jacobi-Gleichung:

\begin{equation}
    \frac{\partial S}{\partial t} + \frac{1}{2m} \left| \vec{\nabla} S \right|^2 + V + Q = 0
\end{equation}

Setzen wir nun die Definition des Quantenpotentials (\refeq{eq:quantenpotential}) ein, erhalten wir die zentrale Gleichung:

\begin{equation}
    \label{eq:modifizierte_hamilton_jacobi}
    \frac{\partial S}{\partial t} + \frac{1}{2m} \left| \vec{\nabla} S \right|^2 + V - \frac{\hbar^2}{2m} \frac{\nabla^2 R}{R} = 0
\end{equation}

\section{Synthese zur Schrödinger-Gleichung}
\label{sec:schrödinger_gleichung}
Wir haben nun die beiden realen Gleichungen, die die Dynamik des Systems beschreiben:

\begin{enumerate}
    \item Die Kontinuitätsgleichung (\refeq{eq:kontinuitätsgleichung})
    \item Die modifizierte Hamilton-Jacobi-Gleichung (\refeq{eq:modifizierte_hamilton_jacobi})
\end{enumerate}

Die geniale Einsicht ist, dass diese beiden Gleichungen äquivalent zur einen komplexen Schrödinger-Gleichung sind. Um dies zu sehen, setzen wir die Wellenfunktion $\psi = R e^{iS/\hbar}$
in die zeitabhängige Schrödinger-Gleichung ein:

\begin{equation}
    \label{eq:schrödinger_gleichung}
    i\hbar \frac{\partial \psi}{\partial t} = \left( -\frac{\hbar^2}{2m} \nabla^2 + V \right) \psi
\end{equation}

Wir führen die Ableitungen explizit aus.

\textbf{Linke Seite:}

\begin{equation}
    \frac{\partial \psi}{\partial t} = \frac{\partial R}{\partial t} e^{iS/\hbar} + R \cdot \frac{i}{\hbar} \frac{\partial S}{\partial t} e^{iS/\hbar}
\end{equation}

\begin{equation}
    i\hbar \frac{\partial \psi}{\partial t} = i\hbar \frac{\partial R}{\partial t} e^{iS/\hbar} - R \frac{\partial S}{\partial t} e^{iS/\hbar}
\end{equation}

\textbf{Rechte Seite:}\\
Wir berechnen zuerst $\nabla^2 \psi$.

\begin{equation}
    \vec{\nabla} \psi = (\vec{\nabla} R) e^{iS/\hbar} + R \frac{i}{\hbar} (\vec{\nabla} S) e^{iS/\hbar}
\end{equation}

\begin{equation}
    \nabla^2 \psi = \left[ \nabla^2 R + \frac{2i}{\hbar} (\vec{\nabla} R \cdot \vec{\nabla} S) + \frac{i}{\hbar} R \nabla^2 S - \frac{1}{\hbar^2} R \left| \vec{\nabla} S \right|^2 \right] e^{iS/\hbar}
\end{equation}

Somit wird die rechte Seite:

\begin{equation}
    \left( -\frac{\hbar^2}{2m} \nabla^2 + V \right) \psi = -\frac{\hbar^2}{2m} \left[ \nabla^2 R + \frac{2i}{\hbar} (\vec{\nabla} R \cdot \vec{\nabla} S) + \frac{i}{\hbar} R \nabla^2 S - \frac{1}{\hbar^2} R \left| \vec{\nabla} S \right|^2 \right] e^{iS/\hbar} + V R e^{iS/\hbar}
\end{equation}

Setzen wir nun linke und rechte Seite gleich und teilen durch den gemeinsamen Faktor $e^{iS/\hbar}$:

\begin{equation}
    i\hbar \frac{\partial R}{\partial t} - R \frac{\partial S}{\partial t} = -\frac{\hbar^2}{2m} \nabla^2 R - \frac{i\hbar}{m} (\vec{\nabla} R \cdot \vec{\nabla} S) - \frac{i\hbar}{2m} R \nabla^2 S + \frac{1}{2m} R \left| \vec{\nabla} S \right|^2 + V R
\end{equation}

Wir trennen nun Real- und Imaginärteil.

\textbf{Imaginärteil:}

\begin{equation}
    \hbar \frac{\partial R}{\partial t} = -\frac{\hbar}{m} (\vec{\nabla} R \cdot \vec{\nabla} S) - \frac{\hbar}{2m} R \nabla^2 S
\end{equation}

Multipliziert man beide Seiten mit $2R/\hbar$, erhält man:

\begin{equation}
    2R \frac{\partial R}{\partial t} = -\frac{2}{m} R (\vec{\nabla} R \cdot \vec{\nabla} S) - \frac{1}{m} R^2 \nabla^2 S
\end{equation}

\begin{equation}
    \frac{\partial (R^2)}{\partial t} + \vec{\nabla} \cdot \left( R^2 \frac{\vec{\nabla} S}{m} \right) = 0 \tag{\refeq{eq:kontinuitätsgleichung}}
\end{equation}

Dies ist exakt die \textbf{Kontinuitätsgleichung (\refeq{eq:kontinuitätsgleichung})} aus Abschnitt (\ref{sec:kontinuitätsgleichung}).

\textbf{Realteil:}

\begin{equation}
    - R \frac{\partial S}{\partial t} = -\frac{\hbar^2}{2m} \nabla^2 R + \frac{1}{2m} R \left| \vec{\nabla} S \right|^2 + V R
\end{equation}

Teilt man durch $R$:

\begin{equation}
    - \frac{\partial S}{\partial t} = -\frac{\hbar^2}{2m} \frac{\nabla^2 R}{R} + \frac{1}{2m} |\nabla S|^2 + V
\end{equation}

\begin{equation}
    \frac{\partial S}{\partial t} + \frac{1}{2m} \left| \vec{\nabla} S \right|^2 + V - \frac{\hbar^2}{2m} \frac{\nabla^2 R}{R} = 0 \tag{\refeq{eq:modifizierte_hamilton_jacobi}}
\end{equation}

Dies ist exakt die modifizierte Hamilton-Jacobi-Gleichung (\refeq{eq:modifizierte_hamilton_jacobi}) aus Abschnitt (\ref{sec:schrödinger_gleichung}).

Somit ist gezeigt, dass die beiden Axiome der \gls{wdbt} – die Bewegungsgleichung (\refeq{eq:deterministische_bewegungsgleichung}) mit Quantenpotential und die Kontinuitätsgleichung
(\refeq{eq:kontinuitätsgleichung}) – mathematisch äquivalent zur postulierten Schrödinger-Gleichung (\refeq{eq:schrödinger_gleichung}) sind.

\section{Fazit: Quantenmechanik als emergente Beschreibung}
Die Schrödinger-Gleichung emergiert somit nicht als fundamentales Postulat, sondern als eine effektive Feldbeschreibung für die Dynamik eines Ensembles von Teilchen, deren individuelle Trajektorien
durch die deterministische Gleichung (\refeq{eq:deterministische_bewegungsgleichung}) der \gls{wdbt} gelenkt werden. Die \gls{wdbt} löst die interpretatorischen Probleme der konventionellen
Quantenmechanik, indem sie eine realistische, kausale und nicht-lokale Theorie bietet, aus der die statistischen Vorhersagen der Quantenmechanik hervorgehen.

\section{Die WDBT als umfassende Theorie: Jenseits der Quantenmechanik}
Emergenz bedeutet immer, dass die übergeordnete Theorie, aus welcher die untergeordnete Theorie emergiert, weitreichendere Aussagen tätigen kann (\textbf{empirische Überlegenheit}); sonst wäre es
nur eine Gleichsetzung.

Die hergeleitete Äquivalenz zur Schrödinger-Gleichung zeigt, dass die \gls{wdbt} alle statistischen Vorhersagen der konventionellen Quantenmechanik reproduziert. Doch die \gls{wdbt} ist mehr als die \gls{qm}.
Während die \gls{qm} sich mit der Berechnung von Wahrscheinlichkeitsdichten $\rho = \left| \psi \right|^2$ begnügt, bietet die \gls{wdbt} eine \textbf{vollständige kinematische und dynamische Beschreibung}
der zugrundeliegenden Realität in Form von wohldefinierten Teilchentrajektorien, die durch die Führungswelle $\psi$ geleitet werden.

Dieser ontologische Überschuss der \gls{wdbt} macht sie zu einer umfassenderen Theorie und ermöglicht die Erklärung und Modellierung von Phänomenen, die in der orthodoxen \gls{qm} nicht nur unerklärt,
sondern \textbf{aktiv ausgeblendet} werden, da sie dem Postulat der Vollständigkeit der Wellenfunktion widersprechen.

Das paradigmatische Beispiel ist das \textbf{Doppelspaltexperiment}:

\begin{itemize}
    \item In der \textbf{konventionellen \gls{qm}} trifft ein \enquote{Teilchen} als delokalisierte Welle auf den Doppelspalt, interferiert mit sich selbst, und die Wahrscheinlichkeit, es hinter dem Spalt an einem bestimmten Ort zu detektieren, wird durch $\left| \psi \right|^2$ gegeben. Der Begriff einer Trajektorie durch einen bestimmten Spalt ist verboten und \enquote{unsinnig}.
    \item In der \textbf{\gls{wdbt}} durchquert jedes Teilchen auf einer wohlbestimmten Trajektorie genau einen der beiden Spalte. Die Führungswelle $\psi$ jedoch durchläuft beide Spalte, interferiert hinter ihnen und bildet das bekannte Interferenzmuster aus. Das Quantenpotential $Q$, das von der gesamten Führungswelle abhängt, lenkt das Teilchen entlang seiner Trajektorie in die Bereiche konstruktiver Interferenz und weg von den Bereichen destruktiver Interferenz.
\end{itemize}

\textbf{Die entscheidende Erweiterung der \gls{wdbt}:} Die Führungswelle wird nicht nur am Spalt, sondern auch am Spaltsteg reflektiert. Dies erzeugt komplexe Interferenzmuster und Variationen im
Quantenpotential in unmittelbarer Nähe des Steges. Teilchen, die sehr nahe am Steg vorbeifliegen, erfahren durch dieses modifizierte Quantenpotential eine abstoßende Kraft (\enquote{kick}), die ihre
Trajektorie leicht in Richtung des anderen Spalts ablenkt.

\textbf{Resultat:} Die \gls{wdbt} sagt vorher, dass ein kleiner, aber nicht verschwindender Anteil der Teilchen, die den linken Spalt passieren, aufgrund dieser Wechselwirkung mit dem Steg final auf
der rechten Seite des Detektorschirms registriert werden, und umgekehrt.

Dieses Phänomen ist:

\begin{enumerate}
    \item \textbf{In der orthodoxen \gls{qm} unsichtbar:} Die Theorie macht keine Aussage über den Weg, nur über das Endergebnis.
    \item \textbf{In der WDBT berechenbar:} Die Trajektorien können (zumindest prinzipiell) für gegebene Anfangsbedingungen berechnet werden.
    \item \textbf{Experimentell überprüfbar:} Moderne Experimente mit empfindlichen Sonden in der Nähe der Spalte könnten diese subtilen Abstoßeffekte möglicherweise nachweisen.
\end{enumerate}

Dies unterstreicht den Status der \gls{wdbt} als fundamentale Theorie: Sie bietet nicht nur eine konsistente Interpretation der etablierten Quantenphänomene, sondern erweitert den Horizont für neue,
testbare Vorhersagen jenseits des Rahmens der konventionellen Quantenmechanik. Die Schrödinger-Gleichung emergiert somit als Teilmenge der \gls{wdbt} – nämlich als die Beschreibung der Dynamik der
Führungswelle –, während die Teilchendynamik durch die Gleichung (\refeq{eq:deterministische_bewegungsgleichung}) eine tiefere Beschreibungsebene eröffnet.

