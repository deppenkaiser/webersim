\chapter{Fazit}
\section{Systematische Widersprüche der etablierten Theorien und ihre Auflösung durch die WDBT}
\subsection{Die Widersprüche der ART}
Die \gls{art} steht auf tönernen Füßen – ihre zentralen Postulate entpuppen sich bei genauer Betrachtung als mathematische Fiktionen ohne physikalische Grundlage.
\begin{enumerate}
    \item \textbf{Singularitäten:} Der Bankrott der Theorie\\Die \gls{art} sagt die Existenz von Punkten unendlicher Dichte in Schwarzen Löchern und beim Urknall voraus – ein klarer Verstoß gegen jedes physikalische Prinzip. Während die \gls{art} hier kapituliert, löst die \gls{wdbt} das Problem durch das Quantenpotential $Q$, das bei kleinen Abständen abstoßend wirkt und so Singularitäten verhindert (Gl. \refeq{eq:wg-dbt-q}).
    \item \textbf{Dunkle Materie:} Der erfundene Rettungsanker\\Seit Jahrzehnten jagt die Physik nach \enquote{dunkler Materie}, um die Diskrepanz zwischen \gls{art}-Vorhersagen und beobachteten Galaxienrotationen zu erklären. Die \gls{wdbt} macht diese Hilfskonstruktion überflüssig: Die fraktale Raumstruktur und das Quantenpotential liefern eine natürliche Erklärung für die Rotationskurven (Gl. \refeq{eq:rotationskurve}).
    \item \textbf{Raumzeitkrümmung:} Ein metaphysisches Konstrukt\\Die \gls{art} beschreibt Gravitation als Krümmung einer abstrakten Raumzeit, bleibt aber die Antwort schuldig, wie Materie diese Krümmung verursacht. Die \gls{wdbt} ersetzt dieses mysteriöse Konzept durch die direkte Weber-Wechselwirkung zwischen Massen (Gl. \refeq{eq:wg-beta}) – eine physikalisch interpretierbare Kraft.
    \item \textbf{Lokalitätsdogma vs. Quantenrealität}\\Während die \gls{art} strikte Lokalität fordert, zeigen Quantenexperimente (\gls{epr}, Bell-Tests) eindeutig nicht-lokale Korrelationen. Die \gls{wdbt} integriert diese Effekte durch das Quantenpotential, das instantan wirkt, ohne die Kausalität zu verletzen.
\end{enumerate}
\subsection{Die Widersprüche der Maxwell-Theorie}
Die klassische Elektrodynamik ist ebenfalls von fundamentalen Inkonsistenzen durchzogen, die in Lehrbüchern systematisch verschleiert werden.
\begin{enumerate}
    \item \textbf{Die Selbstenergie-Katastrophe}\\Die \gls{mt} sagt für Punktladungen eine unendliche Selbstenergie voraus – ein untrügliches Zeichen dafür, dass das Feldkonzept an seine Grenzen stößt. Die Weber-Elektrodynamik umgeht dieses Problem elegant: Da sie ohne Felder auskommt, gibt es keine divergierenden Energien.
    \item \textbf{Das Strahlungsdämpfungs-Paradoxon}\\Nach der \gls{mt} sollte jedes beschleunigte geladene Teilchen strahlen – doch warum tut ein Elektron im homogenen Gravitationsfeld dies nicht? Die Weber-Theorie löst das Rätsel: Strahlung tritt nur bei relativer Beschleunigung zwischen Ladungen auf (Gl. \refeq{eq:weber-em-damp}).
    \item \textbf{Der Aharonov-Bohm-Effekt:} Das Ende des Feld-Dogmas\\Experimente zeigen, dass Quantenteilchen durch das Vektorpotential $\vec{A}$ beeinflusst werden – selbst in Regionen ohne elektromagnetisches Feld. Dies widerlegt die MT-Ansicht, dass nur $\vec{E}$ und $\vec{B}$ physikalisch real seien. Die Weber-Elektrodynamik kommt ganz ohne Potentiale aus und erklärt die Effekte durch direkte Ladungswechselwirkungen.
    \item \textbf{Virtuelle Teilchen:} Die große Illusion\\Die \gls{qed} führt \enquote{virtuelle Photonen} ein, die scheinbar überlichtschnell wechselwirken – ein klarer Verstoß gegen die Relativitätstheorie, der als \enquote{Pfadintegral-Trick} kaschiert wird. Die Weber-Elektrodynamik zeigt: Solche Hilfskonstrukte sind überflüssig, wenn man direkte, geschwindigkeitsabhängige Wechselwirkungen zulässt.
\end{enumerate}
\subsection{Die Heuchelei des Establishments}
Die Doppelstandards der etablierten Physik sind unübersehbar:
\begin{itemize}
    \item \textbf{Für die ART/MT erlaubt:}
    \begin{itemize}
        \item Unendlichkeiten (Singularitäten, Selbstenergien).
        \item Erfundene Entitäten (dunkle Materie, virtuelle Teilchen).
        \item Widersprüche zur Quantenmechanik (Lokalitätsproblem).
    \end{itemize}
    \item \textbf{Für die WDBT verboten:}
    \begin{itemize}
        \item Jede Abweichung vom Feld-Paradigma – trotz experimenteller Anomalien.
        \item Die Forderung nach mechanistischen Erklärungen („Wie krümmt Masse die Raumzeit?“).
    \end{itemize}
\end{itemize}
Gleichzeitig werden Forscher wie David Bohm oder André Koch Torres Assis systematisch ausgegrenzt – nicht weil ihre Theorien falsch wären, sondern weil sie das Machtgefüge der etablierten
Physik bedrohen.

\subsection{Der Weg zur wissenschaftlichen Revolution}
Diese Widersprüche sind keine Lappalien – sie zeigen, dass die \gls{art} und \gls{mt} fundamental unvollständig sind. Die \gls{wdbt} bietet nicht nur Lösungen, sondern eine kohärente Alternative:
\begin{itemize}
    \item Keine Singularitäten (dank Quantenpotential).
    \item Keine dunkle Materie (durch fraktale Raumstruktur).
    \item Keine Felder (direkte Wechselwirkungen).
\end{itemize}
Es ist an der Zeit, diese Wahrheit unverblümt auszusprechen: Die etablierten Theorien sind gescheitert – die \gls{wdbt} ist der Ausweg.
