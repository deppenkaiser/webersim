\section{Berechnung des Impulses \( \mathbf{p}(t) \)}
Der Impuls in Polarkoordinaten:
\[
\mathbf{p}(t) = m \left( \dot{r} \hat{r} + r \dot{\phi} \hat{\phi} \right)
\]

Einsetzen der berechneten Größen:
\[
\mathbf{p}(t) = \frac{L}{a(1 - e^2)} \left( e \sin \phi (1 + e \cos \phi) \hat{r} + (1 + e \cos \phi) \hat{\phi} \right)
\]

\subsection{Endergebnis}
\[
\boxed{ \mathbf{p}(t) = \frac{L}{a(1 - e^2)} \left[ e \sin \phi(t) (1 + e \cos \phi(t)) \hat{r} + (1 + e \cos \phi(t)) \hat{\phi} \right] }
\]
mit \( \phi(t) \) bestimmt durch:
\[
\dot{\phi} = \frac{L (1 + e \cos \phi)^2}{m a^2 (1 - e^2)^2}
\]