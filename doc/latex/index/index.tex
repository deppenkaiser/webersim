\documentclass{article}
\usepackage[utf8]{inputenc}
\usepackage[german]{babel}
\usepackage{amsmath}
\usepackage{amssymb}
\usepackage{graphicx}
\usepackage{hyperref}
\usepackage{xcolor}
\usepackage{booktabs}
\usepackage{tabularx}
\usepackage{enumitem}
\usepackage{geometry}
\usepackage{float}
\usepackage{subfiles}
\usepackage{textcomp}
\usepackage{pifont}

\geometry{a4paper, margin=2cm}

\title{Weber-Kraft als fundamentale Theorie der Quantengravitation}
\author{}
\date{}

\begin{document}

\maketitle

\subfile{sections/00_zusammenfassung}
\subfile{sections/01_einfuehrung}
\subfile{sections/02_periheldrehung}
\subfile{sections/03_interpretation}
\subfile{sections/04_grenzen}
\subfile{sections/05_beta_formel}
\subfile{sections/06_universal_formel}
\subfile{sections/07_rotverschiebung}
\subfile{sections/08_gravitationswellen}
\subfile{sections/09_quantized_space}
\subfile{sections/10_knoten_theorie}
\subfile{sections/11_qed}
\subfile{sections/12_vorhersagekraft}
\subfile{sections/13_history}
\subfile{sections/14_roadmap}
\subfile{sections/15_vergleich}
\subfile{sections/16_literatur}

\end{document}
