\section{Natürliche Zeitdefinition für Himmelskörper}

\subsection{Grundprinzip der Winkelzeit}
\begin{equation}
\tau = N + \frac{\phi}{2\pi}
\end{equation}
\begin{itemize}
    \item $N$ = Anzahl vollendeter Umläufe (ganzzahlig)
    \item $\phi$ = wahre Anomalie ($0 \leq \phi < 2\pi$)
\end{itemize}

\subsection{Erde-Mond-Zeitsystem}
\subsubsection{Erdzeit (ET)}
\begin{equation}
\tau_{\text{Erde}} = N_E + \frac{\phi_E}{2\pi}
\end{equation}
\begin{itemize}
    \item 1 ET-Jahr = 1 Erdumlauf (365.25 Tage)
    \item 1 ET-Tag = $2\pi$ Rotation (24 Stunden)
\end{itemize}

\subsubsection{Mondzeit (LT)}
\begin{equation}
\tau_{\text{Mond}} = N_M + \frac{\phi_M}{2\pi}
\end{equation}
\begin{itemize}
    \item 1 LT-Jahr = 1 Mondumlauf (27.3 Tage)
    \item 1 LT-Tag = $2\pi$ Rotation (29.5 ET-Tage)
\end{itemize}

\subsection{Zeitumrechnung}
\subsubsection{Kepler-Gleichung für den Mond}
\begin{equation}
E - e\sin E = M(t) = \sqrt{\frac{GM}{a^3}} \cdot t
\end{equation}
\begin{equation}
\phi_M = 2 \arctan\left(\sqrt{\frac{1+e}{1-e}} \tan\frac{E}{2}\right)
\end{equation}

\subsection{Kalendersystem}
\begin{tabular}{lll}
    \hline
    Element & Erde & Mond \\
    \hline
    Grundzyklus & Sonnenumlauf (Jahr) & Erdumlauf (Monat) \\
    Untereinheit & Eigenrotation (Tag) & Eigenrotation (Lunation) \\
    Natürliche Zeit & $\tau_E = N_E + \frac{\phi_E}{2\pi}$ & $\tau_M = N_M + \frac{\phi_M}{2\pi}$ \\
    \hline
\end{tabular}

\subsection{Implementierung}
\begin{itemize}
    \item Natürliche Synchronisation mit Himmelskörpern
    \item Keine willkürlichen Zeitzonen
    \item Direkte Korrelation mit Sonnen-/Erdposition
    \item Universelle Anwendbarkeit auf alle Himmelskörper
\end{itemize}

\begin{verbatim}
LOCAL TIME SYSTEM: LUNA-STATION-1
MOON TIME: CYCLES=683.214 [PHI=1.34rad]
EARTH TIME: CYCLES=1969.552 [PHI=4.71rad]
SUN POSITION: 47° ABOVE HORIZON
EARTH POSITION: 23° ABOVE HORIZON
\end{verbatim}