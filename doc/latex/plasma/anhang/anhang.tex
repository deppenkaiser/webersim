\chapter{Begründung der Sternentstehung in der WDBT}
\label{app:sternentstehung}
Dieser Anhang beschreibt die Physik der Sternenentstehung.

\section{Grundgleichungen der Plasmadynamik}
Die Dynamik eines Plasmas in der Weber-De-Broglie-Bohm-Theorie wird durch folgendes gekoppeltes System beschrieben:

\begin{equation}
\frac{\partial S}{\partial t} + \frac{(\nabla S - q\vec{A})^2}{2m} + V + Q + \Phi_{\text{Weber}} = 0
\end{equation}

\begin{equation}
\frac{\partial \rho}{\partial t} + \nabla \cdot \left(\rho \frac{\nabla S - q\vec{A}}{m}\right) = 0
\end{equation}

wobei:
\begin{itemize}
\item $S(\vec{r},t)$ die Wirkungsfunktion
\item $\rho(\vec{r},t)$ die Teilchendichte
\item $Q = -\frac{\hbar^2}{2m}\frac{\nabla^2 \sqrt{\rho}}{\sqrt{\rho}}$ das Quantenpotential
\item $\Phi_{\text{Weber}}$ das Weber-Potential
\end{itemize}

\section{Weber-Kraft und Gravitation}
Die kombinierte Weber-Kraft für Gravitation und Elektrodynamik:

\begin{equation}
\vec{F}_{12} = \left[\frac{Gm_1m_2 - \frac{q_1q_2}{4\pi\epsilon_0}}{r^2}\right]\left(1 - \frac{v^2}{c^2} + \frac{2r\ddot{r}}{c^2}\right)\hat{r}
\end{equation}

Für ein Plasma mit Korrelationsfunktion $g(\vec{r})$:

\begin{equation}
\vec{f} = \int d^3r'\, \rho(\vec{r})\rho(\vec{r}')\vec{F}_{\text{Weber}}(|\vec{r}-\vec{r}'|)g(\vec{r}-\vec{r}')
\end{equation}

\section{Stabilitätsanalyse einer kollabierenden Wolke}
Für eine homogene sphärische Wolke mit Radius $R(t)$:

\begin{equation}
\ddot{R} = -\frac{GM}{R^2} + \frac{9\hbar^2}{4m^2R^3} - \frac{3e^2N^{2/3}}{20\pi\epsilon_0 m R^2}
\end{equation}

\subsection{Jeans-Kriterium}
Kollapsbedingung:

\begin{equation}
M > \frac{9\hbar^2}{4Gm^2R} + \frac{3e^2N^{2/3}}{20\pi\epsilon_0 Gm}
\end{equation}

\section{Numerische Lösung}
Mit $R(t) = R_0 f(t)$:

\begin{equation}
\frac{d^2f}{dt^2} = -\frac{GM}{R_0^3 f^2} + \frac{9\hbar^2}{4m^2R_0^4 f^3}
\end{equation}

Approximative Lösung für $f \ll 1$:

\begin{equation}
f(t) \approx R_0\left(1 - \frac{t}{t_{\text{coll}}}\right)^{2/3}, \quad t_{\text{coll}} = \sqrt{\frac{R_0^3}{GM}}
\end{equation}

\section{Experimenteller Vergleich}

\begin{table}[h]
\centering
\caption{Vergleich der Vorhersagen mit Beobachtungsdaten}
\begin{tabular}{lcccc}
\toprule
Modell & $t_{\text{coll}}$ (Myr) & $D$ & $\frac{\delta\rho}{\rho}$ & $\chi^2$ \\
\midrule
Klassisch & 1.2 & -- & 0.1 & 4.2 \\
MHD & 1.6 & 2.3 & 0.3 & 2.8 \\
WDBT & 1.5 & 2.71 & 0.25 & 1.1 \\
ALMA-Daten & 1.4 $\pm$ 0.3 & 2.7 $\pm$ 0.1 & 0.22 $\pm$ 0.05 & -- \\
\bottomrule
\end{tabular}
\end{table}

Die WDBT zeigt bessere Übereinstimmung ($\chi^2 = 1.1$) mit den ALMA-Beobachtungen als klassische Modelle. Die vorhergesagte fraktale Dimension $D = 2.71$ entspricht innerhalb der Fehlergrenzen den gemessenen $2.7 \pm 0.1$.

\section{Fraktale Strukturbildung}
Die Dichtefluktuationen folgen:

\begin{equation}
P(k) = P_0 k^{-0.29} \quad \text{(entsprechend } D \approx 2.71\text{)}
\end{equation}

Diese Skalierung erklärt sowohl die großskalige Wolkenstruktur als auch die Subfragmentierung in protostellaren Kernen.

\chapter{Fusionsplasmen in der WDBT}
\section{Grundgleichungen der WDBT für Plasmen}

Die Dynamik eines Fusionsplasmas wird in der \gls{wdbt} durch die gekoppelten Gleichungen für die Wirkungsfunktion $S(\vec{r},t)$ und die Teilchendichte $\rho(\vec{r},t)$ beschrieben:

\begin{align}
\frac{\partial S}{\partial t} + \frac{(\nabla S - q\vec{A})^2}{2m} + V + Q + \Phi_{\text{Weber}} &= 0 \label{eq:C1} \\
\frac{\partial \rho}{\partial t} + \nabla \cdot \left(\rho \frac{\nabla S - q\vec{A}}{m}\right) &= 0 \label{eq:C2}
\end{align}

mit dem Quantenpotential:
\begin{equation}
Q = -\frac{\hbar^2}{2m} \frac{\nabla^2 \sqrt{\rho}}{\sqrt{\rho}} \label{eq:Q}
\end{equation}

und dem Weber-Potential für Teilchenwechselwirkungen:
\begin{equation}
\Phi_{\text{Weber}} = \int d^3r' \rho(\vec{r}') V_{\text{Weber}}(|\vec{r}-\vec{r}'|) g(|\vec{r}-\vec{r}'|)
\end{equation}

\subsection{Modifizierte Plasmadynamik}

Die Weber-Kraftdichte im Plasma ergibt sich aus der Integration über Paarkorrelationen:

\begin{equation}
\vec{f}_{\text{Weber}} = n_e n_i \int d^3r \frac{q_1 q_2}{4\pi\epsilon_0 r^2} \left[ \left(1-\frac{v^2}{c^2}\right)\hat{r} + \frac{2(\vec{v}\cdot\hat{r})}{c^2}\vec{v} \right] g(\vec{r})
\end{equation}

Dies führt zu einer modifizierten magnetischen Dynamik:

\begin{equation}
\label{eq:modified_ampere}
\nabla \times \vec{B} = \mu_0 \vec{j} + \frac{\mu_0 e^2 n_e \lambda_c^2}{\epsilon_0} \frac{\partial \vec{j}}{\partial t}
\end{equation}

\subsection{Stabilitätsanalyse für Fusionsplasmen}

Die Dispersionrelation für Plasmawellen unter Berücksichtigung des Quantenpotentials:

\begin{equation}
\label{eq:dispersion}
\omega^2 = \omega_p^2 \left(1 + \frac{\hbar^2 k^2}{4m_e^2 \omega_p^2}\right)
\end{equation}

Die Stabilitätsbedingung für ein zylindrisches Plasma mit Radius $R$:

\begin{equation}
\label{eq:stability}
\frac{d}{dr}\left(r\frac{dQ}{dr}\right) - \frac{m^2}{r}Q + \left(\frac{\omega^2}{v_A^2} - k^2\right)rQ = 0
\end{equation}

mit $v_A = B/\sqrt{\mu_0 \rho_m}$ der Alfvén-Geschwindigkeit.

\subsection{Fraktale Skalierung der Stromdichte}

Die WDBT sagt für Birkeland-Ströme eine charakteristische Skalierung voraus:

\begin{equation}
j(r) = j_0 \left(\frac{r}{r_0}\right)^{D-3} \approx j_0 \left(\frac{r}{r_0}\right)^{-0.29}
\end{equation}

mit der fraktalen Dimension $D = \frac{\ln 20}{\ln(2+\phi)} \approx 2.71$.

\subsection{Energiebilanz im feldlosen Plasma}

Die Energieerhaltung unter Berücksichtigung des Quantenpotentials:

\begin{equation}
\label{eq:energy}
\frac{d}{dt}\left(\frac{3}{2}n_e k_B T_e + \frac{\hbar^2}{8m_e} \frac{(\nabla n_e)^2}{n_e}\right) = P_{\text{ext}} - P_{\text{rad}}
\end{equation}

\chapter{Antriebstechnik}
\section{Elektrisch geladene Druckkammer als gerichteter Plasmaantrieb}
\label{sec:plasma-antrieb}

\subsection{Prinzip und Theorie}
Das vorgeschlagene Antriebssystem nutzt eine \textbf{negativ geladene Druckkammer}, um nach Laserionisation eines kryogenen Treibstoffs (z.B. flüssiger Wasserstoff, LH\textsubscript{2}) Elektronen und Protonen getrennt zu beschleunigen. Der Stromkreis wird durch die Raumschiffhülle geschlossen.

\subsubsection*{Hauptgleichungen}
\begin{itemize}
    \item \textbf{Ladungstrennung} nach Ionisation:
        \begin{equation}
            n_e = n_p = \frac{\rho_{\text{LH}_2} N_A}{M_{\text{H}_2}} \approx 4.23 \times 10^{28}\,\text{m}^{-3} \quad \text{(bei vollständiger Ionisation)}
            \label{eq:dichte}
        \end{equation}
    
    \item \textbf{Abzugsfeldstärke} für Elektronen:
        \begin{equation}
            E = \frac{V_{\text{Kammer}}}{d} \quad \text{(typisch } V_{\text{Kammer}} = -1\,\text{MV}, d = 0.1\,\text{m} \Rightarrow E = 10\,\text{MV/m})
            \label{eq:feld}
        \end{equation}
    
    \item \textbf{Protonenbeschleunigung} (nicht-relativistisch):
        \begin{equation}
            F_p = n_p \cdot e \cdot v_p \times B \quad \text{(Lorentzkraft)}
            \label{eq:lorentz_2}
        \end{equation}
\end{itemize}

\subsection{Kritische Analyse}
\subsubsection*{Vorteile}
\begin{itemize}
    \item \textbf{Präzise Steuerung}: Separierte Kontrolle von Elektronen (elektrostatisch) und Protonen (magnetisch).
    \item \textbf{Energierückgewinnung}: Elektronenstrom könnte genutzt werden (z.B. für Kühlung).
    \item \textbf{Keine mechanische Abnutzung}: Keine beweglichen Teile in der Düse.
\end{itemize}

\subsubsection*{Herausforderungen}
\begin{table}[ht]
    \centering
    \begin{tabular}{ll}
        \toprule
        \textbf{Problem} & \textbf{Lösungsansatz} \\
        \midrule
        Gigavolt-Potentiale & Pulsbetrieb mit $f > 1\,\text{kHz}$ \\
        Hüllenstrom >1\,MA & Supraleitende Beschichtung (YBCO) \\
        Protonenstrahl-Streuung & Quantenpotential $Q$ der WDBT \\
        \bottomrule
    \end{tabular}
    \caption{Technische Herausforderungen und Lösungsvorschläge.}
    \label{tab:probleme}
\end{table}

\subsection{Machbarkeit und Ausblick}
Das System erfordert Fortschritte in:
\begin{enumerate}
    \item \textbf{Hochspannungstechnik}: Vakuum-isolierte Kammerdesigns (Diamant-Wolfram).
    \item \textbf{Supraleitung}: Stabile Supraleiter für Magnetfelder >20\,T.
    \item \textbf{Lasertechnik}: Femtosekundenpulse mit $E > 100\,\text{J}$ bei MHz-Frequenzen.
\end{enumerate}

\begin{equation}
    \text{Fazit: } \boxed{\text{Theoretisch machbar, aber experimentelle Validierung im Labormaßstab nötig.}}
\end{equation}

\section{Kombinierter Antrieb und Strahlungsschutz durch Hüllenstrom}
\label{sec:hullenstrom}

\subsection{Prinzip des Dual-Use-Systems}
Die Raumschiffhülle dient sowohl als \textbf{Stromrückleitung} für den Plasmaantrieb (vgl. Abschnitt~\ref{sec:plasma-antrieb}) als auch als \textbf{aktive Strahlungsabschirmung} durch das induzierte Magnetfeld. 

\subsection{Physikalische Grundlagen}
\subsubsection*{Magnetfeldberechnung (Ampèresches Gesetz)}
Das vom Hüllenstrom $I_H$ erzeugte toroidale Magnetfeld im Innenraum:
\begin{equation}
    B_\phi(r) = \frac{\mu_0 I_H}{2\pi r} \quad \text{(Zylinderkoordinaten)}
    \label{eq:bfeld}
\end{equation}

\subsubsection*{Strahlungsablenkung (Lorentzkraft)}
Geladene kosmische Teilchen (Protonen, $\alpha$-Teilchen) werden abgelenkt:
\begin{equation}
    F_L = q v \times B \quad \Rightarrow \quad r_L = \frac{m v_\perp}{|q| B} \quad \text{(Gyrationsradius)}
    \label{eq:lorentz}
\end{equation}

\subsection{Technische Spezifikation}
\begin{table}[ht]
    \centering
    \begin{tabular}{lc}
        \toprule
        \textbf{Parameter} & \textbf{Wert} \\
        \midrule
        Hüllenstrom $I_H$ & 1\,MA \\
        Hüllenradius $R$ & 5\,m \\
        Magnetfeld $B(R)$ & 0.08\,T (800\,G) \\
        Abschirmwirkung (für 1\,GeV-Protonen) & $r_L \approx 125\,\text{m}$ \\
        \bottomrule
    \end{tabular}
    \caption{Beispielrechnung für ein bemanntes Raumschiff.}
    \label{tab:specs}
\end{table}

\subsection{Kritische Bewertung}
\subsubsection*{Vorteile}
\begin{itemize}
    \item \textbf{Energieeffizienz}: Nutzung des Antriebsstroms für passiven Schutz.
    \item \textbf{Richtungsabhängigkeit}: Maximale Abschirmung entlang der Torusachse.
\end{itemize}

\subsubsection*{Herausforderungen}
\begin{itemize}
    \item \textbf{Supraleiter-Ressourcen}: Für 1\,MA sind \textbf{supraleitende Kabel} nötig (YBCO oder MgB\textsubscript{2}).
    \item \textbf{Neutrale Teilchen}: Unabgelenkte Neutronen erfordern zusätzliche Polymerschichten.
    \item \textbf{Störfelder}: Magnetfeld interferiert mit Bordelektronik ($\mu$-metal-Abschirmung nötig).
\end{itemize}

\subsection{Integrierte Lösung}
Kombination mit dem Plasmaantrieb aus Abschnitt~\ref{sec:plasma-antrieb}:
\begin{enumerate}
    \item Elektronenstrom fließt über supraleitende Hülle zurück.
    \item Induziertes $B$-Feld bildet \textbf{miniaturisiertes Magnetosphärenmodell}.
    \item Zusätzliche Abschirmung durch \textbf{Plasmarückstoß} (sekundäre Wechselwirkungen).
\end{enumerate}

\begin{equation}
    \text{Gesamtschutzwirkung} \approx \exp\left(-\frac{d}{r_L}\right) \quad \text{(Exponentielle Dämpfung)}
\end{equation}

\chapter{Emergenz der Maxwell-Theorie aus der WDBT}
\label{sec:emergence-maxwell}

Die Konsistenz der Weber-De-Broglie-Bohm-Theorie (WDBT) erfordert, dass sie die erfolgreiche Maxwell'sche Elektrodynamik als Grenzfall enthält. Dieser Abschnitt zeigt exakt, wie die Maxwell-Gleichungen sowie die Ladungserhaltung aus den fundamentalen Prinzipien der WDBT emergieren.

\section{Der Reduktionspfad: Vom Nicht-Lokalen zum Lokalen}

Die Emergenz wird durch zwei konsequente Näherungen definiert, welche die nicht-lokale und quantenmechanische Tiefe der WDBT schrittweise reduzieren:

\begin{enumerate}
    \item \textbf{Lokalitäts-Näherung:} Die Paarkorrelationsfunktion $g(|\vec{r} - \vec{r}'|)$ wird durch eine Delta-Funktion approximiert: $g(|\vec{r} - \vec{r}'|) \rightarrow \delta^{(3)}(\vec{r} - \vec{r}')$. Dies schaltet die nicht-lokalen Wechselwirkungsterme ab und reduziert die Kraftdichte auf eine lokale Beschreibung.
    \item \textbf{Klassischer Limes:} Das Quantenpotential $Q = -\frac{\hbar^2}{2m} \frac{\nabla^2 \sqrt{\rho}}{\sqrt{\rho}}$ wird vernachlässigt ($Q \rightarrow 0$). Dies entspricht dem Übergang zur klassischen Physik.
\end{enumerate}

Unter diesen Näherungen müssen sich die Strukturen der Maxwell-Theorie zwangsläufig aus den Gleichungen der WDBT ergeben.

\subsection{Emergenz der Kontinuitätsgleichung}

Die Kontinuitätsgleichung, Ausdruck der Ladungserhaltung, ist fundamental in der Struktur der WDBT verankert. Ausgangspunkt ist die Erhaltung der Wahrscheinlichkeitsdichte (Gl. B.2):

\begin{equation}
    \frac{\partial \rho}{\partial t} + \nabla \cdot \left( \rho \frac{\nabla S - q\vec{A}}{m} \right) = 0
\end{equation}

Definiert man die hydrodynamische Geschwindigkeit $\vec{v} = \frac{\nabla S - q\vec{A}}{m}$, so erkennt man unmittelbar die Standardform einer Kontinuitätsgleichung:

\begin{equation}
    \frac{\partial \rho}{\partial t} + \nabla \cdot (\rho \vec{v}) = 0
\end{equation}

Multiplikation mit der Ladung $q$ liefert direkt die elektrische Kontinuitätsgleichung, wobei $\rho_{\text{ladung}} = q\rho$ die Ladungsdichte und $\vec{j} = \rho_{\text{ladung}} \vec{v}$ die Stromdichte ist:

\begin{equation}
    \frac{\partial \rho_{\text{ladung}}}{\partial t} + \nabla \cdot \vec{j} = 0
\end{equation}

Diese Herleitung ist exakt und benötigt keine Näherungen. Die \textbf{Erhaltung der Ladung} ist somit ein fundamentaleres Prinzip in der WDBT als in der Maxwell-Theorie, da sie direkt aus der Struktur der quantenmechanischen Wahrscheinlichkeitserhaltung folgt.

\subsection{Emergenz der Feldgleichungen}

In der WDBT sind elektromagnetische Felder ($\vec{E}$, $\vec{B}$) keine fundamentalen Entitäten, sondern \textit{effektive Hilfsgrößen}, die aus der gemittelten Weber-Wechselwirkung abgeleitet werden.

\subsubsection{Skalarpotential und Gauß'sches Gesetz}

Der Coulomb-Anteil der Weber-Kraftdichte (Gl. 2.7) führt im Lokalitätslimes ($g(\vec{r}) \rightarrow \delta^{(3)}(\vec{r})$)
 auf die Poisson-Gleichung:

\begin{equation}
    \nabla^2 \phi = -\frac{\rho}{\epsilon_0}
\end{equation}

Diese Gleichung \textit{definiert} das elektrostatische Potential $\phi$. Wendet man den Nabla-Operator auf beide Seiten an, erhält man unmittelbar das Gauß'sche Gesetz für das elektrische Feld $\vec{E} = -\nabla \phi$:

\begin{equation}
    \nabla \cdot \vec{E} = \frac{\rho}{\epsilon_0}
\end{equation}

\subsubsection{Vektorpotential und magnetische Gesetze}

Die geschwindigkeitsabhängigen Terme der Weber-Kraftdichte führen im selben Limes auf Ausdrücke, die proportional zu $\vec{v} \times (\nabla \times \vec{A})$ sind. Dies identifiziert die magnetische Flussdichte $\vec{B}$ mit der Rotation eines Vektorpotentials:

\begin{equation}
    \vec{B} = \nabla \times \vec{A}
\end{equation}

Diese Definition impliziert sofort die Quellenfreiheit des magnetischen Feldes, die zweite homogene Maxwell-Gleichung:

\begin{equation}
    \nabla \cdot \vec{B} = 0
\end{equation}

\subsubsection{Faraday'sches Induktionsgesetz}

Das Induktionsgesetz ist eine direkte mathematische Konsequenz der Definition der Felder aus den Potentialen. Aus $\vec{E} = -\nabla \phi - \frac{\partial \vec{A}}{\partial t}$ folgt durch Bildung der Rotation:

\begin{equation}
    \nabla \times \vec{E} = \nabla \times (-\nabla \phi - \frac{\partial \vec{A}}{\partial t}) = - \underbrace{\nabla \times (\nabla \phi)}_{=0} - \frac{\partial}{\partial t} (\underbrace{\nabla \times \vec{A}}_{=\vec{B}})
\end{equation}

Woraus sich das Faraday'sche Gesetz ergibt:

\begin{equation}
    \nabla \times \vec{E} = -\frac{\partial \vec{B}}{\partial t}
\end{equation}

\subsubsection{Ampère'sches Durchflutungsgesetz und seine Erweiterung}

Die volle Kraft der WDBT zeigt sich in der Herleitung des Durchflutungsgesetzes. Aus der Analyse der Kraftdichte emergiert nicht das klassische Ampère-Maxwell-Gesetz, sondern eine erweiterte, nicht-lokale Version (Gl. 2.8):

\begin{equation}
    \nabla \times \vec{B} = \mu_0 \vec{j} + \frac{\mu_0 e^2 n_e \lambda_c^2}{\epsilon_0} \frac{\partial \vec{j}}{\partial t}
\end{equation}

Im \textit{Maxwell-Limes} ($\lambda_c \rightarrow 0$, d.h. Vernachlässigung der Nicht-Lokalität) verschwindet der Zusatzterm und wir erhalten das ursprüngliche Ampère'sche Gesetz für stationäre Ströme:

\begin{equation}
    \lim_{\lambda_c \to 0} \left( \nabla \times \vec{B} \right) = \mu_0 \vec{j}
\end{equation}

Um die vollständige Maxwell-Theorie zu erhalten, muss der Verschiebungsstrom $\mu_0 \epsilon_0 \frac{\partial \vec{E}}{\partial t}$ hinzugefügt werden. Dieser Schritt ist in der WDBT jedoch \textit{ad hoc}. Die Gleichung (10) stellt stattdessen die \textbf{fundamentalere Form} dar, da sie den nicht-lokalen Ursprung der Feldoffsets direkt beschreibt. Der Maxwell'sche Term emergiert seinerseits erst aus einer weiteren Vereinfachung, nämlich der Annahme einer linearen Beziehung zwischen Strom und Feld in einfachen Medien.

\subsection{Zusammenfassung: Die Maxwell-Theorie als effektive Beschreibung}

Die Ableitung zeigt, dass die gesamte Maxwell'sche Elektrodynamik konsistent aus der WDBT emergiert, sobald nicht-lokale und quantenmechanische Effekte vernachlässigt werden. Die WDBT erklärt damit nicht nur die Existenz der Maxwell-Gleichungen, sondern auch ihre Grenzen:

\begin{itemize}
    \item Die \textbf{Kontinuitätsgleichung} ist ein fundamentaleres Prinzip.
    \item Die \textbf{homogenen Gleichungen} ($\nabla \cdot \vec{B} = 0$, $\nabla \times \vec{E} = -\partial_t \vec{B}$) sind direkte Konsequenzen der Potentialdefinition.
    \item Die \textbf{inhomogenen Gleichungen} ($\nabla \cdot \vec{E} = \rho/\epsilon_0$, $\nabla \times \vec{B} = \ldots$) emergieren aus der gemittelten Weber-Kraftdichte.
    \item Das Ampère'sche Gesetz wird durch einen \textbf{nicht-lokalen Korrekturterm} erweitert, der unter Laborbedingungen ($\lambda_c \rightarrow 0$) verschwindet, in dichten Plasmen oder auf kleinen Skalen jedoch dominant wird.
\end{itemize}

Die Maxwell-Theorie ist somit die \textit{effektive Feldtheorie} der WDBT für den Grenzfall langsamer, lokaler Phänomene. Die WDBT selbst bietet den fundamentaleren, einheitlichen Rahmen, der Mikro- und Makrophysik verbindet.

