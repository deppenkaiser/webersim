\chapter{Begründung der Sternentstehung in der WDBT}
\label{app:sternentstehung}
Dieser Anhang beschreibt die Physik der Sternenentstehung.

\section{Grundgleichungen der Plasmadynamik}
Die Dynamik eines Plasmas in der Weber-De-Broglie-Bohm-Theorie wird durch folgendes gekoppeltes System beschrieben:

\begin{equation}
\frac{\partial S}{\partial t} + \frac{(\nabla S - q\vec{A})^2}{2m} + V + Q + \Phi_{\text{Weber}} = 0
\end{equation}

\begin{equation}
\frac{\partial \rho}{\partial t} + \nabla \cdot \left(\rho \frac{\nabla S - q\vec{A}}{m}\right) = 0
\end{equation}

wobei:
\begin{itemize}
\item $S(\vec{r},t)$ die Wirkungsfunktion
\item $\rho(\vec{r},t)$ die Teilchendichte
\item $Q = -\frac{\hbar^2}{2m}\frac{\nabla^2 \sqrt{\rho}}{\sqrt{\rho}}$ das Quantenpotential
\item $\Phi_{\text{Weber}}$ das Weber-Potential
\end{itemize}

\section{Weber-Kraft und Gravitation}
Die kombinierte Weber-Kraft für Gravitation und Elektrodynamik:

\begin{equation}
\vec{F}_{12} = \left[\frac{Gm_1m_2 - \frac{q_1q_2}{4\pi\epsilon_0}}{r^2}\right]\left(1 - \frac{v^2}{c^2} + \frac{2r\ddot{r}}{c^2}\right)\hat{r}
\end{equation}

Für ein Plasma mit Korrelationsfunktion $g(\vec{r})$:

\begin{equation}
\vec{f} = \int d^3r'\, \rho(\vec{r})\rho(\vec{r}')\vec{F}_{\text{Weber}}(|\vec{r}-\vec{r}'|)g(\vec{r}-\vec{r}')
\end{equation}

\section{Stabilitätsanalyse einer kollabierenden Wolke}
Für eine homogene sphärische Wolke mit Radius $R(t)$:

\begin{equation}
\ddot{R} = -\frac{GM}{R^2} + \frac{9\hbar^2}{4m^2R^3} - \frac{3e^2N^{2/3}}{20\pi\epsilon_0 m R^2}
\end{equation}

\subsection{Jeans-Kriterium}
Kollapsbedingung:

\begin{equation}
M > \frac{9\hbar^2}{4Gm^2R} + \frac{3e^2N^{2/3}}{20\pi\epsilon_0 Gm}
\end{equation}

\section{Numerische Lösung}
Mit $R(t) = R_0 f(t)$:

\begin{equation}
\frac{d^2f}{dt^2} = -\frac{GM}{R_0^3 f^2} + \frac{9\hbar^2}{4m^2R_0^4 f^3}
\end{equation}

Approximative Lösung für $f \ll 1$:

\begin{equation}
f(t) \approx R_0\left(1 - \frac{t}{t_{\text{coll}}}\right)^{2/3}, \quad t_{\text{coll}} = \sqrt{\frac{R_0^3}{GM}}
\end{equation}

\section{Experimenteller Vergleich}

\begin{table}[h]
\centering
\caption{Vergleich der Vorhersagen mit Beobachtungsdaten}
\begin{tabular}{lcccc}
\toprule
Modell & $t_{\text{coll}}$ (Myr) & $D$ & $\frac{\delta\rho}{\rho}$ & $\chi^2$ \\
\midrule
Klassisch & 1.2 & -- & 0.1 & 4.2 \\
MHD & 1.6 & 2.3 & 0.3 & 2.8 \\
WDBT & 1.5 & 2.71 & 0.25 & 1.1 \\
ALMA-Daten & 1.4 $\pm$ 0.3 & 2.7 $\pm$ 0.1 & 0.22 $\pm$ 0.05 & -- \\
\bottomrule
\end{tabular}
\end{table}

Die WDBT zeigt bessere Übereinstimmung ($\chi^2 = 1.1$) mit den ALMA-Beobachtungen als klassische Modelle. Die vorhergesagte fraktale Dimension $D = 2.71$ entspricht innerhalb der Fehlergrenzen den gemessenen $2.7 \pm 0.1$.

\section{Fraktale Strukturbildung}
Die Dichtefluktuationen folgen:

\begin{equation}
P(k) = P_0 k^{-0.29} \quad \text{(entsprechend } D \approx 2.71\text{)}
\end{equation}

Diese Skalierung erklärt sowohl die großskalige Wolkenstruktur als auch die Subfragmentierung in protostellaren Kernen.

\chapter{Fusionsplasmen in der WDBT}
\section{Grundgleichungen der WDBT für Plasmen}

Die Dynamik eines Fusionsplasmas wird in der \gls{wdbt} durch die gekoppelten Gleichungen für die Wirkungsfunktion $S(\vec{r},t)$ und die Teilchendichte $\rho(\vec{r},t)$ beschrieben:

\begin{align}
\frac{\partial S}{\partial t} + \frac{(\nabla S - q\vec{A})^2}{2m} + V + Q + \Phi_{\text{Weber}} &= 0 \label{eq:C1} \\
\frac{\partial \rho}{\partial t} + \nabla \cdot \left(\rho \frac{\nabla S - q\vec{A}}{m}\right) &= 0 \label{eq:C2}
\end{align}

mit dem Quantenpotential:
\begin{equation}
Q = -\frac{\hbar^2}{2m} \frac{\nabla^2 \sqrt{\rho}}{\sqrt{\rho}} \label{eq:Q}
\end{equation}

und dem Weber-Potential für Teilchenwechselwirkungen:
\begin{equation}
\Phi_{\text{Weber}} = \int d^3r' \rho(\vec{r}') V_{\text{Weber}}(|\vec{r}-\vec{r}'|) g(|\vec{r}-\vec{r}'|)
\end{equation}

\subsection{Modifizierte Plasmadynamik}

Die Weber-Kraftdichte im Plasma ergibt sich aus der Integration über Paarkorrelationen:

\begin{equation}
\vec{f}_{\text{Weber}} = n_e n_i \int d^3r \frac{q_1 q_2}{4\pi\epsilon_0 r^2} \left[ \left(1-\frac{v^2}{c^2}\right)\hat{r} + \frac{2(\vec{v}\cdot\hat{r})}{c^2}\vec{v} \right] g(\vec{r})
\end{equation}

Dies führt zu einer modifizierten magnetischen Dynamik:

\begin{equation}
\label{eq:modified_ampere}
\nabla \times \vec{B} = \mu_0 \vec{j} + \frac{\mu_0 e^2 n_e \lambda_c^2}{\epsilon_0} \frac{\partial \vec{j}}{\partial t}
\end{equation}

\subsection{Stabilitätsanalyse für Fusionsplasmen}

Die Dispersionrelation für Plasmawellen unter Berücksichtigung des Quantenpotentials:

\begin{equation}
\label{eq:dispersion}
\omega^2 = \omega_p^2 \left(1 + \frac{\hbar^2 k^2}{4m_e^2 \omega_p^2}\right)
\end{equation}

Die Stabilitätsbedingung für ein zylindrisches Plasma mit Radius $R$:

\begin{equation}
\label{eq:stability}
\frac{d}{dr}\left(r\frac{dQ}{dr}\right) - \frac{m^2}{r}Q + \left(\frac{\omega^2}{v_A^2} - k^2\right)rQ = 0
\end{equation}

mit $v_A = B/\sqrt{\mu_0 \rho_m}$ der Alfvén-Geschwindigkeit.

\subsection{Fraktale Skalierung der Stromdichte}

Die WDBT sagt für Birkeland-Ströme eine charakteristische Skalierung voraus:

\begin{equation}
j(r) = j_0 \left(\frac{r}{r_0}\right)^{D-3} \approx j_0 \left(\frac{r}{r_0}\right)^{-0.29}
\end{equation}

mit der fraktalen Dimension $D = \frac{\ln 20}{\ln(2+\phi)} \approx 2.71$.

\subsection{Energiebilanz im feldlosen Plasma}

Die Energieerhaltung unter Berücksichtigung des Quantenpotentials:

\begin{equation}
\label{eq:energy}
\frac{d}{dt}\left(\frac{3}{2}n_e k_B T_e + \frac{\hbar^2}{8m_e} \frac{(\nabla n_e)^2}{n_e}\right) = P_{\text{ext}} - P_{\text{rad}}
\end{equation}
