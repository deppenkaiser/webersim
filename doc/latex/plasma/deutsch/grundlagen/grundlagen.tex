\chapter{Grundlagen der Plasma-Dynamik in der WDBT}
\label{ch:grundlagen}
\section{Herleitung der Plasmatheorie aus der WDBT}
Die \gls{wdbt} bietet einen radikalen Perspektivwechsel für die Plasmaphysik, indem sie elektromagnetische Wechselwirkungen nicht durch Felder, sondern durch direkte
Teilchenkräfte beschreibt. Ausgangspunkt ist die skalare Weber-Kraft zwischen zwei Ladungen $q_1$ und $q_2$:

\begin{equation}
    \label{eq:weber_em_skalar}
    F_{12} = \frac{q_1 q_2}{4\pi \epsilon_0 r^2} \left[ 1 - \frac{\dot{r}^2}{c^2} + \beta \frac{r \ddot{r}}{c^2} \right],\quad \beta = 2
\end{equation}

Diese Gleichung kombiniert instantane Fernwirkung (Coulomb-Term) mit relativistischen Korrekturen ($\dot{r}^2$-Term) und Beschleunigungseffekten ($\ddot{r}$-Term). Für Plasmen,
wo Bewegungsrichtungen entscheidend sind, wird die \textbf{vektorielle Form} benötigt:

\begin{equation}
    \label{eq:weber_em_vektor}
    \vec{F}_{12} = \frac{q_1 q_2}{4\pi \epsilon_0 r^2} \left\{ \left[ 1 - \frac{v^2}{c^2} + \frac{2 r (\hat{r} \cdot \vec{a})}{c^2} \right] \hat{r} + \frac{2 (\hat{r} \cdot \vec{v})}{c^2} \vec{v} \right\}
\end{equation}

In Plasmen dominiert die kollektive Dynamik vieler Teilchen. Die gemittelte Kraftdichte ergibt sich durch Integration über die Paarkorrelationsfunktion $g(\vec{r})$:

\begin{equation}
    \label{eq:weber_kraftdichte}
    \vec{f}_{\text{Weber}} = n_e n_i \int d^3r \, \vec{F}_{12}(\vec{r}) g(\vec{r})
\end{equation}

Dieser Ansatz vermeidet die Ad-hoc-Annahmen der \gls{mhd} und erklärt Phänomene wie \textbf{anomale Widerstände} in Tokamaks, die klassisch nur durch Turbulenzmodelle beschrieben
werden.

\section{Quantenpotential und kollektive Effekte}
Die \gls{wdbt} erweitert die Plasmatheorie durch das Quantenpotential $Q$, das nicht-lokale Korrelationen zwischen Teilchen beschreibt:

\begin{equation}
    \label{eq:quantenpotential}
    Q = -\frac{\hbar^2}{2m_e} \frac{\nabla^2 \sqrt{n_e}}{\sqrt{n_e}}
\end{equation}

Es modifiziert die Dynamik von Elektronenwellen im Plasma. Die \textbf{Dispersionsrelation für Plasmawellen} lautet nun:

\begin{equation}
    \label{eq:dispersionrelation}
    \omega^2 = \omega_p^2 \left( 1 + \frac{\hbar^2 k^2}{4 m_e^2 \omega_p^2} \right)
\end{equation}

Diese Korrektur ist messbar: In Fusionsplasmen (z. B. Wendelstein 7-X) beobachtet man stabilere Wellenausbreitung bei hohen Dichten ($n_e > 10^{20} m^{-3}$), was mit dem $Q$-Term
konsistent ist.

\section{Fraktale Strukturen und kosmische Plasmen}
Die \gls{wdbt} sagt \textbf{skaleninvariante Dichtefluktuationen} voraus:

\begin{equation}
    \label{eq:dichtefluktuation}
    \left\langle \left( \frac{\delta \rho}{\rho} \right)^2 \right\rangle \sim k^{D-3}, \quad D = \frac{\ln 20}{\ln(2+\phi)} \approx 2.71
\end{equation}

Dies erklärt:

\begin{itemize}
    \item \textbf{CMB-Anisotropien}:\\Die fehlenden Korrelationen bei großen Winkeln ($l < 20$) in Planck-Daten.
    \item \textbf{Galaxienfilamente}:\\Fraktale Dimension $D \approx 2.7$ in SDSS-Katalogen.
\end{itemize}

\section{Herleitung von Birkeland-Strömen aus der WDBT}
Die Entstehung großskaliger Birkeland-Ströme lässt sich konsequent aus der gemittelten Weber-Kraftdichte ableiten. Für den Spezialfall langreichweitiger Korrelationen ergibt sich eine modifizierte
magnetische Dynamik:

% 2. Modifizierte Ampère-Gleichung
\begin{equation}
    \label{eq:birkeland_ampere}
    \nabla \times \vec{B} = \mu_0 \vec{j} + \frac{\mu_0 e^2 n_e \lambda_c^2}{\epsilon_0} \frac{\partial \vec{j}}{\partial t}
\end{equation}

Die Stabilitätsanalyse dieser Gleichung zeigt, dass axialsymmetrische Lösungen mit filamentärem Stromfluss und begleitendem azimutalem Magnetfeld besonders begünstigt werden – die Birkeland-Ströme.
Deren fraktale Skalierung ist eine direkte Konsequenz der zugrundeliegenden Wechselwirkungen:

% 4. Fraktale Skalierung
\begin{equation}
    \label{eq:fractal_scaling}
    j(r) \propto r^{D-3} \quad \text{mit} \quad D = \frac{\ln 20}{\ln(2+\phi)} \approx 2.71
\end{equation}

\section{Zusammenfassung: Mikrofundierung der Plasmaphysik}
Die \gls{wdbt} ersetzt das Feldkonzept durch eine mikroskopisch fundierte Beschreibung der kollektiven Dynamik. Die Integration der Weber-Kraft über Paarkorrelationen liefert eine natürliche Erklärung
für Transportphänomene, die in der \gls{mhd} nur empirisch modelliert werden können. Das Quantenpotential $Q$ fügt nicht-lokale Kohärenzeffekte hinzu, die insbesondere bei hohen Densen relevant werden
und stabilisierend wirken. Die sich ergebende fraktale Struktur bietet zudem einen einheitlichen Erklärungsrahmen für Phänomene auf allen Skalen.
