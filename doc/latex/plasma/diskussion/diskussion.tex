\chapter{Diskussion}

\section{Photonen als Solitonen der Weber-Kraft}
Die \gls{wdbt} interpretiert Photonen nicht als Eichbosonen, sondern als nicht-lokale Solitonen kollektiver Ladungsfluktuationen. Die zugehörige Lagrange-Dichte kombiniert Weber-Wechselwirkung,
Quantenpotential und Polarisationsfeld:

Die Lagrange-Dichte $\mathcal{L}_{\text{WED}}$ kombiniert:  
\begin{equation}
\mathcal{L} = \int d^3r'\, \rho(\vec{r},t) \rho(\vec{r}',t) V_{\text{Weber}}(|\vec{r}-\vec{r}'|) + \frac{\hbar^2}{8m_e} \frac{(\nabla \rho)^2}{\rho} - e \vec{P} \cdot \vec{E}_{\text{eff}},
\end{equation}
mit:
\begin{itemize}
    \item $V_{\text{Weber}} = \dfrac{q_1 q_2}{4\pi \epsilon_0 r} \left(1 - \dfrac{\dot{r}^2}{2c^2} + 2 \dfrac{r \ddot{r}}{c^2}\right)$,
    \item $\rho(\vec{r},t)$: Ladungsdichte des Solitons,
    \item $\vec{P}$: Polarisationsfeld.
\end{itemize}

Variation von $\mathcal{L}$ liefert:  

\begin{equation}
i\hbar \frac{\partial \psi}{\partial t} = -\frac{\hbar^2}{2m_e} \nabla^2 \psi + \left[ V_{\text{Weber}} + Q \right] \psi, \quad \psi = \sqrt{\rho} e^{iS/\hbar}.
\end{equation}

Die Solitonlösung ist:
\begin{equation}
\rho(\vec{r},t) = \rho_0 \, \text{sech}^2\left(\frac{z - ct}{\lambda}\right), \quad \lambda = \sqrt{\frac{\hbar^2}{m_e V_{\text{Weber}}}}.
\end{equation}

\section{Emergenz der QED}
Die \gls{qed} emergiert als \textbf{effektive Theorie} bei $\mathbf{k \ll m_e c/\hbar}$:

\begin{table}[ht]
\centering
\begin{tabular}{ll}
\toprule
\textbf{QED-Konzept} & \textbf{WDBT-Äquivalent} \\
\midrule
Photonen & Ladungssolitonen \\
Virtuelle Photonen & Instanton-ähnliche Weber-Konfigurationen \\
$g-2$ des Elektrons & Weber-Kraft-Korrekturen \\
\bottomrule
\end{tabular}
\caption{Korrespondenz zwischen QED und WDBT}
\end{table}

\subsubsection{Experimentelle Konsequenzen:}

\begin{itemize}
    \item \textbf{Anomale Dispersion in Plasmen}:  
    \begin{equation}
    \frac{\Delta c}{c} \sim \alpha \left(\frac{\omega_p}{\omega}\right)^2.
    \end{equation}
    \item \textbf{Modifizierte Lamb-Shift}:
    \begin{equation}
        \label{eq:lamb_shift}
        \Delta E_{\text{Lamb}}^{\text{WED}} = \Delta E_{\text{QED}} + \frac{e^2 \hbar}{4\pi \epsilon_0 m_e^2 c^3} \langle \ddot{r} \rangle.
    \end{equation}
\end{itemize}

\section{Experimentelle Validierung der WDBT}
\label{sec:experimentelle_konsequenzen}

\begin{enumerate}
    \item \textbf{Plasmaphysikalische Tests:} Stabilisierung von Fusionsplasmen durch $Q$ (Gl. \refeq{eq:quantenpotential}), nicht-lokaler Transport in Tokamaks.
    \item \textbf{Atomphysik:} Modifizierter Lamb-Shift (Gl. \refeq{eq:lamb_shift}), Photonen als Solitonen.
    \item \textbf{Kosmologie:} Fraktale \gls{cmb}-Anisotropien (Gl. \refeq{eq:dichtefluktuation}), Galaxienrotation ohne dunkle Materie.
\end{enumerate}

\section{Plasma-Kosmologie als emergentes Phänomen}
Die \gls{wdbt} bietet einen fundamentalen Rahmen zur Beschreibung nicht-lokaler Wechselwirkungen in Plasmen. Die mathematische Struktur legt nahe, dass sich großskalige Phänomene der Plasma-Kosmologie
(Birkeland-Ströme, galaktische Filamente) natürlich ableiten lassen. Die fraktale Skalierung (Gl. \refeq{eq:dichtefluktuation}) erklärt die beobachtete Hierarchie kosmischer Strukturen ohne
Zusatzannahmen wie dunkle Materie oder Inflation.
