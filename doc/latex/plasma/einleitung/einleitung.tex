\chapter{Einführung}
\section{Plasmen als Schlüssel zu einer neuen Physik}
Seit über einem Jahrhundert dominieren Feldtheorien das Denken – von den Maxwell-Gleichungen bis zur \gls{qed}. Doch gerade dort, wo diese Theorien an ihre Grenzen stoßen, in der
Welt der Plasmen, offenbart sich eine tiefere Wahrheit: \textbf{Die Natur kennt keine Felder}. Was wir als elektromagnetische Wechselwirkungen interpretieren, ist in Wirklichkeit ein
komplexes Geflecht direkter, nicht-lokaler Kräfte zwischen Teilchen – eine Erkenntnis, die bereits in der \gls{wed} angelegt ist und durch die \gls{dbt} ihre volle
Bedeutung erlangt.

\section{Das kosmische Plasma: Eine Herausforderung für die Standardmodelle}
Im großen Maßstab des Universums zeigt sich das Versagen der Feldtheorien besonders deutlich. Die kosmische \gls{cmb}, oft als Beweis für den Urknall gefeiert, könnte
ebenso gut das thermische Gleichgewicht eines unendlichen, statischen Plasmauniversums beschreiben. Die Rotverschiebung ferner Galaxien, die heute als Indiz für die Expansion des
Raumes gedeutet wird, lässt sich alternativ durch Energieverluste des Lichts in intergalaktischen Plasmen erklären – ein Prozess, den die \gls{wed} präziser beschreibt
als die \gls{art}.

Die rätselhaften Rotationskurven der Galaxien, die zur Erfindung der dunklen Materie führten, finden in der Plasma-Kosmologie eine natürliche Erklärung: Elektromagnetische Kräfte,
modifiziert durch die Geschwindigkeitsabhängigkeit der Weber-Wechselwirkung, können die beobachteten Geschwindigkeitsprofile erzeugen, ohne auf unsichtbare Teilchen zurückgreifen
zu müssen. Die filamentären Strukturen des kosmischen Netzes, die sich über Hunderte von Millionen Lichtjahren erstrecken, ähneln verblüffend den Mustern, die in
Plasmadynamik-Experimenten auf Laborskala entstehen – ein Hinweis darauf, dass das Universum in seinem Wesen ein elektrisches Phänomen ist.

\subsection{Sternentstehung und Plasmadynamik}
Auch die Geburt der Sterne wirft Fragen auf, die das Feldparadigma nicht befriedigend beantworten kann. Wie können interstellare Wolken aus diffusem Plasma unter ihrer eigenen
Gravitation kollabieren, wenn die elektromagnetischen Abstoßungskräfte um Größenordnungen stärker sind? Die Standardtheorie greift hier zu ad-hoc-Annahmen über
\enquote{magnetische Unterstützung} oder \enquote{Turbulenzdämpfung}. Die \gls{wdbt} hingegen bietet eine elegante Lösung: Das Quantenpotential der \gls{dbt} wirkt als nicht-lokale,
stabilisierende Kraft, die den Kollaps trotz der elektromagnetischen Barrieren ermöglicht. Gleichzeitig erklärt die Weber-Gravitation mit ihrer geschwindigkeitsabhängigen Komponente,
warum protoplanetare Scheiben rotationsstabil bleiben, ohne dass dunkle Materie als \enquote{Klebstoff} benötigt wird.

\subsection{Kernfusion: Vom ITER zum feldlosen Plasma}
Auf der irdischen Skala zeigt sich das Potential der neuen Sichtweise vielleicht am deutlichsten in der Fusionsforschung. Seit Jahrzehnten kämpfen Projekte wie ITER mit den
Unwägbarkeiten der Plasmaturbulenz – einem Problem, das im Rahmen der \gls{mhd} unlösbar erscheint. Doch was, wenn die Turbulenz gar kein chaotisches Phänomen ist,
sondern die Manifestation einer tieferen, nicht-lokalen Ordnung?

Die \gls{wdbt} legt nahe, dass Plasmen in Fusionsreaktoren nicht durch äußere Magnetfelder kontrolliert werden müssen, sondern sich selbst organisieren können – gesteuert durch
das Quantenpotential und die direkten Teilchenwechselwirkungen der \gls{wed}. Es gibt Hinweise dafür, dass Plasmen in dieser Beschreibung stabilere Konfigurationen
einnehmen, als die Feldtheorie vorhersagt. Sollte sich dies bestätigen, könnte es den Weg zu kompakteren, effizienteren Fusionsreaktoren ebnen – eine Revolution der Energiegewinnung.

\subsection{Die Anwendungen: Von der Medizin zur Raumfahrt}
Die Konsequenzen dieser neuen Physik reichen weit über die Grundlagenforschung hinaus. In der Plasmamedizin, wo kalte Plasmen zur Wundheilung eingesetzt werden, könnte die
\gls{wed} erklären, warum bestimmte Plasma-Konfigurationen biologisch wirksamer sind als andere – nicht wegen der Feldstärke, sondern aufgrund der spezifischen,
nicht-lokalen Wechselwirkung mit Gewebemolekülen.

In der Raumfahrtantriebstechnik zeigen Plasmantriebe wie der VASIMR bereits heute, dass hohe spezifische Impulse möglich sind – doch ihre Effizienz bleibt hinter den theoretischen
Grenzen zurück. Die WDBT bietet hier einen neuen Ansatz: Wenn die Strahlbeschleunigung nicht durch Felder, sondern durch direkt wirkende Weber-Kräfte erfolgt, könnten völlig neue
Antriebskonzepte entstehen, die das Zeitalter der interplanetaren Raumfahrt einläuten.

\section{Eine neue Ära der Physik}
Dieses Buch wird zeigen, dass die Vereinigung von \gls{wed}, \gls{dbt} und Plasmaphysik mehr ist als eine akademische Übung – es ist der Schlüssel zu
einem neuen Verständnis des Universums. Von den größten kosmischen Strukturen bis hin zur Kontrolle von Fusionsplasmen eröffnet sich eine Welt jenseits der Quantenfelder, in der
die Natur nicht durch abstrakte Feldgleichungen, sondern durch reale, messbare Wechselwirkungen beschrieben wird.

Die kommenden Kapitel werden diese Vision mit mathematischer Strenge und experimentellen Belegen untermauern. Die Reise beginnt mit den Grundlagen – einer feldlosen Beschreibung
der Plasmadynamik, die zeigt, warum die \gls{wdbt} nicht nur eine Alternative, sondern die logisch konsistentere Theorie ist.
