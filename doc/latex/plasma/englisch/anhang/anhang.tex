\chapter{Justification of Star Formation in WDBT}
\label{app:sternentstehung}

\section{Basic Equations of Plasma Dynamics}
The dynamics of a plasma in the Weber-De Broglie-Bohm Theory (WDBT) is described by the following coupled system:

\begin{equation}
\frac{\partial S}{\partial t} + \frac{(\nabla S - q\vec{A})^2}{2m} + V + Q + \Phi_{\text{Weber}} = 0
\end{equation}

\begin{equation}
\frac{\partial \rho}{\partial t} + \nabla \cdot \left(\rho \frac{\nabla S - q\vec{A}}{m}\right) = 0
\end{equation}

where:
\begin{itemize}
\item $S(\vec{r},t)$ is the action function
\item $\rho(\vec{r},t)$ is the particle density
\item $Q = -\frac{\hbar^2}{2m}\frac{\nabla^2 \sqrt{\rho}}{\sqrt{\rho}}$ is the quantum potential
\item $\Phi_{\text{Weber}}$ is the Weber potential
\end{itemize}

\section{Weber Force and Gravitation}
The combined Weber force for gravitation and electrodynamics:

\begin{equation}
    \label{eq:weber_em_und_g}
    \vec{F}_{12} = \left[ \frac{G m_1 m_2}{r^2} \left( 1 - \frac{\alpha_g v^2}{c^2} + \frac{\beta_g r \ddot{r}}{c^2} \right) - \frac{q_1 q_2}{4\pi\epsilon_0 r^2} \left( 1 - \frac{\alpha_{em} v^2}{c^2} + \frac{\beta_{em} r \ddot{r}}{c^2} \right) \right] \hat{r}
\end{equation}

\section{Stability Analysis of a Collapsing Cloud}
For a homogeneous spherical cloud with radius $R(t)$:

\begin{equation}
    \label{eq:bewegungsgleichung}
    \ddot{R} = -\frac{G M}{R^2} + \frac{9 \hbar^2}{4 m_e^2 R^3} - \frac{3}{16\pi} \frac{e^2 N}{\epsilon_0 m_e R^2}
    \quad \text{with} \quad M = N m_p
\end{equation}

\subsection{Jeans Criterion}
Collapse condition:

\begin{equation}
    M > \frac{9 \hbar^2}{4 G m_e^2 R} + \frac{3}{16\pi} \frac{e^2 N}{\epsilon_0 G m_e}
    \quad \text{with} \quad M = N m_p
\end{equation}

Here, $N$ is the total number of electron-proton pairs in the system, $m_e$ is the electron mass, and $m_p$ is the proton mass. The total mass of the cloud is $M = N m_p$, neglecting the smaller electron mass. The dominant repulsion term primarily results from the Coulomb barrier of the electrons, enhanced by their quantum pressure.

\subsection{Numerical Solution}
Analyzing the collapse dynamics requires the numerical integration of the equation of motion (Eq. \ref{eq:bewegungsgleichung}). Using the substitution approach $R(t) = R_0 f(t)$ yields the following second-order initial value problem:

\begin{equation}
\label{eq:corrected_dimensionless}
\frac{d^{2}f}{dt^{2}} = -\frac{G M}{R_{0}^{3} f^{2}} + \frac{9 \hbar^{2}}{4 m_e^{2} R_{0}^{4} f^{3}} - \frac{3}{16\pi} \frac{e^{2} N}{\epsilon_{0} m_e R_{0}^{3} f^{2}}
\end{equation}

\noindent with the initial conditions $f(0) = 1$ and $\frac{df}{dt}(0) = 0$.

The characteristic time scale of the problem is defined by the free-fall time neglecting the other terms:
\[
\tau_{\text{ff}} = \sqrt{\frac{R_0^3}{G M}}
\]

Equation~\ref{eq:corrected_dimensionless} is integrated numerically to determine the time evolution $f(t)$ and the collapse time $t_{\text{coll}}$, which is defined as the time when $f(t) \rightarrow 0$.

The numerical integration of the equation of motion (Eq. A.5) was performed for an astrophysically relevant parameter set ($M = 10^3 \, M_\odot$, $R_0 = 1 \, \text{LY}$). Neglecting the small quantum term ($\alpha \approx 2.5 \times 10^{-9}$), the combined effect of gravitation and electromagnetic repulsion ($\beta \approx 0.1$) dominates.

The calculated collapse time is $t_{\text{coll}} \approx 1.3 \times 10^5$ years. This is in good agreement with observations, which suggest collapse times on the order of $10^5$ years for clouds of this mass. The result shows that the corrected theory predicts rapid gravitational collapse despite the inhibiting effect of Coulomb repulsion.

\paragraph{On the Role of the Quantum Potential:} The numerical integration for a macroscopic cloud ($M = 10^3 \, M_\odot$, $R_0 \approx 1 \, \text{LY}$) suggests that the contribution of the quantum potential $Q$ in the equation of motion is negligible. This conclusion is deceptive and based on a scaling effect. The true, crucial function of $Q$ is not to provide a direct force counteracting gravity, but to stabilize the electron cloud against its intrinsic Coulomb repulsion and enable coherent contraction in the first place.

Without the quantum potential, the electron component of the cloud would immediately disperse, and gravitational collapse would be blocked by electrostatic repulsion. $Q$ acts as a non-local, cohesive force that suppresses this dispersion. While its contribution appears quantitatively small in the early phase of contraction (large $R$), it becomes dominant on small scales ($R \to 0$), as it scales with $\propto 1/R^3$ – and thus grows faster than gravity ($\propto 1/R^2$) and Coulomb repulsion ($\propto 1/R^2$).

Thus, $Q$ is not the \textit{driver} of the collapse, but its fundamental \textit{guarantor}. It is the physical entity that enforces fractal structure formation (Eq. \ref{eq:dichtefluktuation}) and explains why stars can form despite the overwhelming electromagnetic barrier. The numerical solution merely confirms that the initial contraction on large scales is dominated by gravity; the actual proof of the theory lies in the successful stabilization on the micro level, which is not visible in the present macroscopic calculation.

\section{Fractal Structure Formation}
The density fluctuations follow:

\begin{equation}
P(k) = P_0 k^{-0.29} \quad \text{(corresponding to } D \approx 2.71\text{)}
\end{equation}

This scaling explains both the large-scale cloud structure and the sub-fragmentation into protostellar cores.

\chapter{Fusion Plasmas in WDBT}
\section{Basic Equations of WDBT for Plasmas}

The dynamics of a fusion plasma in the \gls{wdbt} is described by the coupled equations for the action function $S(\vec{r},t)$ and the particle density $\rho(\vec{r},t)$:

\begin{align}
\frac{\partial S}{\partial t} + \frac{(\nabla S - q\vec{A})^2}{2m} + V + Q + \Phi_{\text{Weber}} &= 0 \label{eq:B1} \\
\frac{\partial \rho}{\partial t} + \nabla \cdot \left(\rho \frac{\nabla S - q\vec{A}}{m}\right) &= 0 \label{eq:B2}
\end{align}

with the quantum potential:
\begin{equation}
Q = -\frac{\hbar^2}{2m} \frac{\nabla^2 \sqrt{\rho}}{\sqrt{\rho}} \label{eq:Q}
\end{equation}

and the Weber potential for particle interactions:
\begin{equation}
\Phi_{\text{Weber}} = \int d^3r' \rho(\vec{r}') V_{\text{Weber}}(|\vec{r}-\vec{r}'|) g(|\vec{r}-\vec{r}'|)
\end{equation}

\subsection{Modified Plasma Dynamics}

The Weber force density in the plasma results from the integration over pair correlations:

\begin{equation}
\vec{f}_{\text{Weber}} = n_e n_i \int d^3r \frac{q_1 q_2}{4\pi\epsilon_0 r^2} \left[ \left(1-\frac{v^2}{c^2}\right)\hat{r} + \frac{2(\vec{v}\cdot\hat{r})}{c^2}\vec{v} \right] g(\vec{r})
\end{equation}

This leads to a modified magnetic dynamics:

\begin{equation}
\label{eq:modified_ampere}
\nabla \times \vec{B} = \mu_0 \vec{j} + \frac{\mu_0 e^2 n_e \lambda_c^2}{\epsilon_0} \frac{\partial \vec{j}}{\partial t}
\end{equation}

\subsection{Stability Analysis for Fusion Plasmas}

The dispersion relation for plasma waves considering the quantum potential:

\begin{equation}
\label{eq:dispersion}
\omega^2 = \omega_p^2 \left(1 + \frac{\hbar^2 k^2}{4m_e^2 \omega_p^2}\right)
\end{equation}

The stability condition for a cylindrical plasma with radius $R$:

\begin{equation}
\label{eq:stability}
\frac{d}{dr}\left(r\frac{dQ}{dr}\right) - \frac{m^2}{r}Q + \left(\frac{\omega^2}{v_A^2} - k^2\right)rQ = 0
\end{equation}

with $v_A = B/\sqrt{\mu_0 \rho_m}$ the Alfvén velocity.

\subsection{Fractal Scaling of Current Density}

The \gls{wdbt} predicts a characteristic scaling for Birkeland currents:

\begin{equation}
j(r) = j_0 \left(\frac{r}{r_0}\right)^{D-3} \approx j_0 \left(\frac{r}{r_0}\right)^{-0.29}
\end{equation}

with the fractal dimension $D = \frac{\ln 20}{\ln(2+\phi)} \approx 2.71$.

\subsection{Energy Balance in a Field-Free Plasma}

Energy conservation considering the quantum potential:

\begin{equation}
\label{eq:energy}
\frac{d}{dt}\left(\frac{3}{2}n_e k_B T_e + \frac{\hbar^2}{8m_e} \frac{(\nabla n_e)^2}{n_e}\right) = P_{\text{ext}} - P_{\text{rad}}
\end{equation}

\chapter{Propulsion Technology}
\section{Electrically Charged Pressure Chamber as Directed Plasma Propulsion}
\label{sec:plasma-antrieb}

\subsection{Principle and Theory}
The proposed propulsion system uses a \textbf{negatively charged pressure chamber} to separately accelerate electrons and protons after laser ionization of a cryogenic propellant (e.g., liquid hydrogen, LH\textsubscript{2}). The electrical circuit is closed through the spacecraft hull.

\subsubsection*{Main Equations}
\begin{itemize}
    \item \textbf{Charge separation} after ionization:
        \begin{equation}
            n_e = n_p = \frac{\rho_{\text{LH}_2} N_A}{M_{\text{H}_2}} \approx 4.23 \times 10^{28}\,\text{m}^{-3} \quad \text{(at complete ionization)}
            \label{eq:density}
        \end{equation}
    
    \item \textbf{Extraction field strength} for electrons:
        \begin{equation}
            E = \frac{V_{\text{chamber}}}{d} \quad \text{(typically } V_{\text{chamber}} = -1\,\text{MV}, d = 0.1\,\text{m} \Rightarrow E = 10\,\text{MV/m})
            \label{eq:field}
        \end{equation}
    
    \item \textbf{Proton acceleration} (non-relativistic):
        \begin{equation}
            F_p = n_p \cdot e \cdot v_p \times B \quad \text{(Lorentz force)}
            \label{eq:lorentz_2}
        \end{equation}
\end{itemize}

\subsection{Critical Analysis}
\subsubsection*{Advantages}
\begin{itemize}
    \item \textbf{Precise control}: Separate control of electrons (electrostatic) and protons (magnetic).
    \item \textbf{Energy recovery}: Electron current could be utilized (e.g., for cooling).
    \item \textbf{No mechanical wear}: No moving parts in the nozzle.
\end{itemize}

\subsubsection*{Challenges}
\begin{table}[ht]
    \centering
    \begin{tabular}{ll}
        \toprule
        \textbf{Problem} & \textbf{Proposed Solution} \\
        \midrule
        Gigavolt potentials & Pulsed operation with $f > 1\,\text{kHz}$ \\
        Hull current >1\,MA & Superconducting coating (YBCO) \\
        Proton beam scattering & Quantum potential $Q$ of the \gls{wdbt} \\
        \bottomrule
    \end{tabular}
    \caption{Technical challenges and proposed solutions.}
    \label{tab:problems}
\end{table}

\subsection{Feasibility and Outlook}
The system requires advances in:
\begin{enumerate}
    \item \textbf{High-voltage technology}: Vacuum-insulated chamber designs (diamond-tungsten).
    \item \textbf{Superconductivity}: Stable superconductors for magnetic fields >20\,T.
    \item \textbf{Laser technology}: Femtosecond pulses with $E > 100\,\text{J}$ at MHz frequencies.
\end{enumerate}

\begin{equation}
    \text{Conclusion: } \boxed{\text{Theoretically feasible, but experimental validation on a laboratory scale is necessary.}}
\end{equation}

\section{Combined Propulsion and Radiation Protection via Hull Current}
\label{sec:hullcurrent}

\subsection{Principle of the Dual-Use System}
The spacecraft hull serves both as a \textbf{current return path} for the plasma propulsion (cf. Section~\ref{sec:plasma-antrieb}) and as an \textbf{active radiation shield} via the induced magnetic field.

\subsection{Physical Foundations}
\subsubsection*{Magnetic Field Calculation (Ampère's Law)}
The toroidal magnetic field generated by the hull current $I_H$ in the interior:
\begin{equation}
    B_\phi(r) = \frac{\mu_0 I_H}{2\pi r} \quad \text{(Cylindrical coordinates)}
    \label{eq:bfield}
\end{equation}

\subsubsection*{Radiation Deflection (Lorentz Force)}
Charged cosmic particles (protons, $\alpha$-particles) are deflected:
\begin{equation}
    F_L = q v \times B \quad \Rightarrow \quad r_L = \frac{m v_\perp}{|q| B} \quad \text{(Gyroradius)}
    \label{eq:lorentz}
\end{equation}

\subsection{Technical Specifications}
\begin{table}[ht]
    \centering
    \begin{tabular}{lc}
        \toprule
        \textbf{Parameter} & \textbf{Value} \\
        \midrule
        Hull current $I_H$ & 1\,MA \\
        Hull radius $R$ & 5\,m \\
        Magnetic field $B(R)$ & 0.08\,T (800\,G) \\
        Shielding effectiveness (for 1\,GeV protons) & $r_L \approx 125\,\text{m}$ \\
        \bottomrule
    \end{tabular}
    \caption{Sample calculation for a manned spacecraft.}
    \label{tab:specs}
\end{table}

\subsection{Critical Assessment}
\subsubsection*{Advantages}
\begin{itemize}
    \item \textbf{Energy efficiency}: Utilization of the propulsion current for passive protection.
    \item \textbf{Direction dependence}: Maximum shielding along the torus axis.
\end{itemize}

\subsubsection*{Challenges}
\begin{itemize}
    \item \textbf{Superconductor resources}: \textbf{Superconducting cables} are needed for 1\,MA (YBCO or MgB\textsubscript{2}).
    \item \textbf{Neutral particles}: Undeflected neutrons require additional polymer layers.
    \item \textbf{Interference fields}: Magnetic field interferes with onboard electronics ($\mu$-metal shielding required).
\end{itemize}

\subsection{Integrated Solution}
Combination with the plasma propulsion from Section~\ref{sec:plasma-antrieb}:
\begin{enumerate}
    \item Electron current flows back via the superconducting hull.
    \item Induced $B$-field forms a \textbf{miniaturized magnetosphere model}.
    \item Additional shielding via \textbf{plasma recoil} (secondary interactions).
\end{enumerate}

\begin{equation}
    \text{Total shielding effectiveness} \approx \exp\left(-\frac{d}{r_L}\right) \quad \text{(Exponential damping)}
\end{equation}

\chapter{Emergence of Maxwell Theory from WDBT}
\label{sec:emergence-maxwell}

The consistency of the Weber-De Broglie-Bohm Theory (\gls{wdbt}) requires that it contains the successful Maxwellian electrodynamics as a limiting case. This section shows exactly how Maxwell's equations and charge conservation emerge from the fundamental principles of the \gls{wdbt}.

\section{The Reduction Path: From Non-Local to Local}

The emergence is defined by two consistent approximations, which gradually reduce the non-local and quantum mechanical depth of the \gls{wdbt}:

\begin{enumerate}
    \item \textbf{Locality Approximation:} The pair correlation function $g(|\vec{r} - \vec{r}'|)$ is approximated by a delta function: $g(|\vec{r} - \vec{r}'|) \rightarrow \delta^{(3)}(\vec{r} - \vec{r}')$. This switches off the non-local interaction terms and reduces the force density to a local description.
    \item \textbf{Classical Limit:} The quantum potential $Q = -\frac{\hbar^2}{2m} \frac{\nabla^2 \sqrt{\rho}}{\sqrt{\rho}}$ is neglected ($Q \rightarrow 0$). This corresponds to the transition to classical physics.
\end{enumerate}

Under these approximations, the structures of Maxwell theory must necessarily emerge from the equations of the \gls{wdbt}.

\subsection{Emergence of the Continuity Equation}

The continuity equation, expressing charge conservation, is fundamentally embedded in the structure of the \gls{wdbt}. The starting point is the conservation of probability density (Eq. \ref{eq:B2}):

\begin{equation}
    \frac{\partial \rho}{\partial t} + \nabla \cdot \left( \rho \frac{\nabla S - q\vec{A}}{m} \right) = 0
\end{equation}

Defining the hydrodynamic velocity $\vec{v} = \frac{\nabla S - q\vec{A}}{m}$, one immediately recognizes the standard form of a continuity equation:

\begin{equation}
    \frac{\partial \rho}{\partial t} + \nabla \cdot (\rho \vec{v}) = 0
\end{equation}

Multiplication by the charge $q$ directly yields the electrical continuity equation, where $\rho_{\text{charge}} = q\rho$ is the charge density and $\vec{j} = \rho_{\text{charge}} \vec{v}$ is the current density:

\begin{equation}
    \frac{\partial \rho_{\text{charge}}}{\partial t} + \nabla \cdot \vec{j} = 0
\end{equation}

This derivation is exact and requires no approximations. \textbf{Charge conservation} is thus a more fundamental principle in the \gls{wdbt} than in Maxwell theory, as it follows directly from the structure of quantum mechanical probability conservation.

\subsection{Emergence of the Field Equations}

In the \gls{wdbt}, electromagnetic fields ($\vec{E}$, $\vec{B}$) are not fundamental entities, but \textit{effective auxiliary quantities} derived from the averaged Weber interaction.

\subsubsection{Scalar Potential and Gauss's Law}

The Coulomb part of the Weber force density (Eq. \ref{eq:weber_kraftdichte}) leads, in the locality limit ($g(\vec{r}) \rightarrow \delta^{(3)}(\vec{r})$), to the Poisson equation:

\begin{equation}
    \nabla^2 \phi = -\frac{\rho}{\epsilon_0}
\end{equation}

This equation \textit{defines} the electrostatic potential $\phi$. Applying the nabla operator to both sides immediately yields Gauss's law for the electric field $\vec{E} = -\nabla \phi$:

\begin{equation}
    \nabla \cdot \vec{E} = \frac{\rho}{\epsilon_0}
\end{equation}

\subsubsection{Vector Potential and Magnetic Laws}

The velocity-dependent terms of the Weber force density lead, in the same limit, to expressions proportional to $\vec{v} \times (\nabla \times \vec{A})$. This identifies the magnetic flux density $\vec{B}$ with the curl of a vector potential:

\begin{equation}
    \vec{B} = \nabla \times \vec{A}
\end{equation}

This definition immediately implies the source-free nature of the magnetic field, the second homogeneous Maxwell equation:

\begin{equation}
    \nabla \cdot \vec{B} = 0
\end{equation}

\subsubsection{Faraday's Law of Induction}

The law of induction is a direct mathematical consequence of the definition of the fields from the potentials. From $\vec{E} = -\nabla \phi - \frac{\partial \vec{A}}{\partial t}$, taking the curl yields:

\begin{equation}
    \nabla \times \vec{E} = \nabla \times (-\nabla \phi - \frac{\partial \vec{A}}{\partial t}) = - \underbrace{\nabla \times (\nabla \phi)}_{=0} - \frac{\partial}{\partial t} (\underbrace{\nabla \times \vec{A}}_{=\vec{B}})
\end{equation}

From which Faraday's law follows:

\begin{equation}
    \nabla \times \vec{E} = -\frac{\partial \vec{B}}{\partial t}
\end{equation}

\subsubsection{Ampère's Circuital Law and its Extension}

The full power of the \gls{wdbt} is shown in the derivation of the circuital law. From the analysis of the force density emerges not the classical Ampère-Maxwell law, but an extended, non-local version (Eq. \ref{eq:modified_ampere}):

\begin{equation}
    \nabla \times \vec{B} = \mu_0 \vec{j} + \frac{\mu_0 e^2 n_e \lambda_c^2}{\epsilon_0} \frac{\partial \vec{j}}{\partial t}
\end{equation}

In the \textit{Maxwell limit} ($\lambda_c \rightarrow 0$, i.e., neglecting non-locality), the additional term vanishes and we obtain the original Ampère's law for stationary currents:

\begin{equation}
    \lim_{\lambda_c \to 0} \left( \nabla \times \vec{B} \right) = \mu_0 \vec{j}
\end{equation}

To obtain the complete Maxwell theory, the displacement current $\mu_0 \epsilon_0 \frac{\partial \vec{E}}{\partial t}$ must be added. However, this step is \textit{ad hoc} in the \gls{wdbt}. Equation (10) instead represents the \textbf{more fundamental form}, as it directly describes the non-local origin of the field offsets. The Maxwellian term itself only emerges from a further simplification, namely the assumption of a linear relationship between current and field in simple media.

\subsection{Summary: Maxwell Theory as an Effective Description}

The derivation shows that the entire Maxwellian electrodynamics consistently emerges from the \gls{wdbt} once non-local and quantum mechanical effects are neglected. The \gls{wdbt} thus not only explains the existence of Maxwell's equations but also their limits:

\begin{itemize}
    \item The \textbf{continuity equation} is a more fundamental principle.
    \item The \textbf{homogeneous equations} ($\nabla \cdot \vec{B} = 0$, $\nabla \times \vec{E} = -\partial_t \vec{B}$) are direct consequences of the potential definition.
    \item The \textbf{inhomogeneous equations} ($\nabla \cdot \vec{E} = \rho/\epsilon_0$, $\nabla \times \vec{B} = \ldots$) emerge from the averaged Weber force density.
    \item Ampère's law is extended by a \textbf{non-local correction term} that vanishes under laboratory conditions ($\lambda_c \rightarrow 0$), but becomes dominant in dense plasmas or on small scales.
\end{itemize}

Maxwell theory is thus the \textit{effective field theory} of the \gls{wdbt} for the limiting case of slow, local phenomena. The \gls{wdbt} itself provides the more fundamental, unified framework connecting micro and macro physics.