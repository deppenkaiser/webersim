\chapter{Astrophysical Plasmas in the Framework of WDBT}
\section{Fractal Plasma Universe: Novel Explanatory Approaches}
The \gls{wdbt} offers a novel interpretation of astrophysical phenomena that fundamentally differs from the magnetohydrodynamic description (\gls{mhd}). In contrast to \gls{mhd}, the \gls{wdbt} postulates that large-scale structures of the universe arise from non-local interactions, described by the Weber force (Eq. \ref{eq:weber_em_vektor}) and the quantum potential (Eq. \ref{eq:quantenpotential}).

\subsubsection{Cosmic Filaments and Fractality:}
The theory predicts a characteristic fractal distribution of plasma density (Eq. \ref{eq:dichtefluktuation}), which agrees remarkably well with the observed large-scale structures of the universe. The scale-invariant solution with $D \approx 2.71$ explains why similar patterns appear in both galactic filaments and laboratory plasmas. Furthermore, the modified Ampère equation (Eq. \ref{eq:birkeland_ampere}) provides a natural explanation for the stability of Birkeland currents over cosmological timescales, without needing to resort to dark matter as a stabilizing element.

\subsubsection{Galaxy Rotation and Dark Matter:}
The velocity-dependent terms of the Weber force (Eq. \ref{eq:weber_em_vektor}) lead to an effective modification of the gravitational effect in plasma systems. This could explain the observed deviations from Newtonian predictions, which are usually interpreted through dark matter. The combination of the Weber force and the quantum potential yields a scaling compatible with the empirical Tully-Fisher relations.

\subsubsection{Cosmic Microwave Background (CMB):}
The fractal density fluctuations (Eq. \ref{eq:dichtefluktuation}) produce an anisotropic pattern that shows qualitative similarity to the observed \gls{cmb} fluctuations. This suggests that at least part of the observed structure can be explained by plasma phenomena, without resorting to inflation theories.

\section{The Sun as a Plasma Phenomenon: New Perspectives from WDBT}
In the \gls{wdbt} model, the Sun does not appear as a nuclear-powered fusion reactor with conventional layering, but as a complex, self-organized plasma structure whose form and dynamics can be derived from the fundamental equations of the theory.

\subsubsection{Structure and Dynamics:}
The structure of the Sun is determined by the interplay of the velocity-dependent Weber forces (Eq. \ref{eq:weber_em_vektor}) with the non-local quantum potential (Eq. \ref{eq:quantenpotential}). The sharp boundary of the photosphere is explained by sudden changes in plasma couplings, while the fractal nature of the convection zones (with $D \approx 2.71$) points to the scale-invariant structure of the underlying interactions.

\subsubsection{Coronal Heating and Solar Wind:}
The extreme temperatures of the solar corona arise from particle acceleration due to the Weber force terms, not from poorly understood wave heating mechanisms. The solar wind is described as a natural result of this plasma dynamics: the characteristic particle acceleration results directly from the velocity-dependent terms of the Weber force, while the observed filamentary structure is a consequence of the fractal scaling (Eq. \ref{eq:fractal_scaling}).

\subsubsection{Solar Activity Phenomena:}
Sunspots arise from complex, non-local current systems, whose bipolar structure emerges from the fundamental equations of the theory. The 11-year sunspot cycle appears as a resonance phenomenon of the global quantum potential, and solar flares are interpreted as sudden discharges occurring when critical Weber force thresholds are exceeded.

\section{The Solar Wind as a Consequence of Continuous Matter Creation and Non-Local Quantum Dynamics}
According to the \gls{wdbt}, the solar wind does not primarily arise from thermal or magnetohydrodynamic processes, but from a combined effect of quantum vacuum fluctuations, the non-local quantum potential and the fractal space structure. Near massive objects like the Sun, spontaneous quantum fluctuations constantly generate new particle-antiparticle pairs. The quantum potential $Q$ preferentially stabilizes matter (protons/electrons), while antiparticles are suppressed through destructive interference or annihilation. Simultaneously, the \gls{wed} accelerates the charged particles to high velocities through direct velocity-dependent interactions. The fractal dimension $D \approx 2.71$ modifies the propagation dynamics: particles follow optimal paths in the space lattice, explaining the observed supersonic flows (up to 800 km/s).

\paragraph{Experimental Consequence:} The \gls{wdbt} predicts that the solar wind exhibits a wavelength-independent component and non-local particle correlations – both testable deviations from the standard model.

\paragraph{Core Message:} The solar wind is not a purely classical plasma phenomenon, but a quantum process of emergent matter, driven by the geometry of spacetime and non-local interactions.