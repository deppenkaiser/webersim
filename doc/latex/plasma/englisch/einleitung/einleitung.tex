\chapter{Introduction}
\section{Plasmas as the Key to a New Physics}
For over a century, field theories have dominated physical thinking. Yet precisely in the world of plasmas, a deeper truth reveals itself: Nature knows no fields. What we interpret as electromagnetic interactions is a complex web of direct, non-local forces between particles – an insight already present in \gls{wed} and which gains its full significance through \gls{dbt}.

\section{The Cosmic Plasma: A Challenge for the Standard Models}
The field paradigm reaches fundamental limits on a cosmic scale. The cosmic \gls{cmb} can be interpreted not only as a relic of a Big Bang but also as the thermal equilibrium of an infinite, static plasma universe \cite{Arp1998}. The redshift of distant galaxies is alternatively explained by energy losses of light in intergalactic plasmas – a process described more precisely by \gls{wed} than by \gls{art} \cite{einstein1915,LaViolette2010}.

The enigmatic rotation curves of galaxies, which led to the postulation of dark matter, find a natural explanation in plasma cosmology: Electromagnetic forces, modified by the velocity dependence of the Weber interaction, generate the observed velocity profiles \cite{rubin1970}, without needing to resort to invisible particles \cite{Milgrom2015}.

\subsection{Star Formation and Plasma Dynamics}
The challenge of star formation lies in the apparent contradiction between the enormous electromagnetic repulsion of charged particles in interstellar clouds and the comparatively weak gravitation. The \gls{wdbt} elegantly solves this problem through the interplay of the quantum potential and Weber gravitation.

The quantum potential acts as a non-local, stabilizing force that keeps particles in coherent trajectories, suppresses electromagnetic repulsion, and enables large-scale condensation despite the barriers. Simultaneously, the velocity-dependent terms of Weber gravitation cause a rotationally stable contraction of the cloud – a self-organized collapse that requires neither dark matter nor ad-hoc assumptions. The fractal structure of the plasma, which emerges naturally from WDBT, also explains the hierarchical arrangement of star-forming regions in filaments.

\subsection{Nuclear Fusion: From ITER to Field-Free Plasma}
In fusion research, the \gls{wdbt} could lead to paradigmatic advances. Unlike \gls{mhd}, which relies on external magnetic field control and struggles with turbulent scattering and anomalous transport, the \gls{wdbt} describes plasmas as self-organizing systems: The quantum potential ($Q$) intrinsically stabilizes instabilities like Edge-Localized Modes (ELMs), and the Weber force density models transport phenomena more precisely through pair correlations rather than statistical turbulence models. Furthermore, the natural emergence of filamentary current structures (Birkeland currents) with fractal scaling suggests that plasmas in fusion reactors could self-organize, potentially leading to more compact reactor designs without elaborate magnetic field coils.

\subsection{Applications: From Medicine to Space Travel}
The consequences of this new physics extend far beyond basic research. In plasma medicine, WED could explain why certain plasma configurations are biologically more effective than others – not because of field strength, but due to the specific, non-local interaction with tissue molecules. In space propulsion engineering, the \gls{wdbt} offers a new approach: If radiation acceleration occurs through directly acting Weber forces, entirely new propulsion concepts could emerge, ushering in the era of interplanetary space travel.

\section{Plasma Propulsion: Thermoelectric Resonance Expansion}
\label{sec:hybrid_antrieb}

The combination of cryogenic propellants with Weber-De Broglie-Bohm Electrodynamics (\gls{wdbt}) leads to a novel propulsion concept that unites the advantages of chemical and electrical systems.
For a liquid ion gas with particle density $n_e$, the \textbf{extended equation of state} holds:

\begin{equation}
p = \underbrace{n_e k_B T_e}_{\text{thermal}} 
+ \underbrace{\frac{e^2 n_e^{4/3}}{4\pi \epsilon_0} \left(1 + \beta \frac{v^2}{c^2}\right)}_{\text{WDBT correction}}
\label{eq:druck}
\end{equation}

with $\beta = 2$ for the Weber force. The \textbf{critical density} for dominance of the Coulomb pressure is:

\begin{equation}
n_c = \left(\frac{4\pi \epsilon_0 k_B T_e}{e^2}\right)^3 \approx 10^{28}\,\text{m}^{-3}\quad\text{(for }T_e=10^4\,\text{K)}
\end{equation}

\subsection{Resonance Conditions}
\label{subsec:resonanz}

The system behaves analogously to a Helmholtz resonator with \textbf{plasma resonance frequency}:

\begin{equation}
f_r = \frac{c_s}{2\pi}\sqrt{\frac{A_d}{V_c L_d}} \quad \text{with} \quad c_s = \sqrt{\gamma \left(\frac{k_B T_e}{m_i} + \frac{\hbar^2}{4m_e m_i}\frac{\nabla^2 n_e}{n_e}\right)}
\label{eq:resonanz}
\end{equation}

\subsection{Energy Transfer Analysis}
\label{subsec:energie}

The \textbf{energy density scaling} shows the \gls{wdbt} advantage:

\begin{table}[ht]
\centering
\caption{Comparison of energy densities}
\label{tab:energie}
\begin{tabular}{lcc}
\toprule
Propellant type & $E$ [MJ/kg] & $p_{\text{max}}$ [GPa] \\
\midrule
TNT & 4.6 & 20 \\
Liquid hydrogen & 142 & 25 \\
\gls{wdbt}-Plasma (LH$_2$) & 175 & 175 \\
\bottomrule
\end{tabular}
\end{table}

\subsection{Technical Implementation}
\label{subsec:tech}

The \textbf{optimal nozzle geometry} follows the fractal scaling:

\begin{equation}
\frac{dA}{dx} = -A^{1-1/D} \quad \text{with} \quad D = \frac{\ln 20}{\ln(2+\phi)} \approx 2.71
\label{eq:duese}
\end{equation}

The principle of hybrid plasma propulsion utilizes the synergy of cryogenic storage, electrostatic explosion, and quantum coherence: An extremely compressed liquid hydrogen tank is instantaneously ionized. The resulting Coulomb explosion is amplified by the velocity-dependent Weber force – similar to a spring releasing energy through resonant oscillations. The key to control lies in the precise tuning of the resonance conditions, where the quantum potential Q acts as an active damper suppressing chaotic turbulence and redirecting energy into a coherent expansion wave. The resulting thrust force surpasses conventional systems through a unique mechanism of collective quantum acceleration.