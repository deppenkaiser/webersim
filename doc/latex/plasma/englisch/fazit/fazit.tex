\chapter{Conclusion}
\section{Significance and Revolutionary Argumentation}
The \gls{wdbt} does not simply present itself as just another alternative physics theory. Its claim is more radical and fundamental: It positions itself as the underlying, fundamental ur-theory (Theory of Everything), from which the successful parts of established 20th-century physics – theory of relativity, quantum mechanics, Maxwellian electrodynamics – emerge as special limiting cases. This emergence, however, is not a simple "zooming out", but a process of correction and validation.

\subsection{WDBT as a Coherent Superstructure}
The conceptual core of the \gls{wdbt} unites three elements:

\begin{itemize}
    \item \textbf{Weber Electrodynamics:} Replaces the field concept with direct, velocity- and acceleration-dependent interactions between particles.
    \item \textbf{De Broglie-Bohm Theory:} Replaces the indeterministic collapse of the wave function with deterministic guidance via the quantum potential ($Q$).
    \item \textbf{Weber Gravitation \& Fractal Space Structure:} Provides a mechanistic alternative to the geometric curvature of \gls{art} in a space with fractal dimension ($D \approx 2.71$).
\end{itemize}

From this combination, it is derived that the equations of Maxwell, Einstein, and Schrödinger emerge under certain approximations (e.g., $Q \to 0$, neglecting velocity terms, localization of the interaction). The \gls{wdbt} thus claims to be the more general framework that does not discard the established theories, but encompasses and extends them.

\section{The Recursive Nature: A Sign of Deeper Mathematical Depth}
A crucial quality of the \gls{wdbt} is its \textbf{recursive mathematical structure}. The Weber force depends not only on the distance ($r$), but also on the relative velocity ($\dot{r}$) and acceleration ($\ddot{r}$) of the interacting particles.\\This recursivity –

\begin{itemize}
    \item ... gives the theory a \textbf{memory} and \textbf{feedback}, enabling stability and high precision (analogous to recursive digital filters).
    \item ... \textbf{builds in non-locality naturalistically} instead of postulating it as "spooky action at a distance".
    \item ... contains \textbf{more information} about the dynamics of an interaction than a non-recursive theory that only considers snapshots.
\end{itemize}

Thus, the \gls{wdbt} appears not as more complicated, but as a mathematically more fundamental and informative approach.

\subsection{Emergence Through Filtering: The Gain in Physical Validity}
This is the core of the argumentative superiority: The \gls{wdbt} does not let the established theories emerge in their entirety, but filters out their conceptual pathologies. Only the valid, empirically confirmed core of a theory emerges, freed from its internal contradictions. The \gls{wdbt} thus explains not only the successes but also the failures of other theories at their limits.

\begin{itemize}
    \item From \textbf{\gls{art}}, its successes emerge (perihelion precession, light deflection), but not its singularities or the need for "dark" entities.
    \item From \textbf{Maxwell Theory / QED}, the force effects and propagation phenomena emerge, but not the infinite self-energies or radiation paradoxes.
    \item From \textbf{Standard QM}, the Schrödinger equation and its statistical predictions emerge, but not the unexplained probabilistic collapse or the measurement problem.
\end{itemize}

The \gls{wdbt} thus functions as a meta-framework and filter for physical validity. Its greatest proof lies not only in new predictions, but in its ability to coherently explain why the established theories work exactly where they do, and why they fail precisely at the points where they do.

\section{A Paradigm Shift in Justification}
The \gls{wdbt} demands a paradigm shift away from fields and undefined spacetime curvature towards direct interactions and non-local wholeness. According to this argumentation, its significance is that of a fundamental operating system that runs the "software" of known physics and corrects its errors. It claims not only to describe the world but also to provide the rules by which a good description functions at all. The remaining challenge is and remains the experimental confirmation of its specific, deviating predictions – but conceptually and mathematically, it raises the claim to be the most coherent and most valid foundation of physics.