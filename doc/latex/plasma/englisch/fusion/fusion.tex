\chapter{Fusion Research}
\section{Fusion Research in the Light of WDBT}
Conventional fusion research, based on \gls{mhd}, is reaching fundamental limits: Turbulence, anomalous particle transport, and plasma instabilities such as Edge-Localized Modes (ELMs) require complex additional models. The \gls{wdbt} offers a paradigm shift through a field-free description based on direct particle interactions and non-local quantum effects.

\subsection{Self-Organized Plasma Stabilization}
A central advantage of the \gls{wdbt} lies in the inclusion of the Bohmian quantum potential $Q$ (Eq. \ref{eq:quantenpotential}), which exerts a stabilizing effect in dense plasmas. While \gls{mhd} relies on external magnetic fields to control instabilities like Edge-Localized Modes (ELMs), the \gls{wdbt} describes an intrinsic damping through $Q$. This explains why surprisingly stable plasma configurations are observed in experiments like Wendelstein 7-X at high densities ($n_e > 10^{20} m^{-3}$) – an effect consistent with the modified dispersion relation (Eq. \ref{eq:dispersionrelation}).

\subsection{Non-Local Transport and Anomalous Resistances}
The classical explanation for anomalous resistance in tokamaks relies on turbulent scattering, but the \gls{wdbt} provides an elegant alternative: The Weber force density (Eq. \ref{eq:weber_kraftdichte}) describes collective interactions via the pair correlation function $g(\vec{r})$, without resorting to statistical approximations. This could be particularly relevant for compact fusion concepts like spherical tokamaks or stellarators, where local transport models often fail.

\subsection{Birkeland Currents and Scalable Fusion Configurations}
Another promising aspect is the natural emergence of filamentary current structures (Birkeland currents) in the \gls{wdbt}. Their fractal scaling (Eq. \ref{eq:fractal_scaling}) with $D \approx 2.71$ suggests that plasmas in fusion reactors could self-organize – similar to astrophysical phenomena. Practically, this could lead to more compact reactor designs, where elaborate magnetic field coils become partially obsolete.

\subsection{Experimental Challenges and Perspectives}
To establish the \gls{wdbt} in fusion research, targeted experiments are necessary:

\begin{enumerate}
    \item \textbf{Quantum Potential Effects:}\\Can the influence of $Q$ on plasma waves be detected in high-density experiments (e.g., SPARC)?
    \item \textbf{Non-Local Transport:}\\Can measurements of anomalous resistance confirm the predictions from Eq. \ref{eq:weber_kraftdichte}?
    \item \textbf{Filamentary Structures:}\\Do laboratory experiments (e.g., Z-pinch arrangements) show the fractal scaling predicted in Eq. \ref{eq:fractal_scaling}?
\end{enumerate}

If these effects are confirmed, the \gls{wdbt} could pave the way for a new type of fusion reactor – more stable, compact, and without the complexity of current magnetic field technologies. Thus, it would not only enrich theoretical plasma physics but also provide practical solutions for future energy problems.

\textbf{In summary}, this chapter shows how the \gls{wdbt} could fundamentally renew fusion research: through microscopically founded stability mechanisms, more precise transport models, and the vision of a field-free fusion plasma. The existing equations of the \gls{wdbt} (Ch. \ref{ch:grundlagen}) already provide a complete framework for this – now it is up to experimental validation to realize this potential.