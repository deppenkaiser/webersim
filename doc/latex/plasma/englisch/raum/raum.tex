\chapter{Plasma Medicine and Space Travel}
\section{Theoretical Perspectives of WDBT}
The Weber-De Broglie-Bohm Theory opens up new ways of thinking for applications in medicine and space travel that fundamentally differ from conventional concepts. In the field of plasma medicine, the theory offers an alternative explanation for the interaction between cold plasmas and biological tissue. While established models attribute the effect to reactive oxygen species and electromagnetic fields, the \gls{wdbt} describes a mechanism of direct non-local interactions through the Weber force (Eq. \ref{eq:weber_em_vektor}). This could explain why certain plasma frequencies show higher biological activity than others. Particularly interesting is the potential role of the Bohmian quantum potential (Eq. \ref{eq:quantenpotential}) in the selective effect on cancer cells, although this effect has not yet been experimentally proven.

For space propulsion, radically new concepts emerge from the \gls{wdbt}. The theory suggests that by exploiting the velocity-dependent terms in the Weber force (Eq. \ref{eq:weber_em_vektor}), direct plasma acceleration without magnetic confinement fields might be possible. However, such systems would require extremely high plasma densities, as described in Eq. \ref{eq:dispersionrelation}, which far exceed the values of current propulsion technologies. Another promising concept concerns the self-organized formation of current filaments with fractal structure (Eq. \ref{eq:fractal_scaling}), which theoretically could lead to more compact propulsion designs.

The practical implementation of these concepts faces significant challenges. In plasma medicine, experimental evidence for the postulated non-local interactions with biological systems is still lacking. For space applications, fundamental questions about the stability of high-density plasmas under vacuum conditions would first need to be clarified. However, both application areas demonstrate the potential of the \gls{wdbt} to complement or replace established technological approaches through fundamentally new physical principles – provided the theoretical predictions can be experimentally confirmed.