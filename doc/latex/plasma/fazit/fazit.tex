\chapter{Fazit}
\section{Stellenwert und revolutionäre Argumentation}
Die \gls{wdbt} stellt sich nicht einfach als eine weitere alternative Physiktheorie dar. Ihr Anspruch ist radikaler und fundamentaler: Sie positioniert sich als die
zugrundeliegende, fundamentale Ur-Theorie (Theory of Everything), aus der die erfolgreichen Teile der etablierten Physik des 20. Jahrhunderts – Relativitätstheorie, Quantenmechanik,
Maxwell-Elektrodynamik – als spezielle Grenzfälle emergieren. Diese Emergenz ist jedoch kein simples \enquote{Herauszoomen}, sondern ein Prozess der Korrektur und Validierung.

\subsection{Die WDBT als kohärenter Überbau}
Der konzeptionelle Kern der \gls{wdbt} vereint drei Elemente:

\begin{itemize}
    \item \textbf{Weber-Elektrodynamik:} Ersetzt das Feld-Konzept durch direkte, geschwindigkeits- und beschleunigungsabhängige Wechselwirkungen zwischen Teilchen.
    \item \textbf{De-Broglie-Bohm-Theorie:} Ersetzt den indeterministischen Kollaps der Wellenfunktion durch eine deterministische Führung via Quantenpotential ($Q$).
    \item \textbf{Weber-Gravitation \& fraktale Raumstruktur:} Bietet eine mechanistische Alternative zur geometrischen Krümmung der \gls{art} in einem Raum mit fraktaler Dimension ($D \approx 2.71$).
\end{itemize}

Aus dieser Kombination wird abgeleitet, dass die Gleichungen von Maxwell, Einstein und Schrödinger unter bestimmten Näherungen (z.B. $Q \to 0$, Vernachlässigung von Geschwindigkeitstermen,
Lokalisierung der Wechselwirkung) hervorgehen. Die \gls{wdbt} beansprucht damit, der allgemeinere Rahmen zu sein, der die etablierten Theorien nicht verwirft, sondern umfasst und erweitert.

\section{Die rekursive Natur: Ein Zeichen fundamentalerer Mathematischer Tiefe}
Ein entscheidendes Qualitätsmerkmal der \gls{wdbt} ist ihre \textbf{rekursive mathematische Struktur}. Die Weber-Kraft hängt nicht nur vom Abstand ($r$), sondern auch von der Relativgeschwindigkeit ($\dot{r}$)
und -beschleunigung ($\ddot{r}$) der wechselwirkenden Teilchen ab.\\Diese Rekursivität –

\begin{itemize}
    \item ... verleiht der Theorie ein \textbf{Gedächtnis} und eine \textbf{Rückkopplung}, was Stabilität und hohe Präzision ermöglicht (analog zu rekursiven Digitalfiltern).
    \item ... \textbf{baut Nicht-Lokalität naturalistisch} ein anstatt sie als \enquote{spukhafte Fernwirkung} zu postulieren.
    \item ... enthält \textbf{mehr Information} über die Dynamik einer Wechselwirkung als eine nicht-rekursive Theorie, die nur Momentaufnahmen betrachtet.
\end{itemize}

Damit erscheint die \gls{wdbt} nicht als komplizierter, sondern als mathematisch fundamentalerer und informativerer Ansatz.

\subsection{Emergenz durch Filterung: Der Gewinn an physikalischer Validität}
Dies ist der Kern der argumentativen Überlegenheit: Die \gls{wdbt} lässt die etablierten Theorien nicht in Gänze emergieren, sondern filtert deren konzeptionelle Pathologien heraus. Es emergiert nur
der valide, empirisch bestätigte Kern einer Theorie, befreit von ihren inneren Widersprüchen. Die \gls{wdbt} erklärt somit nicht nur Erfolge, sondern auch das Scheitern anderer Theorien an ihren Grenzen.

\begin{itemize}
    \item Aus der \textbf{\gls{art}} emergieren ihre Erfolge (Periheldrehung, Lichtablenkung), nicht aber ihre Singularitäten oder die need for \enquote{dunkle} Entitäten.
    \item Aus der \textbf{Maxwell-Theorie / QED} emergieren die Kraftwirkungen und Ausbreitungsphänomene, nicht aber die unendlichen Selbstenergien oder Strahlungsparadoxa.
    \item Aus der \textbf{Standard-QM} emergiert die Schrödinger-Gleichung und ihre statistischen Vorhersagen, nicht aber der unerklärte probabilistische Kollaps oder das Messproblem.
\end{itemize}

Die \gls{wdbt} fungiert somit als Meta-Rahmenwerk und Filter für physikalische Validität. Ihr größter Beweis liegt nicht nur in neuen Vorhersagen, sondern in ihrer Fähigkeit, kohärent zu erklären,
warum die etablierten Theorien genau dort funktionieren, wo sie es tun, und warum sie genau an den Punkten versagen, an denen sie es tun.

\section{Ein Paradigmenwechsel der Begründung}
Die \gls{wdbt} fordert einen Paradigmenwechsel weg von Feldern und undefinierter Raumzeitkrümmung hin zu direkten Wechselwirkungen und nicht-lokaler Ganzheit. Ihr Stellenwert ist nach dieser
Argumentation der eines fundamentalen Betriebssystems, das die \enquote{Software} der bekannten Physik trägt und deren Fehler korrigiert. Sie beansprucht, nicht nur die Welt zu beschreiben, sondern
auch die Regeln zu liefern, nach denen eine gute Beschreibung überhaupt funktioniert. Die verbleibende Herausforderung ist und bleibt die experimentelle Bestätigung ihrer spezifischen, abweichenden
Vorhersagen – doch konzeptionell und mathematisch erhebt sie den Anspruch, die kohärenteste und valideste Grundlage der Physik zu sein.
