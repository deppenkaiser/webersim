\chapter{Fusionsforschung}
\section{Fusionsforschung im Lichte der WDBT}
Die konventionelle Fusionsforschung, die auf der \gls{mhd} basiert, stößt an fundamentale Grenzen: Turbulenz, anomaler Teilchentransport und plasmaphysikalische Instabilitäten wie
Edge-Localized Modes (ELMs) erfordern komplexe Zusatzmodelle. Die \gls{wdbt} bietet einen Paradigmenwechsel durch eine feldlose Beschreibung, die auf direkten Teilchenwechselwirkungen und nicht-lokalen
Quanteneffekten basiert.

\subsection{Selbstorganisierte Plasmastabilisierung}
Ein zentraler Vorteil der \gls{wdbt} liegt in der Einbeziehung des Bohm’schen Quantenpotentials $Q$ (Gl. \refeq{eq:quantenpotential}), das in dichten Plasmen eine stabilisierende Wirkung entfaltet.
Während die \gls{mhd} auf externe Magnetfelder angewiesen ist, um Instabilitäten wie Edge-Localized Modes (ELMs) zu kontrollieren, beschreibt die \gls{wdbt} eine intrinsische
Dämpfung durch $Q$. Dies erklärt, warum in Experimenten wie Wendelstein 7-X bei hohen Dichten ($n_e > 10^{20} m^{-3}$) überraschend stabile Plasmakonfigurationen beobachtet
werden – ein Effekt, der mit der modifizierten Dispersionsrelation (Gl. \refeq{eq:dispersionrelation}) konsistent ist.

\subsection{Nicht-lokaler Transport und anomale Widerstände}
Die klassische Erklärung für anomalen Widerstand in Tokamaks beruht auf turbulenter Streuung, doch die \gls{wdbt} liefert eine elegante Alternative: Die Weber-Kraftdichte (Gl. \refeq{eq:weber_kraftdichte})
beschreibt kollektive Wechselwirkungen über die Paarkorrelationsfunktion $g(\vec{r})$, ohne auf statistische Näherungen zurückzugreifen. Dies könnte insbesondere für kompakte
Fusionskonzepte wie sphärische Tokamaks oder Stellaratoren relevant sein, wo lokale Transportmodelle oft versagen.

\subsection{Birkeland-Ströme und skalierbare Fusionskonfigurationen}
Ein weiterer vielversprechender Aspekt ist die natürliche Entstehung filamentärer Stromstrukturen (Birkeland-Ströme) in der \gls{wdbt}. Deren fraktale Skalierung (Gl. \refeq{eq:fractal_scaling}) mit
$D \approx 2.71$ legt nahe, dass sich Plasmen in Fusionsreaktoren selbstorganisieren könnten – ähnlich wie in astrophysikalischen Phänomenen. Praktisch könnte dies zu kompakteren
Reaktordesigns führen, bei denen aufwendige Magnetfeldspulen teilweise überflüssig werden.

\subsection{Experimentelle Herausforderungen und Perspektiven}
Um die \gls{wdbt} in der Fusionsforschung zu etablieren, sind gezielte Experimente nötig:

\begin{enumerate}
    \item \textbf{Quantenpotential-Effekte:}\\Lässt sich der Einfluss von $Q$ auf Plasmawellen in Hochdichte-Experimenten (z. B. SPARC) nachweisen?
    \item \textbf{Nicht-lokaler Transport:}\\Können Messungen des anomalen Widerstands die Vorhersagen aus Gl. \refeq{eq:weber_kraftdichte} bestätigen?
    \item \textbf{Filamentäre Strukturen:}\\Zeigen Laborexperimente (z. B. Z-Pinch-Anordnungen) die in Gl. \refeq{eq:fractal_scaling} vorhergesagte fraktale Skalierung?
\end{enumerate}

Falls sich diese Effekte bestätigen, könnte die \gls{wdbt} den Weg zu einem neuen Typ von Fusionsreaktoren ebnen – stabiler, kompakter und ohne die Komplexität heutiger
Magnetfeldtechnologien. Damit würde sie nicht nur die theoretische Plasmaphysik bereichern, sondern auch praktische Lösungen für die Energieprobleme der Zukunft liefern.

\textbf{Zusammenfassend} zeigt dieses Kapitel, wie die \gls{wdbt} die Fusionsforschung von Grund auf erneuern könnte: durch mikroskopisch fundierte Stabilitätsmechanismen, präzisere
Transportmodelle und die Vision eines feldlosen Fusionsplasmas. Die bestehenden Gleichungen der \gls{wdbt} (Kap. \ref{ch:grundlagen}) bieten hierfür bereits einen vollständigen Rahmen – nun liegt
es an der experimentellen Validierung, dieses Potenzial auszuschöpfen.
