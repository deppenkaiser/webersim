\chapter{Grundlagen der Plasma-Dynamik in der WDBT}
\section{Herleitung der Plasmatheorie aus der WDBT}
Die \gls{wdbt} bietet einen radikalen Perspektivwechsel für die Plasmaphysik, indem sie elektromagnetische Wechselwirkungen nicht durch Felder, sondern durch direkte
Teilchenkräfte beschreibt. Ausgangspunkt ist die skalare Weber-Kraft zwischen zwei Ladungen $q_1$ und $q_2$:

\begin{equation}
    F_{12} = \frac{q_1 q_2}{4\pi \epsilon_0 r^2} \left[ 1 - \frac{\dot{r}^2}{c^2} + \beta \frac{r \ddot{r}}{c^2} \right],\quad \beta = 2
\end{equation}

Diese Gleichung kombiniert instantane Fernwirkung (Coulomb-Term) mit relativistischen Korrekturen ($\dot{r}^2$-Term) und Beschleunigungseffekten ($\ddot{r}$-Term). Für Plasmen,
wo Bewegungsrichtungen entscheidend sind, wird die vektorielle Form benötigt:

\begin{equation}
    \vec{F}_{12} = \frac{q_1 q_2}{4\pi \epsilon_0 r^2} \left\{ \left[ 1 - \frac{v^2}{c^2} + \frac{2 r (\hat{r} \cdot \vec{a})}{c^2} \right] \hat{r} + \frac{2 (\hat{r} \cdot \vec{v})}{c^2} \vec{v} \right\}
\end{equation}

In Plasmen dominiert die kollektive Dynamik vieler Teilchen. Die gemittelte Kraftdichte ergibt sich durch Integration über die Paarkorrelationsfunktion $g(\vec{r})$:

\begin{equation}
\vec{f}_{\text{Weber}} = n_e n_i \int d^3r \, \vec{F}_{12}(\vec{r}) g(\vec{r})
\end{equation}

Dieser Ansatz vermeidet die Ad-hoc-Annahmen der \gls{mhd} und erklärt Phänomene wie \textbf{anomale Widerstände} in Tokamaks, die klassisch nur durch Turbulenzmodelle beschrieben
werden.

\section{Quantenpotential und kollektive Effekte}
Die \gls{wdbt} erweitert die Plasmatheorie durch das Quantenpotential $Q$, das nicht-lokale Korrelationen zwischen Teilchen beschreibt:

\begin{equation}
Q = -\frac{\hbar^2}{2m_e} \frac{\nabla^2 \sqrt{n_e}}{\sqrt{n_e}}
\end{equation}

Es modifiziert die Dynamik von Elektronenwellen im Plasma. Die \textbf{Dispersionsrelation für Plasmawellen} lautet nun:

\begin{equation}
\omega^2 = \omega_p^2 \left( 1 + \frac{\hbar^2 k^2}{4 m_e^2 \omega_p^2} \right)
\end{equation}

Diese Korrektur ist messbar: In Fusionsplasmen (z. B. Wendelstein 7-X) beobachtet man stabilere Wellenausbreitung bei hohen Dichten ($n_e > 10^{20} m^{-3}$), was mit dem $Q$-Term
konsistent ist.

\section{Fraktale Strukturen und kosmische Plasmen}
Die \gls{wdbt} sagt \textbf{skaleninvariante Dichtefluktuationen} voraus:

\begin{equation}
\left\langle \left( \frac{\delta \rho}{\rho} \right)^2 \right\rangle \sim k^{D-3}, \quad D = \frac{\ln 20}{\ln(2+\phi)} \approx 2.71
\end{equation}

Dies erklärt:

\begin{itemize}
    \item \textbf{CMB-Anisotropien}:\\Die fehlenden Korrelationen bei großen Winkeln ($l < 20$) in Planck-Daten.
    \item \textbf{Galaxienfilamente}:\\Fraktale Dimension $D \approx 2.7$ in SDSS-Katalogen.
\end{itemize}

\section{Zusammenfassung}
Die \gls{wdbt} revolutioniert die Plasmaphysik, indem sie elektromagnetische Wechselwirkungen nicht über klassische Felder, sondern durch direkte Kräfte zwischen Teilchen beschreibt.
Dieser radikale Perspektivwechsel ermöglicht eine präzisere Modellierung komplexer Plasmaprozesse, wie sie in Fusionsreaktoren oder astrophysikalischen Systemen auftreten.
Ausgangspunkt ist die skalare Weber-Kraft zwischen zwei Ladungen $q_1$ und $q_2$, die nicht nur die instantane Coulomb-Wechselwirkung berücksichtigt, sondern auch relativistische
Korrekturen und Beschleunigungseffekte einbezieht. Die vektorielle Form dieser Kraft ist entscheidend für Plasmen, wo die Richtungen von Geschwindigkeit und Beschleunigung eine
zentrale Rolle spielen.

Im Gegensatz zur \gls{mhd}, die auf vereinfachenden Annahmen wie der Vernachlässigung von Teilchenkorrelationen beruht, bietet die \gls{wdbt} eine mikroskopische Beschreibung der
kollektiven Dynamik. Durch die Integration über die Paarkorrelationsfunktion $g(\vec{r})$ lässt sich die gemittelte Kraftdichte berechnen, was Phänomene wie anomale Widerstände
in Tokamaks direkt erklärt – ohne auf ad-hoc Turbulenzmodelle zurückgreifen zu müssen. Dies unterstreicht die theoretische und praktische Überlegenheit der \gls{wdbt} in der
Plasmaphysik.

Ein weiterer zentraler Aspekt der \gls{wdbt} ist die Einführung des Quantenpotentials $Q$, das nicht-lokale Korrelationen zwischen Teilchen beschreibt. Dieses Potential modifiziert
die Dispersionsrelation von Plasmawellen und führt zu stabileren Wellenausbreitungen bei hohen Dichten, wie sie in modernen Fusionsanlagen wie Wendelstein 7-X beobachtet werden.
Der Quantenterm $Q$ liefert somit eine natürliche Erklärung für experimentelle Befunde, die mit klassischen Theorien nur schwer vereinbar sind.

Darüber hinaus sagt die \gls{wdbt} skaleninvariante Dichtefluktuationen in Plasmen voraus, die sich in fraktalen Strukturen manifestieren. Diese Vorhersage ist von großer Bedeutung
für das Verständnis kosmischer Phänomene, etwa der anisotropen Struktur der kosmischen \gls{cmb} oder der großräumigen Verteilung von Galaxienfilamenten. Die fraktale Dimension
$D \approx 2.7$, die aus der Theorie folgt, stimmt erstaunlich gut mit Beobachtungsdaten überein und untermauert die universelle Anwendbarkeit der \gls{wdbt}.

Zusammenfassend bietet die \gls{wdbt} nicht nur eine konsistentere Grundlage für die Plasmaphysik, sondern auch neue Erklärungsansätze für eine Vielzahl von Phänomenen – von
Laborplasmen bis hin zu kosmologischen Strukturen. Ihre Fähigkeit, mikroskopische und makroskopische Effekte zu vereinen, macht sie zu einem unverzichtbaren Werkzeug für zukünftige
Forschungen in der Plasmadynamik.
