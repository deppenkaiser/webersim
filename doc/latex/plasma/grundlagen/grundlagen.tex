\chapter{Grundlagen der Plasma-Dynamik in der WDBT}
\section{Herleitung der Plasmatheorie aus der WDBT}
Die \gls{wdbt} bietet einen radikalen Perspektivwechsel für die Plasmaphysik, indem sie elektromagnetische Wechselwirkungen nicht durch Felder, sondern durch direkte
Teilchenkräfte beschreibt. Ausgangspunkt ist die skalare Weber-Kraft zwischen zwei Ladungen $q_1$ und $q_2$:

\begin{equation}
    F_{12} = \frac{q_1 q_2}{4\pi \epsilon_0 r^2} \left[ 1 - \frac{\dot{r}^2}{c^2} + \beta \frac{r \ddot{r}}{c^2} \right],\quad \beta = 2
\end{equation}

Diese Gleichung kombiniert instantane Fernwirkung (Coulomb-Term) mit relativistischen Korrekturen ($\dot{r}^2$-Term) und Beschleunigungseffekten ($\ddot{r}$-Term). Für Plasmen,
wo Bewegungsrichtungen entscheidend sind, wird die vektorielle Form benötigt:

\begin{equation}
    \vec{F}_{12} = \frac{q_1 q_2}{4\pi \epsilon_0 r^2} \left\{ \left[ 1 - \frac{v^2}{c^2} + \frac{2 r (\hat{r} \cdot \vec{a})}{c^2} \right] \hat{r} + \frac{2 (\hat{r} \cdot \vec{v})}{c^2} \vec{v} \right\}
\end{equation}

In Plasmen dominiert die kollektive Dynamik vieler Teilchen. Die gemittelte Kraftdichte ergibt sich durch Integration über die Paarkorrelationsfunktion $g(\vec{r})$:

\begin{equation}
\vec{f}_{\text{Weber}} = n_e n_i \int d^3r \, \vec{F}_{12}(\vec{r}) g(\vec{r})
\end{equation}

Dieser Ansatz vermeidet die Ad-hoc-Annahmen der \gls{mhd} und erklärt Phänomene wie \textbf{anomale Widerstände} in Tokamaks, die klassisch nur durch Turbulenzmodelle beschrieben
werden.

\section{Quantenpotential und kollektive Effekte}
Die \gls{wdbt} erweitert die Plasmatheorie durch das Quantenpotential $Q$, das nicht-lokale Korrelationen zwischen Teilchen beschreibt:

\begin{equation}
Q = -\frac{\hbar^2}{2m_e} \frac{\nabla^2 \sqrt{n_e}}{\sqrt{n_e}}
\end{equation}

Es modifiziert die Dynamik von Elektronenwellen im Plasma. Die \textbf{Dispersionsrelation für Plasmawellen} lautet nun:

\begin{equation}
\omega^2 = \omega_p^2 \left( 1 + \frac{\hbar^2 k^2}{4 m_e^2 \omega_p^2} \right)
\end{equation}

Diese Korrektur ist messbar: In Fusionsplasmen (z. B. Wendelstein 7-X) beobachtet man stabilere Wellenausbreitung bei hohen Dichten ($n_e > 10^{20} m^{-3}$), was mit dem $Q$-Term
konsistent ist.

\section{Fraktale Strukturen und kosmische Plasmen}
Die \gls{wdbt} sagt \textbf{skaleninvariante Dichtefluktuationen} voraus:

\begin{equation}
\left\langle \left( \frac{\delta \rho}{\rho} \right)^2 \right\rangle \sim k^{D-3}, \quad D = \frac{\ln 20}{\ln(2+\phi)} \approx 2.71
\end{equation}

Dies erklärt:

\begin{itemize}
    \item \textbf{CMB-Anisotropien}:\\Die fehlenden Korrelationen bei großen Winkeln ($l < 20$) in Planck-Daten.
    \item \textbf{Galaxienfilamente}:\\Fraktale Dimension $D \approx 2.7$ in SDSS-Katalogen.
\end{itemize}

\section{Herleitung von Birkeland-Strömen aus der WDBT}
Die Entstehung großskaliger Birkeland-Ströme lässt sich konsequent aus den Grundgleichungen der \gls{wdbt} ableiten. Ausgangspunkt ist die verallgemeinerte Weber-Kraft zwischen
geladenen Teilchen in einem Plasma. Während die klassische \gls{mhd} diese Wechselwirkungen durch gemittelte Felder beschreibt, erfasst die \gls{wdbt} die mikroskopischen
Nicht-Lokalitäten direkt.

Die kollektive Dynamik des Plasmas wird durch die gemittelte Weber-Kraftdichte bestimmt, die sich aus der Integration über alle Paarkorrelationen ergibt. Diese Kraftdichte zeigt
charakteristische Abweichungen von der Lorentzkraft der \gls{mhd}, insbesondere durch geschwindigkeitsabhängige Terme und nicht-lokale Kopplungen. Für den Spezialfall
langreichweitiger Korrelationen - einer typischen Situation in astrophysikalischen Plasmen - lässt sich eine modifizierte Version des Ampèreschen Gesetzes ableiten.

Die Stabilitätsanalyse dieser modifizierten Gleichungen zeigt, dass bestimmte Stromkonfigurationen besonders begünstigt werden. Es ergeben sich axialsymmetrische Lösungen, die
den beobachteten Birkeland-Strömen entsprechen: filamentäre Stromsysteme mit einem longitudinalen Stromfluss und begleitendem azimutalen Magnetfeld. Die Dispersionrelation dieser
Lösungen weist auf eine charakteristische Skaleninvarianz hin, die durch die fraktale Dimension $D \approx 2.71$ der \gls{wdbt} vorhergesagt wird.

Besonders bemerkenswert ist die selbstkonsistente Beschreibung der Stromfilamente. Während in der klassischen Theorie zusätzliche Annahmen über die Anregung und Stabilisierung
solcher Ströme notwendig sind, ergeben sie sich in der \gls{wdbt} natürlich aus den fundamentalen Wechselwirkungen. Die fraktale Skalierung erklärt zudem die beobachtete Hierarchie
der Filamentstrukturen, von Laborexperimenten bis zu kosmischen Dimensionen.

% 1. Weber-Kraftdichte
\begin{equation}
\label{eq:force_density}
\vec{f}_{\text{Weber}} = n_e n_i \int d^3r \, \frac{q_1 q_2}{4\pi\epsilon_0 r^2} 
\left[ \left(1 - \frac{v^2}{c^2}\right)\hat{r} + \frac{2(\vec{v}\cdot\hat{r})}{c^2}\vec{v} \right] g(\vec{r})
\end{equation}

% 2. Modifizierte Ampère-Gleichung
\begin{equation}
\label{eq:birkeland_ampere}
\nabla \times \vec{B} = \mu_0 \vec{j} + \frac{\mu_0 e^2 n_e \lambda_c^2}{\epsilon_0} \frac{\partial \vec{j}}{\partial t}
\end{equation}

% 3. Dispersionsrelation für Stromfilamente
\begin{equation}
\omega^2 = \frac{c^2 k^2}{1 + (e^2 n_e \lambda_c^2)/(\epsilon_0 c^2)} \approx v_A^2 k^2 \quad (k\lambda_c \ll 1)
\end{equation}

% 4. Fraktale Skalierung
\begin{equation}
\label{eq:fractal_scaling}
j(r) \propto r^{D-3} \quad \text{mit} \quad D = \frac{\ln 20}{\ln(2+\phi)} \approx 2.71
\end{equation}

% 5. Quantenpotential im Plasma
\begin{equation}
\label{eq:quantenpotential_plasma}
Q = -\frac{\hbar^2}{2m_e} \frac{\nabla^2 \sqrt{n_e}}{\sqrt{n_e}}
\end{equation}

\subsection{Interpretation und Bedeutung}
Diese Herleitung zeigt, wie die \gls{wdbt} eine Brücke zwischen mikroskopischen Wechselwirkungen und makroskopischen Plasmaphänomenen schlägt. Die modifizierte
Ampère-Gleichung (\refeq{eq:birkeland_ampere}) erklärt die Stabilität der Filamente, während die fraktale Skalierung (\refeq{eq:fractal_scaling}) ihre charakteristische räumliche
Verteilung vorhersagt. Das Quantenpotential (\refeq{eq:quantenpotential_plasma}) sorgt für zusätzliche Kohärenzeffekte, die in der klassischen Beschreibung fehlen.

Die besondere Stärke dieses Ansatzes liegt in der natürlichen Erklärung mehrerer Beobachtungen:

\begin{enumerate}
    \item Die selbstorganisierte Bildung von Stromfilamenten
    \item Ihre Stabilität über kosmologische Zeitskalen
    \item Die charakteristische Skaleninvarianz der Strukturen
    \item Die Kopplung zwischen elektrischen Strömen und Magnetfeldkonfigurationen
\end{enumerate}

Damit bietet die \gls{wdbt} nicht nur eine alternative Beschreibung, sondern einen fundamentaleren Zugang zum Verständnis kosmischer Plasmaprozesse.

\section{Zusammenfassung}
Die \gls{wdbt} revolutioniert die Plasmaphysik, indem sie elektromagnetische Wechselwirkungen nicht über klassische Felder, sondern durch direkte Kräfte zwischen Teilchen beschreibt.
Dieser radikale Perspektivwechsel ermöglicht eine präzisere Modellierung komplexer Plasmaprozesse, wie sie in Fusionsreaktoren oder astrophysikalischen Systemen auftreten.
Ausgangspunkt ist die skalare Weber-Kraft zwischen zwei Ladungen $q_1$ und $q_2$, die nicht nur die instantane Coulomb-Wechselwirkung berücksichtigt, sondern auch relativistische
Korrekturen und Beschleunigungseffekte einbezieht. Die vektorielle Form dieser Kraft ist entscheidend für Plasmen, wo die Richtungen von Geschwindigkeit und Beschleunigung eine
zentrale Rolle spielen.

Im Gegensatz zur \gls{mhd}, die auf vereinfachenden Annahmen wie der Vernachlässigung von Teilchenkorrelationen beruht, bietet die \gls{wdbt} eine mikroskopische Beschreibung der
kollektiven Dynamik. Durch die Integration über die Paarkorrelationsfunktion $g(\vec{r})$ lässt sich die gemittelte Kraftdichte berechnen, was Phänomene wie anomale Widerstände
in Tokamaks direkt erklärt – ohne auf ad-hoc Turbulenzmodelle zurückgreifen zu müssen. Dies unterstreicht die theoretische und praktische Überlegenheit der \gls{wdbt} in der
Plasmaphysik.

Ein weiterer zentraler Aspekt der \gls{wdbt} ist die Einführung des Quantenpotentials $Q$, das nicht-lokale Korrelationen zwischen Teilchen beschreibt. Dieses Potential modifiziert
die Dispersionsrelation von Plasmawellen und führt zu stabileren Wellenausbreitungen bei hohen Dichten, wie sie in modernen Fusionsanlagen wie Wendelstein 7-X beobachtet werden.
Der Quantenterm $Q$ liefert somit eine natürliche Erklärung für experimentelle Befunde, die mit klassischen Theorien nur schwer vereinbar sind.

Darüber hinaus sagt die \gls{wdbt} skaleninvariante Dichtefluktuationen in Plasmen voraus, die sich in fraktalen Strukturen manifestieren. Diese Vorhersage ist von großer Bedeutung
für das Verständnis kosmischer Phänomene, etwa der anisotropen Struktur der kosmischen \gls{cmb} oder der großräumigen Verteilung von Galaxienfilamenten. Die fraktale Dimension
$D \approx 2.7$, die aus der Theorie folgt, stimmt erstaunlich gut mit Beobachtungsdaten überein und untermauert die universelle Anwendbarkeit der \gls{wdbt}.

Zusammenfassend bietet die \gls{wdbt} nicht nur eine konsistentere Grundlage für die Plasmaphysik, sondern auch neue Erklärungsansätze für eine Vielzahl von Phänomenen – von
Laborplasmen bis hin zu kosmologischen Strukturen. Ihre Fähigkeit, mikroskopische und makroskopische Effekte zu vereinen, macht sie zu einem unverzichtbaren Werkzeug für zukünftige
Forschungen in der Plasmadynamik.

\subsection{Vergleich zwischen der WDBT und klassischer MHD in der Plasmaphysik}
Die Plasmaphysik steht vor der Herausforderung, das komplexe Verhalten ionisierter Gase auf verschiedenen Skalen zu beschreiben. Während die klassische \gls{mhd} seit Jahrzehnten
den Standardansatz darstellt, bietet die \gls{wdbt} einen radikal neuen Blickwinkel, der möglicherweise einige der hartnäckigsten Probleme des Feldes lösen könnte.

\subsubsection{Grundlegende Unterschiede in der Beschreibung von Plasmen}
Die \gls{mhd} basiert auf den Maxwell-Gleichungen und der Hydrodynamik, behandelt Plasmen also als kontinuierliche, leitfähige Fluide, die durch elektromagnetische Felder
beeinflusst werden. Dieser Ansatz hat sich zwar in vielen Fällen als nützlich erwiesen, stößt jedoch an Grenzen, wenn mikroskopische Effekte oder nicht-lokale Wechselwirkungen
eine Rolle spielen. Die \gls{wdbt} hingegen geht von direkten Teilchenwechselwirkungen aus, beschrieben durch die Weber-Kraft, und integriert zudem Quanteneffekte über das
Bohm'sche Quantenpotential. Während die \gls{mhd} mit der Lorentzkraft arbeitet, berechnet die \gls{wdbt} die Kraftdichte durch Integration über Paarkorrelationen, was eine
natürlichere Beschreibung kollektiver Phänomene ermöglicht.

\subsubsection{Stabilität und Wellenausbreitung in Plasmen}
Ein zentraler Unterschied zeigt sich in der Beschreibung von Plasmawellen und Instabilitäten. Die klassische \gls{mhd} sagt Alfvén-Wellen vorher, deren Dispersionrelation durch
Magnetfelder und Plasmadruck bestimmt wird. Die \gls{wdbt} führt dagegen eine Quantenkorrektur ein, die besonders bei hohen Dichten relevant wird - ein Effekt, der tatsächlich in
Experimenten wie Wendelstein 7-X beobachtet wurde. Während die \gls{mhd} auf externe Magnetfelder angewiesen ist, um Plasmen zu stabilisieren, erklärt die \gls{wdbt}
Stabilisierungseffekte durch das Quantenpotential, was völlig neue Möglichkeiten für Fusionsreaktoren eröffnen könnte.

\subsubsection{Kosmologische Implikationen und großskalige Phänomene}
Besonders bemerkenswert sind die Unterschiede bei der Erklärung kosmologischer Phänomene. Die \gls{mhd}-basierte Astrophysik benötigt Konzepte wie dunkle Materie, um die
Rotationskurven von Galaxien zu erklären. Die \gls{wdbt} hingegen bietet eine elegante Alternative durch ihre fraktale Beschreibung der Dichteverteilung, die ohne solche
Zusatzannahmen auskommt. Ähnlich verhält es sich mit den Anisotropien der kosmischen Hintergrundstrahlung: Während das Standardmodell die Inflationstheorie benötigt, ergibt sich
die Skaleninvarianz in der \gls{wdbt} natürlich aus den grundlegenden Gleichungen.

\subsubsection{Experimentelle Konsequenzen und zukünftige Entwicklungen}
Die \gls{wdbt} sagt mehrere messbare Abweichungen von \gls{mhd}-Vorhersagen voraus, etwa bei der Lamb-Verschiebung oder der Lichtausbreitung in Plasmen. Diese Effekte könnten in
modernen Experimenten überprüft werden und würden im Erfolgsfall die Plasmaphysik revolutionieren. Besonders vielversprechend ist das Potential der \gls{wdbt} in der
Fusionsforschung, wo sie zu stabileren und effizienteren Reaktordesigns führen könnte.

\subsubsection{Fazit: Paradigmenwechsel in der Plasmaphysik?}
Während die \gls{mhd} nach wie vor ein unverzichtbares Werkzeug für viele praktische Anwendungen bleibt, deutet vieles darauf hin, dass die \gls{wdbt} eine tiefere und umfassendere
Theorie der Plasmadynamik bietet. Ihre Fähigkeit, mikroskopische und makroskopische Phänomene konsistent zu beschreiben, ohne auf ad-hoc-Annahmen zurückgreifen zu müssen, macht
sie zu einem vielversprechenden Kandidaten für den nächsten großen Schritt in unserem Verständnis ionisierter Materie - von Laborplasmen bis hin zur großräumigen Struktur des
Universums.

\section{Emergenz der Birkeland-Ströme in der WDBT}
\label{sec:birkeland_emergence}

Die \gls{wdbt} ermöglicht eine systematische Herleitung der Birkeland-Ströme als kollektives Phänomen. Wir zeigen, wie Birkelands historische Gleichungen als Spezialfall der \gls{wdbt}
erscheinen.

\subsection{Nicht-lokale Stromdynamik}
Aus der Weber-Kraftdichte (Gl.~\ref{eq:force_density}) folgt die modifizierte magnetische Dynamik:

\begin{equation}
\nabla \times \vec{B} = \mu_0\vec{j} + \frac{\mu_0 e^2 n_e \lambda_c^2}{\epsilon_0} \frac{\partial \vec{j}}{\partial t} + \underbrace{\frac{2\mu_0 e^2 n_e^2}{\epsilon_0 c^2} \int d^3r'\, g(\vec{r}')(\vec{v}(\vec{r}')\cdot\hat{r}')\vec{v}(\vec{r}')}_{\text{Nicht-lokaler Weber-Term}}
\label{eq:full_ampere}
\end{equation}

\subsection{Birkeland-Limes}
Für $\lambda_c \to 0$ und $v \ll c$ verschwinden die WDBT-Korrekturen:

\begin{equation}
\nabla \times \vec{B} \approx \mu_0 \vec{j}_{\text{Birkeland}}
\label{eq:birkeland_limit}
\end{equation}

Dies reproduziert Birkelands empirische Relation (Gl.~\ref{eq:birkeland_ampere}).

\subsection{Fraktale Stromverteilung}
Die \gls{wdbt} sagt für die Stromdichte skaleninvariantes Verhalten voraus:

\begin{equation}
\frac{j(r)}{j_0} = \left(\frac{r}{r_0}\right)^{D-3} \approx \left(\frac{r}{r_0}\right)^{-0.29}
\label{eq:fractal_current}
\end{equation}

\subsection{Interpretation}
Die \gls{wdbt} erklärt damit drei Schlüsseleigenschaften der Birkeland-Ströme:
\begin{itemize}
\item Ihre Entstehung durch nicht-lokale Plasmakopplungen (Term $\propto \partial_t j$ in Gl.~\ref{eq:full_ampere})
\item Die beobachtete $r^{-0.3}$-Skalierung (Gl.~\ref{eq:fractal_current})
\item Die Stabilität durch fraktale Selbstorganisation ($D \approx 2.71$)
\end{itemize}
