\chapter{Plasmamedizin und Raumfahrt}
\section{Theoretische Perspektiven der WDBT}
Die Weber-De-Broglie-Bohm-Theorie eröffnet neue Denkansätze für Anwendungen in Medizin und Raumfahrt, die sich grundlegend von konventionellen Konzepten unterscheiden. Im Bereich
der Plasmamedizin bietet die Theorie eine alternative Erklärung für die Wechselwirkung zwischen kalten Plasmen und biologischem Gewebe. Während etablierte Modelle die Wirkung auf
reaktive Sauerstoffspezies und elektromagnetische Felder zurückführen, beschreibt die \gls{wdbt} einen Mechanismus direkter nicht-lokaler Wechselwirkungen durch die Weber-Kraft
(Gl. \refeq{eq:weber_em_vektor}). Diese könnte erklären, warum bestimmte Plasmafrequenzen eine höhere biologische Aktivität zeigen als andere. Besonders interessant ist die mögliche Rolle des
Bohm'schen Quantenpotentials (Gl. \refeq{eq:quantenpotential}) bei der selektiven Wirkung auf Krebszellen, obwohl dieser Effekt bisher nicht experimentell nachgewiesen wurde.

Für Raumfahrtantriebe ergeben sich aus der \gls{wdbt} radikal neue Konzepte. Die Theorie legt nahe, dass durch Ausnutzung der geschwindigkeitsabhängigen Terme in der
Weber-Kraft (Gl. \refeq{eq:weber_em_vektor}) eine direkte Plasmabeschleunigung ohne magnetische Einschlussfelder möglich sein könnte. Allerdings würden solche Systeme extrem hohe Plasmadichten
erfordern, wie sie in Gl. \refeq{eq:dispersionrelation} beschrieben werden und die weit über den Werten aktueller Antriebstechnologien liegen. Ein weiteres vielversprechendes Konzept betrifft die
selbstorganisierte Bildung von Stromfilamenten mit fraktaler Struktur (Gl. \refeq{eq:fractal_scaling}), die theoretisch zu kompakteren Antriebsdesigns führen könnten.

Die praktische Umsetzung dieser Konzepte steht vor erheblichen Herausforderungen. In der Plasmamedizin fehlen bisher experimentelle Nachweise für die postulierten nicht-lokalen
Wechselwirkungen mit biologischen Systemen. Für Raumfahrtanwendungen müssten zunächst grundlegende Fragen zur Stabilität hochdichter Plasmen unter Vakuumbedingungen geklärt werden.
Beide Anwendungsgebiete zeigen jedoch das Potenzial der \gls{wdbt}, etablierte technologische Ansätze durch grundlegend neue physikalische Prinzipien zu ergänzen oder zu
ersetzen - vorausgesetzt, die theoretischen Vorhersagen lassen sich experimentell bestätigen.
