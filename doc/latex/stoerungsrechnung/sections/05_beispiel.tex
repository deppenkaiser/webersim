\section{Konkretes Beispiel: Merkur (j) gestört durch Jupiter (i=1)}

\examplebox{
\subsection*{Beitrag von Jupiter zur Positionsstörung}
\[
\Delta \mathbf{r}_{\text{Jupiter→Merkur}} = \frac{M_{\text{Jupiter}}}{M_\odot} \cdot \frac{a_{\text{Jupiter}}^2}{|\mathbf{r}_{\text{Merkur}} - \mathbf{r}_{\text{Jupiter}}|^2} (\mathbf{r}_{\text{Merkur}} - \mathbf{r}_{\text{Jupiter}})
\]
}

\examplebox{
\subsection*{Beitrag von Jupiter zur Geschwindigkeitsstörung}
\[
\Delta \mathbf{v}_{\text{Jupiter→Merkur}} = \frac{G M_{\text{Jupiter}}}{h_{\text{Merkur}}} \cdot \frac{(\mathbf{r}_{\text{Merkur}} - \mathbf{r}_{\text{Jupiter}}) \times \hat{\mathbf{z}}}{| \mathbf{r}_{\text{Merkur}} - \mathbf{r}_{\text{Jupiter}} |^3}
\]
}

\examplebox{
\subsection*{Beitrag von Jupiter zur Winkelgeschwindigkeitsstörung}
\[
\Delta \omega_{\text{Jupiter→Merkur}} = \frac{(\mathbf{r}_{\text{Merkur}} \times \Delta \mathbf{v}_{\text{Jupiter→Merkur}})_z + (\Delta \mathbf{r}_{\text{Jupiter→Merkur}} \times \mathbf{v}_{\text{Merkur}})_z}{|\mathbf{r}_{\text{Merkur}}|^2}
\]
}
