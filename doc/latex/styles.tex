\geometry{
    a4paper,
    top=25mm,
    inner=30mm,    % Bundsteg (größerer Rand für Buchbindung)
    outer=25mm,
    bottom=30mm,
    headheight=15pt,
}

\pagestyle{fancy}
\fancyhf{}
\fancyhead[LE,RO]{\thepage}
\fancyhead[RE]{\leftmark}    % Kapitelname (gerade Seiten)
\fancyhead[LO]{\rightmark}   % Abschnittname (ungerade Seiten)
\renewcommand{\headrulewidth}{0.4pt}

\theoremstyle{definition}
\newtheorem{definition}{Definition}[chapter]
\newtheorem{law}{Physikalisches Gesetz}[chapter]
\theoremstyle{plain}
\newtheorem{theorem}{Theorem}[chapter]
\newtheorem{lemma}[theorem]{Lemma}
\theoremstyle{remark}
\newtheorem{remark}{Bemerkung}[chapter]

\hypersetup{
    colorlinks=true,
    linkcolor=blue,
    citecolor=black,
    urlcolor=black,
    pdftitle={WDB-Theorie - Eine effektive Quantengravitation},
    pdfauthor={Dipl.-Ing. (FH) Michael Czybor},
}

\addbibresource{literatur.bib}  % Ihre .bib-Datei
\makeglossaries

\setlength{\headheight}{26.76852pt}
