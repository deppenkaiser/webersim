\section{Universelles Zeitformat für Himmelskörper}

\subsection{Standardisiertes Format}
\begin{equation}
\tau = \text{floor}\left(\frac{t}{T}\right) + \frac{\phi(t)}{2\pi}
\end{equation}
wobei:
\begin{itemize}
    \item $t$ = Zeit in Sekunden seit Referenzpunkt
    \item $T$ = Umlaufperiode des Referenzkörpers
    \item $\phi(t)$ = Wahre Anomalie zum Zeitpunkt $t$
\end{itemize}

\subsection{Anwendungsbeispiele}
\begin{itemize}
    \item \textbf{Erde-Mond System:} 2030.5000000
    \begin{itemize}
        \item 2030 = Erdumläufe seit Referenz
        \item 0.5000000 = Mondposition $\phi = \pi$ (180°)
    \end{itemize}
    
    \item \textbf{Mars Mission:} 15.7843210
    \begin{itemize}
        \item 15 = Marsjahre seit Referenz
        \item 0.7843210 = Position $\phi \approx 4.93$ rad (282°)
    \end{itemize}
\end{itemize}

\subsection{Technische Umsetzung}
\begin{verbatim}
typedef struct {
    uint32_t base_cycles;  // Ganzzahlige Umläufe
    double phase;          // Bahnphase [0,1)
} CelestialTime;
\end{verbatim}

\subsection{Vorteile}
\begin{itemize}
    \item Universell anwendbar auf alle Himmelskörper
    \item Präzision: 7 Dezimalstellen ($\pm 0.03$s für Erdumlauf)
    \item Menschenlesbare Darstellung
    \item Keine Schaltsekunden nötig
\end{itemize}

\subsection{Vergleich mit anderen Systemen}
\begin{tabular}{lllll}
    \hline
    System & Präzision & Astronomisch & Mehrkörper & Menschlich \\
    \hline
    UTC & $\pm 1$s & Nein & Nein & Ja \\
    Julianisches Datum & Mikrosekunden & Ja & Nein & Nein \\
    \textbf{YYYY.ZZZZZZZ} & 0.03s (Erde) & Ja & Ja & Ja \\
    \hline
\end{tabular}

\subsection{Mars Rover Beispiel}
\begin{equation}
5.3274510
\end{equation}
\begin{itemize}
    \item 5 = Fünftes Marsjahr seit Landung
    \item 0.3274510 = Position $\phi \approx 2.057$ rad (118°)
\end{itemize}