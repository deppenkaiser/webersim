\chapter{Quantenfeldtheoretische Erweiterung und kosmologische Konsistenz des WBFM}

\section{Von der Ein-Teilchen- zur Vielteilchen-Wellenfunktion}

Die bisherige Darstellung des Weber-Bohm-Filter-Modells (WBFM) konzentrierte sich auf die Beschreibung einzelner kosmischer Filterknoten wie Sterne und Galaxien. In diesem Kapitel erfolgt der Übergang zu einer \textit{quantenfeldtheoretischen Formulierung}, die das Universum als Ganzes beschreibt. Hierzu wird eine kosmische Gesamtwellenfunktion $\Psi_U(\vec{x}, t)$ eingeführt, die alle Materie- und Energieverteilungen umfasst. Diese Funktion erfüllt eine erweiterte WDBT-Gleichung:

\[
i\hbar \frac{\partial \Psi_U}{\partial t} = \left( -\sum_i \frac{\hbar^2}{2m_i} \nabla_i^2 + V_{\text{WG}} + Q_U \right) \Psi_U
\]

wobei $Q_U$ das universelle Quantenpotential bezeichnet und $V_{\text{WG}}$ das\\Weber-Gravitationspotential darstellt.

\section{Quantenfeldtheorie des Weber-Bohm-Vakuums}

Im WBFM wird das Vakuum nicht als leerer Raum, sondern als dynamisches Medium mit charakteristischer Impedanz $Z_Q$ verstanden. Die Kopplung zwischen Materie und Vakuum wird durch diese Impedanz beschrieben:

\[
Z_Q = \sqrt{\frac{\mu_Q}{\epsilon_Q}} = \frac{h}{e^2} \alpha^{-1} \approx 4.8 \times 10^3 \, \Omega
\]

Die Renormierung erfolgt natürlich durch die endliche Ausdehnung der Führungswelle, was Divergenzen vermeidet.

\section{Kosmologische Wellenfunktion und Strukturbildung}

Die fraktale Struktur des Universums mit $D \approx 2.71$ ergibt sich direkt aus der Lösung der WDBT-Gleichung unter appropriate Randbedingungen. Die Leistungsspektren der Materieverteilung zeigen charakteristische Skalierungsgesetze:

\[
P(k) \propto k^{-(3-D)} \approx k^{-0.29}
\]

\section{Dunkle Energie als Impedanzmismatch im Kosmos}

Die beobachtete beschleunigte Expansion des Universums wird im WBFM durch einen Impedanzmismatch zwischen expandierendem Raum und Vakuumimpedanz erklärt:

\[
\Lambda \sim \frac{1}{Z_Q^2} \left( \frac{dZ}{dt} \right)^2
\]

\section{Quantengravitation ohne Singularitäten}

Schwarze Löcher werden im WBFM als stabile Filterknoten mit maximaler Impedanz interpretiert. Die Quantenpotential-Barriere verhindert Singularitäten:

\[
Q(r) \sim \frac{\hbar^2}{2m} \frac{1}{r^2} \quad \text{für } r \to 0
\]

\section{Testbare Vorhersagen auf kosmologischen Skalen}

Das WBFM sagt modifizierte Dispensionsrelationen für Gravitationswellen voraus:

\[
v_g(f) = c \left( 1 + \alpha \left( \frac{f}{f_P} \right)^{D-3} \right)
\]

sowie frequenzabhängige Lichtlaufzeiten bei Gravitationslinsen.

\section{Numerische Implementierung und Simulation}
Die Implementierung des WBFM erfordert gitterbasierte Berechnungen des Quantenpotentials $Q_U$ in einem expandierenden Universum. Vergleiche mit
Standard-$\Lambda$CDM-Simulationen zeigen charakteristische Abweichungen in der Large-Scale Structure.

\section{Philosophische und methodologische Konsequenzen}
Das WBFM vertritt eine ontologische Reduktion physikalischer Entitäten und ein Prinzip der maximalen Empirie. Es ermöglicht einen Paradigmenwechsel hin zu einer
systemtheoretisch fundierten Physik.

\section{Vereinheitlichte Bewegungsgleichung im WBFM}
Die Bewegungsgleichung für ein Teilchen $i$ mit Masse $m_i$ und Ladung $q_i$ in einem System aus $N$ wechselwirkenden Teilchen setzt sich aus folgenden Komponenten zusammen:

\subsection*{1. Weber-Elektrodynamische Kraft}
Die WED-Kraft auf Teilchen $i$ durch Teilchen $j$ ist gegeben durch:
\[
\vec{F}_{ij}^{\mathrm{WED}} = \frac{q_i q_j}{4\pi\epsilon_0 r_{ij}^2} 
\left\{ 
\left[1 - \frac{v_{ij}^2}{c^2} + \frac{2 r_{ij} (\hat{r}_{ij} \cdot \vec{a}_j)}{c^2}\right] \hat{r}_{ij} 
+ \frac{2 (\hat{r}_{ij} \cdot \vec{v}_j)}{c^2} \vec{v}_j 
\right\}
\]
wobei $\vec{r}_{ij} = \vec{x}_i - \vec{x}_j$, $r_{ij} = |\vec{r}_{ij}|$, $\hat{r}_{ij} = \vec{r}_{ij}/r_{ij}$.

\subsection*{2. Weber-Gravitationskraft}
Die WG-Kraft auf Teilchen $i$ durch Teilchen $j$ lautet:
\[
\vec{F}_{ij}^{\mathrm{WG}} = -\frac{G m_i m_j}{r_{ij}^2} 
\left(1 - \frac{\dot{r}_{ij}^2}{c^2} + \beta \frac{r_{ij} \ddot{r}_{ij}}{c^2}\right) \hat{r}_{ij}
\]
mit $\beta = 0.5$ für Massen und $\beta = 1.0$ für Photonen.

\subsection*{3. Quantenpotential-Kraft}
Die Kraft aus dem Quantenpotential ist:
\[
\vec{F}_{i}^{Q} = -\vec{\nabla}_i Q = \frac{\hbar^2}{2m_i} \vec{\nabla}_i \left( \frac{\nabla_i^2 R}{R} \right)
\]
wobei $R$ die Amplitude der Gesamtwellenfunktion $\Psi_U = R e^{iS/\hbar}$ ist.

\subsection*{4. Vollständige Bewegungsgleichung}
Die vereinheitlichte Bewegungsgleichung für das Teilchen $i$ ergibt sich zu:
\[
\boxed{
m_i \frac{d^2 \vec{x}_i}{dt^2} = 
\sum_{j \neq i}^N \left( \vec{F}_{ij}^{\mathrm{WED}} + \vec{F}_{ij}^{\mathrm{WG}} \right) 
+ \frac{\hbar^2}{2m_i} \vec{\nabla}_i \left( \frac{\nabla_i^2 R}{R} \right)
}
\]

\subsection*{Anmerkungen zur Interpretation}
\begin{itemize}
\item Die Gleichung ist \textit{nicht-lokal}, da sowohl die Weber-Kräfte als auch das Quantenpotential von instantanen Wechselwirkungen abhängen
\item Die Bewegung des Teilchens $i$ hängt von der Gesamtwellenfunktion $\Psi_U$ ab, die wiederum von den Positionen aller Teilchen abhängt
\item Es handelt sich um eine Differentialgleichung \textit{dritter Ordnung} aufgrund der Beschleunigungsterme in den Weber-Kräften
\item Die Gleichung verletzt die lokale Lorentz-Invarianz, was konsistent mit der postulierten fundamentalen Nicht-Lokalität des WBFM ist
\end{itemize}

\section{Hierarchische Skalenentkopplung im WBFM}

Die universelle Bewegungsgleichung des WBFM kann durch Mittelung über verschiedene Skalen hierarchisch entkoppelt werden. Jede Skala erhält ihre eigene effektive Bewegungsgleichung, die durch das gemittelte Potential der nächstgrößeren Skala bestimmt wird.

\subsection{Kosmische Skala: Bewegung einer Galaxie}

Auf der kosmischen Skala wird eine Galaxie als Punktmasse $M_{\text{gal}}$ mit Schwerpunkt $\vec{X}_{\text{gal}}$ behandelt. Ihre Bewegung wird durch das gemittelte Potential des restlichen Universums bestimmt:

\begin{equation}
M_{\text{gal}} \frac{d^2 \vec{X}_{\text{gal}}}{dt^2} \approx -\vec{\nabla}_{\vec{X}_{\text{gal}}} \Phi_{\text{eff}}^{\text{cosmic}}
\end{equation}

Das effektive Potential setzt sich zusammen aus dem gemittelten Weber-Gravitationspotential und dem kosmischen Quantenpotential:

\begin{equation}
\Phi_{\text{eff}}^{\text{cosmic}} \approx G \int \frac{\bar{\rho}_{\text{univ}}(\vec{r}', t)}{r'} \left(1 - \frac{\dot{r}'^2}{c^2} + \beta \frac{r' \ddot{r}'}{c^2}\right) d^3r' + \langle Q \rangle_{\text{cosmic}}
\end{equation}

Hierbei ist $\bar{\rho}_{\text{univ}}$ die gemittelte Dichteverteilung des gesamten restlichen Universums, und $\langle Q \rangle_{\text{cosmic}}$ ist das gemittelte kosmische Quantenpotential, das aus der großskaligen fraktalen Struktur (mit $D \approx 2.71$) emergiert.

\subsection{Galaktische Skala: Bewegung eines Sterns}

Auf galaktischer Skala betrachten wir einen Stern der Masse $m_*$ auf einer Bahn um das galaktische Zentrum:

\begin{equation}
m_* \frac{d^2 \vec{x}_*}{dt^2} \approx -\vec{\nabla}_{\vec{x}_*} \Phi_{\text{eff}}^{\text{gal}} - \vec{\nabla} Q_{\text{gal}}
\end{equation}

Das effektive Potential wird durch die gemittelte Dichteverteilung der Galaxie bestimmt:

\begin{equation}
\Phi_{\text{eff}}^{\text{gal}} \approx G \int \frac{\bar{\rho}_{\text{gal}}(\vec{r}', t)}{r'^2} \left(1 - \frac{\dot{r}'^2}{c^2} + \beta \frac{r' \ddot{r}'}{c^2}\right) \hat{r}'  d^3r'
\end{equation}

Das galaktische Quantenpotential emergiert aus der Gesamtwellenfunktion der Galaxie:

\begin{equation}
Q_{\text{gal}} \approx -\frac{\hbar^2}{2 m_*} \frac{\nabla^2 \sqrt{\bar{\rho}_{\text{gal}}}}{\sqrt{\bar{\rho}_{\text{gal}}}} 
\end{equation}

\subsection{Stellare Skala: Bewegung eines Planeten}

Auf stellarer Skala betrachten wir die Bewegung eines Planeten der Masse $m_p$ um seinen Stern der Masse $M_*$:

\begin{equation}
m_p \frac{d^2 \vec{x}_p}{dt^2} \approx -\vec{\nabla} \left( \frac{G M_* m_p}{r} \left(1 - \frac{\dot{r}^2}{c^2} + \beta \frac{r \ddot{r}}{c^2}\right) \right) - \vec{\nabla} Q_{*}
\end{equation}

Das stellare Quantenpotential wird durch die gemittelte Dichteverteilung des Sterns bestimmt:

\begin{equation}
Q_{*} \approx -\frac{\hbar^2}{2 m_p} \frac{\nabla^2 \sqrt{\bar{\rho}_{*}}}{\sqrt{\bar{\rho}_{*}}}
\end{equation}

\subsection{Planetare Skala: Bewegung eines Atoms}

Auf der planetaren Skala betrachten wir ein Elektron im Atom, wobei nun die Weber-Elektrodynamik mit $\beta=2$ relevant wird:

\begin{equation}
m_e \frac{d^2 \vec{x}_e}{dt^2} \approx -\vec{\nabla} \left( \frac{k e^2}{r} \left(1 - \frac{\dot{r}^2}{c^2} + 2 \frac{r \ddot{r}}{c^2}\right) \right) - \vec{\nabla} Q_{\text{atom}}
\end{equation}

Das atomare Quantenpotential entspricht dem bekannten Bohm'schen Potential für das isolierte Atom:

\begin{equation}
Q_{\text{atom}} = -\frac{\hbar^2}{2 m_e} \frac{\nabla^2 |\psi_{\text{atom}}|}{|\psi_{\text{atom}}|}
\end{equation}

\subsection{Zusammenhang zwischen den Skalen}

Die verschiedenen Skalen sind hierarchisch gekoppelt:
\begin{itemize}
\item Die kosmische Skala liefert Randbedingungen und das umgebende Potential für die galaktische Skala
\item Die galaktische Skala bestimmt die Dichteverteilung $\bar{\rho}_{\text{gal}}$ für das Potential $\Phi_{\text{eff}}^{\text{gal}}$
\item Die stellare Skala liefert die Dichteverteilung $\bar{\rho}_{*}$ für $Q_{*}$
\item Die planetare Skala wird von den darüberliegenden Skalen eingehüllt, deren Einfluss vernachlässigbar ist
\end{itemize}

Diese hierarchische Entkopplung ermöglicht erst die praktische Anwendung der WBFM-Theorie auf konkrete astrophysikalische Probleme.
