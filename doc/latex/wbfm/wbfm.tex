\documentclass[11pt, a5paper, twoside, openright]{book}
\usepackage[ngerman]{babel}
\usepackage[T1]{fontenc}
\usepackage[utf8]{inputenc}
\usepackage{lmodern}
\usepackage{microtype}
\usepackage{csquotes}
\usepackage{verbatim}  % Im Kopf des Dokuments einfügen
\usepackage{geometry}
\usepackage{fancyhdr}
\usepackage{amsmath, amssymb, amsthm}  % Mathe
\usepackage{mathtools}                 % \coloneqq, \xrightarrow
\usepackage{bm}                        % Fette Symbole (\bm{B} für Magnetfeld)
\usepackage{siunitx}                   % \SI{1.23}{\meter\per\second}
\usepackage{graphicx}                  % \includegraphics
\usepackage{subcaption}                % Unterabbildungen
\usepackage{booktabs}                  % Professionelle Tabellen
\usepackage{tikz}                      % Für Diagramme
\usepackage{xcolor}                    % Farbige Tabellenzellen
\usepackage[
    backend=biber,
    style=phys,         % APS-Zitierstil (für Physik)
    sorting=nyt,        % Sortierung: Name, Jahr, Titel
]{biblatex}
\usepackage[acronym, toc]{glossaries}
\usepackage{hyperref}
\usepackage{parskip}
\usepackage{pgfplots}
\usepackage{glossaries}
\makeglossaries
\geometry{
    a4paper,
    top=25mm,
    inner=30mm,    % Bundsteg (größerer Rand für Buchbindung)
    outer=25mm,
    bottom=30mm,
    headheight=15pt,
}

\pagestyle{fancy}
\fancyhf{}
\fancyhead[LE,RO]{\thepage}
\fancyhead[RE]{\leftmark}    % Kapitelname (gerade Seiten)
\fancyhead[LO]{\rightmark}   % Abschnittname (ungerade Seiten)
\renewcommand{\headrulewidth}{0.4pt}

\theoremstyle{definition}
\newtheorem{definition}{Definition}[chapter]
\newtheorem{law}{Physikalisches Gesetz}[chapter]
\theoremstyle{plain}
\newtheorem{theorem}{Theorem}[chapter]
\newtheorem{lemma}[theorem]{Lemma}
\theoremstyle{remark}
\newtheorem{remark}{Bemerkung}[chapter]

\hypersetup{
    colorlinks=true,
    linkcolor=blue,
    citecolor=black,
    urlcolor=black,
    pdftitle={WDB-Theorie - Eine effektive Quantengravitation},
    pdfauthor={Dipl.-Ing. (FH) Michael Czybor},
}

\addbibresource{literatur.bib}  % Ihre .bib-Datei
\makeglossaries

\setlength{\headheight}{26.76852pt}

\newacronym{qm}{QM}{Quantenmechanik}
\newacronym{art}{ART}{Allgemeine Relativitätstheorie}
\newacronym{srt}{SRT}{Spezielle Relativitätstheorie}
\newacronym{cmb}{CMB}{Hintergrundstrahlung}
\newacronym{qed}{QED}{Quantenelektrodynamik}
\newacronym{qft}{QFT}{Quantenfeldtheorie}
\newacronym{epr}{EPR-Paradoxon}{Einstein-Podolsky-Rosen-Paradoxon}
\newacronym{wg}{WG}{Weber-Gravitation}
\newacronym{wed}{WED}{Weber-Elektrodynamik}
\newacronym{dbt}{DBT}{De-Broglie-Bohm-Theorie}
\newacronym{wdbt}{WDBT}{Weber-De Broglie-Bohm-Theorie}
\newacronym{mt}{MT}{Maxwell-Theorie}
\newacronym{iwt}{IWT}{Informations-Weber-Theorie}
\newacronym{dstt}{DSTT}{Dynamischen Schwere-Trägheits-Theorie}

\newglossaryentry{gls:quantenmechanik}
{
    name={Quantenmechanik},
    description={Theorie der Materie und Strahlung auf atomarer und subatomarer Ebene}
}
\newglossaryentry{gls:hamiltonian}
{
    name={\ensuremath{\mathcal{H}}},
    description={Hamilton-Operator, beschreibt die Gesamtenergie eines Systems},
    sort={hamiltonian}
}


\begin{document}

\frontmatter
\begin{tikzpicture}[remember picture, overlay]

  % Hintergrund (Dunkel mit fraktalem Gitter)
  \fill[hintergrund] (current page.south west) rectangle (current page.north east);
  \foreach \i in {0,10,...,360} {
    \draw[quantenblau!10, line width=0.1pt] 
      (current page.center) -- +(\i:5cm);
  }

  % Dodekaeder (abstrahiert)
  \node[rotate=25, scale=2, quantenblau!50] at (current page.center) {
    \begin{tikzpicture}[scale=0.3]
      \draw[quantenblau] (0:1) \foreach \a in {72,144,...,360} { -- (\a:1) } -- cycle;
      \foreach \a in {36,108,...,324} { \draw[quantenblau] (0,0) -- (\a:1.6); }
    \end{tikzpicture}
  };

  % Titeltext (mit Schatten-Effekt)
  \node[align=center, text=white, font=\sffamily\bfseries\Huge] 
    at ($(current page.center)+(0,3cm)$) {
    \textbf{WBFM}
  };
  \node[align=center, text=quantenblau!80, font=\sffamily\Large] 
    at ($(current page.center)+(0,1.8cm)$)
    {
        Filter als Modell
    };

  % Kernformeln (rechts unten)
  \node[align=left, anchor=south east, text=weberrot!70, font=\small] 
    at ($(current page.south east)+(-1cm,1cm)$) {
    $\displaystyle \vec{F}_{\text{WG}} = -\frac{GMm}{r^2}\left(1-\frac{\dot{r}^2}{c^2}+\beta\frac{r\ddot{r}}{c^2}\right)$
  };
  \node[align=left, anchor=north east, text=quantenblau!70, font=\small] 
    at ($(current page.south east)+(-1cm,3cm)$) {
    $\displaystyle Q = -\frac{\hbar^2}{2m}\frac{\nabla^2\sqrt{\rho}}{\sqrt{\rho}}$
  };

  % Autor (unten mittig)
  \node[align=center, text=white, font=\sffamily\large] 
    at ($(current page.south)+(0,1cm)$) {
    \textbf{Michael Czybor}
  };

  % Fraktale Dimension (links oben)
  \node[align=right, text=quantenblau!50, font=\small] 
    at ($(current page.north west)+(2cm,-1cm)$) {
    $D = \frac{\ln 20}{\ln(2+\phi)} \approx 2.71$
  };

\end{tikzpicture}

\title{WBFM\\Filter als Modell}
\author{Michael Czybor}
\date{\today}
\maketitle

\chapter*{Vorwort}

Die tiefe Verbindung zwischen Quantenmechanik und Signalverarbeitung stellt sich nicht als bloße Analogie, sondern als fundamentale Entsprechung heraus. Die Mathematik der
Wellenfunktionen, Fourier-Transformationen und nicht-lokalen Korrelationen bildet die gemeinsame Sprache beider Disziplinen. In dieser Arbeit wird aufgezeigt, wie die
\gls{wdbt} diese Verbindung zur Grundlage einer neuen Kosmologie macht.

Das Weber-Bohm-Filter-Modell (WBFM) interpretiert das Universum als ein dynamisches Netzwerk von Filteroperationen, bei dem Sterne und Galaxien als aktive Verarbeitungsknoten
fungieren. Quantenpotentiale wirken als nicht-lokale Übertragungsfunktionen, während die Weber-Kräfte die Rückkopplungsschleifen des Systems bilden. Diese Sichtweise erlaubt
es, scheinbar disparate Phänomene – von der Teilchenphysik bis zur Kosmologie – unter einem einheitlichen systemtheoretischen Rahmen zu beschreiben.

Die hier vorgestellte Modellbildung folgt dem Prinzip der Abstraktion komplexer Zusammenhänge durch filtertheoretische Konzepte. Pol- und Nullstellen-Diagramme ersetzen dabei
traditionelle Feldgleichungen, nicht-lokale Verschaltungen treten an die Stelle von Raumzeit-Krümmung. Diese Herangehensweise ermöglicht nicht nur eine neue Perspektive auf
bestehende Probleme der theoretischen Physik, sondern führt auch zu konkreten, überprüfbaren Vorhersagen.

Die Arbeit verbindet damit zwei scheinbar getrennte Welten: Die mikroskopische Strange der Quantenprozesse mit der makroskopischen Organisation des Kosmos – vereint durch die
Sprache der Systemtheorie und Signalverarbeitung.

\begin{flushright}
    Michael Czybor \\
    \emph{Langenstein/AT, August 2025}
\end{flushright}

\tableofcontents
\listoffigures
\listoftables

\mainmatter
\chapter{Einführung}
\section{Plasmen als Schlüssel zu einer neuen Physik}
Seit über einem Jahrhundert dominieren Feldtheorien das Denken – von den Maxwell-Gleichungen bis zur \gls{qed}. Doch gerade dort, wo diese Theorien an ihre Grenzen stoßen, in der
Welt der Plasmen, offenbart sich eine tiefere Wahrheit: \textbf{Die Natur kennt keine Felder}. Was wir als elektromagnetische Wechselwirkungen interpretieren, ist in Wirklichkeit ein
komplexes Geflecht direkter, nicht-lokaler Kräfte zwischen Teilchen – eine Erkenntnis, die bereits in der \gls{wed} \cite{Weber1846} angelegt ist und durch die \gls{dbt} \cite{bohm1952}
ihre volle Bedeutung erlangt.

\section{Das kosmische Plasma: Eine Herausforderung für die Standardmodelle}
Im großen Maßstab des Universums zeigt sich das Versagen der Feldtheorien besonders deutlich. Die kosmische \gls{cmb}, oft als Beweis für den Urknall gefeiert, könnte
ebenso gut das thermische Gleichgewicht eines unendlichen, statischen Plasmauniversums beschreiben. Die Rotverschiebung ferner Galaxien, die heute als Indiz für die Expansion des
Raumes gedeutet wird, lässt sich alternativ durch Energieverluste des Lichts in intergalaktischen Plasmen erklären – ein Prozess, den die \gls{wed} präziser beschreibt
als die \gls{art} \cite{einstein1915}.

Die rätselhaften Rotationskurven der Galaxien, die zur Erfindung der dunklen Materie führten, finden in der Plasma-Kosmologie eine natürliche Erklärung: Elektromagnetische Kräfte,
modifiziert durch die Geschwindigkeitsabhängigkeit der Weber-Wechselwirkung, können die beobachteten Geschwindigkeitsprofile erzeugen, ohne auf unsichtbare Teilchen zurückgreifen
zu müssen. Die filamentären Strukturen des kosmischen Netzes, die sich über Hunderte von Millionen Lichtjahren erstrecken, ähneln verblüffend den Mustern, die in
Plasmadynamik-Experimenten auf Laborskala entstehen – ein Hinweis darauf, dass das Universum in seinem Wesen ein elektrisches Phänomen ist.

\subsection{Sternentstehung und Plasmadynamik}
Auch die Geburt der Sterne wirft Fragen auf, die das Feldparadigma nicht befriedigend beantworten kann. Wie können interstellare Wolken aus diffusem Plasma unter ihrer eigenen
Gravitation kollabieren, wenn die elektromagnetischen Abstoßungskräfte um Größenordnungen stärker sind? Die \gls{wdbt} hingegen bietet eine elegante Lösung: Das Quantenpotential der \gls{dbt}
wirkt als nicht-lokale, stabilisierende Kraft, die den Kollaps trotz der elektromagnetischen Barrieren ermöglicht. Gleichzeitig erklärt die Weber-Gravitation mit ihrer geschwindigkeitsabhängigen
Komponente, warum protoplanetare Scheiben rotationsstabil bleiben, ohne dass dunkle Materie als \enquote{Klebstoff} benötigt wird. Details hierzu können dem Anhang (\ref{app:sternentstehung})
entnommen werden.

Die Herausforderung der Sternentstehung liegt im scheinbaren Widerspruch zwischen der enormen elektromagnetischen Abstoßung geladener Teilchen in interstellaren Wolken und der
vergleichsweise schwachen Gravitation, die den Kollaps einleiten soll. Während klassische Modelle auf zusätzliche Annahmen wie magnetische Stabilisierung oder Turbulenzdämpfung
zurückgreifen müssen, bietet die \gls{wdbt} eine elegante Lösung durch das Zusammenspiel des Quantenpotentials und der Weber-Gravitation.

Das Quantenpotential wirkt hier nicht nur als quantenmechanische Korrektur, sondern als entscheidender Vermittler zwischen mikroskopischen und makroskopischen Prozessen. Indem es
die Teilchen in kohärenten, geordneten Bahnen hält, verhindert es die sonst dominierende elektromagnetische Abstoßung und ermöglicht eine großräumige Verdichtung der Wolke.
Gleichzeitig stabilisiert es die Struktur gegen turbulente Fragmentierung, ohne den Kollaps selbst zu blockieren – im Gegensatz zu klassischen Modellen, die solche Effekte nur
durch externe Mechanismen erklären können.

Die Weber-Gravitation ergänzt diesen Prozess, indem ihre geschwindigkeitsabhängigen Terme eine rotationsstabile Kontraktion der Wolke bewirken. Dadurch entsteht ein
selbstorganisierter Kollaps, der weder auf hypothetische dunkle Materie noch auf ad-hoc-Annahmen angewiesen ist. Die fraktale Struktur des Plasmas, die sich natürlich aus der
\gls{wdbt} ergibt, erklärt zudem die hierarchische Anordnung von Sternentstehungsregionen in Filamenten – ein Phänomen, das in herkömmlichen Theorien nur schwer abzubilden ist.

Kurz gesagt: Die \gls{wdbt} zeigt, dass Sternentstehung kein Kampf zwischen Gravitation und elektromagnetischen Kräften ist, sondern ein koordinierter Prozess, der durch
nicht-lokale Quanteneffekte und direkte Teilchenwechselwirkungen gesteuert wird. Dieses Bild passt nicht nur besser zu Beobachtungen, sondern vermeidet auch die willkürlichen
Zusatzannahmen der etablierten Modelle.

\subsection{Kernfusion: Vom ITER zum feldlosen Plasma}
Auf der irdischen Skala zeigt sich das Potential der neuen Sichtweise vielleicht am deutlichsten in der Fusionsforschung. Seit Jahrzehnten kämpfen Projekte wie ITER mit den
Unwägbarkeiten der Plasmaturbulenz – einem Problem, das im Rahmen der \gls{mhd} unlösbar erscheint. Doch was, wenn die Turbulenz gar kein chaotisches Phänomen ist,
sondern die Manifestation einer tieferen, nicht-lokalen Ordnung?

Die \gls{wdbt} legt nahe, dass Plasmen in Fusionsreaktoren nicht durch äußere Magnetfelder kontrolliert werden müssen, sondern sich selbst organisieren können – gesteuert durch
das Quantenpotential und die direkten Teilchenwechselwirkungen der \gls{wed}. Es gibt Hinweise dafür, dass Plasmen in dieser Beschreibung stabilere Konfigurationen
einnehmen, als die Feldtheorie vorhersagt. Sollte sich dies bestätigen, könnte es den Weg zu kompakteren, effizienteren Fusionsreaktoren ebnen – eine Revolution der Energiegewinnung.

Die Kernfusion gilt seit Jahrzehnten als vielversprechende Lösung für die Energieprobleme der Menschheit, doch die technischen Herausforderungen bleiben immens. Projekte wie ITER oder
Wendelstein 7-X setzen auf die \gls{mhd}, um Plasmen bei extrem hohen Temperaturen (über 100 Millionen Grad) einzuschließen. Doch trotz enormer Fortschritte kämpfen diese Anlagen mit
unkontrollierbarer Turbulenz, anomalem Teilchentransport und instabilen Plasmarändern – Probleme, die sich mit den klassischen Modellen nur unzureichend beschreiben lassen. Hier setzt
die \gls{wdbt} an und bietet einen radikal neuen Ansatz, der die Fusion revolutionieren könnte.

\subsubsection{Die Grenzen der MHD in der Fusionsforschung}
Die \gls{mhd} beschreibt Plasmen als kontinuierliche Fluide, die durch Magnetfelder geformt werden. Doch diese Näherung vernachlässigt mikroskopische Effekte wie Teilchenkorrelationen
oder nicht-lokale Wechselwirkungen – genau jene Phänomene, die in Fusionsplasmen eine entscheidende Rolle spielen. Turbulenz und anomaler Widerstand entstehen, weil die Lorentzkraft der
\gls{mhd} die komplexe Dynamik geladener Teilchen nur unvollständig erfasst. Die Folge sind unvorhersehbare Energieverluste und instabile Plasmen, die den Betrieb von Tokamaks oder
Stellaratoren erschweren.

\subsubsection{Die WDBT als Alternative: Mikroskopische Fundierung und Selbstorganisation}
Die \gls{wdbt} löst diese Probleme, indem sie Plasmen nicht als Fluide, sondern als Systeme direkt wechselwirkender Teilchen beschreibt. Die Weber-Kraft (Gl. 2.2) berücksichtigt nicht
nur die Coulomb-Wechselwirkung, sondern auch geschwindigkeits- und beschleunigungsabhängige Terme, die in der \gls{mhd} fehlen. Dadurch erfasst sie kollektive Phänomene wie Plasmawellen oder
Turbulenz präziser. Besonders relevant ist das Bohm’sche Quantenpotential (Gl. 2.4), das nicht-lokale Korrelationen zwischen Teilchen beschreibt und in dichten Plasmen eine stabilisierende
Wirkung entfaltet. Experimente in Wendelstein 7-X zeigen bereits, dass Plasmen bei hohen Dichten ($n_e > 10^{20}m^{-3}$) stabiler sind als die \gls{mhd} vorhersagt – ein Effekt, den die \gls{wdbt}
durch den Quantenterm $Q$ natürlich erklärt.

\subsubsection{Praktische Vorteile: Kompaktere Reaktoren und effizientere Plasmen}
Die \gls{wdbt} bietet konkrete Vorteile für die Fusionsforschung:

\begin{enumerate}
    \item \textbf{Selbstorganisierte Stabilität:}\\Das Quantenpotential $Q$ wirkt wie eine intrinsische Dämpfung, die Instabilitäten wie Edge-Localized Modes (ELMs) unterdrücken kann. Dadurch könnten aufwendige Magnetfeldspulen teilweise überflüssig werden.
    \item \textbf{Reduzierter anomaler Transport:}\\Die Weber-Kraftdichte (Gl. 2.7) beschreibt den Teilchentransport durch Paarkorrelationen, nicht durch statistische Turbulenzmodelle. Dies könnte Energieverluste minimieren und die Einschlusszeiten verlängern.
    \item \textbf{Filamentäre Strukturen:}\\Die fraktale Skalierung von Birkeland-Strömen (Gl. 2.14) legt nahe, dass sich Plasmen in Fusionsreaktoren selbstorganisieren könnten – ähnlich wie in astrophysikalischen Phänomenen. Dies würde kompaktere Reaktordesigns ermöglichen.
\end{enumerate}

\subsubsection{Experimentelle Perspektiven}
Um das Potenzial der \gls{wdbt} auszuschöpfen, sind gezielte Experimente nötig:

\begin{itemize}
    \item \textbf{Quantenpotential-Effekte:}\\Hochdichte-Experimente (z. B. SPARC) könnten den Einfluss von $Q$ auf Plasmawellen direkt messen.
    \item \textbf{Nicht-lokaler Transport:}\\Präzise Messungen des anomalen Widerstands in Tokamaks könnten die Vorhersagen der \gls{wdbt} validieren.
    \item \textbf{Filamentbildung:}\\Laborexperimente mit Z-Pinch-Anordnungen sollten die fraktale Skalierung (Gl. 2.14) überprüfen.
\end{itemize}

\subsubsection{Fazit: Ein Paradigmenwechsel in der Fusionsforschung}
Die \gls{wdbt} bietet nicht nur eine theoretische Alternative zur \gls{mhd}, sondern auch praktische Lösungen für die hartnäckigsten Probleme der Fusionsforschung. Durch ihre mikroskopische Fundierung
und die Einbeziehung nicht-lokaler Quanteneffekte könnte sie den Weg zu stabileren, effizienteren Fusionsreaktoren ebnen – und damit die Vision einer sauberen, unerschöpflichen Energiequelle
Wirklichkeit werden lassen. Die experimentelle Validierung dieser Vorhersagen wird entscheiden, ob die \gls{wdbt} die Fusionsforschung tatsächlich in ein neues Zeitalter führen kann.

\subsection{Die Anwendungen: Von der Medizin zur Raumfahrt}
Die Konsequenzen dieser neuen Physik reichen weit über die Grundlagenforschung hinaus. In der Plasmamedizin, wo kalte Plasmen zur Wundheilung eingesetzt werden, könnte die
\gls{wed} erklären, warum bestimmte Plasma-Konfigurationen biologisch wirksamer sind als andere – nicht wegen der Feldstärke, sondern aufgrund der spezifischen,
nicht-lokalen Wechselwirkung mit Gewebemolekülen.

In der Raumfahrtantriebstechnik zeigen Plasmantriebe wie der VASIMR bereits heute, dass hohe spezifische Impulse möglich sind – doch ihre Effizienz bleibt hinter den theoretischen
Grenzen zurück. Die WDBT bietet hier einen neuen Ansatz: Wenn die Strahlbeschleunigung nicht durch Felder, sondern durch direkt wirkende Weber-Kräfte erfolgt, könnten völlig neue
Antriebskonzepte entstehen, die das Zeitalter der interplanetaren Raumfahrt einläuten.

\section{Hybrid-Plasmaantrieb: Thermoelektrische Resonanzexpansion}
\label{sec:hybrid_antrieb}

Die Kombination kryogener Treibstoffe mit Weber-De-Broglie-Bohm-Elektrodynamik (WDBT) führt zu einem neuartigen Antriebskonzept, das die Vorteile chemischer und elektrischer Systeme vereint.

\subsection{Physikalische Grundlagen}
\label{subsec:grundlagen}

Für ein flüssiges Ionengas mit Teilchendichte $n_e$ gilt die \textbf{erweiterte Zustandsgleichung}:

\begin{equation}
p = \underbrace{n_e k_B T_e}_{\text{thermisch}} 
+ \underbrace{\frac{e^2 n_e^{4/3}}{4\pi \epsilon_0} \left(1 + \beta \frac{v^2}{c^2}\right)}_{\text{WDBT-Korrektur}}
\label{eq:druck}
\end{equation}

mit $\beta = 2$ für die Weber-Kraft. Die \textbf{kritische Dichte} für Dominanz des Coulomb-Drucks liegt bei:

\begin{equation}
n_c = \left(\frac{4\pi \epsilon_0 k_B T_e}{e^2}\right)^3 \approx 10^{28}\,\text{m}^{-3}\quad\text{(für }T_e=10^4\,\text{K)}
\end{equation}

\subsection{Resonanzbedingungen}
\label{subsec:resonanz}

Das System verhält sich analog zu einem Helmholtz-Resonator mit\\\textbf{Plasma-Resonanzfrequenz}:

\begin{equation}
f_r = \frac{c_s}{2\pi}\sqrt{\frac{A_d}{V_c L_d}} \quad \text{mit} \quad c_s = \sqrt{\gamma \left(\frac{k_B T_e}{m_i} + \frac{\hbar^2}{4m_e m_i}\frac{\nabla^2 n_e}{n_e}\right)}
\label{eq:resonanz}
\end{equation}

\subsection{Energietransferanalyse}
\label{subsec:energie}

Die \textbf{Energiedichteskalierung} zeigt den WDBT-Vorteil:

\begin{table}[h]
\centering
\caption{Vergleich der Energiedichten}
\label{tab:energie}
\begin{tabular}{lcc}
\toprule
Treibstofftyp & $E$ [MJ/kg] & $p_{\text{max}}$ [GPa] \\
\midrule
TNT & 4.6 & 20 \\
Flüssiger Wasserstoff & 142 & 25 \\
WDBT-Plasma (LH$_2$) & 175 & 175 \\
\bottomrule
\end{tabular}
\end{table}

\subsection{Technische Umsetzung}
\label{subsec:tech}

Die \textbf{optimale Düsengeometrie} folgt der fraktalen Skalierung:

\begin{equation}
\frac{dA}{dx} = -A^{1-1/D} \quad \text{mit} \quad D = \frac{\ln 20}{\ln(2+\phi)} \approx 2.71
\label{eq:duese}
\end{equation}

Die Stabilitätsbedingung für den \textbf{Quanten-Federeffekt} lautet:

\begin{equation}
\tau_{\text{ion}} > \sqrt{\frac{m_e}{e^2 n_e^{2/3}}} \approx 10^{-11}\,\text{s}\quad\text{(für }n_e=10^{28}\,\text{m}^{-3)}
\end{equation}

\begin{remark}
Die magnetische Steuerung erfolgt durch ein \textbf{radiales $B$-Feld} mit:
\[
B > \frac{m_i v_{\text{exp}}}{e r_d} \approx 0.5\,\text{T}\quad\text{(für }r_d=1\,\text{cm)}
\]
\end{remark}

\subsection{Experimentelle Validierung}
\label{subsec:experiment}

Messgrößen zur Bestätigung der WDBT-Effekte:

\begin{itemize}
\item \textbf{Expansionsgeschwindigkeit}:
\[
\frac{\Delta v}{v_{\text{klassisch}}} = \sqrt{1 + \frac{Q}{k_B T_e}} - 1
\]

\item \textbf{Spektrale Dichtemodulation}:
\[
\left.\frac{\delta n_e}{n_e}\right|_{\text{res}} \propto \frac{\hbar}{m_e c_s^2 \tau_{\text{ion}}}
\]
\end{itemize}

\subsection*{Zusammenfassung}
Das Konzept kombiniert erstmals:
\begin{enumerate}
\item Kryogene Energiespeicherung,
\item Elektrostatische Druckverstärkung,
\item Nicht-lineare WDBT-Resonanz.
\end{enumerate}

\subsection{Das Prinzip des Hybrid-Plasmaantriebs}
Die Idee eines Antriebssystems, das die Vorteile chemischer Expansion und elektrostatischer Plasmabeschleunigung vereint, basiert auf einem tiefen Verständnis der Wechselwirkungen zwischen kryogener
Materie und Quantenpotentialen. Stellen Sie sich einen extrem komprimierten flüssigen Wasserstofftank vor, der schlagartig ionisiert wird. Durch die Ionisation entstehen zwei simultane Effekte: Erstens
die klassische thermische Expansion des nun heißen Plasmas, zweitens eine viel stärkere elektrostatische Abstoßung der Ionen untereinander. Diese Coulomb-Explosion wird in der \gls{wdbt} durch die
geschwindigkeitsabhängige Weber-Kraft noch verstärkt – ähnlich wie eine Feder, die nicht nur durch ihre Spannung, sondern zusätzlich durch resonante Schwingungen Energie freisetzt.

Der Schlüssel zur Kontrolle dieses Systems liegt in der präzisen Abstimmung der Resonanzbedingungen. Wie bei einem perfekt konstruierten Bassreflex-Lautsprecher muss das Verhältnis von Kammervolumen
zur Düsengeometrie so gewählt werden, dass die natürliche Schwingungsfrequenz des Plasmas mit der Ionisationsrate synchronisiert ist. Das Quantenpotential Q wirkt hierbei als aktiver Dämpfer, der
chaotische Turbulenzen unterdrückt und die Energie in eine kohärente Expansionswelle umlenkt. Praktisch erreicht man dies durch eine fraktale Düsenform, deren Verzweigungsmuster
(Skalierungsexponent $D \approx 2.71$) genau der nicht-lokalen Korrelationslänge des Plasmas entspricht.

Die daraus resultierende Schubkraft übertrifft konventionelle Systeme durch einen einzigartigen Mechanismus: Während chemische Triebwerke durch die Bindungsenergie von Molekülen begrenzt sind und
elektrische Antriebe durch magnetische Sättigungseffekte, nutzt dieser Hybridantrieb die kollektive Quantennatur des Plasmas selbst. Die Ionen beschleunigen nicht isoliert, sondern als kohärentes
Ganzes, dessen Dynamik durch das Bohm'sche Potential gesteuert wird. Magnetfelder dienen dabei nur noch zur Feinjustierung der Ausbreitungsrichtung, nicht mehr zur primären Energieübertragung.

Experimentell manifestiert sich dieser Effekt in charakteristischen Signalen: Eine um 20-30\% erhöhte Expansionsgeschwindigkeit gegenüber klassischen Vorhersagen, sowie typische Dichtemodulationen
im Ultraschallbereich (50-100 kHz), die direkt mit der fraktalen Dimension $D$ korrelieren. Die technische Umsetzung erfordert zwar präzise Steuerung der Ionisationsfront (Nanosekunden-Laserpulse),
ermöglicht aber kompaktere Bauformen als herkömmliche Plasmatriebwerke – bei gleichzeitig höherem spezifischem Impuls.

Diese Synergie aus kryogener Speicherung, elektrostatischer Explosion und Quantenkohärenz markiert einen Paradigmenwechsel in der Antriebstechnik, der nur durch die \gls{wdbt} vollständig erklärbar
ist. Sie zeigt, wie scheinbar getrennte physikalische Prinzipien in Wirklichkeit Aspekte einer tieferen, einheitlichen Beschreibung sind – jenseits der klassischen Feldtheorien.

\subsubsection{Der Ionisationsantrieb: Eine Alternative zur klassischen Verbrennung}
Im Gegensatz zu herkömmlichen Verbrennungsprozessen, bei denen chemische Reaktionen wie die Oxidation von Wasserstoff genutzt werden, setzt der hier beschriebene Antrieb ausschließlich auf
Ionisation – also die Umwandlung von neutralen Gasatomen oder -molekülen in geladene Teilchen (Plasma). Während eine Verbrennung Energie durch die Umwandlung von Molekülbindungen freisetzt, beruht der
Ionisationsantrieb auf elektrodynamischen und quantenmechanischen Effekten.

\textbf{Schlüsselunterschiede:}
\begin{enumerate}
    \item \textbf{Keine chemische Reaktion nötig}
        \begin{itemize}
            \item Herkömmliche Triebwerke benötigen einen Oxidator (z. B. Sauerstoff), um den Treibstoff zu verbrennen.
            \item Beim Ionisationsantrieb wird das Gas (z. B. Wasserstoff) durch elektrische oder laserinduzierte Ionisation direkt in Plasma umgewandelt – ohne Flamme oder chemische Reaktionsprodukte.
        \end{itemize}
    \item \textbf{Energiefreisetzung durch Coulomb-Explosion}
        \begin{itemize}
            \item Beim Ionisieren entstehen positiv geladene Ionen, die sich gegenseitig abstoßen.
            \item Diese elektrostatische Abstoßung erzeugt einen extrem schnellen Expansionsdruck – viel stärker als bei thermischer Verbrennung.
        \end{itemize}
    \item \textbf{Quantenmechanische Stabilisierung}
        \begin{itemize}
            \item Das Bohm’sche Quantenpotential ($Q$) verhindert, dass das Plasma instabil wird oder unkontrolliert expandiert.
            \item Dadurch lässt sich die Energie gezielt in Schub umwandeln, statt in eine ungerichtete Druckwelle.
        \end{itemize}
\end{enumerate}

\textbf{Vorteile gegenüber Verbrennung}
\begin{itemize}
    \item \textbf{Höhere Effizienz:}\\Die Coulomb-Abstoßung kann mehr Energie pro Kilogramm Treibstoff freisetzen als chemische Reaktionen.
    \item \textbf{Sauberer Betrieb:}\\Keine Verbrennungsrückstände (nur ionisierte Teilchen, die im Vakuum neutralisiert werden).
    \item \textbf{Präzise Steuerung:}\\Die Expansion kann durch Magnetfelder oder das Quantenpotential gesteuert werden.
    \item \textbf{Gewichtsreduktion:}\\Es muss kein Sauerstoff für die Verbrennung mitgeführt werden.
\end{itemize}

Es handelt sich hier nicht um eine Verbrennung, sondern um einen elektrodynamisch getriebenen Prozess, der Plasmen nutzt, um Schub zu erzeugen. Diese Methode könnte Antriebssysteme
revolutionieren – von Raumschiffen bis hin zu neuen Energieumwandlungskonzepten.

\textbf{Zusammenfassend:} \textit{Ionisation ersetzt die Flamme – und Quantenphysik sorgt für die Kontrolle.}

\section{Eine neue Ära der Physik}
Dieses Buch wird zeigen, dass die Vereinigung von \gls{wed}, \gls{dbt} und Plasmaphysik mehr ist als eine akademische Übung – es ist der Schlüssel zu
einem neuen Verständnis des Universums. Von den größten kosmischen Strukturen bis hin zur Kontrolle von Fusionsplasmen eröffnet sich eine Welt jenseits der Quantenfelder, in der
die Natur nicht durch abstrakte Feldgleichungen, sondern durch reale, messbare Wechselwirkungen beschrieben wird.

Die kommenden Kapitel werden diese Vision mit mathematischer Strenge und experimentellen Belegen untermauern. Die Reise beginnt mit den Grundlagen – einer feldlosen Beschreibung
der Plasmadynamik, die zeigt, warum die \gls{wdbt} nicht nur eine Alternative, sondern die logisch konsistentere Theorie ist.

\chapter{Das Sonnenmodell im WBFM-Rahmen}

\section{Die Sonne als aktiver Quantenfilter}

Im Weber-Bohm-Filter-Modell (WBFM) wird die Sonne als ein hochkomplexer, aktiver Filterknoten verstanden, dessen Eigenschaften durch eine nicht-lineare, nicht-lokale Wellengleichung beschrieben werden. Die Wellenfunktion \(\Psi_S(r, \theta, \phi, t) = R_S e^{iS_S/\hbar}\) kodiert dabei sowohl die dynamischen als auch die strukturellen Eigenschaften unseres Zentralsterns.

\section{Energie-Materie-Transformation im solaren Kern}

Der Sonnenkern fungiert als primäre Filterstufe, wo durch nicht-lineare Wechselwirkungen im Quantenpotential \(Q\) Energie in Materie transformiert wird:

\[
E_{\text{fusion}} \rightarrow \eta Q \rightarrow \Delta m c^2
\]

wobei \(\eta\) der Kopplungsparameter zwischen Fusionsenergie und Quantenpotential darstellt. Dieser Prozess führt zu einer messbaren Anreicherung leichter Isotope im Sonnenwind.

\section{Phasenstruktur und Zonierung}

Die verschiedenen solaren Zonen entsprechen charakteristischen Bereichen der Wellenfunktion:

\begin{itemize}
\item \textbf{Kernzone}: Region maximaler Phasenkrümmung (\(\nabla^2 S_S > 0\)) mit dominanter Materiegenerierung
\item \textbf{Strahlungszone}: Bereich linearer Phasenentwicklung mit \(\nabla S_S \approx \text{const}\)
\item \textbf{Tachocline}: Phasensprungstelle mit \(\Delta(\nabla S_S) \neq 0\) für differentielle Rotation
\item \textbf{Konvektionszone}: Bereich chaotischer Phasenfluktuationen mit \(\partial_t S_S \sim \text{turbulent}\)
\item \textbf{Photosphäre}: Wellenfunktions-Knotenfläche mit \(R_S \approx 0\)
\item \textbf{Korona}: Auskopplungsregion mit \(\nabla S_S \rightarrow m_p v_{\text{wind}}\)
\end{itemize}

\section{Transferfunktion des solaren Filters}

Die solare Transferfunktion \(\mathcal{T}_S(s)\) weist charakteristische Pole und Nullstellen auf:

\[
\mathcal{T}_S(s) = G \frac{(s - z_1)(s - z_2)\cdots}{(s - p_1)(s - p_2)\cdots}
\]

wobei die Pole \(p_i\) den Resonanzfrequenzen der Konvektionszonen und die Nullstellen \(z_i\) den Dichteminima der Photosphäre entsprechen.

\section{Numerische Implementierung}

Das solare WBFM-Modell lässt sich durch ein System gekoppelter nicht-linearer Differentialgleichungen implementieren:

\begin{align*}
\frac{\partial R_S}{\partial t} &= -\frac{1}{2m_p}\left(R_S \nabla^2 S_S + 2\nabla R_S \cdot \nabla S_S\right) \\
\frac{\partial S_S}{\partial t} &= -\left(\frac{|\nabla S_S|^2}{2m_p} + V + Q + U_{\text{WG}}\right)
\end{align*}

mit \(Q = -\frac{\hbar^2}{2m_p}\frac{\nabla^2 R_S}{R_S}\) und \(U_{\text{WG}}\) dem Weber-Gravitationspotential.

\section{Testbare Vorhersagen}

Das Modell sagt vorher:
\begin{enumerate}
\item Eine fraktale Skalierung der Sonnenwinddichte mit \(\rho(r) \propto r^{D-3}\)
\item Spezifische Isotopenanomalien im Sonnenwind (\(^3\text{He}/^4\text{He}\), \(^7\text{Li}/^6\text{Li}\))
\item Resonanzfrequenzen in der Helioseismologie bei \(\omega = \text{Im}(p_i)\)
\item Nicht-standard Skalierung der Koronatemperatur mit \(T \propto |Q|^{2/3}\)
\end{enumerate}

\section{Konkretes Sonnenmodell im WBFM}

\subsection{Parameterisierung des solaren Filterknotens}

Basierend auf beobachtbaren Sonnendaten lässt sich das WBFM-Modell konkret parametrisieren:

\begin{align*}
\text{Sternklasse:} &\quad G2V \\
\text{Masse:} &\quad M_\odot = 1.989 \times 10^{30}  \text{kg} \\
\text{Radius:} &\quad R_\odot = 6.957 \times 10^8  \text{m} \\
\text{Korona-Temperatur:} &\quad T_c = 1.5-2.0 \times 10^6  \text{K} \\
\text{Sonnenwind (1 AE):} &\quad v_{sw} = 400-800  \text{km/s} \\
&\quad n_{sw} = 5-10  \text{cm}^{-3} \\
\text{Heliosphärenradius:} &\quad R_H \approx 120  \text{AE}
\end{align*}

\subsection{Wellenfunktions-Parameter}

Die solare Wellenfunktion $\Psi_S(r) = R_S(r)e^{iS_S(r)/\hbar}$ zeigt charakteristische Skalierung:

\[
R_S(r) \propto r^{-\alpha} e^{-r/\lambda_Q}
\]
mit $\alpha \approx 0.32$ (entsprechend $D-2 \approx 0.71$) und $\lambda_Q \approx 0.1 R_\odot$ als Quantenpotential-Länge.

Die Phase $S_S(r)$ folgt:
\[
\frac{dS_S}{dr} = m_p v_{sw} \left(1 + \beta \ln\frac{r}{R_\odot}\right)
\]
mit $\beta \approx 0.1$ für die beobachtete Beschleunigung des Sonnenwinds.

\subsection{Quantenpotential und Temperatur}

Die Korrelation zwischen $Q$ und Temperatur:
\[
k_B T(r) \approx \frac{\hbar^2}{2m_p} \left|\frac{\nabla^2 R_S}{R_S}\right| \approx 100  \text{eV} \left(\frac{R_\odot}{r}\right)^{0.4}
\]

\subsection{Materieerzeugungsrate}

Die Rate der Materiegenerierung im Kern:
\[
\frac{dM}{dt} \approx \eta \frac{L_\odot}{c^2} \approx 2 \times 10^9  \text{kg/s}
\]
mit $\eta \approx 0.001$, konsistent mit beobachteter Sonnenwind-Massenverlustrate.

\subsection{Heliosphären-Randbedingung}

Am Heliopause ($r = R_H$) gilt:
\[
\frac{dS_S}{dr}\Big|_{r=R_H} = 0, \quad R_S(R_H) \propto R_H^{-0.29}
\]

\subsection{Transferfunktion des solaren Filters}

\[
\mathcal{T}_S(s) = \frac{(s + \gamma_1)(s + \gamma_2)}{(s + \Gamma_1)(s + \Gamma_2)(s + \Gamma_3)}
\]
mit:
\begin{align*}
\gamma_1 &\approx 10^{-3}  \text{s}^{-1} \quad \text{(Konvektionszone)} \\
\gamma_2 &\approx 10^{-2}  \text{s}^{-1} \quad \text{(Tachocline)} \\
\Gamma_1 &\approx 10^{-6}  \text{s}^{-1} \quad \text{(Kernfusion)} \\
\Gamma_2 &\approx 10^{-4}  \text{s}^{-1} \quad \text{(Strahlungszone)} \\
\Gamma_3 &\approx 10^{-1}  \text{s}^{-1} \quad \text{(Korona)}
\end{align*}

\subsection{Vorhersagen und Verifikation}

Das Modell sagt konkret vorher:
\begin{itemize}
\item $^3$He/$^4$He-Verhältnis im Sonnenwind: $4.5 \times 10^{-4}$ (vs. $3.0 \times 10^{-4}$ im ISM)
\item Fraktale Dimension des Sonnenwinds: $D = 2.71 \pm 0.01$
\item Charakteristische Frequenzen in Helioseismologie: 0.3 mHz, 2.8 mHz, 5.0 mHz
\end{itemize}

\section{Wellenwiderstand im WBFM: Die Impedanz des Quantenvakuums}

\subsection{Definition der kosmischen Impedanz}

Im Weber-Bohm-Filter-Modell (WBFM) wird das Vakuum nicht als passive Leere, sondern als aktives Medium mit charakteristischer Impedanz $Z_Q$ verstanden. Diese quantenmechanische Impedanz beschreibt den Widerstand, den das Vakuum der Anregung durch Materie und Energie entgegensetzt:

\[
Z_Q = \sqrt{\frac{\mu_Q}{\epsilon_Q}} = \frac{h}{e^2} \alpha^{-1} \approx 4.8 \times 10^3  \Omega
\]

wobei $\mu_Q$ und $\epsilon_Q$ die permeativen Eigenschaften des Quantenvakuums beschreiben und $\alpha$ die Feinstrukturkonstante ist.

\subsection{Impedanzanpassung im solaren Filter}

Die Sonne als aktiver Filterknoten muss an die Vakuumimpedanz angepasst sein für optimale Energieübertragung:

\[
Z_S(r) = Z_Q \left(\frac{r}{R_\odot}\right)^{D-2}
\]

Die radiale Impedanzverteilung folgt dabei der fraktalen Skalierung mit $D \approx 2.71$.

\subsection{Wellenwiderstand und Quantenpotential}

Der Zusammenhang zwischen Impedanz und Quantenpotential wird durch:

\[
Q(r) = \frac{\hbar^2}{2m} \frac{Z_Q^2}{Z_S^2(r)} \left|\frac{\nabla \rho}{\rho}\right|^2
\]

Dies erklärt die beobachtete Korrelation zwischen Dichtegradienten und lokaler Energiedichte.

\subsection{Impedanzsprünge an Phasengrenzen}

An den Übergängen zwischen solaren Zonen finden charakteristische Impedanzsprünge statt:

\begin{align*}
\text{Kern/Strahlungszone:} &\quad \Delta Z \approx +12\% \\
\text{Tachocline:} &\quad \Delta Z \approx -8\% \\
\text{Konvektionszone/Photosphäre:} &\quad \Delta Z \approx +23\% \\
\text{Photosphäre/Korona:} &\quad \Delta Z \approx +180\%
\end{align*}

Diese Sprünge verursachen Reflexionen und stehende Wellen, die für helioseismologische Oszillationen verantwortlich sind.

\subsection{Energieübertragung und Wirkungsgrad}

Der Wirkungsgrad der Energieübertragung vom Kern zur Heliosphäre folgt:

\[
\eta(r) = 1 - \left|\frac{Z_S(r) - Z_Q}{Z_S(r) + Z_Q}\right|^2
\]

Mit $\eta(R_H) \approx 0.98$ an der Heliopause.

\subsection{Messbare Konsequenzen}

\begin{itemize}
\item Charakteristische Impedanz-Mismatch-Oszillationen bei $f = 3.2  \text{mHz}$
- Reflektierte Leistung an der Heliopause: $P_{\text{refl}} \approx 0.02 L_\odot$
- Typische Stehwellenverhältnisse: $\text{SWR} \approx 1.5-2.0$ im Sonnenwind
\end{itemize}

\subsection{Vergleich mit elektromagnetischer Impedanz}

Die Vakuumimpedanz $Z_0 = \sqrt{\mu_0/\epsilon_0} \approx 377  \Omega$ beschreibt die elektromagnetische Kopplung, während $Z_Q \approx 4.8  \text{k}\Omega$ die materielle Kopplung an das Quantenvakuum beschreibt. Das Verhältnis:

\[
\frac{Z_Q}{Z_0} = \frac{1}{\alpha} \approx 137
\]

entspricht genau dem Kehrwert der Feinstrukturkonstanten.

\chapter{Quantenfeldtheoretische Erweiterung und kosmologische Konsistenz des WBFM}

\section{Von der Ein-Teilchen- zur Vielteilchen-Wellenfunktion}

Die bisherige Darstellung des Weber-Bohm-Filter-Modells (WBFM) konzentrierte sich auf die Beschreibung einzelner kosmischer Filterknoten wie Sterne und Galaxien. In diesem Kapitel erfolgt der Übergang zu einer \textit{quantenfeldtheoretischen Formulierung}, die das Universum als Ganzes beschreibt. Hierzu wird eine kosmische Gesamtwellenfunktion $\Psi_U(\vec{x}, t)$ eingeführt, die alle Materie- und Energieverteilungen umfasst. Diese Funktion erfüllt eine erweiterte WDBT-Gleichung:

\[
i\hbar \frac{\partial \Psi_U}{\partial t} = \left( -\sum_i \frac{\hbar^2}{2m_i} \nabla_i^2 + V_{\text{WG}} + Q_U \right) \Psi_U
\]

wobei $Q_U$ das universelle Quantenpotential bezeichnet und $V_{\text{WG}}$ das\\Weber-Gravitationspotential darstellt.

\section{Quantenfeldtheorie des Weber-Bohm-Vakuums}

Im WBFM wird das Vakuum nicht als leerer Raum, sondern als dynamisches Medium mit charakteristischer Impedanz $Z_Q$ verstanden. Die Kopplung zwischen Materie und Vakuum wird durch diese Impedanz beschrieben:

\[
Z_Q = \sqrt{\frac{\mu_Q}{\epsilon_Q}} = \frac{h}{e^2} \alpha^{-1} \approx 4.8 \times 10^3 \, \Omega
\]

Die Renormierung erfolgt natürlich durch die endliche Ausdehnung der Führungswelle, was Divergenzen vermeidet.

\section{Kosmologische Wellenfunktion und Strukturbildung}

Die fraktale Struktur des Universums mit $D \approx 2.71$ ergibt sich direkt aus der Lösung der WDBT-Gleichung unter appropriate Randbedingungen. Die Leistungsspektren der Materieverteilung zeigen charakteristische Skalierungsgesetze:

\[
P(k) \propto k^{-(3-D)} \approx k^{-0.29}
\]

\section{Dunkle Energie als Impedanzmismatch im Kosmos}

Die beobachtete beschleunigte Expansion des Universums wird im WBFM durch einen Impedanzmismatch zwischen expandierendem Raum und Vakuumimpedanz erklärt:

\[
\Lambda \sim \frac{1}{Z_Q^2} \left( \frac{dZ}{dt} \right)^2
\]

\section{Quantengravitation ohne Singularitäten}

Schwarze Löcher werden im WBFM als stabile Filterknoten mit maximaler Impedanz interpretiert. Die Quantenpotential-Barriere verhindert Singularitäten:

\[
Q(r) \sim \frac{\hbar^2}{2m} \frac{1}{r^2} \quad \text{für } r \to 0
\]

\section{Testbare Vorhersagen auf kosmologischen Skalen}

Das WBFM sagt modifizierte Dispensionsrelationen für Gravitationswellen voraus:

\[
v_g(f) = c \left( 1 + \alpha \left( \frac{f}{f_P} \right)^{D-3} \right)
\]

sowie frequenzabhängige Lichtlaufzeiten bei Gravitationslinsen.

\section{Numerische Implementierung und Simulation}
Die Implementierung des WBFM erfordert gitterbasierte Berechnungen des Quantenpotentials $Q_U$ in einem expandierenden Universum. Vergleiche mit
Standard-$\Lambda$CDM-Simulationen zeigen charakteristische Abweichungen in der Large-Scale Structure.

\section{Philosophische und methodologische Konsequenzen}
Das WBFM vertritt eine ontologische Reduktion physikalischer Entitäten und ein Prinzip der maximalen Empirie. Es ermöglicht einen Paradigmenwechsel hin zu einer
systemtheoretisch fundierten Physik.

\section{Vereinheitlichte Bewegungsgleichung im WBFM}
Die Bewegungsgleichung für ein Teilchen $i$ mit Masse $m_i$ und Ladung $q_i$ in einem System aus $N$ wechselwirkenden Teilchen setzt sich aus folgenden Komponenten zusammen:

\subsection*{1. Weber-Elektrodynamische Kraft}
Die WED-Kraft auf Teilchen $i$ durch Teilchen $j$ ist gegeben durch:
\[
\vec{F}_{ij}^{\mathrm{WED}} = \frac{q_i q_j}{4\pi\epsilon_0 r_{ij}^2} 
\left\{ 
\left[1 - \frac{v_{ij}^2}{c^2} + \frac{2 r_{ij} (\hat{r}_{ij} \cdot \vec{a}_j)}{c^2}\right] \hat{r}_{ij} 
+ \frac{2 (\hat{r}_{ij} \cdot \vec{v}_j)}{c^2} \vec{v}_j 
\right\}
\]
wobei $\vec{r}_{ij} = \vec{x}_i - \vec{x}_j$, $r_{ij} = |\vec{r}_{ij}|$, $\hat{r}_{ij} = \vec{r}_{ij}/r_{ij}$.

\subsection*{2. Weber-Gravitationskraft}
Die WG-Kraft auf Teilchen $i$ durch Teilchen $j$ lautet:
\[
\vec{F}_{ij}^{\mathrm{WG}} = -\frac{G m_i m_j}{r_{ij}^2} 
\left(1 - \frac{\dot{r}_{ij}^2}{c^2} + \beta \frac{r_{ij} \ddot{r}_{ij}}{c^2}\right) \hat{r}_{ij}
\]
mit $\beta = 0.5$ für Massen und $\beta = 1.0$ für Photonen.

\subsection*{3. Quantenpotential-Kraft}
Die Kraft aus dem Quantenpotential ist:
\[
\vec{F}_{i}^{Q} = -\vec{\nabla}_i Q = \frac{\hbar^2}{2m_i} \vec{\nabla}_i \left( \frac{\nabla_i^2 R}{R} \right)
\]
wobei $R$ die Amplitude der Gesamtwellenfunktion $\Psi_U = R e^{iS/\hbar}$ ist.

\subsection*{4. Vollständige Bewegungsgleichung}
Die vereinheitlichte Bewegungsgleichung für das Teilchen $i$ ergibt sich zu:
\[
\boxed{
m_i \frac{d^2 \vec{x}_i}{dt^2} = 
\sum_{j \neq i}^N \left( \vec{F}_{ij}^{\mathrm{WED}} + \vec{F}_{ij}^{\mathrm{WG}} \right) 
+ \frac{\hbar^2}{2m_i} \vec{\nabla}_i \left( \frac{\nabla_i^2 R}{R} \right)
}
\]

\subsection*{Anmerkungen zur Interpretation}
\begin{itemize}
\item Die Gleichung ist \textit{nicht-lokal}, da sowohl die Weber-Kräfte als auch das Quantenpotential von instantanen Wechselwirkungen abhängen
\item Die Bewegung des Teilchens $i$ hängt von der Gesamtwellenfunktion $\Psi_U$ ab, die wiederum von den Positionen aller Teilchen abhängt
\item Es handelt sich um eine Differentialgleichung \textit{dritter Ordnung} aufgrund der Beschleunigungsterme in den Weber-Kräften
\item Die Gleichung verletzt die lokale Lorentz-Invarianz, was konsistent mit der postulierten fundamentalen Nicht-Lokalität des WBFM ist
\end{itemize}

\section{Hierarchische Skalenentkopplung im WBFM}

Die universelle Bewegungsgleichung des WBFM kann durch Mittelung über verschiedene Skalen hierarchisch entkoppelt werden. Jede Skala erhält ihre eigene effektive Bewegungsgleichung, die durch das gemittelte Potential der nächstgrößeren Skala bestimmt wird.

\subsection{Kosmische Skala: Bewegung einer Galaxie}

Auf der kosmischen Skala wird eine Galaxie als Punktmasse $M_{\text{gal}}$ mit Schwerpunkt $\vec{X}_{\text{gal}}$ behandelt. Ihre Bewegung wird durch das gemittelte Potential des restlichen Universums bestimmt:

\begin{equation}
M_{\text{gal}} \frac{d^2 \vec{X}_{\text{gal}}}{dt^2} \approx -\vec{\nabla}_{\vec{X}_{\text{gal}}} \Phi_{\text{eff}}^{\text{cosmic}}
\end{equation}

Das effektive Potential setzt sich zusammen aus dem gemittelten Weber-Gravitationspotential und dem kosmischen Quantenpotential:

\begin{equation}
\Phi_{\text{eff}}^{\text{cosmic}} \approx G \int \frac{\bar{\rho}_{\text{univ}}(\vec{r}', t)}{r'} \left(1 - \frac{\dot{r}'^2}{c^2} + \beta \frac{r' \ddot{r}'}{c^2}\right) d^3r' + \langle Q \rangle_{\text{cosmic}}
\end{equation}

Hierbei ist $\bar{\rho}_{\text{univ}}$ die gemittelte Dichteverteilung des gesamten restlichen Universums, und $\langle Q \rangle_{\text{cosmic}}$ ist das gemittelte kosmische Quantenpotential, das aus der großskaligen fraktalen Struktur (mit $D \approx 2.71$) emergiert.

\subsection{Galaktische Skala: Bewegung eines Sterns}

Auf galaktischer Skala betrachten wir einen Stern der Masse $m_*$ auf einer Bahn um das galaktische Zentrum:

\begin{equation}
m_* \frac{d^2 \vec{x}_*}{dt^2} \approx -\vec{\nabla}_{\vec{x}_*} \Phi_{\text{eff}}^{\text{gal}} - \vec{\nabla} Q_{\text{gal}}
\end{equation}

Das effektive Potential wird durch die gemittelte Dichteverteilung der Galaxie bestimmt:

\begin{equation}
\Phi_{\text{eff}}^{\text{gal}} \approx G \int \frac{\bar{\rho}_{\text{gal}}(\vec{r}', t)}{r'^2} \left(1 - \frac{\dot{r}'^2}{c^2} + \beta \frac{r' \ddot{r}'}{c^2}\right) \hat{r}'  d^3r'
\end{equation}

Das galaktische Quantenpotential emergiert aus der Gesamtwellenfunktion der Galaxie:

\begin{equation}
Q_{\text{gal}} \approx -\frac{\hbar^2}{2 m_*} \frac{\nabla^2 \sqrt{\bar{\rho}_{\text{gal}}}}{\sqrt{\bar{\rho}_{\text{gal}}}} 
\end{equation}

\subsection{Stellare Skala: Bewegung eines Planeten}

Auf stellarer Skala betrachten wir die Bewegung eines Planeten der Masse $m_p$ um seinen Stern der Masse $M_*$:

\begin{equation}
m_p \frac{d^2 \vec{x}_p}{dt^2} \approx -\vec{\nabla} \left( \frac{G M_* m_p}{r} \left(1 - \frac{\dot{r}^2}{c^2} + \beta \frac{r \ddot{r}}{c^2}\right) \right) - \vec{\nabla} Q_{*}
\end{equation}

Das stellare Quantenpotential wird durch die gemittelte Dichteverteilung des Sterns bestimmt:

\begin{equation}
Q_{*} \approx -\frac{\hbar^2}{2 m_p} \frac{\nabla^2 \sqrt{\bar{\rho}_{*}}}{\sqrt{\bar{\rho}_{*}}}
\end{equation}

\subsection{Planetare Skala: Bewegung eines Atoms}

Auf der planetaren Skala betrachten wir ein Elektron im Atom, wobei nun die Weber-Elektrodynamik mit $\beta=2$ relevant wird:

\begin{equation}
m_e \frac{d^2 \vec{x}_e}{dt^2} \approx -\vec{\nabla} \left( \frac{k e^2}{r} \left(1 - \frac{\dot{r}^2}{c^2} + 2 \frac{r \ddot{r}}{c^2}\right) \right) - \vec{\nabla} Q_{\text{atom}}
\end{equation}

Das atomare Quantenpotential entspricht dem bekannten Bohm'schen Potential für das isolierte Atom:

\begin{equation}
Q_{\text{atom}} = -\frac{\hbar^2}{2 m_e} \frac{\nabla^2 |\psi_{\text{atom}}|}{|\psi_{\text{atom}}|}
\end{equation}

\subsection{Zusammenhang zwischen den Skalen}

Die verschiedenen Skalen sind hierarchisch gekoppelt:
\begin{itemize}
\item Die kosmische Skala liefert Randbedingungen und das umgebende Potential für die galaktische Skala
\item Die galaktische Skala bestimmt die Dichteverteilung $\bar{\rho}_{\text{gal}}$ für das Potential $\Phi_{\text{eff}}^{\text{gal}}$
\item Die stellare Skala liefert die Dichteverteilung $\bar{\rho}_{*}$ für $Q_{*}$
\item Die planetare Skala wird von den darüberliegenden Skalen eingehüllt, deren Einfluss vernachlässigbar ist
\end{itemize}

Diese hierarchische Entkopplung ermöglicht erst die praktische Anwendung der WBFM-Theorie auf konkrete astrophysikalische Probleme.

\appendix
\chapter{Anhang}
\section{Der Aharonov-Bohm-Effekt}
\label{sec:aharonov-bohm}

Der \textbf{Aharonov-Bohm-Effekt} (AB-Effekt) ist ein grundlegendes Quantenphänomen, das zeigt, dass elektromagnetische Potentiale ($\vec{A}$, $\Phi$) eine direkte physikalische
Wirkung auf Quantenteilchen haben, selbst in Regionen wo die Felder ($\vec{E}$, $\vec{B}$) null sind.

\subsection{Experimentelle Anordnung}
Ein Elektronenstrahl wird in zwei Pfade aufgeteilt, die eine Region mit magnetischem Fluss $\Phi$ umschließen.

\subsection{Theoretische Beschreibung}
Die Wellenfunktion $\psi$ eines Teilchens mit Ladung $q$ wird durch das Vektorpotential $\vec{A}$ modifiziert:

\begin{equation}
\psi \rightarrow \psi \cdot \exp\left(i\frac{q}{\hbar}\int \vec{A}\cdot d\vec{l}\right)
\end{equation}

Die Phasendifferenz zwischen den beiden Pfaden beträgt:

\begin{equation}
\Delta\phi = \frac{q}{\hbar}\oint \vec{A}\cdot d\vec{l} = \frac{q}{\hbar}\Phi_B
\end{equation}

\subsection{Physikalische Bedeutung}
\begin{itemize}
\item \textbf{Nicht-Lokalität}: Quantenteilchen \enquote{spüren} $\vec{A}$ auch in feldfreien Regionen
\item \textbf{Topologische Invariante}: Die Phase hängt nur vom eingeschlossenen Fluss $\Phi_B$ ab
\item \textbf{Paradigmenwechsel}: Widerlegt die klassische Annahme, dass nur $\vec{E}$ und $\vec{B}$ physikalisch relevant sind
\end{itemize}

\subsection{Experimentelle Bestätigung}
\begin{itemize}
\item Theoretische Vorhersage: Aharonov \& Bohm (1959)
\item Erste Experimente: Chambers (1960), Tonomura et al. (1982)
\item Moderne Anwendungen: Quanteninterferometer, topologische Quantenmaterialien
\end{itemize}


\backmatter
\printbibliography[title=Literaturverzeichnis]
\glswritefiles
\printglossary[title=Glossar]
\printglossary[type=acronym, title=Abkürzungen]

\end{document}
