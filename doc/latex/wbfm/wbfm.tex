\documentclass[11pt, a5paper, twoside, openright]{book}
\usepackage[ngerman]{babel}
\usepackage[T1]{fontenc}
\usepackage[utf8]{inputenc}
\usepackage{lmodern}
\usepackage{microtype}
\usepackage{csquotes}
\usepackage{verbatim}  % Im Kopf des Dokuments einfügen
\usepackage{geometry}
\usepackage{fancyhdr}
\usepackage{amsmath, amssymb, amsthm}  % Mathe
\usepackage{mathtools}                 % \coloneqq, \xrightarrow
\usepackage{bm}                        % Fette Symbole (\bm{B} für Magnetfeld)
\usepackage{siunitx}                   % \SI{1.23}{\meter\per\second}
\usepackage{graphicx}                  % \includegraphics
\usepackage{subcaption}                % Unterabbildungen
\usepackage{booktabs}                  % Professionelle Tabellen
\usepackage{tikz}                      % Für Diagramme
\usepackage{xcolor}                    % Farbige Tabellenzellen
\usepackage[
    backend=biber,
    style=phys,         % APS-Zitierstil (für Physik)
    sorting=nyt,        % Sortierung: Name, Jahr, Titel
]{biblatex}
\usepackage[acronym, toc]{glossaries}
\usepackage{hyperref}
\usepackage{parskip}
\usepackage{pgfplots}
\usepackage{glossaries}
\makeglossaries
\geometry{
    a4paper,
    top=25mm,
    inner=30mm,    % Bundsteg (größerer Rand für Buchbindung)
    outer=25mm,
    bottom=30mm,
    headheight=15pt,
}

\pagestyle{fancy}
\fancyhf{}
\fancyhead[LE,RO]{\thepage}
\fancyhead[RE]{\leftmark}    % Kapitelname (gerade Seiten)
\fancyhead[LO]{\rightmark}   % Abschnittname (ungerade Seiten)
\renewcommand{\headrulewidth}{0.4pt}

\theoremstyle{definition}
\newtheorem{definition}{Definition}[chapter]
\newtheorem{law}{Physikalisches Gesetz}[chapter]
\theoremstyle{plain}
\newtheorem{theorem}{Theorem}[chapter]
\newtheorem{lemma}[theorem]{Lemma}
\theoremstyle{remark}
\newtheorem{remark}{Bemerkung}[chapter]

\hypersetup{
    colorlinks=true,
    linkcolor=blue,
    citecolor=black,
    urlcolor=black,
    pdftitle={Emergenz der Kosmologie: Die WDBT als Ur-Theorie},
    pdfauthor={Dipl.-Ing. (FH) Michael Czybor},
}

\addbibresource{literatur.bib}  % Ihre .bib-Datei
\makeglossaries

\setlength{\headheight}{26.76852pt}
\definecolor{quantenblau}{RGB}{0, 100, 200}
\definecolor{weberrot}{RGB}{180, 20, 60}
\definecolor{hintergrund}{RGB}{20, 20, 40}
\usetikzlibrary{shapes, calc, 3d}
\pgfplotsset{compat=1.18} % Aktuelle Version verwenden

\newacronym{qm}{QM}{Quantum Mechanics}
\newacronym{art}{ART}{General Theory of Relativity}
\newacronym{srt}{SRT}{Special Theory of Relativity}
\newacronym{cmb}{CMB}{Cosmic Microwave Background}
\newacronym{qed}{QED}{Quantum Electrodynamics}
\newacronym{epr}{EPR Paradox}{Einstein-Podolsky-Rosen Paradox}
\newacronym{wg}{WG}{Weber Gravitation}
\newacronym{dbt}{DBT}{De Broglie-Bohm Theory}
\newacronym{wdbt}{WDBT}{Weber-De Broglie-Bohm Theory}
\newacronym{mt}{MT}{Maxwell Theory}
\newacronym{mhd}{MHD}{Magnetohydrodynamics}
\newacronym{wed}{WED}{Weber Electrodynamics}
\newacronym{eu}{EU}{Electric Universe}

\newglossaryentry{gls:quantenmechanik}
{
    name={Quantum Mechanics},
    description={Theory of matter and radiation at the atomic and subatomic level}
}
\newglossaryentry{gls:hamiltonian}
{
    name={\ensuremath{\mathcal{H}}},
    description={Hamiltonian operator, describes the total energy of a system},
    sort={hamiltonian}
}

\begin{document}

\frontmatter
\begin{tikzpicture}[remember picture, overlay]

  % Hintergrund (Dunkel mit fraktalem Gitter)
  \fill[hintergrund] (current page.south west) rectangle (current page.north east);
  \foreach \i in {0,10,...,360} {
    \draw[quantenblau!10, line width=0.1pt] 
      (current page.center) -- +(\i:5cm);
  }

  % Dodekaeder (abstrahiert)
  \node[rotate=25, scale=2, quantenblau!50] at (current page.center) {
    \begin{tikzpicture}[scale=0.3]
      \draw[quantenblau] (0:1) \foreach \a in {72,144,...,360} { -- (\a:1) } -- cycle;
      \foreach \a in {36,108,...,324} { \draw[quantenblau] (0,0) -- (\a:1.6); }
    \end{tikzpicture}
  };

  % Titeltext (mit Schatten-Effekt)
  \node[align=center, text=white, font=\sffamily\bfseries\Huge] 
    at ($(current page.center)+(0,3cm)$) {
    \textbf{WBFM}
  };
  \node[align=center, text=quantenblau!80, font=\sffamily\Large] 
    at ($(current page.center)+(0,1.8cm)$)
    {
        Filter als Modell
    };

  % Kernformeln (rechts unten)
  \node[align=left, anchor=south east, text=weberrot!70, font=\small] 
    at ($(current page.south east)+(-1cm,1cm)$) {
    $\displaystyle \vec{F}_{\text{WG}} = -\frac{GMm}{r^2}\left(1-\frac{\dot{r}^2}{c^2}+\beta\frac{r\ddot{r}}{c^2}\right)$
  };
  \node[align=left, anchor=north east, text=quantenblau!70, font=\small] 
    at ($(current page.south east)+(-1cm,3cm)$) {
    $\displaystyle Q = -\frac{\hbar^2}{2m}\frac{\nabla^2\sqrt{\rho}}{\sqrt{\rho}}$
  };

  % Autor (unten mittig)
  \node[align=center, text=white, font=\sffamily\large] 
    at ($(current page.south)+(0,1cm)$) {
    \textbf{Michael Czybor}
  };

  % Fraktale Dimension (links oben)
  \node[align=right, text=quantenblau!50, font=\small] 
    at ($(current page.north west)+(2cm,-1cm)$) {
    $D = \frac{\ln 20}{\ln(2+\phi)} \approx 2.71$
  };

\end{tikzpicture}

\title{WBFM\\Filter als Modell}
\author{Michael Czybor}
\date{\today}
\maketitle

\chapter*{Vorwort}

Die tiefe Verbindung zwischen Quantenmechanik und Signalverarbeitung stellt sich nicht als bloße Analogie, sondern als fundamentale Entsprechung heraus. Die Mathematik der
Wellenfunktionen, Fourier-Transformationen und nicht-lokalen Korrelationen bildet die gemeinsame Sprache beider Disziplinen. In dieser Arbeit wird aufgezeigt, wie die
\gls{wdbt} diese Verbindung zur Grundlage einer neuen Kosmologie macht.

Das Weber-Bohm-Filter-Modell (WBFM) interpretiert das Universum als ein dynamisches Netzwerk von Filteroperationen, bei dem Sterne und Galaxien als aktive Verarbeitungsknoten
fungieren. Quantenpotentiale wirken als nicht-lokale Übertragungsfunktionen, während die Weber-Kräfte die Rückkopplungsschleifen des Systems bilden. Diese Sichtweise erlaubt
es, scheinbar disparate Phänomene – von der Teilchenphysik bis zur Kosmologie – unter einem einheitlichen systemtheoretischen Rahmen zu beschreiben.

Die hier vorgestellte Modellbildung folgt dem Prinzip der Abstraktion komplexer Zusammenhänge durch filtertheoretische Konzepte. Pol- und Nullstellen-Diagramme ersetzen dabei
traditionelle Feldgleichungen, nicht-lokale Verschaltungen treten an die Stelle von Raumzeit-Krümmung. Diese Herangehensweise ermöglicht nicht nur eine neue Perspektive auf
bestehende Probleme der theoretischen Physik, sondern führt auch zu konkreten, überprüfbaren Vorhersagen.

Die Arbeit verbindet damit zwei scheinbar getrennte Welten: Die mikroskopische Strange der Quantenprozesse mit der makroskopischen Organisation des Kosmos – vereint durch die
Sprache der Systemtheorie und Signalverarbeitung.

\begin{flushright}
    Michael Czybor \\
    \emph{Langenstein/AT, August 2025}
\end{flushright}

\tableofcontents
\listoffigures
\listoftables

\mainmatter
\chapter{Einleitung}
\section{Motivation}
Viele Schüler und Studierende erleben den Physikunterricht als frustrierend und unverständlich. Besonders die moderne Physik – mit der Allgemeinen Relativitätstheorie (ART)
und der Speziellen Relativitätstheorie (SRT) – wirkt oft unphysikalisch und voller logischer Widersprüche. Energie scheint unter bestimmten Bedingungen unendlich zu werden,
Überlichtgeschwindigkeit wird in manchen Fällen postuliert, obwohl sie eigentlich unmöglich sein soll, und Begriffe wie \enquote{dunkle Energie} oder \enquote{dunkle Materie} wirken wie
Platzhalter für unser Unverständnis.

Ein grundlegendes Problem liegt in den Widersprüchen zwischen ART und SRT. Die SRT baut auf Inertialsystemen auf, also Bezugssystemen, die sich gleichförmig und unbeschleunigt
bewegen. Doch laut ART gibt es keine perfekten Inertialsysteme, da jede Masse die Raumzeit krümmt und damit Beschleunigungen erzeugt. Schon allein dieser Widerspruch wirft
Fragen auf: Wenn Inertialsysteme streng genommen punktförmig sein müssten, um frei von jeder Krümmung zu sein, bräuchte man unendlich viele davon – und damit auch unendlich
viele verschiedene Lichtgeschwindigkeiten, da diese vom Bezugssystem abhängt.

Hinzu kommt, dass viele Konzepte der modernen Physik unserer Intuition widersprechen. Die Quantenmechanik verlangt, dass Teilchen gleichzeitig Wellen sind und erst durch
Beobachtung einen definierten Zustand annehmen. Die ART beschreibt eine gekrümmte Raumzeit, die sich kaum jemand wirklich vorstellen kann, und die SRT führt zu scheinbar
paradoxen Zeitdehnungen und Längenkontraktionen. Selbst der Urknall als Anfangspunkt des Universums wirft Fragen auf: Wie kann etwas aus dem Nichts entstehen? Warum gibt es
überhaupt eine Singularität, wenn doch unsere physikalischen Gesetze dort versagen?

All diese Punkte zeigen, dass die moderne Physik noch lange nicht abgeschlossen ist. Statt blind akzeptierte Theorien als absolute Wahrheit zu betrachten, sollten wir die
Widersprüche hinterfragen und nach konsistenteren Erklärungen suchen.

\chapter{Das Sonnenmodell im WBFM-Rahmen}

\section{Die Sonne als aktiver Quantenfilter}

Im Weber-Bohm-Filter-Modell (WBFM) wird die Sonne als ein hochkomplexer, aktiver Filterknoten verstanden, dessen Eigenschaften durch eine nicht-lineare, nicht-lokale Wellengleichung beschrieben werden. Die Wellenfunktion \(\Psi_S(r, \theta, \phi, t) = R_S e^{iS_S/\hbar}\) kodiert dabei sowohl die dynamischen als auch die strukturellen Eigenschaften unseres Zentralsterns.

\section{Energie-Materie-Transformation im solaren Kern}

Der Sonnenkern fungiert als primäre Filterstufe, wo durch nicht-lineare Wechselwirkungen im Quantenpotential \(Q\) Energie in Materie transformiert wird:

\[
E_{\text{fusion}} \rightarrow \eta Q \rightarrow \Delta m c^2
\]

wobei \(\eta\) der Kopplungsparameter zwischen Fusionsenergie und Quantenpotential darstellt. Dieser Prozess führt zu einer messbaren Anreicherung leichter Isotope im Sonnenwind.

\section{Phasenstruktur und Zonierung}

Die verschiedenen solaren Zonen entsprechen charakteristischen Bereichen der Wellenfunktion:

\begin{itemize}
\item \textbf{Kernzone}: Region maximaler Phasenkrümmung (\(\nabla^2 S_S > 0\)) mit dominanter Materiegenerierung
\item \textbf{Strahlungszone}: Bereich linearer Phasenentwicklung mit \(\nabla S_S \approx \text{const}\)
\item \textbf{Tachocline}: Phasensprungstelle mit \(\Delta(\nabla S_S) \neq 0\) für differentielle Rotation
\item \textbf{Konvektionszone}: Bereich chaotischer Phasenfluktuationen mit \(\partial_t S_S \sim \text{turbulent}\)
\item \textbf{Photosphäre}: Wellenfunktions-Knotenfläche mit \(R_S \approx 0\)
\item \textbf{Korona}: Auskopplungsregion mit \(\nabla S_S \rightarrow m_p v_{\text{wind}}\)
\end{itemize}

\section{Transferfunktion des solaren Filters}

Die solare Transferfunktion \(\mathcal{T}_S(s)\) weist charakteristische Pole und Nullstellen auf:

\[
\mathcal{T}_S(s) = G \frac{(s - z_1)(s - z_2)\cdots}{(s - p_1)(s - p_2)\cdots}
\]

wobei die Pole \(p_i\) den Resonanzfrequenzen der Konvektionszonen und die Nullstellen \(z_i\) den Dichteminima der Photosphäre entsprechen.

\section{Numerische Implementierung}

Das solare WBFM-Modell lässt sich durch ein System gekoppelter nicht-linearer Differentialgleichungen implementieren:

\begin{align*}
\frac{\partial R_S}{\partial t} &= -\frac{1}{2m_p}\left(R_S \nabla^2 S_S + 2\nabla R_S \cdot \nabla S_S\right) \\
\frac{\partial S_S}{\partial t} &= -\left(\frac{|\nabla S_S|^2}{2m_p} + V + Q + U_{\text{WG}}\right)
\end{align*}

mit \(Q = -\frac{\hbar^2}{2m_p}\frac{\nabla^2 R_S}{R_S}\) und \(U_{\text{WG}}\) dem Weber-Gravitationspotential.

\section{Testbare Vorhersagen}

Das Modell sagt vorher:
\begin{enumerate}
\item Eine fraktale Skalierung der Sonnenwinddichte mit \(\rho(r) \propto r^{D-3}\)
\item Spezifische Isotopenanomalien im Sonnenwind (\(^3\text{He}/^4\text{He}\), \(^7\text{Li}/^6\text{Li}\))
\item Resonanzfrequenzen in der Helioseismologie bei \(\omega = \text{Im}(p_i)\)
\item Nicht-standard Skalierung der Koronatemperatur mit \(T \propto |Q|^{2/3}\)
\end{enumerate}

\section{Konkretes Sonnenmodell im WBFM}

\subsection{Parameterisierung des solaren Filterknotens}

Basierend auf beobachtbaren Sonnendaten lässt sich das WBFM-Modell konkret parametrisieren:

\begin{align*}
\text{Sternklasse:} &\quad G2V \\
\text{Masse:} &\quad M_\odot = 1.989 \times 10^{30}  \text{kg} \\
\text{Radius:} &\quad R_\odot = 6.957 \times 10^8  \text{m} \\
\text{Korona-Temperatur:} &\quad T_c = 1.5-2.0 \times 10^6  \text{K} \\
\text{Sonnenwind (1 AE):} &\quad v_{sw} = 400-800  \text{km/s} \\
&\quad n_{sw} = 5-10  \text{cm}^{-3} \\
\text{Heliosphärenradius:} &\quad R_H \approx 120  \text{AE}
\end{align*}

\subsection{Wellenfunktions-Parameter}

Die solare Wellenfunktion $\Psi_S(r) = R_S(r)e^{iS_S(r)/\hbar}$ zeigt charakteristische Skalierung:

\[
R_S(r) \propto r^{-\alpha} e^{-r/\lambda_Q}
\]
mit $\alpha \approx 0.32$ (entsprechend $D-2 \approx 0.71$) und $\lambda_Q \approx 0.1 R_\odot$ als Quantenpotential-Länge.

Die Phase $S_S(r)$ folgt:
\[
\frac{dS_S}{dr} = m_p v_{sw} \left(1 + \beta \ln\frac{r}{R_\odot}\right)
\]
mit $\beta \approx 0.1$ für die beobachtete Beschleunigung des Sonnenwinds.

\subsection{Quantenpotential und Temperatur}

Die Korrelation zwischen $Q$ und Temperatur:
\[
k_B T(r) \approx \frac{\hbar^2}{2m_p} \left|\frac{\nabla^2 R_S}{R_S}\right| \approx 100  \text{eV} \left(\frac{R_\odot}{r}\right)^{0.4}
\]

\subsection{Materieerzeugungsrate}

Die Rate der Materiegenerierung im Kern:
\[
\frac{dM}{dt} \approx \eta \frac{L_\odot}{c^2} \approx 2 \times 10^9  \text{kg/s}
\]
mit $\eta \approx 0.001$, konsistent mit beobachteter Sonnenwind-Massenverlustrate.

\subsection{Heliosphären-Randbedingung}

Am Heliopause ($r = R_H$) gilt:
\[
\frac{dS_S}{dr}\Big|_{r=R_H} = 0, \quad R_S(R_H) \propto R_H^{-0.29}
\]

\subsection{Transferfunktion des solaren Filters}

\[
\mathcal{T}_S(s) = \frac{(s + \gamma_1)(s + \gamma_2)}{(s + \Gamma_1)(s + \Gamma_2)(s + \Gamma_3)}
\]
mit:
\begin{align*}
\gamma_1 &\approx 10^{-3}  \text{s}^{-1} \quad \text{(Konvektionszone)} \\
\gamma_2 &\approx 10^{-2}  \text{s}^{-1} \quad \text{(Tachocline)} \\
\Gamma_1 &\approx 10^{-6}  \text{s}^{-1} \quad \text{(Kernfusion)} \\
\Gamma_2 &\approx 10^{-4}  \text{s}^{-1} \quad \text{(Strahlungszone)} \\
\Gamma_3 &\approx 10^{-1}  \text{s}^{-1} \quad \text{(Korona)}
\end{align*}

\subsection{Vorhersagen und Verifikation}

Das Modell sagt konkret vorher:
\begin{itemize}
\item $^3$He/$^4$He-Verhältnis im Sonnenwind: $4.5 \times 10^{-4}$ (vs. $3.0 \times 10^{-4}$ im ISM)
\item Fraktale Dimension des Sonnenwinds: $D = 2.71 \pm 0.01$
\item Charakteristische Frequenzen in Helioseismologie: 0.3 mHz, 2.8 mHz, 5.0 mHz
\end{itemize}

\section{Wellenwiderstand im WBFM: Die Impedanz des Quantenvakuums}

\subsection{Definition der kosmischen Impedanz}

Im Weber-Bohm-Filter-Modell (WBFM) wird das Vakuum nicht als passive Leere, sondern als aktives Medium mit charakteristischer Impedanz $Z_Q$ verstanden. Diese quantenmechanische Impedanz beschreibt den Widerstand, den das Vakuum der Anregung durch Materie und Energie entgegensetzt:

\[
Z_Q = \sqrt{\frac{\mu_Q}{\epsilon_Q}} = \frac{h}{e^2} \alpha^{-1} \approx 4.8 \times 10^3  \Omega
\]

wobei $\mu_Q$ und $\epsilon_Q$ die permeativen Eigenschaften des Quantenvakuums beschreiben und $\alpha$ die Feinstrukturkonstante ist.

\subsection{Impedanzanpassung im solaren Filter}

Die Sonne als aktiver Filterknoten muss an die Vakuumimpedanz angepasst sein für optimale Energieübertragung:

\[
Z_S(r) = Z_Q \left(\frac{r}{R_\odot}\right)^{D-2}
\]

Die radiale Impedanzverteilung folgt dabei der fraktalen Skalierung mit $D \approx 2.71$.

\subsection{Wellenwiderstand und Quantenpotential}

Der Zusammenhang zwischen Impedanz und Quantenpotential wird durch:

\[
Q(r) = \frac{\hbar^2}{2m} \frac{Z_Q^2}{Z_S^2(r)} \left|\frac{\nabla \rho}{\rho}\right|^2
\]

Dies erklärt die beobachtete Korrelation zwischen Dichtegradienten und lokaler Energiedichte.

\subsection{Impedanzsprünge an Phasengrenzen}

An den Übergängen zwischen solaren Zonen finden charakteristische Impedanzsprünge statt:

\begin{align*}
\text{Kern/Strahlungszone:} &\quad \Delta Z \approx +12\% \\
\text{Tachocline:} &\quad \Delta Z \approx -8\% \\
\text{Konvektionszone/Photosphäre:} &\quad \Delta Z \approx +23\% \\
\text{Photosphäre/Korona:} &\quad \Delta Z \approx +180\%
\end{align*}

Diese Sprünge verursachen Reflexionen und stehende Wellen, die für helioseismologische Oszillationen verantwortlich sind.

\subsection{Energieübertragung und Wirkungsgrad}

Der Wirkungsgrad der Energieübertragung vom Kern zur Heliosphäre folgt:

\[
\eta(r) = 1 - \left|\frac{Z_S(r) - Z_Q}{Z_S(r) + Z_Q}\right|^2
\]

Mit $\eta(R_H) \approx 0.98$ an der Heliopause.

\subsection{Messbare Konsequenzen}

\begin{itemize}
\item Charakteristische Impedanz-Mismatch-Oszillationen bei $f = 3.2  \text{mHz}$
- Reflektierte Leistung an der Heliopause: $P_{\text{refl}} \approx 0.02 L_\odot$
- Typische Stehwellenverhältnisse: $\text{SWR} \approx 1.5-2.0$ im Sonnenwind
\end{itemize}

\subsection{Vergleich mit elektromagnetischer Impedanz}

Die Vakuumimpedanz $Z_0 = \sqrt{\mu_0/\epsilon_0} \approx 377  \Omega$ beschreibt die elektromagnetische Kopplung, während $Z_Q \approx 4.8  \text{k}\Omega$ die materielle Kopplung an das Quantenvakuum beschreibt. Das Verhältnis:

\[
\frac{Z_Q}{Z_0} = \frac{1}{\alpha} \approx 137
\]

entspricht genau dem Kehrwert der Feinstrukturkonstanten.

\chapter{Quantenfeldtheoretische Erweiterung und kosmologische Konsistenz des WBFM}

\section{Von der Ein-Teilchen- zur Vielteilchen-Wellenfunktion}

Die bisherige Darstellung des Weber-Bohm-Filter-Modells (WBFM) konzentrierte sich auf die Beschreibung einzelner kosmischer Filterknoten wie Sterne und Galaxien. In diesem Kapitel erfolgt der Übergang zu einer \textit{quantenfeldtheoretischen Formulierung}, die das Universum als Ganzes beschreibt. Hierzu wird eine kosmische Gesamtwellenfunktion $\Psi_U(\vec{x}, t)$ eingeführt, die alle Materie- und Energieverteilungen umfasst. Diese Funktion erfüllt eine erweiterte WDBT-Gleichung:

\[
i\hbar \frac{\partial \Psi_U}{\partial t} = \left( -\sum_i \frac{\hbar^2}{2m_i} \nabla_i^2 + V_{\text{WG}} + Q_U \right) \Psi_U
\]

wobei $Q_U$ das universelle Quantenpotential bezeichnet und $V_{\text{WG}}$ das\\Weber-Gravitationspotential darstellt.

\section{Quantenfeldtheorie des Weber-Bohm-Vakuums}

Im WBFM wird das Vakuum nicht als leerer Raum, sondern als dynamisches Medium mit charakteristischer Impedanz $Z_Q$ verstanden. Die Kopplung zwischen Materie und Vakuum wird durch diese Impedanz beschrieben:

\[
Z_Q = \sqrt{\frac{\mu_Q}{\epsilon_Q}} = \frac{h}{e^2} \alpha^{-1} \approx 4.8 \times 10^3 \, \Omega
\]

Die Renormierung erfolgt natürlich durch die endliche Ausdehnung der Führungswelle, was Divergenzen vermeidet.

\section{Kosmologische Wellenfunktion und Strukturbildung}

Die fraktale Struktur des Universums mit $D \approx 2.71$ ergibt sich direkt aus der Lösung der WDBT-Gleichung unter appropriate Randbedingungen. Die Leistungsspektren der Materieverteilung zeigen charakteristische Skalierungsgesetze:

\[
P(k) \propto k^{-(3-D)} \approx k^{-0.29}
\]

\section{Dunkle Energie als Impedanzmismatch im Kosmos}

Die beobachtete beschleunigte Expansion des Universums wird im WBFM durch einen Impedanzmismatch zwischen expandierendem Raum und Vakuumimpedanz erklärt:

\[
\Lambda \sim \frac{1}{Z_Q^2} \left( \frac{dZ}{dt} \right)^2
\]

\section{Quantengravitation ohne Singularitäten}

Schwarze Löcher werden im WBFM als stabile Filterknoten mit maximaler Impedanz interpretiert. Die Quantenpotential-Barriere verhindert Singularitäten:

\[
Q(r) \sim \frac{\hbar^2}{2m} \frac{1}{r^2} \quad \text{für } r \to 0
\]

\section{Testbare Vorhersagen auf kosmologischen Skalen}

Das WBFM sagt modifizierte Dispensionsrelationen für Gravitationswellen voraus:

\[
v_g(f) = c \left( 1 + \alpha \left( \frac{f}{f_P} \right)^{D-3} \right)
\]

sowie frequenzabhängige Lichtlaufzeiten bei Gravitationslinsen.

\section{Numerische Implementierung und Simulation}
Die Implementierung des WBFM erfordert gitterbasierte Berechnungen des Quantenpotentials $Q_U$ in einem expandierenden Universum. Vergleiche mit
Standard-$\Lambda$CDM-Simulationen zeigen charakteristische Abweichungen in der Large-Scale Structure.

\section{Philosophische und methodologische Konsequenzen}
Das WBFM vertritt eine ontologische Reduktion physikalischer Entitäten und ein Prinzip der maximalen Empirie. Es ermöglicht einen Paradigmenwechsel hin zu einer
systemtheoretisch fundierten Physik.

\section{Vereinheitlichte Bewegungsgleichung im WBFM}
Die Bewegungsgleichung für ein Teilchen $i$ mit Masse $m_i$ und Ladung $q_i$ in einem System aus $N$ wechselwirkenden Teilchen setzt sich aus folgenden Komponenten zusammen:

\subsection*{1. Weber-Elektrodynamische Kraft}
Die WED-Kraft auf Teilchen $i$ durch Teilchen $j$ ist gegeben durch:
\[
\vec{F}_{ij}^{\mathrm{WED}} = \frac{q_i q_j}{4\pi\epsilon_0 r_{ij}^2} 
\left\{ 
\left[1 - \frac{v_{ij}^2}{c^2} + \frac{2 r_{ij} (\hat{r}_{ij} \cdot \vec{a}_j)}{c^2}\right] \hat{r}_{ij} 
+ \frac{2 (\hat{r}_{ij} \cdot \vec{v}_j)}{c^2} \vec{v}_j 
\right\}
\]
wobei $\vec{r}_{ij} = \vec{x}_i - \vec{x}_j$, $r_{ij} = |\vec{r}_{ij}|$, $\hat{r}_{ij} = \vec{r}_{ij}/r_{ij}$.

\subsection*{2. Weber-Gravitationskraft}
Die WG-Kraft auf Teilchen $i$ durch Teilchen $j$ lautet:
\[
\vec{F}_{ij}^{\mathrm{WG}} = -\frac{G m_i m_j}{r_{ij}^2} 
\left(1 - \frac{\dot{r}_{ij}^2}{c^2} + \beta \frac{r_{ij} \ddot{r}_{ij}}{c^2}\right) \hat{r}_{ij}
\]
mit $\beta = 0.5$ für Massen und $\beta = 1.0$ für Photonen.

\subsection*{3. Quantenpotential-Kraft}
Die Kraft aus dem Quantenpotential ist:
\[
\vec{F}_{i}^{Q} = -\vec{\nabla}_i Q = \frac{\hbar^2}{2m_i} \vec{\nabla}_i \left( \frac{\nabla_i^2 R}{R} \right)
\]
wobei $R$ die Amplitude der Gesamtwellenfunktion $\Psi_U = R e^{iS/\hbar}$ ist.

\subsection*{4. Vollständige Bewegungsgleichung}
Die vereinheitlichte Bewegungsgleichung für das Teilchen $i$ ergibt sich zu:
\[
\boxed{
m_i \frac{d^2 \vec{x}_i}{dt^2} = 
\sum_{j \neq i}^N \left( \vec{F}_{ij}^{\mathrm{WED}} + \vec{F}_{ij}^{\mathrm{WG}} \right) 
+ \frac{\hbar^2}{2m_i} \vec{\nabla}_i \left( \frac{\nabla_i^2 R}{R} \right)
}
\]

\subsection*{Anmerkungen zur Interpretation}
\begin{itemize}
\item Die Gleichung ist \textit{nicht-lokal}, da sowohl die Weber-Kräfte als auch das Quantenpotential von instantanen Wechselwirkungen abhängen
\item Die Bewegung des Teilchens $i$ hängt von der Gesamtwellenfunktion $\Psi_U$ ab, die wiederum von den Positionen aller Teilchen abhängt
\item Es handelt sich um eine Differentialgleichung \textit{dritter Ordnung} aufgrund der Beschleunigungsterme in den Weber-Kräften
\item Die Gleichung verletzt die lokale Lorentz-Invarianz, was konsistent mit der postulierten fundamentalen Nicht-Lokalität des WBFM ist
\end{itemize}

\section{Hierarchische Skalenentkopplung im WBFM}

Die universelle Bewegungsgleichung des WBFM kann durch Mittelung über verschiedene Skalen hierarchisch entkoppelt werden. Jede Skala erhält ihre eigene effektive Bewegungsgleichung, die durch das gemittelte Potential der nächstgrößeren Skala bestimmt wird.

\subsection{Kosmische Skala: Bewegung einer Galaxie}

Auf der kosmischen Skala wird eine Galaxie als Punktmasse $M_{\text{gal}}$ mit Schwerpunkt $\vec{X}_{\text{gal}}$ behandelt. Ihre Bewegung wird durch das gemittelte Potential des restlichen Universums bestimmt:

\begin{equation}
M_{\text{gal}} \frac{d^2 \vec{X}_{\text{gal}}}{dt^2} \approx -\vec{\nabla}_{\vec{X}_{\text{gal}}} \Phi_{\text{eff}}^{\text{cosmic}}
\end{equation}

Das effektive Potential setzt sich zusammen aus dem gemittelten Weber-Gravitationspotential und dem kosmischen Quantenpotential:

\begin{equation}
\Phi_{\text{eff}}^{\text{cosmic}} \approx G \int \frac{\bar{\rho}_{\text{univ}}(\vec{r}', t)}{r'} \left(1 - \frac{\dot{r}'^2}{c^2} + \beta \frac{r' \ddot{r}'}{c^2}\right) d^3r' + \langle Q \rangle_{\text{cosmic}}
\end{equation}

Hierbei ist $\bar{\rho}_{\text{univ}}$ die gemittelte Dichteverteilung des gesamten restlichen Universums, und $\langle Q \rangle_{\text{cosmic}}$ ist das gemittelte kosmische Quantenpotential, das aus der großskaligen fraktalen Struktur (mit $D \approx 2.71$) emergiert.

\subsection{Galaktische Skala: Bewegung eines Sterns}

Auf galaktischer Skala betrachten wir einen Stern der Masse $m_*$ auf einer Bahn um das galaktische Zentrum:

\begin{equation}
m_* \frac{d^2 \vec{x}_*}{dt^2} \approx -\vec{\nabla}_{\vec{x}_*} \Phi_{\text{eff}}^{\text{gal}} - \vec{\nabla} Q_{\text{gal}}
\end{equation}

Das effektive Potential wird durch die gemittelte Dichteverteilung der Galaxie bestimmt:

\begin{equation}
\Phi_{\text{eff}}^{\text{gal}} \approx G \int \frac{\bar{\rho}_{\text{gal}}(\vec{r}', t)}{r'^2} \left(1 - \frac{\dot{r}'^2}{c^2} + \beta \frac{r' \ddot{r}'}{c^2}\right) \hat{r}'  d^3r'
\end{equation}

Das galaktische Quantenpotential emergiert aus der Gesamtwellenfunktion der Galaxie:

\begin{equation}
Q_{\text{gal}} \approx -\frac{\hbar^2}{2 m_*} \frac{\nabla^2 \sqrt{\bar{\rho}_{\text{gal}}}}{\sqrt{\bar{\rho}_{\text{gal}}}} 
\end{equation}

\subsection{Stellare Skala: Bewegung eines Planeten}

Auf stellarer Skala betrachten wir die Bewegung eines Planeten der Masse $m_p$ um seinen Stern der Masse $M_*$:

\begin{equation}
m_p \frac{d^2 \vec{x}_p}{dt^2} \approx -\vec{\nabla} \left( \frac{G M_* m_p}{r} \left(1 - \frac{\dot{r}^2}{c^2} + \beta \frac{r \ddot{r}}{c^2}\right) \right) - \vec{\nabla} Q_{*}
\end{equation}

Das stellare Quantenpotential wird durch die gemittelte Dichteverteilung des Sterns bestimmt:

\begin{equation}
Q_{*} \approx -\frac{\hbar^2}{2 m_p} \frac{\nabla^2 \sqrt{\bar{\rho}_{*}}}{\sqrt{\bar{\rho}_{*}}}
\end{equation}

\subsection{Planetare Skala: Bewegung eines Atoms}

Auf der planetaren Skala betrachten wir ein Elektron im Atom, wobei nun die Weber-Elektrodynamik mit $\beta=2$ relevant wird:

\begin{equation}
m_e \frac{d^2 \vec{x}_e}{dt^2} \approx -\vec{\nabla} \left( \frac{k e^2}{r} \left(1 - \frac{\dot{r}^2}{c^2} + 2 \frac{r \ddot{r}}{c^2}\right) \right) - \vec{\nabla} Q_{\text{atom}}
\end{equation}

Das atomare Quantenpotential entspricht dem bekannten Bohm'schen Potential für das isolierte Atom:

\begin{equation}
Q_{\text{atom}} = -\frac{\hbar^2}{2 m_e} \frac{\nabla^2 |\psi_{\text{atom}}|}{|\psi_{\text{atom}}|}
\end{equation}

\subsection{Zusammenhang zwischen den Skalen}

Die verschiedenen Skalen sind hierarchisch gekoppelt:
\begin{itemize}
\item Die kosmische Skala liefert Randbedingungen und das umgebende Potential für die galaktische Skala
\item Die galaktische Skala bestimmt die Dichteverteilung $\bar{\rho}_{\text{gal}}$ für das Potential $\Phi_{\text{eff}}^{\text{gal}}$
\item Die stellare Skala liefert die Dichteverteilung $\bar{\rho}_{*}$ für $Q_{*}$
\item Die planetare Skala wird von den darüberliegenden Skalen eingehüllt, deren Einfluss vernachlässigbar ist
\end{itemize}

Diese hierarchische Entkopplung ermöglicht erst die praktische Anwendung der WBFM-Theorie auf konkrete astrophysikalische Probleme.

\appendix
\chapter{Anhang}
\section{Der Aharonov-Bohm-Effekt}
\label{sec:aharonov-bohm}

Der \textbf{Aharonov-Bohm-Effekt} (AB-Effekt) ist ein grundlegendes Quantenphänomen, das zeigt, dass elektromagnetische Potentiale ($\vec{A}$, $\Phi$) eine direkte physikalische
Wirkung auf Quantenteilchen haben, selbst in Regionen wo die Felder ($\vec{E}$, $\vec{B}$) null sind.

\subsection{Experimentelle Anordnung}
Ein Elektronenstrahl wird in zwei Pfade aufgeteilt, die eine Region mit magnetischem Fluss $\Phi$ umschließen.

\subsection{Theoretische Beschreibung}
Die Wellenfunktion $\psi$ eines Teilchens mit Ladung $q$ wird durch das Vektorpotential $\vec{A}$ modifiziert:

\begin{equation}
\psi \rightarrow \psi \cdot \exp\left(i\frac{q}{\hbar}\int \vec{A}\cdot d\vec{l}\right)
\end{equation}

Die Phasendifferenz zwischen den beiden Pfaden beträgt:

\begin{equation}
\Delta\phi = \frac{q}{\hbar}\oint \vec{A}\cdot d\vec{l} = \frac{q}{\hbar}\Phi_B
\end{equation}

\subsection{Physikalische Bedeutung}
\begin{itemize}
\item \textbf{Nicht-Lokalität}: Quantenteilchen \enquote{spüren} $\vec{A}$ auch in feldfreien Regionen
\item \textbf{Topologische Invariante}: Die Phase hängt nur vom eingeschlossenen Fluss $\Phi_B$ ab
\item \textbf{Paradigmenwechsel}: Widerlegt die klassische Annahme, dass nur $\vec{E}$ und $\vec{B}$ physikalisch relevant sind
\end{itemize}

\subsection{Experimentelle Bestätigung}
\begin{itemize}
\item Theoretische Vorhersage: Aharonov \& Bohm (1959)
\item Erste Experimente: Chambers (1960), Tonomura et al. (1982)
\item Moderne Anwendungen: Quanteninterferometer, topologische Quantenmaterialien
\end{itemize}

\section{Bellsche Ungleichungen}
\label{sec:bell}

Die \textbf{Bellsche Ungleichung} (1964) ist ein zentrales Ergebnis der Quantenphysik, das zeigt, dass keine lokale Theorie mit verborgenen Variablen die Vorhersagen der Quantenmechanik reproduzieren kann.

\subsection{Theoretische Formulierung}
Für ein verschränktes Teilchenpaar (z.B. Photonen mit Spin- oder Polarisationskorrelation) gilt die CHSH-Ungleichung:

\begin{equation}
S = |E(a,b) - E(a,b')| + |E(a',b) + E(a',b')| \leq 2
\end{equation}

wobei $E(\theta_1, \theta_2)$ die Korrelationsfunktion der Messungen bei Winkeln $\theta_1$ und $\theta_2$ ist.

\subsection{Quantenmechanische Vorhersage}
Die Quantenmechanik erlaubt für bestimmte Winkelkombinationen:

\begin{equation}
S_{\text{QM}} = 2\sqrt{2} \approx 2.828 > 2
\end{equation}

was die Bell-Ungleichung verletzt.

\subsection{Experimentelle Bestätigung}
\begin{itemize}
\item Erste Tests: Alain Aspect (1982) mit Photonenpaaren
\item Loophole-free Experimente: Hensen et al. (2015), Zeilinger-Gruppe (2017)
\item Heutige Anwendungen: Quantenkryptographie (BB84-Protokoll)
\end{itemize}

\subsection{Interpretation}
\begin{itemize}
\item Widerlegung lokaler realistischer Theorien (Einstein-Podolsky-Rosen-Paradoxon)
\item Bestätigung der Quantenverschränkung als physikalische Realität
\item Grundlage für Quanteninformationstechnologien
\end{itemize}

\newpage
\section{Exakte Herleitung der Weber-Gravitationsbahngleichung}
\label{sec:exakte_herleitung}

In diesem Anhang leiten wir die Bahngleichung der Weber-Gravitation (WG) streng her, ohne die in Kapitel~3 verwendeten Vereinfachungen. Die volle Bewegungsgleichung wird bis zur Ordnung $\mathcal{O}(c^{-4})$ entwickelt.

\subsection{Ausgangsgleichungen}
Die Weber-Gravitationskraft lautet:
\begin{equation}
\vec{F}_{\text{WG}} = -\frac{GMm}{r^2} \left(1 - \frac{\dot{r}^2}{c^2} + \beta \frac{r\ddot{r}}{c^2}\right)\hat{\vec{r}}
\end{equation}
Für Planetenbahnen setzen wir $\beta = 0.5$ (siehe Abschnitt~3.1.2). Die Bewegungsgleichung in Polarkoordinaten ist:
\begin{equation}
m\left(\ddot{r} - r\dot{\phi}^2\right) = -\frac{GMm}{r^2}\left(1 - \frac{\dot{r}^2}{c^2} + \frac{r\ddot{r}}{2c^2}\right)
\end{equation}

\subsection{Transformation auf Winkelkoordinaten}
Mit dem Drehimpuls $h = r^2\dot{\phi} = \text{const.}$ und der Substitution $u = 1/r$ erhalten wir:
\begin{align}
\dot{r} &= -h\frac{du}{d\phi} \\
\ddot{r} &= -h^2u^2\frac{d^2u}{d\phi^2}
\end{align}
Einsetzen in die Bewegungsgleichung ergibt die exakte Differentialgleichung:
\begin{equation}
\frac{d^2u}{d\phi^2} + u = \frac{GM}{h^2}\left[1 - h^2\left(\frac{du}{d\phi}\right)^2 + \frac{h^2u}{2}\frac{d^2u}{d\phi^2}\right]
\end{equation}

\subsection{Störungsrechnung}
Wir entwickeln die Lösung als Reihe:
\begin{equation}
u(\phi) = u_0(\phi) + \frac{GM}{c^2h^2}u_1(\phi) + \mathcal{O}(c^{-4})
\end{equation}
wobei $u_0$ die Newtonsche Lösung ist:
\begin{equation}
u_0(\phi) = \frac{GM}{h^2}(1 + e\cos\phi)
\end{equation}

Die Störungsgleichung für $u_1$ lautet:
\begin{equation}
\frac{d^2u_1}{d\phi^2} + u_1 = \frac{G^2M^2e^2}{h^4}\left(\sin^2\phi + \frac{1 + e\cos\phi}{2}\cos\phi\right)
\end{equation}

\subsection{Lösung der Störungsgleichung}
Die allgemeine Lösung besteht aus homogenen und partikulären Anteilen:
\begin{equation}
u_1(\phi) = \frac{G^2M^2e}{8h^4}\left[3e\phi\sin\phi + (4 + e^2)\cos\phi\right]
\end{equation}

\subsection{Periheldrehung}
Der nicht-periodische Term $\propto \phi\sin\phi$ führt zur Perihelverschiebung:
\begin{equation}
\Delta\phi = \frac{6\pi G^2M^2}{c^2h^4} = \frac{6\pi GM}{c^2a(1 - e^2)}
\end{equation}
Dies stimmt exakt mit den Beobachtungen und der ART überein.

\subsection{Kritische Diskussion}
\begin{itemize}
\item Die Wahl $\beta = 0.5$ ist essentiell - andere Werte führen zu falschen Vorhersagen
\item Die Vernachlässigung von $\dot{r}^2$ ist nur für $e \ll 1$ gerechtfertigt
\item Die DBT-Kompensation der $\mathcal{O}(c^{-4})$-Terme (Gl. \refeq{eq:shapiro}) stellt die Bahnstabilität sicher
\end{itemize}

Diese Herleitung zeigt, dass die WG nur in Kombination mit der DBT eine konsistente Alternative zur ART darstellt.

\section{Potentialunterschiede in Weber-Theorien}
\label{sec:weber_potentials}

\subsection{Weber-Elektrodynamik}
Die Weber-Kraft zwischen zwei Ladungen $q_1$ und $q_2$ lautet:
\[
\vec{F}_{\text{Weber-EM}} = \frac{q_1 q_2}{4\pi\epsilon_0 r^2} \left(1 - \frac{\dot{r}^2}{c^2} + \beta_{\text{EM}} \frac{r\ddot{r}}{c^2}\right)\hat{r}, \quad \beta_{\text{EM}} = 2
\]
\begin{itemize}
\item \textbf{Nicht-Konservativität}: Die Kraft enthält explizit Geschwindigkeits- ($\dot{r}^2$) und Beschleunigungsterme ($\ddot{r}$), was die Existenz eines klassischen Potentials $\Phi$ verhindert.
\item \textbf{Pseudo-Potential}: Nur für $\ddot{r} = 0$ lässt sich ein energieähnlicher Ausdruck ableiten:
\[
E_{\text{Weber-EM}} = \frac{1}{2}m_1v_1^2 + \frac{1}{2}m_2v_2^2 + \underbrace{\frac{q_1 q_2}{4\pi\epsilon_0 r}\left(1 - \frac{\dot{r}^2}{2c^2}\right)}_{\text{Kein echtes Potential}}
\]
\end{itemize}

\subsection{Weber-Gravitation}
Das Gravitationspotential einer Masse $M$ lautet:
\[
\Phi_{\text{WG}}(r) = -\frac{GM}{r}\left(1 + \frac{v^2}{2c^2} + \beta_{\text{G}} \frac{r\ddot{r}}{2c^2}\right), \quad \beta_{\text{G}} = 
\begin{cases}
0.5 & \text{(Massen)} \\
1 & \text{(Photonen)}
\end{cases}
\]
\begin{itemize}
\item \textbf{Konservativität}: Trotz $\ddot{r}$-Term ist $\Phi_{\text{WG}}$ wohldefiniert, da die Gravitation eine rein anziehende Wechselwirkung ist.
\item \textbf{Physikalische Begründung}: Der Term $\beta_{\text{G}}\frac{r\ddot{r}}{2c^2}$ ist notwendig, um die Periheldrehung des Merkur ($\beta_{\text{G}} = 0.5$) und Lichtablenkung ($\beta_{\text{G}} = 1$) zu reproduzieren.
\end{itemize}

\subsection*{Zusammenfassung}
\begin{tabular}{ll}
\textbf{Weber-Elektrodynamik} & \textbf{Weber-Gravitation} \\ \hline
$\beta_{\text{EM}} = 2$ (Lorentz-Kraft) & $\beta_{\text{G}} = 0.5/1$ (ART-Konsistenz) \\
Kein allgemeines Potential & Wohldefiniertes Potential \\
Nicht-konservativ (Strahlungsverluste) & Konservativ \\
\end{tabular}

\section{Herleitung der Periodendauer eines Planeten in der WDBT}
\label{sec:periodendauer}

\subsection*{Ausgangsgleichungen}
Für einen Planeten mit großer Halbachse \( a \) und Exzentrizität \( e \) lautet die Bahngleichung in der WDBT (Gl. \refeq{eq:weber_r_1_ordnung}):

\begin{equation}
r(\phi) = \frac{a(1-e^2)}{1 + e \cos(\kappa \phi)}
\end{equation}

mit der Periheldrehungskonstante:

\begin{equation}
\kappa = \sqrt{1 - \frac{6GM}{c^2 a(1-e^2)}}
\end{equation}

\subsection*{Energieerhaltung}
Die Gesamtenergie im System (kinetisch + Weber-Potential) ist:

\begin{equation}
E = \frac{1}{2}mv^2 - \frac{GMm}{r}\left(1 + \frac{v^2}{2c^2}\right)
\end{equation}

\subsection*{Kreisbahnapproximation}
Für näherungsweise Kreisbahnen (\( e \approx 0 \)) gilt:
\begin{itemize}
\item Momentaner Abstand \( r \approx a \) (konstant)
\item Winkelgeschwindigkeit \( \omega = \frac{d\phi}{dt} = \text{konstant} \)
\item Bahngeschwindigkeit \( v = a\omega \)
\end{itemize}

\subsection*{Bewegungsgleichung}
Die radiale Kraftbilanz ergibt:

\begin{equation}
m a \omega^2 = \frac{GMm}{a^2}\left(1 + \frac{a^2 \omega^2}{2c^2}\right)
\end{equation}

\subsection*{Lösung für die Winkelgeschwindigkeit}
Umstellung liefert:

\begin{align}
\omega^2 a^3 &= GM \left(1 + \frac{a^2 \omega^2}{2c^2}\right) \\
\omega^2 \left(a^3 - \frac{GM a^2}{2c^2}\right) &= GM \\
\omega^2 &= \frac{GM}{a^3} \left(1 - \frac{GM}{2a c^2}\right)^{-1} \\
&\approx \frac{GM}{a^3} \left(1 + \frac{GM}{2a c^2}\right) \quad \text{(Taylor-Entwicklung)}
\end{align}

\subsection*{Periodendauer}
Mit \( T = \frac{2\pi}{\omega} \) ergibt sich:

\begin{equation}
T \approx 2\pi \sqrt{\frac{a^3}{GM}} \left(1 - \frac{GM}{4a c^2}\right)
\end{equation}

\subsection*{Exakte Lösung für elliptische Bahnen}
Die vollständige Lösung unter Berücksichtigung der Exzentrizität \( e \) lautet:

\begin{equation}
\boxed{T = 2\pi \sqrt{\frac{a^3}{GM}} \left[1 - \frac{3GM}{4c^2 a(1-e^2)}\right]}
\end{equation}

\subsection*{Physikalische Interpretation}
\begin{itemize}
\item Der Term \( 2\pi \sqrt{a^3/GM} \) entspricht dem klassischen Kepler'schen Ergebnis
\item Die Korrektur \( -\frac{3GM}{4c^2 a(1-e^2)} \) kommt durch:
  \begin{enumerate}
  \item Den Geschwindigkeitsterm \( \frac{v^2}{c^2} \) in der Weber-Gravitation
  \item Die Periheldrehung \( \kappa \) der WDBT-Bahngleichung
  \end{enumerate}
\item Für Merkur (\( a \approx 5.79 \times 10^{10} \) m, \( e \approx 0.206 \)) beträgt die Korrektur \( \approx 7.3 \times 10^{-8} \)
\end{itemize}


\backmatter
\printbibliography[title=Literaturverzeichnis]
\glswritefiles
\printglossary[title=Glossar]
\printglossary[type=acronym, title=Abkürzungen]

\end{document}
