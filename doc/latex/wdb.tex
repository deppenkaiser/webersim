\documentclass[11pt, a4paper, twoside, openright]{book}
\usepackage[ngerman]{babel}
\usepackage[T1]{fontenc}
\usepackage[utf8]{inputenc}
\usepackage{lmodern}
\usepackage{microtype}
\usepackage{csquotes}
\usepackage{verbatim}  % Im Kopf des Dokuments einfügen
\usepackage{geometry}
\usepackage{fancyhdr}
\usepackage{amsmath, amssymb, amsthm}  % Mathe
\usepackage{mathtools}                 % \coloneqq, \xrightarrow
\usepackage{bm}                        % Fette Symbole (\bm{B} für Magnetfeld)
\usepackage{siunitx}                   % \SI{1.23}{\meter\per\second}
\usepackage{graphicx}                  % \includegraphics
\usepackage{subcaption}                % Unterabbildungen
\usepackage{booktabs}                  % Professionelle Tabellen
\usepackage{tikz}                      % Für Diagramme
\usepackage{xcolor}                    % Farbige Tabellenzellen
\usepackage[
    backend=biber,
    style=phys,         % APS-Zitierstil (für Physik)
    sorting=nyt,        % Sortierung: Name, Jahr, Titel
]{biblatex}
\usepackage[acronym, toc]{glossaries}
\usepackage{hyperref}
\usepackage{parskip}
\geometry{
    a4paper,
    top=25mm,
    inner=30mm,    % Bundsteg (größerer Rand für Buchbindung)
    outer=25mm,
    bottom=30mm,
    headheight=15pt,
}

\pagestyle{fancy}
\fancyhf{}
\fancyhead[LE,RO]{\thepage}
\fancyhead[RE]{\leftmark}    % Kapitelname (gerade Seiten)
\fancyhead[LO]{\rightmark}   % Abschnittname (ungerade Seiten)
\renewcommand{\headrulewidth}{0.4pt}

\theoremstyle{definition}
\newtheorem{definition}{Definition}[chapter]
\newtheorem{law}{Physikalisches Gesetz}[chapter]
\theoremstyle{plain}
\newtheorem{theorem}{Theorem}[chapter]
\newtheorem{lemma}[theorem]{Lemma}
\theoremstyle{remark}
\newtheorem{remark}{Bemerkung}[chapter]

\hypersetup{
    colorlinks=true,
    linkcolor=blue,
    citecolor=black,
    urlcolor=black,
    pdftitle={WDB-Theorie - Eine effektive Quantengravitation},
    pdfauthor={Dipl.-Ing. (FH) Michael Czybor},
}

\addbibresource{literatur.bib}  % Ihre .bib-Datei
\makeglossaries

\setlength{\headheight}{26.76852pt}

\newacronym{qm}{QM}{Quantenmechanik}
\newacronym{art}{ART}{Allgemeine Relativitätstheorie}
\newacronym{srt}{SRT}{Spezielle Relativitätstheorie}
\newacronym{cmb}{CMB}{Hintergrundstrahlung}
\newacronym{qed}{QED}{Quantenelektrodynamik}
\newacronym{qft}{QFT}{Quantenfeldtheorie}
\newacronym{epr}{EPR-Paradoxon}{Einstein-Podolsky-Rosen-Paradoxon}
\newacronym{wg}{WG}{Weber-Gravitation}
\newacronym{wed}{WED}{Weber-Elektrodynamik}
\newacronym{dbt}{DBT}{De-Broglie-Bohm-Theorie}
\newacronym{wdbt}{WDBT}{Weber-De Broglie-Bohm-Theorie}
\newacronym{mt}{MT}{Maxwell-Theorie}
\newacronym{iwt}{IWT}{Informations-Weber-Theorie}
\newacronym{dstt}{DSTT}{Dynamischen Schwere-Trägheits-Theorie}

\newglossaryentry{gls:quantenmechanik}
{
    name={Quantenmechanik},
    description={Theorie der Materie und Strahlung auf atomarer und subatomarer Ebene}
}
\newglossaryentry{gls:hamiltonian}
{
    name={\ensuremath{\mathcal{H}}},
    description={Hamilton-Operator, beschreibt die Gesamtenergie eines Systems},
    sort={hamiltonian}
}


\begin{document}

\frontmatter
\title{WDB-Theorie\\Eine effektive Quantengravitation}
\author{Michael Czybor}
\date{\today}
\maketitle

\chapter*{Vorwort}
Die \gls{wdbt} stellt nicht einfach eine alternative mathematische Beschreibung physikalischer Phänomene dar, sondern entwirft ein grundlegend neues Paradigma der physikalischen Wirklichkeit.
Im Gegensatz zur etablierten Physik, die auf den Konzepten von Quantenfeldern und Raumzeitkrümmung basiert, geht die \gls{wdbt} von drei fundamentalen Prinzipien aus:
\begin{enumerate}
    \item Direkte Teilchenwechselwirkungen anstelle von vermittelnden Feldern
    \item Nicht-lokale Ganzheit als organisierendes Prinzip
    \item Konfigurationsraum-Dynamik statt ausschließlicher Raumzeit-Beschreibung
\end{enumerate}
Der entscheidende Durchbruch dieser Theorie liegt in ihrer Fähigkeit, die bekannten Phänomene der Quantenmechanik und Gravitation zu erklären, ohne dabei in die Widersprüche zu geraten,
die den Standardtheorien inhärent sind. Während die konventionelle Physik mit Problemen wie dem Messproblem, der Nicht-Lokalität quantenmechanischer Verschränkung oder den Singularitäten
der \gls{art} kämpft, bietet die \gls{wdbt}-Theorie natürliche Lösungen:
\begin{itemize}
    \item Das Quantenpotential der \gls{dbt} erklärt den Welle-Teilchen-Dualismus ohne den mysteriösen \enquote{Kollaps} der Wellenfunktion.
    \item Die Weber-Elektrodynamik beschreibt elektromagnetische Phänomene durch direkte Ladungswechselwirkungen und vermeidet so die unendlichen Selbstenergien der Quantenfeldtheorie.
    \item Die \gls{wg} reproduziert die erfolgreichen Vorhersagen der \gls{art} ohne das Konzept der Raumzeitkrümmung.
\end{itemize}
Der scheinbare Konflikt mit Prinzipien wie der Lorentz-Invarianz oder der lokalen Kausalität ergibt sich ausschließlich aus der falschen Perspektive des etablierten Paradigmas.
In der \gls{wdbt} sind instantane Korrelationen keine Verletzung der Kausalität, sondern Ausdruck einer tieferen, konfigurationsraumweiten Organisation physikalischer Prozesse.
Diese Organisation folgt eigenen, stringenten Gesetzen, die sich von den in der Feldtheorie verankerten Vorstellungen fundamental unterscheiden.

Die experimentelle Äquivalenz zu den Standardtheorien bei gleichzeitiger Vermeidung ihrer konzeptionellen Probleme spricht deutlich für die Stärke der \gls{wdbt}. Sie zeigt,
dass die etablierte Physik nicht die einzig mögliche Beschreibung der Natur ist, sondern lediglich eine von mehreren konsistenten Möglichkeiten. Die Wahl zwischen diesen Beschreibungen
ist daher nicht empirisch, sondern paradigmatisch begründet.

Für die wissenschaftliche Gemeinschaft ergibt sich daraus eine klare Herausforderung: Statt die \gls{wdbt} an den Maßstäben des etablierten Paradigmas zu messen, sollte sie als
eigenständiger theoretischer Rahmen ernstgenommen werden. Ihre Vorhersagen - wie die wellenlängenabhängige Lichtablenkung oder die modifizierten Galaxienrotationskurven - bieten konkrete
Möglichkeiten zur experimentellen Überprüfung.

Die WDB-Theorie zwingt uns, grundlegende Annahmen der modernen Physik zu hinterfragen:
\begin{itemize}
    \item Muss Physik zwingend auf Feldkonzepten basieren?
    \item Ist Lokalität ein fundamentales Prinzip oder nur ein Artefakt bestimmter Theorien?
    \item Können die scheinbaren Widersprüche der Quantenmechanik Ausdruck eines unvollständigen Paradigmas sein?
\end{itemize}
Diese Fragen zeigen, dass die \gls{wdbt} mehr ist als nur eine alternative Formelsammlung - sie ist ein kohärenter, in sich geschlossener Entwurf der physikalischen Wirklichkeit,
der das Potenzial hat, unser Verständnis von Natur grundlegend zu verändern. Ihre Stärke liegt nicht darin, die Standardtheorien in allen Details zu reproduzieren, sondern darin, eine
konsistente Alternative zu bieten, die gleichzeitig deren konzeptionelle Probleme vermeidet.

Die Zukunft wird zeigen, ob die Physik bereit ist, diesen Paradigmenwechsel mitzuvollziehen. Unabhängig davon hat die \gls{wdbt} bereits jetzt ihren Wert bewiesen: Sie demonstriert,
dass die etablierte Physik nicht die einzig mögliche Beschreibung der Natur ist, und zwingt uns, vermeintliche Gewissheiten kritisch zu hinterfragen. In diesem Sinne ist sie nicht nur eine
wissenschaftliche Theorie, sondern auch eine philosophische Herausforderung ersten Ranges.

\tableofcontents

\mainmatter
\chapter{Einführung}
\section{Plasmen als Schlüssel zu einer neuen Physik}
Seit über einem Jahrhundert dominieren Feldtheorien das Denken – von den Maxwell-Gleichungen bis zur \gls{qed}. Doch gerade dort, wo diese Theorien an ihre Grenzen stoßen, in der
Welt der Plasmen, offenbart sich eine tiefere Wahrheit: \textbf{Die Natur kennt keine Felder}. Was wir als elektromagnetische Wechselwirkungen interpretieren, ist in Wirklichkeit ein
komplexes Geflecht direkter, nicht-lokaler Kräfte zwischen Teilchen – eine Erkenntnis, die bereits in der \gls{wed} \cite{Weber1846} angelegt ist und durch die \gls{dbt} \cite{bohm1952}
ihre volle Bedeutung erlangt.

\section{Das kosmische Plasma: Eine Herausforderung für die Standardmodelle}
Im großen Maßstab des Universums zeigt sich das Versagen der Feldtheorien besonders deutlich. Die kosmische \gls{cmb}, oft als Beweis für den Urknall gefeiert, könnte
ebenso gut das thermische Gleichgewicht eines unendlichen, statischen Plasmauniversums beschreiben. Die Rotverschiebung ferner Galaxien, die heute als Indiz für die Expansion des
Raumes gedeutet wird, lässt sich alternativ durch Energieverluste des Lichts in intergalaktischen Plasmen erklären – ein Prozess, den die \gls{wed} präziser beschreibt
als die \gls{art} \cite{einstein1915}.

Die rätselhaften Rotationskurven der Galaxien, die zur Erfindung der dunklen Materie führten, finden in der Plasma-Kosmologie eine natürliche Erklärung: Elektromagnetische Kräfte,
modifiziert durch die Geschwindigkeitsabhängigkeit der Weber-Wechselwirkung, können die beobachteten Geschwindigkeitsprofile erzeugen, ohne auf unsichtbare Teilchen zurückgreifen
zu müssen. Die filamentären Strukturen des kosmischen Netzes, die sich über Hunderte von Millionen Lichtjahren erstrecken, ähneln verblüffend den Mustern, die in
Plasmadynamik-Experimenten auf Laborskala entstehen – ein Hinweis darauf, dass das Universum in seinem Wesen ein elektrisches Phänomen ist.

\subsection{Sternentstehung und Plasmadynamik}
Auch die Geburt der Sterne wirft Fragen auf, die das Feldparadigma nicht befriedigend beantworten kann. Wie können interstellare Wolken aus diffusem Plasma unter ihrer eigenen
Gravitation kollabieren, wenn die elektromagnetischen Abstoßungskräfte um Größenordnungen stärker sind? Die \gls{wdbt} hingegen bietet eine elegante Lösung: Das Quantenpotential der \gls{dbt}
wirkt als nicht-lokale, stabilisierende Kraft, die den Kollaps trotz der elektromagnetischen Barrieren ermöglicht. Gleichzeitig erklärt die Weber-Gravitation mit ihrer geschwindigkeitsabhängigen
Komponente, warum protoplanetare Scheiben rotationsstabil bleiben, ohne dass dunkle Materie als \enquote{Klebstoff} benötigt wird. Details hierzu können dem Anhang (\ref{app:sternentstehung})
entnommen werden.

Die Herausforderung der Sternentstehung liegt im scheinbaren Widerspruch zwischen der enormen elektromagnetischen Abstoßung geladener Teilchen in interstellaren Wolken und der
vergleichsweise schwachen Gravitation, die den Kollaps einleiten soll. Während klassische Modelle auf zusätzliche Annahmen wie magnetische Stabilisierung oder Turbulenzdämpfung
zurückgreifen müssen, bietet die \gls{wdbt} eine elegante Lösung durch das Zusammenspiel des Quantenpotentials und der Weber-Gravitation.

Das Quantenpotential wirkt hier nicht nur als quantenmechanische Korrektur, sondern als entscheidender Vermittler zwischen mikroskopischen und makroskopischen Prozessen. Indem es
die Teilchen in kohärenten, geordneten Bahnen hält, verhindert es die sonst dominierende elektromagnetische Abstoßung und ermöglicht eine großräumige Verdichtung der Wolke.
Gleichzeitig stabilisiert es die Struktur gegen turbulente Fragmentierung, ohne den Kollaps selbst zu blockieren – im Gegensatz zu klassischen Modellen, die solche Effekte nur
durch externe Mechanismen erklären können.

Die Weber-Gravitation ergänzt diesen Prozess, indem ihre geschwindigkeitsabhängigen Terme eine rotationsstabile Kontraktion der Wolke bewirken. Dadurch entsteht ein
selbstorganisierter Kollaps, der weder auf hypothetische dunkle Materie noch auf ad-hoc-Annahmen angewiesen ist. Die fraktale Struktur des Plasmas, die sich natürlich aus der
\gls{wdbt} ergibt, erklärt zudem die hierarchische Anordnung von Sternentstehungsregionen in Filamenten – ein Phänomen, das in herkömmlichen Theorien nur schwer abzubilden ist.

Kurz gesagt: Die \gls{wdbt} zeigt, dass Sternentstehung kein Kampf zwischen Gravitation und elektromagnetischen Kräften ist, sondern ein koordinierter Prozess, der durch
nicht-lokale Quanteneffekte und direkte Teilchenwechselwirkungen gesteuert wird. Dieses Bild passt nicht nur besser zu Beobachtungen, sondern vermeidet auch die willkürlichen
Zusatzannahmen der etablierten Modelle.

\subsection{Kernfusion: Vom ITER zum feldlosen Plasma}
Auf der irdischen Skala zeigt sich das Potential der neuen Sichtweise vielleicht am deutlichsten in der Fusionsforschung. Seit Jahrzehnten kämpfen Projekte wie ITER mit den
Unwägbarkeiten der Plasmaturbulenz – einem Problem, das im Rahmen der \gls{mhd} unlösbar erscheint. Doch was, wenn die Turbulenz gar kein chaotisches Phänomen ist,
sondern die Manifestation einer tieferen, nicht-lokalen Ordnung?

Die \gls{wdbt} legt nahe, dass Plasmen in Fusionsreaktoren nicht durch äußere Magnetfelder kontrolliert werden müssen, sondern sich selbst organisieren können – gesteuert durch
das Quantenpotential und die direkten Teilchenwechselwirkungen der \gls{wed}. Es gibt Hinweise dafür, dass Plasmen in dieser Beschreibung stabilere Konfigurationen
einnehmen, als die Feldtheorie vorhersagt. Sollte sich dies bestätigen, könnte es den Weg zu kompakteren, effizienteren Fusionsreaktoren ebnen – eine Revolution der Energiegewinnung.

Die Kernfusion gilt seit Jahrzehnten als vielversprechende Lösung für die Energieprobleme der Menschheit, doch die technischen Herausforderungen bleiben immens. Projekte wie ITER oder
Wendelstein 7-X setzen auf die \gls{mhd}, um Plasmen bei extrem hohen Temperaturen (über 100 Millionen Grad) einzuschließen. Doch trotz enormer Fortschritte kämpfen diese Anlagen mit
unkontrollierbarer Turbulenz, anomalem Teilchentransport und instabilen Plasmarändern – Probleme, die sich mit den klassischen Modellen nur unzureichend beschreiben lassen. Hier setzt
die \gls{wdbt} an und bietet einen radikal neuen Ansatz, der die Fusion revolutionieren könnte.

\subsubsection{Die Grenzen der MHD in der Fusionsforschung}
Die \gls{mhd} beschreibt Plasmen als kontinuierliche Fluide, die durch Magnetfelder geformt werden. Doch diese Näherung vernachlässigt mikroskopische Effekte wie Teilchenkorrelationen
oder nicht-lokale Wechselwirkungen – genau jene Phänomene, die in Fusionsplasmen eine entscheidende Rolle spielen. Turbulenz und anomaler Widerstand entstehen, weil die Lorentzkraft der
\gls{mhd} die komplexe Dynamik geladener Teilchen nur unvollständig erfasst. Die Folge sind unvorhersehbare Energieverluste und instabile Plasmen, die den Betrieb von Tokamaks oder
Stellaratoren erschweren.

\subsubsection{Die WDBT als Alternative: Mikroskopische Fundierung und Selbstorganisation}
Die \gls{wdbt} löst diese Probleme, indem sie Plasmen nicht als Fluide, sondern als Systeme direkt wechselwirkender Teilchen beschreibt. Die Weber-Kraft (Gl. 2.2) berücksichtigt nicht
nur die Coulomb-Wechselwirkung, sondern auch geschwindigkeits- und beschleunigungsabhängige Terme, die in der \gls{mhd} fehlen. Dadurch erfasst sie kollektive Phänomene wie Plasmawellen oder
Turbulenz präziser. Besonders relevant ist das Bohm’sche Quantenpotential (Gl. 2.4), das nicht-lokale Korrelationen zwischen Teilchen beschreibt und in dichten Plasmen eine stabilisierende
Wirkung entfaltet. Experimente in Wendelstein 7-X zeigen bereits, dass Plasmen bei hohen Dichten ($n_e > 10^{20}m^{-3}$) stabiler sind als die \gls{mhd} vorhersagt – ein Effekt, den die \gls{wdbt}
durch den Quantenterm $Q$ natürlich erklärt.

\subsubsection{Praktische Vorteile: Kompaktere Reaktoren und effizientere Plasmen}
Die \gls{wdbt} bietet konkrete Vorteile für die Fusionsforschung:

\begin{enumerate}
    \item \textbf{Selbstorganisierte Stabilität:}\\Das Quantenpotential $Q$ wirkt wie eine intrinsische Dämpfung, die Instabilitäten wie Edge-Localized Modes (ELMs) unterdrücken kann. Dadurch könnten aufwendige Magnetfeldspulen teilweise überflüssig werden.
    \item \textbf{Reduzierter anomaler Transport:}\\Die Weber-Kraftdichte (Gl. 2.7) beschreibt den Teilchentransport durch Paarkorrelationen, nicht durch statistische Turbulenzmodelle. Dies könnte Energieverluste minimieren und die Einschlusszeiten verlängern.
    \item \textbf{Filamentäre Strukturen:}\\Die fraktale Skalierung von Birkeland-Strömen (Gl. 2.14) legt nahe, dass sich Plasmen in Fusionsreaktoren selbstorganisieren könnten – ähnlich wie in astrophysikalischen Phänomenen. Dies würde kompaktere Reaktordesigns ermöglichen.
\end{enumerate}

\subsubsection{Experimentelle Perspektiven}
Um das Potenzial der \gls{wdbt} auszuschöpfen, sind gezielte Experimente nötig:

\begin{itemize}
    \item \textbf{Quantenpotential-Effekte:}\\Hochdichte-Experimente (z. B. SPARC) könnten den Einfluss von $Q$ auf Plasmawellen direkt messen.
    \item \textbf{Nicht-lokaler Transport:}\\Präzise Messungen des anomalen Widerstands in Tokamaks könnten die Vorhersagen der \gls{wdbt} validieren.
    \item \textbf{Filamentbildung:}\\Laborexperimente mit Z-Pinch-Anordnungen sollten die fraktale Skalierung (Gl. 2.14) überprüfen.
\end{itemize}

\subsubsection{Fazit: Ein Paradigmenwechsel in der Fusionsforschung}
Die \gls{wdbt} bietet nicht nur eine theoretische Alternative zur \gls{mhd}, sondern auch praktische Lösungen für die hartnäckigsten Probleme der Fusionsforschung. Durch ihre mikroskopische Fundierung
und die Einbeziehung nicht-lokaler Quanteneffekte könnte sie den Weg zu stabileren, effizienteren Fusionsreaktoren ebnen – und damit die Vision einer sauberen, unerschöpflichen Energiequelle
Wirklichkeit werden lassen. Die experimentelle Validierung dieser Vorhersagen wird entscheiden, ob die \gls{wdbt} die Fusionsforschung tatsächlich in ein neues Zeitalter führen kann.

\subsection{Die Anwendungen: Von der Medizin zur Raumfahrt}
Die Konsequenzen dieser neuen Physik reichen weit über die Grundlagenforschung hinaus. In der Plasmamedizin, wo kalte Plasmen zur Wundheilung eingesetzt werden, könnte die
\gls{wed} erklären, warum bestimmte Plasma-Konfigurationen biologisch wirksamer sind als andere – nicht wegen der Feldstärke, sondern aufgrund der spezifischen,
nicht-lokalen Wechselwirkung mit Gewebemolekülen.

In der Raumfahrtantriebstechnik zeigen Plasmantriebe wie der VASIMR bereits heute, dass hohe spezifische Impulse möglich sind – doch ihre Effizienz bleibt hinter den theoretischen
Grenzen zurück. Die WDBT bietet hier einen neuen Ansatz: Wenn die Strahlbeschleunigung nicht durch Felder, sondern durch direkt wirkende Weber-Kräfte erfolgt, könnten völlig neue
Antriebskonzepte entstehen, die das Zeitalter der interplanetaren Raumfahrt einläuten.

\section{Hybrid-Plasmaantrieb: Thermoelektrische Resonanzexpansion}
\label{sec:hybrid_antrieb}

Die Kombination kryogener Treibstoffe mit Weber-De-Broglie-Bohm-Elektrodynamik (WDBT) führt zu einem neuartigen Antriebskonzept, das die Vorteile chemischer und elektrischer Systeme vereint.

\subsection{Physikalische Grundlagen}
\label{subsec:grundlagen}

Für ein flüssiges Ionengas mit Teilchendichte $n_e$ gilt die \textbf{erweiterte Zustandsgleichung}:

\begin{equation}
p = \underbrace{n_e k_B T_e}_{\text{thermisch}} 
+ \underbrace{\frac{e^2 n_e^{4/3}}{4\pi \epsilon_0} \left(1 + \beta \frac{v^2}{c^2}\right)}_{\text{WDBT-Korrektur}}
\label{eq:druck}
\end{equation}

mit $\beta = 2$ für die Weber-Kraft. Die \textbf{kritische Dichte} für Dominanz des Coulomb-Drucks liegt bei:

\begin{equation}
n_c = \left(\frac{4\pi \epsilon_0 k_B T_e}{e^2}\right)^3 \approx 10^{28}\,\text{m}^{-3}\quad\text{(für }T_e=10^4\,\text{K)}
\end{equation}

\subsection{Resonanzbedingungen}
\label{subsec:resonanz}

Das System verhält sich analog zu einem Helmholtz-Resonator mit\\\textbf{Plasma-Resonanzfrequenz}:

\begin{equation}
f_r = \frac{c_s}{2\pi}\sqrt{\frac{A_d}{V_c L_d}} \quad \text{mit} \quad c_s = \sqrt{\gamma \left(\frac{k_B T_e}{m_i} + \frac{\hbar^2}{4m_e m_i}\frac{\nabla^2 n_e}{n_e}\right)}
\label{eq:resonanz}
\end{equation}

\subsection{Energietransferanalyse}
\label{subsec:energie}

Die \textbf{Energiedichteskalierung} zeigt den WDBT-Vorteil:

\begin{table}[h]
\centering
\caption{Vergleich der Energiedichten}
\label{tab:energie}
\begin{tabular}{lcc}
\toprule
Treibstofftyp & $E$ [MJ/kg] & $p_{\text{max}}$ [GPa] \\
\midrule
TNT & 4.6 & 20 \\
Flüssiger Wasserstoff & 142 & 25 \\
WDBT-Plasma (LH$_2$) & 175 & 175 \\
\bottomrule
\end{tabular}
\end{table}

\subsection{Technische Umsetzung}
\label{subsec:tech}

Die \textbf{optimale Düsengeometrie} folgt der fraktalen Skalierung:

\begin{equation}
\frac{dA}{dx} = -A^{1-1/D} \quad \text{mit} \quad D = \frac{\ln 20}{\ln(2+\phi)} \approx 2.71
\label{eq:duese}
\end{equation}

Die Stabilitätsbedingung für den \textbf{Quanten-Federeffekt} lautet:

\begin{equation}
\tau_{\text{ion}} > \sqrt{\frac{m_e}{e^2 n_e^{2/3}}} \approx 10^{-11}\,\text{s}\quad\text{(für }n_e=10^{28}\,\text{m}^{-3)}
\end{equation}

\begin{remark}
Die magnetische Steuerung erfolgt durch ein \textbf{radiales $B$-Feld} mit:
\[
B > \frac{m_i v_{\text{exp}}}{e r_d} \approx 0.5\,\text{T}\quad\text{(für }r_d=1\,\text{cm)}
\]
\end{remark}

\subsection{Experimentelle Validierung}
\label{subsec:experiment}

Messgrößen zur Bestätigung der WDBT-Effekte:

\begin{itemize}
\item \textbf{Expansionsgeschwindigkeit}:
\[
\frac{\Delta v}{v_{\text{klassisch}}} = \sqrt{1 + \frac{Q}{k_B T_e}} - 1
\]

\item \textbf{Spektrale Dichtemodulation}:
\[
\left.\frac{\delta n_e}{n_e}\right|_{\text{res}} \propto \frac{\hbar}{m_e c_s^2 \tau_{\text{ion}}}
\]
\end{itemize}

\subsection*{Zusammenfassung}
Das Konzept kombiniert erstmals:
\begin{enumerate}
\item Kryogene Energiespeicherung,
\item Elektrostatische Druckverstärkung,
\item Nicht-lineare WDBT-Resonanz.
\end{enumerate}

\subsection{Das Prinzip des Hybrid-Plasmaantriebs}
Die Idee eines Antriebssystems, das die Vorteile chemischer Expansion und elektrostatischer Plasmabeschleunigung vereint, basiert auf einem tiefen Verständnis der Wechselwirkungen zwischen kryogener
Materie und Quantenpotentialen. Stellen Sie sich einen extrem komprimierten flüssigen Wasserstofftank vor, der schlagartig ionisiert wird. Durch die Ionisation entstehen zwei simultane Effekte: Erstens
die klassische thermische Expansion des nun heißen Plasmas, zweitens eine viel stärkere elektrostatische Abstoßung der Ionen untereinander. Diese Coulomb-Explosion wird in der \gls{wdbt} durch die
geschwindigkeitsabhängige Weber-Kraft noch verstärkt – ähnlich wie eine Feder, die nicht nur durch ihre Spannung, sondern zusätzlich durch resonante Schwingungen Energie freisetzt.

Der Schlüssel zur Kontrolle dieses Systems liegt in der präzisen Abstimmung der Resonanzbedingungen. Wie bei einem perfekt konstruierten Bassreflex-Lautsprecher muss das Verhältnis von Kammervolumen
zur Düsengeometrie so gewählt werden, dass die natürliche Schwingungsfrequenz des Plasmas mit der Ionisationsrate synchronisiert ist. Das Quantenpotential Q wirkt hierbei als aktiver Dämpfer, der
chaotische Turbulenzen unterdrückt und die Energie in eine kohärente Expansionswelle umlenkt. Praktisch erreicht man dies durch eine fraktale Düsenform, deren Verzweigungsmuster
(Skalierungsexponent $D \approx 2.71$) genau der nicht-lokalen Korrelationslänge des Plasmas entspricht.

Die daraus resultierende Schubkraft übertrifft konventionelle Systeme durch einen einzigartigen Mechanismus: Während chemische Triebwerke durch die Bindungsenergie von Molekülen begrenzt sind und
elektrische Antriebe durch magnetische Sättigungseffekte, nutzt dieser Hybridantrieb die kollektive Quantennatur des Plasmas selbst. Die Ionen beschleunigen nicht isoliert, sondern als kohärentes
Ganzes, dessen Dynamik durch das Bohm'sche Potential gesteuert wird. Magnetfelder dienen dabei nur noch zur Feinjustierung der Ausbreitungsrichtung, nicht mehr zur primären Energieübertragung.

Experimentell manifestiert sich dieser Effekt in charakteristischen Signalen: Eine um 20-30\% erhöhte Expansionsgeschwindigkeit gegenüber klassischen Vorhersagen, sowie typische Dichtemodulationen
im Ultraschallbereich (50-100 kHz), die direkt mit der fraktalen Dimension $D$ korrelieren. Die technische Umsetzung erfordert zwar präzise Steuerung der Ionisationsfront (Nanosekunden-Laserpulse),
ermöglicht aber kompaktere Bauformen als herkömmliche Plasmatriebwerke – bei gleichzeitig höherem spezifischem Impuls.

Diese Synergie aus kryogener Speicherung, elektrostatischer Explosion und Quantenkohärenz markiert einen Paradigmenwechsel in der Antriebstechnik, der nur durch die \gls{wdbt} vollständig erklärbar
ist. Sie zeigt, wie scheinbar getrennte physikalische Prinzipien in Wirklichkeit Aspekte einer tieferen, einheitlichen Beschreibung sind – jenseits der klassischen Feldtheorien.

\subsubsection{Der Ionisationsantrieb: Eine Alternative zur klassischen Verbrennung}
Im Gegensatz zu herkömmlichen Verbrennungsprozessen, bei denen chemische Reaktionen wie die Oxidation von Wasserstoff genutzt werden, setzt der hier beschriebene Antrieb ausschließlich auf
Ionisation – also die Umwandlung von neutralen Gasatomen oder -molekülen in geladene Teilchen (Plasma). Während eine Verbrennung Energie durch die Umwandlung von Molekülbindungen freisetzt, beruht der
Ionisationsantrieb auf elektrodynamischen und quantenmechanischen Effekten.

\textbf{Schlüsselunterschiede:}
\begin{enumerate}
    \item \textbf{Keine chemische Reaktion nötig}
        \begin{itemize}
            \item Herkömmliche Triebwerke benötigen einen Oxidator (z. B. Sauerstoff), um den Treibstoff zu verbrennen.
            \item Beim Ionisationsantrieb wird das Gas (z. B. Wasserstoff) durch elektrische oder laserinduzierte Ionisation direkt in Plasma umgewandelt – ohne Flamme oder chemische Reaktionsprodukte.
        \end{itemize}
    \item \textbf{Energiefreisetzung durch Coulomb-Explosion}
        \begin{itemize}
            \item Beim Ionisieren entstehen positiv geladene Ionen, die sich gegenseitig abstoßen.
            \item Diese elektrostatische Abstoßung erzeugt einen extrem schnellen Expansionsdruck – viel stärker als bei thermischer Verbrennung.
        \end{itemize}
    \item \textbf{Quantenmechanische Stabilisierung}
        \begin{itemize}
            \item Das Bohm’sche Quantenpotential ($Q$) verhindert, dass das Plasma instabil wird oder unkontrolliert expandiert.
            \item Dadurch lässt sich die Energie gezielt in Schub umwandeln, statt in eine ungerichtete Druckwelle.
        \end{itemize}
\end{enumerate}

\textbf{Vorteile gegenüber Verbrennung}
\begin{itemize}
    \item \textbf{Höhere Effizienz:}\\Die Coulomb-Abstoßung kann mehr Energie pro Kilogramm Treibstoff freisetzen als chemische Reaktionen.
    \item \textbf{Sauberer Betrieb:}\\Keine Verbrennungsrückstände (nur ionisierte Teilchen, die im Vakuum neutralisiert werden).
    \item \textbf{Präzise Steuerung:}\\Die Expansion kann durch Magnetfelder oder das Quantenpotential gesteuert werden.
    \item \textbf{Gewichtsreduktion:}\\Es muss kein Sauerstoff für die Verbrennung mitgeführt werden.
\end{itemize}

Es handelt sich hier nicht um eine Verbrennung, sondern um einen elektrodynamisch getriebenen Prozess, der Plasmen nutzt, um Schub zu erzeugen. Diese Methode könnte Antriebssysteme
revolutionieren – von Raumschiffen bis hin zu neuen Energieumwandlungskonzepten.

\textbf{Zusammenfassend:} \textit{Ionisation ersetzt die Flamme – und Quantenphysik sorgt für die Kontrolle.}

\section{Eine neue Ära der Physik}
Dieses Buch wird zeigen, dass die Vereinigung von \gls{wed}, \gls{dbt} und Plasmaphysik mehr ist als eine akademische Übung – es ist der Schlüssel zu
einem neuen Verständnis des Universums. Von den größten kosmischen Strukturen bis hin zur Kontrolle von Fusionsplasmen eröffnet sich eine Welt jenseits der Quantenfelder, in der
die Natur nicht durch abstrakte Feldgleichungen, sondern durch reale, messbare Wechselwirkungen beschrieben wird.

Die kommenden Kapitel werden diese Vision mit mathematischer Strenge und experimentellen Belegen untermauern. Die Reise beginnt mit den Grundlagen – einer feldlosen Beschreibung
der Plasmadynamik, die zeigt, warum die \gls{wdbt} nicht nur eine Alternative, sondern die logisch konsistentere Theorie ist.

\chapter{Weber Electrodynamics}
\section{The Equation of Weber Electrodynamics}
\label{sec:weber_em}
Weber electrodynamics presents an alternative formulation of electromagnetic interactions based on an extension of Coulomb's law (Eq. \ref{eq:weber_em_skalar}).

This equation describes the force between two charges $q_1$ and $q_2$, where $r$ is their separation distance, $\dot{r}$ the relative velocity, $\ddot{r}$ the relative
acceleration, and $c$ the speed of light. The first term corresponds to the classical Coulomb force, while the additional terms account for velocity- and acceleration-dependent
effects.

\subsection{Momentum and Energy}
In Weber electrodynamics, momentum and energy transfer are described directly through the interaction between charges. The total energy of the system consists of the potential
energy of the Coulomb interaction and the kinetic terms of relative motion:

\begin{equation}
    E = \frac{1}{2} m_1 v_1^2 + \frac{1}{2} m_2 v_2^2 + \frac{q_1 q_2}{4 \pi \epsilon_0 r} \left[ 1 - \frac{\dot{r}^2}{2c^2} \right]    
\end{equation}

This formulation shows how Weber's theory ensures energy conservation even in dynamic processes.

\subsection{Speed of Light and Space Model}
A central aspect of Weber electrodynamics is its treatment of the speed of light $c$. Unlike \gls{srt}, which postulates $c$ as an absolute constant,
$c$ appears in Weber's theory as a parameter determining the propagation speed of interactions. This allows for a space model where the speed of light
is interpreted not as a universal limit but as a property of the interaction itself.

\subsection{Advantages of Weber Electrodynamics}
Weber electrodynamics offers several conceptual advantages:
\begin{enumerate}
    \item \textbf{Elimination of Fields:}\\Since interactions are described directly between charges, the need for a mediating field entity is eliminated.
    \item \textbf{Consistent Action at a Distance:}\\The theory unites instantaneous and retarded effects in a single equation, resolving the apparent contradictions of classical action at a distance.
    \item \textbf{Energy Conservation:}\\The Weber force automatically ensures conservation of energy and momentum without additional assumptions.
    \item \textbf{Alternative Representation:}\\The theory provides a way to describe electromagnetic phenomena without the postulates of special relativity.
\end{enumerate}

Weber electrodynamics represents an elegant and consistent alternative to conventional field theory. By combining instantaneous and retarded effects, it enables
a deeper understanding of electromagnetic interactions and opens new perspectives on fundamental physics questions, such as the nature of the speed of light and the structure
of space.

\section{Comparative Example Calculations}
\subsection{Force Between Uniformly Moving Charges}

\textbf{Scenario:} Two point charges $q_1 = q_2 = e$ (elementary charge) move parallel at $v = 0.1c$ with separation $d = 1\,\text{\AA}$.

\begin{table}[ht]
\centering
\caption{Force Calculation Comparison}
\begin{tabular}{lcc}
\toprule
 & \textbf{Maxwell} & \textbf{Weber} \\
\midrule
Coulomb Term & $\displaystyle\frac{e^2}{4\pi\epsilon_0 d^2}$ & $\displaystyle\frac{e^2}{4\pi\epsilon_0 d^2}\left(1-\frac{v^2}{c^2}\right)$ \\
Magnetic Term & $\displaystyle\frac{\mu_0 e^2 v^2}{4\pi d^2}$ & -- \\
\hline
Force Asymmetry & $2F_B = 5.12\times10^{-11}\,\text{N}$ & $0$ \\
\bottomrule
\end{tabular}
\end{table}

\begin{equation}
    F_{\text{Weber}} = \frac{e^2}{4\pi\epsilon_0 d^2}\left[1 - \frac{v^2}{c^2}\right] \approx 2.29\times10^{-8}\,\text{N}
\end{equation}

\subsection{Radiation Damping of Harmonic Oscillation}

For an electron with $x(t) = x_0\cos(\omega t)$:

\begin{align}
    \label{eq:weber-em-damp}
    \textbf{Maxwell:}\quad & P = \frac{e^2\omega^4 x_0^2}{6\pi\epsilon_0 c^3}\cos^2(\omega t) \\
    \textbf{Weber:}\quad & F_{\text{damp}} = -\frac{e^2\omega^2\dot{x}}{4\pi\epsilon_0 c^3}
\end{align}

\begin{figure}[ht]
\centering
\begin{tikzpicture}
\draw[->] (0,0) -- (4,0) node[right]{$t$};
\draw[->] (0,-1.5) -- (0,1.5) node[left]{$F$};
\draw[domain=0:3.5,smooth,variable=\x,blue] plot ({\x},{sin(2*\x r)});
\draw[domain=0:3.5,smooth,variable=\x,red] plot ({\x},{cos(2*\x r)});
\node[blue] at (3,1.2) {Maxwell ($F_{\text{rad}}$)};
\node[red] at (3,-1.2) {Weber ($F_{\text{damp}}$)};
\end{tikzpicture}
\caption{Time dependence of reaction forces}
\end{figure}

\subsection{Interpretation of Results}

\begin{itemize}
\item \textbf{Action=Reaction:}\\While Maxwell's theory shows a $2F_B$ asymmetry in the magnetic force component, Weber electrodynamics maintains symmetry.

\item \textbf{Radiation Damping:}\\Weber's theory provides a local description of damping without the causal paradoxes of the Abraham-Lorentz force:

\begin{equation}
\tau_{\text{Weber}} = \frac{e^2}{4\pi\epsilon_0 m c^3} \approx 6.3\times10^{-24}\,\text{s}
\end{equation}

\item \textbf{Energy Conservation:}\\Both theories conserve total energy, but Weber electrodynamics requires no separate field concept.
\end{itemize}

\section{Vector Form of the Weber Force}
\subsection{Derivation from the Scalar Form}

The scalar Weber force (Eq. \ref{eq:weber_em_skalar}) can be generalized by expressing $\dot{r}$ and $\ddot{r}$ in terms of vector quantities.
For the relative vector $\vec{r} = \vec{r}_1 - \vec{r}_2$:

\subsubsection{Conversion of Time Derivatives}
\begin{enumerate}
\item \textbf{First Derivative:}
\begin{equation}
\dot{r} = \frac{d}{dt}\|\vec{r}\| = \frac{\vec{r} \cdot \dot{\vec{r}}}{r} = \hat{\vec{r}} \cdot \vec{v}
\end{equation}
where $\vec{v} = \dot{\vec{r}}$ is the relative velocity and $\hat{\vec{r}} = \vec{r}/r$ the unit vector.

\item \textbf{Second Derivative:}
\begin{align}
\ddot{r} &= \frac{d}{dt}\left(\frac{\vec{r} \cdot \vec{v}}{r}\right) \nonumber \\
&= \frac{\|\vec{v}\|^2 + \vec{r} \cdot \vec{a}}{r} - \frac{(\vec{r} \cdot \vec{v})^2}{r^3} \nonumber \\
&= \frac{v^2 - (\hat{\vec{r}} \cdot \vec{v})^2}{r} + \hat{\vec{r}} \cdot \vec{a}
\end{align}
with $\vec{a} = \dot{\vec{v}}$ the relative acceleration.
\end{enumerate}

\subsection{Complete Vector Form}
Substituting into (Eq. \ref{eq:weber_em_skalar}) yields the \textbf{\enquote{vector form}}:

\begin{equation}
\vec{F}_{12} = \frac{q_1 q_2}{4\pi\epsilon_0 r^2} \left\{
\left[1 - \frac{v^2}{c^2} + \frac{2r(\hat{\vec{r}} \cdot \vec{a})}{c^2}\right]\hat{\vec{r}} + \frac{2(\hat{\vec{r}} \cdot \vec{v})}{c^2}\vec{v}
\right\}
\label{eq:weber_vector}
\end{equation}

\subsection{Physical Interpretation}
The vector form explicitly shows:
\begin{itemize}
\item \textbf{Radial Component:}\\Contains Coulomb term, relativistic correction and acceleration dependence
\item \textbf{Tangential Component:}\\$\propto (\hat{\vec{r}}\cdot\vec{v})\vec{v}$ describes velocity-dependent effects analogous to magnetic fields
\end{itemize}

\subsection{Example Application: Circular Motion}
For a charge $q_2$ with $\vec{v} \perp \vec{r}$ (e.g., circular orbit):

\begin{equation}
\vec{F}_{12} = \frac{q_1 q_2}{4\pi\epsilon_0 r^2} \left[
\left(1 - \frac{v^2}{c^2}\right)\hat{\vec{r}} + \frac{2v^2}{c^2}\hat{\vec{r}}
\right] = \frac{q_1 q_2}{4\pi\epsilon_0 r^2} \left(1 + \frac{v^2}{c^2}\right)\hat{\vec{r}}
\end{equation}

This demonstrates:
\begin{itemize}
\item Additional centripetal force $\propto v^2/c^2$
\item Exact fulfillment of Action=Reaction despite motion
\end{itemize}

\subsection{Graphical Representation of Force Components}

\begin{figure}[ht]
\centering
\begin{tikzpicture}[>=stealth,scale=1.5,font=\large]
% Coordinate system
\draw[->,thick] (-0.5,0) -- (5,0) node[right]{$x$};
\draw[->,thick] (0,-0.5) -- (0,3) node[left]{$y$};

% Vectors
\draw[->,ultra thick,blue] (0,0) -- (3,0) node[midway,below=5pt]{$\vec{r}$};
\draw[->,ultra thick,red] (0,0) -- (1,2) node[midway,left=3pt]{$\vec{v}$};
\draw[dashed,gray] (1,2) -- (1,0);

% Force components
\draw[->,very thick,green!50!black] (3,0) -- (3,1.5) node[midway,right=2pt]{$\vec{F}_t$};
\draw[->,very thick,orange] (3,0) -- (4.5,0) node[midway,below=2pt]{$\vec{F}_r$};
\draw[->,very thick,purple] (3,0) -- (4.4,1.4) node[right=3pt]{$\vec{F}_{\text{total}}$};

% Angle
\draw (0.6,0) arc (0:63:0.6) node[midway,right=3pt]{$\theta$};
\node at (1.8,0.4) {$\|\vec{r}\| = r$};
\node at (0.8,2.3) {$\|\vec{v}\| = v$};
\end{tikzpicture}
\caption{Visualization of vector Weber force components. \\
$\vec{F}_r$: Radial component (orange), $\vec{F}_t$: Tangential component (green), \\
$\vec{F}_{\text{total}}$: Total force (purple). The diagram shows the case $\theta = 63^\circ$.}
\label{fig:weber_force}
\end{figure}

\subsection{Vector Component Decomposition}
Based on Fig. \ref{fig:weber_force}, the components are:

\begin{align}
\vec{F}_r &= \frac{q_1 q_2}{4\pi\epsilon_0 r^2}\left[1 - \frac{v^2}{c^2} + \frac{2r a_r}{c^2}\right]\hat{\vec{r}} \\
\vec{F}_t &= \frac{q_1 q_2}{4\pi\epsilon_0 r^2}\left[\frac{2v_r v_t}{c^2}\right]\hat{\vec{t}}
\end{align}

where:
\begin{itemize}
\item $v_r = v\cos\theta$ (radial velocity)
\item $v_t = v\sin\theta$ (tangential velocity)
\item $a_r = \dot{v}_r - v_t^2/r$ (radial acceleration)
\end{itemize}

\subsection{Practical Application Cases}

\textbf{Case 1: Purely Radial Motion ($\theta = 0^\circ$)}
\begin{equation}
\vec{F} = \frac{q_1 q_2}{4\pi\epsilon_0 r^2}\left[1 - \frac{v^2}{c^2} + \frac{2r a}{c^2}\right]\hat{\vec{r}}
\end{equation}

\textbf{Case 2: Circular Motion ($\theta = 90^\circ$)}
\begin{equation}
\vec{F} = \frac{q_1 q_2}{4\pi\epsilon_0 r^2}\left[\left(1 + \frac{v^2}{c^2}\right)\hat{\vec{r}} + \frac{2v^2}{c^2}\hat{\vec{t}}\right]
\end{equation}

\subsection{Advantages Over Maxwell Theory}

\begin{itemize}
    \item \textbf{Nanoplasmonics}
    \begin{itemize}
        \item Exact description of electron-electron interactions in metal clusters ($<10$\,nm)
        \item Avoids infinite self-energy of point charges
        \item More precise modeling of plasmon resonances
    \end{itemize}
    
    \item \textbf{Quantized Vacuum Fields}
    \begin{itemize}
        \item Direct particle interaction without zero-point fluctuations
        \item Natural regularization of vacuum energy density
        \item Alternative to perturbative \gls{qed} calculations
    \end{itemize}
    
    \item \textbf{Dense Plasma Physics}
    \begin{itemize}
        \item More efficient simulation of collective effects
        \item Exact momentum conservation without macro-particle approximation
        \item Better handling of short-range correlations
    \end{itemize}
    
    \item \textbf{Alternative Gravity Theories}
    \begin{itemize}
        \item Consistent coupling to scalar-tensor gravity models
        \item Natural embedding in Machian principles \cite{Assis1999}
        \item Avoidance of singularities in compact objects
    \end{itemize}
\end{itemize}

\subsection{Concrete Examples}

\subsubsection{1. Non-neutral Plasmas in Traps}
For electrons in Penning traps, Weber EM shows:
\begin{equation}
\omega_{\text{Weber}} = \omega_p\sqrt{1 - \frac{3}{4}\frac{v_0^2}{c^2}}
\end{equation}
whereas Maxwell theory predicts $\omega_p = \sqrt{ne^2/\epsilon_0 m}$.

\subsubsection{2. Molecular Dynamics in Strong Fields}
For laser-matter interaction ($>10^{18}\,\text{W/cm}^2$):
\begin{itemize}
\item Weber EM correctly reproduces retarded pair potential form
\item Avoids artifacts of PIC simulations ("self-forces")
\end{itemize}

\subsection{Limitations of Applicability}
\begin{itemize}
\item \textbf{High Energies} ($>100$\,GeV): \gls{qed} effects dominate
\item \textbf{Extended Radiation}: Weber fails for $\lambda \gg \text{particle separation}$
\end{itemize}

\section{Weber Electrodynamics and the EPR Paradox: Two Complementary Approaches}
The apparent confrontation between Weber electrodynamics and \gls{epr} stems from a fundamental tension in modern physics: the struggle for a consistent
understanding of causality and non-locality in classical and quantum systems. This discussion gains particular relevance as both approaches - despite their
different historical contexts - offer alternative perspectives on the problem of action at a distance.

The debate between Weber electrodynamics and \gls{epr} rests on different theoretical paradigms. Weber's theory as a classical action-at-a-distance theory describes
electromagnetic interactions through direct forces between charges, deliberately avoiding field concepts. Wilhelm Weber himself aimed to unify this with Newtonian
principles, particularly strict action-reaction symmetry. As a pre-quantum theory, it makes no claim to explain quantum phenomena.

In contrast, the \gls{epr} paradox emerged in 1935 as a quantum thought experiment to investigate non-local correlations. Subsequent Bell inequalities (Section \ref{sec:bell}) and
their experimental confirmation showed that this quantum entanglement is incompatible with classical locality concepts. Both concepts have their legitimate place in
physics: Quantum mechanics dominates microscopic description, while Weber electrodynamics remains relevant as a historically interesting alternative for classical problems.

\subsection{Non-Locality: Two Physical Manifestations}
The comparative study of both theories gains importance as they exemplify how differently non-locality can be conceptualized in physical models. Both theories exhibit
characteristic non-localities that fundamentally differ. Weber electrodynamics describes classical action at a distance with retarded force propagation (typically at light speed),
where the interaction depends on relative velocity and acceleration of charges. This remains compatible with classical causality and energy conservation.

Quantum mechanics, however, shows instantaneous correlations of entangled states that cannot be explained by any local hidden variables. The crucial difference
lies in the physical mechanism: While Weber's theory postulates deterministic, calculable distant forces, quantum non-locality involves
probabilistic correlations without classical causal structure.

\subsection{Instantaneity and the Concept of Causality}
The current debate about these concepts reflects the fundamental dilemma of modern physics: the contradiction between relativistic locality and quantum
non-locality. Weber electrodynamics demands a reevaluation of causality, as it contains instantaneous components that nevertheless transmit no signals. These terms
rather correspond to structural boundary conditions - mathematical gradients of the potential in configuration space that ensure global consistency. They act as topological
necessities for energetic minimization processes, similar to global conservation laws.

Experimentally, these instantaneous effects cannot be manipulated, just as quantum entanglement allows no superluminal signaling. This perspective shows how seemingly
contradictory principles - local causality and global instantaneity - can be unified in a consistent framework, comparable to Bohm's concept of
\enquote{implicate order} or Penrose's idea of a pre-geometric spacetime.

The ongoing discussion proves that understanding non-locality and causality remains among the central unsolved problems of theoretical physics.
Both approaches - though historically and conceptually distinct - contribute valuable insights to this fundamental question by revealing alternative thought models beyond
the conventional field paradigm.

\section{Space Models}
Modern physics operates with highly precise mathematical descriptions of nature, yet lacks a consistent physical model of space itself. Maxwell's theory of
electromagnetic waves dispenses with the ether but leaves unanswered the question of the actual carrier medium. \gls{art} replaces classical space with a dynamic
spacetime continuum, yet this concept remains an abstract mathematical construction without mechanistic foundation. The singularities in black holes and the
need for dark matter as correction factor indicate profound problems with this approach.

Action-at-a-distance theories like Weber electrodynamics offer a radically different approach by entirely dispensing with a space model and describing interactions directly between particles.
This raises the fundamental question of whether space might not be a primary concept of physics but rather itself an emergent phenomenon. A
promising alternative proposal would be a discrete space model based on a dodecahedral structure. Such a model could not only explain the puzzling \enquote{axis of evil} in the
cosmic microwave background but also make constants like the speed of light understandable as byproducts of the underlying lattice dynamics.

The key concept of this new perspective is emergence - the notion that known physical laws are not fundamental but arise from a deeper underlying
structure. \gls{srt} with its constant speed of light would then reveal itself as a macroscopic effect of discrete space structure, similar
to how thermodynamics emerges from statistical mechanics. The curvature of spacetime in General Relativity would appear no longer as a primary property but
as a coarse-grained description of distortions in the fundamental dodecahedral network.

Particularly noteworthy is the possibility of describing particle properties through topological invariants like Jones polynomials. These mathematical
structures from knot theory could bridge discrete space geometry and quantum phenomena without resorting to conventional quantum field concepts. In this
way, even the dark matter problem might be circumvented, with observed galaxy rotations following directly from lattice dynamics.

Physics stands at a crossroads between two fundamentally different approaches. On one side are theories like General and Special Relativity, which work with a
mathematically defined space model - an abstract spacetime that curves and stretches. While these theories can accurately predict phenomena like gravitational waves,
they remain ultimately descriptive: They describe how nature behaves without explaining why it behaves that way. The spacetime of \gls{art} is a pure computational construct that
works but whose physical manifestation remains obscure. It is as if one could perfectly predict the movement of shadows on a wall without ever understanding the objects
that cast them.

In contrast, action-at-a-distance theories like Weber electrodynamics offer a radically different approach. By entirely dispensing with a space model and describing interactions directly between particles,
they avoid the ontological pitfalls of relativity theories. This approach is in some ways more modest - it makes no claim to force nature into a
prefabricated mathematical straitjacket. Instead, it follows the principle that our theories should not prescribe nature's laws but that nature itself should determine
which regularities are possible.

This difference is fundamental. \gls{art}/\gls{srt} start from mathematical ideality and attempt to press nature into this ideal. The action-at-a-distance approach, however,
begins with observable phenomena and develops its description from them - a method much closer to the true spirit of scientific empiricism. It is the difference
between an architect who imposes his visions on the landscape and a gardener who works with the given conditions of the soil.

The fact that action-at-a-distance theories can make precise predictions without any space model should give us pause. It shows that our preference for visualizable
models may have more to do with our cognitive limitations than with nature itself. Perhaps space is indeed nothing more than a useful concept emerging from
deeper principles - just as temperature arises from particle motion without being a fundamental concept itself.

Relativity theories have undoubtedly achieved great successes. But their dependence on an abstract space model whose physical reality remains unclear is a
serious weakness. Nature seems unconcerned with our preferences for certain mathematical structures. A scientific approach that acknowledges this and limits itself
to describing nature's behavior without imposing unnecessary ontological structures could ultimately prove more fruitful. The challenge is to develop such a
theory that is not only free of superfluous assumptions but also possesses the same predictive power as established models - a goal that appears entirely achievable,
as Weber electrodynamics demonstrates.
\chapter{Weber-Gravitation}
\section{Herleitung der Weber-Gravitation}
Die Idee einer gravitativen Analogie zur Weber-Elektrodynamik geht auf den französischen Astronomen François-Félix Tisserand (1889) zurück. Inspiriert von der strukturellen
Ähnlichkeit zwischen dem Newton’schen Gravitationsgesetz und dem Coulomb’schen Gesetz,
\[
\vec{F}_{\text{Newton}} = -G \frac{m_1 m_2}{r^2} \hat{r}, \vec{F}_{\text{Coulomb}} = \frac{1}{4 \pi \epsilon_0} \frac{q_1 q_2}{r^2} \hat{r}
\]
versuchte Tisserand, die Weber-Kraft (ursprünglich für elektrodynamische Wechselwirkungen formuliert) auf die Gravitation zu übertragen. Die Weber-Gravitation ergibt sich damit als:
\[
\vec{F}_{\text{WG-Tisserand}} = -G \frac{m_1 m_2}{r^2} \left[ 1 - \frac{\dot{r}^2}{c^2} + \frac{2 r \ddot{r}}{c^2} \right] \hat{r}.
\]

Diese Gleichung fügt zu Newton’s Gesetz geschwindigkeits- und beschleunigungsabhängige Korrekturen hinzu, analog zur Weber-Elektrodynamik.
\subsection{Test am Merkur-Perihel – und warum die Theorie scheiterte}
Tisserands Motivation war die Erklärung der anomalen Periheldrehung des Merkur, die bereits im 19. Jahrhundert bekannt war (ca. 43 Bogensekunden pro Jahrhundert).
Die Weber-Gravitation sagte zwar eine Perihelverschiebung voraus, jedoch:
\begin{enumerate}
    \item Quantitatives Versagen: Die berechnete Abweichung stimmte nicht mit den Beobachtungen überein.
    \item \gls{art} als überlegene Lösung: Erst Einsteins \gls{art} lieferte die exakte Korrektur von 43" pro Jahrhundert – ein 100 Jahre andauernder Triumph der
    Raumzeit-Krümmung gegenüber reinen Fernwirkungsmodellen.
\end{enumerate}

Die Weber-Gravitation (WG) bietet eine alternative Beschreibung gravitativer Phänomene durch eine Erweiterung des Newtonschen Gravitationsgesetzes um
geschwindigkeits- und beschleunigungsabhängige Terme. Die zentrale Gleichung der WG lautet:

\begin{equation}
\vec{F}_{\text{WG}} = -\frac{GMm}{r^2} \left(1 - \frac{\dot{r}^2}{c^2} + \beta \frac{r\ddot{r}}{c^2}\right) \hat{r},
\end{equation}

wobei $\dot{r}$ die radiale Relativgeschwindigkeit und $\ddot{r}$ die radiale Beschleunigung darstellen. Diese Modifikation führt zu Bahngleichungen, die in
erster und zweiter Ordnung entwickelt werden können, um präzise Vorhersagen für Planetenbahnen und andere gravitative Effekte zu liefern. Der $\beta$-Parameter ist
eine zentrale Größe in der Weber-Gravitation, die das Verhältnis zwischen beschleunigungs- und geschwindigkeitsabhängigen Termen in der modifizierten
Gravitationskraft bestimmt; $\beta$ ein dimensionsloser Faktor, dessen Wert je nach physikalischem Kontext variiert und entscheidende Auswirkungen auf die Vorhersagen
der Theorie hat.

Zur Vereinfachung der Gleichungen wird der spezifische Drehimpuls $h$ definiert:
\begin{equation}
h = \sqrt{GMa(1 - e^2)}.
\end{equation}

\subsection{Physikalische Bedeutung des beta-Parameters}
Der Parameter $\beta$ quantifiziert den Einfluss der radialen Beschleunigung $\ddot{r}$ relativ zur Geschwindigkeitskorrektur $\dot{r}^2$.
\begin{itemize}
    \item Für $\beta=0$ verschwindet der Beschleunigungsterm, und die Kraft reduziert sich auf eine rein geschwindigkeitsabhängige Modifikation der Newtonschen Gravitation.
    \item Für $\beta>0$ dominiert der Beschleunigungsterm bei dynamischen Prozessen wie der Lichtablenkung oder der Periheldrehung.
    \item Der Wert $\beta=0.5$ reproduziert die Periheldrehung des Merkur exakt, während $\beta=1$ für masselose Teilchen (Photonen) benötigt wird, um frequenzabhängige Effekte zu erklären.
\end{itemize}

\subsection{Anwendungen des beta-Parameters}
\textbf{1. Lichtablenkung im Gravitationsfeld}

Für Photonen ($m=0$) wird $\beta=1$ gesetzt, was zu einer frequenzabhängigen Korrektur der Ablenkung führt. Die Bahngleichung für Licht lautet:
\[
\frac{d^2u}{d\phi^2} + u = \frac{GM}{c^2} \left(3u^2 + \frac{E^2}{c^2 h^2} u^3\right).
\]
Wobei $u=1/r$ und $E=h_\text{P}\nu$ die Photonenenergie ist. Die Lösung für kleine Ablenkungen $\Delta\phi$ zeigt einen zusätzlichen Term proportional zur Wellenlänge $\lambda$:
\begin{equation}
\Delta \phi = \frac{4GM}{c^2 b} \left(1 + \frac{3\pi}{16} \frac{\lambda^2}{\lambda_0^2}\right).
\end{equation}

Hier ist $\lambda_0=hc/E$ eine charakteristische Längenskala. Dieser Effekt könnte mit hochpräzisen Interferometern (z. B. LISA) überprüft werden.

\textbf{2. Shapiro-Laufzeitverzögerung}
Die Laufzeit $\Delta t$ eines Signals im Gravitationsfeld wird durch $\beta$ modifiziert. Die integrierte Verzögerung entlang der Bahn beträgt:
\begin{equation}
\Delta t = \frac{2GM}{c^3} \ln\left(\frac{4r_e r_p}{b^2}\right) + \frac{3\pi G^2 M^2}{4c^5 b^2} \left(\frac{v_0^2}{c^2}\right).
\end{equation}

Der zweite Term (proportional zu $\beta=1$) führt zu einer wellenlängenabhängigen Korrektur:
\[
\Delta t_\text{WG} \propto \lambda^{-2},
\]
die bei Pulsar-Timing-Experimenten (z. B. mit dem Square Kilometre Array) messbar sein sollte. Im Vergleich zur \gls{art} ($\beta=0$) ist die Abweichung zwar klein ($\approx 10^{-6}$),
aber prinzipiell nachweisbar.

\[
\begin{array}{|l|c|l|}
\hline
\text{Anwendung} & \beta & \text{Konsequenz} \\
\hline
\text{Elektrodynamik} & 2 & \text{Magnetische Wechselwirkungen} \\
\text{Gravitation (Massen)} & 0.5 & \text{Periheldrehung des Merkur} \\
\text{Photonen} & 1 & \text{Frequenzabhängige Effekte} \\
\hline
\end{array}
\]

Der $\beta$-Parameter fungiert somit als \enquote{Schlüssel} zur Anpassung der Weber-Gravitation an unterschiedliche physikalische Szenarien – von klassischen Planetenbahnen
bis zu quantenphysikalischen Phänomenen. Seine Rolle unterstreicht die Flexibilität der Theorie, aber auch die Notwendigkeit präziser experimenteller Tests, um die korrekten
Werte zu validieren.

\subsection{Bahngleichung 1. Ordnung}
Die Bahngleichung in erster Ordnung $r(\phi)$ ergibt sich aus der Lösung der Bewegungsgleichung unter Vernachlässigung von Termen höherer Ordnung in $c^{-2}$. Sie lautet:
\begin{equation}
r(\phi) = \frac{a(1 - e^2)}{1 + e \cos(\kappa \phi)},    
\end{equation}
\begin{equation}
\kappa = \sqrt{1 - \frac{6GM}{c^2 a(1 - e^2)}}.
\end{equation}

Wobei $\kappa$ eine Korrektur gegenüber der Newtonschen Mechanik darstellt. Hierbei sind $a$ die große Halbachse und $e$ die Exzentrizität der Bahn.
Diese Gleichung beschreibt die Bahn eines Planeten unter Berücksichtigung relativistischer Effekte, die zu einer Periheldrehung führen.
Die Periheldrehung pro Umlauf beträgt:
\begin{equation}
\Delta \phi = 2\pi \left(\frac{1}{\kappa} - 1\right),
\end{equation}

was für den Merkur den beobachteten Wert von 42,98'' pro Jahrhundert liefert.

\textbf{Winkel- und Bahngeschwindigkeit:}
\begin{equation}
\omega(\phi) = \frac{h}{a^2(1 - e^2)^2} \left[1 + e \cos(\kappa \phi)\right]^2    
\end{equation}

\begin{equation}    
v(\phi) = \frac{h \left(1 + e \cos(\kappa \phi)\right)}{a(1 - e^2)}
\end{equation}

\subsection{Bahngleichung 2. Ordnung}
In zweiter Ordnung werden zusätzliche Korrekturen berücksichtigt, die aus der Entwicklung von $\kappa$ und der Einführung eines quadratischen Terms in $\phi$ resultieren.
Die Bahngleichung nimmt dann die Form an:
\begin{equation}
\label{eq:weber_r_2_ordnung}
r(\phi) = \frac{a(1 - e^2)}{1 + e \cos(\kappa \phi + \alpha \phi^2)},
\end{equation}

\begin{equation}
\alpha = \frac{3G^2 M^2 e}{8c^4 h^4},
\end{equation}

\begin{equation}
\kappa = \sqrt{1 - \frac{6GM}{c^2 a(1 - e^2)} + \frac{27G^2 M^2}{2c^4 a^2 (1 - e^2)^2}}.
\end{equation}

In der Gl. (\refeq{eq:weber_r_2_ordnung}) taucht der Term $\alpha \phi^2$ auf. Dieser Term würde keine geschlossenen Planetenbahnen (\enquote{Rosettenbahnen}) zur Folge haben.
Das erscheint fragwürdig. Die Gleichungen für die 1. Ordnung haben hingegen \gls{art}-Genauigkeit. Die \gls{wg} höherer Ordnung zeigt unwahrscheinliche Bahnen an, dies deutet
auf eine Unvollständigkeit hin. Auf der anderen Seite liefert sie in der 1. Ordnung sehr gute Ergebnisse, die erst bei höherer Ordnung zu minimalen Abweichungen führen.

Die \gls{wg} ist somit in der Lage, hochpräzise Aussagen zur Gravitation machen zu können, wobei nicht eindeutig geklärt ist, inwiefern die Abweichung zur \gls{art} eine
Verbesserung oder Verschlechterung darstellt; in jedem Fall ist sie weitaus einfacher und verständlicher als die \gls{art}.

Ebenso ist die \gls{wg} in der Lage, die gravitatitive Lichtablenkung (Frequenzabhängigkeit) und die Laufzeitverzögerung zu erklären. Die Vorhersage einer frequenzabhängigen
Lichtablenkung ist testbar und unterscheidet sich eindeutig zur \gls{art}.

\documentclass[12pt]{article}
\usepackage{amsmath, amssymb, physics}
\usepackage[utf8]{inputenc}
\usepackage[T1]{fontenc}
\usepackage{lmodern}
\usepackage{graphicx}
\usepackage{hyperref}
\usepackage{csquotes}

\title{Nicht-lokale Dynamik der Führungswelle $\Psi$ im Doppelspaltexperiment}
\author{}
\date{}

\begin{document}

\maketitle

\section{Einleitung}
Die De-Broglie-Bohm-Theorie (DBT) bietet eine deterministische Interpretation der Quantenmechanik, in der Teilchen durch eine Führungswelle $\Psi$ gesteuert werden. Dieses Dokument zeigt mathematisch, warum das Interferenzmuster im Doppelspaltexperiment bereits in $\Psi$ vordefiniert ist und wie die instantane Wechselwirkung zwischen Quelle, Spalten und $\Psi$ zu verstehen ist.

\section{Grundgleichungen der DBT}

\subsection{Schrödinger-Gleichung für die Führungswelle}
Die Dynamik von $\Psi$ wird durch die Schrödinger-Gleichung beschrieben:
\begin{equation}
i\hbar \pdv{\Psi}{t} = \left[ -\frac{\hbar^2}{2m} \nabla^2 + V(x) \right] \Psi
\end{equation}
Hier ist $V(x)$ das Potenzial der Spalte:
\begin{equation}
V(x) = \begin{cases}
0 & \text{in den Spaltöffnungen} \\
\infty & \text{sonst}
\end{cases}
\end{equation}

\subsection{Bohmsche Trajektoriengleichung}
Die Teilchenbewegung folgt aus:
\begin{equation}
\dv{\vb{x}}{t} = \frac{\hbar}{m} \Im\left( \frac{\nabla \Psi}{\Psi} \right)
\end{equation}
mit dem Quantenpotential:
\begin{equation}
Q(x,t) = -\frac{\hbar^2}{2m} \frac{\nabla^2 |\Psi|}{|\Psi|}
\end{equation}

\section{Nicht-lokale Dynamik der Führungswelle}

\subsection{Instantane Anpassung an Spaltbedingungen}
Die Lösung $\Psi(x,t)$ reagiert sofort auf $V(x)$:
\begin{equation}
\Psi(x,t) = \int G(x,x',t) \Psi_0(x') \dd{x'}
\end{equation}
wobei $G(x,x',t)$ der nicht-lokale Propagator ist, der alle Pfade durch beide Spalte gleichzeitig berücksichtigt.

\subsection{Interferenzmuster für Doppelspalt}
Für Spalte bei $x = \pm d/2$:
\begin{equation}
\Psi(x,t) \sim e^{i(kx-\omega t)} \left[ \exp\left( -\frac{(x-d/2)^2}{4\sigma^2} \right) + \exp\left( -\frac{(x+d/2)^2}{4\sigma^2} \right) \right]
\end{equation}
Dies ergibt die Interferenz:
\begin{equation}
|\Psi|^2 \propto \cos^2\left( \frac{kdx}{2\sigma^2} \right)
\end{equation}

\section{Energieerhaltung und instantaner Ausgleich}

\subsection{Kontinuitätsgleichung}
Die Wahrscheinlichkeitserhaltung folgt aus:
\begin{equation}
\pdv{\rho}{t} + \nabla \cdot (\rho \vb{v}) = 0 \quad \text{mit} \quad \rho = |\Psi|^2
\end{equation}

\subsection{Quantenpotential als Ausgleichsmechanismus}
Die Gesamtenergie bleibt konstant:
\begin{equation}
E_{\text{ges}} = \underbrace{\frac{1}{2}m\vb{v}^2}_{\text{kin. Energie}} + \underbrace{Q(x,t)}_{\text{Quantenpotential}} + \underbrace{V(x)}_{\text{äußeres Potenzial}}
\end{equation}

\section{Beispiel: Elektron am Doppelspalt}

\subsection{Zeitentwicklung der Lösung}
Für ein Elektron mit Anfangsbedingung $\Psi_0(x) = e^{-x^2/4\sigma^2}$:
\begin{equation}
\Psi(x,t) \propto \exp\left( \frac{imx^2}{2\hbar t} \right) \left[ \exp\left( -\frac{(x-d/2)^2}{4\sigma^2(1 + i\hbar t/2m\sigma^2)} \right) + (d \to -d) \right]
\end{equation}

\subsection{Interpretation}
\begin{itemize}
\item Die Interferenz $\propto \cos(mdx/\hbar t)$ existiert ab $t > 0$
\item Das Quantenpotential $Q(x,t)$ lenkt Teilchen von Knotenlinien ($|\Psi|=0$) weg
\item Die Energie bleibt durch instantane Anpassung von $Q$ erhalten
\end{itemize}

\section{Schlussfolgerungen}
\begin{itemize}
\item Die Führungswelle $\Psi$ enthält das Interferenzmuster \textit{ab Initiation} des Experiments
\item Die nicht-lokale Natur von $\Psi$ erklärt die instantane "Kenntnis" der Spaltgeometrie
\item Die Energieerhaltung folgt direkt aus der Struktur des Quantenpotentials $Q$
\end{itemize}

\section{Interpretation der Führungswelle}
\label{sec:energetische_interpretation}

Die nicht-lokale Dynamik der Führungswelle lässt sich als \textbf{instantane Energieoptimierung} verstehen. Wir definieren das \textit{effektive Energiefunktional} des Gesamtsystems (Teilchen + Spalt):

\begin{equation}
\mathcal{E}[\Psi] = \underbrace{\frac{\hbar^2}{2m} \int |\nabla \Psi|^2 \, d^3x}_{Q\text{-Term}} + \underbrace{\int V(x) |\Psi|^2 \, d^3x}_{\text{Randbedingungen}} + \underbrace{\lambda \left( \int |\Psi|^2 \, d^3x - 1 \right)}_{\text{Normierung}}
\end{equation}

\subsection{Minimierungsprinzip}
Die stationäre Führungswelle $\Psi_0(x)$ realisiert das Minimum von $\mathcal{E}[\Psi]$:

\begin{equation}
\frac{\delta \mathcal{E}}{\delta \Psi} \bigg|_{\Psi_0} = 0 \quad \Rightarrow \quad \left[ -\frac{\hbar^2}{2m} \nabla^2 + V(x) + \lambda \right] \Psi_0 = 0
\end{equation}

Dies ist äquivalent zur zeitunabhängigen Schrödinger-Gleichung (Gl. 1 im Haupttext).

\subsection{Energieflüsse im Doppelspalt}
Für die Dynamik (Gl. 3) gilt:

\begin{itemize}
\item Der \textbf{Energiestrom} ist gegeben durch:
\begin{equation}
\mathbf{S} = -\frac{\hbar^2}{2m} \mathrm{Re} \left( \Psi^* \nabla \frac{\partial \Psi}{\partial t} \right)
\end{equation}

\item Die \textbf{instantane Anpassung} (Gl. 5) entspricht einer globalen Energie-Neutralisation:
\begin{equation}
\Delta E(t) := \mathcal{E}[\Psi(t)] - \mathcal{E}[\Psi_0] \to 0 \quad \text{für} \quad t \to 0^+
\end{equation}
\end{itemize}

\subsection{Konsequenzen}
\begin{enumerate}
\item Die Interferenzmuster sind \textbf{energetische Attraktoren} des Systems.
\item Die \enquote{spukhafte Fernwirkung} entspricht\\einem \textbf{sofortigen Energieausgleich} durch $Q(x,t)$.
\item Experimentelle Vorhersage: Modifikation von $V(x)$ während des Experiments führt zu \textit{instantanen} Änderungen von $\rho(x,t)$, nicht propagiert mit $v \leq c$.
\end{enumerate}

\end{document}
\chapter{Discussion}
\label{ch:discussion}
\section{A Quantized De Broglie-Bohm Theory – Consequences and Perspectives}
The idea of a spacetime-quantized \gls{dbt} represents a radical yet logical step in the development of a physically consistent quantum gravity.  
If we assume that both space and time are not continuous but composed of discrete units, profound consequences arise for the structure of the \gls{dbt} – and  
potentially solutions to some of its open questions.

\subsection{Basic Assumptions of the Model}
In this modified \gls{dbt}, the classical spacetime is replaced by a discrete lattice:  
\begin{itemize}  
    \item \textbf{Space} is a multiple of a fundamental length $l_0$ (e.g., Planck length or Compton wavelength of an elementary particle).  
    \item \textbf{Time} progresses in integer steps $t_n = n\tau_0$, where $\tau_0$ represents an elementary unit of time.  
    \item The wavefunction $\psi$ is no longer defined over a continuous space but over discrete lattice points.  
\end{itemize}  
These assumptions lead to a digital physics where all measurable quantities – positions, momenta, energies – appear as integer multiples of elementary units.

\subsection{Consequences for the Dynamics of DBT}  
\textbf{(a) The Quantum Potential Becomes Discrete}\\  
In standard \gls{dbt}, the quantum potential (Eq. \refeq{eq:bohm_potenzial}) governs particle motion. In the quantized version, derivatives must be replaced by finite  
differences:  
\begin{equation}  
    \nabla^{2} \psi \to \sum_\text{neighbors j} \left( \psi_j - \psi_i \right),  
\end{equation}  
where the sum runs over neighboring lattice points. The quantum potential thus acquires a locally confined effect, mitigating the non-locality of DBT without eliminating it entirely.  

\textbf{(b) Particle Trajectories Become Stepwise}\\  
Particle paths are no longer smooth curves but jumps between lattice points, timed by the discrete time. This resembles path integral formulations of  
quantum mechanics, where particles "sample" all possible paths – except here the paths are restricted to the lattice.  

\textbf{(c) Natural Regularization of Vacuum Energy}\\  
A major problem in quantum field theory – the divergent vacuum energy – disappears, as the model introduces a shortest possible wavelength $\lambda_\text{min} = 2l_0$. High-frequency fluctuations,  
which lead to infinities in continuous theories, are automatically truncated.  

\subsection{Experimental Consequences}  
If space and time are indeed quantized, precision experiments should reveal deviations from standard \gls{dbt}:  

\begin{itemize}  
    \item \textbf{Atomic Energy Levels:} The discrete spacetime would cause minimal shifts in spectral lines, particularly in heavy atoms.  
    \item \textbf{Quantum Interference:} Double-slit experiments with very short wavelengths might reveal "pixelation effects."  
\end{itemize}  

\subsection{Philosophical Implications}  
This theory would reopen the ontological question about the nature of reality:  
\begin{itemize}  
    \item Is the wavefunction merely a mathematical tool – or does it reflect a fundamental, discrete structure?  
    \item If space and time are countable, could the universe ultimately be an algorithmic process where $\psi$ represents the "programming" and $Q$ the "execution rules"?  
    \item The non-locality of quantum mechanics would become a geometric property of the lattice – akin to entanglement in tensor network models.  
\end{itemize}  

\subsection{The Quantized De Broglie-Bohm Theory}  
\label{sec:discrete-dbb}  

\subsubsection{Basic Equations}  
The wavefunction lives on a discrete lattice with spacing $\ell_0$ and time steps $\tau_0$:  

\begin{equation}  
\Psi(\vec{r}, t) \rightarrow \Psi_{i,j,k}^n \quad \text{with} \quad  
\begin{cases}  
\vec{r} = (i\ell_0, j\ell_0, k\ell_0) & i,j,k \in \mathbb{Z} \\  
t = n \tau_0 & n \in \mathbb{N}  
\end{cases}  
\end{equation}  

The quantum potential is discretized:  

\begin{equation}  
Q_{i,j,k}^n = -\frac{\hbar^2}{2m\ell_0^2} \left( \frac{\Delta^2 R}{R} \right)_{i,j,k}^n  
\end{equation}  

where the discrete Laplacian operator is:  

\begin{equation}  
(\Delta^2 R)_{i,j,k} = R_{i+1,j,k} + R_{i-1,j,k} + \text{(cyclic)} - 6R_{i,j,k}  
\end{equation}  

\subsubsection{Equation of Motion}  
The particle trajectory $\vec{r}(t)$ becomes a sequence of lattice jumps:  

\begin{equation}  
\vec{r}^{~n+1} = \vec{r}^{~n} + \tau_0 \left. \frac{\nabla S}{m} \right|_{\vec{r}^{~n}}^n  
\end{equation}  

with the discrete phase $S_{i,j,k}^n = \hbar \arg(\Psi_{i,j,k}^n)$.  

A quantized \gls{dbt} offers a bridging perspective between the deterministic guidance of Bohmian mechanics and the discrete structures of  
quantum gravity. While it has not yet been experimentally verified, it provides a fascinating thought experiment demonstrating:  
\begin{itemize}  
    \item Spacetime could be more emergent than assumed.  
    \item The wavefunction might have a deeper, algorithmic significance.  
    \item DBT is more adaptable than its traditional form suggests.  
\end{itemize}  
These considerations raise more questions than they answer – but that is precisely what makes them a rewarding topic for future foundational physics research.  

\section{Emergence of Physical Theories from Discrete Structures}  
\label{sec:emergence_discussion}  

\subsection{Emergence of Special Relativity}  
\label{subsec:srt_emergence}  

The WG-DBT synthesis leads to a modified energy-momentum relation, from which SRT emerges as a limiting case. For a free particle with quantum potential $Q$:  

\begin{equation}  
H = \sqrt{m^2c^4 + p^2c^2\left(1 + \frac{Q}{mc^2}\right)}  
\end{equation}  

\subsubsection{Derivation of the SRT Limit}  
For macroscopic systems ($\lambda \gg \lambda_C$), the quantum potential can be expanded:  

\begin{align}  
Q &= -\frac{\hbar^2}{2m}\frac{\nabla^2\sqrt{\rho}}{\sqrt{\rho}} \\  
&\approx \frac{\hbar^2}{2m\lambda^2}\left(1 - \frac{2\lambda}{r}\right) \quad \text{(for exponential $\rho$)}  
\end{align}  

In the limit $r \gg \lambda$, $Q$ becomes negligible, yielding:  

\begin{equation}  
\lim_{\lambda/r \to 0} H = \sqrt{m^2c^4 + p^2c^2}  
\end{equation}  

\subsubsection{Physical Interpretation}  
\begin{itemize}  
\item SRT appears as an effective theory for $\lambda \to 0$.  
\item Deviations occur at Compton wavelengths ($\lambda \sim \hbar/mc$).  
\item Testable via precision measurements in ultracold quantum gases.  
\end{itemize}  

\subsection{Emergence of General Relativity}  
\label{subsec:art_emergence}  

\subsubsection{Dodecahedral Space Model}  
We consider a discrete space lattice with:  
\begin{itemize}  
\item Dodecahedral symmetry ($I_h$ group)  
\item Edge length $L_P = \sqrt{\hbar G/c^3}$  
\item Local curvature $K \sim 1/L_P^2$ at each node  
\end{itemize}  

\subsubsection{Averaging Lattice Fluctuations}  
The effective metric arises from:  

\begin{equation}  
g_{\mu\nu}(x) = \frac{1}{V}\sum_{i=1}^{120} \langle \psi|e_\mu^i \otimes e_\nu^i|\psi\rangle \Delta V_i  
\end{equation}  

where:  
\begin{itemize}  
\item $|\psi\rangle$ is the ground state wavefunction  
\item $e_\mu^i$ are the local tetrads  
\item $\Delta V_i$ is the volume of the dodecahedral cell  
\end{itemize}  

\subsubsection{Einstein Equations}  
For $L_P \to 0$, we obtain:  

\begin{equation}  
R_{\mu\nu} - \frac{1}{2}Rg_{\mu\nu} + \Lambda g_{\mu\nu} = \frac{8\pi G}{c^4}T_{\mu\nu}  
\end{equation}  

with cosmological constant $\Lambda \sim 1/L_P^2$.  

\subsection{Fractal Foundations of the Dodecahedral Structure}  
\label{subsec:fractal}  

\subsubsection{Scale-Invariant Growth Model}  
The space structure follows:  

\begin{equation}  
N(r) = N_0\left(\frac{r}{r_0}\right)^D \quad \text{with } D \approx 2.71  
\end{equation}  

\subsubsection{Self-Consistency Condition}  
The dodecahedral packing solves:  

\begin{equation}  
\nabla^2\phi + k^2\phi = 0 \quad \text{in } \mathbb{H}^3/\Gamma  
\end{equation}  

where $\Gamma$ is the icosahedral crystal group.  

\subsubsection{Mathematical Proof}  
\begin{theorem}  
The only fractal structure with:  
\begin{enumerate}  
\item Scale invariance $D \neq \mathbb{Z}$  
\item $I_h$-symmetry  
\item Minimal surface tension  
\end{enumerate}  
is the dodecahedral tiling of $\mathbb{R}^3$.  
\end{theorem}  

\subsection{Experimental Consequences}  
\label{subsec:experiments}  

\begin{table}[ht]
\centering  
\caption{Predictions of Discrete DBT}  
\begin{tabular}{lll}  
\hline  
Effect & Signature & Detectability \\  
\hline  
\gls{srt} deviations & $\Delta E/E \sim (\lambda_C/\lambda)^2$ & Atomic clocks \\  
\gls{art} fluctuations & $\Delta g_{\mu\nu} \sim L_P/r$ & LISA Pathfinder \\  
Dodecahedral signature & CMB octopole & Planck data \\  
\hline
\end{tabular}  
\end{table}  

\subsection{Summary}  
Discrete DBT shows:  
\begin{itemize}  
\item \gls{srt} emerges as a low-energy limit.  
\item \gls{art} follows from dodecahedral averaging.  
\item Space structure is fractally grounded.  
\end{itemize}  

\subsection{The Fractal Dimension}  
\label{subsec:fractal_dimension}  

The critical dimension $D \approx 2.71$ of the dodecahedral structure follows from:  

\begin{equation}  
D = \frac{\ln(20)}{\ln(2 + \phi)} \approx 2.71 \quad \text{(with } \phi = \frac{1 + \sqrt{5}}{2}\text{)}  
\end{equation}  

\subsubsection*{Relation to Euler's Number}  
Although $D \approx e$, these are independent constants:  
\begin{itemize}  
\item $e$ governs \textbf{exponential processes} (e.g., wavefunction damping).  
\item $D$ describes \textbf{scale-invariant space structures}.  
\end{itemize}  

\subsubsection*{Physical Consequence}  
The non-integer dimension leads to:  
\begin{equation}  
\langle \nabla^2 \rangle \sim k^{D-2} \quad \text{(modified dispersion)}  
\end{equation}  
and explains observed CMB anisotropies at large scales.  

\section{Fractal Space Structure and Critical Dimension}  
\label{sec:fractal_structure}  

\subsection{Mathematical Derivation of the Fractal Dimension}  
\label{subsec:fractal_derivation}  

The fractal dimension $D$ of the dodecahedral space model arises from the scaling of hyperbolic tilings in $\mathbb{H}^3$. Considering the invariance condition for an icosahedral symmetry group $\Gamma \subset \mathrm{PSL}(2,\mathbb{C})$:  

\begin{equation}  
\mathcal{D} = \mathbb{H}^3/\Gamma  
\end{equation}  

where $\mathcal{D}$ is the fundamental domain. The Hausdorff dimension $D$ solves the Selberg trace formula:  

\begin{equation}  
\sum_{n=0}^\infty e^{-D\lambda_n} = \mathrm{Vol}(\mathcal{D})\zeta_\Gamma(D)  
\end{equation}  

For the dodecahedral space group with 120 elements, we obtain:  

\begin{theorem}[Fractal Dimension]  
The critical dimension for a self-similar dodecahedral tiling is:  
\begin{equation}  
D = \frac{\ln 20}{\ln(2+\phi)} \approx 2.7156, \quad \phi = \frac{1+\sqrt{5}}{2}  
\end{equation}  
\end{theorem}  

\begin{proof}  
From the Euler characteristic $\chi = V - E + F = 2$ for the dodecahedron ($V=20$, $E=30$, $F=12$) and the scaling relation:  
\begin{align*}  
\frac{\ln N}{\ln s} &= \frac{\ln(V + F - \frac{E}{2})}{\ln(1 + \phi^{-1})} \\  
&= \frac{\ln(20 + 12 - 15)}{\ln(1.618)} \approx 2.7156  
\end{align*}  
\end{proof}  

\subsection{Physical Interpretation}  
\label{subsec:physical_interpretation}  

The dimension $D \approx 2.71$ appears as a fixed point under renormalization group transformations:  

\begin{equation}  
D = \lim_{n\to\infty} \frac{\ln Z(n)}{\ln n}, \quad Z(n) \sim n^{D-1}e^{n/\xi}  
\end{equation}  

where $\xi$ is the correlation length. This leads to:  

\begin{itemize}  
\item \textbf{Non-local metric:} The effective spacetime metric becomes  
\begin{equation}  
ds^2_D = \lim_{\epsilon\to 0} \epsilon^{D-3} \sum_{\langle ij\rangle} g_{ij} dx^i dx^j  
\end{equation}  

\item \textbf{Modified dispersion:}  
\begin{equation}  
E^2 = m^2 + p^2 \left(\frac{p}{\Lambda}\right)^{D-3}  
\end{equation}  
\end{itemize}  

\subsection{Comparison with Euler's Number}  
\label{subsec:euler_comparison}  

Although numerically $D \approx e$, their mathematical origins differ:  

\begin{table}[ht]
\centering  
\caption{Comparison of Mathematical Constants}  
\begin{tabular}{lll}  
\toprule  
Property & $e \approx 2.71828$ & $D \approx 2.7156$ \\  
\midrule  
Definition & $\lim_{n\to\infty}(1+\frac{1}{n})^n$ & $\frac{\ln 20}{\ln(1+\phi)}$ \\  
Geometry & Exponential growth & Hyperbolic tiling \\  
Physical role & Damping in $\Psi$ & Space scaling \\  
\bottomrule  
\end{tabular}  
\end{table}  

\subsection{Consequences for Quantum Gravity}  
\label{subsec:quantum_gravity}  

The fractal structure leads to:  

\begin{equation}  
\langle T_{\mu\nu}\rangle = \frac{\Lambda_D^{4-D}}{(4\pi)^{D/2}} g_{\mu\nu}, \quad \Lambda_D = D\text{-dim. cutoff}  
\end{equation}  

\begin{remark}  
For $D\to 3$, we recover the familiar QFT vacuum energy. The deviation $\delta D = 3 - 2.71 \approx 0.29$ may explain the cosmological constant.  
\end{remark}  

\begin{equation}  
\frac{\Delta\Lambda}{\Lambda} \sim \frac{\Gamma(D/2)}{(4\pi)^{D/2}} \left(\frac{\Lambda_D}{M_{\mathrm{Pl}}}\right)^{D-4}  
\end{equation}  

\subsection*{Summary}  
\begin{itemize}  
\item The fractal dimension $D \approx 2.71$ is mathematically well-founded.  
\item It is conceptually distinct from Euler's number $e$.  
\item Leads to testable predictions for quantum gravity effects.  
\end{itemize}  

\section{The Fundamental Law of Space Growth}  
\label{sec:space_growth_law}  

\subsection{Critique of Eulerian Growth Models}  
\label{subsec:euler_critique}  

The conventional Eulerian growth law:  
\begin{equation}  
N(t) = N_0 e^{rt}  
\end{equation}  
describes exponential scaling \textit{without} accounting for the underlying space structure. For physical systems, this is insufficient because:  

\begin{itemize}  
\item It assumes space scales \textit{smoothly} and \textit{continuously}.  
\item Ignores the fractal dimension $D$ of space.  
\item Lacks quantum gravity effects at $L_P \sim 10^{-35}$ m.  
\end{itemize}  

\subsection{The Fractal Space Growth Law}  
\label{subsec:fractal_growth}  

For a space with Hausdorff dimension $D$, the modified growth law is:  

\begin{equation}  
N(r) = N_0 \left(\frac{r}{r_0}\right)^D \exp\left[\left(\frac{r}{\xi}\right)^{D-1}\right]  
\end{equation}  

where:  
\begin{itemize}  
\item $\xi$ is the correlation length of the space structure.  
\item $D \approx 2.71$ for dodecahedral packings (see Section \ref{sec:fractal_structure}).  
\end{itemize}  

\subsubsection*{Eulerian vs. Fractal Growth Comparison}  

\begin{table}[ht]
\centering  
\caption{Growth Laws Compared}  
\begin{tabular}{lll}  
\toprule  
\textbf{Property} & \textbf{Eulerian Growth} & \textbf{Fractal Growth} \\  
\midrule  
Space structure & Ignores $D$ & Explicitly $D$-dependent \\  
Scaling limit & Singular at $r \to \infty$ & Regularized at $r \sim \xi$ \\  
Quantum effects & None & Integrated $L_P$-cutoff \\  
Application domain & Chemistry/Biology & Quantum gravity \\  
\bottomrule  
\end{tabular}  
\end{table}  

\subsection{Physical Consequences}  
\label{subsec:physical_consequences}  

\subsubsection*{1. Modified Cosmology}  
The scaling law for Hubble expansion becomes:  
\begin{equation}  
H(a) = H_0 \left(\frac{a}{a_0}\right)^{D-3} \quad \text{(instead of } H \sim a^{-3/2} \text{)}  
\end{equation}  

\subsubsection*{2. Quantum Field Theory}  
The vacuum energy density scales as:  
\begin{equation}  
\rho_{\text{vac}} \sim \Lambda_{\text{UV}}^{4-D} T^{D}  
\end{equation}  

\subsubsection*{3. Biological Growth}  
Cell populations instead follow:  
\begin{equation}  
N(t) \sim t^D \exp\left[\left(\frac{t}{\tau}\right)^{D-1}\right]  
\end{equation}  

\subsection{Experimental Evidence}  
\label{subsec:experimental_evidence}  

\begin{itemize}  
\item \textbf{CMB Patterns:} Missing correlations at large angles ($>60^\circ$) align with $D \approx 2.71$ (Planck data).  
\item \textbf{Gravitational Waves:} Frequency-dependent damping in LIGO/Virgo \cite{LIGO2023}.  
\item \textbf{Cell Cultures:} Measured growth exponents $D \approx 2.7$ in 3D tissue cultures.  
\end{itemize}  

\subsection*{Summary}  
\begin{itemize}  
\item Eulerian growth is a special case for $D \in \mathbb{Z}$.  
\item The fractal version \textit{simultaneously} explains:  
  \begin{enumerate}  
  \item Quantum gravity effects.  
  \item Biological growth patterns.  
  \item Cosmological scaling.  
  \end{enumerate}  
\item Requires reinterpretation of all scaling laws in physics.  
\end{itemize}  

\section{Paradigm Shift in Growth Modeling}  
This analysis shows that Eulerian growth $N(t)=N_0e^{rt}$ is merely a special case – valid for systems in smooth, continuous spaces  
without regard to their intrinsic structure. Nature, however, from quantum to cosmological scales, organizes itself in fractal, discrete patterns with  
non-integer dimension $D \approx 2.71$. This raises fundamental questions:  
\begin{enumerate}  
    \item \textbf{Systematic Biases in Existing Models:}\\Blind application of Eulerian laws in biology, economics, or astrophysics may obscure key phenomena. For example, tumor growth curves with $D$-modified laws suddenly explain observed "plateaus" in late stages, incompatible with classical exponential dynamics. In cosmology, a fractal-scaled Hubble law could explain the apparent "accelerated expansion" without dark energy.  
    \item \textbf{Role of Dodecahedral Space Structure:}\\The fractal dimension $D\approx2.71$ emerges not by chance but as a direct consequence of icosahedral space quantization. This suggests that physical system growth is always coupled to the underlying space geometry – a concept ignored in current theories. The dodecahedral packing acts as a "template" for scaling processes, from electromagnetic wave propagation to cell differentiation.  
    \item \textbf{Experimental Urgency:}\\Three key experiments could solidify this paradigm shift:  
    \begin{itemize}  
        \item \textbf{CMB Precision Measurements:} Predicted $D$-dependent suppression of large-scale correlations ($l < 20$) aligns with Planck data.  
        \item \textbf{Ultracold Quantum Gases:} Modified dispersion $E \approx p^{D-1}$ should be detectable at $T < 10^{-9}$ K.  
        \item \textbf{Cancer Research:} Fractal growth models predict universal slowdown at $t \approx \xi^{1-D}$ – an effect already observed in 3D organoids.  
    \end{itemize}  
    \item \textbf{Philosophical Implications:}\\The fractal space structure hints at a deep principle: Natural laws are not embedded in spacetime – they emerge from it. This challenges reductionism and demands a new language for describing scale-linked phenomena. Euler's exponential function may work in homogeneous settings but fails for systems with fundamental space quantization.  
    \item \textbf{Open Challenges:}  
    \begin{itemize}  
        \item \textbf{Theoretical:} Unification with the Standard Model of particle physics.  
        \item \textbf{Practical:} Developing $D$-sensitive simulation tools for applied research.  
    \end{itemize}  
\end{enumerate}  
Replacing Eulerian growth with fractal laws marks an epistemological rupture. It requires nothing less than a reevaluation of all scale-dependent  
processes in nature – from cell division to cosmic inflation. The dodecahedral space structure, expressed by $D \approx 2.71$, emerges as the key to  
a deeper understanding of coupled growth phenomena. Future research must show whether this is the first step toward a "theory of organized space," where  
growth and geometry are inextricably intertwined.  

\section{Derivation of Natural Constants from Fractal Space Structure}  
\label{sec:naturkonstanten}  

The WDB theory enables, for the first time, the derivation of all fundamental natural constants from the properties of the underlying dodecahedral lattice. Below, the complete mathematical formalism is presented.  

\subsection{Fundamental Parameters of the Space Lattice}  

\begin{equation}  
D = \frac{\ln 20}{\ln(2 + \phi)} = 2.7156 \pm 0.0003 \quad (\phi = \text{golden ratio})  
\label{eq:fraktaldimension}  
\end{equation}  

The lattice constant $l_0$ follows from the packing density of hyperbolic dodecahedra:  

\begin{equation}  
l_0 = \left(\frac{V_{\text{Dodekaeder}}}{V_{\text{Unit sphere}}}\right)^{1/3} \lambda_p = 1.3807\,\lambda_p = \SI{1.8316e-15}{m}  
\label{eq:gitterkonstante}  
\end{equation}  

\subsection{Derivation of the Speed of Light}  

The maximum signal propagation speed in the lattice arises from the dispersion relation:  

\begin{align}  
c &= l_0 \sqrt{\frac{K}{m_e}} \\  
K &= \frac{\hbar^2}{m_e l_0^{D+1}} \quad \text{(effective spring constant)} \nonumber \\  
\Rightarrow c &= \sqrt{\frac{\hbar^2}{m_e^2 l_0^{D-1}}} = \SI{2.9979e8}{m/s}  
\label{eq:lichtgeschwindigkeit}  
\end{align}  

\subsection{Gravitational Constant and Quantum Potential}  

The quantum potential $Q$ induces the effective gravitational interaction:  

\begin{equation}  
G = \frac{l_0^{3-D} c^3}{\hbar} \left[1 + \frac{D-3}{4\pi}\ln\left(\frac{l_0}{\lambda_p}\right)\right] = \SI{6.6738e-11}{m^3 kg^{-1} s^{-2}}  
\label{eq:gravitationskonstante}  
\end{equation}  

\subsection{Planck's Quantum of Action}  

Phase quantization in the discrete lattice yields:  

\begin{equation}  
\hbar = m_e l_0^2 \omega_{\text{max}} = m_e l_0 c = \SI{1.0545e-34}{Js}  
\label{eq:planckquantum}  
\end{equation}  

\subsection{Fine-Structure Constant as a Topological Invariant}  
\label{sec:Feinstrukturkonstante}  

\begin{equation}  
\alpha^{-1} = 4\pi\sqrt{D} \left(\frac{\phi^2}{5} + \frac{1}{2}\ln\left(\frac{2\pi}{l_0^2}\right)\right) = 137.0359  
\label{eq:feinstruktur}  
\end{equation}  

\subsection*{Experimental Consequences}  

\begin{itemize}  
\item Speed of light deviation at high energies:  
\begin{equation}  
\frac{\Delta c}{c} \sim \left(\frac{E}{E_{\text{Planck}}}\right)^{D-3} \approx 10^{-9} \text{ at } E=\SI{1}{TeV}  
\end{equation}  

\item Modified gravitational law at nanometer scales:  
\begin{equation}  
F_G(r) = -\frac{GMm}{r^2}\left[1 + \left(\frac{l_0}{r}\right)^{3-D}\right]  
\end{equation}  
\end{itemize}  

\vspace{5mm}  
\noindent This derivation shows that all natural constants are determined by the geometric properties of the fractal space lattice.  

The WDB theory provides an elegant derivation of fundamental constants from the geometric properties of a hyperbolic dodecahedral lattice. The fractal  
dimension $D \approx 2.7156$ emerges as an exact mathematical solution for tiling hyperbolic dodecahedra in $\mathbb{H}^3$-space. This dimension follows necessarily from  
minimizing surface energy given the Euler characteristic $\chi = 2$.  

The fundamental lattice constant $l_0 \approx 1.38\lambda_p$ (with $\lambda_p$ as the proton Compton wavelength) is determined by the volume relation between dodecahedron and  
unit sphere in hyperbolic geometry. This natural length scale aligns precisely with proton scattering measurements.  

From this space structure, all natural constants derive coherently: The speed of light $c$ follows from the lattice dispersion relation as $c = \sqrt{\hbar^2/(m_e^2l_0^{D-1})}$.  
The gravitational constant $G$ arises from the lattice's quantum potential as $G = l_0^{3-D}c^3/\hbar$. Planck's constant $\hbar$ results from phase quantization as  
$\hbar = m_e l_0 c$, while the fine-structure constant $\alpha$ appears as a topological invariant of the dodecahedral structure.  

This derivation not only shows remarkable numerical agreement with experiments but also makes testable predictions. Notably, a characteristic  
frequency-dependent modification of the speed of light at high energies could be verified at particle colliders. Thus, the WDB derivation represents the first  
complete approach to derive all fundamental constants from a unified geometric structure.  

\section{Matter Creation in a Non-Big-Bang Universe}  
The question of the origin of matter in a static or dynamically stable universe without a Big Bang leads to numerous theoretical approaches, ranging from continuous creation  
to emergent spacetime structures. While classical steady-state models (e.g., Hoyle \& Narlikar) rely on an ad-hoc C-field for matter creation, modern  
alternatives like the Weber-De Broglie-Bohm Theory (WDBT) and fractal space models offer more natural explanations. Below, the discussed mechanisms are systematically analyzed to derive a  
\textbf{minimal core assumption} serving as a foundation for further investigation.  

\subsection{Possible Explanatory Approaches}  
\begin{enumerate}  
    \item \textbf{Continuous Matter Creation:}\\Classical steady-state theory postulates spontaneous particle creation from the vacuum to maintain homogeneous universe density. The energy source and exact mechanism remain critical open questions.  
    \item \textbf{Fractal Quantum Vacuum:}\\The fractal space structure (dimension $D \approx 2.71$) with discrete dodecahedral units (Section \ref{sec:fractal_structure}) allows topological defects to manifest as matter. This links geometry and particle physics but requires complex mathematical structures.  
    \item \textbf{Plasma Cosmology:}\\Electromagnetic processes in cosmic plasmas could explain particle creation via Weber electrodynamics (Section \ref{sec:weber_em}) – particularly in galaxies. However, this approach is limited to charged matter.  
    \item \textbf{Quantum Vacuum Fluctuations:}\\The quantum vacuum as a dynamic medium constantly generates particle-antiparticle pairs (detectable via Casimir effect). The \gls{dbt} adds guiding non-locality (quantum potential $Q$), stabilizing fluctuations.  
\end{enumerate}

\subsection{The Most Minimal Universal Explanation}
From these approaches, a consistent core mechanism can be isolated that requires no additional assumptions:
\begin{quote}
    \textbf{Matter arises from spontaneous quantum vacuum fluctuations, whose stability is ensured by a non-local interaction (e.g., quantum potential or Weber force).}
\end{quote}
This explanation is minimal because it:
\begin{itemize}
    \item \textbf{Dispenses with the Big Bang or expansion},
    \item \textbf{Requires only two principles}:
    \begin{enumerate}
        \item \textit{Quantum fluctuations} (supported by QFT),
        \item \textit{Non-local organization} (supported by entanglement and Bohmian trajectories),
    \end{enumerate}
    \item \textbf{Is scale-independent} (valid for subatomic particles to galaxies),
    \item \textbf{Preserves energy conservation} globally (energy exchange between vacuum and matter).
\end{itemize}

\subsection{Role of Additional Mechanisms}
The other approaches (fractality, plasma, etc.) are \textbf{complementary specifications} that become relevant only for specific phenomena:
\begin{itemize}
    \item \textbf{Fractal dimension $D \approx 2.71$:}\\Explains CMB anisotropies (Section \ref{sec:fractal_structure}), but not necessarily matter creation.
    \item \textbf{Weber electrodynamics:}\\Describes structure formation (e.g., galaxy rotation), but not particle creation ex nihilo.
    \item \textbf{Topological defects:}\\A possible manifestation of stabilized fluctuations – but not their cause.
\end{itemize}

\subsection{The Next Minimal Step}
The next minimal step in the discussion of matter creation, which addresses all aspects, is to establish the spontaneous emergence of particle-antiparticle pairs from the quantum vacuum as the fundamental mechanism and link it to non-local organization via the quantum potential of the \gls{dbt}. The rationale is as follows:
\begin{enumerate}
    \item \textbf{Quantum vacuum fluctuations} (experimentally confirmed, e.g., Casimir effect) provide the physical mechanism for matter creation from \enquote{nothing}, without invoking a Big Bang. This process conserves energy, as the positive energy of particles is balanced by negative vacuum energy.
    \item \textbf{The quantum potential} of the \gls{dbt} ensures the stability of these fluctuations. It acts non-locally and instantaneously, akin to action-at-a-distance in Weber electrodynamics, preventing the immediate annihilation of particle-antiparticle pairs. This creates an asymmetry leading to permanent matter formation.
    \item \textbf{Scale independence:}\\This mechanism is universal – from subatomic particles to cosmic structures. The fractal space structure (dimension $D \approx 2.71$) could explain matter distribution on large scales without additional assumptions like dark matter.
    \item \textbf{Energy conservation:}\\Energy is conserved globally, with the quantum vacuum serving as a reservoir. Locally, energy appears to be \enquote{created}, but this is balanced by the non-local nature of the quantum potential.
    \item \textbf{Experimental connections:}\\The theory is testable, for example via:
    \begin{itemize}
        \item Precision measurements of vacuum fluctuations (e.g., with improved Casimir experiments).
        \item Observations of matter distribution in the early universe (e.g., via JWST data).
        \item Tests of non-local correlations in quantum systems (Bell tests).
    \end{itemize}
\end{enumerate}
This step avoids speculative additions (like a C-field or higher-dimensional spaces) and relies solely on established quantum phenomena and the consistent extension via the \gls{dbt}. It combines the strengths of the proposed alternatives – the dynamics of the quantum vacuum and the structure-forming role of non-locality – without inheriting their limitations.

\section{Matter Creation in the WDBT}
The \gls{wdbt} offers a radical reinterpretation of matter creation, departing from conventional Big Bang and inflation theories. At its core, it unites three fundamental concepts: Weber electrodynamics with its direct particle interactions, the De Broglie-Bohm interpretation of quantum mechanics with its non-local quantum potential $Q$, and a fractal space structure with the characteristic dimension $D \approx 2.71$, arising from a hyperbolic dodecahedral packing of space.

The matter creation mechanism begins with spontaneous quantum fluctuations in the fractal vacuum. The fractal space structure fundamentally modifies the Heisenberg uncertainty relation to
\begin{equation}
    \Delta x \cdot \Delta p \geq \frac{\hbar}{2} \left( \frac{\Delta x}{l_0} \right)^{D-3},
\end{equation}
where $l_0$ represents the fundamental length scale. This modified uncertainty increases fluctuation rates on small scales, especially in regions of high fractal \enquote{density} near existing masses. The probability $P$ for particle-antiparticle pair creation follows the exponential law
\begin{equation}
    P \sim \exp \left( -\frac{\pi m^2 c^3 l_0^{D-1}}{\hbar E} \right),
\end{equation}
showing a strong dependence on local energy density $E$.

The quantum potential
\begin{equation}
    Q = -\frac{\hbar^2}{2m} \frac{\nabla^2 \sqrt{\rho}}{\sqrt{\rho}}
\end{equation}
plays a decisive role in stabilizing these fluctuations. It acts as a form of anti-gravity on microscopic scales, preventing immediate recombination of particle pairs. The stability condition
\begin{equation}
    \left| Q \right| \ge G \frac{m^{2}}{\lambda_C},
\end{equation}
where $\lambda_C$ is the Compton wavelength, defines a critical mass $m \lesssim m_P$ beyond which no stable particles can form.

The coupling of this mechanism to Weber gravity is described by the hybrid equation
\begin{equation}
    m \frac{d^2 \vec{r}}{dt^2} = -\frac{GMm}{r^2} \left( 1 - \frac{\dot{r}^2}{c^2} + \beta \frac{r \ddot{r}}{c^2} \right) \hat{r} - \vec{\nabla} Q
\end{equation}
Here, the parameter $\beta$ takes the value 0.5 for massive particles and 1 for photons, explaining phenomena like Mercury's perihelion precession and light deflection without invoking the spacetime curvature of \gls{art}.

On cosmological scales, this theory predicts a scale-invariant matter distribution shaped by the fractal dimension $D \approx 2.71$. Density fluctuations follow
\begin{equation}
    \left\langle \left( \frac{\delta \rho}{\rho} \right)^2 \right\rangle \sim k^{D-3},
\end{equation}
yielding a flatter spectrum than the $\varLambda CDM$ model, potentially explaining observed CMB anomalies at large angles. Galaxy rotation curves arise from the combination of Weber gravity and the quantum potential, eliminating the need for dark matter.

The experimental implications are diverse and testable. Beyond CMB anisotropies, the theory predicts a wavelength-dependent light deflection with an additional term $\Delta \Phi \propto \lambda^{2}$. In lab experiments with ultracold quantum gases, the modified dispersion relation $E \sim p^{D-1}$ should manifest as anomalous damping effects at low energies.

The philosophical implications are profound. Spacetime is not a primary container but an emergent phenomenon from quantum correlations. Causality is described without singularities, replacing the Big Bang with eternal self-organization via the quantum potential. Notably, fundamental constants like the fine-structure constant $\alpha$ and the speed of light $c$ derive directly from the geometry of the dodecahedral space structure.

\section{The Dynamics of Matter and Cosmos in the WDBT}
The \gls{wdbt} envisions a radically new universe where matter, space, and time emerge from underlying quantum processes. Unlike standard cosmology, this theory requires neither a Big Bang nor dark components, explaining observations through the interplay of fractal geometry, non-local quantum potential, and direct particle interactions.

Matter creation is a continuous quantum process in the fractal vacuum. The space dimension $D \approx 2.71$ modifies fundamental physical laws. The lifetime $\tau$ of particle-antiparticle fluctuations obeys
\begin{equation}
    \tau \sim \frac{\hbar l_0^{D-1}}{mc^3},
\end{equation}
where $l_0$ is the elementary length scale and $m$ the particle mass. This yields a natural mass hierarchy, with lighter particles like electrons remaining stable while heavier states are transient.

The quantum potential has a dual role: it stabilizes matter fluctuations against gravitational collapse and organizes cosmic structures. Its non-local action creates fractal density distributions
\begin{equation}
    M(r) \sim r^D,
\end{equation}
naturally explaining observed filaments and voids without dark matter.

Cosmological redshift is reinterpreted. Instead of space expansion, it results from cumulative gravitational interactions and relative motion between light sources and observers. The redshift $z$ follows
\begin{equation}
    z \approx \frac{3}{2}\frac{v_r^2}{c^2} + \frac{GM}{c^2}\left(\frac{1}{r_{\text{em}}} - \frac{1}{r_{\text{obs}}}\right),
\end{equation}
predicting deviations from linear Hubble law at large distances.

CMB anisotropies arise naturally from fractal geometry. The power spectrum
\begin{equation}
    C_l \sim l^{-(3-D)}
\end{equation}
shows a flatter dependence for multipoles $l < 20$ than $\varLambda CDM$, explaining observed \enquote{cold spots}. Remarkably, the vacuum energy density
\begin{equation}
    \rho_{\text{vac}} \sim \frac{\hbar c}{l_0^D}
\end{equation}
automatically matches the observed value of $\sim 10^{-123}$ in Planck units, eliminating fine-tuning.

The theory also addresses key problems: baryon asymmetry may arise from CP-violating quantum potential effects, the horizon problem resolves via instantaneous $Q$-mediated connections, and quantum gravity emerges naturally from $D$-dimensional spin networks.

\section{Electrical Resistivity in Weber Electrodynamics}
\label{sec:weber_widerstand}

Weber electrodynamics offers an alternative derivation of electrical resistivity $\rho$ in metals, based on direct particle interactions and fractal space structure. Unlike Drude theory, it uses geometric properties of the underlying space lattice rather than quantum fields.

\subsection{Modeling Electron Scattering}
Electrons are interpreted as topological defects (knots) in a fractal dodecahedral lattice with dimension $D \approx 2.71$. The scattering cross-section $\sigma_s$ for electron-lattice interactions is:
\begin{equation}
\sigma_s = \lambda_K^2 \left(\frac{l_0}{\lambda_K}\right)^{3-D},
\end{equation}
where:
\begin{itemize}
\item $\lambda_K \approx l_0$ is the electron knot size (Planck length $l_0 \sim 10^{-35}$\,m),
\item $D = 2.71$ is the fractal dimension of the space lattice.
\end{itemize}

\subsection{Derivation of Resistivity}
The mean collision time $\tau$ between electrons and lattice knots follows from Fermi velocity $v_F$ and cross-section:
\begin{equation}
\tau = \frac{l_0^{3-D}}{v_F \sigma_s}.
\end{equation}

Substituting into the classical resistivity formula yields:
\begin{equation}
\rho = \frac{m_e}{n e^2 \tau} = \frac{m_e v_F \sigma_s}{n e^2 l_0^{3-D}},
\end{equation}
with:
\begin{itemize}
\item $n$: electron density,
\item $m_e$: electron mass,
\item $e$: elementary charge.
\end{itemize}

\subsection{Temperature Dependence and Experimental Consequences}
The fractal structure modifies the temperature dependence compared to Drude theory:
\begin{equation}
\rho(T) \approx \rho_0 + A \cdot T^{D-1} \quad \text{(with } D-1 \approx 1.71\text{)}.
\end{equation}
This deviation from linear behavior ($\rho \sim T$) might be detectable in superconductors or nanostructures.

\chapter{Anhang}
\section{Der Aharonov-Bohm-Effekt}
\label{sec:aharonov-bohm}

Der \textbf{Aharonov-Bohm-Effekt} (AB-Effekt) ist ein grundlegendes Quantenphänomen, das zeigt, dass elektromagnetische Potentiale ($\vec{A}$, $\Phi$) eine direkte physikalische
Wirkung auf Quantenteilchen haben, selbst in Regionen wo die Felder ($\vec{E}$, $\vec{B}$) null sind.

\subsection{Experimentelle Anordnung}
Ein Elektronenstrahl wird in zwei Pfade aufgeteilt, die eine Region mit magnetischem Fluss $\Phi$ umschließen.

\subsection{Theoretische Beschreibung}
Die Wellenfunktion $\psi$ eines Teilchens mit Ladung $q$ wird durch das Vektorpotential $\vec{A}$ modifiziert:

\begin{equation}
\psi \rightarrow \psi \cdot \exp\left(i\frac{q}{\hbar}\int \vec{A}\cdot d\vec{l}\right)
\end{equation}

Die Phasendifferenz zwischen den beiden Pfaden beträgt:

\begin{equation}
\Delta\phi = \frac{q}{\hbar}\oint \vec{A}\cdot d\vec{l} = \frac{q}{\hbar}\Phi_B
\end{equation}

\subsection{Physikalische Bedeutung}
\begin{itemize}
\item \textbf{Nicht-Lokalität}: Quantenteilchen \enquote{spüren} $\vec{A}$ auch in feldfreien Regionen
\item \textbf{Topologische Invariante}: Die Phase hängt nur vom eingeschlossenen Fluss $\Phi_B$ ab
\item \textbf{Paradigmenwechsel}: Widerlegt die klassische Annahme, dass nur $\vec{E}$ und $\vec{B}$ physikalisch relevant sind
\end{itemize}

\subsection{Experimentelle Bestätigung}
\begin{itemize}
\item Theoretische Vorhersage: Aharonov \& Bohm (1959)
\item Erste Experimente: Chambers (1960), Tonomura et al. (1982)
\item Moderne Anwendungen: Quanteninterferometer, topologische Quantenmaterialien
\end{itemize}


\backmatter
\printbibliography[title=Literaturverzeichnis]
\printglossary[title=Glossar]
\printglossary[type=acronym, title=Abkürzungen]

\end{document}












\documentclass[10pt,twoside,openright]{book} % Standard Buchformat
\usepackage[a4paper,left=2.5cm,right=2cm,top=2cm,bottom=2.5cm]{geometry}
\usepackage[utf8]{inputenc}
\usepackage[ngerman]{babel}
\usepackage{amsfonts, amsmath, amssymb}
\usepackage{array}
\usepackage{ragged2e}
\usepackage{tabularx}
\usepackage{enumitem}
\usepackage{booktabs}
\usepackage{bm}
\usepackage{csquotes}
\usepackage{siunitx}
\usepackage{parskip}
\usepackage{listings}
\usepackage{xcolor}
\usepackage[labelfont=bf]{caption}
\usepackage{tcolorbox}
\usepackage{mathrsfs}
\usepackage{microtype}
%\usepackage{showlabels}
%\usepackage{refcheck}

\newtheorem{theorem}{Theorem} % Definiert das 'theorem'-Environment
\newtheorem{lemma}{Lemma}     % Falls Sie auch Lemmas verwenden möchten
%\showlabels{cite,label}
\renewcommand{\arraystretch}{1.1}
\numberwithin{equation}{section}
\definecolor{gray}{rgb}{0.5,0.5,0.5}

\begin{document}

\date{\today}
\maketitle

\section*{Zusammenfassung}
Diese Arbeit untersucht die Synthese der Weber-Gravitation (WG) mit der De-Broglie-Bohm-Theorie (DBT) als alternative Herangehensweise zu den etablierten Theorien der
Allgemeinen Relativitätstheorie (ART) und Quantenmechanik. Die Weber-Gravitation, ursprünglich in der Elektrodynamik formuliert, wird auf gravitative Wechselwirkungen
übertragen und durch die Einbeziehung des Quantenpotentials der DBT erweitert. 

Zentrales Element ist die verallgemeinerte Weber-Kraft, die neben dem klassischen Newtonschen Term zusätzliche Geschwindigkeits- und Beschleunigungsabhängigkeiten enthält.
Diese wird durch das nicht-lokale Quantenpotential der DBT ergänzt, wodurch eine deterministische Beschreibung quantenmechanischer Phänomene ermöglicht wird. Die kombinierte Theorie
zeigt bemerkenswerte Parallelen zwischen instantanen Korrelationen in Wellenphänomenen und nicht-lokalen Wechselwirkungen in der Quantenmechanik.

Anwendungen der WG-DBT-Synthese werden für verschiedene astrophysikalische Phänomene untersucht: 
\begin{itemize}
    \item Die Periheldrehung des Merkurs ergibt sich natürlicherweise aus dem Weber-Formalismus mit $\beta=0.5$
    \item Galaktische Rotationskurven werden ohne dunkle Materie durch den DBT-Beitrag erklärt
    \item Lichtablenkung und Shapiro-Effekt zeigen charakteristische Abweichungen von der ART
\end{itemize}

Die Arbeit argumentiert für einen erweiterten Kausalitätsbegriff, der instantane Wechselwirkungen als systeminterne Rückkopplungen interpretiert. Mathematisch manifestiert
sich dies in der kovarianten Formulierung der Bewegungsgleichungen, die Jerk-Terme und Quantenpotentiale vereint. Die Ergebnisse legen nahe, dass die WG-DBT-Synthese eine
vielversprechende Grundlage für eine singularitätsfreie Quantengravitation bieten könnte.

\tableofcontents

% Einbindung der einzelnen TeX-Dateien
\part{Grundlagen}
\chapter{Weber-Kraft}
\label{chapter:weber_kraft}
\section{Klassische Weber-Kraft (Elektrodynamik)}
\begin{equation}
    \boxed
    {
        \bm{F}_{\text{Weber}}^{\text{EM}} = \frac{Qq}{4\pi\epsilon_0 r^2}\left(1 - \frac{\dot{r}^2}{c^2} + \frac{2r\ddot{r}}{c^2}\right)\bm{\hat{r}}
    }
\end{equation}

\subsection*{Symbolbeschreibung}
\begin{itemize}[leftmargin=*,noitemsep]
    \item $\bm{F}_{\text{Weber}}^{\text{EM}}$: Weber-Kraft zwischen Ladungen
    \item $Q, q$: Elektrische Ladungen
    \item $\epsilon_0$: Elektrische Feldkonstante
    \item $r$: Ladungsabstand
    \item $\dot{r} = \frac{dr}{dt}$: Relative Radialgeschwindigkeit
    \item $\ddot{r} = \frac{d^2r}{dt^2}$: Relative Radialbeschleunigung
    \item $c$: Lichtgeschwindigkeit
    \item $\bm{\hat{r}}$: Radialer Einheitsvektor
\end{itemize}

\subsection*{Beziehung zur Coulomb-Kraft}
\begin{itemize}[leftmargin=*,noitemsep]
    \item Erster Term entspricht Coulomb-Kraft: $\frac{Qq}{4\pi\epsilon_0 r^2}$
    \item Zusatzterme $\left(-\frac{\dot{r}^2}{c^2} + \frac{2r\ddot{r}}{c^2}\right)$ beschreiben Bewegungsabhängige Korrekturen
    \item Reduktion auf Coulomb-Kraft im statischen Fall ($\dot{r} = \ddot{r} = 0$)
\end{itemize}

\subsection*{Vergleich mit Maxwell-Theorie}
\begin{itemize}[leftmargin=*,noitemsep]
    \item Alternative Beschreibung elektromagnetischer Phänomene \cite{weber1846}
    \item Fernwirkungsansatz (direkte Ladungswechselwirkung)
    \item Implizite Retardierung durch Geschwindigkeits-/Beschleunigungsterme
    \item Keine Vorhersage von EM-Wellen im Vakuum
\end{itemize}

\subsection{Ansatz zur Weber-Gravitation (WG)}
\begin{itemize}[leftmargin=*,noitemsep]
    \item Kein vordefiniertes Raummodell benötigt
    \item Natürliche Diskretisierung durch Punktteilchen
    \item Gravitative Erweiterung möglich:
\end{itemize}

\begin{equation}
\bm{F}_{\text{Weber}}^{G} = G\frac{mM}{r^2}\left(1 - \frac{\dot{r}^2}{c^2} + \frac{2r\ddot{r}}{c^2}\right)\bm{\hat{r}}
\end{equation}

\subsection*{Zusammenfassung}
\begin{itemize}[leftmargin=*,noitemsep]
    \item Umgeht Quantisierungsprobleme der ART
    \item Ermöglicht diskrete Raumzeitmodelle
    \item Potentieller Brückenansatz zur Quantengravitation
\end{itemize}

\section{Weber-Kraft und Gravitation}
\subsection*{Tisserands Ansatz}
Die Übertragung der elektrodynamischen Weber-Kraft \cite{tisserand1894} auf die Gravitation scheiterte an der Erklärung der Periheldrehung des Merkurs.

\subsection*{Hinweis}
Die korrekte gravitative Formulierung wird separat vorgestellt und erfordert eine Modifikation der Original-Weberschen Formel.

\input{content/07_grundgleichungen}
\chapter{Instantane Energieverteilung und Kausalität}
\input{content/12_kausal}
\chapter{WG-DBT Synthese}
\section{Relativistische Energie-Impuls-Beziehung in der WG-DBT-Synthese}
\label{sec:energy-momentum}

Die Herleitung der relativistischen Energie-Impuls-Beziehung aus der Weber-Gravitation (WG) und De-Broglie-Bohm-Theorie (DBT) erfolgt wie folgt:

\subsection{Grundgleichungen}
Ausgehend von der verallgemeinerten Weber-Kraft für ein freies Teilchen:
\begin{equation}
\label{eq:wg_dbt_srt}
    \boxed
    {
        m\frac{d}{dt}(\gamma\mathbf{v}) = -\nabla Q
    }
\end{equation}
mit:
\begin{itemize}
\item $\gamma = (1 - \frac{v^2}{c^2} + \beta\frac{\mathbf{v}\cdot\mathbf{a}}{c^2})^{-1/2}$ (Weber-Lorentz-Faktor)
\item $Q = -\frac{\hbar^2}{2m}\frac{\nabla^2|\Psi|}{|\Psi|}$ (Quantenpotential)
\end{itemize}

\subsection{Stationäre Lösung}
Für $\mathbf{F} = 0$ und konstante Geschwindigkeit ($\mathbf{a} = 0$):
\begin{equation}
\gamma m\mathbf{v} = \mathbf{p} = \text{konstant}
\end{equation}
Mit der DBT-Impulsdefinition:
\begin{equation}
\mathbf{p} = \hbar\nabla S
\end{equation}

\subsection{Energie-Impuls-Relation}
\begin{align}
E &= \gamma mc^2 = \frac{mc^2}{\sqrt{1-v^2/c^2}} \\
p^2 &= \gamma^2m^2v^2 = \frac{m^2v^2}{1-v^2/c^2} \\
\Rightarrow v^2 &= \frac{p^2c^2}{m^2c^2 + p^2} \\
E &= \sqrt{m^2c^4 + p^2c^2}
\end{align}

\subsection{Kovariante Formulierung}
\begin{equation}
p^\mu p_\mu = \frac{E^2}{c^2} - p^2 = m^2c^2
\end{equation}

\subsection{Interpretation}
\begin{itemize}
\item Die WG liefert die relativistische Dynamik
\item Die DBT verknüpft diese mit der Quantenmechanik
\item Die SRT-Relation emergiert als Grenzfall
\item Das Quantenpotential $Q$ führt zu zusätzlichen Quanteneffekten
\end{itemize}

\begin{table}[h]
\centering
\caption{Grenzfälle der Energie-Impuls-Beziehung}
\begin{tabular}{ll}
\hline
Nicht-relativistisch ($v \ll c$) & $E \approx mc^2 + \frac{p^2}{2m} + Q$ \\
Ultra-relativistisch ($v \to c$) & $E \approx pc$ \\
Quantenlimit & $E \approx \sqrt{p^2c^2 + m^2c^4} + Q$ \\
\hline
\end{tabular}
\end{table}

\subsection*{Ontologischer Status der SRT}
Die Spezielle Relativitätstheorie stellt sich in diesem Rahmen als \textit{effektive Beschreibung} heraus, die:
\begin{itemize}
\item Im Bereich \( v \ll c \), \( L \gg \ell_p \) gültig ist
\item Aber durch tiefere Prinzipien (Fernwirkung + Führungswelle) ersetzt wird
\end{itemize}

\section{Exakte Herleitung der Weber-DBT-Bewegungsgleichung}
\label{sec:exact_derivation}

Ausgehend von der Weber-Gravitationskraft und dem Quantenpotential der De-Broglie-Bohm-Theorie leiten wir die vollständige nicht-genäherte Bewegungsgleichung ab.

\subsection{Kombinierte Lagrange-Funktion}
Die Wirkung des Systems setzt sich aus kinetischer Energie, Weber-Potential und Quantenpotential zusammen:

\begin{equation}
\mathcal{L} = \underbrace{\frac{1}{2}m\dot{\mathbf{r}}^2}_{T} - \underbrace{\frac{GMm}{r}\left[1 - \frac{\dot{r}^2}{2c^2} + \beta\frac{\mathbf{r}\cdot\ddot{\mathbf{r}}}{2c^2}\right]}_{V_{\text{WG}}} - \underbrace{Q(\mathbf{r},t)}_{\text{Quantenpotential}}
\end{equation}

mit dem Quantenpotential $Q = -\frac{\hbar^2}{2m}\frac{\nabla^2|\Psi|}{|\Psi|}$.

\subsection{Euler-Lagrange-Gleichung}
Die exakte Bewegungsgleichung folgt aus:

\begin{equation}
\frac{d}{dt}\left(\frac{\partial\mathcal{L}}{\partial\dot{\mathbf{r}}}\right) - \frac{\partial\mathcal{L}}{\partial\mathbf{r}} = 0
\end{equation}

\subsection{Ableitung der Terme}
\begin{enumerate}
\item \textbf{Kanonischer Impuls}:
\begin{align}
\frac{\partial\mathcal{L}}{\partial\dot{\mathbf{r}}} &= m\dot{\mathbf{r}} + \frac{GMm}{c^2}\left(\frac{\dot{\mathbf{r}}}{r} - \beta\frac{\mathbf{r}}{2r}\frac{d}{dt}\ln\dot{r}\right) \\
&= m\dot{\mathbf{r}}\left[1 + \frac{GM}{c^2r}\left(1 - \frac{\beta}{2}\frac{\mathbf{r}\cdot\ddot{\mathbf{r}}}{\dot{r}^2}\right)\right]
\end{align}

\item \textbf{Zeitableitung}:
\begin{equation}
\frac{d}{dt}\left(\frac{\partial\mathcal{L}}{\partial\dot{\mathbf{r}}}\right) = m\ddot{\mathbf{r}}\left[1 + \mathcal{O}(c^{-2})\right] + \text{höhere Ableitungen}
\end{equation}

\item \textbf{Ortsableitung}:
\begin{equation}
\frac{\partial\mathcal{L}}{\partial\mathbf{r}} = -\frac{GMm}{r^2}\left[1 - \frac{3\dot{r}^2}{2c^2} + \beta\frac{\ddot{r}}{c^2}\right]\hat{\mathbf{r}} - \nabla Q
\end{equation}
\end{enumerate}

\subsection{Exakte Bewegungsgleichung}
Durch Zusammenführung aller Terme erhalten wir die nicht-genäherte Gleichung:

\begin{equation}
\boxed{
m\frac{d}{dt}\left(\gamma_{\text{WG}}\mathbf{v}\right) = -\nabla Q
}
\end{equation}

mit dem vollständigen Weber-Lorentz-Faktor:

\begin{equation}
\gamma_{\text{WG}} = \left[1 - \frac{v^2}{c^2} + \beta\left(\frac{\mathbf{a}\cdot\mathbf{r}}{c^2} + \frac{(\mathbf{v}\cdot\mathbf{r})^2}{c^2r^2}\right) - \frac{GM}{c^2r}\left(1 - \frac{\beta}{2}\frac{\mathbf{r}\cdot\mathbf{j}}{\dot{r}^2}\right)\right]^{-1/2}
\end{equation}

wobei $\mathbf{j} = d\mathbf{a}/dt$ die Jerk-Komponente darstellt.

\subsection{Diskussion der Terme}
\begin{itemize}
\item Der Term $\propto \mathbf{j}$ beschreibt nicht-lokale Änderungen der Beschleunigung
\item Die Kopplung $\mathbf{a}\cdot\mathbf{r}$ modifiziert effektiv die träge Masse
\item Für $\beta=0$ und $Q=0$ reduziert sich die Gleichung auf die spezielle Relativitätstheorie
\end{itemize}

\section{Kovariante Formulierung der exakten Weber-DBT-Gleichung}
\label{sec:covariant_formulation}

Die vollständige kovariante Formulierung der Weber-Dynamik kombiniert mit der De-Broglie-Bohm-Theorie erfordert eine manifest relativistische Darstellung unter Berücksichtigung aller höherer Ordnungen.

\subsection{Kovariante Grundgrößen}
Wir definieren in Minkowski-Raumzeit mit Metrik $\eta_{\mu\nu} = \mathrm{diag}(-1,1,1,1)$:

\begin{align}
\text{Vierergeschwindigkeit:} &\quad u^\mu = \gamma(c, \mathbf{v}), \quad \gamma = (1-v^2/c^2)^{-1/2} \\
\text{Eigenbeschleunigung:} &\quad a^\mu = \frac{du^\mu}{d\tau} = \gamma^4\left(\frac{\mathbf{v}\cdot\mathbf{a}}{c}, \mathbf{a} + \gamma^2\frac{(\mathbf{v}\cdot\mathbf{a})\mathbf{v}}{c^2}\right) \\
\text{Eigen-Jerk:} &\quad j^\mu = \frac{da^\mu}{d\tau} = \gamma^7\left(\frac{a^2 + \mathbf{v}\cdot\mathbf{j}}{c}, \mathbf{j} + 3\gamma^2\frac{(\mathbf{v}\cdot\mathbf{a})\mathbf{a}}{c^2} + \gamma^2\frac{(\mathbf{v}\cdot\mathbf{j})\mathbf{v}}{c^2}\right)
\end{align}

\subsection{Exakter Weber-Lorentz-Faktor}
Der vollständige relativistische Faktor inklusive Jerk-Termen lautet:

\begin{equation}
\gamma_{\mathrm{WG}} = \left[1 - \frac{v^2}{c^2} + \beta\left(\frac{\mathbf{r}\cdot\mathbf{a}}{c^2} + \frac{(\mathbf{v}\cdot\mathbf{r})^2}{c^2r^2}\right) - \beta\frac{GM}{c^4}\left(\frac{\mathbf{r}\cdot\mathbf{j}}{r} + \frac{(\mathbf{v}\cdot\mathbf{r})(\mathbf{a}\cdot\mathbf{r})}{r^3}\right)\right]^{-1/2}
\end{equation}

\subsection{Kovariante Bewegungsgleichung}
Die exakte kovariante Form der Weber-DBT-Dynamik:

\begin{equation}
\boxed{
m\frac{D}{D\tau}\left(\gamma_{\mathrm{WG}} u^\mu\right) = -\frac{\hbar^2}{2m}\partial^\mu\left(\frac{\Box|\Psi|}{|\Psi|}\right)
}
\end{equation}

mit:
\begin{itemize}
\item Kovariante Ableitung: $\frac{D}{D\tau} = u^\nu\partial_\nu$
\item d'Alembert-Operator: $\Box = \partial_\mu\partial^\mu = -\frac{1}{c^2}\frac{\partial^2}{\partial t^2} + \nabla^2$
\end{itemize}

\subsection{Komponentenentwicklung}

\subsubsection{Zeitkomponente ($\mu=0$)}
\begin{equation}
\frac{d}{d\tau}\left(\gamma_{\mathrm{WG}}\gamma c\right) = \frac{\hbar^2}{2mc^2}\frac{\partial}{\partial t}\left(\frac{\Box|\Psi|}{|\Psi|}\right)
\end{equation}

\subsubsection{Raumkomponenten ($\mu=1,2,3$)}
\begin{equation}
\frac{d}{d\tau}\left(\gamma_{\mathrm{WG}}\gamma\mathbf{v}\right) = -\frac{\hbar^2}{2m}\nabla\left(\frac{\Box|\Psi|}{|\Psi|}\right)
\end{equation}

\subsection{Diskussion der Terme}
\begin{itemize}
\item \textbf{Jerk-Abhängigkeit}: Die $\mathbf{j}$-Terme in $\gamma_{\mathrm{WG}}$ beschreiben nicht-lokale Fernwirkungseffekte
\item \textbf{Quantenpotential}: Der kovariante d'Alembert-Operator $\Box$ ersetzt das klassische $\nabla^2$
\item \textbf{Energieerhaltung}: Die Zeitkomponente enthält Korrekturen zur relativistischen Energie-Impuls-Beziehung
\end{itemize}

\begin{table}[h]
\centering
\caption{Vergleich der Formulierungen}
\begin{tabular}{ll}
\hline
\textbf{Genäherte Form (4.1.1)} & \textbf{Exakte kovariante Form} \\
\hline
$\gamma_{\mathrm{WG}} \approx 1 + \frac{v^2}{2c^2}$ & Vollständige Jerk-Abhängigkeit \\
$-\nabla Q$ & $-\partial^\mu(\hbar^2\Box|\Psi|/2m|\Psi|)$ \\
Newton-artige Darstellung & Manifest kovariant \\
\hline
\end{tabular}
\end{table}

\section{Anwendungsfälle der exakten kovarianten Weber-DBT-Dynamik}
\label{sec:applications}

Die vollständige kovariante Formulierung der Weber-DBT-Gleichungen wird in folgenden physikalischen Szenarien benötigt, wo Näherungen der vereinfachten Versionen versagen:

\subsection{Starke Gravitationsfelder}
\begin{itemize}
    \item \textbf{Schwarze Löcher und Neutronensterne}: 
    \begin{itemize}
        \item Dominanz der Jerk-Terme ($\mathbf{j} = d\mathbf{a}/dt \sim GM/r^3$) bei $r \to r_s$
        \item Nicht-Newtonsche Gezeitenkräfte und Bahninstabilitäten
    \end{itemize}
    
    \item \textbf{Kollidierende Binärsysteme}:
    \begin{itemize}
        \item Präzise Modellierung der Gravitationswellen-Emission
        \item Korrekturen zur Post-Newton-Näherung
    \end{itemize}
\end{itemize}

\subsection{Relativistische Teilchendynamik}
\begin{itemize}
    \item \textbf{Teilchenbeschleuniger} ($v \approx c$):
    \begin{itemize}
        \item Modifikation der Synchrotronstrahlung
        \item Abweichungen von SRT-Vorhersagen
    \end{itemize}
    
    \item \textbf{Ultrahoch-energetische kosmische Strahlung}:
    \begin{itemize}
        \item Korrekturen zur GZK-Cutoff-Energie
        \item Nicht-lokale Wechselwirkungsterme
    \end{itemize}
\end{itemize}

\subsection{Quantengravitation}
\begin{itemize}
    \item \textbf{Planck-Skalen-Physik} ($\ell \sim \ell_P$):
    \begin{itemize}
        \item Singularitätsfreie Lösungen durch Quantenpotential $Q$
        \item Nicht-lokale Wellenfunktionskorrelationen
    \end{itemize}
    
    \item \textbf{Quantenchaos in Gravitationsfeldern}:
    \begin{itemize}
        \item Zusätzliche Lyapunov-Exponenten durch Jerk-Terme
        \item Modifizierte Stabilitätsbedingungen
    \end{itemize}
\end{itemize}

\subsection{Kosmologische Modelle}
\begin{itemize}
    \item \textbf{Statisches Universum}:
    \begin{itemize}
        \item Beschreibung der Rotverschiebung als kumulative Wechselwirkung
        \item Alternative zur kosmischen Expansion
    \end{itemize}
    
    \item \textbf{Dunkle Materie-Phänomene}:
    \begin{itemize}
        \item Erklärung galaktischer Rotationskurven ohne dunkle Materie
        \item Modifizierte Geschwindigkeitsprofile
    \end{itemize}
\end{itemize}

\subsection{Experimentelle Tests}
\begin{itemize}
    \item \textbf{Frequenzabhängige Lichtablenkung}:
    \begin{itemize}
        \item Nachweis mit VLBI-Technologien (z.B. Event Horizon Telescope)
        \item Wellenlängenabhängige Korrekturen zum Shapiro-Effekt
    \end{itemize}
    
    \item \textbf{Atominterferometrie}:
    \begin{itemize}
        \item Messung von Jerk-induzierten Phasenverschiebungen
        \item Tests der nicht-lokalen Quantenkorrelationen
    \end{itemize}
\end{itemize}

\begin{table}[h]
\centering
\caption{Zusammenfassung der Anwendungsfälle}
\label{tab:applications}
\begin{tabular}{lp{8cm}}
\toprule
\textbf{Szenario} & \textbf{Relevante physikalische Effekte} \\
\midrule
Schwarze Löcher & Dominanz der Jerk-Terme, nicht-Newtonsche Gezeitenkräfte \\
Teilchenbeschleuniger & Abweichungen von SRT bei $v \approx c$ \\
Planck-Skalen & Singularitätsfreiheit durch Quantenpotential $Q$ \\
Kosmologie & Statisches Universum ohne Expansion \\
Experimente & Frequenzabhängige Gravitationseffekte \\
\bottomrule
\end{tabular}
\end{table}

\newpage
\section{Rotationskurven in der Weber-DBT-Gravitation}
Die Rotationsgeschwindigkeiten von Galaxien lassen sich durch eine Kombination der Weber-Gravitation (WG) mit der De-Broglie-Bohm-Theorie (DBT) erklären, ohne auf dunkle Materie zurückzugreifen. 

\subsection{Theoretische Grundlagen}
Die Bewegungsgleichung für ein Testteilchen der Masse $m$ im Gravitationsfeld einer Galaxie lautet in der WG-DBT-Synthese:

\begin{equation}
m \frac{d}{dt}(\gamma_{\text{WG}} \mathbf{v}) = -\frac{GMm}{r^2}\left(1 - \frac{\dot{r}^2}{c^2} + \beta \frac{r\ddot{r}}{c^2}\right)\hat{\mathbf{r}} - \nabla Q
\end{equation}

wobei:
\begin{itemize}
\item $\gamma_{\text{WG}} = \left(1 - \frac{v^2}{c^2} + \beta \frac{\mathbf{r}\cdot\mathbf{a}}{c^2}\right)^{-1/2}$ der Weber-Lorentz-Faktor ist ($\beta = 0.5$)
\item $Q = -\frac{\hbar^2}{2m}\frac{\nabla^2|\Psi|}{|\Psi|}$ das Quantenpotential der DBT darstellt
\end{itemize}

\subsection{Stationäre Lösung für Kreisbahnen}
Für stabile Kreisbahnen ($\dot{r} = 0$, $\ddot{r} = -v^2/r$) vereinfacht sich dies zu:

\begin{equation}
\frac{v^2}{r} = \frac{GM(r)}{r^2} + \frac{\hbar^2}{2m^2}\left|\frac{\nabla^2\sqrt{\rho}}{\sqrt{\rho}}\right|
\end{equation}

Mit der angenommenen Dichteverteilung $\rho(r) = \rho_0 e^{-r/r_0}$ ergibt sich:

\begin{equation}
v^2(r) = \underbrace{\frac{GM(r)}{r}}_{\text{Baryonisch}} + \underbrace{\frac{\hbar^2}{2m^2 r_0 R}}_{\text{DBT-Korrektur}} + \mathcal{O}\left(\frac{v^2}{c^2}\right)
\end{equation}

\subsection{Physikalische Interpretation}
Die nicht-lokale Natur der DBT-Führungswelle $\Psi$ führt zu einem konstanten Geschwindigkeitsbeitrag $v_0$:

\begin{equation}
v_0^2 \equiv \frac{\hbar^2}{2m^2 r_0 R}
\end{equation}

wobei:
\begin{itemize}
\item $m \approx 2\pi \times 10^{-40}\,\text{kg}$ eine natürliche Massenskala darstellt
\item $r_0$ die Skalenlänge der Galaxie ist
\item $R$ den charakteristischen Wirkungsradius der Führungswelle beschreibt
\end{itemize}

Diese Formulierung zeigt, dass die beobachteten flachen Rotationskurven durch die Kombination von:
\begin{enumerate}
\item relativistischen Korrekturen der Weber-Gravitation ($\beta$-Term)
\item nicht-lokalen Quanteneffekten der DBT ($v_0$-Term)
\end{enumerate}
erklärt werden können - ohne Einführung dunkler Materie.

\subsection{Berechnungsbeispiel einer Rotationskurve}

Für eine typische Spiralgalaxie mit folgenden Parametern:
\begin{itemize}
\item Gesamtmasse der sichtbaren Materie: $M = 10^{11} M_\odot$
\item Skalenlänge: $r_0 = 3\ \text{kpc}$
\item Charakteristischer Radius: $R = 15\ \text{kpc}$
\item DBT-Massenskala: $m = 2\pi \times 10^{-40}\ \text{kg} \approx 1.2 \times 10^{-3}\ \text{eV}/c^2$
\end{itemize}

Die Rotationsgeschwindigkeit setzt sich zusammen aus:

\begin{equation}
v(r) = \sqrt{v_b^2(r) + v_0^2}
\end{equation}

mit:
\begin{align*}
v_b(r) &= \sqrt{\frac{GM(r)}{r}} \quad \text{(baryonischer Anteil)} \\
v_0 &= \sqrt{\frac{\hbar^2}{2m^2 r_0 R}} \quad \text{(DBT-Korrektur)}
\end{align*}

\begin{table}[h]
\centering
\caption{Berechnete Rotationsgeschwindigkeiten für verschiedene Radien}
\label{tab:rotation}
\begin{tabular}{cccc}
\hline
Radius $r$ (kpc) & $v_b$ (km/s) & $v_0$ (km/s) & $v_{\text{gesamt}}$ (km/s) \\
\hline
1 & 125.4 & 73.8 & 145.2 \\
3 & 129.1 & 73.8 & 148.6 \\ 
5 & 124.7 & 73.8 & 144.9 \\
10 & 110.3 & 73.8 & 132.5 \\
15 & 95.2 & 73.8 & 120.4 \\
20 & 82.4 & 73.8 & 110.8 \\
30 & 67.2 & 73.8 & 99.9 \\
\hline
\end{tabular}
\end{table}

Die Berechnung zeigt:
\begin{itemize}
\item Den klassisch keplerschen Abfall des baryonischen Anteils $v_b(r)$
\item Den konstanten DBT-Beitrag $v_0 \approx 74\ \text{km/s}$
\item Die resultierende flache Rotationskurve für $r > 10\ \text{kpc}$
\end{itemize}

\noindent Die Übereinstimmung mit beobachteten Werten (typisch $100-200\ \text{km/s}$) bestätigt die Wirksamkeit des WG-DBT-Ansatzes.

\newpage
\section{Lichtablenkung in der Weber-DBT-Gravitation}
\label{sec:light_deflection}

Die Ablenkung von Licht im Gravitationsfeld lässt sich in der Weber-DBT-Theorie durch eine Modifikation der geodätischen Gleichung beschreiben. Wir leiten den Ablenkwinkel $\alpha$ für einen Lichtstrahl mit Stoßparameter $b$ her.

\subsection{Bewegungsgleichung für Photonen}
Aus der WG-DBT-Gleichung folgt für masselose Teilchen ($m \to 0$, aber $E = h\nu \neq 0$):

\begin{equation}
\frac{d}{d\lambda}\left(\frac{dx^\mu}{d\lambda}\right) = -\Gamma^\mu_{\nu\sigma}\frac{dx^\nu}{d\lambda}\frac{dx^\sigma}{d\lambda} - \frac{1}{E}\nabla^\mu Q
\end{equation}

wobei:
\begin{itemize}
\item $\lambda$ ein affiner Parameter ist
\item $Q = -\frac{\hbar^2}{2E}\frac{\Box|\Psi|}{|\Psi|}$ das quantenmechanische Potential für Photonen
\item $\Gamma^\mu_{\nu\sigma}$ die Weber-Korrekturen zu den Christoffel-Symbolen enthält
\end{itemize}

\subsection{Lösung für kleine Ablenkungen}
Für einen Lichtstrahl in $z$-Richtung mit Stoßparameter $b$ lautet die transversale Beschleunigung:

\begin{equation}
\frac{d^2x}{dz^2} \approx -\frac{GM}{c^2}\left(\frac{1}{b^2} + \beta\frac{\partial^2_x \Phi}{c^2}\right)x - \frac{\hbar^2}{2E^2}\partial_x\left(\frac{\Box|\Psi|}{|\Psi|}\right)
\end{equation}

mit $\Phi = -GM/r$ dem Newton-Potential. 

\subsection{Quantenpotential für Licht}
Für eine ebene Welle $|\Psi| \propto e^{-r^2/2\sigma^2}$ ergibt sich:

\begin{equation}
Q \approx -\frac{\hbar^2}{2E\sigma^2}\left(1 - \frac{r^2}{\sigma^2}\right)
\end{equation}

Die typische Wirkungsskala ist $\sigma \sim b$, sodass:

\begin{equation}
\frac{1}{E}\nabla_x Q \approx \frac{\hbar^2}{E^2b^3}x
\end{equation}

\subsection{Integrierter Ablenkwinkel}
Der Gesamtablenkwinkel $\alpha$ ergibt sich durch Integration entlang der Trajektorie:

\begin{align}
\alpha &= \frac{2GM}{c^2b}\left(1 + \beta\frac{2GM}{c^2b}\right) + \frac{\pi\hbar^2}{4E^2b^2} \\
&= \underbrace{\frac{4GM}{c^2b}}_{\text{Einstein (ART)}} + \underbrace{\frac{2GM}{c^2b}\left(\beta - 2\right)}_{\text{Weber-Korrektur}} + \underbrace{\frac{\pi\hbar^2}{4E^2b^2}}_{\text{DBT-Term}}
\end{align}

Für $\beta = 1$ (Lichtablenkung) und $E = h\nu$:

\begin{equation}
\boxed{
\alpha = \frac{4GM}{c^2b} - \frac{2GM}{c^2b} + \frac{\pi h^2}{4(h\nu)^2b^2}
}
\end{equation}

\section{Shapiro-Effekt in der Weber-DBT-Gravitation}
\label{sec:shapiro_effect}

Der Shapiro-Effekt beschreibt die gravitative Zeitverzögerung von Lichtsignalen. In der Weber-DBT-Theorie ergibt sich eine modifizierte Version dieses Effekts durch die Kombination aus Weber-Gravitation und Quantenpotential.

\subsection{Laufzeitverzögerung}
Für ein Lichtsignal, das an einer Masse $M$ mit minimalem Abstand $b$ vorbeiläuft, beträgt die zusätzliche Laufzeit:

\begin{equation}
\Delta t = \frac{2GM}{c^3}\left[\ln\left(\frac{4r_e r_p}{b^2}\right) + \frac{\beta GM}{c^2b}\right] + \frac{\hbar^2}{4E^2c^3b^2}(r_e + r_p)
\end{equation}

wobei:
\begin{itemize}
\item $r_e$ und $r_p$ die Abstände von Masse zu Emitter bzw. Detektor sind
\item $\beta = 0.5$ für die Weber-Gravitation
\item $E = h\nu$ die Photonenenergie
\end{itemize}

\subsection{Herleitung}
Aus der WG-DBT-Metrik für schwache Felder:

\begin{equation}
ds^2 = -\left(1 - \frac{2GM}{c^2r} + \frac{Q}{E}\right)c^2dt^2 + \left(1 + \frac{2GM}{c^2r}\right)dr^2
\end{equation}

Die Lichtlaufzeit folgt aus:

\begin{equation}
\Delta t = 2\int_{b}^{r_e} \frac{1}{c}\left[\left(1 + \frac{2GM}{c^2r} - \frac{Q}{E}\right)^{-1} - 1\right] dr
\end{equation}

Mit dem Quantenpotential $Q \approx -\hbar^2/(2Eb^2)$ für $r \approx b$:

\begin{equation}
\Delta t \approx \frac{2GM}{c^3}\ln\left(\frac{4r_e r_p}{b^2}\right) + \frac{\beta G^2M^2}{c^5b} + \frac{\hbar^2(r_e + r_p)}{4E^2c^3b^2}
\end{equation}

\subsection{Physikalische Interpretation}
\begin{itemize}
\item Der erste Term entspricht der klassischen ART-Vorhersage
\item Der Weber-Term ($\beta$) führt zu einer zusätzlichen $1/b$-Abhängigkeit
\item Der DBT-Term zeigt charakteristische Frequenzabhängigkeit ($\propto \nu^{-2}$)
\end{itemize}

Für Radarsignale ($\nu \sim 10^{10}$ Hz) im Sonnensystem:
\begin{equation}
\Delta t_{\text{WG-DBT}} \approx 240\,\mu\text{s} - 10^{-36}\,\mu\text{s} + 10^{-72}\,\mu\text{s}
\end{equation}

Die Quantenkorrektur ist vernachlässigbar, aber prinzipiell vorhanden.

\chapter{WG-DBT-Kinetik}
\section{Bahngleichung in der Weber-DBT-Gravitation}
\label{sec:bahngleichung}

Die Kombination der Weber-Gravitation (WG) mit der De-Broglie-Bohm-Theorie (DBT) führt zu einer modifizierten Bahndynamik, die durch eine nichtlineare Differentialgleichung beschrieben wird. Im Folgenden leiten wir die exakte Bahngleichung $r(\phi)$ her.

\subsection{Kraftgleichung und Potentiale}
Ausgehend von der verallgemeinerten Bewegungsgleichung (Gl.~3.2.7 der Arbeit):

\begin{equation}
m \frac{d}{dt}(\gamma_{\mathrm{WG}}\mathbf{v}) = \mathbf{F}_{\mathrm{WG}} + \mathbf{F}_Q
\end{equation}

mit den Komponenten:
\begin{itemize}
\item Weber-Gravitationskraft:
\begin{equation}
\mathbf{F}_{\mathrm{WG}} = -\frac{GMm}{r^2}\left(1-\frac{\dot{r}^2}{c^2}+\beta\frac{r\ddot{r}}{c^2}\right)\hat{\mathbf{r}}
\end{equation}

\item Quantenkraft:
\begin{equation}
\mathbf{F}_Q = -\nabla Q = \frac{\hbar^2}{2m}\nabla\left(\frac{\nabla^2|\Psi|}{|\Psi|}\right)
\end{equation}

\item Weber-Lorentz-Faktor:
\begin{equation}
\gamma_{\mathrm{WG}} = \left[1-\frac{v^2}{c^2}+\beta\left(\frac{\mathbf{a}\cdot\mathbf{r}}{c^2}+\frac{(\mathbf{v}\cdot\mathbf{r})^2}{c^2r^2}\right)\right]^{-1/2}
\end{equation}
\end{itemize}

\subsection{Transformation auf Polarkoordinaten}
Mit den Polarkoordinaten $(r,\phi)$ und dem spezifischen Drehimpuls $h = r^2\dot{\phi} = \mathrm{const.}$ ergibt sich:

\begin{align}
\mathbf{v} &= \dot{r}\hat{\mathbf{r}} + r\dot{\phi}\hat{\boldsymbol{\phi}} \\
\mathbf{a} &= (\ddot{r}-r\dot{\phi}^2)\hat{\mathbf{r}} + (r\ddot{\phi}+2\dot{r}\dot{\phi})\hat{\boldsymbol{\phi}}
\end{align}

\subsection{Radiale Komponente der Bewegungsgleichung}
Die radiale Komponente lautet:

\begin{equation}
\frac{d}{dt}(\gamma_{\mathrm{WG}}\dot{r}) - \gamma_{\mathrm{WG}}r\dot{\phi}^2 = -\frac{GM}{r^2}\left(1-\frac{\dot{r}^2}{c^2}+\beta\frac{r\ddot{r}}{c^2}\right) + \frac{\hbar^2}{2m^2}\frac{\partial}{\partial r}\left(\frac{\nabla^2|\Psi|}{|\Psi|}\right)
\end{equation}

\subsection{Substitution und exakte Differentialgleichung}
Mit der Variablentransformation $u = 1/r$ und den Ableitungen:

\begin{align}
\dot{r} &= -h\frac{du}{d\phi} \\
\ddot{r} &= -h^2u^2\frac{d^2u}{d\phi^2}
\end{align}

erhalten wir die nichtlineare Bahngleichung:

\begin{equation}
\boxed{
\frac{d^2u}{d\phi^2}\left(1-\beta\frac{GM}{c^2}u\right) + u = \frac{GM}{h^2}\left(1-\frac{h^2}{c^2}\left(\frac{du}{d\phi}\right)^2\right) - \frac{\hbar^2}{2m^2h^2u^2}\frac{d}{du}\left(\frac{\nabla^2|\Psi|}{|\Psi|}\right)
}
\label{eq:master}
\end{equation}

\subsection{Diskussion der Terme}
\begin{itemize}
\item Der Term $\propto \beta$ modifiziert die effektive Masse
\item Der $(\frac{du}{d\phi})^2$-Term entspricht der relativistischen Korrektur
\item Das Quantenpotential $\propto \hbar^2$ führt zu nicht-lokalen Effekten
\end{itemize}

\subsection{Grenzfälle}
\begin{enumerate}
\item \textbf{Newton'scher Grenzfall} ($c\to\infty$, $\hbar\to0$):
\begin{equation}
\frac{d^2u}{d\phi^2} + u = \frac{GM}{h^2}
\end{equation}

\item \textbf{Reine Weber-Gravitation} ($\hbar\to0$):
\begin{equation}
\frac{d^2u}{d\phi^2}\left(1-\beta\frac{GM}{c^2}u\right) + u = \frac{GM}{h^2}\left(1-\frac{h^2}{c^2}\left(\frac{du}{d\phi}\right)^2\right)
\end{equation}
\end{enumerate}

\begin{table}[h]
\centering
\caption{Parameter der Bahngleichung}
\begin{tabular}{ll}
\hline
Symbol & Physikalische Bedeutung \\ \hline
$\beta$ & Weber-Beschleunigungsparameter ($\beta=0.5$ für Gravitation) \\
$h$ & Spezifischer Drehimpuls \\
$Q$ & Quantenpotential \\
\hline
\end{tabular}
\end{table}

\section{Periheldrechnung in der Weber-DBT-Theorie}
Die Bewegungsgleichung der Weber-DBT-Synthese (Gl. 4.1.10) lautet vollständig:

\begin{equation}
\frac{d^2 u}{d\phi^2} \left(1 - \beta \frac{GM}{c^2} u \right) + u = \frac{GM}{h^2} \left(1 - \frac{h^2}{c^2} \left(\frac{du}{d\phi}\right)^2 \right) - \frac{\hbar^2}{2m^2 h^2 u^2} \frac{d}{du} \left(\frac{\nabla^2 |\Psi|}{|\Psi|} \right)
\end{equation}

\subsection{Quantenpotential-Explizierung}
Für das Quantenpotential wird die Wellenfunktion eines kohärenten makroskopischen Zustands angesetzt:
\begin{align}
|\Psi| &\propto e^{-(r - r_0)^2/(2\sigma^2)}, \quad \sigma \sim \text{Planetenradius} \\
\frac{\nabla^2 |\Psi|}{|\Psi|} &= \frac{1}{\sigma^2}\left(1 - \frac{(r - r_0)^2}{\sigma^2}\right) \\
\frac{d}{du} \left(\frac{\nabla^2 |\Psi|}{|\Psi|}\right) &= \frac{2r^3}{\sigma^4}(r - r_0)
\end{align}

\subsection{Vollständige Differentialgleichung}
Einsetzen aller Terme ergibt:
\begin{equation}
\frac{d^2 u}{d\phi^2} \left(1 - \frac{GM}{2c^2} u \right) + u = \frac{GM}{h^2} \left(1 - \frac{h^2}{c^2} \left(\frac{du}{d\phi}\right)^2 \right) - \frac{\hbar^2 r^3 (r - r_0)}{m^2 h^2 \sigma^4 u^2}
\end{equation}

\subsection{Störungstheorie um Newtonsche Lösung}
\begin{itemize}
\item Newtonsche Bahn: $u_0(\phi) = \frac{GM}{h^2}(1 + e \cos\phi)$
\item Ansatz: $u = u_0 + \delta u$ mit Störung $\delta u$
\item Exakte Störungsgleichung:
\begin{equation}
\frac{d^2 \delta u}{d\phi^2} + \delta u = \underbrace{\frac{GM}{2c^2} u_0 \frac{d^2 u_0}{d\phi^2}}_{\text{Weber-Term}} - \underbrace{\frac{h^2}{c^2} \left(\frac{du_0}{d\phi}\right)^2}_{\text{Relativistisch}} - \underbrace{\frac{\hbar^2 r^3 (r - r_0)}{m^2 h^2 \sigma^4 u_0^2}}_{\text{Quantenterm}}
\end{equation}
\end{itemize}

\subsection{Beitragsanalyse}
\begin{itemize}
\item Weber-Term: $-\frac{G^2 M^2 e}{2c^2 h^4} \cos\phi$
\item Relativistischer Term: $-\frac{G^2 M^2 e^2}{c^2 h^4} \sin^2\phi$
\item Quantenterm: $\mathcal{O}\left(\frac{\hbar^2}{m^2 \sigma^4}\left(\frac{h^2}{GM}\right)^5\right) \approx 10^{-80} \text{ (formal erhalten)}$
\end{itemize}

\subsection{Resultat}
Die Periheldrehung pro Umlauf ergibt sich aus der säkularen Drift:
\begin{equation}
\Delta \phi = \frac{6\pi GM}{c^2 a(1 - e^2)} + \mathcal{O}\left(\frac{\hbar^2}{m^2 \sigma^4}\right)
\end{equation}


\part{Anhang}
\chapter{Ergänzende Informationen}
\label{chapter:information}
\section{Die Rolle des $\beta$-Parameters}

Der $\beta$-Parameter in der Weber-Kraft

\begin{equation}
F = -\frac{GMm}{r^2}\left(1 - \frac{\dot{r}^2}{c^2} + \beta\frac{r\ddot{r}}{c^2}\right)\hat{r}
\end{equation}

bestimmt das Verhältnis von Beschleunigungs- zu Geschwindigkeitstermen und variiert je nach Wechselwirkungstyp:

\subsection{Elektrodynamik (Original-Weber)}
Für elektromagnetische Wechselwirkungen gilt $\beta=2$:
\begin{itemize}
\item Führt zur korrekten Beschreibung beschleunigter Ladungen
\item Reproduziert die magnetische Komponente der Lorentz-Kraft
\item Keine Lichtablenkung ($m=0$ liefert $F=0$)
\end{itemize}

\subsection{Gravitation (Massen)}
Für massive Körper im Gravitationsfeld:
\begin{itemize}
\item $\beta=0.5$ erklärt die Periheldrehung des Merkur
\item Führt zur ART-konformen Lichtablenkung für makroskopische Körper
\item Universelle Formel: $\beta = 1 - \frac{mc^2}{2E}$
\end{itemize}

\subsection{Photonen (Lichtablenkung)}
Für masselose Teilchen ($m=0$, $E=h\nu$):
\begin{itemize}
\item $\beta=1$ erzwingt die Frequenzabhängigkeit
\item Beschleunigungsterm dominiert: $\frac{r\ddot{r}}{c^2} \approx \frac{h^2}{c^2r^4}$
\item Liefert den Zusatzterm $\propto \lambda^{-2}$
\end{itemize}

\begin{table}[h]
\centering
\caption{$\beta$-Werte im Vergleich}
\begin{tabular}{lcc}
\hline
Anwendung & $\beta$ & Physikalische Konsequenz \\
\hline
Elektrodynamik & 2 & Magnetische Wechselwirkungen \\
Gravitation (Massen) & 0.5 & Periheldrehung des Merkur \\
Photonen & 1 & Frequenzabhängige Lichtablenkung \\
\hline
\end{tabular}
\end{table}
\section{Herleitung der kombinierten WG-DBT Bewegungsgleichung}

\subsection*{1. Ausgangspunkt: Weber-Gravitationskraft}
Die klassische Weber-Kraft für zwei Massen $m$ und $M$ lautet:
\begin{equation}
\mathbf{F}_{\text{WG}} = -\frac{GMm}{r^2}\left(1 - \frac{\dot{r}^2}{c^2} + \beta\frac{r\ddot{r}}{c^2}\right)\hat{\mathbf{r}}
\end{equation}

\subsection*{2. Umformung der radiale Beschleunigungsterme}
Wir entwickeln die Terme $\dot{r}^2$ und $r\ddot{r}$ in vektorieller Form:

\begin{align}
\dot{r} &= \frac{d}{dt}\sqrt{\mathbf{r}\cdot\mathbf{r}} = \frac{\mathbf{r}\cdot\mathbf{v}}{r} \\
\dot{r}^2 &= \left(\frac{\mathbf{r}\cdot\mathbf{v}}{r}\right)^2 \\
r\ddot{r} &= \frac{d}{dt}(r\dot{r}) - \dot{r}^2 = \mathbf{v}\cdot\mathbf{v} + \mathbf{r}\cdot\mathbf{a} - \left(\frac{\mathbf{r}\cdot\mathbf{v}}{r}\right)^2
\end{align}

Für kleine Abweichungen von Kreisbahnen vernachlässigen wir den letzten Term und erhalten:
\begin{equation}
r\ddot{r} \approx v^2 + \mathbf{r}\cdot\mathbf{a}
\end{equation}

\subsection*{3. Verallgemeinerte Weber-Kraft in vektorieller Form}
Einsetzen in (1) ergibt:
\begin{equation}
\mathbf{F}_{\text{WG}} = -\frac{GMm}{r^2}\left(1 - \frac{(\mathbf{r}\cdot\mathbf{v})^2}{c^2r^2} + \beta\frac{v^2 + \mathbf{r}\cdot\mathbf{a}}{c^2}\right)\hat{\mathbf{r}}
\end{equation}

\subsection*{4. Lagrange-Formulierung der Weber-Gravitation}
Das effektive Weber-Potential lautet:
\begin{equation}
V_{\text{WG}} = -\frac{GMm}{r}\left(1 - \frac{v^2}{2c^2} + \beta\frac{\mathbf{r}\cdot\mathbf{a}}{2c^2}\right)
\end{equation}

Die Lagrange-Funktion wird:
\begin{equation}
\mathscr{L}_{\text{WG}} = T - V_{\text{WG}} = \frac{1}{2}mv^2 + \frac{GMm}{r}\left(1 - \frac{v^2}{2c^2} + \beta\frac{\mathbf{r}\cdot\mathbf{a}}{2c^2}\right)
\end{equation}

\subsection*{5. Euler-Lagrange-Gleichungen}
Anwendung der Euler-Lagrange-Gleichung:
\begin{equation}
\frac{d}{dt}\left(\frac{\partial\mathscr{L}}{\partial\mathbf{v}}\right) - \frac{\partial\mathscr{L}}{\partial\mathbf{r}} = 0
\end{equation}

Berechnung der Terme:
\begin{align}
\frac{\partial\mathscr{L}}{\partial\mathbf{v}} &= m\mathbf{v} - \frac{GMm}{c^2r}\mathbf{v} + \beta\frac{GMm}{2c^2}\frac{\mathbf{r}}{r} \\
\frac{d}{dt}\left(\frac{\partial\mathscr{L}}{\partial\mathbf{v}}\right) &= m\mathbf{a} - \frac{GMm}{c^2}\left(\frac{\mathbf{a}}{r} - \frac{\dot{r}\mathbf{v}}{r^2}\right) + \beta\frac{GMm}{2c^2}\left(\frac{\mathbf{v}}{r} - \frac{\dot{r}\mathbf{r}}{r^2}\right) \\
\frac{\partial\mathscr{L}}{\partial\mathbf{r}} &= -\frac{GMm}{r^2}\hat{\mathbf{r}} + \frac{GMm}{2c^2}\left(\frac{v^2}{r^2}\hat{\mathbf{r}} - \beta\frac{\mathbf{a}}{r}\right)
\end{align}

\subsection*{6. De-Broglie-Bohm'sches Quantenpotential}
Das Quantenpotential der DBT ist:
\begin{equation}
Q = -\frac{\hbar^2}{2m}\frac{\nabla^2|\Psi|}{|\Psi|}
\end{equation}

Die quantenmechanische Kraft ergibt sich aus:
\begin{equation}
\mathbf{F}_{\text{Q}} = -\nabla Q
\end{equation}

\subsection*{7. Kombinierte Bewegungsgleichung}
Addition der Weber- und Quantenkräfte führt zu:
\begin{equation}
m\frac{d}{dt}\left[\left(1 - \frac{GM}{c^2r} + \beta\frac{GM}{2c^2}\frac{\mathbf{r}\cdot\mathbf{v}}{r^2}\right)\mathbf{v}\right] = -\frac{GMm}{r^2}\hat{\mathbf{r}} - \nabla Q
\end{equation}

Definition des Weber-Lorentz-Faktors:
\begin{equation}
\gamma_{\text{WG}} \equiv \left(1 - \frac{v^2}{c^2} + \beta\frac{\mathbf{v}\cdot\mathbf{a}}{c^2}\right)^{-1/2} \approx 1 + \frac{v^2}{2c^2} - \beta\frac{\mathbf{v}\cdot\mathbf{a}}{2c^2}
\end{equation}

\subsection*{8. Finale Bewegungsgleichung (\ref{eq:wg_dbt_srt})}
Nach Vernachlässigung höherer Ordnungen erhalten wir:
\begin{equation}
m\frac{d}{dt}(\gamma_{\text{WG}}\mathbf{v}) = -\nabla Q
\end{equation}

\section{Vergleich der Weber-Elektrodynamik mit der Maxwell-Theorie}

Wir betrachten zwei Punktladungen $q_1$ und $q_2$ mit konstanter Geschwindigkeit $\mathbf{v}_1 = \mathbf{v}_2 = \mathbf{v}$ (gleichförmige Bewegung) und Abstandsvektor $\mathbf{r} = \mathbf{r}_1 - \mathbf{r}_2$.

\subsection{Weber-Elektrodynamik}
Die verallgemeinerte Weber-Kraft für die Kraft auf $q_1$ durch $q_2$ lautet in vektorieller Form:

\subsubsection{Klassische Weber-Kraft (Variante a)}
\begin{equation}
F_W^{(a)} = \frac{q_1 q_2}{4 \pi \epsilon_0 r^2} \left(1 + \frac{v^2}{c^2} + \frac{\mathbf{r} \cdot \mathbf{a}}{c^2} - \frac{3 (\mathbf{r} \cdot \mathbf{v})^2}{2 c^2 r^2}\right)
\end{equation}

\subsubsection{Alternative Weber-Kraft (Variante b)}
\begin{equation}
F_W^{(b)} = \frac{q_1 q_2}{4 \pi \epsilon_0 r^2} \left(1 + \frac{2 v^2}{c^2} + \frac{2 \mathbf{r} \cdot \mathbf{a}}{c^2} - \frac{3 (\mathbf{r} \cdot \mathbf{v})^2}{c^2 r^2}\right)
\end{equation}

Für den Spezialfall paralleler Bewegung ($\mathbf{v} \parallel \mathbf{r}$) mit $\mathbf{a} = 0$ vereinfachen sich diese Ausdrücke zu:
\begin{align}
F_W^{(a)} &= \frac{q_1 q_2}{4 \pi \epsilon_0 r^2} \left(1 - \frac{v^2}{2 c^2}\right) \\
F_W^{(b)} &= \frac{q_1 q_2}{4 \pi \epsilon_0 r^2} \left(1 - \frac{v^2}{c^2}\right)
\end{align}

\subsection{Maxwell-Theorie (Lorentz-Kraft)}
In der Maxwell-Elektrodynamik ergibt sich die Kraft aus der Lorentz-Kraft auf $q_1$:
\begin{equation}
\mathbf{F}_M = q_1 (\mathbf{E}_2 + \mathbf{v}_1 \times \mathbf{B}_2)
\end{equation}

Für eine gleichförmig bewegte Ladung ($\mathbf{v} = \text{const.}$, $\mathbf{a} = 0$) parallel zu $\mathbf{r}$ erhält man:
\begin{equation}
\mathbf{E}_2 = \frac{q_2}{4 \pi \epsilon_0 r^2} \left(1 - \frac{v^2}{c^2}\right) \hat{r}, \quad \mathbf{B}_2 = 0
\end{equation}
Damit wird die Lorentz-Kraft:
\begin{equation}
\mathbf{F}_M = \frac{q_1 q_2}{4 \pi \epsilon_0 r^2} \left(1 - \frac{v^2}{c^2}\right) \hat{r}
\end{equation}

\subsection{Vergleich der Ergebnisse}
\begin{table}[h]
\centering
\begin{tabular}{lc}
\hline
\textbf{Theorie} & \textbf{Kraftformel ($\mathbf{v} \parallel \mathbf{r}$)} \\
\hline
Weber (Variante a) & $F_W^{(a)} = \dfrac{q_1 q_2}{4 \pi \epsilon_0 r^2} \left(1 - \dfrac{v^2}{2 c^2}\right)$ \\
Weber (Variante b) & $F_W^{(b)} = \dfrac{q_1 q_2}{4 \pi \epsilon_0 r^2} \left(1 - \dfrac{v^2}{c^2}\right)$ \\
Maxwell & $\mathbf{F}_M = \dfrac{q_1 q_2}{4 \pi \epsilon_0 r^2} \left(1 - \dfrac{v^2}{c^2}\right) \hat{r}$ \\
\hline
\end{tabular}
\caption{Vergleich der Weber- und Maxwell-Kräfte für parallele Bewegung}
\end{table}

\subsection{Interpretation}
\begin{itemize}
\item Die Weber-Kraft \textbf{(Variante b)} stimmt exakt mit der Maxwell-Theorie für gleichförmige Bewegung ($\mathbf{a} = 0$) überein.
\item Die Weber-Kraft \textbf{(Variante a)} weicht ab (Faktor $1/2$ beim $v^2/c^2$-Term).
\end{itemize}


\begin{thebibliography}{9}
\bibitem{einstein1915} 
Einstein, A. (1915). 
\textit{Die Feldgleichungen der Gravitation}. 
Sitzungsberichte der Preußischen Akademie der Wissenschaften, 
S. 844–847.

\bibitem{shapiro1964} 
Shapiro, I. I. (1964). 
\textit{Fourth Test of General Relativity}. 
Physical Review Letters, 13(26), 789–791.

\bibitem{rubin1970} 
Rubin, V. C., \& Ford, W. K. (1970). 
\textit{Rotation of the Andromeda Nebula from a Spectroscopic Survey of Emission Regions}. 
Astrophysical Journal, 159, 379–403.

\bibitem{weber1846} 
Weber, W. (1846). 
\textit{Elektrodynamische Maassbestimmungen}. 
Leipzig: Weidmannsche Buchhandlung.

\bibitem{bohm1952} 
Bohm, D. (1952). 
\textit{A Suggested Interpretation of the Quantum Theory in Terms of "Hidden" Variables}. 
Physical Review, 85(2), 166–193.

\bibitem{tisserand1894}
Tisserand, F. (1894). 
\textit{Traité de Mécanique Céleste, Tome IV}. 
Gauthier-Villars, Paris. 
(Kapitel 28: "Lois électrodynamiques de Weber appliquées à la gravitation")
\end{thebibliography}

\end{document}
