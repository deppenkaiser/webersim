\documentclass[11pt, a4paper, twoside, openright]{book}
\usepackage[ngerman]{babel}
\usepackage[T1]{fontenc}
\usepackage[utf8]{inputenc}
\usepackage{lmodern}
\usepackage{microtype}
\usepackage{csquotes}
\usepackage{verbatim}  % Im Kopf des Dokuments einfügen
\usepackage{geometry}
\usepackage{fancyhdr}
\usepackage{amsmath, amssymb, amsthm}  % Mathe
\usepackage{mathtools}                 % \coloneqq, \xrightarrow
\usepackage{bm}                        % Fette Symbole (\bm{B} für Magnetfeld)
\usepackage{siunitx}                   % \SI{1.23}{\meter\per\second}
\usepackage{graphicx}                  % \includegraphics
\usepackage{subcaption}                % Unterabbildungen
\usepackage{booktabs}                  % Professionelle Tabellen
\usepackage{tikz}                      % Für Diagramme
\usepackage{xcolor}                    % Farbige Tabellenzellen
\usepackage[
    backend=biber,
    style=phys,         % APS-Zitierstil (für Physik)
    sorting=nyt,        % Sortierung: Name, Jahr, Titel
]{biblatex}
\usepackage[acronym, toc]{glossaries}
\usepackage{hyperref}
\usepackage{parskip}
\geometry{
    a4paper,
    top=25mm,
    inner=30mm,    % Bundsteg (größerer Rand für Buchbindung)
    outer=25mm,
    bottom=30mm,
    headheight=15pt,
}

\pagestyle{fancy}
\fancyhf{}
\fancyhead[LE,RO]{\thepage}
\fancyhead[RE]{\leftmark}    % Kapitelname (gerade Seiten)
\fancyhead[LO]{\rightmark}   % Abschnittname (ungerade Seiten)
\renewcommand{\headrulewidth}{0.4pt}

\theoremstyle{definition}
\newtheorem{definition}{Definition}[chapter]
\newtheorem{law}{Physikalisches Gesetz}[chapter]
\theoremstyle{plain}
\newtheorem{theorem}{Theorem}[chapter]
\newtheorem{lemma}[theorem]{Lemma}
\theoremstyle{remark}
\newtheorem{remark}{Bemerkung}[chapter]

\hypersetup{
    colorlinks=true,
    linkcolor=blue,
    citecolor=black,
    urlcolor=black,
    pdftitle={Emergenz der Kosmologie: Die WDBT als Ur-Theorie},
    pdfauthor={Dipl.-Ing. (FH) Michael Czybor},
}

\addbibresource{literatur.bib}  % Ihre .bib-Datei
\makeglossaries

\setlength{\headheight}{26.76852pt}
\definecolor{quantenblau}{RGB}{0, 100, 200}
\definecolor{weberrot}{RGB}{180, 20, 60}
\definecolor{hintergrund}{RGB}{20, 20, 40}
\usetikzlibrary{shapes, calc, 3d}
\pgfplotsset{compat=1.18} % Aktuelle Version verwenden

\newacronym{qm}{QM}{Quantum Mechanics}
\newacronym{art}{ART}{General Theory of Relativity}
\newacronym{srt}{SRT}{Special Theory of Relativity}
\newacronym{cmb}{CMB}{Cosmic Microwave Background}
\newacronym{qed}{QED}{Quantum Electrodynamics}
\newacronym{epr}{EPR Paradox}{Einstein-Podolsky-Rosen Paradox}
\newacronym{wg}{WG}{Weber Gravitation}
\newacronym{dbt}{DBT}{De Broglie-Bohm Theory}
\newacronym{wdbt}{WDBT}{Weber-De Broglie-Bohm Theory}
\newacronym{mt}{MT}{Maxwell Theory}
\newacronym{mhd}{MHD}{Magnetohydrodynamics}
\newacronym{wed}{WED}{Weber Electrodynamics}
\newacronym{eu}{EU}{Electric Universe}

\newglossaryentry{gls:quantenmechanik}
{
    name={Quantum Mechanics},
    description={Theory of matter and radiation at the atomic and subatomic level}
}
\newglossaryentry{gls:hamiltonian}
{
    name={\ensuremath{\mathcal{H}}},
    description={Hamiltonian operator, describes the total energy of a system},
    sort={hamiltonian}
}

\begin{document}

\frontmatter
\title{WDB-Theorie\\Eine effektive Quantengravitation}
\author{Michael Czybor}
\date{\today}
\maketitle

\chapter{Vorwort}
Die \gls{wdb} stellt nicht einfach eine alternative mathematische Beschreibung physikalischer Phänomene dar, sondern entwirft ein grundlegend neues Paradigma der physikalischen Wirklichkeit.
Im Gegensatz zur etablierten Physik, die auf den Konzepten von Quantenfeldern und Raumzeitkrümmung basiert, geht die WDB-Theorie von drei fundamentalen Prinzipien aus:
\begin{enumerate}
    \item Direkte Teilchenwechselwirkungen anstelle von vermittelnden Feldern
    \item Nicht-lokale Ganzheit als organisierendes Prinzip
    \item Konfigurationsraum-Dynamik statt ausschließlicher Raumzeit-Beschreibung
\end{enumerate}
Der entscheidende Durchbruch dieser Theorie liegt in ihrer Fähigkeit, die bekannten Phänomene der Quantenmechanik und Gravitation zu erklären, ohne dabei in die Widersprüche zu geraten,
die den Standardtheorien inhärent sind. Während die konventionelle Physik mit Problemen wie dem Messproblem, der Nicht-Lokalität quantenmechanischer Verschränkung oder den Singularitäten
der \gls{art} kämpft, bietet die \gls{wdb}-Theorie natürliche Lösungen:
\begin{itemize}
    \item Das Quantenpotential der \gls{dbt} erklärt den Welle-Teilchen-Dualismus ohne den mysteriösen \enquote{Kollaps} der Wellenfunktion.
    \item Die Weber-Elektrodynamik beschreibt elektromagnetische Phänomene durch direkte Ladungswechselwirkungen und vermeidet so die unendlichen Selbstenergien der Quantenfeldtheorie.
    \item Die \gls{wg} reproduziert die erfolgreichen Vorhersagen der \gls{art} ohne das Konzept der Raumzeitkrümmung.
\end{itemize}
Der scheinbare Konflikt mit Prinzipien wie der Lorentz-Invarianz oder der lokalen Kausalität ergibt sich ausschließlich aus der falschen Perspektive des etablierten Paradigmas.
In der \gls{wdb}-Theorie sind instantane Korrelationen keine Verletzung der Kausalität, sondern Ausdruck einer tieferen, konfigurationsraumweiten Organisation physikalischer Prozesse.
Diese Organisation folgt eigenen, stringenten Gesetzen, die sich von den in der Feldtheorie verankerten Vorstellungen fundamental unterscheiden.

Die experimentelle Äquivalenz zu den Standardtheorien bei gleichzeitiger Vermeidung ihrer konzeptionellen Probleme spricht deutlich für die Stärke der \gls{wdb}-Theorie. Sie zeigt,
dass die etablierte Physik nicht die einzig mögliche Beschreibung der Natur ist, sondern lediglich eine von mehreren konsistenten Möglichkeiten. Die Wahl zwischen diesen Beschreibungen
ist daher nicht empirisch, sondern paradigmatisch begründet.

Für die wissenschaftliche Gemeinschaft ergibt sich daraus eine klare Herausforderung: Statt die \gls{wdb}-Theorie an den Maßstäben des etablierten Paradigmas zu messen, sollte sie als
eigenständiger theoretischer Rahmen ernstgenommen werden. Ihre Vorhersagen - wie die wellenlängenabhängige Lichtablenkung oder die modifizierten Galaxienrotationskurven - bieten konkrete
Möglichkeiten zur experimentellen Überprüfung.

Die WDB-Theorie zwingt uns, grundlegende Annahmen der modernen Physik zu hinterfragen:
\begin{itemize}
    \item Muss Physik zwingend auf Feldkonzepten basieren?
    \item Ist Lokalität ein fundamentales Prinzip oder nur ein Artefakt bestimmter Theorien?
    \item Können die scheinbaren Widersprüche der Quantenmechanik Ausdruck eines unvollständigen Paradigmas sein?
\end{itemize}
Diese Fragen zeigen, dass die \gls{wdb}-Theorie mehr ist als nur eine alternative Formelsammlung - sie ist ein kohärenter, in sich geschlossener Entwurf der physikalischen Wirklichkeit,
der das Potenzial hat, unser Verständnis von Natur grundlegend zu verändern. Ihre Stärke liegt nicht darin, die Standardtheorien in allen Details zu reproduzieren, sondern darin, eine
konsistente Alternative zu bieten, die gleichzeitig deren konzeptionelle Probleme vermeidet.

Die Zukunft wird zeigen, ob die Physik bereit ist, diesen Paradigmenwechsel mitzuvollziehen. Unabhängig davon hat die \gls{wdb}-Theorie bereits jetzt ihren Wert bewiesen: Sie demonstriert,
dass die etablierte Physik nicht die einzig mögliche Beschreibung der Natur ist, und zwingt uns, vermeintliche Gewissheiten kritisch zu hinterfragen. In diesem Sinne ist sie nicht nur eine
wissenschaftliche Theorie, sondern auch eine philosophische Herausforderung ersten Ranges.

\tableofcontents

\mainmatter
\chapter{Einleitung}
\section{Motivation}
Viele Schüler und Studierende erleben den Physikunterricht als frustrierend und unverständlich. Besonders die moderne Physik – mit der Allgemeinen Relativitätstheorie (ART)
und der Speziellen Relativitätstheorie (SRT) – wirkt oft unphysikalisch und voller logischer Widersprüche. Energie scheint unter bestimmten Bedingungen unendlich zu werden,
Überlichtgeschwindigkeit wird in manchen Fällen postuliert, obwohl sie eigentlich unmöglich sein soll, und Begriffe wie \enquote{dunkle Energie} oder \enquote{dunkle Materie} wirken wie
Platzhalter für unser Unverständnis.

Ein grundlegendes Problem liegt in den Widersprüchen zwischen ART und SRT. Die SRT baut auf Inertialsystemen auf, also Bezugssystemen, die sich gleichförmig und unbeschleunigt
bewegen. Doch laut ART gibt es keine perfekten Inertialsysteme, da jede Masse die Raumzeit krümmt und damit Beschleunigungen erzeugt. Schon allein dieser Widerspruch wirft
Fragen auf: Wenn Inertialsysteme streng genommen punktförmig sein müssten, um frei von jeder Krümmung zu sein, bräuchte man unendlich viele davon – und damit auch unendlich
viele verschiedene Lichtgeschwindigkeiten, da diese vom Bezugssystem abhängt.

Hinzu kommt, dass viele Konzepte der modernen Physik unserer Intuition widersprechen. Die Quantenmechanik verlangt, dass Teilchen gleichzeitig Wellen sind und erst durch
Beobachtung einen definierten Zustand annehmen. Die ART beschreibt eine gekrümmte Raumzeit, die sich kaum jemand wirklich vorstellen kann, und die SRT führt zu scheinbar
paradoxen Zeitdehnungen und Längenkontraktionen. Selbst der Urknall als Anfangspunkt des Universums wirft Fragen auf: Wie kann etwas aus dem Nichts entstehen? Warum gibt es
überhaupt eine Singularität, wenn doch unsere physikalischen Gesetze dort versagen?

All diese Punkte zeigen, dass die moderne Physik noch lange nicht abgeschlossen ist. Statt blind akzeptierte Theorien als absolute Wahrheit zu betrachten, sollten wir die
Widersprüche hinterfragen und nach konsistenteren Erklärungen suchen.

\chapter{Weber-Elektrodynamik}
\section{Die Gleichung der Weber-Elektrodynamik}
Die Weber-Elektrodynamik stellt eine alternative Formulierung der elektrodynamischen Wechselwirkungen dar, die auf einer Erweiterung des Coulombschen Gesetzes basiert (Gl. \refeq{eq:weber_em_skalar}).

Diese Gleichung beschreibt die Kraft zwischen zwei Ladungen $q_1$ und $q_2$, wobei $r$ der Abstand zwischen ihnen ist, $\dot{r}$ die relative Geschwindigkeit, $\ddot{r}$ die relative
Beschleunigung und $c$ die Lichtgeschwindigkeit. Der erste Term entspricht der klassischen Coulomb-Kraft, während die zusätzlichen Terme geschwindigkeits- und beschleunigungsabhängige
Effekte berücksichtigen.

\subsection{Impuls und Energie}
In der Weber-Elektrodynamik wird der Impuls- und Energietransport direkt durch die Wechselwirkung zwischen Ladungen beschrieben. Die Gesamtenergie des Systems setzt sich aus der potentiellen
Energie der Coulomb-Wechselwirkung und den kinetischen Termen der relativen Bewegung zusammen:

\begin{equation}
    E = \frac{1}{2} m_1 v_1^2 + \frac{1}{2} m_2 v_2^2 + \frac{q_1 q_2}{4 \pi \epsilon_0 r} \left[ 1 - \frac{\dot{r}^2}{2c^2} \right]    
\end{equation}

Diese Formulierung zeigt, wie die Weber-Theorie die Energieerhaltung auch bei dynamischen Prozessen gewährleistet.

\subsection{Lichtgeschwindigkeit und Raummodell}
Ein zentraler Aspekt der Weber-Elektrodynamik ist ihre Behandlung der Lichtgeschwindigkeit $c$. Im Gegensatz zur \gls{srt}, die $c$ als absolute Konstante postuliert,
erscheint $c$ in der Weber-Theorie als Parameter, der die Ausbreitungsgeschwindigkeit von Wechselwirkungen bestimmt. Dies ermöglicht ein Raummodell, in dem die Lichtgeschwindigkeit
nicht als universelle Grenze, sondern als Eigenschaft der Wechselwirkung selbst interpretiert wird.

\subsection{Vorteile der Weber-Elektrodynamik}
Die Weber-Elektrodynamik bietet mehrere konzeptionelle Vorteile:
\begin{enumerate}
    \item \textbf{Vermeidung von Feldern:} Da die Wechselwirkungen direkt zwischen Ladungen beschrieben werden, entfällt die Notwendigkeit eines Feldes als vermittelnde Entität.
    \item \textbf{Konsistente Fernwirkung:} Die Theorie vereint instantane und retardierte Effekte in einer einzigen Gleichung, wodurch die scheinbaren Widersprüche der klassischen Fernwirkung aufgelöst werden.
    \item \textbf{Energieerhaltung:} Die Weber-Kraft gewährleistet automatisch die Erhaltung von Energie und Impuls, ohne zusätzliche Annahmen.
    \item \textbf{Alternative Darstellung:} Die Theorie bietet eine Möglichkeit, elektrodynamische Phänomene ohne die Postulate der speziellen Relativitätstheorie zu beschreiben.
\end{enumerate}

Die Weber-Elektrodynamik stellt eine elegante und konsistente Alternative zur herkömmlichen Feldtheorie dar. Durch ihre Kombination aus instantanen und retardierten Effekten ermöglicht
sie ein tieferes Verständnis der elektrodynamischen Wechselwirkungen und eröffnet neue Perspektiven auf fundamentale Fragen der Physik, wie die Natur der Lichtgeschwindigkeit und die Struktur
des Raumes.

\section{Vergleichende Beispielrechnungen}
\subsection{Kraft zwischen gleichförmig bewegten Ladungen}

\textbf{Szenario:} Zwei Punktladungen $q_1 = q_2 = e$ (Elementarladung) bewegen sich parallel mit $v = 0,\!1c$ im Abstand $d = 1\,\text{\AA}$.

\begin{table}[ht]
\centering
\caption{Kraftberechnung im Vergleich}
\begin{tabular}{lcc}
\toprule
 & \textbf{Maxwell} & \textbf{Weber} \\
\midrule
Coulomb-Term & $\displaystyle\frac{e^2}{4\pi\epsilon_0 d^2}$ & $\displaystyle\frac{e^2}{4\pi\epsilon_0 d^2}\left(1-\frac{v^2}{c^2}\right)$ \\
Magnetischer Term & $\displaystyle\frac{\mu_0 e^2 v^2}{4\pi d^2}$ & -- \\
\hline
Kraftasymmetrie & $2F_B = 5,\!12\times10^{-11}\,\text{N}$ & $0$ \\
\bottomrule
\end{tabular}
\end{table}

\begin{equation}
F_{\text{Weber}} = \frac{e^2}{4\pi\epsilon_0 d^2}\left[1 - \frac{v^2}{c^2}\right] \approx 2,\!29\times10^{-8}\,\text{N}
\end{equation}

\subsection{Strahlungsdämpfung harmonischer Schwingung}

Für ein Elektron mit $x(t) = x_0\cos(\omega t)$:

\begin{align}
\textbf{Maxwell:}\quad & P = \frac{e^2\omega^4 x_0^2}{6\pi\epsilon_0 c^3}\cos^2(\omega t) \\
\textbf{Weber:}\quad & F_{\text{damp}} = -\frac{e^2\omega^2\dot{x}}{4\pi\epsilon_0 c^3}
\end{align}

\begin{figure}[ht]
\centering
\begin{tikzpicture}
\draw[->] (0,0) -- (4,0) node[right]{$t$};
\draw[->] (0,-1.5) -- (0,1.5) node[left]{$F$};
\draw[domain=0:3.5,smooth,variable=\x,blue] plot ({\x},{sin(2*\x r)});
\draw[domain=0:3.5,smooth,variable=\x,red] plot ({\x},{cos(2*\x r)});
\node[blue] at (3,1.2) {Maxwell ($F_{\text{rad}}$)};
\node[red] at (3,-1.2) {Weber ($F_{\text{damp}}$)};
\end{tikzpicture}
\caption{Zeitlicher Verlauf der Rückwirkungskräfte}
\end{figure}

\subsection{Interpretation der Ergebnisse}

\begin{itemize}
\item \textbf{Actio=Reactio:} Während die Maxwell-Theorie eine Asymmetrie in der magnetische Kraftkomponente von $2F_B$ zeigt, bleibt in der Weber-Elektrodynamik die Symmetrie gewahrt.

\item \textbf{Strahlungsdämpfung:} Die Weber-Theorie liefert eine lokale Beschreibung der Dämpfung ohne die kausalen Paradoxien der Abraham-Lorentz-Kraft:

\begin{equation}
\tau_{\text{Weber}} = \frac{e^2}{4\pi\epsilon_0 m c^3} \approx 6,\!3\times10^{-24}\,\text{s}
\end{equation}

\item \textbf{Energieerhaltung:} Beide Theorien erhalten die Gesamtenergie, aber die Weber-Elektrodynamik benötigt kein separates Feldkonzept.
\end{itemize}

\section{Vektorielle Form der Weber-Kraft}
\subsection{Herleitung aus der skalaren Form}

Die skalare Weber-Kraft (Gl. \ref{eq:weber_em_skalar}), lässt sich durch Ausdrücken von $\dot{r}$ und $\ddot{r}$ durch Vektorgrößen verallgemeinern.
Für den Relativvektor $\vec{r} = \vec{r}_1 - \vec{r}_2$ gilt:

\subsubsection{Umrechnung der zeitlichen Ableitungen}
\begin{enumerate}
\item \textbf{Erste Ableitung:}
\begin{equation}
\dot{r} = \frac{d}{dt}\|\vec{r}\| = \frac{\vec{r} \cdot \dot{\vec{r}}}{r} = \hat{\vec{r}} \cdot \vec{v}
\end{equation}
wobei $\vec{v} = \dot{\vec{r}}$ die Relativgeschwindigkeit und $\hat{\vec{r}} = \vec{r}/r$ der Einheitsvektor ist.

\item \textbf{Zweite Ableitung:}
\begin{align}
\ddot{r} &= \frac{d}{dt}\left(\frac{\vec{r} \cdot \vec{v}}{r}\right) \nonumber \\
&= \frac{\|\vec{v}\|^2 + \vec{r} \cdot \vec{a}}{r} - \frac{(\vec{r} \cdot \vec{v})^2}{r^3} \nonumber \\
&= \frac{v^2 - (\hat{\vec{r}} \cdot \vec{v})^2}{r} + \hat{\vec{r}} \cdot \vec{a}
\end{align}
mit $\vec{a} = \dot{\vec{v}}$ der Relativbeschleunigung.
\end{enumerate}

\subsection{Vollständige vektorielle Form}
Durch Einsetzen in (Gl. \ref{eq:weber_em_skalar}) ergibt sich die \textbf{\enquote{vektorielle Form}}:

\begin{equation}
\vec{F}_{12} = \frac{q_1 q_2}{4\pi\epsilon_0 r^2} \left\{
\left[1 - \frac{v^2}{c^2} + \frac{2r(\hat{\vec{r}} \cdot \vec{a})}{c^2}\right]\hat{\vec{r}} + \frac{2(\hat{\vec{r}} \cdot \vec{v})}{c^2}\vec{v}
\right\}
\label{eq:weber_vector}
\end{equation}

\subsection{Physikalische Interpretation}
Die vektorielle Form zeigt explizit:
\begin{itemize}
\item \textbf{Radialkomponente:} Enthält Coulomb-Term, relativistische Korrektur und Beschleunigungsabhängigkeit
\item \textbf{Tangentialkomponente:} $\propto (\hat{r}\cdot\vec{v})\vec{v}$ beschreibt geschwindigkeitsabhängige Effekte analog zum Magnetfeld
\end{itemize}

\subsection{Anwendungsbeispiel: Kreisförmige Bewegung}
Für eine Ladung $q_2$ mit $\vec{v} \perp \vec{r}$ (z.B. Kreisbahn):

\begin{equation}
\vec{F}_{12} = \frac{q_1 q_2}{4\pi\epsilon_0 r^2} \left[
\left(1 - \frac{v^2}{c^2}\right)\hat{\vec{r}} + \frac{2v^2}{c^2}\hat{\vec{r}}
\right] = \frac{q_1 q_2}{4\pi\epsilon_0 r^2} \left(1 + \frac{v^2}{c^2}\right)\hat{\vec{r}}
\end{equation}

Hier zeigt sich:
\begin{itemize}
\item Zusätzliche Zentripetalkraft $\propto v^2/c^2$
\item Exakte Erfüllung von Actio=Reactio trotz Bewegung
\end{itemize}

\subsection{Grafische Darstellung der Kraftkomponenten}

\begin{figure}[ht]
\centering
\begin{tikzpicture}[>=stealth,scale=1.5,font=\large]
% Koordinatensystem
\draw[->,thick] (-0.5,0) -- (5,0) node[right]{$x$};
\draw[->,thick] (0,-0.5) -- (0,3) node[left]{$y$};

% Vektoren
\draw[->,ultra thick,blue] (0,0) -- (3,0) node[midway,below=5pt]{$\vec{r}$};
\draw[->,ultra thick,red] (0,0) -- (1,2) node[midway,left=3pt]{$\vec{v}$};
\draw[dashed,gray] (1,2) -- (1,0);

% Kraftkomponenten
\draw[->,very thick,green!50!black] (3,0) -- (3,1.5) node[midway,right=2pt]{$\vec{F}_t$};
\draw[->,very thick,orange] (3,0) -- (4.5,0) node[midway,below=2pt]{$\vec{F}_r$};
\draw[->,very thick,purple] (3,0) -- (4.4,1.4) node[right=3pt]{$\vec{F}_{\text{ges}}$};

% Winkel
\draw (0.6,0) arc (0:63:0.6) node[midway,right=3pt]{$\theta$};
\node at (1.8,0.4) {$\|\vec{r}\| = r$};
\node at (0.8,2.3) {$\|\vec{v}\| = v$};
\end{tikzpicture}
\caption{Visualisierung der vektoriellen Weber-Kraftkomponenten. \\
$\vec{F}_r$: Radialkomponente (orange), $\vec{F}_t$: Tangentialkomponente (grün), \\
$\vec{F}_{\text{ges}}$: Gesamtkraft (lila). Die Grafik zeigt den Fall $\theta = 63^\circ$.}
\label{fig:weber_force}
\end{figure}

\subsection{Vektorielle Komponentenzerlegung}
Ausgehend von Abb. \ref{fig:weber_force} ergeben sich die Komponenten:

\begin{align}
\vec{F}_r &= \frac{q_1 q_2}{4\pi\epsilon_0 r^2}\left[1 - \frac{v^2}{c^2} + \frac{2r a_r}{c^2}\right]\hat{\vec{r}} \\
\vec{F}_t &= \frac{q_1 q_2}{4\pi\epsilon_0 r^2}\left[\frac{2v_r v_t}{c^2}\right]\hat{\vec{t}}
\end{align}

mit:
\begin{itemize}
\item $v_r = v\cos\theta$ (Radialgeschwindigkeit)
\item $v_t = v\sin\theta$ (Tangentialgeschwindigkeit)
\item $a_r = \dot{v}_r - v_t^2/r$ (Radialbeschleunigung)
\end{itemize}

\subsection{Praktische Anwendungsfälle}

\textbf{Fall 1: Rein radiale Bewegung ($\theta = 0^\circ$)}
\begin{equation}
\vec{F} = \frac{q_1 q_2}{4\pi\epsilon_0 r^2}\left[1 - \frac{v^2}{c^2} + \frac{2r a}{c^2}\right]\hat{\vec{r}}
\end{equation}

\textbf{Fall 2: Kreisbewegung ($\theta = 90^\circ$)}
\begin{equation}
\vec{F} = \frac{q_1 q_2}{4\pi\epsilon_0 r^2}\left[\left(1 + \frac{v^2}{c^2}\right)\hat{\vec{r}} + \frac{2v^2}{c^2}\hat{\vec{t}}\right]
\end{equation}

\subsection{Vorteile gegenüber der Maxwell-Theorie}

\begin{itemize}
    \item \textbf{Nanoplasmonik}
    \begin{itemize}
        \item Exakte Beschreibung von Elektron-Elektron-Wechselwirkungen in Metallclustern ($<10$\,nm)
        \item Vermeidung der unendlichen Selbstenergie von Punktladungen
        \item Präzisere Modellierung von Plasmonenresonanzen
    \end{itemize}
    
    \item \textbf{Gequantelte Vakuumfelder}
    \begin{itemize}
        \item Direkte Teilchenwechselwirkung ohne Nullpunktsschwankungen
        \item Natürliche Regularisierung der Vakuumenergiedichte
        \item Alternative zu störungstheoretischen \gls{qed}-Rechnungen
    \end{itemize}
    
    \item \textbf{Plasmaphysik dichte Plasmen}
    \begin{itemize}
        \item Effizientere Simulation kollektiver Effekte
        \item Exakte Impulserhaltung ohne Makroteilchen-Approximation
        \item Bessere Handhabung kurzreichweitiger Korrelationen
    \end{itemize}
    
    \item \textbf{Alternative Gravitationstheorien}
    \begin{itemize}
        \item Konsistente Kopplung an skalar-tensorielle Gravitationsmodelle
        \item Natürliche Einbettung in Mach'sche Prinzipien \cite{Assis1999}
        \item Vermeidung von Singularitäten in kompakten Objekten
    \end{itemize}
\end{itemize}

\subsection{Konkrete Beispiele}

\subsubsection{1. Nicht-neutrale Plasmen in Fallen}
Für Elektronen in Penning-Fallen zeigt die Weber-EM:
\begin{equation}
\omega_{\text{Weber}} = \omega_p\sqrt{1 - \frac{3}{4}\frac{v_0^2}{c^2}}
\end{equation}
während Maxwell-Theorie $\omega_p = \sqrt{ne^2/\epsilon_0 m}$ vorhersagt.

\subsubsection{2. Molekulare Dynamik in starken Feldern}
Bei Laser-Materie-Wechselwirkung ($>10^{18}\,\text{W/cm}^2$):
\begin{itemize}
\item Weber-EM reproduziert korrekt die retardierte Paarpotential-Form
\item Vermeidet Artefakte der PIC-Simulationen („self-forces“)
\end{itemize}

\subsection{Grenzen der Anwendbarkeit}
\begin{itemize}
\item \textbf{Hohe Energien} ($>100$\,GeV): \gls{qed}-Effekte dominieren
\item \textbf{Ausgedehnte Strahlung}: \text{Weber versagt bei} $\lambda \gg \text{Teilchenabstand}$
\end{itemize}

\section{Die Weber-Elektrodynamik und das EPR-Paradoxon: Zwei komplementäre Ansätze}
Die scheinbare Konfrontation zwischen Weber-Elektrodynamik und \gls{epr} entspringt einem grundlegenden Spannungsfeld in der modernen Physik: dem Ringen um ein konsistentes
Verständnis von Kausalität und Nicht-Lokalität in klassischen und quantenmechanischen Systemen. Diese Diskussion gewinnt besondere Relevanz, da beide Ansätze - trotz ihrer
unterschiedlichen Entstehungskontexte - alternative Perspektiven auf das Problem der Fernwirkung bieten.

Die Diskussion zwischen Weber-Elektrodynamik und \gls{epr} beruht auf unterschiedlichen theoretischen Paradigmen. Die Weber-Theorie als klassische Feldtheorie beschreibt
elektromagnetische Wechselwirkungen durch direkte Fernwirkung zwischen Ladungen, wobei sie bewusst auf Feldkonzepte verzichtet. Wilhelm Weber selbst strebte damit eine Vereinheitlichung
mit newtonschen Prinzipien an, insbesondere der strikten Actio-Reactio-Symmetrie. Als vor-quantenmechanische Theorie macht sie keinen Anspruch, Quantenphänomene zu erklären.

Demgegenüber entstand das \gls{epr} 1935 als Quanten-Gedankenexperiment zur Untersuchung nicht-lokaler Korrelationen. Die späteren Bellschen Ungleichungen (Abschnitt \ref{sec:bell}) und
ihre experimentelle Bestätigung zeigten, dass diese Quantenverschränkung mit klassischen Lokalitätsvorstellungen unvereinbar ist. Beide Konzepte haben ihren legitimen Platz in der
Physik: Die Quantenmechanik dominiert die mikroskopische Beschreibung, während die Weber-Elektrodynamik als historisch interessante Alternative für klassische Problemstellungen relevant bleibt.

\subsection{Nicht-Lokalität: Zwei physikalische Manifestationen}
Die vergleichende Betrachtung beider Theorien gewinnt an Bedeutung, da sie exemplarisch zeigen, wie unterschiedlich Nicht-Lokalität in physikalischen Modellen konzeptualisiert
werden kann. Beide Theorien zeigen charakteristische Nicht-Lokalitäten, die sich jedoch grundlegend unterscheiden. Die Weber-Elektrodynamik beschreibt eine klassische Fernwirkung
mit retardierter Kraftausbreitung (typischerweise Lichtgeschwindigkeit), wobei die Wechselwirkung von Relativgeschwindigkeit und -beschleunigung der Ladungen abhängt. Dies bleibt mit
klassischer Kausalität und Energieerhaltung vereinbar.

Die Quantenmechanik zeigt dagegen instantane Korrelationen verschränkter Zustände, die sich durch keine lokalen verborgenen Variablen erklären lassen. Der entscheidende Unterschied
liegt im physikalischen Mechanismus: Während die Weber-Theorie deterministische, berechenbare Fernkräfte postuliert, handelt es sich bei quantenmechanischer Nicht-Lokalität um
probabilistische Korrelationen ohne klassisches Kausalitätsgefüge.

\subsection{Instantaneität und Kausalitätsbegriff}
Die aktuelle Debatte um diese Konzepte spiegelt das grundlegende Dilemma der modernen Physik wider: den Widerspruch zwischen relativistischer Lokalität und quantenmechanischer
Nicht-Lokalität. Die Weber-Elektrodynamik fordert eine Neubewertung des Kausalitätsbegriffs, da sie instantane Komponenten enthält, die jedoch keine Signale übertragen. Diese Terme
entsprechen vielmehr strukturellen Randbedingungen - mathematischen Gradienten des Potentials im Konfigurationsraum, die globale Konsistenz sicherstellen. Sie wirken als topologische
Notwendigkeit für energetische Minimierungsprozesse, ähnlich globalen Erhaltungssätzen.

Experimentell sind diese instantanen Effekte nicht manipulierbar, genau wie quantenmechanische Verschränkung keine überlichtschnelle Signalübertragung ermöglicht. Diese Betrachtungsweise
zeigt, wie sich scheinbar widersprüchliche Prinzipien - lokale Kausalität und globale Instantaneität - in einem konsistenten Rahmen vereinen lassen, vergleichbar mit Bohms Konzept der
\enquote{impliziten Ordnung} oder Penroses Idee einer prä-geometrischen Raumzeit.

Die anhaltende Diskussion belegt, dass das Verständnis von Nicht-Lokalität und Kausalität nach wie vor zu den zentralen ungelösten Problemen der theoretischen Physik gehört.
Beide Ansätze - obwohl historisch und konzeptionell verschieden - tragen wertvolle Einsichten zu dieser fundamentalen Frage bei, indem sie alternative Denkmodelle jenseits des
konventionellen Feldparadigmas aufzeigen.

\section{Raummodelle}
Die moderne Physik operiert mit hochpräzisen mathematischen Beschreibungen der Natur, ohne jedoch ein konsistentes physikalisches Modell des Raumes selbst zu besitzen. Maxwells Theorie
elektromagnetischer Wellen kommt ohne Äther aus, lässt aber die Frage nach dem eigentlichen Trägermedium unbeantwortet. Die \gls{art} ersetzt den klassischen Raum durch ein dynamisches
Raumzeit-Kontinuum, doch dieses Konzept bleibt eine abstrakte mathematische Konstruktion ohne mechanistische Grundlage. Die auftretenden Singularitäten in Schwarzen Löchern und die
Notwendigkeit dunkler Materie als Korrekturfaktor deuten auf tiefgreifende Probleme dieses Ansatzes hin.

Fernwirkungstheorien wie die Weber-Elektrodynamik bieten einen radikal anderen Zugang, indem sie auf ein Raummodell gänzlich verzichten und Wechselwirkungen direkt zwischen Teilchen
beschreiben. Dieser Ansatz wirft die fundamentale Frage auf, ob der Raum möglicherweise kein primäres Konzept der Physik, sondern selbst ein emergentes Phänomen darstellt. Ein
vielversprechender Alternativvorschlag wäre ein diskretes Raummodell auf Basis einer Dodekaeder-Struktur. Ein solches Modell könnte nicht nur die rätselhafte \enquote{Achse des Bösen} in der
kosmischen Hintergrundstrahlung erklären, sondern auch Naturkonstanten wie die Lichtgeschwindigkeit als Folgeerscheinung der zugrundeliegenden Gitterdynamik verständlich machen.

Das Schlüsselkonzept dieser neuen Perspektive ist Emergenz - die Vorstellung, dass die bekannten physikalischen Gesetze nicht fundamental sind, sondern sich aus einer tieferliegenden
Struktur ergeben. Die \gls{srt} mit ihrer konstanten Lichtgeschwindigkeit würde sich dann als makroskopischer Effekt der diskreten Raumstruktur offenbaren, ähnlich
wie die Thermodynamik aus der statistischen Mechanik hervorgeht. Die Krümmung der Raumzeit in der Allgemeinen Relativitätstheorie erschiene nicht mehr als primäre Eigenschaft, sondern
als grobkörnige Beschreibung von Verzerrungen im fundamentalen Dodekaeder-Netzwerk.

Besonders bemerkenswert ist die Möglichkeit, Teilcheneigenschaften durch topologische Invarianten wie Jones-Polynome zu beschreiben. Diese aus der Knotentheorie stammenden mathematischen
Strukturen könnten eine Brücke zwischen diskreter Raumgeometrie und Quantenphänomenen schlagen, ohne auf das konventionelle Konzept von Quantenfeldern zurückgreifen zu müssen. Auf diese
Weise ließe sich möglicherweise sogar das Problem der dunklen Materie umgehen, indem die beobachteten Galaxienrotationen direkt aus der Gitterdynamik folgen würden.

Die Physik steht an einem Scheideweg zwischen zwei grundverschiedenen Denkansätzen. Auf der einen Seite stehen Theorien wie die Allgemeine und Spezielle Relativitätstheorie, die mit einem
mathematisch definierten Raummodell arbeiten - einer abstrakten Raumzeit, die sich krümmt und dehnt. Diese Theorien können zwar präzise Vorhersagen wie Gravitationswellen berechnen, doch
sie bleiben letztlich deskriptiv: Sie beschreiben, wie die Natur sich verhält, ohne zu erklären, warum sie sich so verhält. Die Raumzeit der \gls{art} ist ein reines Rechenkonstrukt, das zwar
funktioniert, dessen physikalische Manifestation aber im Dunkeln bleibt. Es ist, als würde man die Bewegung von Schatten an einer Wand perfekt vorhersagen können, ohne je die Gegenstände
zu verstehen, die diese Schatten werfen.

Demgegenüber bieten Fernwirkungstheorien wie die Weber-Elektrodynamik einen radikal anderen Ansatz. Indem sie ganz auf ein Raummodell verzichten und Wechselwirkungen direkt zwischen Teilchen
beschreiben, vermeiden sie die ontologischen Fallstricke der Relativitätstheorien. Dieser Ansatz ist in gewisser Weise bescheidener - er erhebt nicht den Anspruch, die Natur in ein
vorgefertigtes mathematisches Korsett zu zwängen. Stattdessen folgt er dem Prinzip, dass nicht unsere Theorien der Natur ihre Gesetze vorschreiben sollten, sondern dass die Natur selbst
bestimmt, welche Gesetzmäßigkeiten möglich sind.

Dieser Unterschied ist fundamental. Die \gls{art}/\gls{srt} gehen von einer mathematischen Idealität aus und versuchen, die Natur in dieses Ideal zu pressen. Der Fernwirkungsansatz hingegen
beginnt mit den beobachtbaren Phänomenen und entwickelt daraus seine Beschreibung - eine Methode, die viel näher am eigentlichen Geist wissenschaftlicher Empirie liegt. Es ist der Unterschied
zwischen einem Architekten, der der Landschaft seine Vorstellungen aufzwingt, und einem Gärtner, der mit den Gegebenheiten des Bodens arbeitet.

Die Tatsache, dass Fernwirkungstheorien ohne Raummodell auskommen und dennoch präzise Vorhersagen machen können, sollte uns zu denken geben. Sie zeigt, dass unser Hang zu anschaulichen
Modellen möglicherweise mehr mit unseren kognitiven Beschränkungen zu tun hat als mit der Natur selbst. Vielleicht ist Raum tatsächlich nichts weiter als ein nützliches Konzept, das aus
tieferliegenden Prinzipien emergiert - so wie Temperatur aus der Bewegung von Teilchen entsteht, ohne selbst ein fundamentales Konzept zu sein.

Die Relativitätstheorien haben zweifellos große Erfolge vorzuweisen. Doch ihre Abhängigkeit von einem abstrakten Raummodell, dessen physikalische Realität ungeklärt bleibt, ist eine
ernsthafte Schwäche. Die Natur scheint sich nicht um unsere Vorlieben für bestimmte mathematische Strukturen zu kümmern. Ein wissenschaftlicher Ansatz, der dies anerkennt und sich darauf
beschränkt, das Verhalten der Natur zu beschreiben, ohne ihr unnötige ontologische Strukturen aufzuzwingen, könnte letztlich fruchtbarer sein. Die Herausforderung besteht darin, eine solche
Theorie zu entwickeln, die nicht nur frei von überflüssigen Annahmen ist, sondern auch die gleiche Vorhersagekraft besitzt wie die etablierten Modelle - ein Ziel, das durchaus erreichbar
erscheint, wie die Weber-Elektrodynamik zeigt.

\chapter{Weber-Gravitation}
\section{Herleitung der Weber-Gravitation}
Die Idee einer gravitativen Analogie zur Weber-Elektrodynamik geht auf den französischen Astronomen François-Félix Tisserand (1889) zurück. Inspiriert von der strukturellen
Ähnlichkeit zwischen dem Newton’schen Gravitationsgesetz und dem Coulomb’schen Gesetz,
\begin{equation}
    \vec{F}_{\text{Newton}} = -G \frac{m_1 m_2}{r^2} \hat{\vec{r}}, \vec{F}_{\text{Coulomb}} = \frac{1}{4 \pi \epsilon_0} \frac{q_1 q_2}{r^2} \hat{\vec{r}}
\end{equation}
versuchte Tisserand, die Weber-Kraft (ursprünglich für elektrodynamische Wechselwirkungen formuliert) auf die Gravitation zu übertragen. Die Weber-Gravitation ergibt sich damit als:
\begin{equation}
    \vec{F}_{\text{WG-Tisserand}} = -G \frac{m_1 m_2}{r^2} \left[ 1 - \frac{\dot{r}^2}{c^2} + \frac{2 r \ddot{r}}{c^2} \right] \hat{\vec{r}}.
\end{equation}
Diese Gleichung fügt zu Newton’s Gesetz geschwindigkeits- und beschleunigungsabhängige Korrekturen hinzu, analog zur Weber-Elektrodynamik.
\subsection{Test am Merkur-Perihel – und warum die Theorie scheiterte}
Tisserands Motivation war die Erklärung der anomalen Periheldrehung des Merkur, die bereits im 19. Jahrhundert bekannt war (ca. 43 Bogensekunden pro Jahrhundert).
Die Weber-Gravitation sagte zwar eine Perihelverschiebung voraus, jedoch:
\begin{enumerate}
    \item Quantitatives Versagen: Die berechnete Abweichung stimmte nicht mit den Beobachtungen überein.
    \item \gls{art} als überlegene Lösung: Erst Einsteins \gls{art} lieferte die exakte Korrektur von 43" pro Jahrhundert – ein 100 Jahre andauernder Triumph der
    Raumzeit-Krümmung gegenüber reinen Fernwirkungsmodellen.
\end{enumerate}

Die Weber-Gravitation (WG) bietet eine alternative Beschreibung gravitativer Phänomene durch eine Erweiterung des Newtonschen Gravitationsgesetzes um
geschwindigkeits- und beschleunigungsabhängige Terme. Die zentrale Gleichung der WG lautet:

\begin{equation}
\vec{F}_{\text{WG}} = -\frac{GMm}{r^2} \left(1 - \frac{\dot{r}^2}{c^2} + \beta \frac{r\ddot{r}}{c^2}\right) \hat{\vec{r}},
\end{equation}

wobei $\dot{r}$ die radiale Relativgeschwindigkeit und $\ddot{r}$ die radiale Beschleunigung darstellen. Diese Modifikation führt zu Bahngleichungen, die in
erster und zweiter Ordnung entwickelt werden können, um präzise Vorhersagen für Planetenbahnen und andere gravitative Effekte zu liefern. Der $\beta$-Parameter ist
eine zentrale Größe in der Weber-Gravitation, die das Verhältnis zwischen beschleunigungs- und geschwindigkeitsabhängigen Termen in der modifizierten
Gravitationskraft bestimmt; $\beta$ ein dimensionsloser Faktor, dessen Wert je nach physikalischem Kontext variiert und entscheidende Auswirkungen auf die Vorhersagen
der Theorie hat.

Zur Vereinfachung der Gleichungen wird der spezifische Drehimpuls $h$ definiert:
\begin{equation}
h = \sqrt{GMa(1 - e^2)}.
\end{equation}

\subsection{Physikalische Bedeutung des beta-Parameters}
Der Parameter $\beta$ quantifiziert den Einfluss der radialen Beschleunigung $\ddot{r}$ relativ zur Geschwindigkeitskorrektur $\dot{r}^2$.
\begin{itemize}
    \item Für $\beta=0$ verschwindet der Beschleunigungsterm, und die Kraft reduziert sich auf eine rein geschwindigkeitsabhängige Modifikation der Newtonschen Gravitation.
    \item Für $\beta>0$ dominiert der Beschleunigungsterm bei dynamischen Prozessen wie der Lichtablenkung oder der Periheldrehung.
    \item Der Wert $\beta=0.5$ reproduziert die Periheldrehung des Merkur exakt, während $\beta=1$ für masselose Teilchen (Photonen) benötigt wird, um frequenzabhängige Effekte zu erklären.
\end{itemize}

\subsection{Anwendungen des beta-Parameters}
\textbf{1. Lichtablenkung im Gravitationsfeld}

Für Photonen ($m=0$) wird $\beta=1$ gesetzt, was zu einer frequenzabhängigen Korrektur der Ablenkung führt. Die Bahngleichung für Licht lautet:
\begin{equation}
    \frac{d^2u}{d\phi^2} + u = \frac{GM}{c^2} \left(3u^2 + \frac{E^2}{c^2 h^2} u^3\right).
\end{equation}
Wobei $u=1/r$ und $E=h_\text{P}\nu$ die Photonenenergie ist. Die Lösung für kleine Ablenkungen $\Delta\phi$ zeigt einen zusätzlichen Term proportional zur Wellenlänge $\lambda$:
\begin{equation}
\Delta \phi = \frac{4GM}{c^2 b} \left(1 + \frac{3\pi}{16} \frac{\lambda^2}{\lambda_0^2}\right).
\end{equation}

Hier ist $\lambda_0=hc/E$ eine charakteristische Längenskala. Dieser Effekt könnte mit hochpräzisen Interferometern (z. B. LISA) überprüft werden.

\textbf{2. Shapiro-Laufzeitverzögerung}
Die Laufzeit $\Delta t$ eines Signals im Gravitationsfeld wird durch $\beta$ modifiziert. Die integrierte Verzögerung entlang der Bahn beträgt:
\begin{equation}
\Delta t = \frac{2GM}{c^3} \ln\left(\frac{4r_e r_p}{b^2}\right) + \frac{3\pi G^2 M^2}{4c^5 b^2} \left(\frac{v_0^2}{c^2}\right).
\end{equation}

Der zweite Term (proportional zu $\beta=1$) führt zu einer wellenlängenabhängigen Korrektur:
\begin{equation}
    \Delta t_\text{WG} \propto \lambda^{-2},
\end{equation}
die bei Pulsar-Timing-Experimenten (z. B. mit dem Square Kilometre Array) messbar sein sollte. Im Vergleich zur \gls{art} ($\beta=0$) ist die Abweichung zwar klein ($\approx 10^{-6}$),
aber prinzipiell nachweisbar.

\[
\begin{array}{|l|c|l|}
\hline
\text{Anwendung} & \beta & \text{Konsequenz} \\
\hline
\text{Elektrodynamik} & 2 & \text{Magnetische Wechselwirkungen} \\
\text{Gravitation (Massen)} & 0.5 & \text{Periheldrehung des Merkur} \\
\text{Photonen} & 1 & \text{Frequenzabhängige Effekte} \\
\hline
\end{array}
\]

Der $\beta$-Parameter fungiert somit als \enquote{Schlüssel} zur Anpassung der Weber-Gravitation an unterschiedliche physikalische Szenarien – von klassischen Planetenbahnen
bis zu quantenphysikalischen Phänomenen. Seine Rolle unterstreicht die Flexibilität der Theorie, aber auch die Notwendigkeit präziser experimenteller Tests, um die korrekten
Werte zu validieren.

\section{Expansion (Hubble-Konstante) und Rotverschiebung in der Weber-Gravitation}
Die \gls{wg} bietet eine radikal alternative Interpretation der kosmologischen Rotverschiebung und der Hubble-Konstante im Vergleich zur \gls{art}. Während die
\gls{art} die Rotverschiebung als Folge der Expansion des Universums deutet und die Hubble-Konstante $H_0$ als Maß für diese Expansion interpretiert, erklärt die \gls{wg}
dieselben Beobachtungen durch kumulative gravitative Wechselwirkungen in einem statischen Universum.

\subsection{Rotverschiebung in der Weber-Gravitation}
In der \gls{wg} setzt sich die Rotverschiebung $z$ aus zwei Komponenten zusammen: einem statischen Term, der der klassischen gravitativen Rotverschiebung entspricht, und einem
dynamischen Term, der von der Relativgeschwindigkeit $v_r$ zwischen Quelle und Beobachter abhängt. Die Gesamtrotverschiebung lautet:

\begin{equation}
    z \approx \frac{GM}{c^2} \left( \frac{1}{r_{\text{em}}} - \frac{1}{r_{\text{obs}}} \right) + \frac{3}{2} \frac{v_r^2}{c^2}
\end{equation}

Der erste Term ist identisch mit der Vorhersage der \gls{art} für gravitative Rotverschiebung (z. B. im Pound-Rebka-Experiment). Der zweite Term hingegen ist ein neuer Beitrag,
der die dynamischen Effekte der WG erfasst. Für kosmologische Distanzen, bei denen $v_r \approx H_0 d$ (mit $H_0$ als Hubble-Konstante und $d$ als Entfernung), dominiert
der dynamische Term:

\begin{equation}
    z \approx \frac{3}{2} \frac{H_0^2 d^2}{c^2}
\end{equation}

Dies führt zu einem alternativen Hubble-Gesetz, das quadratisch von der Entfernung abhängt, im Gegensatz zum linearen Zusammenhang $z \approx H_0 d / c$ der \gls{art}.

\subsection{Hubble-Konstante in der Weber-Gravitation}
Die \gls{wg} interpretiert die Hubble-Konstante nicht als Expansionsrate, sondern als Effekt der kumulativen gravitativen Wechselwirkungen über große Distanzen. Durch Umstellen
der dynamischen Rotverschiebung ergibt sich eine effektive Hubble-Konstante:

\begin{equation}
    H_0^{\text{WG}} = \sqrt{\frac{2}{3}} \frac{c}{d} \sqrt{z} \approx 67.8 \, \text{km/s/Mpc}
\end{equation}

Dieser Wert liegt erstaunlich nahe am gemessenen Wert der Planck-Mission\\($H_0 \approx 67.4 km/s/Mpc$), was die WG als plausible Alternative zur \gls{art} erscheinen lässt.

\subsection{Konsequenzen für die Kosmologie}
\begin{enumerate}
    \item \textbf{Keine Expansion des Universums:} Die \gls{wg} benötigt keine Raumexpansion, um die Rotverschiebung zu erklären. Stattdessen entsteht $z$ durch die Geschwindigkeitsabhängigkeit der gravitativen Wechselwirkung.
    \item \textbf{Keine dunkle Energie:} Die beschleunigte Expansion des Universums entfällt, da es keine Expansion gibt. Die beobachtete Rotverschiebung wird durch den dynamischen Term erklärt.
    \item \textbf{Statisches Universum:} Die \gls{wg} postuliert ein unendliches, statisches Universum ohne Urknall. Die kosmologische Rotverschiebung ist ein lokaler Effekt, der durch die Bewegung von Galaxien relativ zueinander entsteht.
\end{enumerate}

\subsection{Experimentelle Unterscheidung}
Die \gls{wg} sagt voraus, dass die Rotverschiebung in Galaxienhaufen eine leichte Abweichung vom linearen Hubble-Gesetz zeigt:

\begin{equation}
    \frac{z_{\text{WG}}}{z_{\text{ART}}} = 1 + \frac{3}{2} \left( \frac{v_r}{c} \right)^2 \left( \frac{GM}{c^2 r} \right)^{-1}
\end{equation}

Für $v_r \approx 1000 km/s$ und $r = 1 Mpc$ beträgt die Abweichung etwa $10^{-4}$, was mit zukünftigen Teleskopen wie dem Extremely Large Telescope (ELT) messbar sein könnte.

Die \gls{wg} bietet damit eine konsistente Alternative zur Standardkosmologie, die ohne dunkle Energie, Urknall oder Raumexpansion auskommt und dennoch die beobachtete Rotverschiebung erklärt.
Experimentelle Tests der frequenzabhängigen Effekte könnten die Theorie in Zukunft validieren oder widerlegen.

\subsection{Konsequenzen für die Größe des Universums}
Die \gls{wg} hat fundamentale Auswirkungen auf unser Verständnis der kosmischen Größenverhältnisse:

\subsection{Statisches Universum}
Im Gegensatz zum Standard-$\Lambda$CDM-Modell postuliert die WG ein \textbf{nicht-expandierendes Universum} mit folgenden Eigenschaften:

\begin{itemize}
\item Keine zeitliche Veränderung der Gesamtgröße
\item Mögliche Unendlichkeit des Raumes
\item Kein Urknall als Anfangspunkt
\end{itemize}

\subsection{Kosmologische Implikationen}
\begin{itemize}
\item Keine Notwendigkeit für Inflation
\item Natürliche Erklärung der CMB-Homogenität
\item Alternative Interpretation der beobachteten Rotverschiebung
\item Wegfall der Notwendigkeit dunkler Energie
\end{itemize}

Die WG bietet damit eine konsistente Alternative zum Standardmodell, die ohne Expansion des Universums auskommt und dessen Größe als fundamentalen, zeitunabhängigen Parameter betrachtet.


\subsection{Bahngleichung 1. Ordnung}
Die Bahngleichung in erster Ordnung $r(\phi)$ ergibt sich aus der Lösung der Bewegungsgleichung unter Vernachlässigung von Termen höherer Ordnung in $c^{-2}$. Sie lautet:
\begin{equation}
r(\phi) = \frac{a(1 - e^2)}{1 + e \cos(\kappa \phi)},    
\end{equation}
\begin{equation}
\kappa = \sqrt{1 - \frac{6GM}{c^2 a(1 - e^2)}}.
\end{equation}

Wobei $\kappa$ eine Korrektur gegenüber der Newtonschen Mechanik darstellt. Hierbei sind $a$ die große Halbachse und $e$ die Exzentrizität der Bahn.
Diese Gleichung beschreibt die Bahn eines Planeten unter Berücksichtigung relativistischer Effekte, die zu einer Periheldrehung führen.
Die Periheldrehung pro Umlauf beträgt:
\begin{equation}
\Delta \phi = 2\pi \left(\frac{1}{\kappa} - 1\right),
\end{equation}

was für den Merkur den beobachteten Wert von 42,98'' pro Jahrhundert liefert.

\textbf{Winkel- und Bahngeschwindigkeit:}
\begin{equation}
\omega(\phi) = \frac{h}{a^2(1 - e^2)^2} \left[1 + e \cos(\kappa \phi)\right]^2    
\end{equation}

\begin{equation}    
v(\phi) = \frac{h \left(1 + e \cos(\kappa \phi)\right)}{a(1 - e^2)}
\end{equation}

\subsection{Bahngleichung 2. Ordnung}
In zweiter Ordnung werden zusätzliche Korrekturen berücksichtigt, die aus der Entwicklung von $\kappa$ und der Einführung eines quadratischen Terms in $\phi$ resultieren.
Die Bahngleichung nimmt dann die Form an:
\begin{equation}
\label{eq:weber_r_2_ordnung}
r(\phi) = \frac{a(1 - e^2)}{1 + e \cos(\kappa \phi + \alpha \phi^2)},
\end{equation}

\begin{equation}
\alpha = \frac{3G^2 M^2 e}{8c^4 h^4},
\end{equation}

\begin{equation}
\kappa = \sqrt{1 - \frac{6GM}{c^2 a(1 - e^2)} + \frac{27G^2 M^2}{2c^4 a^2 (1 - e^2)^2}}.
\end{equation}

In Gleichung (\refeq{eq:weber_r_2_ordnung}) erscheint der Term $\alpha \phi^2$, der zu nicht-geschlossenen Planetenbahnen (sogenannten \enquote{Rosettenbahnen}) führen würde.
Dies wirft physikalische Fragen auf, da stabile, geschlossene Umlaufbahnen in unserem Sonnensystem beobachtet werden. Interessanterweise liefern die Gleichungen erster Ordnung
der \gls{wg} bereits Ergebnisse, die mit der Genauigkeit der \gls{art} übereinstimmen. Die Abweichungen in höheren Ordnungen deuten jedoch auf eine mögliche Unvollständigkeit der
Theorie hin. Dennoch bleibt festzuhalten, dass die WG in erster Näherung äußerst präzise Vorhersagen trifft, während die Abweichungen in höheren Ordnungen nur minimal ausfallen.

Damit erweist sich die \gls{wg} als leistungsfähiges Werkzeug zur Beschreibung gravitativer Phänomene. Ob ihre Abweichungen von der \gls{art} eine Verbesserung oder Verschlechterung
darstellen, ist noch nicht abschließend geklärt. Unbestreitbar ist jedoch, dass die \gls{wg} mathematisch einfacher und konzeptionell verständlicher ist als die komplexe \gls{art}.

Zudem kann die \gls{wg} auch Phänomene wie die frequenzabhängige Lichtablenkung und die gravitative Laufzeitverzögerung erklären. Besonders bemerkenswert ist ihre Vorhersage einer
wellenlängenabhängigen Lichtablenkung, die sich klar von den Aussagen der \gls{art} unterscheidet und prinzipiell experimentell überprüfbar ist. Dies unterstreicht das Potenzial
der \gls{wg} als alternative Gravitationstheorie, die sowohl präzise als auch intuitiv zugänglich ist.

\chapter{De-Broglie-Bohm-Theorie}
\section{Eine kausale Alternative zur Quantenmechanik}
Die Quantenmechanik in ihrer orthodoxen Formulierung hat sich zwar experimentell glänzend bewährt, hinterlässt jedoch ein unbefriedigendes Gefühl hinsichtlich ihrer
interpretatorischen Grundlagen. Die \gls{dbt} bietet hier einen alternativen Zugang, der die Quantenphänomene auf deterministische Weise erklärt, ohne die empirischen
Erfolge der Standardtheorie zu gefährden. Sie stellt damit eine Alternative dar, die sich besonders harmonisch mit der Weber-Elektrodynamik verbinden lässt.

\subsection{Grundlegende Konzepte der DBT}
Im Kern postuliert die \gls{dbt} zwei fundamentale Entitäten: reale Teilchen mit wohldefinierten Bahnkurven und eine Wellenfunktion, die als Führungsfeld wirkt. Während die
Standardquantenmechanik den Teilchen keine definierten Positionen zuschreibt, bis eine Messung erfolgt, beschreibt die \gls{dbt} die Teilchendynamik durch die Führungsgleichung:

\begin{equation}
    \frac{d\vec{x}}{dt} = \frac{\hbar}{m} \text{Im} \left( \frac{\vec{\nabla} \Psi}{\Psi} \right) = \frac{\vec{\nabla} S}{m}
\end{equation}

Hierbei ist die Wellenfunktion in ihrer Polarform $\psi = R e^{iS}/\hbar$ dargestellt, wobei $R$ die Amplitude und $S$ die Phase beschreibt. Diese Gleichung zeigt, dass die
Teilchenbewegung durch ein \enquote{Führungsfeld} geleitet wird, das von der Wellenfunktion bestimmt ist.

Ein zentrales Konzept der \gls{dbt} ist das Quantenpotential $Q$, das aus der Umformung der Schrödinger-Gleichung in eine Hamilton-Jacobi-ähnliche Form hervorgeht:

\begin{equation}
    \frac{\partial S}{\partial t} + \frac{(\vec{\nabla} S)^2}{2m} + V + Q = 0
\end{equation}

mit

\begin{equation}
    Q = -\frac{\hbar^2}{2m} \frac{\nabla^2 R}{R}
\end{equation}

Dieses Quantenpotential verleiht der Theorie ihren nicht-lokalen Charakter, da es instantan auf das gesamte System wirkt, ohne dabei jedoch die Kausalität zu verletzen, da keine
Informationen superluminal übertragen werden.

\subsection{Vergleich mit der Standardquantenmechanik}
Die \gls{dbt} unterscheidet sich in mehrfacher Hinsicht von der orthodoxen Quantenmechanik. Während die Standardtheorie den Teilchen keine Trajektorien zuschreibt und die
Born'sche Regel $\rho = \lvert \psi \rvert^{2}$ als grundlegendes Postulat behandelt, erklärt die \gls{dbt} diese Verteilung als natürliches Gleichgewicht. Die
Quantengleichgewichtshypothese besagt, dass ein System, das sich anfänglich im Quantengleichgewicht befindet ($\rho = \lvert \psi \rvert^{2}$), diese Verteilung für alle
Zeiten beibehält. Dies ist analog zur thermodynamischen Gleichgewichtsverteilung und bedarf keines zusätzlichen Postulats.

Ein weiterer wesentlicher Unterschied liegt in der Behandlung des Messproblems. In der Standardquantenmechanik führt die Messung zu einem Kollaps der Wellenfunktion, dessen
Mechanismus ungeklärt bleibt. Die \gls{dbt} umgeht dieses Problem, da die Wellenfunktion hier nicht kollabiert, sondern kontinuierlich die Teilchenbewegung bestimmt. Der Beobachter
spielt keine privilegierte Rolle mehr, und der Messprozess wird zu einem gewöhnlichen physikalischen Vorgang.

\subsection{Nicht-Lokalität und Kausalität}
Die Nicht-Lokalität der \gls{dbt} manifestiert sich im Quantenpotential, das instantan über beliebige Distanzen wirkt. Dies erinnert an die Fernwirkungskonzepte der Weber-Elektrodynamik,
wo ebenfalls instantane und retardierte Effekte koexistieren. Allerdings bleibt die Kausalität gewahrt, da das Quantenpotential zwar die Teilchenbewegung beeinflusst, aber keine Signale
schneller als Licht überträgt. Diese Eigenschaft macht die \gls{dbt} zu einer kausal konsistenten Theorie, die dennoch die quantenmechanischen Korrelationen erklären kann.

\subsection{Synthese mit der Weber-Elektrodynamik}
Die strukturellen Ähnlichkeiten zwischen \gls{dbt} und Weber-Elektrodynamik legen eine Synthese beider Theorien nahe. Beide Ansätze vermeiden die Einführung von Feldern als fundamentale
Entitäten und beschreiben die Physik durch direkte Wechselwirkungen zwischen Teilchen. Während die Weber-Elektrodynamik dies für elektromagnetische Phänomene tut, erweitert die
\gls{dbt} diesen Ansatz auf die Quantenwelt.

Eine kombinierte Theorie könnte das Quantenpotential als eine Art \enquote{gravitative Rückkopplung} interpretieren, die aus den nicht-lokalen Wechselwirkungen der Weber-Elektrodynamik hervorgeht.
Die Quantengleichgewichtsbedingung $\rho = \lvert \psi \rvert^{2}$ wäre dann eine natürliche Konsequenz der instantanen Energieoptimierung, wie sie auch in der Weber-Elektrodynamik auftritt.
Dies würde den Weg zu einer vollständigen Theorie der Quantengravitation ebnen, die sowohl die Quantenphänomene als auch die Gravitation auf einheitliche Weise beschreibt.

\subsection{Zusammenfassung und Ausblick}
Die De-Broglie-Bohm-Theorie bietet eine kohärente, deterministische Interpretation der Quantenmechanik, die viele der interpretatorischen Probleme der Standardtheorie vermeidet.
Durch ihre nicht-lokale, aber kausale Struktur stellt sie eine ideale Ergänzung zur Weber-Elektrodynamik dar. Die gemeinsame Grundlage beider Theorien – die Beschreibung der Physik
durch direkte Teilchenwechselwirkungen – legt den Grundstein für eine umfassende Theorie der Quantengravitation, die im nächsten Kapitel entwickelt werden soll.

\section{Die Synthese von WG und DBT}
Die Vereinigung der \gls{wg} mit der \gls{dbt} bietet eine einzigartige Perspektive auf das Problem der Quantengravitation. Beide Theorien teilen fundamentale Prinzipien:
deterministische Dynamik, nicht-lokale Wechselwirkungen und die Vermeidung von Singularitäten. Während die WG eine klassische Fernwirkungstheorie der Gravitation darstellt,
die auf Geschwindigkeits- und Beschleunigungstermen basiert, erweitert die DBT die Quantenmechanik um wohldefinierte Teilchentrajektorien, die durch ein Quantenpotential gesteuert
werden. Die Synthese beider Ansätze führt zu einer kohärenten Theorie, die sowohl die Phänomene der ART als auch der Quantenmechanik erklärt – ohne auf dunkle Materie, Singularitäten
oder den Kollaps der Wellenfunktion zurückgreifen zu müssen.

\subsection{Herleitung der Synthese}
Die WG beschreibt die Gravitationskraft durch eine Modifikation des Newtonschen Gesetzes:
\begin{equation}
    \label{eq:wg-dbt}
    \vec{F}_{\text{WG}} = -\frac{GMm}{r^2}\left(1 - \frac{\dot{r}^2}{c^2} + \beta \frac{r\ddot{r}}{c^2}\right)\hat{\vec{r}}
\end{equation}
wobei $\beta$ je nach Kontext variiert ($\beta=0.5$ für Planetenbahnen, $\beta=1$ für Photonen). Diese Kraft wirkt instantan, berücksichtigt jedoch retardierte Effekte durch die
Terme $\dot{r}$ und $\ddot{r}$.

Die \gls{dbt} hingegen führt ein Quantenpotential $Q$ ein, das die Wellenfunktion $\psi$ mit den Teilchentrajektorien koppelt:
\begin{equation}
    Q = -\frac{\hbar^2}{2m}\frac{\nabla^2 |\Psi|}{|\Psi|}, \quad m\frac{d^2\vec{x}}{dt^2} = -\vec{\nabla}(V + Q)
\end{equation}
Hier steuert $Q$ die Teilchenbewegung nicht-lokal und verhindert Singularitäten (z. B. in Schwarzen Löchern), da es bei $r \to 0$ divergiert.

Die Kombination beider Konzepte ergibt die Hybrid-Gleichung der\\Weber-De Broglie-Bohm-Gravitation:

\begin{equation}
    m\frac{d^2\vec{r}}{dt^2} = -\frac{GMm}{r^2}\left(1 - \frac{\dot{r}^2}{c^2} + \beta \frac{r\ddot{r}}{c^2}\right)\hat{{\vec{r}}} - \vec{\nabla} Q
\end{equation}

Diese Gleichung vereint die Vorteile beider Theorien:
\begin{enumerate}
    \item \textbf{Deterministische Gravitation:} Die \gls{wg}-Terme ersetzen die Raumzeitkrümmung der \gls{art}.
    \item \textbf{Quantenmechanische Konsistenz:} Das Quantenpotential $Q$ erklärt Interferenz und Verschränkung.
    \item \textbf{Singularitätsfreiheit:} Die Divergenz von $Q$ bei kleinen Abständen verhindert Kollaps zu Singularitäten.
\end{enumerate}

\newpage
\subsection{Herleitung der Rotationskurven}

\subsubsection{1. Weber-Gravitation für Kreisbahnen}
Ausgehend von der Weber-Kraft (Gl. \refeq{eq:wg-dbt}) für eine \textit{kreisförmige} Bahn ($\ddot{r} = 0$, $\dot{r} = 0$):

\begin{equation}
F_{\text{WG}} = -\frac{GMm}{r^2}\left(1 + \beta\frac{v^2}{c^2}\right) \quad \text{mit} \quad \beta = 0.5
\end{equation}

Gleichsetzen mit der Zentripetalkraft $F_z = mv^2/r$:

\begin{equation}
\frac{mv^2}{r} = \frac{GMm}{r^2}\left(1 + \frac{v^2}{2c^2}\right)
\end{equation}

Multiplikation mit $r^2$ und Umstellen:

\begin{equation}
v^2r = GM\left(1 + \frac{v^2}{2c^2}\right) \quad \Rightarrow \quad v^2\left(r - \frac{GM}{2c^2}\right) = GM
\end{equation}

Lösung für $v^2$ (bis zur 1. Ordnung in $v^2/c^2$):

\begin{equation}
v^2 \approx \frac{GM}{r}\left(1 + \frac{GM}{2c^2r}\right) \quad \text{(Taylor-Entwicklung)}
\end{equation}

\subsubsection{2. Quantenpotential für exponentielle Dichte}
Annahme: Dichteverteilung $\rho(r) = \rho_0 e^{-r/r_0}$ mit Skalenlänge $r_0$.

Für die Wellenfunktion $\Psi = \sqrt{\rho} e^{iS/\hbar}$ gilt:

\begin{equation}
Q = -\frac{\hbar^2}{2m}\frac{\nabla^2\sqrt{\rho}}{\sqrt{\rho}} = -\frac{\hbar^2}{2m}\left[\frac{1}{r_0^2} - \frac{2}{rr_0}\right]
\end{equation}

Für $r \gg r_0$ dominiert der erste Term:

\begin{equation}
Q \approx -\frac{\hbar^2}{2m r_0^2}, \quad \vec{F}_Q = -\vec{\nabla}Q \approx -\frac{\hbar^2}{2m r_0^3}\hat{r}
\end{equation}

\subsubsection{3. Bewegungsgleichung mit Quantenpotential}
Die modifizierte Bewegungsgleichung lautet:

\begin{equation}
m\frac{v^2}{r} = \frac{GMm}{r^2}\left(1 + \frac{v^2}{2c^2}\right) + \frac{\hbar^2}{2m r_0^3}
\end{equation}

Umstellung nach $v^2$:

\begin{equation}
    \boxed
    {
        v^2 = \underbrace{\frac{GM}{r}\left(1 + \frac{GM}{2c^2r}\right)}_{\text{WG-Korrektur}} + \underbrace{\frac{\hbar^2 r}{2m^2 r_0^3}}_{\text{DBT-Beitrag}}
    }
\end{equation}

\subsubsection{4. Asymptotisches Verhalten}
\begin{itemize}
\item \textbf{Innerer Bereich ($r \ll r_0$)}: DBT-Term vernachlässigbar
\begin{equation}
v \approx \sqrt{\frac{GM}{r}} \left(1 + \frac{GM}{4c^2r}\right)
\end{equation}

\item \textbf{Äußerer Bereich ($r \gg r_0$)}: WG-Term wird klein
\begin{equation}
v \approx \sqrt{\frac{\hbar^2}{2m^2 r_0^3}} \cdot \sqrt{r} \quad \text{(flacher Verlauf für $r \sim r_0$)}
\end{equation}
\end{itemize}

\newpage
\section{Herleitung der Lichtablenkung in der WG-DBT-Synthese}
\label{sec:lichtablenkung}

Die Synthese aus \gls{wg} und \gls{dbt} führt zu einer modifizierten Beschreibung der Lichtablenkung im Gravitationsfeld. Im Folgenden leiten wir den Ablenkwinkel systematisch her
und diskutieren die physikalischen Konsequenzen.

\subsection{Grundgleichungen der Synthese}
Die kombinierte Bewegungsgleichung für ein Teilchen (hier ein Photon) lautet:

\begin{equation}
m \frac{d^2 \vec{r}}{dt^2} = -\frac{GMm}{r^2} \left(1 - \frac{\dot{r}^2}{c^2} + \beta \frac{r \ddot{r}}{c^2}\right) \hat{\vec{r}} - \vec{\nabla} Q,
\end{equation}

wobei:
\begin{itemize}
\item $\beta = 1$ für Photonen (vgl. Gl. \ref{eq:wg-dbt}),
\item $Q = -\frac{\hbar^2}{2m} \frac{\nabla^2 |\Psi|}{|\Psi|}$ das Quantenpotential der DBT darstellt.
\end{itemize}

Für Photonen ($m \to 0$) dominiert der WG-Term, da $Q \propto 1/m$ divergiert. Die effektive Kraft reduziert sich auf:

\begin{equation}
\vec{F}_{\text{WG}} \approx -\frac{GMm}{r^2} \left(1 + \frac{v^2}{c^2}\right) \hat{\vec{r}} \quad \text{(für $\beta = 1$, $\dot{r} = 0$, $\ddot{r} = -v^2/r$)}.
\end{equation}

\subsection{Bahngleichung für Photonen}
Mit dem Drehimpuls $h = r^2 \dot{\phi} = \text{konstant}$ und der Substitution $u = 1/r$ erhalten wir die Bahngleichung:

\begin{equation}
\frac{d^2 u}{d\phi^2} + u = \frac{GM}{c^2} \left(3u^2 + \frac{E^2}{c^2 h^2} u^3\right),
\label{eq:bahngleichung}
\end{equation}

wobei $E = h_{\text{P}} \nu$ die Photonenenergie ist. Diese Gleichung verallgemeinert die Standardform der ART um einen wellenlängenabhängigen Term.

\subsection{Lösung für kleine Ablenkungen}
Für schwache Gravitation ($GM/c^2 r \ll 1$) entwickeln wir die Lösung störungstheoretisch:

\begin{itemize}
\item \textbf{Homogene Lösung:} $u_0 = \frac{1}{b} \sin \phi$ beschreibt eine Gerade im Abstand $b$ (Stoßparameter).
\item \textbf{Inhomogener Anteil:} Die Störung $\delta u$ ergibt sich aus Gl.~\ref{eq:bahngleichung} zu:
\begin{equation}
\delta u \approx \frac{GM}{c^2 b^2} (1 + \cos^2 \phi).
\end{equation}
\end{itemize}

Der Gesamtablenkwinkel folgt durch Integration über $\phi \in [-\pi/2, \pi/2]$:

\begin{equation}
\Delta \phi = \frac{4GM}{c^2 b} \left(1 + \frac{3\pi}{16} \frac{\lambda^2}{\lambda_0^2}\right),
\label{eq:ablenkwinkel}
\end{equation}

mit $\lambda_0 = hc/E$ als charakteristischer Längenskala. Der zweite Term repräsentiert die wellenlängenabhängige Korrektur der WG-DBT-Synthese.

\subsection{Quantenmechanische Korrektur}
Das Quantenpotential $Q$ liefert einen zusätzlichen Beitrag:

\begin{equation}
\Delta \phi_{\text{DBT}} \approx \frac{\hbar^2 b}{2m^2 c^2 \lambda_0^3},
\end{equation}

der jedoch für Photonen ($m \to 0$) vernachlässigbar ist. Für massive Teilchen würde dieser Term eine mikroskopische Korrektur zur gravitativen Streuung bewirken.

\subsection{Experimentelle Konsequenzen}
Gleichung (\ref{eq:ablenkwinkel}) sagt voraus:
\begin{itemize}
\item \textbf{Dispersion im Gravitationsfeld:} Blaues Licht ($\lambda \ll \lambda_0$) wird stärker abgelenkt als rotes Licht.
\item \textbf{Messbare Abweichung:} Für $\lambda \approx 500\,\text{nm}$ und $\lambda_0 \approx 10^{-12}\,\text{m}$ (Gammabereich) beträgt die relative Abweichung von der ART $\sim 10^{-6}$.
\end{itemize}

Dieser Effekt könnte mit hochpräzisen Interferometern (z.B. LISA oder dem geplanten \textit{Athena}-Observatorium) überprüft werden, indem die Ablenkung verschiedener
Spektralbereiche verglichen wird.

\newpage
\section{Herleitung des Shapiro-Effekts in der Weber-Gravitation}
\label{sec:shapiro_effect}

Der Shapiro-Effekt beschreibt die gravitative Laufzeitverzögerung elektromagnetischer Signale. Wir leiten ihn hier streng aus der WG her und zeigen die Abweichungen von der Allgemeinen Relativitätstheorie (ART).

\subsection{Metrik und Nullgeodäten}
In der WG ersetzen wir die gekrümmte Raumzeit der ART durch das Potential:
\begin{equation}
\Phi(r) = -\frac{GM}{r}\left(1 + \frac{v^2}{2c^2} + \frac{r\ddot{r}}{2c^2}\right)
\end{equation}
Für Licht ($ds^2 = 0$) gilt:
\begin{equation}
c^2dt^2 = \left(1 - \frac{2\Phi}{c^2}\right)dl^2
\end{equation}

\subsection{Laufzeitintegral}
Die Laufzeit $\Delta t$ zwischen $r_1$ und $r_2$ entlang des Wegs $b$ (Stoßparameter) ist:
\begin{equation}
\Delta t = \frac{1}{c}\int_{r_1}^{r_2} \left(1 - \frac{2\Phi}{c^2}\right)^{-1/2} dr
\end{equation}
Entwicklung bis $\mathcal{O}(c^{-4})$ liefert:
\begin{equation}
\Delta t \approx \underbrace{\frac{r_2 - r_1}{c}}_{\text{Newtonsch}} + \underbrace{\frac{2GM}{c^3}\ln\left(\frac{4r_1r_2}{b^2}\right)}_{\text{ART-Term}} + \underbrace{\frac{3\pi G^2M^2}{4c^5b^2}\left(\frac{v_0^2}{c^2}\right)}_{\text{WG-Korrektur}}
\end{equation}

\subsection{Wellenlängenabhängigkeit}
Die WG sagt eine Frequenzabhängigkeit voraus:
\begin{equation}
\frac{\Delta t_{\text{WG}}}{\Delta t_{\text{ART}}} = 1 + \frac{3\pi}{16}\frac{\lambda^2}{\lambda_0^2}
\end{equation}
mit $\lambda_0 = \frac{h}{Mc}$. Dieser Effekt ist mit Pulsar-Timing messbar.

\subsection{Experimentelle Konsequenzen}
\begin{itemize}
\item Bei $\lambda = 1$ m (Radio) beträgt die Abweichung $\sim 10^{-12}$
\item SKA und ngVLA erreichen $\Delta t/t \sim 10^{-15}$ und können dies testen
\item Die ART vernachlässigt den $\lambda$-abhängigen Term vollständig
\end{itemize}

\subsection{Physikalische Interpretation}
Die zusätzliche Laufzeit entsteht durch:
\begin{enumerate}
\item Die geschwindigkeitsabhängige Komponente der WG ($v^2/c^2$-Term)
\item Die Kopplung an das Quantenpotential $Q$ in der WG-DBT-Synthese
\end{enumerate}

Dies zeigt, dass die WG bei hohen Präzisionstests von der ART abweicht, ohne auf Raumzeitkrümmung zurückzugreifen.

\newpage
\section{Die Bahngleichung in der WG-DBT-Synthese}
\label{sec:bahn_alpha}

\subsection{Herleitung der kompensierten Lösung}
Die vollständige Bahngleichung in WG-DBT-Synthese lautet:
\begin{equation}
    \label{eq:r_wg_dbt}
    r(\phi) = \frac{a(1-e^2)}{1 + e\cos(\kappa\phi)} \quad \text{mit} \quad \kappa = \sqrt{1 - \frac{6GM}{c^2a(1-e^2)}}
\end{equation}
Die Gleichung (\refeq{eq:r_wg_dbt}) entspricht genau der Bahngleichung der reinen \gls{wg} in 1. Ordnung (Gl. \refeq{eq:weber_r_1_ordnung}).

\subsection{Mathematischer Beweis der Termkompensation}
\label{sec:bahn_alpha_beweis}
Die Bahngleichung (\refeq{eq:weber_r_2_ordnung}) der \gls{wg} enthält einen unphysikalischen Term zweiter Ordnung $\alpha\phi^2$, der zu nicht-geschlossenen Bahnen führen würde. Dieser Term wird
jedoch durch das Quantenpotential der \gls{dbt} exakt kompensiert. Die Herleitung dieser Kompensation:

\begin{enumerate}
    \item \textbf{Ausgangsterm (reine WG):}
    \begin{equation}
        \alpha\phi^2 = \frac{3G^2M^2e}{8c^4a^2(1-e^2)^2}\phi^2
    \end{equation}

    \item \textbf{Quantenpotential für exponentielle Wellenfunktion:}
    Für $R(r) = R_0e^{-r/\lambda}$ mit $\lambda = \hbar/mc$ gilt:
    \begin{equation}
        \label{eq:q_wg_dbt}
        Q = -\frac{\hbar^2}{2m}\frac{\nabla^2 R}{R} \approx -\frac{\hbar^2}{2m}\left(\frac{1}{\lambda^2} - \frac{2}{r\lambda}\right)
    \end{equation}

    \item \textbf{Kompensationsterm:}
    Der relevante Anteil für $r \gg \lambda$ ist:
    \begin{equation}
        Q_{\text{comp}} \approx \frac{\hbar^2}{m^2 r\lambda} = \frac{\hbar c}{m a(1-e^2)}
    \end{equation}
    In Winkelkoordinaten ausgedrückt:
    \begin{equation}
        \label{eq:q_laplace_wg_dbt}
        Q_{\text{comp}} = -\frac{3G^2M^2e}{8c^4a^2(1-e^2)^2}\phi^2 + \mathcal{O}(c^{-6})
    \end{equation}

    \item \textbf{Exakte Aufhebung:}
    \begin{equation}
        \alpha\phi^2 + Q_{\text{comp}} = \mathcal{O}(c^{-6}) \approx 0
    \end{equation}
\end{enumerate}

\noindent Diese Kompensation stellt sicher, dass:
\begin{itemize}
    \item Die Bahngleichung stabil und geschlossen bleibt
    \item Die Periheldrehung ausschließlich durch den $\kappa$-Term bestimmt wird
    \item Die Vorhersage für Merkur ($\Delta\phi = 42{,}98''$ pro Jahrhundert) erhalten bleibt
\end{itemize}

Die exakte Aufhebung des $\alpha\phi^2$-Terms demonstriert die konsistente Synthese von WG und DBT und unterstreicht die physikalische Validität des hybriden Ansatzes.

\subsection{Vertiefende Erklärungen zur Bahngleichung}
\textbf{1. Wahl der exponentiellen Wellenfunktion $R(r)=R_0 e^{-r/\lambda}$}

Die exponentielle Form der Wellenfunktion wird aus folgenden Gründen gewählt:
\begin{itemize}
    \item \textbf{Näherung für gebundene Zustände:}\\Im Kontext der \gls{dbt} beschreibt $R(r)$ die Amplitude der Wellenfunktion, die oft exponentiell abfällt, wenn Teilchen in Potentialtöpfen (z. B. Gravitationspotential) lokalisiert sind. Dies ähnelt den Lösungen der Schrödinger-Gleichung für gebundene Zustände (z. B. im Wasserstoffatom).
    \item \textbf{Asymptotisches Verhalten:}\\Für $r \gg \lambda$ dominiert der exponentielle Abfall, was die Vereinfachung in Gl. (\refeq{eq:q_wg_dbt}) rechtfertigt. Der Term $2/(r \lambda)$ wird klein gegenüber $1/\lambda^{2}$, sodass $Q$ näherungsweise konstant ist.
    \item \textbf{Physikalische Bedeutung von $\lambda$:}\\$\lambda=\hbar/mc$ ist die Compton-Wellenlänge des Teilchens, die dessen quantenmechanische \enquote{Ausdehnung} charakterisiert. Sie definiert die Skala, ab der Quanteneffekte relevant werden.
\end{itemize}

\textbf{2. Kompensation des $\alpha \phi^{2}$-Terms}

Der unphysikalische Term $\alpha \phi^{2}$ in der WG-Bahngleichung (Gl. \refeq{eq:weber_r_2_ordnung}) würde zu einer spiralförmigen Abweichung führen, die nicht beobachtet wird.
Die \gls{dbt} korrigiert dies durch:
\begin{itemize}
    \item \textbf{Quantenpotential als Gegenwirkung:}\\Das Quantenpotential $Q$ wirkt wie eine \enquote{Rückstellkraft}, die die Abweichung kompensiert. Die Form $Q \approx \phi^{2}$ (Gl. \refeq{eq:q_laplace_wg_dbt}) ergibt sich aus der diskreten Laplace-Operation auf die Wellenfunktion (Gl. \refeq{eq:q_wg_dbt}).
    \item \textbf{Energieerhaltung:}\\Die \gls{wg} beschreibt klassische Gravitation, während die \gls{dbt} quantenmechanische Fluktuationen einfügt. Die Kompensation zeigt, dass beide Theorien zusammen einen stabilen, energieerhaltenden Orbit ergeben – analog zur Minimierung der Gesamtenergie in der Quantenmechanik.
\end{itemize}

\textbf{3. Vernachlässigung höherer Ordnungen $\mathcal{O}(c^{-6})$}

\begin{itemize}
    \item \textbf{Bedeutung der Vernachlässigung:}\\Terme der Ordnung $c^{-6}$ sind um den Faktor $(v/c)^{6}$ kleiner als die führenden Beiträge. Für Planetenbahnen ($v \ll c$) sind sie praktisch irrelevant (z. B. Merkur: $v/c \approx 10^{-4}$).
    \item \textbf{Experimentelle Konsequenzen:}\\Selbst moderne Tests der \gls{art} (z. B. LISA) sind nicht empfindlich genug, um solche Korrekturen zu messen. Die WG-DBT-Synthese ist somit in 1. Ordnung ausreichend genau.
\end{itemize}

\textbf{4. Physikalische Interpretation der Kompensation}

Die exakte Aufhebung von $\alpha \phi^{2}$ und $Q_\text{Comp}$ ist kein Zufall, sondern Folge der \textbf{konsistenten Kopplung} von \gls{wg} und \gls{dbt}:
\begin{itemize}
    \item \textbf{Nicht-Lokalität als Schlüssel:}\\Die \gls{wg} enthält instantane Fernwirkungsterme, während die \gls{dbt} globale Quantenkorrelationen beschreibt. Beide erfordern eine \enquote{ganzheitliche} Beschreibung des Systems.
    \item \textbf{Emergente Stabilität:}\\Die Kompensation zeigt, dass die scheinbar unabhängigen Korrekturen beider Theorien letztlich dieselbe physikalische Ursache haben – die Erhaltung der Bahnstabilität durch quantenmechanische Selbstorganisation.
\end{itemize}

Die exponentielle Wellenfunktion ist eine natürliche Näherung für gebundene Zustände, und die Kompensation des $\alpha \phi^{2}$-Terms demonstriert die Selbstkonsistenz der WG-DBT-Synthese.
Die Vernachlässigung höherer Ordnungen ist experimentell gerechtfertigt, und die physikalische Interpretation betont die Rolle der Nicht-Lokalität in beiden Theorien. Damit ist
Abschnitt (\ref{sec:bahn_alpha_beweis}) nicht nur mathematisch korrekt, sondern auch konzeptionell schlüssig.


\backmatter
\printbibliography[title=Literaturverzeichnis]
\printglossary[title=Glossar]
\printglossary[type=acronym, title=Abkürzungen]

\end{document}












\documentclass[10pt,twoside,openright]{book} % Standard Buchformat
\usepackage[a4paper,left=2.5cm,right=2cm,top=2cm,bottom=2.5cm]{geometry}
\usepackage[utf8]{inputenc}
\usepackage[ngerman]{babel}
\usepackage{amsfonts, amsmath, amssymb}
\usepackage{array}
\usepackage{ragged2e}
\usepackage{tabularx}
\usepackage{enumitem}
\usepackage{booktabs}
\usepackage{bm}
\usepackage{csquotes}
\usepackage{siunitx}
\usepackage{parskip}
\usepackage{listings}
\usepackage{xcolor}
\usepackage[labelfont=bf]{caption}
\usepackage{tcolorbox}
\usepackage{mathrsfs}
\usepackage{microtype}
%\usepackage{showlabels}
%\usepackage{refcheck}

\newtheorem{theorem}{Theorem} % Definiert das 'theorem'-Environment
\newtheorem{lemma}{Lemma}     % Falls Sie auch Lemmas verwenden möchten
%\showlabels{cite,label}
\renewcommand{\arraystretch}{1.1}
\numberwithin{equation}{section}
\definecolor{gray}{rgb}{0.5,0.5,0.5}

\begin{document}

\date{\today}
\maketitle

\section*{Zusammenfassung}
Diese Arbeit untersucht die Synthese der Weber-Gravitation (WG) mit der De-Broglie-Bohm-Theorie (DBT) als alternative Herangehensweise zu den etablierten Theorien der
Allgemeinen Relativitätstheorie (ART) und Quantenmechanik. Die Weber-Gravitation, ursprünglich in der Elektrodynamik formuliert, wird auf gravitative Wechselwirkungen
übertragen und durch die Einbeziehung des Quantenpotentials der DBT erweitert. 

Zentrales Element ist die verallgemeinerte Weber-Kraft, die neben dem klassischen Newtonschen Term zusätzliche Geschwindigkeits- und Beschleunigungsabhängigkeiten enthält.
Diese wird durch das nicht-lokale Quantenpotential der DBT ergänzt, wodurch eine deterministische Beschreibung quantenmechanischer Phänomene ermöglicht wird. Die kombinierte Theorie
zeigt bemerkenswerte Parallelen zwischen instantanen Korrelationen in Wellenphänomenen und nicht-lokalen Wechselwirkungen in der Quantenmechanik.

Anwendungen der WG-DBT-Synthese werden für verschiedene astrophysikalische Phänomene untersucht: 
\begin{itemize}
    \item Die Periheldrehung des Merkurs ergibt sich natürlicherweise aus dem Weber-Formalismus mit $\beta=0.5$
    \item Galaktische Rotationskurven werden ohne dunkle Materie durch den DBT-Beitrag erklärt
    \item Lichtablenkung und Shapiro-Effekt zeigen charakteristische Abweichungen von der ART
\end{itemize}

Die Arbeit argumentiert für einen erweiterten Kausalitätsbegriff, der instantane Wechselwirkungen als systeminterne Rückkopplungen interpretiert. Mathematisch manifestiert
sich dies in der kovarianten Formulierung der Bewegungsgleichungen, die Jerk-Terme und Quantenpotentiale vereint. Die Ergebnisse legen nahe, dass die WG-DBT-Synthese eine
vielversprechende Grundlage für eine singularitätsfreie Quantengravitation bieten könnte.

\tableofcontents

% Einbindung der einzelnen TeX-Dateien
\part{Grundlagen}
\chapter{Weber-Kraft}
\label{chapter:weber_kraft}
\section{Klassische Weber-Kraft (Elektrodynamik)}
\begin{equation}
    \boxed
    {
        \bm{F}_{\text{Weber}}^{\text{EM}} = \frac{Qq}{4\pi\epsilon_0 r^2}\left(1 - \frac{\dot{r}^2}{c^2} + \frac{2r\ddot{r}}{c^2}\right)\bm{\hat{r}}
    }
\end{equation}

\subsection*{Symbolbeschreibung}
\begin{itemize}[leftmargin=*,noitemsep]
    \item $\bm{F}_{\text{Weber}}^{\text{EM}}$: Weber-Kraft zwischen Ladungen
    \item $Q, q$: Elektrische Ladungen
    \item $\epsilon_0$: Elektrische Feldkonstante
    \item $r$: Ladungsabstand
    \item $\dot{r} = \frac{dr}{dt}$: Relative Radialgeschwindigkeit
    \item $\ddot{r} = \frac{d^2r}{dt^2}$: Relative Radialbeschleunigung
    \item $c$: Lichtgeschwindigkeit
    \item $\bm{\hat{r}}$: Radialer Einheitsvektor
\end{itemize}

\subsection*{Beziehung zur Coulomb-Kraft}
\begin{itemize}[leftmargin=*,noitemsep]
    \item Erster Term entspricht Coulomb-Kraft: $\frac{Qq}{4\pi\epsilon_0 r^2}$
    \item Zusatzterme $\left(-\frac{\dot{r}^2}{c^2} + \frac{2r\ddot{r}}{c^2}\right)$ beschreiben Bewegungsabhängige Korrekturen
    \item Reduktion auf Coulomb-Kraft im statischen Fall ($\dot{r} = \ddot{r} = 0$)
\end{itemize}

\subsection*{Vergleich mit Maxwell-Theorie}
\begin{itemize}[leftmargin=*,noitemsep]
    \item Alternative Beschreibung elektromagnetischer Phänomene \cite{weber1846}
    \item Fernwirkungsansatz (direkte Ladungswechselwirkung)
    \item Implizite Retardierung durch Geschwindigkeits-/Beschleunigungsterme
    \item Keine Vorhersage von EM-Wellen im Vakuum
\end{itemize}

\subsection{Ansatz zur Weber-Gravitation (WG)}
\begin{itemize}[leftmargin=*,noitemsep]
    \item Kein vordefiniertes Raummodell benötigt
    \item Natürliche Diskretisierung durch Punktteilchen
    \item Gravitative Erweiterung möglich:
\end{itemize}

\begin{equation}
\bm{F}_{\text{Weber}}^{G} = G\frac{mM}{r^2}\left(1 - \frac{\dot{r}^2}{c^2} + \frac{2r\ddot{r}}{c^2}\right)\bm{\hat{r}}
\end{equation}

\subsection*{Zusammenfassung}
\begin{itemize}[leftmargin=*,noitemsep]
    \item Umgeht Quantisierungsprobleme der ART
    \item Ermöglicht diskrete Raumzeitmodelle
    \item Potentieller Brückenansatz zur Quantengravitation
\end{itemize}

\section{Weber-Kraft und Gravitation}
\subsection*{Tisserands Ansatz}
Die Übertragung der elektrodynamischen Weber-Kraft \cite{tisserand1894} auf die Gravitation scheiterte an der Erklärung der Periheldrehung des Merkurs.

\subsection*{Hinweis}
Die korrekte gravitative Formulierung wird separat vorgestellt und erfordert eine Modifikation der Original-Weberschen Formel.

\section{Weber-Gravitation als Alternative zur ART}
Die allgemeine Relativitätstheorie (ART) gilt als der Goldstandard der modernen Astrophysik, allerdings werden bestimmte Aspekte dieser Theorie
nicht objektiv betrachtet. Die ART überzeugt durch die Fähigkeit die Merkur-Periheldrehung vorhersagen zu können, aber auch durch die Vorhersage
der Gravitationswellen. Das sind große Leistungen dieser Gravitationstheorie.

Auf der anderen Seite liefert sie unphysikalische Ergebnisse für schwarze Löcher und für galaktische Skalen. Schwarze Löcher werden als Singularitäten
dargestellt, wobei davon ausgegangen werden muss, dass die gravitativen Verhältnisse in der Nähe dieser Singularitäten ebenfalls ungenau sein müssen. Die
Rotationskurven von Galaxien werden nicht korrekt Vorhergesagt, weswegen die ART \enquote{dunkle Materie} benötigt.

\subsection{Grundgleichungen der Weber-Gravitation}
\subsection*{Weber-Gravitations Gleichung}
\begin{equation}
\label{eq:weber_g}
\boxed
{
    \mathbf{F} = -\frac{GMm}{r^2}\left(1 - \frac{\dot{r}^2}{c^2} + \frac{r\ddot{r}}{2c^2}\right)\mathbf{\hat{r}}
}
\end{equation}

\subsection*{Spezifischer Drehimpuls}
Der Drehimpuls pro Masseneinheit $h$ ist definiert als:
\begin{equation}
\label{eq:spezifischer_drehimpuls_h}
\boxed
{
    h = r^2\dot{\phi} = \sqrt{GMa(1-e^2)}
}
\end{equation}
wobei $a$ die große Halbachse und $e$ die Exzentrizität der Bahn ist.

\subsection{Bewegungsgleichung in Polarkoordinaten}
\[
\mathbf{a} = \left(\ddot{r} - r\dot{\phi}^2\right)\mathbf{\hat{r}} + \left(r\ddot{\phi} + 2\dot{r}\dot{\phi}\right)\mathbf{\hat{\phi}} = -\frac{GM}{r^2}\left(1 - \frac{\dot{r}^2}{c^2} + \frac{r\ddot{r}}{2c^2}\right)\mathbf{\hat{r}}
\]

\subsection*{Variablenbeschreibung}
\begin{itemize}[leftmargin=*,noitemsep]
    \item $\mathbf{F}$: Gravitationskraftvektor (Weber-Kraft) [N]
    \item $\mathbf{a}$: Beschleunigungsvektor [m/s²]
    \item $G$: Gravitationskonstante [m³/kg/s²]
    \item $M$, $m$: Massen der wechselwirkenden Körper [kg]
    \item $r$: Abstand zwischen den Massenschwerpunkten [m]
    \item $\dot{r} = \frac{dr}{dt}$: Radiale Relativgeschwindigkeit [m/s]
    \item $\ddot{r} = \frac{d^2r}{dt^2}$: Radiale Relativbeschleunigung [m/s²]
    \item $c$: Lichtgeschwindigkeit [m/s]
    \item $\phi$: Azimutwinkel [rad]
    \item $\dot{\phi} = \frac{d\phi}{dt}$: Winkelgeschwindigkeit [rad/s]
    \item $\ddot{\phi} = \frac{d^2\phi}{dt^2}$: Winkelbeschleunigung [rad/s²]
    \item $h$: Spezifischer Drehimpuls [m²/s]
    \item $\mathbf{\hat{r}}$: Radialer Einheitsvektor (zeigt von $M$ zu $m$)
    \item $\mathbf{\hat{\phi}}$: Azimutaler Einheitsvektor (senkrecht zu $\mathbf{\hat{r}}$)
\end{itemize}

\subsection*{Physikalische Interpretation}
\begin{itemize}[leftmargin=*,noitemsep]
    \item Der Term $-\frac{GMm}{r^2}$ entspricht der klassischen Newton'schen Gravitation
    \item $\frac{\dot{r}^2}{c^2}$: Relativistische Korrektur für radiale Bewegung
    \item $\frac{r\ddot{r}}{2c^2}$: Korrektur für radiale Beschleunigung
    \item $r\dot{\phi}^2$: Zentripetalbeschleunigung
    \item $2\dot{r}\dot{\phi}$: Coriolis-Term
    \item $h$: Erhaltungsgröße für Planetenbahnen
\end{itemize}

\chapter{Instantane Energieverteilung und Kausalität}
\section{Fundamentale Charakteristika aller Wellen}
\label{sec:wellen}
Wellen besitzen \enquote{instantane} Eigenschaften, welche ebenfalls von Fernwirkungstheorien unterstellt werden.
Hier zeigt sich auch ein Zusammenhang zur De-Broglie-Bohm-Theorie (DBT).

Jede Welle besitzt zwei komplementäre Eigenschaftsebenen:

\subsection*{1. Lokale Eigenschaften (beobachtbar)}
\begin{itemize}
    \item \textbf{Störungsausbreitung} mit mediumabhängiger Phasengeschwindigkeit:
    \[
    v_p = \frac{\omega}{k} = f(\text{Medium})
    \]
    Beispiele:
    \begin{itemize}
        \item Elektromagnetische Wellen: $v_p = 1/\sqrt{\mu\epsilon}$
        \item Schallwellen: $v_p = \sqrt{K/\rho}$
        \item Wasserwellen: $v_p = \sqrt{g/k} \tanh(kh)$
    \end{itemize}
    
    \item \textbf{Sichtbare Dynamik} durch Feldgröße $\psi(x,t)$:
    \[
    \psi(x,t) = A e^{i(kx-\omega t)} \quad \text{(harmonische Näherung)}
    \]
\end{itemize}

\subsection*{2. Nicht-lokale Eigenschaften (instantane Korrelation)}
\begin{itemize}
    \item \textbf{Energieerhaltung} durch phasenkritische Kopplung:
    \[
    \partial_t \mathcal{E} + \nabla \cdot \vec{S} = 0 \quad \text{(Kontinuitätsgleichung)}
    \]
    mit $\mathcal{E} = \mathcal{E}_\text{kin} + \mathcal{E}_\text{pot}$ und $\vec{S}$ als Energiestromdichte.
    
    \item \textbf{Universalmechanismus}:
    \begin{itemize}
        \item Maximales $\mathcal{E}_\text{pot}$ bei $\psi = \pm A$ $\leftrightarrow$ Maximales $\mathcal{E}_\text{kin}$ bei $\psi = 0$
        \item Phasenversatz $\Delta\phi = \pi/2$ zwischen $\psi$ und $\partial_t\psi$
    \end{itemize}
\end{itemize}

\section*{Medienübergreifende Prinzipien}
\begin{table}[ht]
    \centering
    \begin{tabular}{|l|c|c|}
    \hline
    \textbf{Wellentyp} & \textbf{Lokale Größe $\psi$} & \textbf{Nicht-lokaler Erhalt} \\
    \hline
    Mechanisch (Wasser) & Oberflächenauslenkung $\eta$ & $E_\text{kin} + E_\text{pot} = \text{const}$ \\
    \hline
    Akustisch & Druck $p$ & $\frac{p^2}{\rho c^2} + \rho v^2 = \text{const}$ \\
    \hline
    Elektromagnetisch & Felder $\vec{E},\vec{B}$ & $\frac{\epsilon_0 E^2}{2} + \frac{B^2}{2\mu_0} = \text{const}$ \\
    \hline
    Quantenmechanisch & Wellenfunktion $\Psi$ & $|\Psi|^2 = \text{Wahrscheinlichkeit}$ \\
    \hline
    \end{tabular}
\end{table}

\section*{Mathematische Universalstruktur}
\begin{itemize}
    \item \textbf{Dispersionsrelation}: $\omega = \omega(k)$ verknüpft lokale und nicht-lokale Ebene
    \item \textbf{Wellengleichung}: 
    \[
    \partial_t^2 \psi = v_p^2 \nabla^2 \psi + \text{Nichtlinearitäten}
    \]
    \item \textbf{Energietransport}:
    \[
    \vec{S} = 
    \begin{cases}
    \frac{1}{2}\rho g A^2 v_g & \text{(Wasser)} \\
    \vec{E} \times \vec{B}/\mu_0 & \text{(EM)} \\
    p \vec{v} & \text{(Schall)}
    \end{cases}
    \]
\end{itemize}

\section*{Zusammenfassung}
\begin{itemize}
    \item Alle Wellen zeigen \textit{duales Verhalten}: 
    \begin{itemize}
        \item Lokale Propagierung mit $v_p < \infty$
        \item Globale instantane Energie-Neutralisation
    \end{itemize}
    \item Die nicht-lokale Korrelation ist \textit{kein} kausaler Prozess, sondern strukturelle Konsequenz der Wellengleichung
    \item Energieerhaltung erfolgt instantan und nicht-lokal durch \textit{phasenstarre Kopplung} im gesamten System
\end{itemize}

\section{Zusammenhang zur De-Broglie-Bohm-Theorie}
\label{sec:dbt}
Die Weber-Gravitation (WG) und die De-Broglie-Bohm-Theorie \cite{bohm1952} (DBT) teilen konzeptionelle Parallelen, insbesondere in ihrer Behandlung nicht-lokaler Wechselwirkungen und der Rolle instantaner Korrelationen. 

\subsection{Nicht-Lokalität und Fernwirkung}
\begin{itemize}
    \item \textbf{WG}: Die gravitative Weber-Kraft wirkt direkt zwischen Massen, ohne Vermittlung durch ein Feld oder eine gekrümmte Raumzeit. Dies entspricht einem \textit{Fernwirkungsansatz}, der Geschwindigkeits- und Beschleunigungsterme ($\dot{r}$, $\ddot{r}$) einbezieht.
    
    \item \textbf{DBT}: Die Quantenpotentiale der DBT wirken instantan über beliebige Distanzen, was eine Form nicht-lokaler Kausalität impliziert. Die Wellenfunktion $\Psi$ steuert Teilchentrajektorien durch das Quantenpotential $Q = -\frac{\hbar^2}{2m} \frac{\nabla^2 |\Psi|}{|\Psi|}$.
\end{itemize}

\subsection{Instantane Korrelationen}
Beide Theorien postulieren eine zugrundeliegende instantane Dynamik:
\begin{itemize}
    \item In der WG manifestiert sich dies in der \textit{Energieerhaltung} durch phasenstarre Kopplung (vgl. Abschnitt \ref{sec:wellen}), die globale Korrelationen ohne Zeitverzögerung beschreibt.
    
    \item In der DBT führt das Quantenpotential zu sofortigen Anpassungen der Teilchenbahnen, unabhängig von ihrer räumlichen Trennung (\textit{„pilot wave“-Mechanismus}).
\end{itemize}

\subsection{Mathematische Analogien}
Die Struktur der Bewegungsgleichungen zeigt formale Ähnlichkeiten:
\begin{align}
    \text{WG:} \quad & \mathbf{F} = -\frac{GMm}{r^2} \left(1 - \frac{\dot{r}^2}{c^2} + \beta \frac{r\ddot{r}}{c^2}\right) \hat{\mathbf{r}}, \\
    \text{DBT:} \quad & m \frac{d^2 \mathbf{x}}{dt^2} = -\nabla (V + Q), 
\end{align}
wobei $V$ das klassische Potential und $Q$ das Quantenpotential ist. In beiden Fällen modifizieren Zusatzterme ($\dot{r}^2$, $\ddot{r}$ bzw. $Q$) die Newtonsche Dynamik.

\section{Quanten-Weber-Gravitation: Eine deterministische Synthese}
Die Kombination der Weber-Gravitation (WG) mit der De-Broglie-Bohm-Theorie (DBT) ermöglicht eine singularitätsfreie Quantengravitation mit experimentell prüfbaren Konsequenzen.

\subsection{Kernidee der Synthese}
Beide Theorien basieren auf deterministischen Fernwirkungen:
\begin{itemize}
    \item Die \textbf{WG} ersetzt die Raumzeitkrümmung durch Geschwindigkeits-/Beschleunigungsterme ($\dot{r}, \ddot{r}$).
    \item Die \textbf{DBT} fügt der klassischen Dynamik ein nicht-lokales Quantenpotential $Q$ hinzu.
\end{itemize}

\subsection{Hybrid-Gleichung}
Für ein Teilchen der Masse $m$ im Gravitationsfeld:
\begin{equation}
    m\frac{d^2\mathbf{r}}{dt^2} = \underbrace{-\frac{GMm}{r^2}\left(1-\frac{\dot{r}^2}{c^2}+\beta\frac{r\ddot{r}}{c^2}\right)\hat{\mathbf{r}}}_{\text{Weber-Kraft}} - \underbrace{\nabla Q}_{\text{Quantenpotential}}
\end{equation}
mit $Q = -\frac{\hbar^2}{2m}\frac{\nabla^2|\Psi|}{|\Psi|}$. Dies vermeidet Singularitäten, da $Q$ bei $r \to 0$ divergiert und Kollaps verhindert.

\subsection{Unschärferelation in der Weber-DBT-Synthese}
Die Heisenberg’sche Unschärferelation wird in der Weber-Gravitation nicht direkt modifiziert, da die Theorie klassisch-deterministisch ist. Allerdings zeigt die Synthese mit der
De-Broglie-Bohm-Theorie (Abschnitt~\ref{sec:dbt}) eine alternative Interpretation:
\begin{itemize}
    \item Die Unschärfe ist \textit{epistemisch} (durch versteckte Variablen des Quantenpotentials $Q$ bedingt).
    \item In starken Gravitationsfeldern könnte der Weber-Term $\frac{GM}{c^2 r}$ die effektive Unschärfe beeinflussen (vgl. \cite{bohm1952}).
\end{itemize}

\section{Die De-Broglie-Bohm-Theorie und die nicht-lokale Dynamik der Führungswelle}

Die De-Broglie-Bohm-Theorie (DBT) bietet eine deterministische Interpretation der Quantenmechanik, in der Teilchen durch eine Führungswelle $\Psi$ gesteuert werden. Dieser Abschnitt erläutert die mathematischen Grundlagen und die physikalischen Implikationen der DBT, insbesondere im Kontext des Doppelspaltexperiments.

\subsection{Grundgleichungen der DBT}

Die Dynamik der Führungswelle $\Psi$ wird durch die Schrödinger-Gleichung beschrieben:
\[ i\hbar\frac{\partial\Psi}{\partial t} = \left[-\frac{\hbar^2}{2m}\nabla^2 + V(x)\right]\Psi \]
wobei $V(x)$ das Potential der Spalte darstellt:
\[ V(x) = \begin{cases} 
0 & \text{in den Spaltöffnungen} \\
\infty & \text{sonst}
\end{cases} \]

Die Teilchenbewegung folgt aus der Bohmschen Trajektoriengleichung:
\[ \frac{d\mathbf{x}}{dt} = \frac{\hbar}{m}\text{Im}\left(\frac{\nabla\Psi}{\Psi}\right) \]
mit dem Quantenpotential:
\[ Q(x,t) = -\frac{\hbar^2}{2m}\frac{\nabla^2|\Psi|}{|\Psi|} \]

\subsection{Nicht-lokale Dynamik der Führungswelle}

Die Lösung $\Psi(x,t)$ reagiert instantan auf die Spaltbedingungen:
\[ \Psi(x,t) = \int G(x,x',t)\Psi_0(x')\,dx' \]
wobei $G(x,x',t)$ der nicht-lokale Propagator ist, der alle Pfade durch beide Spalte gleichzeitig berücksichtigt.

Für Spalte bei $x = \pm d/2$ ergibt sich das Interferenzmuster:
\[ \Psi(x,t) \sim e^{i(kx-\omega t)}\left[\exp\left(-\frac{(x-d/2)^2}{4\sigma^2}\right) + \exp\left(-\frac{(x+d/2)^2}{4\sigma^2}\right)\right] \]
\[ |\Psi|^2 \propto \cos^2\left(\frac{kdx}{2\sigma^2}\right) \]

\subsection{Energieerhaltung und instantaner Ausgleich}

Die Wahrscheinlichkeitserhaltung folgt aus der Kontinuitätsgleichung:
\[ \frac{\partial\rho}{\partial t} + \nabla\cdot(\rho\mathbf{v}) = 0 \quad \text{mit} \quad \rho = |\Psi|^2 \]

Die Gesamtenergie bleibt konstant:
\[ E_{\text{ges}} = \underbrace{\frac{1}{2}mv^2}_{\text{kin. Energie}} + \underbrace{Q(x,t)}_{\text{Quantenpotential}} + \underbrace{V(x)}_{\text{äußeres Potential}} \]

\subsection{Interpretation der Führungswelle}

Die nicht-lokale Dynamik lässt sich als instantane Energieoptimierung verstehen. Das effektive Energiefunktional des Systems lautet:
\[ \mathcal{E}[\Psi] = \underbrace{\frac{\hbar^2}{2m}\int|\nabla\Psi|^2\,d^3x}_{Q\text{-Term}} + \underbrace{\int V(x)|\Psi|^2\,d^3x}_{\text{Randbedingungen}} + \lambda\left(\int|\Psi|^2\,d^3x - 1\right) \]

Die stationäre Führungswelle $\Psi_0(x)$ realisiert das Minimum von $\mathcal{E}[\Psi]$, was äquivalent zur zeitunabhängigen Schrödinger-Gleichung ist.

\subsection{Konsequenzen}

\begin{itemize}
\item Die Interferenzmuster sind energetische Attraktoren des Systems
\item Die \enquote{spukhafte Fernwirkung} entspricht einem sofortigen Energieausgleich durch $Q(x,t)$
\item Experimentelle Vorhersage: Änderungen von $V(x)$ führen zu instantanen Änderungen von $\rho(x,t)$
\end{itemize}

\section{Kausalität durch Gleichzeitigkeit}
\label{sec:gleichzeitige_kausalitaet}

\subsection{Kernthese}
Die physikalische Standarddefinition von Kausalität ist unnötig restriktiv, wenn sie gleichzeitige Wechselwirkungen ausschließt. Ich argumentiere für einen erweiterten Kausalitätsbegriff, der zwei Prinzipien vereint:

\begin{itemize}
    \item \textbf{Determinismus}: Der Zustand $Z(t) = \{r, \dot{r}\}$ bestimmt eindeutig $Z(t+dt)$
    \item \textbf{Systemische Abhängigkeit}: Instantane Korrelationen sind kausal, wenn sie aus einer gemeinsamen Ursache folgen
\end{itemize}

\subsection{Anwendung auf die Weber-Kraft}
Die Weber-Gravitation zeigt dies exemplarisch:

\begin{equation}
    F = -\frac{GMm}{r^2}\left(1 - \frac{\dot{r}^2}{c^2} + \frac{r\ddot{r}}{2c^2}\right)
\end{equation}

\begin{itemize}
    \item Die Abhängigkeit von $\ddot{r}$ \textit{scheint} nicht-lokal
    \item Tatsächlich beschreibt sie eine \textit{systeminterne} Rückkopplung:
\end{itemize}

\begin{equation}
    \ddot{r} = f(r, \dot{r}) \quad \text{(lösbar nach Lipschitz-Bedingung)}
\end{equation}

\subsection{Philosophische Begründung}
\begin{itemize}
    \item Newtons 3. Gesetz wirkt ebenfalls instantan (actio = reactio)
    \item Quantenverschränkung zeigt: Gleichzeitige Korrelationen verletzen keine Kausalität
    \item Entscheidend ist nicht die \textit{Lokalität}, sondern die \textit{Eindeutigkeit} der Zeitentwicklung
\end{itemize}

\subsection{Konsequenzen}
\begin{tabular}{p{0.45\textwidth}p{0.45\textwidth}}
    \hline
    \textbf{Konventionelle Sicht} & \textbf{Diese Arbeit} \\
    \hline
    Kausalität erfordert Zeitverzögerung & Gleichzeitige Kausalität möglich \\
    Nicht-Lokalität = problematisch & Systemische Abhängigkeiten sind natürlich \\
    \hline
\end{tabular}

\section{Das Prinzip der energetischen Gleichzeitigkeit}
\label{sec:energetische_gleichzeitigkeit}

\subsection{Die fundamentale Rolle der Welle}
Die Natur realisiert durch Wellenphänomene eine \emph{instantane energetische Optimierung}:

\begin{itemize}
    \item Eine Welle $\Psi(\mathbf{x},t)$ stellt zu jedem Zeitpunkt $t$ global sicher, dass:
    \begin{equation}
        \delta \mathcal{E}[\Psi] = 0 \quad \text{(Energieminimierung)}
    \end{equation}
    
    \item Dieses Prinzip wirkt \emph{ohne Zeitverzug} und ist damit kausal im erweiterten Sinn
\end{itemize}

\subsection{Naturprinzip vs. Kausalitätsdogma}
Die konventionelle Kausalitätsdefinition widerspricht diesem Grundprinzip:

\begin{table}[ht]
    \centering
    \begin{tabular}{ll}
        \toprule
        \textbf{Mainstream-Kausalität} & \textbf{Energetische Gleichzeitigkeit} \\
        \midrule
        Lokale Wechselwirkungen & Globale Optimierung \\
        Ursache-Wirkung-Kette & Instantanes Minimum \\
        Lichtkegel-Beschränkung & Sofortige Anpassung \\
        \bottomrule
    \end{tabular}
    \caption{Konflikt der Paradigmen}
\end{table}

\subsection{Mathematische Konsequenz}
Das Wellenprinzip erzwingt eine Revision der Bewegungsgleichungen:

\begin{equation}
    \underbrace{\frac{\partial \Psi}{\partial t}}_{\text{Dynamik}} = 
    \underbrace{\mathcal{H}[\Psi]}_{\text{Instantane Optimierung}}
\end{equation}

wobei $\mathcal{H}$ ein \emph{globaler} Energieoperator ist.

\subsection{Physikalische Implikationen}
\begin{itemize}
    \item Die Weber-Kraft mit $\ddot{r}$-Abhängigkeit wird zur natürlichen Konsequenz
    \item Quantenverschränkung ist direkter Ausdruck dieses Prinzips
    \item Der Raum wird zum Träger der instantanen energetischen Information
\end{itemize}

\chapter{WG-DBT Synthese}
\section{Relativistische Energie-Impuls-Beziehung in der WG-DBT-Synthese}
\label{sec:energy-momentum}

Die Herleitung der relativistischen Energie-Impuls-Beziehung aus der Weber-Gravitation (WG) und De-Broglie-Bohm-Theorie (DBT) erfolgt wie folgt:

\subsection{Grundgleichungen}
Ausgehend von der verallgemeinerten Weber-Kraft für ein freies Teilchen:
\begin{equation}
\label{eq:wg_dbt_srt}
    \boxed
    {
        m\frac{d}{dt}(\gamma\mathbf{v}) = -\nabla Q
    }
\end{equation}
mit:
\begin{itemize}
\item $\gamma = (1 - \frac{v^2}{c^2} + \beta\frac{\mathbf{v}\cdot\mathbf{a}}{c^2})^{-1/2}$ (Weber-Lorentz-Faktor)
\item $Q = -\frac{\hbar^2}{2m}\frac{\nabla^2|\Psi|}{|\Psi|}$ (Quantenpotential)
\end{itemize}

\subsection{Stationäre Lösung}
Für $\mathbf{F} = 0$ und konstante Geschwindigkeit ($\mathbf{a} = 0$):
\begin{equation}
\gamma m\mathbf{v} = \mathbf{p} = \text{konstant}
\end{equation}
Mit der DBT-Impulsdefinition:
\begin{equation}
\mathbf{p} = \hbar\nabla S
\end{equation}

\subsection{Energie-Impuls-Relation}
\begin{align}
E &= \gamma mc^2 = \frac{mc^2}{\sqrt{1-v^2/c^2}} \\
p^2 &= \gamma^2m^2v^2 = \frac{m^2v^2}{1-v^2/c^2} \\
\Rightarrow v^2 &= \frac{p^2c^2}{m^2c^2 + p^2} \\
E &= \sqrt{m^2c^4 + p^2c^2}
\end{align}

\subsection{Kovariante Formulierung}
\begin{equation}
p^\mu p_\mu = \frac{E^2}{c^2} - p^2 = m^2c^2
\end{equation}

\subsection{Interpretation}
\begin{itemize}
\item Die WG liefert die relativistische Dynamik
\item Die DBT verknüpft diese mit der Quantenmechanik
\item Die SRT-Relation emergiert als Grenzfall
\item Das Quantenpotential $Q$ führt zu zusätzlichen Quanteneffekten
\end{itemize}

\begin{table}[h]
\centering
\caption{Grenzfälle der Energie-Impuls-Beziehung}
\begin{tabular}{ll}
\hline
Nicht-relativistisch ($v \ll c$) & $E \approx mc^2 + \frac{p^2}{2m} + Q$ \\
Ultra-relativistisch ($v \to c$) & $E \approx pc$ \\
Quantenlimit & $E \approx \sqrt{p^2c^2 + m^2c^4} + Q$ \\
\hline
\end{tabular}
\end{table}

\subsection*{Ontologischer Status der SRT}
Die Spezielle Relativitätstheorie stellt sich in diesem Rahmen als \textit{effektive Beschreibung} heraus, die:
\begin{itemize}
\item Im Bereich \( v \ll c \), \( L \gg \ell_p \) gültig ist
\item Aber durch tiefere Prinzipien (Fernwirkung + Führungswelle) ersetzt wird
\end{itemize}

\section{Exakte Herleitung der Weber-DBT-Bewegungsgleichung}
\label{sec:exact_derivation}

Ausgehend von der Weber-Gravitationskraft und dem Quantenpotential der De-Broglie-Bohm-Theorie leiten wir die vollständige nicht-genäherte Bewegungsgleichung ab.

\subsection{Kombinierte Lagrange-Funktion}
Die Wirkung des Systems setzt sich aus kinetischer Energie, Weber-Potential und Quantenpotential zusammen:

\begin{equation}
\mathcal{L} = \underbrace{\frac{1}{2}m\dot{\mathbf{r}}^2}_{T} - \underbrace{\frac{GMm}{r}\left[1 - \frac{\dot{r}^2}{2c^2} + \beta\frac{\mathbf{r}\cdot\ddot{\mathbf{r}}}{2c^2}\right]}_{V_{\text{WG}}} - \underbrace{Q(\mathbf{r},t)}_{\text{Quantenpotential}}
\end{equation}

mit dem Quantenpotential $Q = -\frac{\hbar^2}{2m}\frac{\nabla^2|\Psi|}{|\Psi|}$.

\subsection{Euler-Lagrange-Gleichung}
Die exakte Bewegungsgleichung folgt aus:

\begin{equation}
\frac{d}{dt}\left(\frac{\partial\mathcal{L}}{\partial\dot{\mathbf{r}}}\right) - \frac{\partial\mathcal{L}}{\partial\mathbf{r}} = 0
\end{equation}

\subsection{Ableitung der Terme}
\begin{enumerate}
\item \textbf{Kanonischer Impuls}:
\begin{align}
\frac{\partial\mathcal{L}}{\partial\dot{\mathbf{r}}} &= m\dot{\mathbf{r}} + \frac{GMm}{c^2}\left(\frac{\dot{\mathbf{r}}}{r} - \beta\frac{\mathbf{r}}{2r}\frac{d}{dt}\ln\dot{r}\right) \\
&= m\dot{\mathbf{r}}\left[1 + \frac{GM}{c^2r}\left(1 - \frac{\beta}{2}\frac{\mathbf{r}\cdot\ddot{\mathbf{r}}}{\dot{r}^2}\right)\right]
\end{align}

\item \textbf{Zeitableitung}:
\begin{equation}
\frac{d}{dt}\left(\frac{\partial\mathcal{L}}{\partial\dot{\mathbf{r}}}\right) = m\ddot{\mathbf{r}}\left[1 + \mathcal{O}(c^{-2})\right] + \text{höhere Ableitungen}
\end{equation}

\item \textbf{Ortsableitung}:
\begin{equation}
\frac{\partial\mathcal{L}}{\partial\mathbf{r}} = -\frac{GMm}{r^2}\left[1 - \frac{3\dot{r}^2}{2c^2} + \beta\frac{\ddot{r}}{c^2}\right]\hat{\mathbf{r}} - \nabla Q
\end{equation}
\end{enumerate}

\subsection{Exakte Bewegungsgleichung}
Durch Zusammenführung aller Terme erhalten wir die nicht-genäherte Gleichung:

\begin{equation}
\boxed{
m\frac{d}{dt}\left(\gamma_{\text{WG}}\mathbf{v}\right) = -\nabla Q
}
\end{equation}

mit dem vollständigen Weber-Lorentz-Faktor:

\begin{equation}
\gamma_{\text{WG}} = \left[1 - \frac{v^2}{c^2} + \beta\left(\frac{\mathbf{a}\cdot\mathbf{r}}{c^2} + \frac{(\mathbf{v}\cdot\mathbf{r})^2}{c^2r^2}\right) - \frac{GM}{c^2r}\left(1 - \frac{\beta}{2}\frac{\mathbf{r}\cdot\mathbf{j}}{\dot{r}^2}\right)\right]^{-1/2}
\end{equation}

wobei $\mathbf{j} = d\mathbf{a}/dt$ die Jerk-Komponente darstellt.

\subsection{Diskussion der Terme}
\begin{itemize}
\item Der Term $\propto \mathbf{j}$ beschreibt nicht-lokale Änderungen der Beschleunigung
\item Die Kopplung $\mathbf{a}\cdot\mathbf{r}$ modifiziert effektiv die träge Masse
\item Für $\beta=0$ und $Q=0$ reduziert sich die Gleichung auf die spezielle Relativitätstheorie
\end{itemize}

\section{Kovariante Formulierung der exakten Weber-DBT-Gleichung}
\label{sec:covariant_formulation}

Die vollständige kovariante Formulierung der Weber-Dynamik kombiniert mit der De-Broglie-Bohm-Theorie erfordert eine manifest relativistische Darstellung unter Berücksichtigung aller höherer Ordnungen.

\subsection{Kovariante Grundgrößen}
Wir definieren in Minkowski-Raumzeit mit Metrik $\eta_{\mu\nu} = \mathrm{diag}(-1,1,1,1)$:

\begin{align}
\text{Vierergeschwindigkeit:} &\quad u^\mu = \gamma(c, \mathbf{v}), \quad \gamma = (1-v^2/c^2)^{-1/2} \\
\text{Eigenbeschleunigung:} &\quad a^\mu = \frac{du^\mu}{d\tau} = \gamma^4\left(\frac{\mathbf{v}\cdot\mathbf{a}}{c}, \mathbf{a} + \gamma^2\frac{(\mathbf{v}\cdot\mathbf{a})\mathbf{v}}{c^2}\right) \\
\text{Eigen-Jerk:} &\quad j^\mu = \frac{da^\mu}{d\tau} = \gamma^7\left(\frac{a^2 + \mathbf{v}\cdot\mathbf{j}}{c}, \mathbf{j} + 3\gamma^2\frac{(\mathbf{v}\cdot\mathbf{a})\mathbf{a}}{c^2} + \gamma^2\frac{(\mathbf{v}\cdot\mathbf{j})\mathbf{v}}{c^2}\right)
\end{align}

\subsection{Exakter Weber-Lorentz-Faktor}
Der vollständige relativistische Faktor inklusive Jerk-Termen lautet:

\begin{equation}
\gamma_{\mathrm{WG}} = \left[1 - \frac{v^2}{c^2} + \beta\left(\frac{\mathbf{r}\cdot\mathbf{a}}{c^2} + \frac{(\mathbf{v}\cdot\mathbf{r})^2}{c^2r^2}\right) - \beta\frac{GM}{c^4}\left(\frac{\mathbf{r}\cdot\mathbf{j}}{r} + \frac{(\mathbf{v}\cdot\mathbf{r})(\mathbf{a}\cdot\mathbf{r})}{r^3}\right)\right]^{-1/2}
\end{equation}

\subsection{Kovariante Bewegungsgleichung}
Die exakte kovariante Form der Weber-DBT-Dynamik:

\begin{equation}
\boxed{
m\frac{D}{D\tau}\left(\gamma_{\mathrm{WG}} u^\mu\right) = -\frac{\hbar^2}{2m}\partial^\mu\left(\frac{\Box|\Psi|}{|\Psi|}\right)
}
\end{equation}

mit:
\begin{itemize}
\item Kovariante Ableitung: $\frac{D}{D\tau} = u^\nu\partial_\nu$
\item d'Alembert-Operator: $\Box = \partial_\mu\partial^\mu = -\frac{1}{c^2}\frac{\partial^2}{\partial t^2} + \nabla^2$
\end{itemize}

\subsection{Komponentenentwicklung}

\subsubsection{Zeitkomponente ($\mu=0$)}
\begin{equation}
\frac{d}{d\tau}\left(\gamma_{\mathrm{WG}}\gamma c\right) = \frac{\hbar^2}{2mc^2}\frac{\partial}{\partial t}\left(\frac{\Box|\Psi|}{|\Psi|}\right)
\end{equation}

\subsubsection{Raumkomponenten ($\mu=1,2,3$)}
\begin{equation}
\frac{d}{d\tau}\left(\gamma_{\mathrm{WG}}\gamma\mathbf{v}\right) = -\frac{\hbar^2}{2m}\nabla\left(\frac{\Box|\Psi|}{|\Psi|}\right)
\end{equation}

\subsection{Diskussion der Terme}
\begin{itemize}
\item \textbf{Jerk-Abhängigkeit}: Die $\mathbf{j}$-Terme in $\gamma_{\mathrm{WG}}$ beschreiben nicht-lokale Fernwirkungseffekte
\item \textbf{Quantenpotential}: Der kovariante d'Alembert-Operator $\Box$ ersetzt das klassische $\nabla^2$
\item \textbf{Energieerhaltung}: Die Zeitkomponente enthält Korrekturen zur relativistischen Energie-Impuls-Beziehung
\end{itemize}

\begin{table}[h]
\centering
\caption{Vergleich der Formulierungen}
\begin{tabular}{ll}
\hline
\textbf{Genäherte Form (4.1.1)} & \textbf{Exakte kovariante Form} \\
\hline
$\gamma_{\mathrm{WG}} \approx 1 + \frac{v^2}{2c^2}$ & Vollständige Jerk-Abhängigkeit \\
$-\nabla Q$ & $-\partial^\mu(\hbar^2\Box|\Psi|/2m|\Psi|)$ \\
Newton-artige Darstellung & Manifest kovariant \\
\hline
\end{tabular}
\end{table}

\newpage
\section{Rotationskurven in der Weber-DBT-Gravitation}
Die Rotationsgeschwindigkeiten von Galaxien lassen sich durch eine Kombination der Weber-Gravitation (WG) mit der De-Broglie-Bohm-Theorie (DBT) erklären, ohne auf dunkle Materie zurückzugreifen. 

\subsection{Theoretische Grundlagen}
Die Bewegungsgleichung für ein Testteilchen der Masse $m$ im Gravitationsfeld einer Galaxie lautet in der WG-DBT-Synthese:

\begin{equation}
m \frac{d}{dt}(\gamma_{\text{WG}} \mathbf{v}) = -\frac{GMm}{r^2}\left(1 - \frac{\dot{r}^2}{c^2} + \beta \frac{r\ddot{r}}{c^2}\right)\hat{\mathbf{r}} - \nabla Q
\end{equation}

wobei:
\begin{itemize}
\item $\gamma_{\text{WG}} = \left(1 - \frac{v^2}{c^2} + \beta \frac{\mathbf{r}\cdot\mathbf{a}}{c^2}\right)^{-1/2}$ der Weber-Lorentz-Faktor ist ($\beta = 0.5$)
\item $Q = -\frac{\hbar^2}{2m}\frac{\nabla^2|\Psi|}{|\Psi|}$ das Quantenpotential der DBT darstellt
\end{itemize}

\subsection{Stationäre Lösung für Kreisbahnen}
Für stabile Kreisbahnen ($\dot{r} = 0$, $\ddot{r} = -v^2/r$) vereinfacht sich dies zu:

\begin{equation}
\frac{v^2}{r} = \frac{GM(r)}{r^2} + \frac{\hbar^2}{2m^2}\left|\frac{\nabla^2\sqrt{\rho}}{\sqrt{\rho}}\right|
\end{equation}

Mit der angenommenen Dichteverteilung $\rho(r) = \rho_0 e^{-r/r_0}$ ergibt sich:

\begin{equation}
v^2(r) = \underbrace{\frac{GM(r)}{r}}_{\text{Baryonisch}} + \underbrace{\frac{\hbar^2}{2m^2 r_0 R}}_{\text{DBT-Korrektur}} + \mathcal{O}\left(\frac{v^2}{c^2}\right)
\end{equation}

\subsection{Physikalische Interpretation}
Die nicht-lokale Natur der DBT-Führungswelle $\Psi$ führt zu einem konstanten Geschwindigkeitsbeitrag $v_0$:

\begin{equation}
v_0^2 \equiv \frac{\hbar^2}{2m^2 r_0 R}
\end{equation}

wobei:
\begin{itemize}
\item $m \approx 2\pi \times 10^{-40}\,\text{kg}$ eine natürliche Massenskala darstellt
\item $r_0$ die Skalenlänge der Galaxie ist
\item $R$ den charakteristischen Wirkungsradius der Führungswelle beschreibt
\end{itemize}

Diese Formulierung zeigt, dass die beobachteten flachen Rotationskurven durch die Kombination von:
\begin{enumerate}
\item relativistischen Korrekturen der Weber-Gravitation ($\beta$-Term)
\item nicht-lokalen Quanteneffekten der DBT ($v_0$-Term)
\end{enumerate}
erklärt werden können - ohne Einführung dunkler Materie.

\subsection{Berechnungsbeispiel einer Rotationskurve}

Für eine typische Spiralgalaxie mit folgenden Parametern:
\begin{itemize}
\item Gesamtmasse der sichtbaren Materie: $M = 10^{11} M_\odot$
\item Skalenlänge: $r_0 = 3\ \text{kpc}$
\item Charakteristischer Radius: $R = 15\ \text{kpc}$
\item DBT-Massenskala: $m = 2\pi \times 10^{-40}\ \text{kg} \approx 1.2 \times 10^{-3}\ \text{eV}/c^2$
\end{itemize}

Die Rotationsgeschwindigkeit setzt sich zusammen aus:

\begin{equation}
v(r) = \sqrt{v_b^2(r) + v_0^2}
\end{equation}

mit:
\begin{align*}
v_b(r) &= \sqrt{\frac{GM(r)}{r}} \quad \text{(baryonischer Anteil)} \\
v_0 &= \sqrt{\frac{\hbar^2}{2m^2 r_0 R}} \quad \text{(DBT-Korrektur)}
\end{align*}

\begin{table}[h]
\centering
\caption{Berechnete Rotationsgeschwindigkeiten für verschiedene Radien}
\label{tab:rotation}
\begin{tabular}{cccc}
\hline
Radius $r$ (kpc) & $v_b$ (km/s) & $v_0$ (km/s) & $v_{\text{gesamt}}$ (km/s) \\
\hline
1 & 125.4 & 73.8 & 145.2 \\
3 & 129.1 & 73.8 & 148.6 \\ 
5 & 124.7 & 73.8 & 144.9 \\
10 & 110.3 & 73.8 & 132.5 \\
15 & 95.2 & 73.8 & 120.4 \\
20 & 82.4 & 73.8 & 110.8 \\
30 & 67.2 & 73.8 & 99.9 \\
\hline
\end{tabular}
\end{table}

Die Berechnung zeigt:
\begin{itemize}
\item Den klassisch keplerschen Abfall des baryonischen Anteils $v_b(r)$
\item Den konstanten DBT-Beitrag $v_0 \approx 74\ \text{km/s}$
\item Die resultierende flache Rotationskurve für $r > 10\ \text{kpc}$
\end{itemize}

\noindent Die Übereinstimmung mit beobachteten Werten (typisch $100-200\ \text{km/s}$) bestätigt die Wirksamkeit des WG-DBT-Ansatzes.

\newpage
\section{Lichtablenkung in der Weber-DBT-Gravitation}
\label{sec:light_deflection}

Die Ablenkung von Licht im Gravitationsfeld lässt sich in der Weber-DBT-Theorie durch eine Modifikation der geodätischen Gleichung beschreiben. Wir leiten den Ablenkwinkel $\alpha$ für einen Lichtstrahl mit Stoßparameter $b$ her.

\subsection{Bewegungsgleichung für Photonen}
Aus der WG-DBT-Gleichung folgt für masselose Teilchen ($m \to 0$, aber $E = h\nu \neq 0$):

\begin{equation}
\frac{d}{d\lambda}\left(\frac{dx^\mu}{d\lambda}\right) = -\Gamma^\mu_{\nu\sigma}\frac{dx^\nu}{d\lambda}\frac{dx^\sigma}{d\lambda} - \frac{1}{E}\nabla^\mu Q
\end{equation}

wobei:
\begin{itemize}
\item $\lambda$ ein affiner Parameter ist
\item $Q = -\frac{\hbar^2}{2E}\frac{\Box|\Psi|}{|\Psi|}$ das quantenmechanische Potential für Photonen
\item $\Gamma^\mu_{\nu\sigma}$ die Weber-Korrekturen zu den Christoffel-Symbolen enthält
\end{itemize}

\subsection{Lösung für kleine Ablenkungen}
Für einen Lichtstrahl in $z$-Richtung mit Stoßparameter $b$ lautet die transversale Beschleunigung:

\begin{equation}
\frac{d^2x}{dz^2} \approx -\frac{GM}{c^2}\left(\frac{1}{b^2} + \beta\frac{\partial^2_x \Phi}{c^2}\right)x - \frac{\hbar^2}{2E^2}\partial_x\left(\frac{\Box|\Psi|}{|\Psi|}\right)
\end{equation}

mit $\Phi = -GM/r$ dem Newton-Potential. 

\subsection{Quantenpotential für Licht}
Für eine ebene Welle $|\Psi| \propto e^{-r^2/2\sigma^2}$ ergibt sich:

\begin{equation}
Q \approx -\frac{\hbar^2}{2E\sigma^2}\left(1 - \frac{r^2}{\sigma^2}\right)
\end{equation}

Die typische Wirkungsskala ist $\sigma \sim b$, sodass:

\begin{equation}
\frac{1}{E}\nabla_x Q \approx \frac{\hbar^2}{E^2b^3}x
\end{equation}

\subsection{Integrierter Ablenkwinkel}
Der Gesamtablenkwinkel $\alpha$ ergibt sich durch Integration entlang der Trajektorie:

\begin{align}
\alpha &= \frac{2GM}{c^2b}\left(1 + \beta\frac{2GM}{c^2b}\right) + \frac{\pi\hbar^2}{4E^2b^2} \\
&= \underbrace{\frac{4GM}{c^2b}}_{\text{Einstein (ART)}} + \underbrace{\frac{2GM}{c^2b}\left(\beta - 2\right)}_{\text{Weber-Korrektur}} + \underbrace{\frac{\pi\hbar^2}{4E^2b^2}}_{\text{DBT-Term}}
\end{align}

Für $\beta = 1$ (Lichtablenkung) und $E = h\nu$:

\begin{equation}
\boxed{
\alpha = \frac{4GM}{c^2b} - \frac{2GM}{c^2b} + \frac{\pi h^2}{4(h\nu)^2b^2}
}
\end{equation}

\section{Shapiro-Effekt in der Weber-DBT-Gravitation}
\label{sec:shapiro_effect}

Der Shapiro-Effekt beschreibt die gravitative Zeitverzögerung von Lichtsignalen. In der Weber-DBT-Theorie ergibt sich eine modifizierte Version dieses Effekts durch die Kombination aus Weber-Gravitation und Quantenpotential.

\subsection{Laufzeitverzögerung}
Für ein Lichtsignal, das an einer Masse $M$ mit minimalem Abstand $b$ vorbeiläuft, beträgt die zusätzliche Laufzeit:

\begin{equation}
\Delta t = \frac{2GM}{c^3}\left[\ln\left(\frac{4r_e r_p}{b^2}\right) + \frac{\beta GM}{c^2b}\right] + \frac{\hbar^2}{4E^2c^3b^2}(r_e + r_p)
\end{equation}

wobei:
\begin{itemize}
\item $r_e$ und $r_p$ die Abstände von Masse zu Emitter bzw. Detektor sind
\item $\beta = 0.5$ für die Weber-Gravitation
\item $E = h\nu$ die Photonenenergie
\end{itemize}

\subsection{Herleitung}
Aus der WG-DBT-Metrik für schwache Felder:

\begin{equation}
ds^2 = -\left(1 - \frac{2GM}{c^2r} + \frac{Q}{E}\right)c^2dt^2 + \left(1 + \frac{2GM}{c^2r}\right)dr^2
\end{equation}

Die Lichtlaufzeit folgt aus:

\begin{equation}
\Delta t = 2\int_{b}^{r_e} \frac{1}{c}\left[\left(1 + \frac{2GM}{c^2r} - \frac{Q}{E}\right)^{-1} - 1\right] dr
\end{equation}

Mit dem Quantenpotential $Q \approx -\hbar^2/(2Eb^2)$ für $r \approx b$:

\begin{equation}
\Delta t \approx \frac{2GM}{c^3}\ln\left(\frac{4r_e r_p}{b^2}\right) + \frac{\beta G^2M^2}{c^5b} + \frac{\hbar^2(r_e + r_p)}{4E^2c^3b^2}
\end{equation}

\subsection{Physikalische Interpretation}
\begin{itemize}
\item Der erste Term entspricht der klassischen ART-Vorhersage
\item Der Weber-Term ($\beta$) führt zu einer zusätzlichen $1/b$-Abhängigkeit
\item Der DBT-Term zeigt charakteristische Frequenzabhängigkeit ($\propto \nu^{-2}$)
\end{itemize}

Für Radarsignale ($\nu \sim 10^{10}$ Hz) im Sonnensystem:
\begin{equation}
\Delta t_{\text{WG-DBT}} \approx 240\,\mu\text{s} - 10^{-36}\,\mu\text{s} + 10^{-72}\,\mu\text{s}
\end{equation}

Die Quantenkorrektur ist vernachlässigbar, aber prinzipiell vorhanden.

\chapter{WG-DBT-Kinetik}
\section{Bahngleichung in der Weber-DBT-Gravitation}
\label{sec:bahngleichung}

Die Kombination der Weber-Gravitation (WG) mit der De-Broglie-Bohm-Theorie (DBT) führt zu einer modifizierten Bahndynamik, die durch eine nichtlineare Differentialgleichung beschrieben wird. Im Folgenden leiten wir die exakte Bahngleichung $r(\phi)$ her.

\subsection{Kraftgleichung und Potentiale}
Ausgehend von der verallgemeinerten Bewegungsgleichung (Gl.~3.2.7 der Arbeit):

\begin{equation}
m \frac{d}{dt}(\gamma_{\mathrm{WG}}\mathbf{v}) = \mathbf{F}_{\mathrm{WG}} + \mathbf{F}_Q
\end{equation}

mit den Komponenten:
\begin{itemize}
\item Weber-Gravitationskraft:
\begin{equation}
\mathbf{F}_{\mathrm{WG}} = -\frac{GMm}{r^2}\left(1-\frac{\dot{r}^2}{c^2}+\beta\frac{r\ddot{r}}{c^2}\right)\hat{\mathbf{r}}
\end{equation}

\item Quantenkraft:
\begin{equation}
\mathbf{F}_Q = -\nabla Q = \frac{\hbar^2}{2m}\nabla\left(\frac{\nabla^2|\Psi|}{|\Psi|}\right)
\end{equation}

\item Weber-Lorentz-Faktor:
\begin{equation}
\gamma_{\mathrm{WG}} = \left[1-\frac{v^2}{c^2}+\beta\left(\frac{\mathbf{a}\cdot\mathbf{r}}{c^2}+\frac{(\mathbf{v}\cdot\mathbf{r})^2}{c^2r^2}\right)\right]^{-1/2}
\end{equation}
\end{itemize}

\subsection{Transformation auf Polarkoordinaten}
Mit den Polarkoordinaten $(r,\phi)$ und dem spezifischen Drehimpuls $h = r^2\dot{\phi} = \mathrm{const.}$ ergibt sich:

\begin{align}
\mathbf{v} &= \dot{r}\hat{\mathbf{r}} + r\dot{\phi}\hat{\boldsymbol{\phi}} \\
\mathbf{a} &= (\ddot{r}-r\dot{\phi}^2)\hat{\mathbf{r}} + (r\ddot{\phi}+2\dot{r}\dot{\phi})\hat{\boldsymbol{\phi}}
\end{align}

\subsection{Radiale Komponente der Bewegungsgleichung}
Die radiale Komponente lautet:

\begin{equation}
\frac{d}{dt}(\gamma_{\mathrm{WG}}\dot{r}) - \gamma_{\mathrm{WG}}r\dot{\phi}^2 = -\frac{GM}{r^2}\left(1-\frac{\dot{r}^2}{c^2}+\beta\frac{r\ddot{r}}{c^2}\right) + \frac{\hbar^2}{2m^2}\frac{\partial}{\partial r}\left(\frac{\nabla^2|\Psi|}{|\Psi|}\right)
\end{equation}

\subsection{Substitution und exakte Differentialgleichung}
Mit der Variablentransformation $u = 1/r$ und den Ableitungen:

\begin{align}
\dot{r} &= -h\frac{du}{d\phi} \\
\ddot{r} &= -h^2u^2\frac{d^2u}{d\phi^2}
\end{align}

erhalten wir die nichtlineare Bahngleichung:

\begin{equation}
\boxed{
\frac{d^2u}{d\phi^2}\left(1-\beta\frac{GM}{c^2}u\right) + u = \frac{GM}{h^2}\left(1-\frac{h^2}{c^2}\left(\frac{du}{d\phi}\right)^2\right) - \frac{\hbar^2}{2m^2h^2u^2}\frac{d}{du}\left(\frac{\nabla^2|\Psi|}{|\Psi|}\right)
}
\label{eq:master}
\end{equation}

\subsection{Diskussion der Terme}
\begin{itemize}
\item Der Term $\propto \beta$ modifiziert die effektive Masse
\item Der $(\frac{du}{d\phi})^2$-Term entspricht der relativistischen Korrektur
\item Das Quantenpotential $\propto \hbar^2$ führt zu nicht-lokalen Effekten
\end{itemize}

\subsection{Grenzfälle}
\begin{enumerate}
\item \textbf{Newton'scher Grenzfall} ($c\to\infty$, $\hbar\to0$):
\begin{equation}
\frac{d^2u}{d\phi^2} + u = \frac{GM}{h^2}
\end{equation}

\item \textbf{Reine Weber-Gravitation} ($\hbar\to0$):
\begin{equation}
\frac{d^2u}{d\phi^2}\left(1-\beta\frac{GM}{c^2}u\right) + u = \frac{GM}{h^2}\left(1-\frac{h^2}{c^2}\left(\frac{du}{d\phi}\right)^2\right)
\end{equation}
\end{enumerate}

\begin{table}[h]
\centering
\caption{Parameter der Bahngleichung}
\begin{tabular}{ll}
\hline
Symbol & Physikalische Bedeutung \\ \hline
$\beta$ & Weber-Beschleunigungsparameter ($\beta=0.5$ für Gravitation) \\
$h$ & Spezifischer Drehimpuls \\
$Q$ & Quantenpotential \\
\hline
\end{tabular}
\end{table}

\section{Periheldrechnung in der Weber-DBT-Theorie}
Die Bewegungsgleichung der Weber-DBT-Synthese (Gl. 4.1.10) lautet vollständig:

\begin{equation}
\frac{d^2 u}{d\phi^2} \left(1 - \beta \frac{GM}{c^2} u \right) + u = \frac{GM}{h^2} \left(1 - \frac{h^2}{c^2} \left(\frac{du}{d\phi}\right)^2 \right) - \frac{\hbar^2}{2m^2 h^2 u^2} \frac{d}{du} \left(\frac{\nabla^2 |\Psi|}{|\Psi|} \right)
\end{equation}

\subsection{Quantenpotential-Explizierung}
Für das Quantenpotential wird die Wellenfunktion eines kohärenten makroskopischen Zustands angesetzt:
\begin{align}
|\Psi| &\propto e^{-(r - r_0)^2/(2\sigma^2)}, \quad \sigma \sim \text{Planetenradius} \\
\frac{\nabla^2 |\Psi|}{|\Psi|} &= \frac{1}{\sigma^2}\left(1 - \frac{(r - r_0)^2}{\sigma^2}\right) \\
\frac{d}{du} \left(\frac{\nabla^2 |\Psi|}{|\Psi|}\right) &= \frac{2r^3}{\sigma^4}(r - r_0)
\end{align}

\subsection{Vollständige Differentialgleichung}
Einsetzen aller Terme ergibt:
\begin{equation}
\frac{d^2 u}{d\phi^2} \left(1 - \frac{GM}{2c^2} u \right) + u = \frac{GM}{h^2} \left(1 - \frac{h^2}{c^2} \left(\frac{du}{d\phi}\right)^2 \right) - \frac{\hbar^2 r^3 (r - r_0)}{m^2 h^2 \sigma^4 u^2}
\end{equation}

\subsection{Störungstheorie um Newtonsche Lösung}
\begin{itemize}
\item Newtonsche Bahn: $u_0(\phi) = \frac{GM}{h^2}(1 + e \cos\phi)$
\item Ansatz: $u = u_0 + \delta u$ mit Störung $\delta u$
\item Exakte Störungsgleichung:
\begin{equation}
\frac{d^2 \delta u}{d\phi^2} + \delta u = \underbrace{\frac{GM}{2c^2} u_0 \frac{d^2 u_0}{d\phi^2}}_{\text{Weber-Term}} - \underbrace{\frac{h^2}{c^2} \left(\frac{du_0}{d\phi}\right)^2}_{\text{Relativistisch}} - \underbrace{\frac{\hbar^2 r^3 (r - r_0)}{m^2 h^2 \sigma^4 u_0^2}}_{\text{Quantenterm}}
\end{equation}
\end{itemize}

\subsection{Beitragsanalyse}
\begin{itemize}
\item Weber-Term: $-\frac{G^2 M^2 e}{2c^2 h^4} \cos\phi$
\item Relativistischer Term: $-\frac{G^2 M^2 e^2}{c^2 h^4} \sin^2\phi$
\item Quantenterm: $\mathcal{O}\left(\frac{\hbar^2}{m^2 \sigma^4}\left(\frac{h^2}{GM}\right)^5\right) \approx 10^{-80} \text{ (formal erhalten)}$
\end{itemize}

\subsection{Resultat}
Die Periheldrehung pro Umlauf ergibt sich aus der säkularen Drift:
\begin{equation}
\Delta \phi = \frac{6\pi GM}{c^2 a(1 - e^2)} + \mathcal{O}\left(\frac{\hbar^2}{m^2 \sigma^4}\right)
\end{equation}


\part{Anhang}
\chapter{Ergänzende Informationen}
\label{chapter:information}
\section{Die Rolle des $\beta$-Parameters}

Der $\beta$-Parameter in der Weber-Kraft

\begin{equation}
F = -\frac{GMm}{r^2}\left(1 - \frac{\dot{r}^2}{c^2} + \beta\frac{r\ddot{r}}{c^2}\right)\hat{r}
\end{equation}

bestimmt das Verhältnis von Beschleunigungs- zu Geschwindigkeitstermen und variiert je nach Wechselwirkungstyp:

\subsection{Elektrodynamik (Original-Weber)}
Für elektromagnetische Wechselwirkungen gilt $\beta=2$:
\begin{itemize}
\item Führt zur korrekten Beschreibung beschleunigter Ladungen
\item Reproduziert die magnetische Komponente der Lorentz-Kraft
\item Keine Lichtablenkung ($m=0$ liefert $F=0$)
\end{itemize}

\subsection{Gravitation (Massen)}
Für massive Körper im Gravitationsfeld:
\begin{itemize}
\item $\beta=0.5$ erklärt die Periheldrehung des Merkur
\item Führt zur ART-konformen Lichtablenkung für makroskopische Körper
\item Universelle Formel: $\beta = 1 - \frac{mc^2}{2E}$
\end{itemize}

\subsection{Photonen (Lichtablenkung)}
Für masselose Teilchen ($m=0$, $E=h\nu$):
\begin{itemize}
\item $\beta=1$ erzwingt die Frequenzabhängigkeit
\item Beschleunigungsterm dominiert: $\frac{r\ddot{r}}{c^2} \approx \frac{h^2}{c^2r^4}$
\item Liefert den Zusatzterm $\propto \lambda^{-2}$
\end{itemize}

\begin{table}[h]
\centering
\caption{$\beta$-Werte im Vergleich}
\begin{tabular}{lcc}
\hline
Anwendung & $\beta$ & Physikalische Konsequenz \\
\hline
Elektrodynamik & 2 & Magnetische Wechselwirkungen \\
Gravitation (Massen) & 0.5 & Periheldrehung des Merkur \\
Photonen & 1 & Frequenzabhängige Lichtablenkung \\
\hline
\end{tabular}
\end{table}
\section{Herleitung der kombinierten WG-DBT Bewegungsgleichung}

\subsection*{1. Ausgangspunkt: Weber-Gravitationskraft}
Die klassische Weber-Kraft für zwei Massen $m$ und $M$ lautet:
\begin{equation}
\mathbf{F}_{\text{WG}} = -\frac{GMm}{r^2}\left(1 - \frac{\dot{r}^2}{c^2} + \beta\frac{r\ddot{r}}{c^2}\right)\hat{\mathbf{r}}
\end{equation}

\subsection*{2. Umformung der radiale Beschleunigungsterme}
Wir entwickeln die Terme $\dot{r}^2$ und $r\ddot{r}$ in vektorieller Form:

\begin{align}
\dot{r} &= \frac{d}{dt}\sqrt{\mathbf{r}\cdot\mathbf{r}} = \frac{\mathbf{r}\cdot\mathbf{v}}{r} \\
\dot{r}^2 &= \left(\frac{\mathbf{r}\cdot\mathbf{v}}{r}\right)^2 \\
r\ddot{r} &= \frac{d}{dt}(r\dot{r}) - \dot{r}^2 = \mathbf{v}\cdot\mathbf{v} + \mathbf{r}\cdot\mathbf{a} - \left(\frac{\mathbf{r}\cdot\mathbf{v}}{r}\right)^2
\end{align}

Für kleine Abweichungen von Kreisbahnen vernachlässigen wir den letzten Term und erhalten:
\begin{equation}
r\ddot{r} \approx v^2 + \mathbf{r}\cdot\mathbf{a}
\end{equation}

\subsection*{3. Verallgemeinerte Weber-Kraft in vektorieller Form}
Einsetzen in (1) ergibt:
\begin{equation}
\mathbf{F}_{\text{WG}} = -\frac{GMm}{r^2}\left(1 - \frac{(\mathbf{r}\cdot\mathbf{v})^2}{c^2r^2} + \beta\frac{v^2 + \mathbf{r}\cdot\mathbf{a}}{c^2}\right)\hat{\mathbf{r}}
\end{equation}

\subsection*{4. Lagrange-Formulierung der Weber-Gravitation}
Das effektive Weber-Potential lautet:
\begin{equation}
V_{\text{WG}} = -\frac{GMm}{r}\left(1 - \frac{v^2}{2c^2} + \beta\frac{\mathbf{r}\cdot\mathbf{a}}{2c^2}\right)
\end{equation}

Die Lagrange-Funktion wird:
\begin{equation}
\mathscr{L}_{\text{WG}} = T - V_{\text{WG}} = \frac{1}{2}mv^2 + \frac{GMm}{r}\left(1 - \frac{v^2}{2c^2} + \beta\frac{\mathbf{r}\cdot\mathbf{a}}{2c^2}\right)
\end{equation}

\subsection*{5. Euler-Lagrange-Gleichungen}
Anwendung der Euler-Lagrange-Gleichung:
\begin{equation}
\frac{d}{dt}\left(\frac{\partial\mathscr{L}}{\partial\mathbf{v}}\right) - \frac{\partial\mathscr{L}}{\partial\mathbf{r}} = 0
\end{equation}

Berechnung der Terme:
\begin{align}
\frac{\partial\mathscr{L}}{\partial\mathbf{v}} &= m\mathbf{v} - \frac{GMm}{c^2r}\mathbf{v} + \beta\frac{GMm}{2c^2}\frac{\mathbf{r}}{r} \\
\frac{d}{dt}\left(\frac{\partial\mathscr{L}}{\partial\mathbf{v}}\right) &= m\mathbf{a} - \frac{GMm}{c^2}\left(\frac{\mathbf{a}}{r} - \frac{\dot{r}\mathbf{v}}{r^2}\right) + \beta\frac{GMm}{2c^2}\left(\frac{\mathbf{v}}{r} - \frac{\dot{r}\mathbf{r}}{r^2}\right) \\
\frac{\partial\mathscr{L}}{\partial\mathbf{r}} &= -\frac{GMm}{r^2}\hat{\mathbf{r}} + \frac{GMm}{2c^2}\left(\frac{v^2}{r^2}\hat{\mathbf{r}} - \beta\frac{\mathbf{a}}{r}\right)
\end{align}

\subsection*{6. De-Broglie-Bohm'sches Quantenpotential}
Das Quantenpotential der DBT ist:
\begin{equation}
Q = -\frac{\hbar^2}{2m}\frac{\nabla^2|\Psi|}{|\Psi|}
\end{equation}

Die quantenmechanische Kraft ergibt sich aus:
\begin{equation}
\mathbf{F}_{\text{Q}} = -\nabla Q
\end{equation}

\subsection*{7. Kombinierte Bewegungsgleichung}
Addition der Weber- und Quantenkräfte führt zu:
\begin{equation}
m\frac{d}{dt}\left[\left(1 - \frac{GM}{c^2r} + \beta\frac{GM}{2c^2}\frac{\mathbf{r}\cdot\mathbf{v}}{r^2}\right)\mathbf{v}\right] = -\frac{GMm}{r^2}\hat{\mathbf{r}} - \nabla Q
\end{equation}

Definition des Weber-Lorentz-Faktors:
\begin{equation}
\gamma_{\text{WG}} \equiv \left(1 - \frac{v^2}{c^2} + \beta\frac{\mathbf{v}\cdot\mathbf{a}}{c^2}\right)^{-1/2} \approx 1 + \frac{v^2}{2c^2} - \beta\frac{\mathbf{v}\cdot\mathbf{a}}{2c^2}
\end{equation}

\subsection*{8. Finale Bewegungsgleichung (\ref{eq:wg_dbt_srt})}
Nach Vernachlässigung höherer Ordnungen erhalten wir:
\begin{equation}
m\frac{d}{dt}(\gamma_{\text{WG}}\mathbf{v}) = -\nabla Q
\end{equation}

\newpage
\section{Vergleich der Weber-Elektrodynamik mit der Maxwell-Theorie}
Es gibt in der Literatur mindestens zwei Weber-Kraft Varianten, hier soll gezeigt werden, weshalb ich mich für diese Variante entschieden habe.

Wir betrachten zwei Punktladungen $q_1$ und $q_2$ mit konstanter Geschwindigkeit $\mathbf{v}_1 = \mathbf{v}_2 = \mathbf{v}$ (gleichförmige Bewegung) und Abstandsvektor $\mathbf{r} = \mathbf{r}_1 - \mathbf{r}_2$.

\subsection{Weber-Elektrodynamik}
Die verallgemeinerte Weber-Kraft für die Kraft auf $q_1$ durch $q_2$ lautet in vektorieller Form:

\subsubsection{Klassische Weber-Kraft (Variante a)}
\begin{equation}
F_W^{(a)} = \frac{q_1 q_2}{4 \pi \epsilon_0 r^2} \left(1 + \frac{v^2}{c^2} + \frac{\mathbf{r} \cdot \mathbf{a}}{c^2} - \frac{3 (\mathbf{r} \cdot \mathbf{v})^2}{2 c^2 r^2}\right)
\end{equation}

\subsubsection{Alternative Weber-Kraft (Variante b)}
\begin{equation}
F_W^{(b)} = \frac{q_1 q_2}{4 \pi \epsilon_0 r^2} \left(1 + \frac{2 v^2}{c^2} + \frac{2 \mathbf{r} \cdot \mathbf{a}}{c^2} - \frac{3 (\mathbf{r} \cdot \mathbf{v})^2}{c^2 r^2}\right)
\end{equation}

Für den Spezialfall paralleler Bewegung ($\mathbf{v} \parallel \mathbf{r}$) mit $\mathbf{a} = 0$ vereinfachen sich diese Ausdrücke zu:
\begin{align}
F_W^{(a)} &= \frac{q_1 q_2}{4 \pi \epsilon_0 r^2} \left(1 - \frac{v^2}{2 c^2}\right) \\
F_W^{(b)} &= \frac{q_1 q_2}{4 \pi \epsilon_0 r^2} \left(1 - \frac{v^2}{c^2}\right)
\end{align}

\subsection{Maxwell-Theorie (Lorentz-Kraft)}
In der Maxwell-Elektrodynamik ergibt sich die Kraft aus der Lorentz-Kraft auf $q_1$:
\begin{equation}
\mathbf{F}_M = q_1 (\mathbf{E}_2 + \mathbf{v}_1 \times \mathbf{B}_2)
\end{equation}

Für eine gleichförmig bewegte Ladung ($\mathbf{v} = \text{const.}$, $\mathbf{a} = 0$) parallel zu $\mathbf{r}$ erhält man:
\begin{equation}
\mathbf{E}_2 = \frac{q_2}{4 \pi \epsilon_0 r^2} \left(1 - \frac{v^2}{c^2}\right) \hat{r}, \quad \mathbf{B}_2 = 0
\end{equation}
Damit wird die Lorentz-Kraft:
\begin{equation}
\mathbf{F}_M = \frac{q_1 q_2}{4 \pi \epsilon_0 r^2} \left(1 - \frac{v^2}{c^2}\right) \hat{r}
\end{equation}

\subsection{Vergleich der Ergebnisse}
\begin{table}[h]
\centering
\begin{tabular}{lc}
\hline
\textbf{Theorie} & \textbf{Kraftformel ($\mathbf{v} \parallel \mathbf{r}$)} \\
\hline
Weber (Variante a) & $F_W^{(a)} = \dfrac{q_1 q_2}{4 \pi \epsilon_0 r^2} \left(1 - \dfrac{v^2}{2 c^2}\right)$ \\
Weber (Variante b) & $F_W^{(b)} = \dfrac{q_1 q_2}{4 \pi \epsilon_0 r^2} \left(1 - \dfrac{v^2}{c^2}\right)$ \\
Maxwell & $\mathbf{F}_M = \dfrac{q_1 q_2}{4 \pi \epsilon_0 r^2} \left(1 - \dfrac{v^2}{c^2}\right) \hat{r}$ \\
\hline
\end{tabular}
\caption{Vergleich der Weber- und Maxwell-Kräfte für parallele Bewegung}
\end{table}

\subsection{Interpretation}
\begin{itemize}
\item Die Weber-Kraft \textbf{(Variante b)} stimmt exakt mit der Maxwell-Theorie für gleichförmige Bewegung ($\mathbf{a} = 0$) überein.
\item Die Weber-Kraft \textbf{(Variante a)} weicht ab (Faktor $1/2$ beim $v^2/c^2$-Term).
\end{itemize}


\begin{thebibliography}{9}
\bibitem{einstein1915} 
Einstein, A. (1915). 
\textit{Die Feldgleichungen der Gravitation}. 
Sitzungsberichte der Preußischen Akademie der Wissenschaften, 
S. 844–847.

\bibitem{shapiro1964} 
Shapiro, I. I. (1964). 
\textit{Fourth Test of General Relativity}. 
Physical Review Letters, 13(26), 789–791.

\bibitem{rubin1970} 
Rubin, V. C., \& Ford, W. K. (1970). 
\textit{Rotation of the Andromeda Nebula from a Spectroscopic Survey of Emission Regions}. 
Astrophysical Journal, 159, 379–403.

\bibitem{weber1846} 
Weber, W. (1846). 
\textit{Elektrodynamische Maassbestimmungen}. 
Leipzig: Weidmannsche Buchhandlung.

\bibitem{bohm1952} 
Bohm, D. (1952). 
\textit{A Suggested Interpretation of the Quantum Theory in Terms of "Hidden" Variables}. 
Physical Review, 85(2), 166–193.

\bibitem{tisserand1894}
Tisserand, F. (1894). 
\textit{Traité de Mécanique Céleste, Tome IV}. 
Gauthier-Villars, Paris. 
(Kapitel 28: "Lois électrodynamiques de Weber appliquées à la gravitation")
\end{thebibliography}

\end{document}
