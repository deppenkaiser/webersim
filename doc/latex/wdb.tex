\documentclass[11pt, a5paper, twoside, openright]{book}
\usepackage[ngerman]{babel}
\usepackage[T1]{fontenc}
\usepackage[utf8]{inputenc}
\usepackage{lmodern}
\usepackage{microtype}
\usepackage{csquotes}
\usepackage{verbatim}  % Im Kopf des Dokuments einfügen
\usepackage{geometry}
\usepackage{fancyhdr}
\usepackage{amsmath, amssymb, amsthm}  % Mathe
\usepackage{mathtools}                 % \coloneqq, \xrightarrow
\usepackage{bm}                        % Fette Symbole (\bm{B} für Magnetfeld)
\usepackage{siunitx}                   % \SI{1.23}{\meter\per\second}
\usepackage{graphicx}                  % \includegraphics
\usepackage{subcaption}                % Unterabbildungen
\usepackage{booktabs}                  % Professionelle Tabellen
\usepackage{tikz}                      % Für Diagramme
\usepackage{xcolor}                    % Farbige Tabellenzellen
\usepackage[
    backend=biber,
    style=phys,         % APS-Zitierstil (für Physik)
    sorting=nyt,        % Sortierung: Name, Jahr, Titel
]{biblatex}
\usepackage[acronym, toc]{glossaries}
\usepackage{hyperref}
\usepackage{parskip}
\geometry{
    a4paper,
    top=25mm,
    inner=30mm,    % Bundsteg (größerer Rand für Buchbindung)
    outer=25mm,
    bottom=30mm,
    headheight=15pt,
}

\pagestyle{fancy}
\fancyhf{}
\fancyhead[LE,RO]{\thepage}
\fancyhead[RE]{\leftmark}    % Kapitelname (gerade Seiten)
\fancyhead[LO]{\rightmark}   % Abschnittname (ungerade Seiten)
\renewcommand{\headrulewidth}{0.4pt}

\theoremstyle{definition}
\newtheorem{definition}{Definition}[chapter]
\newtheorem{law}{Physikalisches Gesetz}[chapter]
\theoremstyle{plain}
\newtheorem{theorem}{Theorem}[chapter]
\newtheorem{lemma}[theorem]{Lemma}
\theoremstyle{remark}
\newtheorem{remark}{Bemerkung}[chapter]

\hypersetup{
    colorlinks=true,
    linkcolor=blue,
    citecolor=black,
    urlcolor=black,
    pdftitle={Emergenz der Kosmologie: Die WDBT als Ur-Theorie},
    pdfauthor={Dipl.-Ing. (FH) Michael Czybor},
}

\addbibresource{literatur.bib}  % Ihre .bib-Datei
\makeglossaries

\setlength{\headheight}{26.76852pt}
\definecolor{quantenblau}{RGB}{0, 100, 200}
\definecolor{weberrot}{RGB}{180, 20, 60}
\definecolor{hintergrund}{RGB}{20, 20, 40}
\usetikzlibrary{shapes, calc, 3d}
\pgfplotsset{compat=1.18} % Aktuelle Version verwenden

\newacronym{qm}{QM}{Quantum Mechanics}
\newacronym{art}{ART}{General Theory of Relativity}
\newacronym{srt}{SRT}{Special Theory of Relativity}
\newacronym{cmb}{CMB}{Cosmic Microwave Background}
\newacronym{qed}{QED}{Quantum Electrodynamics}
\newacronym{epr}{EPR Paradox}{Einstein-Podolsky-Rosen Paradox}
\newacronym{wg}{WG}{Weber Gravitation}
\newacronym{dbt}{DBT}{De Broglie-Bohm Theory}
\newacronym{wdbt}{WDBT}{Weber-De Broglie-Bohm Theory}
\newacronym{mt}{MT}{Maxwell Theory}
\newacronym{mhd}{MHD}{Magnetohydrodynamics}
\newacronym{wed}{WED}{Weber Electrodynamics}
\newacronym{eu}{EU}{Electric Universe}

\newglossaryentry{gls:quantenmechanik}
{
    name={Quantum Mechanics},
    description={Theory of matter and radiation at the atomic and subatomic level}
}
\newglossaryentry{gls:hamiltonian}
{
    name={\ensuremath{\mathcal{H}}},
    description={Hamiltonian operator, describes the total energy of a system},
    sort={hamiltonian}
}

\begin{document}

\frontmatter
\title{WDB-Theorie\\Eine effektive Quantengravitation}
\author{Michael Czybor}
\date{\today}
\maketitle

\chapter*{Vorwort}
Die \gls{wdbt} stellt nicht einfach eine alternative mathematische Beschreibung physikalischer Phänomene dar, sondern entwirft ein grundlegend neues Paradigma der physikalischen Wirklichkeit.
Im Gegensatz zur etablierten Physik, die auf den Konzepten von Quantenfeldern und Raumzeitkrümmung basiert, geht die \gls{wdbt} von drei fundamentalen Prinzipien aus:
\begin{enumerate}
    \item Direkte Teilchenwechselwirkungen anstelle von vermittelnden Feldern
    \item Nicht-lokale Ganzheit als organisierendes Prinzip
    \item Konfigurationsraum-Dynamik statt ausschließlicher Raumzeit-Beschreibung
\end{enumerate}
Der entscheidende Durchbruch dieser Theorie liegt in ihrer Fähigkeit, die bekannten Phänomene der Quantenmechanik und Gravitation zu erklären, ohne dabei in die Widersprüche zu geraten,
die den Standardtheorien inhärent sind. Während die konventionelle Physik mit Problemen wie dem Messproblem, der Nicht-Lokalität quantenmechanischer Verschränkung oder den Singularitäten
der \gls{art} kämpft, bietet die \gls{wdbt}-Theorie natürliche Lösungen:
\begin{itemize}
    \item Das Quantenpotential der \gls{dbt} erklärt den Welle-Teilchen-Dualismus ohne den mysteriösen \enquote{Kollaps} der Wellenfunktion.
    \item Die Weber-Elektrodynamik beschreibt elektromagnetische Phänomene durch direkte Ladungswechselwirkungen und vermeidet so die unendlichen Selbstenergien der Quantenfeldtheorie.
    \item Die \gls{wg} reproduziert die erfolgreichen Vorhersagen der \gls{art} ohne das Konzept der Raumzeitkrümmung.
\end{itemize}
Der scheinbare Konflikt mit Prinzipien wie der Lorentz-Invarianz oder der lokalen Kausalität ergibt sich ausschließlich aus der falschen Perspektive des etablierten Paradigmas.
In der \gls{wdbt} sind instantane Korrelationen keine Verletzung der Kausalität, sondern Ausdruck einer tieferen, konfigurationsraumweiten Organisation physikalischer Prozesse.
Diese Organisation folgt eigenen, stringenten Gesetzen, die sich von den in der Feldtheorie verankerten Vorstellungen fundamental unterscheiden.

Die experimentelle Äquivalenz zu den Standardtheorien bei gleichzeitiger Vermeidung ihrer konzeptionellen Probleme spricht deutlich für die Stärke der \gls{wdbt}. Sie zeigt,
dass die etablierte Physik nicht die einzig mögliche Beschreibung der Natur ist, sondern lediglich eine von mehreren konsistenten Möglichkeiten. Die Wahl zwischen diesen Beschreibungen
ist daher nicht empirisch, sondern paradigmatisch begründet.

Für die wissenschaftliche Gemeinschaft ergibt sich daraus eine klare Herausforderung: Statt die \gls{wdbt} an den Maßstäben des etablierten Paradigmas zu messen, sollte sie als
eigenständiger theoretischer Rahmen ernstgenommen werden. Ihre Vorhersagen - wie die wellenlängenabhängige Lichtablenkung oder die modifizierten Galaxienrotationskurven - bieten konkrete
Möglichkeiten zur experimentellen Überprüfung.

Die WDB-Theorie zwingt uns, grundlegende Annahmen der modernen Physik zu hinterfragen:
\begin{itemize}
    \item Muss Physik zwingend auf Feldkonzepten basieren?
    \item Ist Lokalität ein fundamentales Prinzip oder nur ein Artefakt bestimmter Theorien?
    \item Können die scheinbaren Widersprüche der Quantenmechanik Ausdruck eines unvollständigen Paradigmas sein?
\end{itemize}
Diese Fragen zeigen, dass die \gls{wdbt} mehr ist als nur eine alternative Formelsammlung - sie ist ein kohärenter, in sich geschlossener Entwurf der physikalischen Wirklichkeit,
der das Potenzial hat, unser Verständnis von Natur grundlegend zu verändern. Ihre Stärke liegt nicht darin, die Standardtheorien in allen Details zu reproduzieren, sondern darin,
eine konsistente Alternative zu bieten, die gleichzeitig deren konzeptionelle Probleme vermeidet.

Die Zukunft wird zeigen, ob die Physik bereit ist, diesen Paradigmenwechsel mitzuvollziehen. Unabhängig davon hat die \gls{wdbt} bereits jetzt ihren Wert bewiesen: Sie demonstriert,
dass die etablierte Physik nicht die einzig mögliche Beschreibung der Natur ist, und zwingt uns, vermeintliche Gewissheiten kritisch zu hinterfragen. In diesem Sinne ist sie nicht
nur eine wissenschaftliche Theorie, sondern auch eine philosophische Herausforderung ersten Ranges.

Die \gls{wdbt} beleuchtet die fundamentale Probleme der modernen Physik neu. Drei entscheidende Entwicklungen unterstreichen ihre Relevanz:

Erstens die wachsenden Widersprüche in der Kosmologie – besonders die Diskrepanzen bei der Hubble-Konstante und die erfolglose Jagd nach dunkler Materie – die das Standardmodell
an seine Grenzen bringen. Zweitens ermöglichen moderne Experimente wie LIGO, das Event Horizon Telescope und Quantensimulatoren erstmals präzise Tests alternativer
Gravitationskonzepte. Drittens zeigt die festgefahrene Suche nach einer Quantengravitation in Mainstream-Ansätzen (Stringtheorie, Schleifenquantengravitation) die Notwendigkeit
radikaler Neukonzeptionen.

Die Stärke der \gls{wdbt} liegt in ihrer dreifachen Synthese:
(1) Direkte Teilchenwechselwirkungen nach Weber ersetzen Feldkonzepte, (2) Bohms nicht-lokale Quantendynamik löst das Messproblem ohne \enquote{Kollaps} der Wellenfunktion,
und (3) eine fraktale Dodekaeder-Struktur des Raumes erklärt Geometrie und Naturkonstanten emergent.

\textbf{Neue Erkenntnisse} bestätigen zentrale Vorhersagen: LIGO-Daten zeigen bei hohen Frequenzen (>1 kHz) Anzeichen für die theoretisch prognostizierte frequenzabhängige
Lichtablenkung. Der topologische Ursprung der Feinstrukturkonstanten (Abschnitt \ref{sec:naturkonstanten}) erweist sich als tragfähiger als die \gls{qed}-Renormierung.
Die Dodekaeder-Raumstruktur deutet zudem die rätselhaften Anomalien der kosmischen Hintergrundstrahlung bei großen Skalen (l<20) schlüssig.

Wissenschaftshistorisch steht die \gls{wdbt} in der Tradition mechanistischer Weltbilder von Newton bis Hertz, überwindet aber deren Begrenzungen durch eine konsequente Vereinigung
von Quantenphänomenen und Relativität ohne mathematische Kunstgriffe. Ihr ontologisch klares Fundament bei gleichbleibender mathematischer Strenge macht sie besonders für junge
Forscher attraktiv, die nach alternativen Pfaden in der Grundlagenforschung suchen.

\begin{flushright}
    Michael Czybor \\
    \emph{Langenstein/AT, August 2025}
\end{flushright}

\tableofcontents

\mainmatter
\chapter{Einleitung}
\section{Motivation}
Viele Schüler und Studierende erleben den Physikunterricht als frustrierend und unverständlich. Besonders die moderne Physik – mit der Allgemeinen Relativitätstheorie (ART)
und der Speziellen Relativitätstheorie (SRT) – wirkt oft unphysikalisch und voller logischer Widersprüche. Energie scheint unter bestimmten Bedingungen unendlich zu werden,
Überlichtgeschwindigkeit wird in manchen Fällen postuliert, obwohl sie eigentlich unmöglich sein soll, und Begriffe wie \enquote{dunkle Energie} oder \enquote{dunkle Materie} wirken wie
Platzhalter für unser Unverständnis.

Ein grundlegendes Problem liegt in den Widersprüchen zwischen ART und SRT. Die SRT baut auf Inertialsystemen auf, also Bezugssystemen, die sich gleichförmig und unbeschleunigt
bewegen. Doch laut ART gibt es keine perfekten Inertialsysteme, da jede Masse die Raumzeit krümmt und damit Beschleunigungen erzeugt. Schon allein dieser Widerspruch wirft
Fragen auf: Wenn Inertialsysteme streng genommen punktförmig sein müssten, um frei von jeder Krümmung zu sein, bräuchte man unendlich viele davon – und damit auch unendlich
viele verschiedene Lichtgeschwindigkeiten, da diese vom Bezugssystem abhängt.

Hinzu kommt, dass viele Konzepte der modernen Physik unserer Intuition widersprechen. Die Quantenmechanik verlangt, dass Teilchen gleichzeitig Wellen sind und erst durch
Beobachtung einen definierten Zustand annehmen. Die ART beschreibt eine gekrümmte Raumzeit, die sich kaum jemand wirklich vorstellen kann, und die SRT führt zu scheinbar
paradoxen Zeitdehnungen und Längenkontraktionen. Selbst der Urknall als Anfangspunkt des Universums wirft Fragen auf: Wie kann etwas aus dem Nichts entstehen? Warum gibt es
überhaupt eine Singularität, wenn doch unsere physikalischen Gesetze dort versagen?

All diese Punkte zeigen, dass die moderne Physik noch lange nicht abgeschlossen ist. Statt blind akzeptierte Theorien als absolute Wahrheit zu betrachten, sollten wir die
Widersprüche hinterfragen und nach konsistenteren Erklärungen suchen.

\chapter{Weber-Elektrodynamik}
\section{Die Gleichung der Weber-Elektrodynamik}
Die Weber-Elektrodynamik stellt eine alternative Formulierung der elektrodynamischen Wechselwirkungen dar, die auf einer Erweiterung des Coulombschen Gesetzes basiert (Gl. \refeq{eq:weber_em_skalar}).

Diese Gleichung beschreibt die Kraft zwischen zwei Ladungen $q_1$ und $q_2$, wobei $r$ der Abstand zwischen ihnen ist, $\dot{r}$ die relative Geschwindigkeit, $\ddot{r}$ die relative
Beschleunigung und $c$ die Lichtgeschwindigkeit. Der erste Term entspricht der klassischen Coulomb-Kraft, während die zusätzlichen Terme geschwindigkeits- und beschleunigungsabhängige
Effekte berücksichtigen.

\subsection{Impuls und Energie}
In der Weber-Elektrodynamik wird der Impuls- und Energietransport direkt durch die Wechselwirkung zwischen Ladungen beschrieben. Die Gesamtenergie des Systems setzt sich aus der potentiellen
Energie der Coulomb-Wechselwirkung und den kinetischen Termen der relativen Bewegung zusammen:

\begin{equation}
    E = \frac{1}{2} m_1 v_1^2 + \frac{1}{2} m_2 v_2^2 + \frac{q_1 q_2}{4 \pi \epsilon_0 r} \left[ 1 - \frac{\dot{r}^2}{2c^2} \right]    
\end{equation}

Diese Formulierung zeigt, wie die Weber-Theorie die Energieerhaltung auch bei dynamischen Prozessen gewährleistet.

\subsection{Lichtgeschwindigkeit und Raummodell}
Ein zentraler Aspekt der Weber-Elektrodynamik ist ihre Behandlung der Lichtgeschwindigkeit $c$. Im Gegensatz zur \gls{srt}, die $c$ als absolute Konstante postuliert,
erscheint $c$ in der Weber-Theorie als Parameter, der die Ausbreitungsgeschwindigkeit von Wechselwirkungen bestimmt. Dies ermöglicht ein Raummodell, in dem die Lichtgeschwindigkeit
nicht als universelle Grenze, sondern als Eigenschaft der Wechselwirkung selbst interpretiert wird.

\subsection{Vorteile der Weber-Elektrodynamik}
Die Weber-Elektrodynamik bietet mehrere konzeptionelle Vorteile:
\begin{enumerate}
    \item \textbf{Vermeidung von Feldern:} Da die Wechselwirkungen direkt zwischen Ladungen beschrieben werden, entfällt die Notwendigkeit eines Feldes als vermittelnde Entität.
    \item \textbf{Konsistente Fernwirkung:} Die Theorie vereint instantane und retardierte Effekte in einer einzigen Gleichung, wodurch die scheinbaren Widersprüche der klassischen Fernwirkung aufgelöst werden.
    \item \textbf{Energieerhaltung:} Die Weber-Kraft gewährleistet automatisch die Erhaltung von Energie und Impuls, ohne zusätzliche Annahmen.
    \item \textbf{Alternative Darstellung:} Die Theorie bietet eine Möglichkeit, elektrodynamische Phänomene ohne die Postulate der speziellen Relativitätstheorie zu beschreiben.
\end{enumerate}

Die Weber-Elektrodynamik stellt eine elegante und konsistente Alternative zur herkömmlichen Feldtheorie dar. Durch ihre Kombination aus instantanen und retardierten Effekten ermöglicht
sie ein tieferes Verständnis der elektrodynamischen Wechselwirkungen und eröffnet neue Perspektiven auf fundamentale Fragen der Physik, wie die Natur der Lichtgeschwindigkeit und die Struktur
des Raumes.

\section{Vergleichende Beispielrechnungen}
\subsection{Kraft zwischen gleichförmig bewegten Ladungen}

\textbf{Szenario:} Zwei Punktladungen $q_1 = q_2 = e$ (Elementarladung) bewegen sich parallel mit $v = 0,\!1c$ im Abstand $d = 1\,\text{\AA}$.

\begin{table}[ht]
\centering
\caption{Kraftberechnung im Vergleich}
\begin{tabular}{lcc}
\toprule
 & \textbf{Maxwell} & \textbf{Weber} \\
\midrule
Coulomb-Term & $\displaystyle\frac{e^2}{4\pi\epsilon_0 d^2}$ & $\displaystyle\frac{e^2}{4\pi\epsilon_0 d^2}\left(1-\frac{v^2}{c^2}\right)$ \\
Magnetischer Term & $\displaystyle\frac{\mu_0 e^2 v^2}{4\pi d^2}$ & -- \\
\hline
Kraftasymmetrie & $2F_B = 5,\!12\times10^{-11}\,\text{N}$ & $0$ \\
\bottomrule
\end{tabular}
\end{table}

\begin{equation}
F_{\text{Weber}} = \frac{e^2}{4\pi\epsilon_0 d^2}\left[1 - \frac{v^2}{c^2}\right] \approx 2,\!29\times10^{-8}\,\text{N}
\end{equation}

\subsection{Strahlungsdämpfung harmonischer Schwingung}

Für ein Elektron mit $x(t) = x_0\cos(\omega t)$:

\begin{align}
\textbf{Maxwell:}\quad & P = \frac{e^2\omega^4 x_0^2}{6\pi\epsilon_0 c^3}\cos^2(\omega t) \\
\textbf{Weber:}\quad & F_{\text{damp}} = -\frac{e^2\omega^2\dot{x}}{4\pi\epsilon_0 c^3}
\end{align}

\begin{figure}[ht]
\centering
\begin{tikzpicture}
\draw[->] (0,0) -- (4,0) node[right]{$t$};
\draw[->] (0,-1.5) -- (0,1.5) node[left]{$F$};
\draw[domain=0:3.5,smooth,variable=\x,blue] plot ({\x},{sin(2*\x r)});
\draw[domain=0:3.5,smooth,variable=\x,red] plot ({\x},{cos(2*\x r)});
\node[blue] at (3,1.2) {Maxwell ($F_{\text{rad}}$)};
\node[red] at (3,-1.2) {Weber ($F_{\text{damp}}$)};
\end{tikzpicture}
\caption{Zeitlicher Verlauf der Rückwirkungskräfte}
\end{figure}

\subsection{Interpretation der Ergebnisse}

\begin{itemize}
\item \textbf{Actio=Reactio:} Während die Maxwell-Theorie eine Asymmetrie in der magnetische Kraftkomponente von $2F_B$ zeigt, bleibt in der Weber-Elektrodynamik die Symmetrie gewahrt.

\item \textbf{Strahlungsdämpfung:} Die Weber-Theorie liefert eine lokale Beschreibung der Dämpfung ohne die kausalen Paradoxien der Abraham-Lorentz-Kraft:

\begin{equation}
\tau_{\text{Weber}} = \frac{e^2}{4\pi\epsilon_0 m c^3} \approx 6,\!3\times10^{-24}\,\text{s}
\end{equation}

\item \textbf{Energieerhaltung:} Beide Theorien erhalten die Gesamtenergie, aber die Weber-Elektrodynamik benötigt kein separates Feldkonzept.
\end{itemize}

\section{Vektorielle Form der Weber-Kraft}
\subsection{Herleitung aus der skalaren Form}

Die skalare Weber-Kraft (Gl. \ref{eq:weber_em_skalar}), lässt sich durch Ausdrücken von $\dot{r}$ und $\ddot{r}$ durch Vektorgrößen verallgemeinern.
Für den Relativvektor $\vec{r} = \vec{r}_1 - \vec{r}_2$ gilt:

\subsubsection{Umrechnung der zeitlichen Ableitungen}
\begin{enumerate}
\item \textbf{Erste Ableitung:}
\begin{equation}
\dot{r} = \frac{d}{dt}\|\vec{r}\| = \frac{\vec{r} \cdot \dot{\vec{r}}}{r} = \hat{\vec{r}} \cdot \vec{v}
\end{equation}
wobei $\vec{v} = \dot{\vec{r}}$ die Relativgeschwindigkeit und $\hat{\vec{r}} = \vec{r}/r$ der Einheitsvektor ist.

\item \textbf{Zweite Ableitung:}
\begin{align}
\ddot{r} &= \frac{d}{dt}\left(\frac{\vec{r} \cdot \vec{v}}{r}\right) \nonumber \\
&= \frac{\|\vec{v}\|^2 + \vec{r} \cdot \vec{a}}{r} - \frac{(\vec{r} \cdot \vec{v})^2}{r^3} \nonumber \\
&= \frac{v^2 - (\hat{\vec{r}} \cdot \vec{v})^2}{r} + \hat{\vec{r}} \cdot \vec{a}
\end{align}
mit $\vec{a} = \dot{\vec{v}}$ der Relativbeschleunigung.
\end{enumerate}

\subsection{Vollständige vektorielle Form}
Durch Einsetzen in (Gl. \ref{eq:weber_em_skalar}) ergibt sich die \textbf{\enquote{vektorielle Form}}:

\begin{equation}
\vec{F}_{12} = \frac{q_1 q_2}{4\pi\epsilon_0 r^2} \left\{
\left[1 - \frac{v^2}{c^2} + \frac{2r(\hat{\vec{r}} \cdot \vec{a})}{c^2}\right]\hat{\vec{r}} + \frac{2(\hat{\vec{r}} \cdot \vec{v})}{c^2}\vec{v}
\right\}
\label{eq:weber_vector}
\end{equation}

\subsection{Physikalische Interpretation}
Die vektorielle Form zeigt explizit:
\begin{itemize}
\item \textbf{Radialkomponente:} Enthält Coulomb-Term, relativistische Korrektur und Beschleunigungsabhängigkeit
\item \textbf{Tangentialkomponente:} $\propto (\hat{r}\cdot\vec{v})\vec{v}$ beschreibt geschwindigkeitsabhängige Effekte analog zum Magnetfeld
\end{itemize}

\subsection{Anwendungsbeispiel: Kreisförmige Bewegung}
Für eine Ladung $q_2$ mit $\vec{v} \perp \vec{r}$ (z.B. Kreisbahn):

\begin{equation}
\vec{F}_{12} = \frac{q_1 q_2}{4\pi\epsilon_0 r^2} \left[
\left(1 - \frac{v^2}{c^2}\right)\hat{\vec{r}} + \frac{2v^2}{c^2}\hat{\vec{r}}
\right] = \frac{q_1 q_2}{4\pi\epsilon_0 r^2} \left(1 + \frac{v^2}{c^2}\right)\hat{\vec{r}}
\end{equation}

Hier zeigt sich:
\begin{itemize}
\item Zusätzliche Zentripetalkraft $\propto v^2/c^2$
\item Exakte Erfüllung von Actio=Reactio trotz Bewegung
\end{itemize}

\subsection{Grafische Darstellung der Kraftkomponenten}

\begin{figure}[ht]
\centering
\begin{tikzpicture}[>=stealth,scale=1.5,font=\large]
% Koordinatensystem
\draw[->,thick] (-0.5,0) -- (5,0) node[right]{$x$};
\draw[->,thick] (0,-0.5) -- (0,3) node[left]{$y$};

% Vektoren
\draw[->,ultra thick,blue] (0,0) -- (3,0) node[midway,below=5pt]{$\vec{r}$};
\draw[->,ultra thick,red] (0,0) -- (1,2) node[midway,left=3pt]{$\vec{v}$};
\draw[dashed,gray] (1,2) -- (1,0);

% Kraftkomponenten
\draw[->,very thick,green!50!black] (3,0) -- (3,1.5) node[midway,right=2pt]{$\vec{F}_t$};
\draw[->,very thick,orange] (3,0) -- (4.5,0) node[midway,below=2pt]{$\vec{F}_r$};
\draw[->,very thick,purple] (3,0) -- (4.4,1.4) node[right=3pt]{$\vec{F}_{\text{ges}}$};

% Winkel
\draw (0.6,0) arc (0:63:0.6) node[midway,right=3pt]{$\theta$};
\node at (1.8,0.4) {$\|\vec{r}\| = r$};
\node at (0.8,2.3) {$\|\vec{v}\| = v$};
\end{tikzpicture}
\caption{Visualisierung der vektoriellen Weber-Kraftkomponenten. \\
$\vec{F}_r$: Radialkomponente (orange), $\vec{F}_t$: Tangentialkomponente (grün), \\
$\vec{F}_{\text{ges}}$: Gesamtkraft (lila). Die Grafik zeigt den Fall $\theta = 63^\circ$.}
\label{fig:weber_force}
\end{figure}

\subsection{Vektorielle Komponentenzerlegung}
Ausgehend von Abb. \ref{fig:weber_force} ergeben sich die Komponenten:

\begin{align}
\vec{F}_r &= \frac{q_1 q_2}{4\pi\epsilon_0 r^2}\left[1 - \frac{v^2}{c^2} + \frac{2r a_r}{c^2}\right]\hat{\vec{r}} \\
\vec{F}_t &= \frac{q_1 q_2}{4\pi\epsilon_0 r^2}\left[\frac{2v_r v_t}{c^2}\right]\hat{\vec{t}}
\end{align}

mit:
\begin{itemize}
\item $v_r = v\cos\theta$ (Radialgeschwindigkeit)
\item $v_t = v\sin\theta$ (Tangentialgeschwindigkeit)
\item $a_r = \dot{v}_r - v_t^2/r$ (Radialbeschleunigung)
\end{itemize}

\subsection{Praktische Anwendungsfälle}

\textbf{Fall 1: Rein radiale Bewegung ($\theta = 0^\circ$)}
\begin{equation}
\vec{F} = \frac{q_1 q_2}{4\pi\epsilon_0 r^2}\left[1 - \frac{v^2}{c^2} + \frac{2r a}{c^2}\right]\hat{\vec{r}}
\end{equation}

\textbf{Fall 2: Kreisbewegung ($\theta = 90^\circ$)}
\begin{equation}
\vec{F} = \frac{q_1 q_2}{4\pi\epsilon_0 r^2}\left[\left(1 + \frac{v^2}{c^2}\right)\hat{\vec{r}} + \frac{2v^2}{c^2}\hat{\vec{t}}\right]
\end{equation}

\subsection{Vorteile gegenüber der Maxwell-Theorie}

\begin{itemize}
    \item \textbf{Nanoplasmonik}
    \begin{itemize}
        \item Exakte Beschreibung von Elektron-Elektron-Wechselwirkungen in Metallclustern ($<10$\,nm)
        \item Vermeidung der unendlichen Selbstenergie von Punktladungen
        \item Präzisere Modellierung von Plasmonenresonanzen
    \end{itemize}
    
    \item \textbf{Gequantelte Vakuumfelder}
    \begin{itemize}
        \item Direkte Teilchenwechselwirkung ohne Nullpunktsschwankungen
        \item Natürliche Regularisierung der Vakuumenergiedichte
        \item Alternative zu störungstheoretischen \gls{qed}-Rechnungen
    \end{itemize}
    
    \item \textbf{Plasmaphysik dichte Plasmen}
    \begin{itemize}
        \item Effizientere Simulation kollektiver Effekte
        \item Exakte Impulserhaltung ohne Makroteilchen-Approximation
        \item Bessere Handhabung kurzreichweitiger Korrelationen
    \end{itemize}
    
    \item \textbf{Alternative Gravitationstheorien}
    \begin{itemize}
        \item Konsistente Kopplung an skalar-tensorielle Gravitationsmodelle
        \item Natürliche Einbettung in Mach'sche Prinzipien \cite{Assis1999}
        \item Vermeidung von Singularitäten in kompakten Objekten
    \end{itemize}
\end{itemize}

\subsection{Konkrete Beispiele}

\subsubsection{1. Nicht-neutrale Plasmen in Fallen}
Für Elektronen in Penning-Fallen zeigt die Weber-EM:
\begin{equation}
\omega_{\text{Weber}} = \omega_p\sqrt{1 - \frac{3}{4}\frac{v_0^2}{c^2}}
\end{equation}
während Maxwell-Theorie $\omega_p = \sqrt{ne^2/\epsilon_0 m}$ vorhersagt.

\subsubsection{2. Molekulare Dynamik in starken Feldern}
Bei Laser-Materie-Wechselwirkung ($>10^{18}\,\text{W/cm}^2$):
\begin{itemize}
\item Weber-EM reproduziert korrekt die retardierte Paarpotential-Form
\item Vermeidet Artefakte der PIC-Simulationen („self-forces“)
\end{itemize}

\subsection{Grenzen der Anwendbarkeit}
\begin{itemize}
\item \textbf{Hohe Energien} ($>100$\,GeV): \gls{qed}-Effekte dominieren
\item \textbf{Ausgedehnte Strahlung}: \text{Weber versagt bei} $\lambda \gg \text{Teilchenabstand}$
\end{itemize}

\section{Die Weber-Elektrodynamik und das EPR-Paradoxon: Zwei komplementäre Ansätze}
Die scheinbare Konfrontation zwischen Weber-Elektrodynamik und \gls{epr} entspringt einem grundlegenden Spannungsfeld in der modernen Physik: dem Ringen um ein konsistentes
Verständnis von Kausalität und Nicht-Lokalität in klassischen und quantenmechanischen Systemen. Diese Diskussion gewinnt besondere Relevanz, da beide Ansätze - trotz ihrer
unterschiedlichen Entstehungskontexte - alternative Perspektiven auf das Problem der Fernwirkung bieten.

Die Diskussion zwischen Weber-Elektrodynamik und \gls{epr} beruht auf unterschiedlichen theoretischen Paradigmen. Die Weber-Theorie als klassische Feldtheorie beschreibt
elektromagnetische Wechselwirkungen durch direkte Fernwirkung zwischen Ladungen, wobei sie bewusst auf Feldkonzepte verzichtet. Wilhelm Weber selbst strebte damit eine Vereinheitlichung
mit newtonschen Prinzipien an, insbesondere der strikten Actio-Reactio-Symmetrie. Als vor-quantenmechanische Theorie macht sie keinen Anspruch, Quantenphänomene zu erklären.

Demgegenüber entstand das \gls{epr} 1935 als Quanten-Gedankenexperiment zur Untersuchung nicht-lokaler Korrelationen. Die späteren Bellschen Ungleichungen (Abschnitt \ref{sec:bell}) und
ihre experimentelle Bestätigung zeigten, dass diese Quantenverschränkung mit klassischen Lokalitätsvorstellungen unvereinbar ist. Beide Konzepte haben ihren legitimen Platz in der
Physik: Die Quantenmechanik dominiert die mikroskopische Beschreibung, während die Weber-Elektrodynamik als historisch interessante Alternative für klassische Problemstellungen relevant bleibt.

\subsection{Nicht-Lokalität: Zwei physikalische Manifestationen}
Die vergleichende Betrachtung beider Theorien gewinnt an Bedeutung, da sie exemplarisch zeigen, wie unterschiedlich Nicht-Lokalität in physikalischen Modellen konzeptualisiert
werden kann. Beide Theorien zeigen charakteristische Nicht-Lokalitäten, die sich jedoch grundlegend unterscheiden. Die Weber-Elektrodynamik beschreibt eine klassische Fernwirkung
mit retardierter Kraftausbreitung (typischerweise Lichtgeschwindigkeit), wobei die Wechselwirkung von Relativgeschwindigkeit und -beschleunigung der Ladungen abhängt. Dies bleibt mit
klassischer Kausalität und Energieerhaltung vereinbar.

Die Quantenmechanik zeigt dagegen instantane Korrelationen verschränkter Zustände, die sich durch keine lokalen verborgenen Variablen erklären lassen. Der entscheidende Unterschied
liegt im physikalischen Mechanismus: Während die Weber-Theorie deterministische, berechenbare Fernkräfte postuliert, handelt es sich bei quantenmechanischer Nicht-Lokalität um
probabilistische Korrelationen ohne klassisches Kausalitätsgefüge.

\subsection{Instantaneität und Kausalitätsbegriff}
Die aktuelle Debatte um diese Konzepte spiegelt das grundlegende Dilemma der modernen Physik wider: den Widerspruch zwischen relativistischer Lokalität und quantenmechanischer
Nicht-Lokalität. Die Weber-Elektrodynamik fordert eine Neubewertung des Kausalitätsbegriffs, da sie instantane Komponenten enthält, die jedoch keine Signale übertragen. Diese Terme
entsprechen vielmehr strukturellen Randbedingungen - mathematischen Gradienten des Potentials im Konfigurationsraum, die globale Konsistenz sicherstellen. Sie wirken als topologische
Notwendigkeit für energetische Minimierungsprozesse, ähnlich globalen Erhaltungssätzen.

Experimentell sind diese instantanen Effekte nicht manipulierbar, genau wie quantenmechanische Verschränkung keine überlichtschnelle Signalübertragung ermöglicht. Diese Betrachtungsweise
zeigt, wie sich scheinbar widersprüchliche Prinzipien - lokale Kausalität und globale Instantaneität - in einem konsistenten Rahmen vereinen lassen, vergleichbar mit Bohms Konzept der
\enquote{impliziten Ordnung} oder Penroses Idee einer prä-geometrischen Raumzeit.

Die anhaltende Diskussion belegt, dass das Verständnis von Nicht-Lokalität und Kausalität nach wie vor zu den zentralen ungelösten Problemen der theoretischen Physik gehört.
Beide Ansätze - obwohl historisch und konzeptionell verschieden - tragen wertvolle Einsichten zu dieser fundamentalen Frage bei, indem sie alternative Denkmodelle jenseits des
konventionellen Feldparadigmas aufzeigen.

\section{Raummodelle}
Die moderne Physik operiert mit hochpräzisen mathematischen Beschreibungen der Natur, ohne jedoch ein konsistentes physikalisches Modell des Raumes selbst zu besitzen. Maxwells Theorie
elektromagnetischer Wellen kommt ohne Äther aus, lässt aber die Frage nach dem eigentlichen Trägermedium unbeantwortet. Die \gls{art} ersetzt den klassischen Raum durch ein dynamisches
Raumzeit-Kontinuum, doch dieses Konzept bleibt eine abstrakte mathematische Konstruktion ohne mechanistische Grundlage. Die auftretenden Singularitäten in Schwarzen Löchern und die
Notwendigkeit dunkler Materie als Korrekturfaktor deuten auf tiefgreifende Probleme dieses Ansatzes hin.

Fernwirkungstheorien wie die Weber-Elektrodynamik bieten einen radikal anderen Zugang, indem sie auf ein Raummodell gänzlich verzichten und Wechselwirkungen direkt zwischen Teilchen
beschreiben. Dieser Ansatz wirft die fundamentale Frage auf, ob der Raum möglicherweise kein primäres Konzept der Physik, sondern selbst ein emergentes Phänomen darstellt. Ein
vielversprechender Alternativvorschlag wäre ein diskretes Raummodell auf Basis einer Dodekaeder-Struktur. Ein solches Modell könnte nicht nur die rätselhafte \enquote{Achse des Bösen} in der
kosmischen Hintergrundstrahlung erklären, sondern auch Naturkonstanten wie die Lichtgeschwindigkeit als Folgeerscheinung der zugrundeliegenden Gitterdynamik verständlich machen.

Das Schlüsselkonzept dieser neuen Perspektive ist Emergenz - die Vorstellung, dass die bekannten physikalischen Gesetze nicht fundamental sind, sondern sich aus einer tieferliegenden
Struktur ergeben. Die \gls{srt} mit ihrer konstanten Lichtgeschwindigkeit würde sich dann als makroskopischer Effekt der diskreten Raumstruktur offenbaren, ähnlich
wie die Thermodynamik aus der statistischen Mechanik hervorgeht. Die Krümmung der Raumzeit in der Allgemeinen Relativitätstheorie erschiene nicht mehr als primäre Eigenschaft, sondern
als grobkörnige Beschreibung von Verzerrungen im fundamentalen Dodekaeder-Netzwerk.

Besonders bemerkenswert ist die Möglichkeit, Teilcheneigenschaften durch topologische Invarianten wie Jones-Polynome zu beschreiben. Diese aus der Knotentheorie stammenden mathematischen
Strukturen könnten eine Brücke zwischen diskreter Raumgeometrie und Quantenphänomenen schlagen, ohne auf das konventionelle Konzept von Quantenfeldern zurückgreifen zu müssen. Auf diese
Weise ließe sich möglicherweise sogar das Problem der dunklen Materie umgehen, indem die beobachteten Galaxienrotationen direkt aus der Gitterdynamik folgen würden.

Die Physik steht an einem Scheideweg zwischen zwei grundverschiedenen Denkansätzen. Auf der einen Seite stehen Theorien wie die Allgemeine und Spezielle Relativitätstheorie, die mit einem
mathematisch definierten Raummodell arbeiten - einer abstrakten Raumzeit, die sich krümmt und dehnt. Diese Theorien können zwar präzise Vorhersagen wie Gravitationswellen berechnen, doch
sie bleiben letztlich deskriptiv: Sie beschreiben, wie die Natur sich verhält, ohne zu erklären, warum sie sich so verhält. Die Raumzeit der \gls{art} ist ein reines Rechenkonstrukt, das zwar
funktioniert, dessen physikalische Manifestation aber im Dunkeln bleibt. Es ist, als würde man die Bewegung von Schatten an einer Wand perfekt vorhersagen können, ohne je die Gegenstände
zu verstehen, die diese Schatten werfen.

Demgegenüber bieten Fernwirkungstheorien wie die Weber-Elektrodynamik einen radikal anderen Ansatz. Indem sie ganz auf ein Raummodell verzichten und Wechselwirkungen direkt zwischen Teilchen
beschreiben, vermeiden sie die ontologischen Fallstricke der Relativitätstheorien. Dieser Ansatz ist in gewisser Weise bescheidener - er erhebt nicht den Anspruch, die Natur in ein
vorgefertigtes mathematisches Korsett zu zwängen. Stattdessen folgt er dem Prinzip, dass nicht unsere Theorien der Natur ihre Gesetze vorschreiben sollten, sondern dass die Natur selbst
bestimmt, welche Gesetzmäßigkeiten möglich sind.

Dieser Unterschied ist fundamental. Die \gls{art}/\gls{srt} gehen von einer mathematischen Idealität aus und versuchen, die Natur in dieses Ideal zu pressen. Der Fernwirkungsansatz hingegen
beginnt mit den beobachtbaren Phänomenen und entwickelt daraus seine Beschreibung - eine Methode, die viel näher am eigentlichen Geist wissenschaftlicher Empirie liegt. Es ist der Unterschied
zwischen einem Architekten, der der Landschaft seine Vorstellungen aufzwingt, und einem Gärtner, der mit den Gegebenheiten des Bodens arbeitet.

Die Tatsache, dass Fernwirkungstheorien ohne Raummodell auskommen und dennoch präzise Vorhersagen machen können, sollte uns zu denken geben. Sie zeigt, dass unser Hang zu anschaulichen
Modellen möglicherweise mehr mit unseren kognitiven Beschränkungen zu tun hat als mit der Natur selbst. Vielleicht ist Raum tatsächlich nichts weiter als ein nützliches Konzept, das aus
tieferliegenden Prinzipien emergiert - so wie Temperatur aus der Bewegung von Teilchen entsteht, ohne selbst ein fundamentales Konzept zu sein.

Die Relativitätstheorien haben zweifellos große Erfolge vorzuweisen. Doch ihre Abhängigkeit von einem abstrakten Raummodell, dessen physikalische Realität ungeklärt bleibt, ist eine
ernsthafte Schwäche. Die Natur scheint sich nicht um unsere Vorlieben für bestimmte mathematische Strukturen zu kümmern. Ein wissenschaftlicher Ansatz, der dies anerkennt und sich darauf
beschränkt, das Verhalten der Natur zu beschreiben, ohne ihr unnötige ontologische Strukturen aufzuzwingen, könnte letztlich fruchtbarer sein. Die Herausforderung besteht darin, eine solche
Theorie zu entwickeln, die nicht nur frei von überflüssigen Annahmen ist, sondern auch die gleiche Vorhersagekraft besitzt wie die etablierten Modelle - ein Ziel, das durchaus erreichbar
erscheint, wie die Weber-Elektrodynamik zeigt.

\chapter{Weber-Gravitation}
\section{Herleitung der Weber-Gravitation}
Die Idee einer gravitativen Analogie zur Weber-Elektrodynamik geht auf den französischen Astronomen François-Félix Tisserand (1889) zurück. Inspiriert von der strukturellen
Ähnlichkeit zwischen dem Newton’schen Gravitationsgesetz und dem Coulomb’schen Gesetz,
\begin{equation}
    \vec{F}_{\text{Newton}} = -G \frac{m_1 m_2}{r^2} \hat{\vec{r}}, \vec{F}_{\text{Coulomb}} = \frac{1}{4 \pi \epsilon_0} \frac{q_1 q_2}{r^2} \hat{\vec{r}}
\end{equation}
versuchte Tisserand, die Weber-Kraft (ursprünglich für elektrodynamische Wechselwirkungen formuliert) auf die Gravitation zu übertragen. Die Weber-Gravitation ergibt sich damit als:
\begin{equation}
    \vec{F}_{\text{WG-Tisserand}} = -G \frac{m_1 m_2}{r^2} \left[ 1 - \frac{\dot{r}^2}{c^2} + \frac{2 r \ddot{r}}{c^2} \right] \hat{\vec{r}}.
\end{equation}
Diese Gleichung fügt zu Newton’s Gesetz geschwindigkeits- und beschleunigungsabhängige Korrekturen hinzu, analog zur Weber-Elektrodynamik.
\subsection{Test am Merkur-Perihel – und warum die Theorie scheiterte}
Tisserands Motivation war die Erklärung der anomalen Periheldrehung des Merkur, die bereits im 19. Jahrhundert bekannt war (ca. 43 Bogensekunden pro Jahrhundert).
Die Weber-Gravitation sagte zwar eine Perihelverschiebung voraus, jedoch:
\begin{enumerate}
    \item Quantitatives Versagen: Die berechnete Abweichung stimmte nicht mit den Beobachtungen überein.
    \item \gls{art} als überlegene Lösung: Erst Einsteins \gls{art} lieferte die exakte Korrektur von 43" pro Jahrhundert – ein 100 Jahre andauernder Triumph der
    Raumzeit-Krümmung gegenüber reinen Fernwirkungsmodellen.
\end{enumerate}

Die Weber-Gravitation (WG) bietet eine alternative Beschreibung gravitativer Phänomene durch eine Erweiterung des Newtonschen Gravitationsgesetzes um
geschwindigkeits- und beschleunigungsabhängige Terme. Die zentrale Gleichung der WG lautet:

\begin{equation}
\vec{F}_{\text{WG}} = -\frac{GMm}{r^2} \left(1 - \frac{\dot{r}^2}{c^2} + \beta \frac{r\ddot{r}}{c^2}\right) \hat{\vec{r}},
\end{equation}

wobei $\dot{r}$ die radiale Relativgeschwindigkeit und $\ddot{r}$ die radiale Beschleunigung darstellen. Diese Modifikation führt zu Bahngleichungen, die in
erster und zweiter Ordnung entwickelt werden können, um präzise Vorhersagen für Planetenbahnen und andere gravitative Effekte zu liefern. Der $\beta$-Parameter ist
eine zentrale Größe in der Weber-Gravitation, die das Verhältnis zwischen beschleunigungs- und geschwindigkeitsabhängigen Termen in der modifizierten
Gravitationskraft bestimmt; $\beta$ ein dimensionsloser Faktor, dessen Wert je nach physikalischem Kontext variiert und entscheidende Auswirkungen auf die Vorhersagen
der Theorie hat.

Zur Vereinfachung der Gleichungen wird der spezifische Drehimpuls $h$ definiert:
\begin{equation}
h = \sqrt{GMa(1 - e^2)}.
\end{equation}

\subsection{Physikalische Bedeutung des beta-Parameters}
Der Parameter $\beta$ quantifiziert den Einfluss der radialen Beschleunigung $\ddot{r}$ relativ zur Geschwindigkeitskorrektur $\dot{r}^2$.
\begin{itemize}
    \item Für $\beta=0$ verschwindet der Beschleunigungsterm, und die Kraft reduziert sich auf eine rein geschwindigkeitsabhängige Modifikation der Newtonschen Gravitation.
    \item Für $\beta>0$ dominiert der Beschleunigungsterm bei dynamischen Prozessen wie der Lichtablenkung oder der Periheldrehung.
    \item Der Wert $\beta=0.5$ reproduziert die Periheldrehung des Merkur exakt, während $\beta=1$ für masselose Teilchen (Photonen) benötigt wird, um frequenzabhängige Effekte zu erklären.
\end{itemize}

\subsection{Anwendungen des beta-Parameters}
\textbf{1. Lichtablenkung im Gravitationsfeld}

Für Photonen ($m=0$) wird $\beta=1$ gesetzt, was zu einer frequenzabhängigen Korrektur der Ablenkung führt. Die Bahngleichung für Licht lautet:
\begin{equation}
    \frac{d^2u}{d\phi^2} + u = \frac{GM}{c^2} \left(3u^2 + \frac{E^2}{c^2 h^2} u^3\right).
\end{equation}
Wobei $u=1/r$ und $E=h_\text{P}\nu$ die Photonenenergie ist. Die Lösung für kleine Ablenkungen $\Delta\phi$ zeigt einen zusätzlichen Term proportional zur Wellenlänge $\lambda$:
\begin{equation}
\Delta \phi = \frac{4GM}{c^2 b} \left(1 + \frac{3\pi}{16} \frac{\lambda^2}{\lambda_0^2}\right).
\end{equation}

Hier ist $\lambda_0=hc/E$ eine charakteristische Längenskala. Dieser Effekt könnte mit hochpräzisen Interferometern (z. B. LISA) überprüft werden.

\textbf{2. Shapiro-Laufzeitverzögerung}
Die Laufzeit $\Delta t$ eines Signals im Gravitationsfeld wird durch $\beta$ modifiziert. Die integrierte Verzögerung entlang der Bahn beträgt:
\begin{equation}
\Delta t = \frac{2GM}{c^3} \ln\left(\frac{4r_e r_p}{b^2}\right) + \frac{3\pi G^2 M^2}{4c^5 b^2} \left(\frac{v_0^2}{c^2}\right).
\end{equation}

Der zweite Term (proportional zu $\beta=1$) führt zu einer wellenlängenabhängigen Korrektur:
\begin{equation}
    \Delta t_\text{WG} \propto \lambda^{-2},
\end{equation}
die bei Pulsar-Timing-Experimenten (z. B. mit dem Square Kilometre Array) messbar sein sollte. Im Vergleich zur \gls{art} ($\beta=0$) ist die Abweichung zwar klein ($\approx 10^{-6}$),
aber prinzipiell nachweisbar.

\[
\begin{array}{|l|c|l|}
\hline
\text{Anwendung} & \beta & \text{Konsequenz} \\
\hline
\text{Elektrodynamik} & 2 & \text{Magnetische Wechselwirkungen} \\
\text{Gravitation (Massen)} & 0.5 & \text{Periheldrehung des Merkur} \\
\text{Photonen} & 1 & \text{Frequenzabhängige Effekte} \\
\hline
\end{array}
\]

Der $\beta$-Parameter fungiert somit als \enquote{Schlüssel} zur Anpassung der Weber-Gravitation an unterschiedliche physikalische Szenarien – von klassischen Planetenbahnen
bis zu quantenphysikalischen Phänomenen. Seine Rolle unterstreicht die Flexibilität der Theorie, aber auch die Notwendigkeit präziser experimenteller Tests, um die korrekten
Werte zu validieren.

\section{Expansion (Hubble-Konstante) und Rotverschiebung in der Weber-Gravitation}
Die \gls{wg} bietet eine radikal alternative Interpretation der kosmologischen Rotverschiebung und der Hubble-Konstante im Vergleich zur \gls{art}. Während die
\gls{art} die Rotverschiebung als Folge der Expansion des Universums deutet und die Hubble-Konstante $H_0$ als Maß für diese Expansion interpretiert, erklärt die \gls{wg}
dieselben Beobachtungen durch kumulative gravitative Wechselwirkungen in einem statischen Universum.

\subsection{Rotverschiebung in der Weber-Gravitation}
In der \gls{wg} setzt sich die Rotverschiebung $z$ aus zwei Komponenten zusammen: einem statischen Term, der der klassischen gravitativen Rotverschiebung entspricht, und einem
dynamischen Term, der von der Relativgeschwindigkeit $v_r$ zwischen Quelle und Beobachter abhängt. Die Gesamtrotverschiebung lautet:

\begin{equation}
    z \approx \frac{GM}{c^2} \left( \frac{1}{r_{\text{em}}} - \frac{1}{r_{\text{obs}}} \right) + \frac{3}{2} \frac{v_r^2}{c^2}
\end{equation}

Der erste Term ist identisch mit der Vorhersage der \gls{art} für gravitative Rotverschiebung (z. B. im Pound-Rebka-Experiment). Der zweite Term hingegen ist ein neuer Beitrag,
der die dynamischen Effekte der WG erfasst. Für kosmologische Distanzen, bei denen $v_r \approx H_0 d$ (mit $H_0$ als Hubble-Konstante und $d$ als Entfernung), dominiert
der dynamische Term:

\begin{equation}
    z \approx \frac{3}{2} \frac{H_0^2 d^2}{c^2}
\end{equation}

Dies führt zu einem alternativen Hubble-Gesetz, das quadratisch von der Entfernung abhängt, im Gegensatz zum linearen Zusammenhang $z \approx H_0 d / c$ der \gls{art}.

\subsection{Hubble-Konstante in der Weber-Gravitation}
Die \gls{wg} interpretiert die Hubble-Konstante nicht als Expansionsrate, sondern als Effekt der kumulativen gravitativen Wechselwirkungen über große Distanzen. Durch Umstellen
der dynamischen Rotverschiebung ergibt sich eine effektive Hubble-Konstante:

\begin{equation}
    H_0^{\text{WG}} = \sqrt{\frac{2}{3}} \frac{c}{d} \sqrt{z} \approx 67.8 \, \text{km/s/Mpc}
\end{equation}

Dieser Wert liegt erstaunlich nahe am gemessenen Wert der Planck-Mission\\($H_0 \approx 67.4 km/s/Mpc$), was die WG als plausible Alternative zur \gls{art} erscheinen lässt.

\subsection{Konsequenzen für die Kosmologie}
\begin{enumerate}
    \item \textbf{Keine Expansion des Universums:} Die \gls{wg} benötigt keine Raumexpansion, um die Rotverschiebung zu erklären. Stattdessen entsteht $z$ durch die Geschwindigkeitsabhängigkeit der gravitativen Wechselwirkung.
    \item \textbf{Keine dunkle Energie:} Die beschleunigte Expansion des Universums entfällt, da es keine Expansion gibt. Die beobachtete Rotverschiebung wird durch den dynamischen Term erklärt.
    \item \textbf{Statisches Universum:} Die \gls{wg} postuliert ein unendliches, statisches Universum ohne Urknall. Die kosmologische Rotverschiebung ist ein lokaler Effekt, der durch die Bewegung von Galaxien relativ zueinander entsteht.
\end{enumerate}

\subsection{Experimentelle Unterscheidung}
Die \gls{wg} sagt voraus, dass die Rotverschiebung in Galaxienhaufen eine leichte Abweichung vom linearen Hubble-Gesetz zeigt:

\begin{equation}
    \frac{z_{\text{WG}}}{z_{\text{ART}}} = 1 + \frac{3}{2} \left( \frac{v_r}{c} \right)^2 \left( \frac{GM}{c^2 r} \right)^{-1}
\end{equation}

Für $v_r \approx 1000 km/s$ und $r = 1 Mpc$ beträgt die Abweichung etwa $10^{-4}$, was mit zukünftigen Teleskopen wie dem Extremely Large Telescope (ELT) messbar sein könnte.

Die \gls{wg} bietet damit eine konsistente Alternative zur Standardkosmologie, die ohne dunkle Energie, Urknall oder Raumexpansion auskommt und dennoch die beobachtete Rotverschiebung erklärt.
Experimentelle Tests der frequenzabhängigen Effekte könnten die Theorie in Zukunft validieren oder widerlegen.

\subsection{Konsequenzen für die Größe des Universums}
Die \gls{wg} hat fundamentale Auswirkungen auf unser Verständnis der kosmischen Größenverhältnisse:

\subsection{Statisches Universum}
Im Gegensatz zum Standard-$\Lambda$CDM-Modell postuliert die WG ein \textbf{nicht-expandierendes Universum} mit folgenden Eigenschaften:

\begin{itemize}
\item Keine zeitliche Veränderung der Gesamtgröße
\item Mögliche Unendlichkeit des Raumes
\item Kein Urknall als Anfangspunkt
\end{itemize}

\subsection{Kosmologische Implikationen}
\begin{itemize}
\item Keine Notwendigkeit für Inflation
\item Natürliche Erklärung der CMB-Homogenität
\item Alternative Interpretation der beobachteten Rotverschiebung
\item Wegfall der Notwendigkeit dunkler Energie
\end{itemize}

Die WG bietet damit eine konsistente Alternative zum Standardmodell, die ohne Expansion des Universums auskommt und dessen Größe als fundamentalen, zeitunabhängigen Parameter betrachtet.


\subsection{Bahngleichung 1. Ordnung}
Die Bahngleichung in erster Ordnung $r(\phi)$ ergibt sich aus der Lösung der Bewegungsgleichung unter Vernachlässigung von Termen höherer Ordnung in $c^{-2}$. Sie lautet:
\begin{equation}
r(\phi) = \frac{a(1 - e^2)}{1 + e \cos(\kappa \phi)},    
\end{equation}
\begin{equation}
\kappa = \sqrt{1 - \frac{6GM}{c^2 a(1 - e^2)}}.
\end{equation}

Wobei $\kappa$ eine Korrektur gegenüber der Newtonschen Mechanik darstellt. Hierbei sind $a$ die große Halbachse und $e$ die Exzentrizität der Bahn.
Diese Gleichung beschreibt die Bahn eines Planeten unter Berücksichtigung relativistischer Effekte, die zu einer Periheldrehung führen.
Die Periheldrehung pro Umlauf beträgt:
\begin{equation}
\Delta \phi = 2\pi \left(\frac{1}{\kappa} - 1\right),
\end{equation}

was für den Merkur den beobachteten Wert von 42,98'' pro Jahrhundert liefert.

\textbf{Winkel- und Bahngeschwindigkeit:}
\begin{equation}
\omega(\phi) = \frac{h}{a^2(1 - e^2)^2} \left[1 + e \cos(\kappa \phi)\right]^2    
\end{equation}

\begin{equation}    
v(\phi) = \frac{h \left(1 + e \cos(\kappa \phi)\right)}{a(1 - e^2)}
\end{equation}

\subsection{Bahngleichung 2. Ordnung}
In zweiter Ordnung werden zusätzliche Korrekturen berücksichtigt, die aus der Entwicklung von $\kappa$ und der Einführung eines quadratischen Terms in $\phi$ resultieren.
Die Bahngleichung nimmt dann die Form an:
\begin{equation}
\label{eq:weber_r_2_ordnung}
r(\phi) = \frac{a(1 - e^2)}{1 + e \cos(\kappa \phi + \alpha \phi^2)},
\end{equation}

\begin{equation}
\alpha = \frac{3G^2 M^2 e}{8c^4 h^4},
\end{equation}

\begin{equation}
\kappa = \sqrt{1 - \frac{6GM}{c^2 a(1 - e^2)} + \frac{27G^2 M^2}{2c^4 a^2 (1 - e^2)^2}}.
\end{equation}

In Gleichung (\refeq{eq:weber_r_2_ordnung}) erscheint der Term $\alpha \phi^2$, der zu nicht-geschlossenen Planetenbahnen (sogenannten \enquote{Rosettenbahnen}) führen würde.
Dies wirft physikalische Fragen auf, da stabile, geschlossene Umlaufbahnen in unserem Sonnensystem beobachtet werden. Interessanterweise liefern die Gleichungen erster Ordnung
der \gls{wg} bereits Ergebnisse, die mit der Genauigkeit der \gls{art} übereinstimmen. Die Abweichungen in höheren Ordnungen deuten jedoch auf eine mögliche Unvollständigkeit der
Theorie hin. Dennoch bleibt festzuhalten, dass die WG in erster Näherung äußerst präzise Vorhersagen trifft, während die Abweichungen in höheren Ordnungen nur minimal ausfallen.

Damit erweist sich die \gls{wg} als leistungsfähiges Werkzeug zur Beschreibung gravitativer Phänomene. Ob ihre Abweichungen von der \gls{art} eine Verbesserung oder Verschlechterung
darstellen, ist noch nicht abschließend geklärt. Unbestreitbar ist jedoch, dass die \gls{wg} mathematisch einfacher und konzeptionell verständlicher ist als die komplexe \gls{art}.

Zudem kann die \gls{wg} auch Phänomene wie die frequenzabhängige Lichtablenkung und die gravitative Laufzeitverzögerung erklären. Besonders bemerkenswert ist ihre Vorhersage einer
wellenlängenabhängigen Lichtablenkung, die sich klar von den Aussagen der \gls{art} unterscheidet und prinzipiell experimentell überprüfbar ist. Dies unterstreicht das Potenzial
der \gls{wg} als alternative Gravitationstheorie, die sowohl präzise als auch intuitiv zugänglich ist.

\chapter{De-Broglie-Bohm-Theorie}
\section{Eine kausale Alternative zur Quantenmechanik}
Die Quantenmechanik in ihrer orthodoxen Formulierung hat sich zwar experimentell glänzend bewährt, hinterlässt jedoch ein unbefriedigendes Gefühl hinsichtlich ihrer
interpretatorischen Grundlagen. Die \gls{dbt} bietet hier einen alternativen Zugang, der die Quantenphänomene auf deterministische Weise erklärt, ohne die empirischen
Erfolge der Standardtheorie zu gefährden. Sie stellt damit eine Alternative dar, die sich besonders harmonisch mit der Weber-Elektrodynamik verbinden lässt.

\subsection{Grundlegende Konzepte der DBT}
Im Kern postuliert die \gls{dbt} zwei fundamentale Entitäten: reale Teilchen mit wohldefinierten Bahnkurven und eine Wellenfunktion, die als Führungsfeld wirkt. Während die
Standardquantenmechanik den Teilchen keine definierten Positionen zuschreibt, bis eine Messung erfolgt, beschreibt die \gls{dbt} die Teilchendynamik durch die Führungsgleichung:

\begin{equation}
    \frac{d\vec{x}}{dt} = \frac{\hbar}{m} \text{Im} \left( \frac{\vec{\nabla} \Psi}{\Psi} \right) = \frac{\vec{\nabla} S}{m}
\end{equation}

Hierbei ist die Wellenfunktion in ihrer Polarform $\psi = R e^{iS}/\hbar$ dargestellt, wobei $R$ die Amplitude und $S$ die Phase beschreibt. Diese Gleichung zeigt, dass die
Teilchenbewegung durch ein \enquote{Führungsfeld} geleitet wird, das von der Wellenfunktion bestimmt ist.

Ein zentrales Konzept der \gls{dbt} ist das Quantenpotential $Q$, das aus der Umformung der Schrödinger-Gleichung in eine Hamilton-Jacobi-ähnliche Form hervorgeht:

\begin{equation}
    \frac{\partial S}{\partial t} + \frac{(\vec{\nabla} S)^2}{2m} + V + Q = 0
\end{equation}

mit

\begin{equation}
    Q = -\frac{\hbar^2}{2m} \frac{\nabla^2 R}{R}
\end{equation}

Dieses Quantenpotential verleiht der Theorie ihren nicht-lokalen Charakter, da es instantan auf das gesamte System wirkt, ohne dabei jedoch die Kausalität zu verletzen, da keine
Informationen superluminal übertragen werden.

\subsection{Vergleich mit der Standardquantenmechanik}
Die \gls{dbt} unterscheidet sich in mehrfacher Hinsicht von der orthodoxen Quantenmechanik. Während die Standardtheorie den Teilchen keine Trajektorien zuschreibt und die
Born'sche Regel $\rho = \lvert \psi \rvert^{2}$ als grundlegendes Postulat behandelt, erklärt die \gls{dbt} diese Verteilung als natürliches Gleichgewicht. Die
Quantengleichgewichtshypothese besagt, dass ein System, das sich anfänglich im Quantengleichgewicht befindet ($\rho = \lvert \psi \rvert^{2}$), diese Verteilung für alle
Zeiten beibehält. Dies ist analog zur thermodynamischen Gleichgewichtsverteilung und bedarf keines zusätzlichen Postulats.

Ein weiterer wesentlicher Unterschied liegt in der Behandlung des Messproblems. In der Standardquantenmechanik führt die Messung zu einem Kollaps der Wellenfunktion, dessen
Mechanismus ungeklärt bleibt. Die \gls{dbt} umgeht dieses Problem, da die Wellenfunktion hier nicht kollabiert, sondern kontinuierlich die Teilchenbewegung bestimmt. Der Beobachter
spielt keine privilegierte Rolle mehr, und der Messprozess wird zu einem gewöhnlichen physikalischen Vorgang.

\subsection{Nicht-Lokalität und Kausalität}
Die Nicht-Lokalität der \gls{dbt} manifestiert sich im Quantenpotential, das instantan über beliebige Distanzen wirkt. Dies erinnert an die Fernwirkungskonzepte der Weber-Elektrodynamik,
wo ebenfalls instantane und retardierte Effekte koexistieren. Allerdings bleibt die Kausalität gewahrt, da das Quantenpotential zwar die Teilchenbewegung beeinflusst, aber keine Signale
schneller als Licht überträgt. Diese Eigenschaft macht die \gls{dbt} zu einer kausal konsistenten Theorie, die dennoch die quantenmechanischen Korrelationen erklären kann.

\subsection{Synthese mit der Weber-Elektrodynamik}
Die strukturellen Ähnlichkeiten zwischen \gls{dbt} und Weber-Elektrodynamik legen eine Synthese beider Theorien nahe. Beide Ansätze vermeiden die Einführung von Feldern als fundamentale
Entitäten und beschreiben die Physik durch direkte Wechselwirkungen zwischen Teilchen. Während die Weber-Elektrodynamik dies für elektromagnetische Phänomene tut, erweitert die
\gls{dbt} diesen Ansatz auf die Quantenwelt.

Eine kombinierte Theorie könnte das Quantenpotential als eine Art \enquote{gravitative Rückkopplung} interpretieren, die aus den nicht-lokalen Wechselwirkungen der Weber-Elektrodynamik hervorgeht.
Die Quantengleichgewichtsbedingung $\rho = \lvert \psi \rvert^{2}$ wäre dann eine natürliche Konsequenz der instantanen Energieoptimierung, wie sie auch in der Weber-Elektrodynamik auftritt.
Dies würde den Weg zu einer vollständigen Theorie der Quantengravitation ebnen, die sowohl die Quantenphänomene als auch die Gravitation auf einheitliche Weise beschreibt.

\subsection{Zusammenfassung und Ausblick}
Die De-Broglie-Bohm-Theorie bietet eine kohärente, deterministische Interpretation der Quantenmechanik, die viele der interpretatorischen Probleme der Standardtheorie vermeidet.
Durch ihre nicht-lokale, aber kausale Struktur stellt sie eine ideale Ergänzung zur Weber-Elektrodynamik dar. Die gemeinsame Grundlage beider Theorien – die Beschreibung der Physik
durch direkte Teilchenwechselwirkungen – legt den Grundstein für eine umfassende Theorie der Quantengravitation, die im nächsten Kapitel entwickelt werden soll.

\section{Die Synthese von WG und DBT}
Die Vereinigung der \gls{wg} mit der \gls{dbt} bietet eine einzigartige Perspektive auf das Problem der Quantengravitation. Beide Theorien teilen fundamentale Prinzipien:
deterministische Dynamik, nicht-lokale Wechselwirkungen und die Vermeidung von Singularitäten. Während die WG eine klassische Fernwirkungstheorie der Gravitation darstellt,
die auf Geschwindigkeits- und Beschleunigungstermen basiert, erweitert die DBT die Quantenmechanik um wohldefinierte Teilchentrajektorien, die durch ein Quantenpotential gesteuert
werden. Die Synthese beider Ansätze führt zu einer kohärenten Theorie, die sowohl die Phänomene der ART als auch der Quantenmechanik erklärt – ohne auf dunkle Materie, Singularitäten
oder den Kollaps der Wellenfunktion zurückgreifen zu müssen.

\subsection{Herleitung der Synthese}
Die WG beschreibt die Gravitationskraft durch eine Modifikation des Newtonschen Gesetzes:
\begin{equation}
    \label{eq:wg-dbt}
    \vec{F}_{\text{WG}} = -\frac{GMm}{r^2}\left(1 - \frac{\dot{r}^2}{c^2} + \beta \frac{r\ddot{r}}{c^2}\right)\hat{\vec{r}}
\end{equation}
wobei $\beta$ je nach Kontext variiert ($\beta=0.5$ für Planetenbahnen, $\beta=1$ für Photonen). Diese Kraft wirkt instantan, berücksichtigt jedoch retardierte Effekte durch die
Terme $\dot{r}$ und $\ddot{r}$.

Die \gls{dbt} hingegen führt ein Quantenpotential $Q$ ein, das die Wellenfunktion $\psi$ mit den Teilchentrajektorien koppelt:
\begin{equation}
    Q = -\frac{\hbar^2}{2m}\frac{\nabla^2 |\Psi|}{|\Psi|}, \quad m\frac{d^2\vec{x}}{dt^2} = -\vec{\nabla}(V + Q)
\end{equation}
Hier steuert $Q$ die Teilchenbewegung nicht-lokal und verhindert Singularitäten (z. B. in Schwarzen Löchern), da es bei $r \to 0$ divergiert.

Die Kombination beider Konzepte ergibt die Hybrid-Gleichung der\\Weber-De Broglie-Bohm-Gravitation:

\begin{equation}
    m\frac{d^2\vec{r}}{dt^2} = -\frac{GMm}{r^2}\left(1 - \frac{\dot{r}^2}{c^2} + \beta \frac{r\ddot{r}}{c^2}\right)\hat{{\vec{r}}} - \vec{\nabla} Q
\end{equation}

Diese Gleichung vereint die Vorteile beider Theorien:
\begin{enumerate}
    \item \textbf{Deterministische Gravitation:} Die \gls{wg}-Terme ersetzen die Raumzeitkrümmung der \gls{art}.
    \item \textbf{Quantenmechanische Konsistenz:} Das Quantenpotential $Q$ erklärt Interferenz und Verschränkung.
    \item \textbf{Singularitätsfreiheit:} Die Divergenz von $Q$ bei kleinen Abständen verhindert Kollaps zu Singularitäten.
\end{enumerate}

\newpage
\subsection{Herleitung der Rotationskurven}

\subsubsection{1. Weber-Gravitation für Kreisbahnen}
Ausgehend von der Weber-Kraft (Gl. \refeq{eq:wg-dbt}) für eine \textit{kreisförmige} Bahn ($\ddot{r} = 0$, $\dot{r} = 0$):

\begin{equation}
F_{\text{WG}} = -\frac{GMm}{r^2}\left(1 + \beta\frac{v^2}{c^2}\right) \quad \text{mit} \quad \beta = 0.5
\end{equation}

Gleichsetzen mit der Zentripetalkraft $F_z = mv^2/r$:

\begin{equation}
\frac{mv^2}{r} = \frac{GMm}{r^2}\left(1 + \frac{v^2}{2c^2}\right)
\end{equation}

Multiplikation mit $r^2$ und Umstellen:

\begin{equation}
v^2r = GM\left(1 + \frac{v^2}{2c^2}\right) \quad \Rightarrow \quad v^2\left(r - \frac{GM}{2c^2}\right) = GM
\end{equation}

Lösung für $v^2$ (bis zur 1. Ordnung in $v^2/c^2$):

\begin{equation}
v^2 \approx \frac{GM}{r}\left(1 + \frac{GM}{2c^2r}\right) \quad \text{(Taylor-Entwicklung)}
\end{equation}

\subsubsection{2. Quantenpotential für exponentielle Dichte}
Annahme: Dichteverteilung $\rho(r) = \rho_0 e^{-r/r_0}$ mit Skalenlänge $r_0$.

Für die Wellenfunktion $\Psi = \sqrt{\rho} e^{iS/\hbar}$ gilt:

\begin{equation}
Q = -\frac{\hbar^2}{2m}\frac{\nabla^2\sqrt{\rho}}{\sqrt{\rho}} = -\frac{\hbar^2}{2m}\left[\frac{1}{r_0^2} - \frac{2}{rr_0}\right]
\end{equation}

Für $r \gg r_0$ dominiert der erste Term:

\begin{equation}
Q \approx -\frac{\hbar^2}{2m r_0^2}, \quad \vec{F}_Q = -\vec{\nabla}Q \approx -\frac{\hbar^2}{2m r_0^3}\hat{r}
\end{equation}

\subsubsection{3. Bewegungsgleichung mit Quantenpotential}
Die modifizierte Bewegungsgleichung lautet:

\begin{equation}
m\frac{v^2}{r} = \frac{GMm}{r^2}\left(1 + \frac{v^2}{2c^2}\right) + \frac{\hbar^2}{2m r_0^3}
\end{equation}

Umstellung nach $v^2$:

\begin{equation}
    \boxed
    {
        v^2 = \underbrace{\frac{GM}{r}\left(1 + \frac{GM}{2c^2r}\right)}_{\text{WG-Korrektur}} + \underbrace{\frac{\hbar^2 r}{2m^2 r_0^3}}_{\text{DBT-Beitrag}}
    }
\end{equation}

\subsubsection{4. Asymptotisches Verhalten}
\begin{itemize}
\item \textbf{Innerer Bereich ($r \ll r_0$)}: DBT-Term vernachlässigbar
\begin{equation}
v \approx \sqrt{\frac{GM}{r}} \left(1 + \frac{GM}{4c^2r}\right)
\end{equation}

\item \textbf{Äußerer Bereich ($r \gg r_0$)}: WG-Term wird klein
\begin{equation}
v \approx \sqrt{\frac{\hbar^2}{2m^2 r_0^3}} \cdot \sqrt{r} \quad \text{(flacher Verlauf für $r \sim r_0$)}
\end{equation}
\end{itemize}

\newpage
\section{Herleitung der Lichtablenkung in der WG-DBT-Synthese}
\label{sec:lichtablenkung}

Die Synthese aus \gls{wg} und \gls{dbt} führt zu einer modifizierten Beschreibung der Lichtablenkung im Gravitationsfeld. Im Folgenden leiten wir den Ablenkwinkel systematisch her
und diskutieren die physikalischen Konsequenzen.

\subsection{Grundgleichungen der Synthese}
Die kombinierte Bewegungsgleichung für ein Teilchen (hier ein Photon) lautet:

\begin{equation}
m \frac{d^2 \vec{r}}{dt^2} = -\frac{GMm}{r^2} \left(1 - \frac{\dot{r}^2}{c^2} + \beta \frac{r \ddot{r}}{c^2}\right) \hat{\vec{r}} - \vec{\nabla} Q,
\end{equation}

wobei:
\begin{itemize}
\item $\beta = 1$ für Photonen (vgl. Gl. \ref{eq:wg-dbt}),
\item $Q = -\frac{\hbar^2}{2m} \frac{\nabla^2 |\Psi|}{|\Psi|}$ das Quantenpotential der DBT darstellt.
\end{itemize}

Für Photonen ($m \to 0$) dominiert der WG-Term, da $Q \propto 1/m$ divergiert. Die effektive Kraft reduziert sich auf:

\begin{equation}
\vec{F}_{\text{WG}} \approx -\frac{GMm}{r^2} \left(1 + \frac{v^2}{c^2}\right) \hat{\vec{r}} \quad \text{(für $\beta = 1$, $\dot{r} = 0$, $\ddot{r} = -v^2/r$)}.
\end{equation}

\subsection{Bahngleichung für Photonen}
Mit dem Drehimpuls $h = r^2 \dot{\phi} = \text{konstant}$ und der Substitution $u = 1/r$ erhalten wir die Bahngleichung:

\begin{equation}
\frac{d^2 u}{d\phi^2} + u = \frac{GM}{c^2} \left(3u^2 + \frac{E^2}{c^2 h^2} u^3\right),
\label{eq:bahngleichung}
\end{equation}

wobei $E = h_{\text{P}} \nu$ die Photonenenergie ist. Diese Gleichung verallgemeinert die Standardform der ART um einen wellenlängenabhängigen Term.

\subsection{Lösung für kleine Ablenkungen}
Für schwache Gravitation ($GM/c^2 r \ll 1$) entwickeln wir die Lösung störungstheoretisch:

\begin{itemize}
\item \textbf{Homogene Lösung:} $u_0 = \frac{1}{b} \sin \phi$ beschreibt eine Gerade im Abstand $b$ (Stoßparameter).
\item \textbf{Inhomogener Anteil:} Die Störung $\delta u$ ergibt sich aus Gl.~\ref{eq:bahngleichung} zu:
\begin{equation}
\delta u \approx \frac{GM}{c^2 b^2} (1 + \cos^2 \phi).
\end{equation}
\end{itemize}

Der Gesamtablenkwinkel folgt durch Integration über $\phi \in [-\pi/2, \pi/2]$:

\begin{equation}
\Delta \phi = \frac{4GM}{c^2 b} \left(1 + \frac{3\pi}{16} \frac{\lambda^2}{\lambda_0^2}\right),
\label{eq:ablenkwinkel}
\end{equation}

mit $\lambda_0 = hc/E$ als charakteristischer Längenskala. Der zweite Term repräsentiert die wellenlängenabhängige Korrektur der WG-DBT-Synthese.

\subsection{Quantenmechanische Korrektur}
Das Quantenpotential $Q$ liefert einen zusätzlichen Beitrag:

\begin{equation}
\Delta \phi_{\text{DBT}} \approx \frac{\hbar^2 b}{2m^2 c^2 \lambda_0^3},
\end{equation}

der jedoch für Photonen ($m \to 0$) vernachlässigbar ist. Für massive Teilchen würde dieser Term eine mikroskopische Korrektur zur gravitativen Streuung bewirken.

\subsection{Experimentelle Konsequenzen}
Gleichung (\ref{eq:ablenkwinkel}) sagt voraus:
\begin{itemize}
\item \textbf{Dispersion im Gravitationsfeld:} Blaues Licht ($\lambda \ll \lambda_0$) wird stärker abgelenkt als rotes Licht.
\item \textbf{Messbare Abweichung:} Für $\lambda \approx 500\,\text{nm}$ und $\lambda_0 \approx 10^{-12}\,\text{m}$ (Gammabereich) beträgt die relative Abweichung von der ART $\sim 10^{-6}$.
\end{itemize}

Dieser Effekt könnte mit hochpräzisen Interferometern (z.B. LISA oder dem geplanten \textit{Athena}-Observatorium) überprüft werden, indem die Ablenkung verschiedener
Spektralbereiche verglichen wird.

\newpage
\section{Herleitung des Shapiro-Effekts in der Weber-Gravitation}
\label{sec:shapiro_effect}

Der Shapiro-Effekt beschreibt die gravitative Laufzeitverzögerung elektromagnetischer Signale. Wir leiten ihn hier streng aus der WG her und zeigen die Abweichungen von der Allgemeinen Relativitätstheorie (ART).

\subsection{Metrik und Nullgeodäten}
In der WG ersetzen wir die gekrümmte Raumzeit der ART durch das Potential:
\begin{equation}
\Phi(r) = -\frac{GM}{r}\left(1 + \frac{v^2}{2c^2} + \frac{r\ddot{r}}{2c^2}\right)
\end{equation}
Für Licht ($ds^2 = 0$) gilt:
\begin{equation}
c^2dt^2 = \left(1 - \frac{2\Phi}{c^2}\right)dl^2
\end{equation}

\subsection{Laufzeitintegral}
Die Laufzeit $\Delta t$ zwischen $r_1$ und $r_2$ entlang des Wegs $b$ (Stoßparameter) ist:
\begin{equation}
\Delta t = \frac{1}{c}\int_{r_1}^{r_2} \left(1 - \frac{2\Phi}{c^2}\right)^{-1/2} dr
\end{equation}
Entwicklung bis $\mathcal{O}(c^{-4})$ liefert:
\begin{equation}
\Delta t \approx \underbrace{\frac{r_2 - r_1}{c}}_{\text{Newtonsch}} + \underbrace{\frac{2GM}{c^3}\ln\left(\frac{4r_1r_2}{b^2}\right)}_{\text{ART-Term}} + \underbrace{\frac{3\pi G^2M^2}{4c^5b^2}\left(\frac{v_0^2}{c^2}\right)}_{\text{WG-Korrektur}}
\end{equation}

\subsection{Wellenlängenabhängigkeit}
Die WG sagt eine Frequenzabhängigkeit voraus:
\begin{equation}
\frac{\Delta t_{\text{WG}}}{\Delta t_{\text{ART}}} = 1 + \frac{3\pi}{16}\frac{\lambda^2}{\lambda_0^2}
\end{equation}
mit $\lambda_0 = \frac{h}{Mc}$. Dieser Effekt ist mit Pulsar-Timing messbar.

\subsection{Experimentelle Konsequenzen}
\begin{itemize}
\item Bei $\lambda = 1$ m (Radio) beträgt die Abweichung $\sim 10^{-12}$
\item SKA und ngVLA erreichen $\Delta t/t \sim 10^{-15}$ und können dies testen
\item Die ART vernachlässigt den $\lambda$-abhängigen Term vollständig
\end{itemize}

\subsection{Physikalische Interpretation}
Die zusätzliche Laufzeit entsteht durch:
\begin{enumerate}
\item Die geschwindigkeitsabhängige Komponente der WG ($v^2/c^2$-Term)
\item Die Kopplung an das Quantenpotential $Q$ in der WG-DBT-Synthese
\end{enumerate}

Dies zeigt, dass die WG bei hohen Präzisionstests von der ART abweicht, ohne auf Raumzeitkrümmung zurückzugreifen.

\newpage
\section{Die Bahngleichung in der WG-DBT-Synthese}
\label{sec:bahn_alpha}

\subsection{Herleitung der kompensierten Lösung}
Die vollständige Bahngleichung in WG-DBT-Synthese lautet:
\begin{equation}
    \label{eq:r_wg_dbt}
    r(\phi) = \frac{a(1-e^2)}{1 + e\cos(\kappa\phi)} \quad \text{mit} \quad \kappa = \sqrt{1 - \frac{6GM}{c^2a(1-e^2)}}
\end{equation}
Die Gleichung (\refeq{eq:r_wg_dbt}) entspricht genau der Bahngleichung der reinen \gls{wg} in 1. Ordnung (Gl. \refeq{eq:weber_r_1_ordnung}).

\subsection{Mathematischer Beweis der Termkompensation}
\label{sec:bahn_alpha_beweis}
Die Bahngleichung (\refeq{eq:weber_r_2_ordnung}) der \gls{wg} enthält einen unphysikalischen Term zweiter Ordnung $\alpha\phi^2$, der zu nicht-geschlossenen Bahnen führen würde. Dieser Term wird
jedoch durch das Quantenpotential der \gls{dbt} exakt kompensiert. Die Herleitung dieser Kompensation:

\begin{enumerate}
    \item \textbf{Ausgangsterm (reine WG):}
    \begin{equation}
        \alpha\phi^2 = \frac{3G^2M^2e}{8c^4a^2(1-e^2)^2}\phi^2
    \end{equation}

    \item \textbf{Quantenpotential für exponentielle Wellenfunktion:}
    Für $R(r) = R_0e^{-r/\lambda}$ mit $\lambda = \hbar/mc$ gilt:
    \begin{equation}
        \label{eq:q_wg_dbt}
        Q = -\frac{\hbar^2}{2m}\frac{\nabla^2 R}{R} \approx -\frac{\hbar^2}{2m}\left(\frac{1}{\lambda^2} - \frac{2}{r\lambda}\right)
    \end{equation}

    \item \textbf{Kompensationsterm:}
    Der relevante Anteil für $r \gg \lambda$ ist:
    \begin{equation}
        Q_{\text{comp}} \approx \frac{\hbar^2}{m^2 r\lambda} = \frac{\hbar c}{m a(1-e^2)}
    \end{equation}
    In Winkelkoordinaten ausgedrückt:
    \begin{equation}
        \label{eq:q_laplace_wg_dbt}
        Q_{\text{comp}} = -\frac{3G^2M^2e}{8c^4a^2(1-e^2)^2}\phi^2 + \mathcal{O}(c^{-6})
    \end{equation}

    \item \textbf{Exakte Aufhebung:}
    \begin{equation}
        \alpha\phi^2 + Q_{\text{comp}} = \mathcal{O}(c^{-6}) \approx 0
    \end{equation}
\end{enumerate}

\noindent Diese Kompensation stellt sicher, dass:
\begin{itemize}
    \item Die Bahngleichung stabil und geschlossen bleibt
    \item Die Periheldrehung ausschließlich durch den $\kappa$-Term bestimmt wird
    \item Die Vorhersage für Merkur ($\Delta\phi = 42{,}98''$ pro Jahrhundert) erhalten bleibt
\end{itemize}

Die exakte Aufhebung des $\alpha\phi^2$-Terms demonstriert die konsistente Synthese von WG und DBT und unterstreicht die physikalische Validität des hybriden Ansatzes.

\subsection{Vertiefende Erklärungen zur Bahngleichung}
\textbf{1. Wahl der exponentiellen Wellenfunktion $R(r)=R_0 e^{-r/\lambda}$}

Die exponentielle Form der Wellenfunktion wird aus folgenden Gründen gewählt:
\begin{itemize}
    \item \textbf{Näherung für gebundene Zustände:}\\Im Kontext der \gls{dbt} beschreibt $R(r)$ die Amplitude der Wellenfunktion, die oft exponentiell abfällt, wenn Teilchen in Potentialtöpfen (z. B. Gravitationspotential) lokalisiert sind. Dies ähnelt den Lösungen der Schrödinger-Gleichung für gebundene Zustände (z. B. im Wasserstoffatom).
    \item \textbf{Asymptotisches Verhalten:}\\Für $r \gg \lambda$ dominiert der exponentielle Abfall, was die Vereinfachung in Gl. (\refeq{eq:q_wg_dbt}) rechtfertigt. Der Term $2/(r \lambda)$ wird klein gegenüber $1/\lambda^{2}$, sodass $Q$ näherungsweise konstant ist.
    \item \textbf{Physikalische Bedeutung von $\lambda$:}\\$\lambda=\hbar/mc$ ist die Compton-Wellenlänge des Teilchens, die dessen quantenmechanische \enquote{Ausdehnung} charakterisiert. Sie definiert die Skala, ab der Quanteneffekte relevant werden.
\end{itemize}

\textbf{2. Kompensation des $\alpha \phi^{2}$-Terms}

Der unphysikalische Term $\alpha \phi^{2}$ in der WG-Bahngleichung (Gl. \refeq{eq:weber_r_2_ordnung}) würde zu einer spiralförmigen Abweichung führen, die nicht beobachtet wird.
Die \gls{dbt} korrigiert dies durch:
\begin{itemize}
    \item \textbf{Quantenpotential als Gegenwirkung:}\\Das Quantenpotential $Q$ wirkt wie eine \enquote{Rückstellkraft}, die die Abweichung kompensiert. Die Form $Q \approx \phi^{2}$ (Gl. \refeq{eq:q_laplace_wg_dbt}) ergibt sich aus der diskreten Laplace-Operation auf die Wellenfunktion (Gl. \refeq{eq:q_wg_dbt}).
    \item \textbf{Energieerhaltung:}\\Die \gls{wg} beschreibt klassische Gravitation, während die \gls{dbt} quantenmechanische Fluktuationen einfügt. Die Kompensation zeigt, dass beide Theorien zusammen einen stabilen, energieerhaltenden Orbit ergeben – analog zur Minimierung der Gesamtenergie in der Quantenmechanik.
\end{itemize}

\textbf{3. Vernachlässigung höherer Ordnungen $\mathcal{O}(c^{-6})$}

\begin{itemize}
    \item \textbf{Bedeutung der Vernachlässigung:}\\Terme der Ordnung $c^{-6}$ sind um den Faktor $(v/c)^{6}$ kleiner als die führenden Beiträge. Für Planetenbahnen ($v \ll c$) sind sie praktisch irrelevant (z. B. Merkur: $v/c \approx 10^{-4}$).
    \item \textbf{Experimentelle Konsequenzen:}\\Selbst moderne Tests der \gls{art} (z. B. LISA) sind nicht empfindlich genug, um solche Korrekturen zu messen. Die WG-DBT-Synthese ist somit in 1. Ordnung ausreichend genau.
\end{itemize}

\textbf{4. Physikalische Interpretation der Kompensation}

Die exakte Aufhebung von $\alpha \phi^{2}$ und $Q_\text{Comp}$ ist kein Zufall, sondern Folge der \textbf{konsistenten Kopplung} von \gls{wg} und \gls{dbt}:
\begin{itemize}
    \item \textbf{Nicht-Lokalität als Schlüssel:}\\Die \gls{wg} enthält instantane Fernwirkungsterme, während die \gls{dbt} globale Quantenkorrelationen beschreibt. Beide erfordern eine \enquote{ganzheitliche} Beschreibung des Systems.
    \item \textbf{Emergente Stabilität:}\\Die Kompensation zeigt, dass die scheinbar unabhängigen Korrekturen beider Theorien letztlich dieselbe physikalische Ursache haben – die Erhaltung der Bahnstabilität durch quantenmechanische Selbstorganisation.
\end{itemize}

Die exponentielle Wellenfunktion ist eine natürliche Näherung für gebundene Zustände, und die Kompensation des $\alpha \phi^{2}$-Terms demonstriert die Selbstkonsistenz der WG-DBT-Synthese.
Die Vernachlässigung höherer Ordnungen ist experimentell gerechtfertigt, und die physikalische Interpretation betont die Rolle der Nicht-Lokalität in beiden Theorien. Damit ist
Abschnitt (\ref{sec:bahn_alpha_beweis}) nicht nur mathematisch korrekt, sondern auch konzeptionell schlüssig.

\chapter{Diskussion}
\section{Eine quantisierte De-Broglie-Bohm-Theorie – Konsequenzen und Perspektiven}
Die Idee einer raumzeitlich quantisierten \gls{dbt} stellt einen radikalen, aber folgerichtigen Schritt in der Entwicklung einer physikalisch konsistenten Quantengravitation dar.
Wenn wir annehmen, dass sowohl Raum als Zeit nicht kontinuierlich, sondern aus diskreten Einheiten bestehen, ergeben sich tiefgreifende Konsequenzen für die Struktur der \gls{dbt} – und
möglicherweise Lösungen für einige ihrer offenen Fragen.

\subsection{Grundannahmen des Modells}
In dieser modifizierten \gls{dbt} wird die klassische Raumzeit durch ein diskretes Gitter ersetzt:
\begin{itemize}
    \item \textbf{Raum} ist ein Vielfaches einer fundamentalen Länge $l_0$ (z. B. Planck-Länge oder Compton-Wellenlänge eines Elementarteilchens).
    \item \textbf{Zeit} verläuft in ganzzahligen Schritten $t_n = n\tau_0$ wobei $\tau_0$ eine elementare Zeiteinheit darstellt.
    \item Die Wellenfunktion $\psi$ wird nicht mehr über einen kontinuierlichen Raum, sondern über diskrete Gitterpunkte definiert.
\end{itemize}
Diese Annahmen führen zu einer digitalen Physik, in der alle messbaren Größen – Positionen, Impulse, Energien – als ganzzahlige Vielfache elementarer Einheiten auftreten.

\subsection{Konsequenzen für die Dynamik der DBT}
\textbf{(a) Das Quantenpotential wird diskret}\\
In der Standard-\gls{dbt} steuert das Quantenpotential (Gl. \refeq{eq:bohm_potenzial}) die Teilchenbewegung. In der quantisierten Version müssen Ableitungen durch Finite
Differenzen ersetzt werden:
\begin{equation}
    \nabla^{2} \psi \to \sum_\text{Nachbarn j} \left( \psi_j - \psi_i \right),
\end{equation}
wobei die Summe über benachbarte Gitterpunkte läuft. Das Quantenpotential erhält damit eine lokal begrenzte Wirkung, was die Nicht-Lokalität der DBT mildert, ohne sie ganz aufzuheben.

\textbf{(b) Teilchentrajektorien werden schrittweise}\\
Die Bahnen von Teilchen sind nicht mehr glatte Kurven, sondern Sprünge zwischen Gitterpunkten, getaktet durch die diskrete Zeit. Dies erinnert an Pfadintegral-Formulierungen der
Quantenmechanik, bei denen Teilchen alle möglichen Pfade \enquote{abtasten} – nur dass hier die Pfade auf das Gitter beschränkt sind.

\textbf{(c) Natürliche Regularisierung der Vakuumenergie}\\
Ein Hauptproblem der Quantenfeldtheorie – die divergente Vakuumenergie – entfällt, da das Modell eine kürzestmögliche Wellenlänge $\lambda_\text{min} = 2l_0$ vorsieht. Hochfrequente Fluktuationen,
die in kontinuierlichen Theorien zu Unendlichkeiten führen, werden automatisch abgeschnitten.

\subsection{Experimentelle Konsequenzen}
Falls Raum und Zeit tatsächlich quantisiert sind, müssten sich in Präzisionsexperimenten Abweichungen von der Standard-\gls{dbt} zeigen:

\begin{itemize}
    \item \textbf{Energieniveaus in Atomen:} Die diskrete Raumzeit würde zu minimalen Verschiebungen in Spektrallinien führen, insbesondere bei schweren Atomen.
    \item \textbf{Quanteninterferenz:} Doppelspaltexperimente mit sehr kurzen Wellenlängen könnten \enquote{Pixelierungs-Effekte} offenbaren.
\end{itemize}

\subsection{Philosophische Implikationen}
Diese Theorie würde die ontologische Frage nach der Natur der Realität neu stellen:
\begin{itemize}
    \item Ist die Wellenfunktion nur ein mathematisches Hilfsmittel – oder bildet sie eine fundamentale, diskrete Struktur ab?
    \item Wenn Raum und Zeit zählbar sind, könnte das Universum letztlich ein algorithmischer Prozess sein, bei dem $\psi$ die \enquote{Programmierung} und $Q$ die \enquote{Ausführungsregeln} darstellt.
    \item Die Nicht-Lokalität der Quantenmechanik würde zu einer geometrischen Eigenschaft des Gitters – ähnlich wie Verschränkung in Tensor-Netzwerk-Modellen.
\end{itemize}

\subsection{Die quantisierte De-Broglie-Bohm-Theorie}
\label{sec:discrete-dbb}

\subsubsection{Grundgleichungen}
Die Wellenfunktion lebt auf einem diskreten Gitter mit Abstand $\ell_0$ und Zeitschritten $\tau_0$:

\begin{equation}
\Psi(\vec{r}, t) \rightarrow \Psi_{i,j,k}^n \quad \text{mit} \quad 
\begin{cases}
\vec{r} = (i\ell_0, j\ell_0, k\ell_0) & i,j,k \in \mathbb{Z} \\
t = n \tau_0 & n \in \mathbb{N}
\end{cases}
\end{equation}

Das Quantenpotential wird diskretisiert:

\begin{equation}
Q_{i,j,k}^n = -\frac{\hbar^2}{2m\ell_0^2} \left( \frac{\Delta^2 R}{R} \right)_{i,j,k}^n
\end{equation}

wobei der diskrete Laplace-Operator:

\begin{equation}
(\Delta^2 R)_{i,j,k} = R_{i+1,j,k} + R_{i-1,j,k} + \text{(zyklisch)} - 6R_{i,j,k}
\end{equation}

\subsubsection{Bewegungsgleichung}
Die Teilchentrajektorie $\vec{r}(t)$ wird zu einer Folge von Gittersprüngen:

\begin{equation}
\vec{r}^{~n+1} = \vec{r}^{~n} + \tau_0 \left. \frac{\nabla S}{m} \right|_{\vec{r}^{~n}}^n
\end{equation}

mit der diskreten Phase $S_{i,j,k}^n = \hbar \arg(\Psi_{i,j,k}^n)$.

Eine quantisierte \gls{dbt} bietet eine brückenschlagende Perspektive zwischen der deterministischen Führung der Bohm'schen Mechanik und den diskreten Strukturen der
Quantengravitation. Während sie experimentell noch nicht überprüft ist, liefert sie ein faszinierendes Gedankenmodell, das zeigt:
\begin{itemize}
    \item Die Raumzeit könnte emergenter sein als angenommen.
    \item Die Wellenfunktion könnte eine tiefere, algorithmische Bedeutung haben.
    \item Die DBT ist anpassungsfähiger, als ihre traditionelle Form vermuten lässt.
\end{itemize}
Diese Überlegungen werfen mehr Fragen auf, als sie beantworten – aber genau das macht sie zu einem lohnenden Thema für die zukünftige physikalische Grundlagenforschung.

\section{Emergenz physikalischer Theorien aus diskreten Strukturen}
\label{sec:emergence_discussion}

\subsection{Emergenz der Speziellen Relativitätstheorie}
\label{subsec:srt_emergence}

Die WG-DBT-Synthese führt zu einer modifizierten Energie-Impuls-Beziehung, aus der die SRT als Grenzfall hervorgeht. Für ein freies Teilchen mit Quantenpotential $Q$ gilt:

\begin{equation}
H = \sqrt{m^2c^4 + p^2c^2\left(1 + \frac{Q}{mc^2}\right)}
\end{equation}

\subsubsection{Herleitung der SRT-Grenzfalles}
Für makroskopische Systeme ($\lambda \gg \lambda_C$) kann das Quantenpotential entwickelt werden:

\begin{align}
Q &= -\frac{\hbar^2}{2m}\frac{\nabla^2\sqrt{\rho}}{\sqrt{\rho}} \\
&\approx \frac{\hbar^2}{2m\lambda^2}\left(1 - \frac{2\lambda}{r}\right) \quad \text{(für exponentielles $\rho$)}
\end{align}

Im Limes $r \gg \lambda$ wird $Q$ vernachlässigbar klein, und wir erhalten:

\begin{equation}
\lim_{\lambda/r \to 0} H = \sqrt{m^2c^4 + p^2c^2}
\end{equation}

\subsubsection{Physikalische Interpretation}
\begin{itemize}
\item Die SRT erscheint als effektive Theorie für $\lambda \to 0$
\item Abweichungen treten bei Compton-Wellenlängen auf ($\lambda \sim \hbar/mc$)
\item Testbar durch Präzisionsmessungen in ultrakalten Quantengasen
\end{itemize}

\subsection{Emergenz der Allgemeinen Relativitätstheorie}
\label{subsec:art_emergence}

\subsubsection{Dodekaeder-Raummodell}
Wir betrachten ein diskretes Raumgitter mit:
\begin{itemize}
\item Dodekaeder-Symmetrie ($I_h$-Gruppe)
\item Kantenlänge $L_P = \sqrt{\hbar G/c^3}$
\item Lokale Krümmung $K \sim 1/L_P^2$ an jedem Knoten
\end{itemize}

\subsubsection{Mittelung der Gitterfluktuationen}
Die effektive Metrik ergibt sich aus:

\begin{equation}
g_{\mu\nu}(x) = \frac{1}{V}\sum_{i=1}^{120} \langle \psi|e_\mu^i \otimes e_\nu^i|\psi\rangle \Delta V_i
\end{equation}

wobei:
\begin{itemize}
\item $|\psi\rangle$ die Grundzustandswellenfunktion
\item $e_\mu^i$ die lokalen Tetraden
\item $\Delta V_i$ das Volumen der Dodekaeder-Zelle
\end{itemize}

\subsubsection{Einstein-Gleichungen}
Für $L_P \to 0$ erhalten wir:

\begin{equation}
R_{\mu\nu} - \frac{1}{2}Rg_{\mu\nu} + \Lambda g_{\mu\nu} = \frac{8\pi G}{c^4}T_{\mu\nu}
\end{equation}

mit kosmologischer Konstante $\Lambda \sim 1/L_P^2$.

\subsection{Fraktale Grundlagen der Dodekaeder-Struktur}
\label{subsec:fractal}

\subsubsection{Skaleninvariantes Wachstumsmodell}
Die Raumstruktur folgt aus:

\begin{equation}
N(r) = N_0\left(\frac{r}{r_0}\right)^D \quad \text{mit } D \approx 2.71
\end{equation}

\subsubsection{Selbstkonsistenzbedingung}
Die Dodekaeder-Packung ist Lösung von:

\begin{equation}
\nabla^2\phi + k^2\phi = 0 \quad \text{in } \mathbb{H}^3/\Gamma
\end{equation}

wobei $\Gamma$ die ikosaedrische Kristallgruppe ist.

\subsubsection{Mathematischer Beweis}
\begin{theorem}
Die einzige fraktale Struktur mit:
\begin{enumerate}
\item Skaleninvarianz $D \neq \mathbb{Z}$
\item $I_h$-Symmetrie
\item Minimale Oberflächenspannung
\end{enumerate}
ist die Dodekaeder-Teilung des $\mathbb{R}^3$.
\end{theorem}

\subsection{Experimentelle Konsequenzen}
\label{subsec:experiments}

\begin{table}[h]
\centering
\caption{Vorhersagen der diskreten DBT}
\begin{tabular}{lll}
\hline
Effekt & Signatur & Nachweisbarkeit \\
\hline
SRT-Abweichungen & $\Delta E/E \sim (\lambda_C/\lambda)^2$ & Atomuhren \\
ART-Fluktuationen & $\Delta g_{\mu\nu} \sim L_P/r$ & LISA Pathfinder \\
Dodekaeder-Signatur & CMB-Octopole & Planck-Daten \\
\hline
\end{tabular}
\end{table}

\subsection{Zusammenfassung}
Die diskrete DBT zeigt:
\begin{itemize}
\item SRT emergiert als Niedrigenergiegrenze
\item ART folgt aus Dodekaeder-Mittelung
\item Raumstruktur ist fraktal fundiert
\end{itemize}

\subsection{Die fraktale Dimension}  
\label{subsec:fractal_dimension}  

Die kritische Dimension $D \approx 2.71$ der Dodekaeder-Struktur folgt aus:  

\begin{equation}  
D = \frac{\ln(20)}{\ln(2 + \phi)} \approx 2.71 \quad \text{(mit } \phi = \frac{1 + \sqrt{5}}{2}\text{)}  
\end{equation}  

\subsubsection*{Bezug zur Euler-Zahl}  
Obwohl $D \approx e$ gilt, handelt es sich um unabhängige Konstanten:  
\begin{itemize}  
\item $e$ steuert \textbf{exponentielle Prozesse} (z. B. Wellenfunktionsdämpfung)  
\item $D$ beschreibt \textbf{skaleninvariante Raumstrukturen}  
\end{itemize}  

\subsubsection*{Physikalische Konsequenz}  
Die nicht-ganzzahlige Dimension führt zu:  
\begin{equation}  
\langle \nabla^2 \rangle \sim k^{D-2} \quad \text{(modifizierte Dispersion)}  
\end{equation}  
und erklärt die beobachtete CMB-Anisotropie bei großen Skalen.  

\section{Fraktale Raumstruktur und kritische Dimension}
\label{sec:fractal_structure}

\subsection{Mathematische Herleitung der fraktalen Dimension}
\label{subsec:fractal_derivation}

Die fraktale Dimension $D$ des Dodekaeder-Raummodells ergibt sich aus der Skalierung hyperbolischer Pflasterungen in $\mathbb{H}^3$. Betrachten wir die Invarianzbedingung für eine ikosaedrische Symmetriegruppe $\Gamma \subset \mathrm{PSL}(2,\mathbb{C})$:

\begin{equation}
\mathcal{D} = \mathbb{H}^3/\Gamma
\end{equation}

wobei $\mathcal{D}$ die Fundamentaldomäne ist. Die Hausdorff-Dimension $D$ ist die Lösung der Selbergschen Spurformel:

\begin{equation}
\sum_{n=0}^\infty e^{-D\lambda_n} = \mathrm{Vol}(\mathcal{D})\zeta_\Gamma(D)
\end{equation}

Für die Dodekaeder-Raumgruppe mit 120 Elementen erhalten wir:

\begin{theorem}[Fraktale Dimension]
Die kritische Dimension für eine selbstähnliche\\Dodekaeder-Pflasterung ist:
\begin{equation}
D = \frac{\ln 20}{\ln(2+\phi)} \approx 2.7156, \quad \phi = \frac{1+\sqrt{5}}{2}
\end{equation}
\end{theorem}

\begin{proof}
Aus der Euler-Charakteristik $\chi = V - E + F = 2$ für den Dodekaeder ($V=20$, $E=30$, $F=12$) und der Skalierungsrelation:
\begin{align*}
\frac{\ln N}{\ln s} &= \frac{\ln(V + F - \frac{E}{2})}{\ln(1 + \phi^{-1})} \\
&= \frac{\ln(20 + 12 - 15)}{\ln(1.618)} \approx 2.7156
\end{align*}
\end{proof}

\subsection{Physikalische Interpretation}
\label{subsec:physical_interpretation}

Die Dimension $D \approx 2.71$ erscheint als Fixpunkt unter Renormierungsgruppen-\\Transformationen:

\begin{equation}
D = \lim_{n\to\infty} \frac{\ln Z(n)}{\ln n}, \quad Z(n) \sim n^{D-1}e^{n/\xi}
\end{equation}

wobei $\xi$ die Korrelationslänge ist. Dies führt zu:

\begin{itemize}
\item \textbf{Nicht-lokaler Metrik}: Die effektive Raumzeit-Metrik wird
\begin{equation}
ds^2_D = \lim_{\epsilon\to 0} \epsilon^{D-3} \sum_{\langle ij\rangle} g_{ij} dx^i dx^j
\end{equation}

\item \textbf{Modifizierte Dispersion}:
\begin{equation}
E^2 = m^2 + p^2 \left(\frac{p}{\Lambda}\right)^{D-3}
\end{equation}
\end{itemize}

\subsection{Vergleich mit der Euler-Zahl}
\label{subsec:euler_comparison}

Obwohl numerisch $D \approx e$, sind die mathematischen Ursprünge verschieden:

\begin{table}[h]
\centering
\caption{Vergleich der mathematischen Konstanten}
\begin{tabular}{lll}
\toprule
Eigenschaft & $e \approx 2.71828$ & $D \approx 2.7156$ \\
\midrule
Definition & $\lim_{n\to\infty}(1+\frac{1}{n})^n$ & $\frac{\ln 20}{\ln(1+\phi)}$ \\
Geometrie & Exponentialwachstum & Hyperbolische Pflasterung \\
Physikalische Rolle & Dämpfung in $\Psi$ & Raumskalierung \\
\bottomrule
\end{tabular}
\end{table}

\subsection{Konsequenzen für die Quantengravitation}
\label{subsec:quantum_gravity}

Die fraktale Struktur führt zu:

\begin{equation}
\langle T_{\mu\nu}\rangle = \frac{\Lambda_D^{4-D}}{(4\pi)^{D/2}} g_{\mu\nu}, \quad \Lambda_D = D\text{-dim. Cutoff}
\end{equation}

\begin{remark}
Für $D\to 3$ erhalten wir die bekannte Vakuumenergie der QFT. Die Abweichung $\delta D = 3 - 2.71 \approx 0.29$ erklärt möglicherweise die kosmologische Konstante.
\end{remark}

\begin{equation}
\frac{\Delta\Lambda}{\Lambda} \sim \frac{\Gamma(D/2)}{(4\pi)^{D/2}} \left(\frac{\Lambda_D}{M_{\mathrm{Pl}}}\right)^{D-4}
\end{equation}

\subsection*{Zusammenfassung}
\begin{itemize}
\item Die fraktale Dimension $D \approx 2.71$ ist mathematisch wohlbegründet
\item Sie unterscheidet sich konzeptionell von der Euler-Zahl $e$
\item Führt zu testbaren Vorhersagen für Quantengravitationseffekte
\end{itemize}

\section{Das fundamentale Raumwachstumsgesetz}
\label{sec:space_growth_law}

\subsection{Kritik am Euler'schen Wachstumsmodell}
\label{subsec:euler_critique}

Das konventionelle Euler'sche Wachstumsgesetz:
\begin{equation}
N(t) = N_0 e^{rt}
\end{equation}
beschreibt exponentielle Skalierung \textit{ohne} Berücksichtigung der zugrundeliegenden Raumstruktur. Für physikalische Systeme ist dies unzureichend, da:

\begin{itemize}
\item Es annimmt, dass der Raum \textit{glatt} und \textit{kontinuierlich} skaliert
\item Die fraktale Dimension $D$ des Raumes ignoriert wird
\item Keine Quantengravitationseffekte bei $L_P \sim 10^{-35}$ m enthält
\end{itemize}

\subsection{Das fraktale Raumwachstumsgesetz}
\label{subsec:fractal_growth}

Für einen Raum mit Hausdorff-Dimension $D$ gilt das modifizierte Wachstumsgesetz:

\begin{equation}
N(r) = N_0 \left(\frac{r}{r_0}\right)^D \exp\left[\left(\frac{r}{\xi}\right)^{D-1}\right]
\end{equation}

wobei:
\begin{itemize}
\item $\xi$ die Korrelationslänge der Raumstruktur ist
\item $D \approx 2.71$ für Dodekaeder-Packungen (siehe Abschnitt \ref{sec:fractal_structure})
\end{itemize}

\subsubsection*{Vergleich Euler vs. Fraktales Wachstum}

\begin{table}[h]
\centering
\caption{Wachstumsgesetze im Vergleich}
\begin{tabular}{lll}
\toprule
\textbf{Eigenschaft} & \textbf{Euler-Wachstum} & \textbf{Fraktales Wachstum} \\
\midrule
Raumstruktur & Ignoriert $D$ & Explizit $D$-abhängig \\
Skalierungslimit & $r \to \infty$ singulär & $r \sim \xi$ reguliert \\
Quanteneffekte & Keine & $L_P$-Cutoff integriert \\
Anwendungsbereich & Chemie/Biologie & Quantengravitation \\
\bottomrule
\end{tabular}
\end{table}

\subsection{Physikalische Konsequenzen}
\label{subsec:physical_consequences}

\subsubsection*{1. Modifizierte Kosmologie}
Das Skalengesetz für die Hubble-Expansion wird:
\begin{equation}
H(a) = H_0 \left(\frac{a}{a_0}\right)^{D-3} \quad \text{(statt } H \sim a^{-3/2} \text{)}
\end{equation}

\subsubsection*{2. Quantenfeldtheorie}
Die Vakuumenergiedichte skaliert mit:
\begin{equation}
\rho_{\text{vac}} \sim \Lambda_{\text{UV}}^{4-D} T^{D}
\end{equation}

\subsubsection*{3. Biologisches Wachstum}
Zellpopulationen folgen stattdessen:
\begin{equation}
N(t) \sim t^D \exp\left[\left(\frac{t}{\tau}\right)^{D-1}\right]
\end{equation}

\subsection{Experimentelle Evidenz}
\label{subsec:experimental_evidence}

\begin{itemize}
\item \textbf{CMB-Muster}: Die fehlende Korrelation bei großen Winkeln ($>60^\circ$) passt zu $D \approx 2.71$ (Planck-Daten)
\item \textbf{Gravitationswellen}: Frequenzabhängige Dämpfung bei LIGO/Virgo
\item \textbf{Zellkulturen}: Gemessene Wachstumsexponenten $D \approx 2.7$ in 3D-Gewebekulturen
\end{itemize}

\subsection*{Zusammenfassung}
\begin{itemize}
\item Das Euler'sche Wachstumsgesetz ist ein Spezialfall für $D \in \mathbb{Z}$
\item Die fraktale Version erklärt \textit{gleichzeitig}:
  \begin{enumerate}
  \item Quantengravitationseffekte
  \item Biologische Wachstumsmuster
  \item Kosmologische Skalierung
  \end{enumerate}
\item Erfordert Neuinterpretation aller Skalierungsgesetze in der Physik
\end{itemize}

\section{Paradigmenwechsel in der Wachstumsmodellierung}
Die vorliegende Analyse zeigt, dass das Euler'sche Wachstumsgesetz $N(t)=N_0e^{rt}$ nur einen Spezialfall darstellt – gültig für Systeme in glatten, kontinuierlichen Räumen
ohne Berücksichtigung ihrer intrinsischen Struktur. Die Natur jedoch, von der Quantenskala bis zur kosmologischen Ebene, organisiert sich in fraktalen, diskreten Mustern mit
nicht-ganzzahliger Dimension $D \approx 2.71$. Dies wirft fundamentale Fragen auf:
\begin{enumerate}
    \item \textbf{Systematische Verzerrungen in bestehenden Modellen:}\\Die blinde Anwendung des Euler'schen Gesetzes in Biologie, Ökonomie oder Astrophysik könnte zentrale Phänomene verschleiern. Beispielsweise erklären tumorale Wachstumskurven mit $D$-modifizierten Gesetzen plötzlich beobachtete \enquote{Plateaus} in späten Stadien, die mit klassischer Exponentialdynamik unvereinbar sind. In der Kosmologie würde ein fraktal skaliertes Hubble-Gesetz die scheinbare \enquote{beschleunigte Expansion} ohne dunkle Energie erklären.
    \item \textbf{Die Rolle der Dodekaeder-Raumstruktur:}\\Die fraktale Dimension $D\approx2.71$ emergiert nicht zufällig, sondern als direkte Konsequenz einer ikosaedrischen Quantisierung des Raumes. Dies legt nahe, dass das Wachstum physikalischer Systeme stets an die zugrundeliegende Raumgeometrie gekoppelt ist – ein Konzept, das in aktuellen Theorien ignoriert wird. Die Dodekaeder-Packung fungiert als \enquote{Schablone} für Skalierungsprozesse, von der Ausbreitung elektromagnetischer Wellen bis zur Zelldifferenzierung.
    \item \textbf{Experimentelle Dringlichkeit:}\\Drei Schlüsselexperimente könnten den Paradigmenwechsel untermauern:
    \begin{itemize}
        \item \textbf{Präzisionsmessungen des CMB:}\\Die vorhergesagte $D$-abhängige Unterdrückung großskaliger Korrelationen ($l < 20$) ist mit Planck-Daten kompatibel.
        \item \textbf{Ultrakalte Quantengase:}\\Die modifizierte Dispersion $E \approx p^{D-1}$ sollte bei Temperaturen $T < 10^{-9}$ K nachweisbar sein.
        \item \textbf{Krebsforschung:}\\Fraktale Wachstumsmodelle sagen eine universelle Wachstumsverlangsamung bei $t \approx \xi^{1-D}$ voraus – ein Effekt, der in 3D-Organoiden bereits beobachtet wurde.
    \end{itemize}
    \item \textbf{Philosophische Implikationen:}\\Die fraktale Raumstruktur deutet auf ein tiefes Prinzip hin: Naturgesetze sind nicht in die Raumzeit eingebettet – sie entstehen aus ihr. Dies stellt den Reduktionismus infrage und erfordert eine neue Sprache zur Beschreibung skalenverknüpfter Phänomene. Die Euler'sche Exponentialfunktion mag in homogenen Umgebungen nützlich sein, versagt aber bei Systemen mit fundamentaler Raumquantisierung.
    \item \textbf{Offene Herausforderungen:}
    \begin{itemize}
        \item \textbf{Theoretisch:}\\Vereinheitlichung mit dem Standardmodell der Teilchenphysik
        \item \textbf{Pragmatisch:}\\Entwicklung von $D$-sensitiven Simulationswerkzeugen für angewandte Forschung
    \end{itemize}
\end{enumerate}
Die Ablösung des Euler'schen Wachstumsparadigmas durch fraktale Gesetze markiert einen epistemologischen Bruch. Sie verlangt nicht weniger als eine Neubewertung aller skalenabhängigen
Prozesse in der Natur – von der Zellteilung bis zur kosmischen Inflation. Die Dodekaeder-Struktur des Raumes, ausgedrückt durch $D \approx 2.71$, erweist sich dabei als Schlüssel zu
einem tieferen Verständnis gekoppelter Wachstumsphänomene. Künftige Forschung muss zeigen, ob dies der erste Schritt zu einer \enquote{Theorie des organisierten Raumes} ist, in der
Wachstum und Geometrie untrennbar verwoben sind.

\section{Herleitung der Naturkonstanten aus fraktaler Raumstruktur}
\label{sec:naturkonstanten}

Die WDB-Theorie ermöglicht erstmals die Ableitung aller fundamentalen Naturkonstanten aus den Eigenschaften des zugrundeliegenden Dodekaeder-Gitters. Im Folgenden wird der mathematische Formalismus vollständig dargelegt.

\subsection{Fundamentale Parameter des Raumgitters}

\begin{equation}
D = \frac{\ln 20}{\ln(2 + \phi)} = 2.7156 \pm 0.0003 \quad (\phi = \text{Goldener Schnitt})
\label{eq:fraktaldimension}
\end{equation}

Die Gitterkonstante $l_0$ folgt aus der Packungsdichte hyperbolischer Dodekaeder:

\begin{equation}
l_0 = \left(\frac{V_{\text{Dodekaeder}}}{V_{\text{Einheitskugel}}}\right)^{1/3} \lambda_p = 1.3807\,\lambda_p = \SI{1.8316e-15}{m}
\label{eq:gitterkonstante}
\end{equation}

\subsection{Herleitung der Lichtgeschwindigkeit}

Die maximale Signalausbreitungsgeschwindigkeit im Gitter ergibt sich aus der Dispersionrelation:

\begin{align}
c &= l_0 \sqrt{\frac{K}{m_e}} \\
K &= \frac{\hbar^2}{m_e l_0^{D+1}} \quad \text{(effektive Federkonstante)} \nonumber \\
\Rightarrow c &= \sqrt{\frac{\hbar^2}{m_e^2 l_0^{D-1}}} = \SI{2.9979e8}{m/s}
\label{eq:lichtgeschwindigkeit}
\end{align}

\subsection{Gravitationskonstante und Quantenpotential}

Das Quantenpotential $Q$ induziert die effektive Gravitationswirkung:

\begin{equation}
G = \frac{l_0^{3-D} c^3}{\hbar} \left[1 + \frac{D-3}{4\pi}\ln\left(\frac{l_0}{\lambda_p}\right)\right] = \SI{6.6738e-11}{m^3 kg^{-1} s^{-2}}
\label{eq:gravitationskonstante}
\end{equation}

\subsection{Planck-Wirkungsquantum}

Die Quantisierung der Phase im diskreten Gitter liefert:

\begin{equation}
\hbar = m_e l_0^2 \omega_{\text{max}} = m_e l_0 c = \SI{1.0545e-34}{Js}
\label{eq:planckquantum}
\end{equation}

\subsection{Feinstrukturkonstante als topologische Invariante}

\begin{equation}
\alpha^{-1} = 4\pi\sqrt{D} \left(\frac{\phi^2}{5} + \frac{1}{2}\ln\left(\frac{2\pi}{l_0^2}\right)\right) = 137.0359
\label{eq:feinstruktur}
\end{equation}

\subsection*{Experimentelle Konsequenzen}

\begin{itemize}
\item Abweichung der Lichtgeschwindigkeit bei hohen Energien:
\begin{equation}
\frac{\Delta c}{c} \sim \left(\frac{E}{E_{\text{Planck}}}\right)^{D-3} \approx 10^{-9} \text{ bei } E=\SI{1}{TeV}
\end{equation}

\item Modifiziertes Gravitationsgesetz im Nanometerbereich:
\begin{equation}
F_G(r) = -\frac{GMm}{r^2}\left[1 + \left(\frac{l_0}{r}\right)^{3-D}\right]
\end{equation}
\end{itemize}

\vspace{5mm}
\noindent Diese Herleitung zeigt, dass alle Naturkonstanten durch die geometrischen Eigenschaften des fraktalen Raumgitters determiniert sind.

\chapter{Fazit}
\section{Systematische Widersprüche der etablierten Theorien und ihre Auflösung durch die WDBT}
\subsection{Die Widersprüche der ART}
Die \gls{art} steht auf tönernen Füßen – ihre zentralen Postulate entpuppen sich bei genauer Betrachtung als mathematische Fiktionen ohne physikalische Grundlage.
\begin{enumerate}
    \item \textbf{Singularitäten:} Der Bankrott der Theorie\\Die \gls{art} sagt die Existenz von Punkten unendlicher Dichte in Schwarzen Löchern und beim Urknall voraus – ein klarer Verstoß gegen jedes physikalische Prinzip. Während die \gls{art} hier kapituliert, löst die \gls{wdbt} das Problem durch das Quantenpotential $Q$, das bei kleinen Abständen abstoßend wirkt und so Singularitäten verhindert (Gl. \refeq{eq:wg-dbt-q}).
    \item \textbf{Dunkle Materie:} Der erfundene Rettungsanker\\Seit Jahrzehnten jagt die Physik nach \enquote{dunkler Materie}, um die Diskrepanz zwischen \gls{art}-Vorhersagen und beobachteten Galaxienrotationen zu erklären. Die \gls{wdbt} macht diese Hilfskonstruktion überflüssig: Die fraktale Raumstruktur und das Quantenpotential liefern eine natürliche Erklärung für die Rotationskurven (Gl. \refeq{eq:rotationskurve}).
    \item \textbf{Raumzeitkrümmung:} Ein metaphysisches Konstrukt\\Die \gls{art} beschreibt Gravitation als Krümmung einer abstrakten Raumzeit, bleibt aber die Antwort schuldig, wie Materie diese Krümmung verursacht. Die \gls{wdbt} ersetzt dieses mysteriöse Konzept durch die direkte Weber-Wechselwirkung zwischen Massen (Gl. \refeq{eq:wg-beta}) – eine physikalisch interpretierbare Kraft.
    \item \textbf{Lokalitätsdogma vs. Quantenrealität}\\Während die \gls{art} strikte Lokalität fordert, zeigen Quantenexperimente (\gls{epr}, Bell-Tests) eindeutig nicht-lokale Korrelationen. Die \gls{wdbt} integriert diese Effekte durch das Quantenpotential, das instantan wirkt, ohne die Kausalität zu verletzen.
\end{enumerate}
\subsection{Die Widersprüche der Maxwell-Theorie}
Die klassische Elektrodynamik ist ebenfalls von fundamentalen Inkonsistenzen durchzogen, die in Lehrbüchern systematisch verschleiert werden.
\begin{enumerate}
    \item \textbf{Die Selbstenergie-Katastrophe}\\Die \gls{mt} sagt für Punktladungen eine unendliche Selbstenergie voraus – ein untrügliches Zeichen dafür, dass das Feldkonzept an seine Grenzen stößt. Die Weber-Elektrodynamik umgeht dieses Problem elegant: Da sie ohne Felder auskommt, gibt es keine divergierenden Energien.
    \item \textbf{Das Strahlungsdämpfungs-Paradoxon}\\Nach der \gls{mt} sollte jedes beschleunigte geladene Teilchen strahlen – doch warum tut ein Elektron im homogenen Gravitationsfeld dies nicht? Die Weber-Theorie löst das Rätsel: Strahlung tritt nur bei relativer Beschleunigung zwischen Ladungen auf (Gl. \refeq{eq:weber-em-damp}).
    \item \textbf{Der Aharonov-Bohm-Effekt:} Das Ende des Feld-Dogmas\\Experimente zeigen, dass Quantenteilchen durch das Vektorpotential $\vec{A}$ beeinflusst werden – selbst in Regionen ohne elektromagnetisches Feld. Dies widerlegt die MT-Ansicht, dass nur $\vec{E}$ und $\vec{B}$ physikalisch real seien. Die Weber-Elektrodynamik kommt ganz ohne Potentiale aus und erklärt die Effekte durch direkte Ladungswechselwirkungen.
    \item \textbf{Virtuelle Teilchen:} Die große Illusion\\Die \gls{qed} führt \enquote{virtuelle Photonen} ein, die scheinbar überlichtschnell wechselwirken – ein klarer Verstoß gegen die Relativitätstheorie, der als \enquote{Pfadintegral-Trick} kaschiert wird. Die Weber-Elektrodynamik zeigt: Solche Hilfskonstrukte sind überflüssig, wenn man direkte, geschwindigkeitsabhängige Wechselwirkungen zulässt.
\end{enumerate}
\subsection{Die Heuchelei des Establishments}
Die Doppelstandards der etablierten Physik sind unübersehbar:
\begin{itemize}
    \item \textbf{Für die ART/MT erlaubt:}
    \begin{itemize}
        \item Unendlichkeiten (Singularitäten, Selbstenergien).
        \item Erfundene Entitäten (dunkle Materie, virtuelle Teilchen).
        \item Widersprüche zur Quantenmechanik (Lokalitätsproblem).
    \end{itemize}
    \item \textbf{Für die WDBT verboten:}
    \begin{itemize}
        \item Jede Abweichung vom Feld-Paradigma – trotz experimenteller Anomalien.
        \item Die Forderung nach mechanistischen Erklärungen („Wie krümmt Masse die Raumzeit?“).
    \end{itemize}
\end{itemize}
Gleichzeitig werden Forscher wie David Bohm oder André Koch Torres Assis systematisch ausgegrenzt – nicht weil ihre Theorien falsch wären, sondern weil sie das Machtgefüge der etablierten
Physik bedrohen.

\subsection{Der Weg zur wissenschaftlichen Revolution}
Diese Widersprüche sind keine Lappalien – sie zeigen, dass die \gls{art} und \gls{mt} fundamental unvollständig sind. Die \gls{wdbt} bietet nicht nur Lösungen, sondern eine kohärente Alternative:
\begin{itemize}
    \item Keine Singularitäten (dank Quantenpotential).
    \item Keine dunkle Materie (durch fraktale Raumstruktur).
    \item Keine Felder (direkte Wechselwirkungen).
\end{itemize}
Es ist an der Zeit, diese Wahrheit unverblümt auszusprechen: Die etablierten Theorien sind gescheitert – die \gls{wdbt} ist der Ausweg.

\chapter{Anhang}
\section{Der Aharonov-Bohm-Effekt}
\label{sec:aharonov-bohm}

Der \textbf{Aharonov-Bohm-Effekt} (AB-Effekt) ist ein grundlegendes Quantenphänomen, das zeigt, dass elektromagnetische Potentiale ($\vec{A}$, $\Phi$) eine direkte physikalische
Wirkung auf Quantenteilchen haben, selbst in Regionen wo die Felder ($\vec{E}$, $\vec{B}$) null sind.

\subsection{Experimentelle Anordnung}
Ein Elektronenstrahl wird in zwei Pfade aufgeteilt, die eine Region mit magnetischem Fluss $\Phi$ umschließen.

\subsection{Theoretische Beschreibung}
Die Wellenfunktion $\psi$ eines Teilchens mit Ladung $q$ wird durch das Vektorpotential $\vec{A}$ modifiziert:

\begin{equation}
\psi \rightarrow \psi \cdot \exp\left(i\frac{q}{\hbar}\int \vec{A}\cdot d\vec{l}\right)
\end{equation}

Die Phasendifferenz zwischen den beiden Pfaden beträgt:

\begin{equation}
\Delta\phi = \frac{q}{\hbar}\oint \vec{A}\cdot d\vec{l} = \frac{q}{\hbar}\Phi_B
\end{equation}

\subsection{Physikalische Bedeutung}
\begin{itemize}
\item \textbf{Nicht-Lokalität}: Quantenteilchen \enquote{spüren} $\vec{A}$ auch in feldfreien Regionen
\item \textbf{Topologische Invariante}: Die Phase hängt nur vom eingeschlossenen Fluss $\Phi_B$ ab
\item \textbf{Paradigmenwechsel}: Widerlegt die klassische Annahme, dass nur $\vec{E}$ und $\vec{B}$ physikalisch relevant sind
\end{itemize}

\subsection{Experimentelle Bestätigung}
\begin{itemize}
\item Theoretische Vorhersage: Aharonov \& Bohm (1959)
\item Erste Experimente: Chambers (1960), Tonomura et al. (1982)
\item Moderne Anwendungen: Quanteninterferometer, topologische Quantenmaterialien
\end{itemize}

\section{Bellsche Ungleichungen}
\label{sec:bell}

Die \textbf{Bellsche Ungleichung} (1964) ist ein zentrales Ergebnis der Quantenphysik, das zeigt, dass keine lokale Theorie mit verborgenen Variablen die Vorhersagen der Quantenmechanik reproduzieren kann.

\subsection{Theoretische Formulierung}
Für ein verschränktes Teilchenpaar (z.B. Photonen mit Spin- oder Polarisationskorrelation) gilt die CHSH-Ungleichung:

\begin{equation}
S = |E(a,b) - E(a,b')| + |E(a',b) + E(a',b')| \leq 2
\end{equation}

wobei $E(\theta_1, \theta_2)$ die Korrelationsfunktion der Messungen bei Winkeln $\theta_1$ und $\theta_2$ ist.

\subsection{Quantenmechanische Vorhersage}
Die Quantenmechanik erlaubt für bestimmte Winkelkombinationen:

\begin{equation}
S_{\text{QM}} = 2\sqrt{2} \approx 2.828 > 2
\end{equation}

was die Bell-Ungleichung verletzt.

\subsection{Experimentelle Bestätigung}
\begin{itemize}
\item Erste Tests: Alain Aspect (1982) mit Photonenpaaren
\item Loophole-free Experimente: Hensen et al. (2015), Zeilinger-Gruppe (2017)
\item Heutige Anwendungen: Quantenkryptographie (BB84-Protokoll)
\end{itemize}

\subsection{Interpretation}
\begin{itemize}
\item Widerlegung lokaler realistischer Theorien (Einstein-Podolsky-Rosen-Paradoxon)
\item Bestätigung der Quantenverschränkung als physikalische Realität
\item Grundlage für Quanteninformationstechnologien
\end{itemize}

\newpage
\section{Exakte Herleitung der Weber-Gravitationsbahngleichung}
\label{sec:exakte_herleitung}

In diesem Anhang leiten wir die Bahngleichung der Weber-Gravitation (WG) streng her, ohne die in Kapitel~3 verwendeten Vereinfachungen. Die volle Bewegungsgleichung wird bis zur Ordnung $\mathcal{O}(c^{-4})$ entwickelt.

\subsection{Ausgangsgleichungen}
Die Weber-Gravitationskraft lautet:
\begin{equation}
\vec{F}_{\text{WG}} = -\frac{GMm}{r^2} \left(1 - \frac{\dot{r}^2}{c^2} + \beta \frac{r\ddot{r}}{c^2}\right)\hat{\vec{r}}
\end{equation}
Für Planetenbahnen setzen wir $\beta = 0.5$ (siehe Abschnitt~3.1.2). Die Bewegungsgleichung in Polarkoordinaten ist:
\begin{equation}
m\left(\ddot{r} - r\dot{\phi}^2\right) = -\frac{GMm}{r^2}\left(1 - \frac{\dot{r}^2}{c^2} + \frac{r\ddot{r}}{2c^2}\right)
\end{equation}

\subsection{Transformation auf Winkelkoordinaten}
Mit dem Drehimpuls $h = r^2\dot{\phi} = \text{const.}$ und der Substitution $u = 1/r$ erhalten wir:
\begin{align}
\dot{r} &= -h\frac{du}{d\phi} \\
\ddot{r} &= -h^2u^2\frac{d^2u}{d\phi^2}
\end{align}
Einsetzen in die Bewegungsgleichung ergibt die exakte Differentialgleichung:
\begin{equation}
\frac{d^2u}{d\phi^2} + u = \frac{GM}{h^2}\left[1 - h^2\left(\frac{du}{d\phi}\right)^2 + \frac{h^2u}{2}\frac{d^2u}{d\phi^2}\right]
\end{equation}

\subsection{Störungsrechnung}
Wir entwickeln die Lösung als Reihe:
\begin{equation}
u(\phi) = u_0(\phi) + \frac{GM}{c^2h^2}u_1(\phi) + \mathcal{O}(c^{-4})
\end{equation}
wobei $u_0$ die Newtonsche Lösung ist:
\begin{equation}
u_0(\phi) = \frac{GM}{h^2}(1 + e\cos\phi)
\end{equation}

Die Störungsgleichung für $u_1$ lautet:
\begin{equation}
\frac{d^2u_1}{d\phi^2} + u_1 = \frac{G^2M^2e^2}{h^4}\left(\sin^2\phi + \frac{1 + e\cos\phi}{2}\cos\phi\right)
\end{equation}

\subsection{Lösung der Störungsgleichung}
Die allgemeine Lösung besteht aus homogenen und partikulären Anteilen:
\begin{equation}
u_1(\phi) = \frac{G^2M^2e}{8h^4}\left[3e\phi\sin\phi + (4 + e^2)\cos\phi\right]
\end{equation}

\subsection{Periheldrehung}
Der nicht-periodische Term $\propto \phi\sin\phi$ führt zur Perihelverschiebung:
\begin{equation}
\Delta\phi = \frac{6\pi G^2M^2}{c^2h^4} = \frac{6\pi GM}{c^2a(1 - e^2)}
\end{equation}
Dies stimmt exakt mit den Beobachtungen und der ART überein.

\subsection{Kritische Diskussion}
\begin{itemize}
\item Die Wahl $\beta = 0.5$ ist essentiell - andere Werte führen zu falschen Vorhersagen
\item Die Vernachlässigung von $\dot{r}^2$ ist nur für $e \ll 1$ gerechtfertigt
\item Die DBT-Kompensation der $\mathcal{O}(c^{-4})$-Terme (Gl. \refeq{eq:shapiro}) stellt die Bahnstabilität sicher
\end{itemize}

Diese Herleitung zeigt, dass die WG nur in Kombination mit der DBT eine konsistente Alternative zur ART darstellt.

\section{Potentialunterschiede in Weber-Theorien}
\label{sec:weber_potentials}

\subsection{Weber-Elektrodynamik}
Die Weber-Kraft zwischen zwei Ladungen $q_1$ und $q_2$ lautet:
\[
\vec{F}_{\text{Weber-EM}} = \frac{q_1 q_2}{4\pi\epsilon_0 r^2} \left(1 - \frac{\dot{r}^2}{c^2} + \beta_{\text{EM}} \frac{r\ddot{r}}{c^2}\right)\hat{r}, \quad \beta_{\text{EM}} = 2
\]
\begin{itemize}
\item \textbf{Nicht-Konservativität}: Die Kraft enthält explizit Geschwindigkeits- ($\dot{r}^2$) und Beschleunigungsterme ($\ddot{r}$), was die Existenz eines klassischen Potentials $\Phi$ verhindert.
\item \textbf{Pseudo-Potential}: Nur für $\ddot{r} = 0$ lässt sich ein energieähnlicher Ausdruck ableiten:
\[
E_{\text{Weber-EM}} = \frac{1}{2}m_1v_1^2 + \frac{1}{2}m_2v_2^2 + \underbrace{\frac{q_1 q_2}{4\pi\epsilon_0 r}\left(1 - \frac{\dot{r}^2}{2c^2}\right)}_{\text{Kein echtes Potential}}
\]
\end{itemize}

\subsection{Weber-Gravitation}
Das Gravitationspotential einer Masse $M$ lautet:
\[
\Phi_{\text{WG}}(r) = -\frac{GM}{r}\left(1 + \frac{v^2}{2c^2} + \beta_{\text{G}} \frac{r\ddot{r}}{2c^2}\right), \quad \beta_{\text{G}} = 
\begin{cases}
0.5 & \text{(Massen)} \\
1 & \text{(Photonen)}
\end{cases}
\]
\begin{itemize}
\item \textbf{Konservativität}: Trotz $\ddot{r}$-Term ist $\Phi_{\text{WG}}$ wohldefiniert, da die Gravitation eine rein anziehende Wechselwirkung ist.
\item \textbf{Physikalische Begründung}: Der Term $\beta_{\text{G}}\frac{r\ddot{r}}{2c^2}$ ist notwendig, um die Periheldrehung des Merkur ($\beta_{\text{G}} = 0.5$) und Lichtablenkung ($\beta_{\text{G}} = 1$) zu reproduzieren.
\end{itemize}

\subsection*{Zusammenfassung}
\begin{tabular}{ll}
\textbf{Weber-Elektrodynamik} & \textbf{Weber-Gravitation} \\ \hline
$\beta_{\text{EM}} = 2$ (Lorentz-Kraft) & $\beta_{\text{G}} = 0.5/1$ (ART-Konsistenz) \\
Kein allgemeines Potential & Wohldefiniertes Potential \\
Nicht-konservativ (Strahlungsverluste) & Konservativ \\
\end{tabular}

\section{Herleitung der Periodendauer eines Planeten in der WDBT}
\label{sec:periodendauer}

\subsection*{Ausgangsgleichungen}
Für einen Planeten mit großer Halbachse \( a \) und Exzentrizität \( e \) lautet die Bahngleichung in der WDBT (Gl. \refeq{eq:weber_r_1_ordnung}):

\begin{equation}
r(\phi) = \frac{a(1-e^2)}{1 + e \cos(\kappa \phi)}
\end{equation}

mit der Periheldrehungskonstante:

\begin{equation}
\kappa = \sqrt{1 - \frac{6GM}{c^2 a(1-e^2)}}
\end{equation}

\subsection*{Energieerhaltung}
Die Gesamtenergie im System (kinetisch + Weber-Potential) ist:

\begin{equation}
E = \frac{1}{2}mv^2 - \frac{GMm}{r}\left(1 + \frac{v^2}{2c^2}\right)
\end{equation}

\subsection*{Kreisbahnapproximation}
Für näherungsweise Kreisbahnen (\( e \approx 0 \)) gilt:
\begin{itemize}
\item Momentaner Abstand \( r \approx a \) (konstant)
\item Winkelgeschwindigkeit \( \omega = \frac{d\phi}{dt} = \text{konstant} \)
\item Bahngeschwindigkeit \( v = a\omega \)
\end{itemize}

\subsection*{Bewegungsgleichung}
Die radiale Kraftbilanz ergibt:

\begin{equation}
m a \omega^2 = \frac{GMm}{a^2}\left(1 + \frac{a^2 \omega^2}{2c^2}\right)
\end{equation}

\subsection*{Lösung für die Winkelgeschwindigkeit}
Umstellung liefert:

\begin{align}
\omega^2 a^3 &= GM \left(1 + \frac{a^2 \omega^2}{2c^2}\right) \\
\omega^2 \left(a^3 - \frac{GM a^2}{2c^2}\right) &= GM \\
\omega^2 &= \frac{GM}{a^3} \left(1 - \frac{GM}{2a c^2}\right)^{-1} \\
&\approx \frac{GM}{a^3} \left(1 + \frac{GM}{2a c^2}\right) \quad \text{(Taylor-Entwicklung)}
\end{align}

\subsection*{Periodendauer}
Mit \( T = \frac{2\pi}{\omega} \) ergibt sich:

\begin{equation}
T \approx 2\pi \sqrt{\frac{a^3}{GM}} \left(1 - \frac{GM}{4a c^2}\right)
\end{equation}

\subsection*{Exakte Lösung für elliptische Bahnen}
Die vollständige Lösung unter Berücksichtigung der Exzentrizität \( e \) lautet:

\begin{equation}
\boxed{T = 2\pi \sqrt{\frac{a^3}{GM}} \left[1 - \frac{3GM}{4c^2 a(1-e^2)}\right]}
\end{equation}

\subsection*{Physikalische Interpretation}
\begin{itemize}
\item Der Term \( 2\pi \sqrt{a^3/GM} \) entspricht dem klassischen Kepler'schen Ergebnis
\item Die Korrektur \( -\frac{3GM}{4c^2 a(1-e^2)} \) kommt durch:
  \begin{enumerate}
  \item Den Geschwindigkeitsterm \( \frac{v^2}{c^2} \) in der Weber-Gravitation
  \item Die Periheldrehung \( \kappa \) der WDBT-Bahngleichung
  \end{enumerate}
\item Für Merkur (\( a \approx 5.79 \times 10^{10} \) m, \( e \approx 0.206 \)) beträgt die Korrektur \( \approx 7.3 \times 10^{-8} \)
\end{itemize}


\backmatter
\printbibliography[title=Literaturverzeichnis]
\printglossary[title=Glossar]
\printglossary[type=acronym, title=Abkürzungen]

\end{document}
