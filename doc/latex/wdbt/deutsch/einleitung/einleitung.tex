\chapter{Grundlagen einer alternativen Quantengravitation}
\section{Motivation}
Die moderne Physik steht vor grundlegenden Widersprüchen: Während die \gls{art} die Gravitation als Krümmung der Raumzeit beschreibt, basiert die \gls{srt} auf idealisierten Inertialsystemen,
die in einer gekrümmten Raumzeit streng genommen nicht existieren können. Dieser Konflikt wirft Fragen auf – etwa zur Natur der Lichtgeschwindigkeit, die in der \gls{srt} absolut ist, in der
\gls{art} jedoch lokal variabel.
\begin{quote}
    \enquote{Einstein's postulates contain inherent contradictions when applied to real gravitational systems, challenging the universality of special relativity.} \cite{Rubcic1998}
\end{quote}
Hinzu kommen ungelöste Probleme der Quantenmechanik: der Welle-Teilchen-Dualismus, der \enquote{Kollaps} der Wellenfunktion bei Messungen und nicht-lokale Verschränkung. Selbst erfolgreiche
Theorien wie die \gls{qed} postulieren scheinbar paradoxe Phänomene, etwa virtuelle Photonen mit Überlichtgeschwindigkeit im Pfadintegralformalismus.

Diese Spannungen deuten darauf hin, dass die etablierten Modelle möglicherweise nur Annäherungen an eine tiefere Realität sind. Statt Dogmen zu folgen, sollten wir alternative Perspektiven
prüfen – wie die Weber-Elektrodynamik oder die \gls{dbt}, die in diesem Buch vorgestellt werden.
\begin{quote}
    \enquote{The observer-dependent collapse of the wavefunction is not a fundamental feature of nature but a limitation of the standard interpretation.} \cite{bohm1952}
\end{quote}

\subsection{Dogmatismus und blinde Flecken der modernen Physik}
Die heutige Physik leidet unter einer paradoxen Situation: Einerseits werden etablierte Theorien wie die \gls{art} oder Quantenfeldtheorie kaum hinterfragt, obwohl sie fundamentale Schwächen
aufweisen – Singularitäten in Schwarzen Löchern, unendliche Selbstenergien von Teilchen oder die Notwendigkeit \enquote{dunkler} Entitäten. Andererseits werden unorthodoxe Ansätze oft bereits
im Peer-Review aussortiert, obwohl sie Lösungen für diese Probleme bieten könnten.

Ein Beispiel ist die Interpretation der \gls{cmb} als Beweis für den Urknall. Alternative Erklärungen – etwa thermische Gleichgewichtsprozesse in Plasmen – werden kaum diskutiert, obwohl sie
ohne Singularitäten auskommen. Ähnlich verhält es sich mit der Rotverschiebung von Galaxien, die nicht zwingend auf eine Expansion des Universums hindeuten muss.
\begin{quote}
    \enquote{Theoretical physics has become stuck in a paradigm that values mathematical elegance over empirical testability, leading to a stagnation of genuine progress.} \cite{Smolin2006}
\end{quote}

\subsection{Spekulation statt Fortschritt}
Seit den revolutionären Durchbrüchen der Quantenmechanik und Relativitätstheorie vor einem Jahrhundert gab es kaum vergleichbare Fortschritte. Stattdessen dominieren spekulative Konzepte wie
höhere Dimensionen oder Multiversen, die empirisch kaum überprüfbar sind.

Doch Wissenschaft sollte sich auf beobachtbare Phänomene konzentrieren. Die Weber-Elektrodynamik zeigt, wie sich elektromagnetische Effekte ohne Felder beschreiben lassen – durch direkte
Wechselwirkungen zwischen Ladungen. Solche Ansätze könnten den Weg zu einer konsistenteren Physik ebnen.

\subsection{Alternative Theorien}
Ein zentrales Problem der modernen Physik liegt in ihrem übermäßigen Vertrauen in die Mathematik. Nur weil etwas mathematisch formulierbar ist,
muss es noch lange nicht der physikalischen Realität entsprechen. Doch statt diese Grenzen anzuerkennen, werden grundlegende Prinzipien der klassischen Physik – wie Energieerhaltung
oder die Gesetze der Thermodynamik – zugunsten abstrakter Gleichungen aufgegeben. Die \gls{art} beispielsweise postuliert eine dynamische Raumzeit, die scheinbar Energie aus dem Nichts
erzeugen oder vernichten kann. Wo bleibt da die strenge Bilanz der Physik?

Konkrete Widersprüche zeigen sich in der Praxis: Nach der \gls{art} müssten Planeten durch die Abstrahlung von Gravitationswellen Energie verlieren – doch warum sind Planetenbahnen dann über
Milliarden Jahre stabil? Wenn die Raumzeit als elastisches Gebilde beschrieben wird, das sich verformen und bewegen lässt: Welche Kraft verrichtet hier Arbeit, und woher kommt die Energie
dafür? Die Standarderklärungen bleiben vage oder weichen auf mathematische Tricks aus.

Auch die vermeintlichen Beweise für den Urknall sind keineswegs so eindeutig, wie oft behauptet wird. Die kosmische \gls{cmb} wird automatisch als Echo des Urknalls
interpretiert – doch es gibt alternative Erklärungen, etwa thermische Gleichgewichtsprozesse oder Streuphänomene.
\begin{quote}
    \enquote{The interpretation of cosmic microwave background as proof of the Big Bang ignores alternative explanations, such as intrinsic redshifts in plasma cosmology.} \cite{Arp1998}
\end{quote}
Ebenso könnte die Rotverschiebung von Galaxien nicht nur durch Expansion, sondern auch durch andere Mechanismen verursacht werden. Selbst Phänomene wie die Lichtablenkung oder der
Shapiro-Effekt lassen sich ohne \gls{art} erklären, wenn man alternative Gravitationsmodelle zulässt.
\begin{quote}
    \enquote{Weber's formulation of electrodynamics provides a consistent framework for gravitational phenomena without invoking curved spacetime.} \cite{WeberElectrodynamics}
\end{quote}
In diesem Buch sollen solche alternativen Erklärungen aufgezeigt werden. Die Physik darf nicht bei mathematischen Dogmen stehen bleiben – sie muss sich wieder auf Logik, Experiment und
echte Kausalität besinnen.

\section{Abweichende Perspektiven in der Physik: Licht, Relativität und alternative Modelle}
\subsection{Feynmans Teilchenmodell des Lichts}
Richard Feynman argumentierte, dass selbst Interferenzphänomene durch Teilchen (Photonen) erklärbar sind – ohne Wellen. Dies wirft die Frage auf: Ist der Welle-Teilchen-Dualismus wirklich
notwendig, oder spiegelt er nur die Grenzen unserer Modelle wider?

\subsection{Widersprüche in der QED: Überlichtschnelle Photonen und Pfadintegrale}
Der Pfadintegralformalismus der \gls{qed} summiert über alle möglichen Photonenpfade – inklusive solcher mit Überlichtgeschwindigkeit. Mathematisch führt dies zu korrekten Vorhersagen,
doch physikalisch bleibt unklar:
\begin{itemize}
    \item Wenn Photonen virtuell schneller als Licht sein können, widerspricht dies nicht der \gls{srt}?
    \item Ist die Lichtgeschwindigkeit wirklich eine absolute Grenze, oder nur ein makroskopischer Effekt?
\end{itemize}

\subsection{Energieabhängige Lichtgeschwindigkeit? Experimentelle Hinweise}
Einige alternative Theorien (z. B. Schleifenquantengravitation oder VSL-Modelle) schlagen vor, dass die Lichtgeschwindigkeit von der Photonenenergie abhängen könnte.
Mögliche Indizien:

\begin{itemize}
    \item Gammablitze mit extrem hohen Energien zeigen minimale Laufzeitunterschiede (z. B. Fermi-Teleskop-Daten).
    \item Quantengravitationseffekte könnten bei hohen Energien zu Dispersion führen.
\end{itemize}

\begin{quote}
    \enquote{The constancy of the speed of light is not an immutable law but a parameter that may vary under extreme conditions, offering solutions to cosmological puzzles.} \cite{Magueijo2003}
\end{quote}

\section{Die Entwicklung des Wellenkonzepts in der Physik}
Das Verständnis von Wellen in der Physik hat sich im Laufe der Zeit radikal gewandelt. Während klassische Wellen wie Schall oder Wasserwellen als Störungen eines materiellen Mediums
beschrieben werden konnten, führten elektromagnetische Wellen und Quantenphänomene zu grundlegenden Umbrüchen. Maxwell zeigte 1865, dass Licht sich als elektromagnetische Welle
auch ohne Äther ausbreitet – was die Frage aufwarf, wie Energie ohne Trägermedium transportiert wird. Die \gls{srt} etablierte die Lichtgeschwindigkeit
als absolute Grenze, während die \gls{art} sie als lokal variabel beschreibt – ein scheinbarer Widerspruch, den alternative Theorien wie die Weber-Elektrodynamik
zu lösen versuchen.

Die Quantenphysik revolutionierte das Wellenkonzept weiter: De Broglie verband Teilchen- und Welleneigenschaften, und die \gls{qed} beschreibt Photonen als Felder mit überlichtschnellen
Pfadintegral-Komponenten. Doch diese mathematische Eleganz wirft physikalische Deutungsprobleme auf – etwa die Rolle des Beobachters beim Kollaps der Wellenfunktion oder die nicht-lokale
Natur der Quantenverschränkung. Auch Gravitationswellen in der \gls{art} bleiben rätselhaft: Wenn Raumzeit als schwingendes Medium gilt, woher stammt die Energie für ihre Verformung?

Diese Widersprüche zeigen, dass die etablierten Theorien möglicherweise nur Annäherungen an eine tiefere Wahrheit sind. Die Physik steht vor grundlegenden Fragen: Ist die Lichtgeschwindigkeit
wirklich konstant? Wie vereint man Quantenphysik und Relativität? Gibt es eine objektive Realität jenseits des Beobachters? Die Suche nach Antworten könnte eine neue wissenschaftliche
Revolution auslösen – eine, die unser Verständnis von Wellen und der fundamentalen Natur der Wirklichkeit grundlegend verändert.

\section{Wellenphänomene: Die Dualität von instantaner Ganzheit und lokaler Ausbreitung}
Wellen besitzen eine einzigartige Doppelnatur, die sich durch die gesamte Physik zieht. Einerseits zeigen sie lokale Ausbreitungsphänomene, andererseits weisen sie instantane globale
Eigenschaften auf, die sich jeder klassischen Kausalität entziehen. Diese Dualität wird besonders deutlich, wenn wir die fundamentalen Wechselwirkungen betrachten.

Die newtonsche Mechanik postuliert mit \enquote{actio = reactio} eine instantane Fernwirkung - eine Kraft wirkt unverzüglich zwischen zwei Körpern. Mathematisch ausgedrückt:
\begin{equation}
    \vec{F}{12} = -\vec{F}{21}
\end{equation}
Diese Gleichung beschreibt eine momentane Wechselwirkung ohne zeitliche Verzögerung. Ähnlich verhält es sich im Coulombschen Gesetz:
\begin{equation}
    \vec{F} = \frac{1}{4\pi\epsilon_0}\frac{q_1q_2}{r^2}\hat{\vec{r}}
\end{equation}
Diese Fernwirkungstheorien funktionieren in ihrem Gültigkeitsbereich bemerkenswert gut, wie die erfolgreiche Beschreibung planetarer Bewegungen zeigt. Sie bleiben aber unvollständig,
da sie den Energie- und Impulstransport zwischen den wechselwirkenden Körpern nicht erklären können.

Interferenzphänomene offenbaren eine weitere tiefgreifende Eigenschaft von Wellen. Betrachten wir das Doppelspaltexperiment: Die Wahrscheinlichkeitsdichte an einem Punkt x auf dem
Schirm ergibt sich aus der Überlagerung der Teilwellen:
\begin{equation}
    |\Psi(x)|^2 = |\psi_1(x) + \psi_2(x)|^2
\end{equation}
Dieses Muster erfüllt einen energetischen Zweck - es minimiert die Gesamtenergie des Systems. Die Welle \enquote{weiß} instantan, wie sie sich verteilen muss, um dieses Minimum zu erreichen,
ohne dass eine lokale Wechselwirkung dies erklären könnte.

Die Weber-Elektrodynamik bietet hier einen interessanten Brückenschlag. Sie erweitert das Coulombsche Gesetz um geschwindigkeits- und beschleunigungsabhängige Terme:
\begin{equation}
    \label{eq:weber_em_skalar}
    \vec{F} = \frac{q_1q_2}{4\pi\epsilon_0r^2}\left[1 - \frac{\dot{r}^2}{c^2} + \frac{2r\ddot{r}}{c^2}\right]\hat{\vec{r}}
\end{equation}
Diese Gleichung beschreibt:
\begin{enumerate}
    \item Den statischen Coulomb-Term (instantane Fernwirkung)
    \item Einen geschwindigkeitsabhängigen Term (magnetische Effekte)
    \item Einen beschleunigungsabhängigen Term (Strahlungswiderstand)
\end{enumerate}
Die instantane Komponente bleibt erhalten, wird aber durch retardierte Effekte ergänzt. Dies zeigt, wie eine Theorie sowohl momentane Wechselwirkungen als auch Ausbreitungseffekte vereinen kann.
Diese Form soll als \textbf{\enquote{skalare Form}} bezeichnet werden.

Die energetische Betrachtung offenbart den tieferen Sinn dieser Doppelnatur. Eine Welle im Gleichgewichtszustand minimiert stets die Gesamtenergie:

\begin{equation}
    \delta \int \left[\frac{\hbar^2}{2m}|\nabla\Psi|^2 + V|\Psi|^2\right] d^3x = 0    
\end{equation}

Diese Bedingung wird global instantan erfüllt, während sich lokale Störungen mit endlicher Geschwindigkeit ausbreiten. Die Weber-Elektrodynamik zeigt, dass ähnliche Prinzipien auch in der
klassischen Physik wirksam sind - die instantane Komponente sorgt für die Energieerhaltung, während die retardierten Terme den Energietransport beschreiben.

Die Konsequenzen dieser Betrachtung sind weitreichend:
\begin{enumerate}
    \item Instantane Effekte sind nicht unbedingt unphysikalisch, sondern können energetische Zwangsbedingungen darstellen
    \item Die Ausbreitungsgeschwindigkeit beschreibt nur den Energietransport, nicht die globale Struktur
    \item Fernwirkungstheorien enthalten einen wahren Kern, der in modernen Theorien oft vernachlässigt wird
\end{enumerate}
Diese Erkenntnisse ebnen den Weg für ein neues Verständnis von Wellenphänomenen, das die scheinbaren Widersprüche zwischen instantaner Ganzheit und lokaler Kausalität auflösen könnte.

\section{Das erweiterte Kausalitätskonzept: Eine synthetische Betrachtung von instantaner Ganzheit und lokaler Dynamik}
Die herkömmliche Vorstellung von Kausalität als linearer Ursache-Wirkungs-Kette mit strenger Lokalität und endlicher Ausbreitungsgeschwindigkeit erweist sich bei genauer Betrachtung
der Wellenphänomene als zu eng gefasst. Die Physik steht vor dem Paradoxon, dass einerseits die Relativitätstheorie eine maximale Signalgeschwindigkeit postuliert, während andererseits
Quantenphänomene wie Verschränkung und das \gls{epr} \cite{EPR1935} nahelegen, dass bestimmte Korrelationen instantan über beliebige Distanzen hinweg bestehen können. Dieses Spannungsfeld verlangt
nach einem neuen, umfassenderen Kausalitätsbegriff.

Der Schlüssel zum Verständnis liegt in der Anerkennung zweier komplementärer, aber gleichberechtigter Kausalitätsebenen, die gemeinsam die Dynamik physikalischer Systeme bestimmen.
Auf der einen Seite steht die lokale Kausalität, wie sie durch die Maxwellschen Gleichungen oder die relativistische Feldtheorie beschrieben wird. Diese Ebene regelt den Energietransport
und die Ausbreitung von Störungen im Raum mit endlicher Geschwindigkeit. Die bekannte Lichtkegel-Struktur der Raumzeit mit ihrer strikten Trennung von zeitartigen, lichtartigen und
raumartigen Abständen gehört in diesen Bereich.

Parallel dazu existiert eine systemische Kausalitätsebene, die für die globale Organisation des Wellenfeldes verantwortlich ist. Diese manifestiert sich in Phänomenen wie der
spontanen Symmetriebrechung, dem Aharonov-Bohm-Effekt (Abschnitt \ref{sec:aharonov-bohm}) oder den bereits erwähnten verschränkten Quantenzuständen. Während die lokale Kausalität durch Differentialgleichungen mit
Randbedingungen beschrieben wird, folgt die systemische Kausalität einem Variationsprinzip, dass das gesamte System simultan optimiert.
Das Quantenpotential \cite{bohm1952}
\begin{equation}
    \label{eq:bohm_potenzial}
    Q(\vec{r},t) = -\frac{\hbar^2}{2m} \frac{\nabla^2 \sqrt{\rho(\vec{r},t)}}{\sqrt{\rho(\vec{r},t)}}
\end{equation}
ist hierfür ein ausgezeichnetes Beispiel - es wirkt nicht durch lokale Wechselwirkungen, sondern durch die instantane Anpassung der gesamten
Wellenfunktion an die globalen Randbedingungen.

Die Weber-Elektrodynamik mit ihrer charakteristischen Kraftgleichung (Gl. \refeq{eq:weber_em_skalar}) zeigt exemplarisch, wie beide Kausalitätsebenen in einer konsistenten Theorie vereint werden können.
Der erste Term repräsentiert die systemische Komponente - eine instantane Coulomb-Wechselwirkung, die für die grundlegende Struktur der Fernwirkung sorgt. Die zusätzlichen Geschwindigkeits- und
Beschleunigungsterme hingegen beschreiben die lokale Dynamik des Energietransports, einschließlich retardierter Effekte und Strahlungsphänomene.

Diese duale Struktur der Kausalität löst zahlreiche konzeptionelle Probleme der modernen Physik. So erklärt sie beispielsweise, warum im Doppelspaltexperiment das Interferenzmuster
auch dann entsteht, wenn die Teilchen nacheinander das Experiment durchlaufen - die systemische Kausalität \enquote{kennt} die Gesamtanordnung und organisiert die Wahrscheinlichkeitsverteilung
entsprechend. Gleichzeitig bleibt der Energietransport durch die lokale Kausalität begrenzt, was Relativitätsprinzipien nicht verletzt.

Die Konsequenzen dieses erweiterten Kausalitätsverständnisses sind tiefgreifend. Es ermöglicht eine physikalische Interpretation der Quantenmechanik, die ohne den problematischen
\enquote{Kollaps} der Wellenfunktion auskommt. Messprozesse erscheinen nicht mehr als mysteriöse Eingriffe in das System, sondern als besondere Fälle der systemischen Selbstorganisation.
Die scheinbare Beobachterabhängigkeit quantenmechanischer Phänomene erweist sich als Spezialfall der allgemeinen systemischen Kausalität, die immer dann besonders augenfällig wird, wenn
ein Teilsystem (der \enquote{Beobachter}) mit einem anderen (dem \enquote{beobachteten System}) korreliert.

Letztlich führt dieser Ansatz zu einer natürlichen Synthese von klassischer und Quantenphysik, von Relativitätstheorie und Wellenmechanik. Anstatt die paradoxen Aspekte der Quantentheorie
als grundlegende Prinzipien zu akzeptieren, erklärt sie sie als Konsequenz des Zusammenspiels zweier komplementärer Kausalitätsebenen - einer systemischen Ganzheit, die instantan wirkt,
und einer lokalen Dynamik, die den Energie- und Impulstransport mit endlicher Geschwindigkeit vermittelt.
