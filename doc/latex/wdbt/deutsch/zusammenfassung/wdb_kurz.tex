\documentclass[11pt, a4paper]{article}
\usepackage[utf8]{inputenc}
\usepackage[T1]{fontenc}
\usepackage[ngerman]{babel}
\usepackage{amsmath, amssymb}
\usepackage{hyperref}

\title{Zusammenfassung: Die Weber-De Broglie-Bohm-Theorie}
\author{Michael Czybor}
\date{18. August 2025}

\begin{document}

\maketitle

\section*{Überblick}
Die \textbf{Weber-De Broglie-Bohm-Theorie (WDBT)} stellt eine radikale Alternative zur etablierten Physik dar. Sie vereint drei Ansätze:
\begin{enumerate}
    \item \textbf{Weber-Elektrodynamik}: Direkte Teilchenwechselwirkungen ohne Felder.
    \item \textbf{Weber-Gravitation (WG)}: Geschwindigkeits- und beschleunigungsabhängige Gravitationskraft.
    \item \textbf{De-Broglie-Bohm-Theorie (DBT)}: Deterministische Quantenmechanik mit Führungswelle und Quantenpotential.
\end{enumerate}

\section*{Grundlegende Gleichungen}

\subsection*{Weber-Elektrodynamik}
Die Kraft zwischen zwei Ladungen:
\[
\vec{F} = \frac{q_1 q_2}{4\pi\epsilon_0 r^2} \left[ 1 - \frac{\dot{r}^2}{c^2} + \frac{2r\ddot{r}}{c^2} \right] \hat{\vec{r}}
\]

\subsection*{Weber-Gravitation}
Die gravitative Kraft:
\[
\vec{F}_{\text{WG}} = -\frac{GMm}{r^2} \left( 1 - \frac{\dot{r}^2}{c^2} + \beta \frac{r\ddot{r}}{c^2} \right) \hat{\vec{r}}
\]
mit \(\beta = 0.5\) für Massen, \(\beta = 1\) für Photonen.

\subsection*{De-Broglie-Bohm-Theorie}
Quantenpotential:
\[
Q = -\frac{\hbar^2}{2m} \frac{\nabla^2 \sqrt{\rho}}{\sqrt{\rho}}
\]
Führungsgleichung:
\[
\frac{d\vec{x}}{dt} = \frac{\hbar}{m} \operatorname{Im} \left( \frac{\vec{\nabla} \Psi}{\Psi} \right) = \frac{\vec{\nabla} S}{m}
\]

\section*{Kernaussagen}
\begin{itemize}
    \item \textbf{Keine Felder}: Wechselwirkungen sind direkt zwischen Teilchen.
    \item \textbf{Keine Singularitäten}: Das Quantenpotential verhindert unendliche Dichten.
    \item \textbf{Keine dunkle Materie}: Galaxienrotationen werden durch fraktale Raumstruktur und \(Q\) erklärt.
    \item \textbf{Statisches Universum}: Die Hubble-Konstante wird durch kumulative Gravitationseffekte erklärt, nicht durch Expansion.
    \item \textbf{Fraktale Raumdimension}: \( D = \frac{\ln 20}{\ln(2 + \phi)} \approx 2.71 \) mit \(\phi = \frac{1+\sqrt{5}}{2}\).
    \item \textbf{Emergez:} Die SRT und ART emergieren aus der WDBT mit Dodekaeder Raummodel ($D \approx 2.71$).
\end{itemize}

\section*{Experimentelle Vorhersagen}
\begin{itemize}
    \item Wellenlängenabhängige Lichtablenkung: \(\Delta \phi \propto \lambda^2\)
    \item Frequenzabhängige Shapiro-Laufzeitverzögerung
    \item Abweichungen vom linearen Hubble-Gesetz auf großen Skalen
    \item Modifizierte Dispensionsrelation in ultrakalten Quantengasen
    \item Perihelberechnung (z. B. Merkur) mit ART-Genauigkeit
\end{itemize}

\section*{Kritik an etablierten Theorien}
\begin{itemize}
    \item \textbf{ART}: Singularitäten, dunkle Materie, metaphysische Raumzeitkrümmung.
    \item \textbf{Maxwell-Theorie}: Unendliche Selbstenergien, Strahlungsparadoxa, Aharonov-Bohm-Effekt.
    \item \textbf{QM}: Kollaps der Wellenfunktion, Messproblem, nicht-lokale Verschränkung.
\end{itemize}

\section*{Philosophische Implikationen}
\begin{itemize}
    \item Physik sollte sich auf beobachtbare Phänomene konzentrieren, nicht auf mathematische Eleganz.
    \item Die Wahl zwischen Theorien ist paradigmatisch, nicht empirisch.
    \item Raum und Zeit könnten emergente Phänomene aus einer tieferen diskreten Struktur sein.
\end{itemize}

\section*{Zusammenfassung}
Die WDBT bietet eine kohärente Alternative zur modernen Physik, die konzeptionelle Probleme vermeidet und testbare Vorhersagen macht. Sie fordert einen Paradigmenwechsel weg von Feldtheorien und Raumzeitkrümmung hin zu direkten Wechselwirkungen und nicht-lokaler Ganzheit.

\end{document}