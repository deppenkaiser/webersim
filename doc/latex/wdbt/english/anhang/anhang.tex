\chapter{Additional Information and Supplements}
\section{The Aharonov-Bohm Effect}
\label{sec:aharonov-bohm}

The \textbf{Aharonov-Bohm effect} (AB effect) is a fundamental quantum phenomenon demonstrating that electromagnetic potentials ($\vec{A}$, $\Phi$) have a direct physical influence on quantum particles, even in regions where the fields ($\vec{E}$, $\vec{B}$) are zero.

\subsection{Experimental Setup}
An electron beam is split into two paths that enclose a region with magnetic flux $\Phi$.

\subsection{Theoretical Description}
The wavefunction $\psi$ of a particle with charge $q$ is modified by the vector potential $\vec{A}$:

\begin{equation}
\psi \rightarrow \psi \cdot \exp\left(i\frac{q}{\hbar}\int \vec{A}\cdot d\vec{l}\right)
\end{equation}

The phase difference between the two paths is:

\begin{equation}
\Delta\phi = \frac{q}{\hbar}\oint \vec{A}\cdot d\vec{l} = \frac{q}{\hbar}\Phi_B
\end{equation}

\subsection{Physical Significance}
\begin{itemize}
\item \textbf{Non-Locality}: Quantum particles "sense" $\vec{A}$ even in field-free regions.
\item \textbf{Topological Invariant}: The phase depends only on the enclosed flux $\Phi_B$.
\item \textbf{Paradigm Shift}: Challenges the classical assumption that only $\vec{E}$ and $\vec{B}$ are physically relevant.
\end{itemize}

\subsection{Experimental Confirmation}
\begin{itemize}
\item Theoretical prediction: Aharonov \& Bohm (1959).
\item First experiments: Chambers (1960), Tonomura et al. (1982).
\item Modern applications: Quantum interferometers, topological quantum materials.
\end{itemize}

\section{Bell Inequalities}
\label{sec:bell}

The \textbf{Bell inequality} (1964) is a central result in quantum physics, showing that no local hidden-variable theory can reproduce the predictions of quantum mechanics.

\subsection{Theoretical Formulation}
For an entangled particle pair (e.g., photons with spin or polarization correlations), the CHSH inequality holds:

\begin{equation}
S = |E(a,b) - E(a,b')| + |E(a',b) + E(a',b')| \leq 2
\end{equation}

where $E(\theta_1, \theta_2)$ is the correlation function for measurements at angles $\theta_1$ and $\theta_2$.

\subsection{Quantum Mechanical Prediction}
Quantum mechanics allows for certain angle combinations:

\begin{equation}
S_{\text{QM}} = 2\sqrt{2} \approx 2.828 > 2
\end{equation}

which violates the Bell inequality.

\subsection{Experimental Confirmation}
\begin{itemize}
\item First tests: Alain Aspect (1982) with photon pairs.
\item Loophole-free experiments: Hensen et al. (2015), Zeilinger group (2017).
\item Current applications: Quantum cryptography (BB84 protocol).
\end{itemize}

\subsection{Interpretation}
\begin{itemize}
\item Refutation of local realistic theories (Einstein-Podolsky-Rosen paradox).
\item Confirmation of quantum entanglement as physical reality.
\item Foundation for quantum information technologies.
\end{itemize}

\newpage
\section{Exact Derivation of the Weber Gravitational Orbital Equation}
\label{sec:exact_derivation}

This appendix rigorously derives the orbital equation of Weber Gravitation (WG) without the simplifications used in Chapter~3. The full equation of motion is developed up to order $\mathcal{O}(c^{-4})$.

\subsection{Starting Equations}
The Weber gravitational force is:
\begin{equation}
\vec{F}_{\text{WG}} = -\frac{GMm}{r^2} \left(1 - \frac{\dot{r}^2}{c^2} + \beta \frac{r\ddot{r}}{c^2}\right)\hat{\vec{r}}
\end{equation}
For planetary orbits, we set $\beta = 0.5$ (see Section~3.1.2). The equation of motion in polar coordinates is:
\begin{equation}
m\left(\ddot{r} - r\dot{\phi}^2\right) = -\frac{GMm}{r^2}\left(1 - \frac{\dot{r}^2}{c^2} + \frac{r\ddot{r}}{2c^2}\right)
\end{equation}

\subsection{Transformation to Angular Coordinates}
Using the angular momentum $h = r^2\dot{\phi} = \text{const.}$ and the substitution $u = 1/r$, we obtain:
\begin{align}
\dot{r} &= -h\frac{du}{d\phi} \\
\ddot{r} &= -h^2u^2\frac{d^2u}{d\phi^2}
\end{align}
Substituting into the equation of motion yields the exact differential equation:
\begin{equation}
\frac{d^2u}{d\phi^2} + u = \frac{GM}{h^2}\left[1 - h^2\left(\frac{du}{d\phi}\right)^2 + \frac{h^2u}{2}\frac{d^2u}{d\phi^2}\right]
\end{equation}

\subsection{Perturbation Theory}
We expand the solution as a series:
\begin{equation}
u(\phi) = u_0(\phi) + \frac{GM}{c^2h^2}u_1(\phi) + \mathcal{O}(c^{-4})
\end{equation}
where $u_0$ is the Newtonian solution:
\begin{equation}
u_0(\phi) = \frac{GM}{h^2}(1 + e\cos\phi)
\end{equation}

The perturbation equation for $u_1$ is:
\begin{equation}
\frac{d^2u_1}{d\phi^2} + u_1 = \frac{G^2M^2e^2}{h^4}\left(\sin^2\phi + \frac{1 + e\cos\phi}{2}\cos\phi\right)
\end{equation}

\subsection{Solution of the Perturbation Equation}
The general solution consists of homogeneous and particular parts:
\begin{equation}
u_1(\phi) = \frac{G^2M^2e}{8h^4}\left[3e\phi\sin\phi + (4 + e^2)\cos\phi\right]
\end{equation}

\subsection{Perihelion Shift}
The non-periodic term $\propto \phi\sin\phi$ leads to the perihelion shift:
\begin{equation}
\Delta\phi = \frac{6\pi G^2M^2}{c^2h^4} = \frac{6\pi GM}{c^2a(1 - e^2)}
\end{equation}
This matches observations and general relativity exactly.

\subsection{Critical Discussion}
\begin{itemize}
\item The choice $\beta = 0.5$ is essential—other values lead to incorrect predictions.
\item Neglecting $\dot{r}^2$ is justified only for $e \ll 1$.
\item The DBT compensation of $\mathcal{O}(c^{-4})$ terms (Eq. \ref{eq:shapiro}) ensures orbital stability.
\end{itemize}

This derivation shows that WG, in combination with DBT, provides a consistent alternative to general relativity.

\section{Potential Differences in Weber Theories}
\label{sec:weber_potentials}

\subsection{Weber Electrodynamics}
The Weber force between two charges $q_1$ and $q_2$ is:
\[
\vec{F}_{\text{Weber-EM}} = \frac{q_1 q_2}{4\pi\epsilon_0 r^2} \left(1 - \frac{\dot{r}^2}{c^2} + \beta_{\text{EM}} \frac{r\ddot{r}}{c^2}\right)\hat{r}, \quad \beta_{\text{EM}} = 2
\]
\begin{itemize}
\item \textbf{Non-Conservativity}: The force explicitly includes velocity ($\dot{r}^2$) and acceleration ($\ddot{r}$) terms, preventing the existence of a classical potential $\Phi$.
\item \textbf{Pseudo-Potential}: Only for $\ddot{r} = 0$ can an energy-like expression be derived:
\[
E_{\text{Weber-EM}} = \frac{1}{2}m_1v_1^2 + \frac{1}{2}m_2v_2^2 + \underbrace{\frac{q_1 q_2}{4\pi\epsilon_0 r}\left(1 - \frac{\dot{r}^2}{2c^2}\right)}_{\text{Not a true potential}}
\]
\end{itemize}

\subsection{Weber Gravitation}
The gravitational potential of a mass $M$ is:
\[
\Phi_{\text{WG}}(r) = -\frac{GM}{r}\left(1 + \frac{v^2}{2c^2} + \beta_{\text{G}} \frac{r\ddot{r}}{2c^2}\right), \quad \beta_{\text{G}} = 
\begin{cases}
0.5 & \text{(masses)} \\
1 & \text{(photons)}
\end{cases}
\]
\begin{itemize}
\item \textbf{Conservativity}: Despite the $\ddot{r}$ term, $\Phi_{\text{WG}}$ is well-defined because gravity is purely attractive.
\item \textbf{Physical Justification}: The term $\beta_{\text{G}}\frac{r\ddot{r}}{2c^2}$ is necessary to reproduce Mercury's perihelion shift ($\beta_{\text{G}} = 0.5$) and light deflection ($\beta_{\text{G}} = 1$).
\end{itemize}

\subsection*{Summary}
\begin{tabular}{ll}
\textbf{Weber Electrodynamics} & \textbf{Weber Gravitation} \\ \hline
$\beta_{\text{EM}} = 2$ (Lorentz force) & $\beta_{\text{G}} = 0.5/1$ (GR consistency) \\
No general potential & Well-defined potential \\
Non-conservative (radiation losses) & Conservative \\
\end{tabular}

\section{Derivation of a Planet's Orbital Period in WDBT}
\label{sec:orbital_period}

\subsection*{Starting Equations}
For a planet with semi-major axis \( a \) and eccentricity \( e \), the orbital equation in WDBT (Eq. \ref{eq:weber_r_1_ordnung}) is:

\begin{equation}
r(\phi) = \frac{a(1-e^2)}{1 + e \cos(\kappa \phi)}
\end{equation}

with the perihelion shift constant:

\begin{equation}
\kappa = \sqrt{1 - \frac{6GM}{c^2 a(1-e^2)}}
\end{equation}

\subsection*{Energy Conservation}
The total energy (kinetic + Weber potential) is:

\begin{equation}
E = \frac{1}{2}mv^2 - \frac{GMm}{r}\left(1 + \frac{v^2}{2c^2}\right)
\end{equation}

\subsection*{Circular Orbit Approximation}
For nearly circular orbits (\( e \approx 0 \)):
\begin{itemize}
\item Instantaneous distance \( r \approx a \) (constant).
\item Angular velocity \( \omega = \frac{d\phi}{dt} = \text{constant} \).
\item Orbital velocity \( v = a\omega \).
\end{itemize}

\subsection*{Equation of Motion}
The radial force balance yields:

\begin{equation}
m a \omega^2 = \frac{GMm}{a^2}\left(1 + \frac{a^2 \omega^2}{2c^2}\right)
\end{equation}

\subsection*{Solution for Angular Velocity}
Rearranging gives:

\begin{align}
\omega^2 a^3 &= GM \left(1 + \frac{a^2 \omega^2}{2c^2}\right) \\
\omega^2 \left(a^3 - \frac{GM a^2}{2c^2}\right) &= GM \\
\omega^2 &= \frac{GM}{a^3} \left(1 - \frac{GM}{2a c^2}\right)^{-1} \\
&\approx \frac{GM}{a^3} \left(1 + \frac{GM}{2a c^2}\right) \quad \text{(Taylor expansion)}
\end{align}

\subsection*{Orbital Period}
With \( T = \frac{2\pi}{\omega} \), we obtain:

\begin{equation}
T \approx 2\pi \sqrt{\frac{a^3}{GM}} \left(1 - \frac{GM}{4a c^2}\right)
\end{equation}

\subsection*{Exact Solution for Elliptical Orbits}
The full solution, accounting for eccentricity \( e \), is:

\begin{equation}
\boxed{T = 2\pi \sqrt{\frac{a^3}{GM}} \left[1 - \frac{3GM}{4c^2 a(1-e^2)}\right]}
\end{equation}

\subsection*{Physical Interpretation}
\begin{itemize}
\item The term \( 2\pi \sqrt{a^3/GM} \) corresponds to the classical Keplerian result.
\item The correction \( -\frac{3GM}{4c^2 a(1-e^2)} \) arises from:
  \begin{enumerate}
  \item The velocity term \( \frac{v^2}{c^2} \) in Weber gravitation.
  \item The perihelion shift \( \kappa \) in the WDBT orbital equation.
  \end{enumerate}
\item For Mercury (\( a \approx 5.79 \times 10^{10} \) m, \( e \approx 0.206 \)), the correction is \( \approx 7.3 \times 10^{-8} \).
\end{itemize}

\section{Dynamics of True Anomaly}
The time evolution of \(\phi(t)\) follows the differential equation:
\[
\frac{d\phi}{dt} = \frac{h (1 + e \cos(\kappa \phi))^2}{a^2 (1-e^2)^2},
\]
where:
\begin{itemize}
\item \(h = \sqrt{GMa(1-e^2)}\) is the specific angular momentum,
\item \(\kappa = \sqrt{1 - \frac{6GM}{c^2 a(1-e^2)}}\) is the perihelion shift correction.
\end{itemize}
The solution \(\phi(t)\) must be numerically integrated to determine \(\Delta \phi\) for \(\Delta T = T_1 - T_0\).