\section{Herleitung der kombinierten WG-DBT Bewegungsgleichung}

\subsection*{1. Ausgangspunkt: Weber-Gravitationskraft}
Die klassische Weber-Kraft für zwei Massen $m$ und $M$ lautet:
\begin{equation}
\mathbf{F}_{\text{WG}} = -\frac{GMm}{r^2}\left(1 - \frac{\dot{r}^2}{c^2} + \beta\frac{r\ddot{r}}{c^2}\right)\hat{\mathbf{r}}
\end{equation}

\subsection*{2. Umformung der radiale Beschleunigungsterme}
Wir entwickeln die Terme $\dot{r}^2$ und $r\ddot{r}$ in vektorieller Form:

\begin{align}
\dot{r} &= \frac{d}{dt}\sqrt{\mathbf{r}\cdot\mathbf{r}} = \frac{\mathbf{r}\cdot\mathbf{v}}{r} \\
\dot{r}^2 &= \left(\frac{\mathbf{r}\cdot\mathbf{v}}{r}\right)^2 \\
r\ddot{r} &= \frac{d}{dt}(r\dot{r}) - \dot{r}^2 = \mathbf{v}\cdot\mathbf{v} + \mathbf{r}\cdot\mathbf{a} - \left(\frac{\mathbf{r}\cdot\mathbf{v}}{r}\right)^2
\end{align}

Für kleine Abweichungen von Kreisbahnen vernachlässigen wir den letzten Term und erhalten:
\begin{equation}
r\ddot{r} \approx v^2 + \mathbf{r}\cdot\mathbf{a}
\end{equation}

\subsection*{3. Verallgemeinerte Weber-Kraft in vektorieller Form}
Einsetzen in (1) ergibt:
\begin{equation}
\mathbf{F}_{\text{WG}} = -\frac{GMm}{r^2}\left(1 - \frac{(\mathbf{r}\cdot\mathbf{v})^2}{c^2r^2} + \beta\frac{v^2 + \mathbf{r}\cdot\mathbf{a}}{c^2}\right)\hat{\mathbf{r}}
\end{equation}

\subsection*{4. Lagrange-Formulierung der Weber-Gravitation}
Das effektive Weber-Potential lautet:
\begin{equation}
V_{\text{WG}} = -\frac{GMm}{r}\left(1 - \frac{v^2}{2c^2} + \beta\frac{\mathbf{r}\cdot\mathbf{a}}{2c^2}\right)
\end{equation}

Die Lagrange-Funktion wird:
\begin{equation}
\mathscr{L}_{\text{WG}} = T - V_{\text{WG}} = \frac{1}{2}mv^2 + \frac{GMm}{r}\left(1 - \frac{v^2}{2c^2} + \beta\frac{\mathbf{r}\cdot\mathbf{a}}{2c^2}\right)
\end{equation}

\subsection*{5. Euler-Lagrange-Gleichungen}
Anwendung der Euler-Lagrange-Gleichung:
\begin{equation}
\frac{d}{dt}\left(\frac{\partial\mathscr{L}}{\partial\mathbf{v}}\right) - \frac{\partial\mathscr{L}}{\partial\mathbf{r}} = 0
\end{equation}

Berechnung der Terme:
\begin{align}
\frac{\partial\mathscr{L}}{\partial\mathbf{v}} &= m\mathbf{v} - \frac{GMm}{c^2r}\mathbf{v} + \beta\frac{GMm}{2c^2}\frac{\mathbf{r}}{r} \\
\frac{d}{dt}\left(\frac{\partial\mathscr{L}}{\partial\mathbf{v}}\right) &= m\mathbf{a} - \frac{GMm}{c^2}\left(\frac{\mathbf{a}}{r} - \frac{\dot{r}\mathbf{v}}{r^2}\right) + \beta\frac{GMm}{2c^2}\left(\frac{\mathbf{v}}{r} - \frac{\dot{r}\mathbf{r}}{r^2}\right) \\
\frac{\partial\mathscr{L}}{\partial\mathbf{r}} &= -\frac{GMm}{r^2}\hat{\mathbf{r}} + \frac{GMm}{2c^2}\left(\frac{v^2}{r^2}\hat{\mathbf{r}} - \beta\frac{\mathbf{a}}{r}\right)
\end{align}

\subsection*{6. De-Broglie-Bohm'sches Quantenpotential}
Das Quantenpotential der DBT ist:
\begin{equation}
Q = -\frac{\hbar^2}{2m}\frac{\nabla^2|\Psi|}{|\Psi|}
\end{equation}

Die quantenmechanische Kraft ergibt sich aus:
\begin{equation}
\mathbf{F}_{\text{Q}} = -\nabla Q
\end{equation}

\subsection*{7. Kombinierte Bewegungsgleichung}
Addition der Weber- und Quantenkräfte führt zu:
\begin{equation}
m\frac{d}{dt}\left[\left(1 - \frac{GM}{c^2r} + \beta\frac{GM}{2c^2}\frac{\mathbf{r}\cdot\mathbf{v}}{r^2}\right)\mathbf{v}\right] = -\frac{GMm}{r^2}\hat{\mathbf{r}} - \nabla Q
\end{equation}

Definition des Weber-Lorentz-Faktors:
\begin{equation}
\gamma_{\text{WG}} \equiv \left(1 - \frac{v^2}{c^2} + \beta\frac{\mathbf{v}\cdot\mathbf{a}}{c^2}\right)^{-1/2} \approx 1 + \frac{v^2}{2c^2} - \beta\frac{\mathbf{v}\cdot\mathbf{a}}{2c^2}
\end{equation}

\subsection*{8. Finale Bewegungsgleichung (\ref{eq:wg_dbt_srt})}
Nach Vernachlässigung höherer Ordnungen erhalten wir:
\begin{equation}
m\frac{d}{dt}(\gamma_{\text{WG}}\mathbf{v}) = -\nabla Q
\end{equation}
