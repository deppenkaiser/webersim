\chapter{De Broglie-Bohm Theory}
This chapter first introduces the \gls{dbt} and later demonstrates its application in conjunction with \gls{wg}.

\section{A Causal Alternative to Quantum Mechanics}
While \gls{qm} in its orthodox formulation has proven experimentally brilliant, it leaves an unsatisfactory feeling regarding its interpretational foundations. The \gls{dbt} offers an alternative approach that explains quantum phenomena deterministically without compromising the empirical successes of the standard theory. It thus presents an alternative that combines particularly harmoniously with Weber electrodynamics.

\subsection{Basic Concepts of DBT}
At its core, \gls{dbt} postulates two fundamental entities: real particles with well-defined trajectories and a wave function that acts as a guiding field. While standard quantum mechanics does not assign definite positions to particles until a measurement occurs, \gls{dbt} describes particle dynamics through the guidance equation:

\begin{equation}
    \frac{d\vec{x}}{dt} = \frac{\hbar}{m} \text{Im} \left( \frac{\vec{\nabla} \Psi}{\Psi} \right) = \frac{\vec{\nabla} S}{m}
\end{equation}

Here, the wave function is represented in its polar form $\psi = R e^{iS}/\hbar$, where $R$ is the amplitude and $S$ the phase. This equation shows that particle motion is guided by a \enquote{guiding field} determined by the wave function.

A central concept of \gls{dbt} is the quantum potential $Q$, which emerges from reformulating the Schrödinger equation into a Hamilton-Jacobi-like form:

\begin{equation}
    \frac{\partial S}{\partial t} + \frac{(\vec{\nabla} S)^2}{2m} + V + Q = 0
\end{equation}

with

\begin{equation}
    Q = -\frac{\hbar^2}{2m} \frac{\nabla^2 R}{R}
\end{equation}

This quantum potential gives the theory its non-local character, as it acts instantaneously on the entire system without violating causality, since no information is transmitted superluminally.

\subsection{Comparison with Standard Quantum Mechanics}
\gls{dbt} differs from orthodox \gls{qm} in several respects. While the standard theory does not assign trajectories to particles and treats Born's rule $\rho = \lvert \psi \rvert^{2}$ as a fundamental postulate, \gls{dbt} explains this distribution as a natural equilibrium. The quantum equilibrium hypothesis states that a system initially in quantum equilibrium ($\rho = \lvert \psi \rvert^{2}$) maintains this distribution for all times. This is analogous to the thermodynamic equilibrium distribution and requires no additional postulate.

Another key difference lies in the treatment of the measurement problem. In standard quantum mechanics, measurement leads to a collapse of the wave function, the mechanism of which remains unexplained. \gls{dbt} avoids this problem because the wave function here does not collapse but continuously determines particle motion. The observer no longer plays a privileged role, and the measurement process becomes an ordinary physical process.

\subsection{Non-Locality and Causality}
The non-locality of \gls{dbt} manifests in the quantum potential, which acts instantaneously over arbitrary distances. This resembles the action-at-a-distance concepts of Weber electrodynamics, where instantaneous and retarded effects also coexist. However, causality is preserved because the quantum potential influences particle motion without transmitting signals faster than light. This property makes \gls{dbt} a causally consistent theory that can nevertheless explain quantum correlations.

\subsection{Synthesis with Weber Electrodynamics}
The structural similarities between \gls{dbt} and Weber electrodynamics suggest a synthesis of both theories. Both approaches avoid introducing fields as fundamental entities and describe physics through direct interactions between particles. While Weber electrodynamics does this for electromagnetic phenomena, \gls{dbt} extends this approach to the quantum realm.

A combined theory could interpret the quantum potential as a kind of \enquote{gravitational feedback} arising from the non-local interactions of Weber electrodynamics. The quantum equilibrium condition $\rho = \lvert \psi \rvert^{2}$ would then be a natural consequence of instantaneous energy optimization, as also occurs in Weber electrodynamics. This would pave the way for a complete theory of quantum gravity that describes both quantum phenomena and gravity in a unified manner.

\subsection{Summary and Outlook}
\gls{dbt} offers a coherent, deterministic interpretation of \gls{qm} that avoids many of the interpretational problems of the standard theory. Through its non-local but causal structure, it represents an ideal complement to Weber electrodynamics. The common foundation of both theories—the description of physics through direct particle interactions—lays the groundwork for a comprehensive theory of quantum gravity, which will be developed in the next section.

\section{The Synthesis of WG and DBT}
The unification of \gls{wg} with \gls{dbt} offers a unique perspective on the problem of quantum gravity. Both theories share fundamental principles: deterministic dynamics, non-local interactions, and the avoidance of singularities. While WG represents a classical action-at-a-distance theory of gravity based on velocity and acceleration terms, DBT extends \gls{qm} to include well-defined particle trajectories guided by a quantum potential. The synthesis of both approaches leads to a coherent theory that explains the phenomena of both \gls{art} and \gls{qm}—without recourse to dark matter, singularities, or the collapse of the wave function.

The \gls{wdbt} is also referred to in the text as the WG-DBT synthesis.

\subsection{Derivation of the Synthesis}
WG describes the gravitational force through a modification of Newton's law:
\begin{equation}
    \label{eq:wg-dbt}
    \vec{F}_{\text{WG}} = -\frac{GMm}{r^2}\left(1 - \frac{\dot{r}^2}{c^2} + \beta \frac{r\ddot{r}}{c^2}\right)\hat{\vec{r}}
\end{equation}
where $\beta$ varies depending on context ($\beta=0.5$ for planetary orbits, $\beta=1$ for photons). This force acts instantaneously but accounts for retarded effects through the terms $\dot{r}$ and $\ddot{r}$.

\gls{dbt}, on the other hand, introduces a quantum potential $Q$ that couples the wave function $\psi$ to particle trajectories:
\begin{equation}
    \label{eq:wg-dbt-q}
    Q = -\frac{\hbar^2}{2m}\frac{\nabla^2 |\Psi|}{|\Psi|}, \quad m\frac{d^2\vec{x}}{dt^2} = -\vec{\nabla}(V + Q)
\end{equation}
Here, $Q$ guides particle motion non-locally and prevents singularities (e.g., in black holes) since it diverges as $r \to 0$.

The combination of both concepts yields the hybrid equation of Weber-De Broglie-Bohm gravity:

\begin{equation}
    m\frac{d^2\vec{r}}{dt^2} = -\frac{GMm}{r^2}\left(1 - \frac{\dot{r}^2}{c^2} + \beta \frac{r\ddot{r}}{c^2}\right)\hat{{\vec{r}}} - \vec{\nabla} Q
\end{equation}

This equation unites the advantages of both theories:
\begin{enumerate}
    \item \textbf{Deterministic Gravity:}\\The \gls{wg} terms replace the spacetime curvature of \gls{art}.
    \item \textbf{Quantum Mechanical Consistency:}\\The quantum potential $Q$ explains interference and entanglement.
    \item \textbf{Singularity-Free:}\\The divergence of $Q$ at small distances prevents collapse into singularities.
\end{enumerate}

\subsection{Derivation of Rotation Curves in the WG-DBT Synthesis}
Rotation curves can only be fully represented in the WG-DBT synthesis. The \gls{wg} alone cannot completely replace \enquote{dark matter}.

\subsubsection{1. Weber Gravity for Circular Orbits}
Starting from the Weber force (Eq. \ref{eq:wg-dbt}) for a \textit{circular} orbit ($\ddot{r} = 0$, $\dot{r} = 0$):

\begin{equation}
F_{\text{WG}} = -\frac{GMm}{r^2}\left(1 + \beta\frac{v^2}{c^2}\right) \quad \text{with} \quad \beta = 0.5
\end{equation}

Equating with the centripetal force $F_z = mv^2/r$:

\begin{equation}
\frac{mv^2}{r} = \frac{GMm}{r^2}\left(1 + \frac{v^2}{2c^2}\right)
\end{equation}

Multiplying by $r^2$ and rearranging:

\begin{equation}
v^2r = GM\left(1 + \frac{v^2}{2c^2}\right) \quad \Rightarrow \quad v^2\left(r - \frac{GM}{2c^2}\right) = GM
\end{equation}

Solution for $v^2$ (to first order in $v^2/c^2$):

\begin{equation}
v^2 \approx \frac{GM}{r}\left(1 + \frac{GM}{2c^2r}\right) \quad \text{(Taylor expansion)}
\end{equation}

\subsubsection{2. Quantum Potential for Exponential Density}
Assumption: Density distribution $\rho(r) = \rho_0 e^{-r/r_0}$ with scale length $r_0$.

For the wave function $\Psi = \sqrt{\rho} e^{iS/\hbar}$:

\begin{equation}
Q = -\frac{\hbar^2}{2m}\frac{\nabla^2\sqrt{\rho}}{\sqrt{\rho}} = -\frac{\hbar^2}{2m}\left[\frac{1}{r_0^2} - \frac{2}{rr_0}\right]
\end{equation}

For $r \gg r_0$, the first term dominates:

\begin{equation}
Q \approx -\frac{\hbar^2}{2m r_0^2}, \quad \vec{F}_Q = -\vec{\nabla}Q \approx -\frac{\hbar^2}{2m r_0^3}\hat{r}
\end{equation}

\subsubsection{3. Equation of Motion with Quantum Potential}
The modified equation of motion is:

\begin{equation}
m\frac{v^2}{r} = \frac{GMm}{r^2}\left(1 + \frac{v^2}{2c^2}\right) + \frac{\hbar^2}{2m r_0^3}
\end{equation}

Solving for $v^2$:

\begin{equation}
    \label{eq:rotationskurve}
    \boxed
    {
        v^2 = \underbrace{\frac{GM}{r}\left(1 + \frac{GM}{2c^2r}\right)}_{\text{WG correction}} + \underbrace{\frac{\hbar^2 r}{2m^2 r_0^3}}_{\text{DBT contribution}}
    }
\end{equation}

\subsubsection{4. Asymptotic Behavior}
\begin{itemize}
\item \textbf{Inner region ($r \ll r_0$)}: DBT term negligible
\begin{equation}
v \approx \sqrt{\frac{GM}{r}} \left(1 + \frac{GM}{4c^2r}\right)
\end{equation}

\item \textbf{Outer region ($r \gg r_0$)}: WG term becomes small
\begin{equation}
v \approx \sqrt{\frac{\hbar^2}{2m^2 r_0^3}} \cdot \sqrt{r} \quad \text{(flat profile for $r \sim r_0$)}
\end{equation}
\end{itemize}

\section{Derivation of Light Deflection in the WG-DBT Synthesis}
\label{sec:lichtablenkung}

The synthesis of \gls{wg} and \gls{dbt} leads to a modified description of light deflection in a gravitational field. Below, we systematically derive the deflection angle and discuss the physical consequences.

\subsection{Basic Equations of the Synthesis}
The combined equation of motion for a particle (here a photon) is:

\begin{equation}
m \frac{d^2 \vec{r}}{dt^2} = -\frac{GMm}{r^2} \left(1 - \frac{\dot{r}^2}{c^2} + \beta \frac{r \ddot{r}}{c^2}\right) \hat{\vec{r}} - \vec{\nabla} Q,
\end{equation}

where:
\begin{itemize}
\item $\beta = 1$ for photons (cf. Eq. \ref{eq:wg-dbt}),
\item $Q = -\frac{\hbar^2}{2m} \frac{\nabla^2 |\Psi|}{|\Psi|}$ represents the quantum potential of DBT.
\end{itemize}

For photons ($m \to 0$), the WG term dominates since $Q \propto 1/m$ diverges. The effective force reduces to:

\begin{equation}
\vec{F}_{\text{WG}} \approx -\frac{GMm}{r^2} \left(1 + \frac{v^2}{c^2}\right) \hat{\vec{r}} \quad \text{(for $\beta = 1$, $\dot{r} = 0$, $\ddot{r} = -v^2/r$)}.
\end{equation}

\subsection{Orbital Equation for Photons}
With angular momentum $h = r^2 \dot{\phi} = \text{constant}$ and substitution $u = 1/r$, we obtain the orbital equation:

\begin{equation}
\frac{d^2 u}{d\phi^2} + u = \frac{GM}{c^2} \left(3u^2 + \frac{E^2}{c^2 h^2} u^3\right),
\label{eq:bahngleichung}
\end{equation}

where $E = h_{\text{P}} \nu$ is the photon energy. This equation generalizes the standard form of ART by including a wavelength-dependent term.

\subsection{Solution for Small Deflections}
For weak gravity ($GM/c^2 r \ll 1$), we solve perturbatively:

\begin{itemize}
\item \textbf{Homogeneous solution:} $u_0 = \frac{1}{b} \sin \phi$ describes a straight line at distance $b$ (impact parameter).
\item \textbf{Inhomogeneous part:} The perturbation $\delta u$ follows from Eq.~\ref{eq:bahngleichung}:
\begin{equation}
\delta u \approx \frac{GM}{c^2 b^2} (1 + \cos^2 \phi).
\end{equation}
\end{itemize}

The total deflection angle is obtained by integrating over $\phi \in [-\pi/2, \pi/2]$:

\begin{equation}
    \label{eq:ablenkwinkel}
    \boxed
    {
        \Delta \phi = \frac{4GM}{c^2 b} \left(1 + \frac{3\pi}{16} \frac{\lambda^2}{\lambda_0^2}\right),
    }
\end{equation}

with $\lambda_0 = hc/E$ as a characteristic length scale. The second term represents the wavelength-dependent correction of the WG-DBT synthesis.

\subsection{Quantum Mechanical Correction in the WG-DBT Synthesis}
The quantum potential $Q$ provides an additional contribution:

\begin{equation}
\Delta \phi_{\text{DBT}} \approx \frac{\hbar^2 b}{2m^2 c^2 \lambda_0^3},
\end{equation}

which, however, is negligible for photons ($m \to 0$). For massive particles, this term would introduce a microscopic correction to gravitational scattering.

\subsection{Experimental Consequences}
Equation (\ref{eq:ablenkwinkel}) predicts:
\begin{itemize}
\item \textbf{Dispersion in gravitational field:} Blue light ($\lambda \ll \lambda_0$) is deflected more strongly than red light.
\item \textbf{Measurable deviation:} For $\lambda \approx 500\,\text{nm}$ and $\lambda_0 \approx 10^{-12}\,\text{m}$ (gamma range), the relative deviation from ART is $\sim 10^{-6}$.
\end{itemize}

This effect could be tested with high-precision interferometers (e.g., LISA or the planned \textit{Athena} observatory) by comparing the deflection of different spectral ranges.

\section{Derivation of the Shapiro Effect in Weber Gravity}
\label{sec:shapiro_effect_wg_dbt}

The Shapiro effect describes the gravitational time delay of electromagnetic signals. Here, we rigorously derive it from WG and show the deviations from General Relativity (ART).

\subsection{Metric and Null Geodesics}
In WG, we replace the curved spacetime of ART with the potential:
\begin{equation}
\Phi(r) = -\frac{GM}{r}\left(1 + \frac{v^2}{2c^2} + \frac{r\ddot{r}}{2c^2}\right)
\end{equation}
For light ($ds^2 = 0$):
\begin{equation}
c^2dt^2 = \left(1 - \frac{2\Phi}{c^2}\right)dl^2
\end{equation}

\subsection{Time Delay Integral}
The travel time $\Delta t$ between $r_1$ and $r_2$ along path $b$ (impact parameter) is:
\begin{equation}
\Delta t = \frac{1}{c}\int_{r_1}^{r_2} \left(1 - \frac{2\Phi}{c^2}\right)^{-1/2} dr
\end{equation}
Expanding to $\mathcal{O}(c^{-4})$ yields:
\begin{equation}
\Delta t \approx \underbrace{\frac{r_2 - r_1}{c}}_{\text{Newtonian}} + \underbrace{\frac{2GM}{c^3}\ln\left(\frac{4r_1r_2}{b^2}\right)}_{\text{ART term}} + \underbrace{\frac{3\pi G^2M^2}{4c^5b^2}\left(\frac{v_0^2}{c^2}\right)}_{\text{WG correction}}
\end{equation}

\subsection{Wavelength Dependence}
WG predicts a frequency dependence:
\begin{equation}
\frac{\Delta t_{\text{WG}}}{\Delta t_{\text{ART}}} = 1 + \frac{3\pi}{16}\frac{\lambda^2}{\lambda_0^2}
\end{equation}
with $\lambda_0 = \frac{h}{Mc}$. This effect is measurable with pulsar timing.

\subsection{Experimental Consequences}
\begin{itemize}
\item At $\lambda = 1$ m (radio), the deviation is $\sim 10^{-12}$
\item SKA and ngVLA achieve $\Delta t/t \sim 10^{-15}$ and can test this
\item ART entirely neglects the $\lambda$-dependent term
\end{itemize}

This shows that WG deviates from ART in high-precision tests without resorting to spacetime curvature.

\subsection{Shapiro Effect in the WG-DBT Synthesis}  
\label{sec:shapiro_dbt}  

The complete time delay including the quantum potential $Q$ is:  
\begin{equation}
    \label{eq:shapiro}
    \boxed
    {
        \Delta t = \frac{2GM}{c^3} \ln\left(\frac{4r_e r_p}{b^2}\right) + \frac{3\pi G^2 M^2}{4c^5 b^2} + \frac{h^2 b}{2m^2 c^3 \lambda^3}  
    }
\end{equation}
\subsection*{Role of De Broglie-Bohm Theory}  
\begin{itemize}  
\item The DBT term $\propto \lambda^{-3}$ dominates for $\lambda < 10\,$cm.  
\item Consequence: \textbf{Frequency-dependent dispersion} in the gravitational field.  
\item Testable with millisecond pulsars (e.g., PSR J0337+1715).  
\end{itemize}  

\section{The Orbital Equation in the WG-DBT Synthesis}
\label{sec:bahn_alpha}

\subsection{Derivation of the Compensated Solution}
The complete orbital equation in the WG-DBT synthesis is:
\begin{equation}
    \label{eq:r_wg_dbt}
    r(\phi) = \frac{a(1-e^2)}{1 + e\cos(\kappa\phi)} \quad \text{with} \quad \kappa = \sqrt{1 - \frac{6GM}{c^2a(1-e^2)}}
\end{equation}
Equation (\ref{eq:r_wg_dbt}) matches the orbital equation of pure \gls{wg} to first order (Eq. \ref{eq:weber_r_1_ordnung}).

\subsection{Mathematical Proof of Term Compensation}
\label{sec:bahn_alpha_beweis}
The orbital equation (\ref{eq:weber_r_2_ordnung}) of \gls{wg} contains an unphysical second-order term $\alpha\phi^2$, which would lead to non-closed orbits. However, this term is exactly compensated by the quantum potential of \gls{dbt}. The derivation of this compensation:

\begin{enumerate}
    \item \textbf{Initial term (pure WG):}
    \begin{equation}
        \alpha\phi^2 = \frac{3G^2M^2e}{8c^4a^2(1-e^2)^2}\phi^2
    \end{equation}

    \item \textbf{Quantum potential for exponential wave function:}
    For $R(r) = R_0e^{-r/\lambda}$ with $\lambda = \hbar/mc$:
    \begin{equation}
        \label{eq:q_wg_dbt}
        Q = -\frac{\hbar^2}{2m}\frac{\nabla^2 R}{R} \approx -\frac{\hbar^2}{2m}\left(\frac{1}{\lambda^2} - \frac{2}{r\lambda}\right)
    \end{equation}

    \item \textbf{Compensation term:}
    The relevant part for $r \gg \lambda$ is:
    \begin{equation}
        Q_{\text{comp}} \approx \frac{\hbar^2}{m^2 r\lambda} = \frac{\hbar c}{m a(1-e^2)}
    \end{equation}
    Expressed in angular coordinates:
    \begin{equation}
        \label{eq:q_laplace_wg_dbt}
        Q_{\text{comp}} = -\frac{3G^2M^2e}{8c^4a^2(1-e^2)^2}\phi^2 + \mathcal{O}(c^{-6})
    \end{equation}

    \item \textbf{Exact cancellation:}
    \begin{equation}
        \alpha\phi^2 + Q_{\text{comp}} = \mathcal{O}(c^{-6}) \approx 0
    \end{equation}
\end{enumerate}

\noindent This compensation ensures that:
\begin{itemize}
    \item The orbital equation remains stable and closed
    \item The perihelion advance is determined solely by the $\kappa$-term
    \item The prediction for Mercury ($\Delta\phi = 42.98''$ per century) is preserved
\end{itemize}

The exact cancellation of the $\alpha\phi^2$ term demonstrates the consistent synthesis of WG and DBT and underscores the physical validity of the hybrid approach.

\subsection{In-Depth Explanations of the Orbital Equation}
\textbf{1. Choice of exponential wave function $R(r)=R_0 e^{-r/\lambda}$}

The exponential form of the wave function is chosen for the following reasons:
\begin{itemize}
    \item \textbf{Approximation for bound states:}\\In the context of \gls{dbt}, $R(r)$ describes the wave function amplitude, which often decays exponentially when particles are localized in potential wells (e.g., gravitational potential). This resembles the solutions of the Schrödinger equation for bound states (e.g., in the hydrogen atom).
    \item \textbf{Asymptotic behavior:}\\For $r \gg \lambda$, the exponential decay dominates, justifying the simplification in Eq. (\ref{eq:q_wg_dbt}). The term $2/(r \lambda)$ becomes negligible compared to $1/\lambda^{2}$, making $Q$ approximately constant.
    \item \textbf{Physical meaning of $\lambda$:}\\$\lambda=\hbar/mc$ is the particle's Compton wavelength, characterizing its quantum mechanical \enquote{extension}. It defines the scale at which quantum effects become relevant.
\end{itemize}

\textbf{2. Compensation of the $\alpha \phi^{2}$ term}

The unphysical term $\alpha \phi^{2}$ in the WG orbital equation (Eq. \ref{eq:weber_r_2_ordnung}) would lead to a spiral deviation not observed in nature.
\gls{dbt} corrects this through:
\begin{itemize}
    \item \textbf{Quantum potential as counteraction:}\\The quantum potential $Q$ acts like a \enquote{restoring force} compensating the deviation. The form $Q \approx \phi^{2}$ (Eq. \ref{eq:q_laplace_wg_dbt}) results from the discrete Laplace operation on the wave function (Eq. \ref{eq:q_wg_dbt}).
    \item \textbf{Energy conservation:}\\While \gls{wg} describes classical gravity, \gls{dbt} introduces quantum fluctuations. The compensation shows that together they yield a stable, energy-conserving orbit—analogous to total energy minimization in quantum mechanics.
\end{itemize}

\textbf{3. Neglect of higher orders $\mathcal{O}(c^{-6})$}

\begin{itemize}
    \item \textbf{Significance of neglect:}\\Terms of order $c^{-6}$ are smaller than leading contributions by a factor of $(v/c)^{6}$. For planetary orbits ($v \ll c$), they are practically irrelevant (e.g., Mercury: $v/c \approx 10^{-4}$).
    \item \textbf{Experimental consequences:}\\Even modern tests of \gls{art} (e.g., LISA) lack the sensitivity to measure such corrections. The WG-DBT synthesis is thus sufficiently accurate to first order.
\end{itemize}

\textbf{4. Physical Interpretation of Compensation}

The exact cancellation of $\alpha \phi^{2}$ and $Q_\text{Comp}$ is no coincidence but results from the \textbf{consistent coupling} of \gls{wg} and \gls{dbt}:
\begin{itemize}
    \item \textbf{Non-locality as key:}\\While \gls{wg} contains instantaneous action-at-a-distance terms, \gls{dbt} describes global quantum correlations. Both require a \enquote{holistic} system description.
    \item \textbf{Emergent stability:}\\The compensation shows that the seemingly independent corrections of both theories ultimately share the same physical cause—the preservation of orbital stability through quantum mechanical self-organization.
\end{itemize}

The exponential wave function is a natural approximation for bound states, and the compensation of the $\alpha \phi^{2}$ term demonstrates the self-consistency of the WG-DBT synthesis.
The neglect of higher orders is experimentally justified, and the physical interpretation emphasizes the role of non-locality in both theories. Thus,
Section (\ref{sec:bahn_alpha_beweis}) is not only mathematically correct but also conceptually coherent.

\section{Derivation of Angular Velocity}
\label{sec:angular_velocity}

The angular velocity $\omega(\phi) = d\phi/dt$ follows from the WG orbital equation:

\begin{equation}
r(\phi) = \frac{a(1 - e^2)}{1 + e \cos(\kappa \phi)},
\end{equation}
where $\kappa = \sqrt{1 - \frac{6GM}{c^2 a (1 - e^2)}}$. 

With angular momentum $h = r^2 \dot{\phi}$:
\begin{equation}
\omega(\phi) = \frac{h}{r^2} = \sqrt{\frac{GM}{a^3 (1 - e^2)^3}} \left(1 + \frac{3GM}{c^2 a (1 - e^2)}\right) \left(1 + e \cos(\kappa \phi)\right)^2.
\end{equation}

\subsection*{Interpretation}
\begin{itemize}
\item WG leads to a \textbf{modulated angular velocity} depending on $\phi$.
\item At perihelion ($\phi = 0$), $\omega$ is maximal; at aphelion ($\phi = \pi/\kappa$), minimal.
\item ART provides an equivalent but structurally different prediction.
\end{itemize}

\subsection{Angular Velocity in the WG-DBT Synthesis}  
\label{sec:omega_dbt}  

The angular velocity $\omega(\phi)$ under the influence of quantum potential $Q$ is:  

\begin{equation}  
\omega(\phi) = \sqrt{\frac{GM}{r^3(\phi)}} \left[1 + \frac{3GM}{2c^2 r(\phi)} - \frac{\hbar^2}{4m^2 c^2 \lambda^3 r(\phi)}\right],  
\end{equation}  

where $r(\phi) = \frac{a(1-e^2)}{1 + e \cos(\kappa \phi)}$ describes the WG orbit.  

\subsection*{Consequences}  
\begin{itemize}  
\item \textbf{Microscopic systems:} The DBT correction $\propto \lambda^{-3}$ is measurable for electrons in atoms ($\Delta \omega/\omega \sim 10^{-4}$).  
\item \textbf{Planetary orbits:} The effect vanishes ($\lambda \gg r$), but the WG correction $\propto c^{-2}$ remains.  
\item \textbf{Difference from ART:} DBT introduces a \textbf{repulsive} component preventing singularities.  
\end{itemize}