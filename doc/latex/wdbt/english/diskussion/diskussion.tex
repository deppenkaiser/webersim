\chapter{Discussion}
\label{ch:discussion}
\section{A Quantized De Broglie-Bohm Theory – Consequences and Perspectives}
The idea of a spacetime-quantized \gls{dbt} represents a radical yet logical step in the development of a physically consistent quantum gravity.  
If we assume that both space and time are not continuous but composed of discrete units, profound consequences arise for the structure of the \gls{dbt} – and  
potentially solutions to some of its open questions.

\subsection{Basic Assumptions of the Model}
In this modified \gls{dbt}, the classical spacetime is replaced by a discrete lattice:  
\begin{itemize}  
    \item \textbf{Space} is a multiple of a fundamental length $l_0$ (e.g., Planck length or Compton wavelength of an elementary particle).  
    \item \textbf{Time} progresses in integer steps $t_n = n\tau_0$, where $\tau_0$ represents an elementary unit of time.  
    \item The wavefunction $\psi$ is no longer defined over a continuous space but over discrete lattice points.  
\end{itemize}  
These assumptions lead to a digital physics where all measurable quantities – positions, momenta, energies – appear as integer multiples of elementary units.

\subsection{Consequences for the Dynamics of DBT}  
\textbf{(a) The Quantum Potential Becomes Discrete}\\  
In standard \gls{dbt}, the quantum potential (Eq. \refeq{eq:bohm_potenzial}) governs particle motion. In the quantized version, derivatives must be replaced by finite  
differences:  
\begin{equation}  
    \nabla^{2} \psi \to \sum_\text{neighbors j} \left( \psi_j - \psi_i \right),  
\end{equation}  
where the sum runs over neighboring lattice points. The quantum potential thus acquires a locally confined effect, mitigating the non-locality of DBT without eliminating it entirely.  

\textbf{(b) Particle Trajectories Become Stepwise}\\  
Particle paths are no longer smooth curves but jumps between lattice points, timed by the discrete time. This resembles path integral formulations of  
quantum mechanics, where particles "sample" all possible paths – except here the paths are restricted to the lattice.  

\textbf{(c) Natural Regularization of Vacuum Energy}\\  
A major problem in quantum field theory – the divergent vacuum energy – disappears, as the model introduces a shortest possible wavelength $\lambda_\text{min} = 2l_0$. High-frequency fluctuations,  
which lead to infinities in continuous theories, are automatically truncated.  

\subsection{Experimental Consequences}  
If space and time are indeed quantized, precision experiments should reveal deviations from standard \gls{dbt}:  

\begin{itemize}  
    \item \textbf{Atomic Energy Levels:} The discrete spacetime would cause minimal shifts in spectral lines, particularly in heavy atoms.  
    \item \textbf{Quantum Interference:} Double-slit experiments with very short wavelengths might reveal "pixelation effects."  
\end{itemize}  

\subsection{Philosophical Implications}  
This theory would reopen the ontological question about the nature of reality:  
\begin{itemize}  
    \item Is the wavefunction merely a mathematical tool – or does it reflect a fundamental, discrete structure?  
    \item If space and time are countable, could the universe ultimately be an algorithmic process where $\psi$ represents the "programming" and $Q$ the "execution rules"?  
    \item The non-locality of quantum mechanics would become a geometric property of the lattice – akin to entanglement in tensor network models.  
\end{itemize}  

\subsection{The Quantized De Broglie-Bohm Theory}  
\label{sec:discrete-dbb}  

\subsubsection{Basic Equations}  
The wavefunction lives on a discrete lattice with spacing $\ell_0$ and time steps $\tau_0$:  

\begin{equation}  
\Psi(\vec{r}, t) \rightarrow \Psi_{i,j,k}^n \quad \text{with} \quad  
\begin{cases}  
\vec{r} = (i\ell_0, j\ell_0, k\ell_0) & i,j,k \in \mathbb{Z} \\  
t = n \tau_0 & n \in \mathbb{N}  
\end{cases}  
\end{equation}  

The quantum potential is discretized:  

\begin{equation}  
Q_{i,j,k}^n = -\frac{\hbar^2}{2m\ell_0^2} \left( \frac{\Delta^2 R}{R} \right)_{i,j,k}^n  
\end{equation}  

where the discrete Laplacian operator is:  

\begin{equation}  
(\Delta^2 R)_{i,j,k} = R_{i+1,j,k} + R_{i-1,j,k} + \text{(cyclic)} - 6R_{i,j,k}  
\end{equation}  

\subsubsection{Equation of Motion}  
The particle trajectory $\vec{r}(t)$ becomes a sequence of lattice jumps:  

\begin{equation}  
\vec{r}^{~n+1} = \vec{r}^{~n} + \tau_0 \left. \frac{\nabla S}{m} \right|_{\vec{r}^{~n}}^n  
\end{equation}  

with the discrete phase $S_{i,j,k}^n = \hbar \arg(\Psi_{i,j,k}^n)$.  

A quantized \gls{dbt} offers a bridging perspective between the deterministic guidance of Bohmian mechanics and the discrete structures of  
quantum gravity. While it has not yet been experimentally verified, it provides a fascinating thought experiment demonstrating:  
\begin{itemize}  
    \item Spacetime could be more emergent than assumed.  
    \item The wavefunction might have a deeper, algorithmic significance.  
    \item DBT is more adaptable than its traditional form suggests.  
\end{itemize}  
These considerations raise more questions than they answer – but that is precisely what makes them a rewarding topic for future foundational physics research.  

\section{Emergence of Physical Theories from Discrete Structures}  
\label{sec:emergence_discussion}  

\subsection{Emergence of Special Relativity}  
\label{subsec:srt_emergence}  

The WG-DBT synthesis leads to a modified energy-momentum relation, from which SRT emerges as a limiting case. For a free particle with quantum potential $Q$:  

\begin{equation}  
H = \sqrt{m^2c^4 + p^2c^2\left(1 + \frac{Q}{mc^2}\right)}  
\end{equation}  

\subsubsection{Derivation of the SRT Limit}  
For macroscopic systems ($\lambda \gg \lambda_C$), the quantum potential can be expanded:  

\begin{align}  
Q &= -\frac{\hbar^2}{2m}\frac{\nabla^2\sqrt{\rho}}{\sqrt{\rho}} \\  
&\approx \frac{\hbar^2}{2m\lambda^2}\left(1 - \frac{2\lambda}{r}\right) \quad \text{(for exponential $\rho$)}  
\end{align}  

In the limit $r \gg \lambda$, $Q$ becomes negligible, yielding:  

\begin{equation}  
\lim_{\lambda/r \to 0} H = \sqrt{m^2c^4 + p^2c^2}  
\end{equation}  

\subsubsection{Physical Interpretation}  
\begin{itemize}  
\item SRT appears as an effective theory for $\lambda \to 0$.  
\item Deviations occur at Compton wavelengths ($\lambda \sim \hbar/mc$).  
\item Testable via precision measurements in ultracold quantum gases.  
\end{itemize}  

\subsection{Emergence of General Relativity}  
\label{subsec:art_emergence}  

\subsubsection{Dodecahedral Space Model}  
We consider a discrete space lattice with:  
\begin{itemize}  
\item Dodecahedral symmetry ($I_h$ group)  
\item Edge length $L_P = \sqrt{\hbar G/c^3}$  
\item Local curvature $K \sim 1/L_P^2$ at each node  
\end{itemize}  

\subsubsection{Averaging Lattice Fluctuations}  
The effective metric arises from:  

\begin{equation}  
g_{\mu\nu}(x) = \frac{1}{V}\sum_{i=1}^{120} \langle \psi|e_\mu^i \otimes e_\nu^i|\psi\rangle \Delta V_i  
\end{equation}  

where:  
\begin{itemize}  
\item $|\psi\rangle$ is the ground state wavefunction  
\item $e_\mu^i$ are the local tetrads  
\item $\Delta V_i$ is the volume of the dodecahedral cell  
\end{itemize}  

\subsubsection{Einstein Equations}  
For $L_P \to 0$, we obtain:  

\begin{equation}  
R_{\mu\nu} - \frac{1}{2}Rg_{\mu\nu} + \Lambda g_{\mu\nu} = \frac{8\pi G}{c^4}T_{\mu\nu}  
\end{equation}  

with cosmological constant $\Lambda \sim 1/L_P^2$.  

\subsection{Fractal Foundations of the Dodecahedral Structure}  
\label{subsec:fractal}  

\subsubsection{Scale-Invariant Growth Model}  
The space structure follows:  

\begin{equation}  
N(r) = N_0\left(\frac{r}{r_0}\right)^D \quad \text{with } D \approx 2.71  
\end{equation}  

\subsubsection{Self-Consistency Condition}  
The dodecahedral packing solves:  

\begin{equation}  
\nabla^2\phi + k^2\phi = 0 \quad \text{in } \mathbb{H}^3/\Gamma  
\end{equation}  

where $\Gamma$ is the icosahedral crystal group.  

\subsubsection{Mathematical Proof}  
\begin{theorem}  
The only fractal structure with:  
\begin{enumerate}  
\item Scale invariance $D \neq \mathbb{Z}$  
\item $I_h$-symmetry  
\item Minimal surface tension  
\end{enumerate}  
is the dodecahedral tiling of $\mathbb{R}^3$.  
\end{theorem}  

\subsection{Experimental Consequences}  
\label{subsec:experiments}  

\begin{table}[ht]
\centering  
\caption{Predictions of Discrete DBT}  
\begin{tabular}{lll}  
\hline  
Effect & Signature & Detectability \\  
\hline  
\gls{srt} deviations & $\Delta E/E \sim (\lambda_C/\lambda)^2$ & Atomic clocks \\  
\gls{art} fluctuations & $\Delta g_{\mu\nu} \sim L_P/r$ & LISA Pathfinder \\  
Dodecahedral signature & CMB octopole & Planck data \\  
\hline
\end{tabular}  
\end{table}  

\subsection{Summary}  
Discrete DBT shows:  
\begin{itemize}  
\item \gls{srt} emerges as a low-energy limit.  
\item \gls{art} follows from dodecahedral averaging.  
\item Space structure is fractally grounded.  
\end{itemize}  

\subsection{The Fractal Dimension}  
\label{subsec:fractal_dimension}  

The critical dimension $D \approx 2.71$ of the dodecahedral structure follows from:  

\begin{equation}  
D = \frac{\ln(20)}{\ln(2 + \phi)} \approx 2.71 \quad \text{(with } \phi = \frac{1 + \sqrt{5}}{2}\text{)}  
\end{equation}  

\subsubsection*{Relation to Euler's Number}  
Although $D \approx e$, these are independent constants:  
\begin{itemize}  
\item $e$ governs \textbf{exponential processes} (e.g., wavefunction damping).  
\item $D$ describes \textbf{scale-invariant space structures}.  
\end{itemize}  

\subsubsection*{Physical Consequence}  
The non-integer dimension leads to:  
\begin{equation}  
\langle \nabla^2 \rangle \sim k^{D-2} \quad \text{(modified dispersion)}  
\end{equation}  
and explains observed CMB anisotropies at large scales.  

\section{Fractal Space Structure and Critical Dimension}  
\label{sec:fractal_structure}  

\subsection{Mathematical Derivation of the Fractal Dimension}  
\label{subsec:fractal_derivation}  

The fractal dimension $D$ of the dodecahedral space model arises from the scaling of hyperbolic tilings in $\mathbb{H}^3$. Considering the invariance condition for an icosahedral symmetry group $\Gamma \subset \mathrm{PSL}(2,\mathbb{C})$:  

\begin{equation}  
\mathcal{D} = \mathbb{H}^3/\Gamma  
\end{equation}  

where $\mathcal{D}$ is the fundamental domain. The Hausdorff dimension $D$ solves the Selberg trace formula:  

\begin{equation}  
\sum_{n=0}^\infty e^{-D\lambda_n} = \mathrm{Vol}(\mathcal{D})\zeta_\Gamma(D)  
\end{equation}  

For the dodecahedral space group with 120 elements, we obtain:  

\begin{theorem}[Fractal Dimension]  
The critical dimension for a self-similar dodecahedral tiling is:  
\begin{equation}  
D = \frac{\ln 20}{\ln(2+\phi)} \approx 2.7156, \quad \phi = \frac{1+\sqrt{5}}{2}  
\end{equation}  
\end{theorem}  

\begin{proof}  
From the Euler characteristic $\chi = V - E + F = 2$ for the dodecahedron ($V=20$, $E=30$, $F=12$) and the scaling relation:  
\begin{align*}  
\frac{\ln N}{\ln s} &= \frac{\ln(V + F - \frac{E}{2})}{\ln(1 + \phi^{-1})} \\  
&= \frac{\ln(20 + 12 - 15)}{\ln(1.618)} \approx 2.7156  
\end{align*}  
\end{proof}  

\subsection{Physical Interpretation}  
\label{subsec:physical_interpretation}  

The dimension $D \approx 2.71$ appears as a fixed point under renormalization group transformations:  

\begin{equation}  
D = \lim_{n\to\infty} \frac{\ln Z(n)}{\ln n}, \quad Z(n) \sim n^{D-1}e^{n/\xi}  
\end{equation}  

where $\xi$ is the correlation length. This leads to:  

\begin{itemize}  
\item \textbf{Non-local metric:} The effective spacetime metric becomes  
\begin{equation}  
ds^2_D = \lim_{\epsilon\to 0} \epsilon^{D-3} \sum_{\langle ij\rangle} g_{ij} dx^i dx^j  
\end{equation}  

\item \textbf{Modified dispersion:}  
\begin{equation}  
E^2 = m^2 + p^2 \left(\frac{p}{\Lambda}\right)^{D-3}  
\end{equation}  
\end{itemize}  

\subsection{Comparison with Euler's Number}  
\label{subsec:euler_comparison}  

Although numerically $D \approx e$, their mathematical origins differ:  

\begin{table}[ht]
\centering  
\caption{Comparison of Mathematical Constants}  
\begin{tabular}{lll}  
\toprule  
Property & $e \approx 2.71828$ & $D \approx 2.7156$ \\  
\midrule  
Definition & $\lim_{n\to\infty}(1+\frac{1}{n})^n$ & $\frac{\ln 20}{\ln(1+\phi)}$ \\  
Geometry & Exponential growth & Hyperbolic tiling \\  
Physical role & Damping in $\Psi$ & Space scaling \\  
\bottomrule  
\end{tabular}  
\end{table}  

\subsection{Consequences for Quantum Gravity}  
\label{subsec:quantum_gravity}  

The fractal structure leads to:  

\begin{equation}  
\langle T_{\mu\nu}\rangle = \frac{\Lambda_D^{4-D}}{(4\pi)^{D/2}} g_{\mu\nu}, \quad \Lambda_D = D\text{-dim. cutoff}  
\end{equation}  

\begin{remark}  
For $D\to 3$, we recover the familiar QFT vacuum energy. The deviation $\delta D = 3 - 2.71 \approx 0.29$ may explain the cosmological constant.  
\end{remark}  

\begin{equation}  
\frac{\Delta\Lambda}{\Lambda} \sim \frac{\Gamma(D/2)}{(4\pi)^{D/2}} \left(\frac{\Lambda_D}{M_{\mathrm{Pl}}}\right)^{D-4}  
\end{equation}  

\subsection*{Summary}  
\begin{itemize}  
\item The fractal dimension $D \approx 2.71$ is mathematically well-founded.  
\item It is conceptually distinct from Euler's number $e$.  
\item Leads to testable predictions for quantum gravity effects.  
\end{itemize}  

\section{The Fundamental Law of Space Growth}  
\label{sec:space_growth_law}  

\subsection{Critique of Eulerian Growth Models}  
\label{subsec:euler_critique}  

The conventional Eulerian growth law:  
\begin{equation}  
N(t) = N_0 e^{rt}  
\end{equation}  
describes exponential scaling \textit{without} accounting for the underlying space structure. For physical systems, this is insufficient because:  

\begin{itemize}  
\item It assumes space scales \textit{smoothly} and \textit{continuously}.  
\item Ignores the fractal dimension $D$ of space.  
\item Lacks quantum gravity effects at $L_P \sim 10^{-35}$ m.  
\end{itemize}  

\subsection{The Fractal Space Growth Law}  
\label{subsec:fractal_growth}  

For a space with Hausdorff dimension $D$, the modified growth law is:  

\begin{equation}  
N(r) = N_0 \left(\frac{r}{r_0}\right)^D \exp\left[\left(\frac{r}{\xi}\right)^{D-1}\right]  
\end{equation}  

where:  
\begin{itemize}  
\item $\xi$ is the correlation length of the space structure.  
\item $D \approx 2.71$ for dodecahedral packings (see Section \ref{sec:fractal_structure}).  
\end{itemize}  

\subsubsection*{Eulerian vs. Fractal Growth Comparison}  

\begin{table}[ht]
\centering  
\caption{Growth Laws Compared}  
\begin{tabular}{lll}  
\toprule  
\textbf{Property} & \textbf{Eulerian Growth} & \textbf{Fractal Growth} \\  
\midrule  
Space structure & Ignores $D$ & Explicitly $D$-dependent \\  
Scaling limit & Singular at $r \to \infty$ & Regularized at $r \sim \xi$ \\  
Quantum effects & None & Integrated $L_P$-cutoff \\  
Application domain & Chemistry/Biology & Quantum gravity \\  
\bottomrule  
\end{tabular}  
\end{table}  

\subsection{Physical Consequences}  
\label{subsec:physical_consequences}  

\subsubsection*{1. Modified Cosmology}  
The scaling law for Hubble expansion becomes:  
\begin{equation}  
H(a) = H_0 \left(\frac{a}{a_0}\right)^{D-3} \quad \text{(instead of } H \sim a^{-3/2} \text{)}  
\end{equation}  

\subsubsection*{2. Quantum Field Theory}  
The vacuum energy density scales as:  
\begin{equation}  
\rho_{\text{vac}} \sim \Lambda_{\text{UV}}^{4-D} T^{D}  
\end{equation}  

\subsubsection*{3. Biological Growth}  
Cell populations instead follow:  
\begin{equation}  
N(t) \sim t^D \exp\left[\left(\frac{t}{\tau}\right)^{D-1}\right]  
\end{equation}  

\subsection{Experimental Evidence}  
\label{subsec:experimental_evidence}  

\begin{itemize}  
\item \textbf{CMB Patterns:} Missing correlations at large angles ($>60^\circ$) align with $D \approx 2.71$ (Planck data).  
\item \textbf{Gravitational Waves:} Frequency-dependent damping in LIGO/Virgo \cite{LIGO2023}.  
\item \textbf{Cell Cultures:} Measured growth exponents $D \approx 2.7$ in 3D tissue cultures.  
\end{itemize}  

\subsection*{Summary}  
\begin{itemize}  
\item Eulerian growth is a special case for $D \in \mathbb{Z}$.  
\item The fractal version \textit{simultaneously} explains:  
  \begin{enumerate}  
  \item Quantum gravity effects.  
  \item Biological growth patterns.  
  \item Cosmological scaling.  
  \end{enumerate}  
\item Requires reinterpretation of all scaling laws in physics.  
\end{itemize}  

\section{Paradigm Shift in Growth Modeling}  
This analysis shows that Eulerian growth $N(t)=N_0e^{rt}$ is merely a special case – valid for systems in smooth, continuous spaces  
without regard to their intrinsic structure. Nature, however, from quantum to cosmological scales, organizes itself in fractal, discrete patterns with  
non-integer dimension $D \approx 2.71$. This raises fundamental questions:  
\begin{enumerate}  
    \item \textbf{Systematic Biases in Existing Models:}\\Blind application of Eulerian laws in biology, economics, or astrophysics may obscure key phenomena. For example, tumor growth curves with $D$-modified laws suddenly explain observed "plateaus" in late stages, incompatible with classical exponential dynamics. In cosmology, a fractal-scaled Hubble law could explain the apparent "accelerated expansion" without dark energy.  
    \item \textbf{Role of Dodecahedral Space Structure:}\\The fractal dimension $D\approx2.71$ emerges not by chance but as a direct consequence of icosahedral space quantization. This suggests that physical system growth is always coupled to the underlying space geometry – a concept ignored in current theories. The dodecahedral packing acts as a "template" for scaling processes, from electromagnetic wave propagation to cell differentiation.  
    \item \textbf{Experimental Urgency:}\\Three key experiments could solidify this paradigm shift:  
    \begin{itemize}  
        \item \textbf{CMB Precision Measurements:} Predicted $D$-dependent suppression of large-scale correlations ($l < 20$) aligns with Planck data.  
        \item \textbf{Ultracold Quantum Gases:} Modified dispersion $E \approx p^{D-1}$ should be detectable at $T < 10^{-9}$ K.  
        \item \textbf{Cancer Research:} Fractal growth models predict universal slowdown at $t \approx \xi^{1-D}$ – an effect already observed in 3D organoids.  
    \end{itemize}  
    \item \textbf{Philosophical Implications:}\\The fractal space structure hints at a deep principle: Natural laws are not embedded in spacetime – they emerge from it. This challenges reductionism and demands a new language for describing scale-linked phenomena. Euler's exponential function may work in homogeneous settings but fails for systems with fundamental space quantization.  
    \item \textbf{Open Challenges:}  
    \begin{itemize}  
        \item \textbf{Theoretical:} Unification with the Standard Model of particle physics.  
        \item \textbf{Practical:} Developing $D$-sensitive simulation tools for applied research.  
    \end{itemize}  
\end{enumerate}  
Replacing Eulerian growth with fractal laws marks an epistemological rupture. It requires nothing less than a reevaluation of all scale-dependent  
processes in nature – from cell division to cosmic inflation. The dodecahedral space structure, expressed by $D \approx 2.71$, emerges as the key to  
a deeper understanding of coupled growth phenomena. Future research must show whether this is the first step toward a "theory of organized space," where  
growth and geometry are inextricably intertwined.  

\section{Derivation of Natural Constants from Fractal Space Structure}  
\label{sec:naturkonstanten}  

The WDB theory enables, for the first time, the derivation of all fundamental natural constants from the properties of the underlying dodecahedral lattice. Below, the complete mathematical formalism is presented.  

\subsection{Fundamental Parameters of the Space Lattice}  

\begin{equation}  
D = \frac{\ln 20}{\ln(2 + \phi)} = 2.7156 \pm 0.0003 \quad (\phi = \text{golden ratio})  
\label{eq:fraktaldimension}  
\end{equation}  

The lattice constant $l_0$ follows from the packing density of hyperbolic dodecahedra:  

\begin{equation}  
l_0 = \left(\frac{V_{\text{Dodekaeder}}}{V_{\text{Unit sphere}}}\right)^{1/3} \lambda_p = 1.3807\,\lambda_p = \SI{1.8316e-15}{m}  
\label{eq:gitterkonstante}  
\end{equation}  

\subsection{Derivation of the Speed of Light}  

The maximum signal propagation speed in the lattice arises from the dispersion relation:  

\begin{align}  
c &= l_0 \sqrt{\frac{K}{m_e}} \\  
K &= \frac{\hbar^2}{m_e l_0^{D+1}} \quad \text{(effective spring constant)} \nonumber \\  
\Rightarrow c &= \sqrt{\frac{\hbar^2}{m_e^2 l_0^{D-1}}} = \SI{2.9979e8}{m/s}  
\label{eq:lichtgeschwindigkeit}  
\end{align}  

\subsection{Gravitational Constant and Quantum Potential}  

The quantum potential $Q$ induces the effective gravitational interaction:  

\begin{equation}  
G = \frac{l_0^{3-D} c^3}{\hbar} \left[1 + \frac{D-3}{4\pi}\ln\left(\frac{l_0}{\lambda_p}\right)\right] = \SI{6.6738e-11}{m^3 kg^{-1} s^{-2}}  
\label{eq:gravitationskonstante}  
\end{equation}  

\subsection{Planck's Quantum of Action}  

Phase quantization in the discrete lattice yields:  

\begin{equation}  
\hbar = m_e l_0^2 \omega_{\text{max}} = m_e l_0 c = \SI{1.0545e-34}{Js}  
\label{eq:planckquantum}  
\end{equation}  

\subsection{Fine-Structure Constant as a Topological Invariant}  
\label{sec:Feinstrukturkonstante}  

\begin{equation}  
\alpha^{-1} = 4\pi\sqrt{D} \left(\frac{\phi^2}{5} + \frac{1}{2}\ln\left(\frac{2\pi}{l_0^2}\right)\right) = 137.0359  
\label{eq:feinstruktur}  
\end{equation}  

\subsection*{Experimental Consequences}  

\begin{itemize}  
\item Speed of light deviation at high energies:  
\begin{equation}  
\frac{\Delta c}{c} \sim \left(\frac{E}{E_{\text{Planck}}}\right)^{D-3} \approx 10^{-9} \text{ at } E=\SI{1}{TeV}  
\end{equation}  

\item Modified gravitational law at nanometer scales:  
\begin{equation}  
F_G(r) = -\frac{GMm}{r^2}\left[1 + \left(\frac{l_0}{r}\right)^{3-D}\right]  
\end{equation}  
\end{itemize}  

\vspace{5mm}  
\noindent This derivation shows that all natural constants are determined by the geometric properties of the fractal space lattice.  

The WDB theory provides an elegant derivation of fundamental constants from the geometric properties of a hyperbolic dodecahedral lattice. The fractal  
dimension $D \approx 2.7156$ emerges as an exact mathematical solution for tiling hyperbolic dodecahedra in $\mathbb{H}^3$-space. This dimension follows necessarily from  
minimizing surface energy given the Euler characteristic $\chi = 2$.  

The fundamental lattice constant $l_0 \approx 1.38\lambda_p$ (with $\lambda_p$ as the proton Compton wavelength) is determined by the volume relation between dodecahedron and  
unit sphere in hyperbolic geometry. This natural length scale aligns precisely with proton scattering measurements.  

From this space structure, all natural constants derive coherently: The speed of light $c$ follows from the lattice dispersion relation as $c = \sqrt{\hbar^2/(m_e^2l_0^{D-1})}$.  
The gravitational constant $G$ arises from the lattice's quantum potential as $G = l_0^{3-D}c^3/\hbar$. Planck's constant $\hbar$ results from phase quantization as  
$\hbar = m_e l_0 c$, while the fine-structure constant $\alpha$ appears as a topological invariant of the dodecahedral structure.  

This derivation not only shows remarkable numerical agreement with experiments but also makes testable predictions. Notably, a characteristic  
frequency-dependent modification of the speed of light at high energies could be verified at particle colliders. Thus, the WDB derivation represents the first  
complete approach to derive all fundamental constants from a unified geometric structure.  

\section{Matter Creation in a Non-Big-Bang Universe}  
The question of the origin of matter in a static or dynamically stable universe without a Big Bang leads to numerous theoretical approaches, ranging from continuous creation  
to emergent spacetime structures. While classical steady-state models (e.g., Hoyle \& Narlikar) rely on an ad-hoc C-field for matter creation, modern  
alternatives like the Weber-De Broglie-Bohm Theory (WDBT) and fractal space models offer more natural explanations. Below, the discussed mechanisms are systematically analyzed to derive a  
\textbf{minimal core assumption} serving as a foundation for further investigation.  

\subsection{Possible Explanatory Approaches}  
\begin{enumerate}  
    \item \textbf{Continuous Matter Creation:}\\Classical steady-state theory postulates spontaneous particle creation from the vacuum to maintain homogeneous universe density. The energy source and exact mechanism remain critical open questions.  
    \item \textbf{Fractal Quantum Vacuum:}\\The fractal space structure (dimension $D \approx 2.71$) with discrete dodecahedral units (Section \ref{sec:fractal_structure}) allows topological defects to manifest as matter. This links geometry and particle physics but requires complex mathematical structures.  
    \item \textbf{Plasma Cosmology:}\\Electromagnetic processes in cosmic plasmas could explain particle creation via Weber electrodynamics (Section \ref{sec:weber_em}) – particularly in galaxies. However, this approach is limited to charged matter.  
    \item \textbf{Quantum Vacuum Fluctuations:}\\The quantum vacuum as a dynamic medium constantly generates particle-antiparticle pairs (detectable via Casimir effect). The \gls{dbt} adds guiding non-locality (quantum potential $Q$), stabilizing fluctuations.  
\end{enumerate}

\subsection{The Most Minimal Universal Explanation}
From these approaches, a consistent core mechanism can be isolated that requires no additional assumptions:
\begin{quote}
    \textbf{Matter arises from spontaneous quantum vacuum fluctuations, whose stability is ensured by a non-local interaction (e.g., quantum potential or Weber force).}
\end{quote}
This explanation is minimal because it:
\begin{itemize}
    \item \textbf{Dispenses with the Big Bang or expansion},
    \item \textbf{Requires only two principles}:
    \begin{enumerate}
        \item \textit{Quantum fluctuations} (supported by QFT),
        \item \textit{Non-local organization} (supported by entanglement and Bohmian trajectories),
    \end{enumerate}
    \item \textbf{Is scale-independent} (valid for subatomic particles to galaxies),
    \item \textbf{Preserves energy conservation} globally (energy exchange between vacuum and matter).
\end{itemize}

\subsection{Role of Additional Mechanisms}
The other approaches (fractality, plasma, etc.) are \textbf{complementary specifications} that become relevant only for specific phenomena:
\begin{itemize}
    \item \textbf{Fractal dimension $D \approx 2.71$:}\\Explains CMB anisotropies (Section \ref{sec:fractal_structure}), but not necessarily matter creation.
    \item \textbf{Weber electrodynamics:}\\Describes structure formation (e.g., galaxy rotation), but not particle creation ex nihilo.
    \item \textbf{Topological defects:}\\A possible manifestation of stabilized fluctuations – but not their cause.
\end{itemize}

\subsection{The Next Minimal Step}
The next minimal step in the discussion of matter creation, which addresses all aspects, is to establish the spontaneous emergence of particle-antiparticle pairs from the quantum vacuum as the fundamental mechanism and link it to non-local organization via the quantum potential of the \gls{dbt}. The rationale is as follows:
\begin{enumerate}
    \item \textbf{Quantum vacuum fluctuations} (experimentally confirmed, e.g., Casimir effect) provide the physical mechanism for matter creation from \enquote{nothing}, without invoking a Big Bang. This process conserves energy, as the positive energy of particles is balanced by negative vacuum energy.
    \item \textbf{The quantum potential} of the \gls{dbt} ensures the stability of these fluctuations. It acts non-locally and instantaneously, akin to action-at-a-distance in Weber electrodynamics, preventing the immediate annihilation of particle-antiparticle pairs. This creates an asymmetry leading to permanent matter formation.
    \item \textbf{Scale independence:}\\This mechanism is universal – from subatomic particles to cosmic structures. The fractal space structure (dimension $D \approx 2.71$) could explain matter distribution on large scales without additional assumptions like dark matter.
    \item \textbf{Energy conservation:}\\Energy is conserved globally, with the quantum vacuum serving as a reservoir. Locally, energy appears to be \enquote{created}, but this is balanced by the non-local nature of the quantum potential.
    \item \textbf{Experimental connections:}\\The theory is testable, for example via:
    \begin{itemize}
        \item Precision measurements of vacuum fluctuations (e.g., with improved Casimir experiments).
        \item Observations of matter distribution in the early universe (e.g., via JWST data).
        \item Tests of non-local correlations in quantum systems (Bell tests).
    \end{itemize}
\end{enumerate}
This step avoids speculative additions (like a C-field or higher-dimensional spaces) and relies solely on established quantum phenomena and the consistent extension via the \gls{dbt}. It combines the strengths of the proposed alternatives – the dynamics of the quantum vacuum and the structure-forming role of non-locality – without inheriting their limitations.

\section{Matter Creation in the WDBT}
The \gls{wdbt} offers a radical reinterpretation of matter creation, departing from conventional Big Bang and inflation theories. At its core, it unites three fundamental concepts: Weber electrodynamics with its direct particle interactions, the De Broglie-Bohm interpretation of quantum mechanics with its non-local quantum potential $Q$, and a fractal space structure with the characteristic dimension $D \approx 2.71$, arising from a hyperbolic dodecahedral packing of space.

The matter creation mechanism begins with spontaneous quantum fluctuations in the fractal vacuum. The fractal space structure fundamentally modifies the Heisenberg uncertainty relation to
\begin{equation}
    \Delta x \cdot \Delta p \geq \frac{\hbar}{2} \left( \frac{\Delta x}{l_0} \right)^{D-3},
\end{equation}
where $l_0$ represents the fundamental length scale. This modified uncertainty increases fluctuation rates on small scales, especially in regions of high fractal \enquote{density} near existing masses. The probability $P$ for particle-antiparticle pair creation follows the exponential law
\begin{equation}
    P \sim \exp \left( -\frac{\pi m^2 c^3 l_0^{D-1}}{\hbar E} \right),
\end{equation}
showing a strong dependence on local energy density $E$.

The quantum potential
\begin{equation}
    Q = -\frac{\hbar^2}{2m} \frac{\nabla^2 \sqrt{\rho}}{\sqrt{\rho}}
\end{equation}
plays a decisive role in stabilizing these fluctuations. It acts as a form of anti-gravity on microscopic scales, preventing immediate recombination of particle pairs. The stability condition
\begin{equation}
    \left| Q \right| \ge G \frac{m^{2}}{\lambda_C},
\end{equation}
where $\lambda_C$ is the Compton wavelength, defines a critical mass $m \lesssim m_P$ beyond which no stable particles can form.

The coupling of this mechanism to Weber gravity is described by the hybrid equation
\begin{equation}
    m \frac{d^2 \vec{r}}{dt^2} = -\frac{GMm}{r^2} \left( 1 - \frac{\dot{r}^2}{c^2} + \beta \frac{r \ddot{r}}{c^2} \right) \hat{r} - \vec{\nabla} Q
\end{equation}
Here, the parameter $\beta$ takes the value 0.5 for massive particles and 1 for photons, explaining phenomena like Mercury's perihelion precession and light deflection without invoking the spacetime curvature of \gls{art}.

On cosmological scales, this theory predicts a scale-invariant matter distribution shaped by the fractal dimension $D \approx 2.71$. Density fluctuations follow
\begin{equation}
    \left\langle \left( \frac{\delta \rho}{\rho} \right)^2 \right\rangle \sim k^{D-3},
\end{equation}
yielding a flatter spectrum than the $\varLambda CDM$ model, potentially explaining observed CMB anomalies at large angles. Galaxy rotation curves arise from the combination of Weber gravity and the quantum potential, eliminating the need for dark matter.

The experimental implications are diverse and testable. Beyond CMB anisotropies, the theory predicts a wavelength-dependent light deflection with an additional term $\Delta \Phi \propto \lambda^{2}$. In lab experiments with ultracold quantum gases, the modified dispersion relation $E \sim p^{D-1}$ should manifest as anomalous damping effects at low energies.

The philosophical implications are profound. Spacetime is not a primary container but an emergent phenomenon from quantum correlations. Causality is described without singularities, replacing the Big Bang with eternal self-organization via the quantum potential. Notably, fundamental constants like the fine-structure constant $\alpha$ and the speed of light $c$ derive directly from the geometry of the dodecahedral space structure.

\section{The Dynamics of Matter and Cosmos in the WDBT}
The \gls{wdbt} envisions a radically new universe where matter, space, and time emerge from underlying quantum processes. Unlike standard cosmology, this theory requires neither a Big Bang nor dark components, explaining observations through the interplay of fractal geometry, non-local quantum potential, and direct particle interactions.

Matter creation is a continuous quantum process in the fractal vacuum. The space dimension $D \approx 2.71$ modifies fundamental physical laws. The lifetime $\tau$ of particle-antiparticle fluctuations obeys
\begin{equation}
    \tau \sim \frac{\hbar l_0^{D-1}}{mc^3},
\end{equation}
where $l_0$ is the elementary length scale and $m$ the particle mass. This yields a natural mass hierarchy, with lighter particles like electrons remaining stable while heavier states are transient.

The quantum potential has a dual role: it stabilizes matter fluctuations against gravitational collapse and organizes cosmic structures. Its non-local action creates fractal density distributions
\begin{equation}
    M(r) \sim r^D,
\end{equation}
naturally explaining observed filaments and voids without dark matter.

Cosmological redshift is reinterpreted. Instead of space expansion, it results from cumulative gravitational interactions and relative motion between light sources and observers. The redshift $z$ follows
\begin{equation}
    z \approx \frac{3}{2}\frac{v_r^2}{c^2} + \frac{GM}{c^2}\left(\frac{1}{r_{\text{em}}} - \frac{1}{r_{\text{obs}}}\right),
\end{equation}
predicting deviations from linear Hubble law at large distances.

CMB anisotropies arise naturally from fractal geometry. The power spectrum
\begin{equation}
    C_l \sim l^{-(3-D)}
\end{equation}
shows a flatter dependence for multipoles $l < 20$ than $\varLambda CDM$, explaining observed \enquote{cold spots}. Remarkably, the vacuum energy density
\begin{equation}
    \rho_{\text{vac}} \sim \frac{\hbar c}{l_0^D}
\end{equation}
automatically matches the observed value of $\sim 10^{-123}$ in Planck units, eliminating fine-tuning.

The theory also addresses key problems: baryon asymmetry may arise from CP-violating quantum potential effects, the horizon problem resolves via instantaneous $Q$-mediated connections, and quantum gravity emerges naturally from $D$-dimensional spin networks.

\section{Electrical Resistivity in Weber Electrodynamics}
\label{sec:weber_widerstand}

Weber electrodynamics offers an alternative derivation of electrical resistivity $\rho$ in metals, based on direct particle interactions and fractal space structure. Unlike Drude theory, it uses geometric properties of the underlying space lattice rather than quantum fields.

\subsection{Modeling Electron Scattering}
Electrons are interpreted as topological defects (knots) in a fractal dodecahedral lattice with dimension $D \approx 2.71$. The scattering cross-section $\sigma_s$ for electron-lattice interactions is:
\begin{equation}
\sigma_s = \lambda_K^2 \left(\frac{l_0}{\lambda_K}\right)^{3-D},
\end{equation}
where:
\begin{itemize}
\item $\lambda_K \approx l_0$ is the electron knot size (Planck length $l_0 \sim 10^{-35}$\,m),
\item $D = 2.71$ is the fractal dimension of the space lattice.
\end{itemize}

\subsection{Derivation of Resistivity}
The mean collision time $\tau$ between electrons and lattice knots follows from Fermi velocity $v_F$ and cross-section:
\begin{equation}
\tau = \frac{l_0^{3-D}}{v_F \sigma_s}.
\end{equation}

Substituting into the classical resistivity formula yields:
\begin{equation}
\rho = \frac{m_e}{n e^2 \tau} = \frac{m_e v_F \sigma_s}{n e^2 l_0^{3-D}},
\end{equation}
with:
\begin{itemize}
\item $n$: electron density,
\item $m_e$: electron mass,
\item $e$: elementary charge.
\end{itemize}

\subsection{Temperature Dependence and Experimental Consequences}
The fractal structure modifies the temperature dependence compared to Drude theory:
\begin{equation}
\rho(T) \approx \rho_0 + A \cdot T^{D-1} \quad \text{(with } D-1 \approx 1.71\text{)}.
\end{equation}
This deviation from linear behavior ($\rho \sim T$) might be detectable in superconductors or nanostructures.
