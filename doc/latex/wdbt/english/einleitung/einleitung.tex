\chapter{Foundations of an Alternative Quantum Gravity}
\section{Motivation}
Modern physics faces fundamental contradictions: While \gls{art} describes gravity as the curvature of spacetime, \gls{srt} is based on idealized inertial frames
that strictly speaking cannot exist in curved spacetime. This conflict raises questions—for instance about the nature of the speed of light, which is absolute in \gls{srt} but
locally variable in \gls{art}.
\begin{quote}
    \enquote{Einstein's postulates contain inherent contradictions when applied to real gravitational systems, challenging the universality of special relativity.} \cite{Rubcic1998}
\end{quote}
Additionally, there are unresolved problems in quantum mechanics: wave-particle duality, the \enquote{collapse} of the wave function upon measurement, and non-local entanglement. Even successful
theories like \gls{qed} postulate seemingly paradoxical phenomena, such as virtual photons with superluminal speeds in the path integral formalism.

These tensions suggest that established models may only be approximations of a deeper reality. Rather than following dogmas, we should examine alternative perspectives
like Weber electrodynamics or \gls{dbt}, which are introduced in this book.
\begin{quote}
    \enquote{The observer-dependent collapse of the wavefunction is not a fundamental feature of nature but a limitation of the standard interpretation.} \cite{bohm1952}
\end{quote}

\subsection{Dogmatism and Blind Spots in Modern Physics}
Contemporary physics suffers from a paradoxical situation: On one hand, established theories like \gls{art} or quantum field theory are rarely questioned, despite their fundamental weaknesses
—singularities in black holes, infinite self-energies of particles, or the need for \enquote{dark} entities. On the other hand, unorthodox approaches are often filtered out during peer review,
even though they could offer solutions to these problems.

An example is the interpretation of the \gls{cmb} as evidence for the Big Bang. Alternative explanations—such as thermal equilibrium processes in plasmas—are hardly discussed, even though they
avoid singularities. The same applies to galactic redshift, which does not necessarily imply universal expansion.
\begin{quote}
    \enquote{Theoretical physics has become stuck in a paradigm that values mathematical elegance over empirical testability, leading to a stagnation of genuine progress.} \cite{Smolin2006}
\end{quote}

\subsection{Speculation Over Progress}
Since the revolutionary breakthroughs of quantum mechanics and relativity theory a century ago, there have been few comparable advances. Instead, speculative concepts like
higher dimensions or multiverses dominate, which are hardly empirically verifiable.

Yet science should focus on observable phenomena. Weber electrodynamics demonstrates how electromagnetic effects can be described without fields—through direct
interactions between charges. Such approaches could pave the way for a more consistent physics.

\subsection{Alternative Theories}
A central problem in modern physics lies in its excessive reliance on mathematics. Just because something can be formulated mathematically
does not mean it corresponds to physical reality. Yet instead of acknowledging these limits, fundamental principles of classical physics—such as energy conservation
or the laws of thermodynamics—are abandoned in favor of abstract equations. \gls{art}, for example, postulates a dynamic spacetime that seemingly creates or destroys energy
out of nothing. Where is the strict balance of physics in this?

Concrete contradictions emerge in practice: According to \gls{art}, planets should lose energy through gravitational wave emission—so why have planetary orbits remained
stable for billions of years? If spacetime is described as an elastic entity that can deform and move: What force performs work here, and where does the energy come from?
Standard explanations remain vague or resort to mathematical tricks.

The supposed evidence for the Big Bang is also far from unambiguous. The cosmic \gls{cmb} is automatically interpreted as an echo of the Big Bang
—but alternative explanations exist, such as thermal equilibrium processes or scattering phenomena.
\begin{quote}
    \enquote{The interpretation of cosmic microwave background as proof of the Big Bang ignores alternative explanations, such as intrinsic redshifts in plasma cosmology.} \cite{Arp1998}
\end{quote}
Similarly, galactic redshift might not only result from expansion but also from other mechanisms. Even phenomena like light deflection or the
Shapiro effect can be explained without \gls{art} if alternative gravitational models are considered.
\begin{quote}
    \enquote{Weber's formulation of electrodynamics provides a consistent framework for gravitational phenomena without invoking curved spacetime.} \cite{WeberElectrodynamics}
\end{quote}
This book aims to present such alternative explanations. Physics must not stop at mathematical dogmas—it must return to logic, experiment, and
real causality.

\section{Divergent Perspectives in Physics: Light, Relativity, and Alternative Models}
\subsection{Feynman's Particle Model of Light}
Richard Feynman argued that even interference phenomena can be explained by particles (photons)—without waves. This raises the question: Is wave-particle duality truly
necessary, or does it merely reflect the limits of our models?

\subsection{Contradictions in QED: Superluminal Photons and Path Integrals}
The path integral formalism of \gls{qed} sums over all possible photon paths—including those with superluminal speeds. Mathematically, this leads to correct predictions,
but physically, it remains unclear:
\begin{itemize}
    \item If photons can virtually exceed the speed of light, does this not contradict \gls{srt}?
    \item Is the speed of light truly an absolute limit, or just a macroscopic effect?
\end{itemize}

\subsection{Energy-Dependent Speed of Light? Experimental Hints}
Some alternative theories (e.g., loop quantum gravity or VSL models) propose that the speed of light might depend on photon energy.
Possible evidence:

\begin{itemize}
    \item Gamma-ray bursts with extremely high energies show minimal time-of-flight differences (e.g., Fermi telescope data).
    \item Quantum gravity effects could cause dispersion at high energies.
\end{itemize}

\begin{quote}
    \enquote{The constancy of the speed of light is not an immutable law but a parameter that may vary under extreme conditions, offering solutions to cosmological puzzles.} \cite{Magueijo2003}
\end{quote}

\section{The Evolution of the Wave Concept in Physics}
The understanding of waves in physics has radically changed over time. While classical waves like sound or water waves could be described as disturbances of a material medium,
electromagnetic waves and quantum phenomena led to fundamental upheavals. In 1865, Maxwell showed that light propagates as an electromagnetic wave
even without an ether—raising the question of how energy is transported without a carrier medium. \gls{srt} established the speed of light
as an absolute limit, while \gls{art} describes it as locally variable—an apparent contradiction that alternative theories like Weber electrodynamics
attempt to resolve.

Quantum physics further revolutionized the wave concept: De Broglie linked particle and wave properties, and \gls{qed} describes photons as fields with superluminal
path integral components. Yet this mathematical elegance creates physical interpretation problems—such as the role of the observer in wave function collapse or the non-local
nature of quantum entanglement. Gravitational waves in \gls{art} also remain enigmatic: If spacetime is considered an oscillating medium, where does the energy for its deformation come from?

These contradictions suggest that established theories may only be approximations of a deeper truth. Physics faces fundamental questions: Is the speed of light
truly constant? How can quantum physics and relativity be unified? Is there an objective reality beyond the observer? The search for answers could trigger a new scientific
revolution—one that fundamentally changes our understanding of waves and the fundamental nature of reality.

\section{Wave Phenomena: The Duality of Instantaneous Wholeness and Local Propagation}
Waves possess a unique dual nature that permeates all of physics. On one hand, they exhibit local propagation phenomena; on the other, they display instantaneous global
properties that defy classical causality. This duality becomes particularly evident when examining fundamental interactions.

Newtonian mechanics postulates instantaneous action at a distance with \enquote{actio = reactio}—a force acts immediately between two bodies. Mathematically expressed:
\begin{equation}
    \vec{F}{12} = -\vec{F}{21}
\end{equation}
This equation describes an instantaneous interaction without time delay. The same applies to Coulomb's law:
\begin{equation}
    \vec{F} = \frac{1}{4\pi\epsilon_0}\frac{q_1q_2}{r^2}\hat{\vec{r}}
\end{equation}
These action-at-a-distance theories work remarkably well within their domain, as demonstrated by the successful description of planetary motion. Yet they remain incomplete,
as they cannot explain the energy and momentum transfer between interacting bodies.

Interference phenomena reveal another profound property of waves. Consider the double-slit experiment: The probability density at a point x on the
screen results from the superposition of partial waves:
\begin{equation}
    |\Psi(x)|^2 = |\psi_1(x) + \psi_2(x)|^2
\end{equation}
This pattern serves an energetic purpose—it minimizes the total energy of the system. The wave \enquote{knows} instantaneously how to distribute itself to achieve this minimum,
without any local interaction explaining it.

Weber electrodynamics offers an interesting bridge here. It extends Coulomb's law with velocity- and acceleration-dependent terms:
\begin{equation}
    \label{eq:weber_em_skalar}
    \vec{F} = \frac{q_1q_2}{4\pi\epsilon_0r^2}\left[1 - \frac{\dot{r}^2}{c^2} + \frac{2r\ddot{r}}{c^2}\right]\hat{\vec{r}}
\end{equation}
This equation describes:
\begin{enumerate}
    \item The static Coulomb term (instantaneous action at a distance)
    \item A velocity-dependent term (magnetic effects)
    \item An acceleration-dependent term (radiation resistance)
\end{enumerate}
The instantaneous component remains but is supplemented by retarded effects. This shows how a theory can unify both instantaneous interactions and propagation phenomena.
This form shall be referred to as the \textbf{\enquote{scalar form}}.

Energetic considerations reveal the deeper meaning of this duality. A wave in equilibrium always minimizes the total energy:

\begin{equation}
    \delta \int \left[\frac{\hbar^2}{2m}|\nabla\Psi|^2 + V|\Psi|^2\right] d^3x = 0    
\end{equation}

This condition is fulfilled globally and instantaneously, while local disturbances propagate at finite speed. Weber electrodynamics demonstrates that similar principles are at work in
classical physics—the instantaneous component ensures energy conservation, while the retarded terms describe energy transport.

The implications of this perspective are far-reaching:
\begin{enumerate}
    \item Instantaneous effects are not necessarily unphysical but can represent energetic constraints
    \item Propagation speed describes only energy transport, not global structure
    \item Action-at-a-distance theories contain a kernel of truth often neglected in modern theories
\end{enumerate}
These insights pave the way for a new understanding of wave phenomena that could resolve the apparent contradictions between instantaneous wholeness and local causality.

\section{The Expanded Concept of Causality: A Synthetic View of Instantaneous Wholeness and Local Dynamics}
The conventional notion of causality as a linear cause-effect chain with strict locality and finite propagation speed proves too narrow upon closer examination
of wave phenomena. Physics faces the paradox that, on one hand, relativity theory postulates a maximum signal speed, while on the other,
quantum phenomena like entanglement and the \gls{epr} \cite{EPR1935} suggest that certain correlations can exist instantaneously across arbitrary distances. This tension demands
a new, more comprehensive concept of causality.

The key to understanding lies in recognizing two complementary but equally valid levels of causality that jointly determine the dynamics of physical systems.
On one side is local causality, as described by Maxwell's equations or relativistic field theory. This level governs energy transport
and the propagation of disturbances through space at finite speed. The familiar light-cone structure of spacetime with its strict separation of timelike, lightlike, and
spacelike intervals belongs to this domain.

Parallel to this exists a systemic level of causality responsible for the global organization of the wave field. This manifests in phenomena like
spontaneous symmetry breaking, the Aharonov-Bohm effect (Section \ref{sec:aharonov-bohm}), or the aforementioned entangled quantum states. While local causality is described by differential equations with
boundary conditions, systemic causality follows a variational principle that optimizes the entire system simultaneously.
The quantum potential \cite{bohm1952}
\begin{equation}
    \label{eq:bohm_potenzial}
    Q(\vec{r},t) = -\frac{\hbar^2}{2m} \frac{\nabla^2 \sqrt{\rho(\vec{r},t)}}{\sqrt{\rho(\vec{r},t)}}
\end{equation}
is a prime example—it acts not through local interactions but through the instantaneous adaptation of the entire
wave function to global boundary conditions.

Weber electrodynamics with its characteristic force equation (Eq. \ref{eq:weber_em_skalar}) exemplifies how both levels of causality can be unified in a consistent theory.
The first term represents the systemic component—an instantaneous Coulomb interaction that ensures the basic structure of action at a distance. The additional velocity- and
acceleration terms, however, describe the local dynamics of energy transport, including retarded effects and radiation phenomena.

This dual structure of causality resolves numerous conceptual problems in modern physics. For instance, it explains why the interference pattern in the double-slit experiment
emerges even when particles pass through the experiment one by one—the systemic causality \enquote{knows} the overall setup and organizes the probability distribution
accordingly. At the same time, energy transport remains limited by local causality, preserving relativistic principles.

The implications of this expanded understanding of causality are profound. It enables a physical interpretation of quantum mechanics that avoids the problematic
\enquote{collapse} of the wave function. Measurement processes no longer appear as mysterious interventions but as special cases of systemic self-organization.
The apparent observer-dependence of quantum phenomena reveals itself as a special case of general systemic causality, which becomes particularly conspicuous when
a subsystem (the \enquote{observer}) correlates with another (the \enquote{observed system}).

Ultimately, this approach leads to a natural synthesis of classical and quantum physics, of relativity theory and wave mechanics. Instead of accepting the paradoxical aspects of quantum theory
as fundamental principles, it explains them as consequences of the interplay between two complementary levels of causality—a systemic wholeness that acts instantaneously,
and a local dynamics that mediates energy and momentum transfer at finite speed.