\chapter{Weber Gravity}
This chapter deals exclusively with \gls{wg} and aims to demonstrate the capabilities of \gls{wg} compared to \gls{art}.
\section{Derivation of Weber Gravity}
The idea of a gravitational analogue to Weber electrodynamics dates back to the French astronomer François-Félix Tisserand (1889). Inspired by the structural
similarity between Newton's law of gravitation and Coulomb's law,
\begin{equation}
    \vec{F}_{\text{Newton}} = -G \frac{m_1 m_2}{r^2} \hat{\vec{r}}, \vec{F}_{\text{Coulomb}} = \frac{1}{4 \pi \epsilon_0} \frac{q_1 q_2}{r^2} \hat{\vec{r}}
\end{equation}
Tisserand attempted to transfer the Weber force (originally formulated for electrodynamic interactions) to gravity. Weber gravity thus emerges as:
\begin{equation}
    \vec{F}_{\text{WG-Tisserand}} = -G \frac{m_1 m_2}{r^2} \left[ 1 - \frac{\dot{r}^2}{c^2} + \frac{2 r \ddot{r}}{c^2} \right] \hat{\vec{r}}.
\end{equation}
This equation adds velocity- and acceleration-dependent corrections to Newton's law, analogous to Weber electrodynamics.
\subsection{Test at Mercury's Perihelion – and Why the Theory Failed}
Tisserand's motivation was explaining Mercury's anomalous perihelion advance, already known in the 19th century (~43 arcseconds per century).
While Weber gravity predicted a perihelion shift:
\begin{enumerate}
    \item \textbf{Quantitative Failure:}\\The calculated deviation did not match observations.
    \item \textbf{\gls{art} as Superior Solution:}\\Only Einstein's \gls{art} provided the exact correction of 43"/century – a 100-year triumph of
    spacetime curvature over pure action-at-a-distance models.
\end{enumerate}

\subsection{Modified Tisserand Approach}
The \gls{wg} (\textbf{modified Tisserand approach}) offers an alternative description of gravitational phenomena by extending Newton's law of gravitation with
velocity- and acceleration-dependent terms. The central WG equation is:

\begin{equation}
    \label{eq:wg-beta}
    \vec{F}_{\text{WG}} = -\frac{GMm}{r^2} \left(1 - \frac{\dot{r}^2}{c^2} + \beta \frac{r\ddot{r}}{c^2}\right) \hat{\vec{r}},
\end{equation}

where $\dot{r}$ is the radial relative velocity and $\ddot{r}$ the radial acceleration. This modification leads to orbital equations that can be
expanded to first and second order to provide precise predictions for planetary orbits and other gravitational effects. The \textbf{$\beta$-parameter} is
a central quantity in Weber gravity that determines the ratio between acceleration- and velocity-dependent terms in the modified
gravitational force; $\beta$ is a dimensionless factor whose value varies depending on physical context and has \textbf{decisive impacts on the theory's predictions}.

For simplification of the equations, the specific angular momentum $h$ is defined:
\begin{equation}
h = \sqrt{GMa(1 - e^2)}.
\end{equation}

\subsection{Physical Meaning of the beta Parameter}
The parameter $\beta$ quantifies the influence of radial acceleration $\ddot{r}$ relative to the velocity correction $\dot{r}^2$.
\begin{itemize}
    \item For $\beta=0$, the acceleration term vanishes, and the force reduces to a purely velocity-dependent modification of Newtonian gravity.
    \item For $\beta>0$, the acceleration term dominates in dynamic processes like light deflection or perihelion advance.
    \item The value $\beta=0.5$ reproduces Mercury's perihelion advance exactly, while $\beta=1$ is needed for massless particles (photons) to explain frequency-dependent effects.
\end{itemize}

\subsection{Applications of the beta Parameter}
\textbf{1. Light Deflection in Gravitational Field}

For photons ($m=0$), $\beta=1$ is set, leading to a frequency-dependent correction of the deflection. The orbital equation for light is:
\begin{equation}
    \frac{d^2u}{d\phi^2} + u = \frac{GM}{c^2} \left(3u^2 + \frac{E^2}{c^2 h^2} u^3\right).
\end{equation}
Where $u=1/r$ and $E=h_\text{P}\nu$ is the photon energy. The solution for small deflections $\Delta\phi$ shows an additional term proportional to wavelength $\lambda$:
\begin{equation}
\Delta \phi = \frac{4GM}{c^2 b} \left(1 + \frac{3\pi}{16} \frac{\lambda^2}{\lambda_0^2}\right).
\end{equation}

Here $\lambda_0=hc/E$ is a characteristic length scale. This effect could be verified with high-precision interferometers (e.g., LISA).

\textbf{2. Shapiro Time Delay}
The travel time $\Delta t$ of a signal in a gravitational field is modified by $\beta$. The integrated delay along the path is:
\begin{equation}
\Delta t = \frac{2GM}{c^3} \ln\left(\frac{4r_e r_p}{b^2}\right) + \frac{3\pi G^2 M^2}{4c^5 b^2} \left(\frac{v_0^2}{c^2}\right).
\end{equation}

The second term (proportional to $\beta=1$) leads to a wavelength-dependent correction:
\begin{equation}
    \Delta t_\text{WG} \propto \lambda^{-2},
\end{equation}
which should be measurable in pulsar timing experiments (e.g., with the Square Kilometre Array). Compared to \gls{art} ($\beta=0$), the deviation is small ($\approx 10^{-6}$),
but in principle detectable.

\[
\begin{array}{|l|c|l|}
\hline
\text{Application} & \beta & \text{Consequence} \\
\hline
\text{Electrodynamics} & 2 & \text{Magnetic interactions} \\
\text{Gravity (masses)} & 0.5 & \text{Mercury's perihelion advance} \\
\text{Photons} & 1 & \text{Frequency-dependent effects} \\
\hline
\end{array}
\]

The $\beta$-parameter thus acts as a \enquote{key} for adapting Weber gravity to different physical scenarios – from classical planetary orbits
to quantum phenomena. Its role highlights the theory's flexibility but also the need for precise experimental tests to validate the correct
values.

\section{Expansion (Hubble Constant) and Redshift in Weber Gravity}
The \gls{wg} offers a radically alternative interpretation of cosmological redshift and the Hubble constant compared to \gls{art}. While
\gls{art} interprets redshift as a consequence of universal expansion and the Hubble constant $H_0$ as a measure of this expansion, \gls{wg}
explains the same observations through cumulative gravitational interactions in a static universe.

\subsection{Redshift in Weber Gravity}
In \gls{wg}, redshift $z$ consists of two components: a static term corresponding to classical gravitational redshift and a
dynamic term depending on the relative velocity $v_r$ between source and observer. The total redshift is:

\begin{equation}
    z \approx \frac{GM}{c^2} \left( \frac{1}{r_{\text{em}}} - \frac{1}{r_{\text{obs}}} \right) + \frac{3}{2} \frac{v_r^2}{c^2}
\end{equation}

The first term is identical to \gls{art}'s prediction for gravitational redshift (e.g., in the Pound-Rebka experiment). The second term, however, is a new contribution
capturing WG's dynamic effects. For cosmological distances where $v_r \approx H_0 d$ (with $H_0$ as Hubble constant and $d$ as distance), the dynamic term dominates:

\begin{equation}
    z \approx \frac{3}{2} \frac{H_0^2 d^2}{c^2}
\end{equation}

This leads to an alternative Hubble law that depends quadratically on distance, unlike the linear relation $z \approx H_0 d / c$ of \gls{art}.

\subsection{Hubble Constant in Weber Gravity}
The \gls{wg} interprets the Hubble constant not as an expansion rate but as an effect of cumulative gravitational interactions over large distances. Rearranging
the dynamic redshift yields an effective Hubble constant:

\begin{equation}
    H_0^{\text{WG}} = \sqrt{\frac{2}{3}} \frac{c}{d} \sqrt{z} \approx 67.8 \, \text{km/s/Mpc}
\end{equation}

This value remarkably matches the Planck mission's measured value ($H_0 \approx 67.4 km/s/Mpc$), making WG a plausible alternative to \gls{art}.

\subsection{Consequences for Cosmology}
\begin{enumerate}
    \item \textbf{No Universal Expansion:} \gls{wg} requires no space expansion to explain redshift. Instead, $z$ arises from the velocity dependence of gravitational interaction.
    \item \textbf{No Dark Energy:} The universe's accelerated expansion disappears since there is no expansion. Observed redshift is explained by the dynamic term.
    \item \textbf{Static Universe:} \gls{wg} postulates an infinite, static universe without a Big Bang. Cosmological redshift is a local effect caused by galaxies' relative motion.
\end{enumerate}

\subsection{Experimental Discrimination}
\gls{wg} predicts that redshift in galaxy clusters shows a slight deviation from the linear Hubble law:

\begin{equation}
    \frac{z_{\text{WG}}}{z_{\text{\gls{art}}}} = 1 + \frac{3}{2} \left( \frac{v_r}{c} \right)^2 \left( \frac{GM}{c^2 r} \right)^{-1}
\end{equation}

For $v_r \approx 1000 km/s$ and $r = 1 Mpc$, the deviation is about $10^{-4}$, potentially measurable with future telescopes like the Extremely Large Telescope (ELT).

Thus, \gls{wg} offers a consistent alternative to standard cosmology, requiring no dark energy, Big Bang, or space expansion while explaining observed redshift.
Experimental tests of frequency-dependent effects could validate or refute the theory in the future.

\subsection{Consequences for the Universe's Size}
\gls{wg} has fundamental implications for our understanding of cosmic scales:

\subsection{Static Universe}
Unlike the standard $\Lambda$CDM model, WG postulates a \textbf{non-expanding universe} with these properties:

\begin{itemize}
\item No temporal change in overall size
\item Possible infinity of space
\item No Big Bang as starting point
\end{itemize}

\subsection{Cosmological Implications}
\begin{itemize}
\item No need for inflation
\item Natural explanation of CMB homogeneity
\item Alternative interpretation of observed redshift
\item Elimination of dark energy necessity
\end{itemize}

WG thus provides a consistent alternative to the standard model, requiring no universe expansion and treating its size as a fundamental, time-independent parameter.


\subsection{Orbital Equation 1st Order}
The 1st-order orbital equation $r(\phi)$ results from solving the equation of motion while neglecting higher-order terms in $c^{-2}$. It reads:
\begin{equation}
    \label{eq:weber_r_1_ordnung}
    r(\phi) = \frac{a(1 - e^2)}{1 + e \cos(\kappa \phi)},    
\end{equation}
\begin{equation}
    \kappa = \sqrt{1 - \frac{6GM}{c^2 a(1 - e^2)}}.
\end{equation}

Where $\kappa$ represents a correction to Newtonian mechanics. Here, $a$ is the semi-major axis and $e$ the orbital eccentricity.
This equation describes a planet's orbit including relativistic effects leading to perihelion advance.
The perihelion advance per orbit is:
\begin{equation}
    \Delta \phi = 2\pi \left(\frac{1}{\kappa} - 1\right),
\end{equation}

yielding Mercury's observed value of 42.98'' per century.

\textbf{Angular and Orbital Velocity:}
\begin{equation}
    \omega(\phi) = \frac{h}{a^2(1 - e^2)^2} \left[1 + e \cos(\kappa \phi)\right]^2    
\end{equation}

\begin{equation}
    v(\phi) = \frac{h \left(1 + e \cos(\kappa \phi)\right)}{a(1 - e^2)}
\end{equation}

\subsection{Orbital Equation 2nd Order}
In 2nd order, additional corrections are considered, resulting from expanding $\kappa$ and introducing a quadratic term in $\phi$.
The orbital equation then takes the form:
\begin{equation}
    \label{eq:weber_r_2_ordnung}
    r(\phi) = \frac{a(1 - e^2)}{1 + e \cos(\kappa \phi + \alpha \phi^2)},
\end{equation}

\begin{equation}
\alpha = \frac{3G^2 M^2 e}{8c^4 h^4},
\end{equation}

\begin{equation}
\kappa = \sqrt{1 - \frac{6GM}{c^2 a(1 - e^2)} + \frac{27G^2 M^2}{2c^4 a^2 (1 - e^2)^2}}.
\end{equation}

In equation (Eq. \ref{eq:weber_r_2_ordnung}), the term $\alpha \phi^2$ appears, which would lead to non-closed planetary orbits (so-called \enquote{rosette orbits}).
This raises physical questions, as stable, closed orbits are observed in our solar system. Interestingly, \gls{wg}'s 1st-order equations already yield results
matching \gls{art}'s accuracy. The higher-order deviations, however, suggest possible incompleteness of the theory. Nonetheless, it remains clear that WG provides
highly precise predictions in first approximation, while higher-order deviations are minimal.

Thus, \gls{wg} proves to be a powerful tool for describing gravitational phenomena. Whether its deviations from \gls{art} represent an improvement or deterioration
is not yet conclusively settled. What is undeniable, however, is that \gls{wg} is mathematically simpler and conceptually more accessible than the complex \gls{art}.

Moreover, \gls{wg} can also explain phenomena like frequency-dependent light deflection and gravitational time delay.

Particularly noteworthy is its prediction of
wavelength-dependent light deflection, which clearly differs from \gls{art}'s predictions and is \textbf{in principle experimentally verifiable}. This underscores \gls{wg}'s potential
as an alternative gravity theory that is both precise and intuitively accessible.