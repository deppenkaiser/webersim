\chapter{Mathematische Grundlagen der diskreten Informations-Weber-Theorie}
\label{app:mathematische-grundlagen}

\section{Einleitung}
Dieser Anhang stellt die mathematischen Werkzeuge bereit, auf denen die diskrete Formulierung der \gls{iwt} basiert. Im Gegensatz zu den kontinuierlichen
Approximationen im Haupttext werden hier ausschließlich die fundamentalen diskreten Strukturen entwickelt. Der Fokus liegt auf:
\begin{enumerate}
    \item Diskrete Variationsrechnung für Informationsnetzwerke
    \item Differenzengleichungen und rekursive Update-Regeln
    \item Diskrete Noether-Theoreme und Erhaltungssätze
    \item Definition und Berechnung der diskreten Informationsmetrik
    \item Fraktale Analysis für diskrete Netzwerke
\end{enumerate}

\section{Diskrete Variationsrechnung}
\label{sec:diskrete-variationsrechnung}

\subsection{Diskretes Funktional}
Ein diskretes Informationssystem wird beschrieben durch Knoten $k = 1, \ldots, N$ mit Informationswerten $I_k^{(n)}$ zum Zeitschritt $n$. Das fundamentale diskrete
Funktional lautet:
\[
S_d\left[I_k^{(n)}\right] = \sum_{n=0}^{M-1} \sum_{k=1}^N \mathcal{F}_d\left(I_k^{(n)}, \Delta_t I_k^{(n)}, \Delta_s I_k^{(n)}\right) \cdot T \cdot V_k
\]
mit:
\begin{itemize}
    \item $\Delta_t I_k^{(n)} = I_k^{(n)} - I_k^{(n-1)}$: Zeitliche Differenz
    \item $\Delta_s I_k^{(n)} = \sum_{l \in \mathcal{N}(k)} w_{kl}(I_l^{(n)} - I_k^{(n)})$: Räumliche Differenz
    \item $T$: Fundamentaler Zeitschritt
    \item $V_k$: Diskretes Volumenelement am Knoten $k$
\end{itemize}

\subsection{Variation diskreter Funktionale}
Eine Variation des diskreten Feldes sei:
\[
I_k^{(n)} \to I_k^{(n)} + \epsilon \eta_k^{(n)}
\]
mit beliebigen Testwerten $\eta_k^{(n)}$, die am Rand verschwinden.

Die Variation des Funktionals ist:
\[
\delta S_d = \frac{d}{d\epsilon} S_d\left[I_k^{(n)} + \epsilon \eta_k^{(n)}\right]_{\epsilon=0}
\]

\subsection{Diskrete Euler-Lagrange-Gleichungen}
Durch Ableitung und Umordnung erhält man:
\[
\frac{\partial \mathcal{F}_d}{\partial I_k^{(n)}} 
- \Delta_t^+ \left( \frac{\partial \mathcal{F}_d}{\partial (\Delta_t I_k^{(n)})} \right)
- \Delta_s \left( \frac{\partial \mathcal{F}_d}{\partial (\Delta_s I_k^{(n)})} \right) = 0
\]
mit den diskreten Operatoren:
\begin{align*}
\Delta_t^+ f^{(n)} &= \frac{f^{(n+1)} - f^{(n)}}{T} \quad &\text{(Vorwärtsdifferenz)} \\
\Delta_s f_k &= \sum_{l \in \mathcal{N}(k)} w_{kl}(f_l - f_k) \quad &\text{(Diskreter Gradient)}
\end{align*}

\subsection{Beispiel: Diskrete Wellengleichung}
Für $\mathcal{F}_d = \alpha (\Delta_t I_k^{(n)})^2 + \beta (\Delta_s I_k^{(n)})^2$:
\[
\alpha \frac{I_k^{(n+1)} - 2I_k^{(n)} + I_k^{(n-1)}}{T^2} 
+ \beta \sum_{l \in \mathcal{N}(k)} w_{kl} (I_l^{(n)} - I_k^{(n)}) = 0
\]

\section{Diskrete Noether-Theoreme}
\label{sec:diskrete-noether}

\subsection{Symmetrien in diskreten Systemen}
Eine diskrete Transformation:
\[
I_k^{(n)} \to I_k^{(n)} + \epsilon \cdot \xi_k^{(n)}
\]
ist eine Symmetrie, wenn:
\[
\mathcal{F}_d\left(I_k^{(n)} + \epsilon \xi_k^{(n)}, \Delta_t(I+\epsilon\xi), \Delta_s(I+\epsilon\xi)\right)
= \mathcal{F}_d\left(I_k^{(n)}, \Delta_t I, \Delta_s I\right) + \Delta_t^+ K^{(n)} + \Delta_s J_k^{(n)}
\]

\subsection{Erhaltungsgrößen}
Aus jeder solchen Symmetrie folgt eine diskrete Erhaltungsgröße:
\[
Q^{(n+1)} - Q^{(n)} = -\sum_k \Delta_s J_k^{(n)}
\]
mit der erhaltenen diskreten Ladung:
\[
Q^{(n)} = \sum_k \left[ \xi_k^{(n)} \frac{\partial \mathcal{F}_d}{\partial (\Delta_t I_k^{(n)})} - K^{(n)} \right]
\]

\subsection{Wichtige Symmetrien und Erhaltungssätze}
\begin{table}[ht]
\centering
\begin{tabular}{p{0.25\textwidth}p{0.35\textwidth}p{0.3\textwidth}}
\hline
\textbf{Symmetrie} & \textbf{Transformation} & \textbf{Erhaltene Größe} \\
\hline
Zeittranslation & $I_k^{(n)} \to I_k^{(n+1)}$ & Diskrete Energie $E^{(n)}$ \\
\hline
Raumtranslation & $I_k^{(n)} \to I_{k+\delta}^{(n)}$ & Diskreter Impuls $\vec{P}^{(n)}$ \\
\hline
Rotation & $I_k^{(n)} \to I_{R(k)}^{(n)}$ & Diskreter Drehimpuls $\vec{L}^{(n)}$ \\
\hline
Informations-Shift & $I_k^{(n)} \to I_k^{(n)} + c$ & Gesamtinformation $\sum_k I_k^{(n)}$ \\
\hline
\end{tabular}
\caption{Diskrete Symmetrien und zugehörige Erhaltungssätze}
\end{table}

\section{Diskrete Informationsmetrik}
\label{sec:diskrete-informationsmetrik}

\subsection{Definition aus dem Funktional}
Die diskrete Informationsmetrik zwischen Knoten $k$ und $l$ ist definiert als:
\[
g_{kl}^{(n)} = \frac{\partial^2 \mathcal{F}_d}{\partial (\Delta_s I_k^{(n)}) \partial (\Delta_s I_l^{(n)})}
\]

\subsection{Alternative Definition über Kopplungsmatrix}
Für viele Anwendungen ist die direkte Definition über die Kopplungsmatrix $K_{kl}^{(n)}$ nützlicher:
\[
g_{kl}^{(n)} = \frac{K_{kl}^{(n)}}{\sqrt{K_{kk}^{(n)} K_{ll}^{(n)}}}
\]

\subsection{Eigenschaften}
\begin{itemize}
    \item \textbf{Symmetrie}: $g_{kl}^{(n)} = g_{lk}^{(n)}$
    \item \textbf{Positive Definitheit}: $\sum_{k,l} v_k g_{kl}^{(n)} v_l \geq 0$ für alle $v_k$
    \item \textbf{Normierung}: $g_{kk}^{(n)} = 1$
    \item \textbf{Lokalität}: $g_{kl}^{(n)} \to 0$ für $|k-l| \to \infty$
\end{itemize}

\section{Fraktale Analysis diskreter Netzwerke}
\label{sec:fraktale-analysis}

\subsection{Fraktale Dimension eines diskreten Netzwerks}
Gegeben sei ein diskretes Netzwerk mit Adjazenzmatrix $A_{kl}$. Die fraktale Dimension $D$ wird bestimmt durch das Skalierungsverhalten der Koordinationszahl:
\[
N(R) \propto R^D \quad \text{für } R \to \infty
\]
wobei $N(R)$ die Anzahl der Knoten innerhalb einer Distanz $R$ von einem Referenzknoten ist.

\subsection{Praktische Berechnung}
\begin{enumerate}
    \item Wähle einen Referenzknoten $k_0$
    \item Berechne die kürzesten Pfade zu allen anderen Knoten: $d(k_0, k)$
    \item Zähle: $N(R) = \#\{k : d(k_0, k) < R\}$
    \item Bestimme $D$ aus der Steigung: $D = \frac{d \ln N(R)}{d \ln R}$
\end{enumerate}

\subsection{Beispiel: Fraktales Netz}
Für das in der \gls{iwt} verwendete fraktale Netzwerk gilt:
\[
D = \frac{\ln 20}{\ln(2+\phi)} \approx 2.71
\]
Dieses Netzwerk zeigt selbstähnliche Struktur auf allen Skalen.

\section{Diskrete Differenzengleichungen höherer Ordnung}
\label{sec:diskrete-differenzengleichungen}

\subsection{Rekursive Update-Regeln}
Viele Gleichungen der \gls{iwt} haben die Form:
\[
I_k^{(n+1)} = F\left(I_k^{(n)}, I_k^{(n-1)}, \{I_l^{(n)}\}\right)
\]

\subsection{Stabilitätsanalyse}
Für eine lineare Rekursion:
\[
I_k^{(n+1)} = a I_k^{(n)} + b I_k^{(n-1)}
\]
erhält man die charakteristische Gleichung:
\[
\lambda^2 - a\lambda - b = 0
\]
Stabilitätsbedingung: $|\lambda| \leq 1$ für alle Lösungen.

\subsection{Numerische Stabilität der diskreten Weber-Kraft}
Die diskrete Weber-Kraft:
\[
\vec{F}^{(n)} = \frac{q_1 q_2}{4\pi\varepsilon_0 (r^{(n)})^2} 
\left[1 - \frac{1}{c^2}\left(\frac{r^{(n)}-r^{(n-1)}}{T}\right)^2 
+ \frac{2r^{(n)}}{c^2} \cdot \frac{r^{(n+1)}-2r^{(n)}+r^{(n-1)}}{T^2}\right] \hat{r}^{(n)}
\]
ist stabil für:
\[
T < T_{\text{max}} = \frac{r_{\text{min}}}{c}
\]

\section{Diskrete Fourier-Analyse}
\label{sec:diskrete-fourier}

\subsection{Diskrete Fourier-Transformation}
Für ein diskretes Feld $I_k$ auf $N$ regelmäßig angeordneten Knoten:
\[
\tilde{I}_q = \frac{1}{\sqrt{N}} \sum_{k=0}^{N-1} I_k e^{-2\pi i q k / N}
\]

\subsection{Dispensionsrelationen}
Aus diskreten Differenzengleichungen erhält man charakteristische Dispersionsrelationen:
\[
\omega(q) = \frac{2}{T} \arcsin\left(\frac{cT}{2a} \sin(\pi q a)\right)
\]

\section{Matrixdarstellungen diskreter Operatoren}
\label{sec:matrixdarstellungen}

\subsection{Laplace-Matrix}
Der diskrete Laplace-Operator wird durch die Laplace-Matrix $L$ dargestellt:
\[
(L f)_k = \sum_{l} L_{kl} f_l
\]
mit
\[
L_{kl} = 
\begin{cases}
\sum_{m \neq k} w_{km} & \text{für } k = l \\
-w_{kl} & \text{für } k \neq l
\end{cases}
\]

\subsection{Eigenwertanalyse}
Die Eigenwerte $\lambda_i$ und Eigenvektoren $\vec{v}_i$ von $L$ charakterisieren die kollektiven Moden des Netzwerks.

\section{Zusammenfassung der mathematischen Struktur}
Die diskrete \gls{iwt} basiert auf folgenden mathematischen Grundlagen:

\begin{table}[ht]
\centering
\begin{tabular}{p{0.3\textwidth}p{0.6\textwidth}}
\hline
\textbf{Konzept} & \textbf{Mathematische Beschreibung} \\
\hline
Zustandsraum & Diskrete Felder $I_k^{(n)}$ auf Netzwerkknoten \\
\hline
Dynamik & Diskretes Funktional $S_d[I_k^{(n)}]$ mit Variationsprinzip \\
\hline
Bewegungsgleichungen & Diskrete Euler-Lagrange-Gleichungen \\
\hline
Symmetrien & Diskrete Noether-Theoreme mit Erhaltungssätzen \\
\hline
Geometrie & Diskrete Metrik $g_{kl}^{(n)}$ aus Kopplungsstruktur \\
\hline
Skalierung & Fraktale Dimension $D$ charakterisiert Netzwerk \\
\hline
Stabilität & Eigenwertanalyse rekursiver Update-Regeln \\
\hline
\end{tabular}
\caption{Übersicht der mathematischen Struktur}
\end{table}

\subsection{Hinweise zur praktischen Anwendung}
\begin{enumerate}
    \item Alle kontinuierlichen Gleichungen im Haupttext sind Grenzfälle dieser diskreten Strukturen
    \item Numerische Simulationen implementieren direkt diese diskreten Gleichungen
    \item Die mathematische Konsistenz wird durch die diskrete Formulierung gewährleistet
    \item Fraktale Eigenschaften ergeben sich natürlich aus der Netzwerkstruktur
\end{enumerate}
Dieser Anhang bildet damit die mathematische Grundlage für das gesamte Werk und ermöglicht die rigorose Herleitung aller im Haupttext verwendeten Gleichungen.
