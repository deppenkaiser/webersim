\chapter{Numerische Simulation der Informationsdynamik}
\label{chap:numerische-simulation}

\section{Einleitung}
Dieses Kapitel beschreibt die numerische Umsetzung der Informations-Weber-Theorie. Da die Theorie fundamental diskret formuliert ist, lässt sie sich direkt als
algorithmische Simulation implementieren. Die numerischen Methoden dienen drei Hauptzwecken:
\begin{enumerate}
    \item \textbf{Validierung}: Überprüfung der internen Konsistenz der Theorie
    \item \textbf{Emergenznachweis}: Demonstration, wie kontinuierliche Physik aus diskreter Dynamik entsteht
    \item \textbf{Vorhersagegenerierung}: Berechnung testbarer experimenteller Konsequenzen
\end{enumerate}
Alle Simulationen basieren auf den in Teil~I entwickelten fundamental diskreten Gleichungen.

\section{Das diskrete Simulationsnetzwerk}
\subsection{Grundstruktur}
Die Simulation arbeitet mit einem diskreten Netzwerk $\mathcal{N}$, bestehend aus:
\[
\mathcal{N} = \left\{ I_k^{(n)},\, K_{kl}^{(n)},\, g_{kl}^{(n)} \mid k,l=1,\ldots,N \right\}
\]
\begin{itemize}
    \item $N$: Anzahl der Informationsknoten (typisch $10^3$–$10^6$)
    \item $I_k^{(n)}$: Informationswert am Knoten $k$ zum Zeitschritt $n$
    \item $K_{kl}^{(n)}$: Kopplungsstärke zwischen Knoten $k$ und $l$
    \item $g_{kl}^{(n)}$: Emergente Metrik zwischen $k$ und $l$
\end{itemize}

\subsection{Initialisierungsprotokolle}
Je nach Simulationsziel werden unterschiedliche Initialisierungen verwendet:

\begin{table}[ht]
\centering
\begin{tabular}{p{0.25\textwidth}p{0.3\textwidth}p{0.35\textwidth}}
\hline
\textbf{Simulationstyp} & \textbf{Initialisierung $I_k^{(0)}$} & \textbf{Kopplungsmatrix $K_{kl}^{(0)}$} \\
\hline
Leeres Vakuum & Gleichverteilung $I_k^{(0)} = I_0$ & Fraktales Muster mit $D \approx 2.71$ \\
Teilchensystem & Lokalisierte Peaks an Teilchenpositionen & Nachbarschaftskopplung \\
Kosmologie & Große Skalenfluktuationen & Skaleninvariantes fraktales Netz \\
Quantensystem & Kohärente Phasenverteilung & Globale Langreichweitkopplung \\
\hline
\end{tabular}
\caption{Initialisierungsprotokolle für verschiedene Simulationstypen}
\end{table}

\section{Implementierung der diskreten Dynamik}
\subsection{Hauptalgorithmus}
Der Kern der Simulation ist die rekursive Update-Schleife:

\noindent\textbf{Algorithmus 1: Hauptsimulationsschleife}
\begin{enumerate}
    \item \textbf{Eingabe}: Knotenzahl $N$, Zeitschritte $N_{\text{steps}}$, Zeitschrittweite $T$
    \item \textbf{Initialisierung}: Setze $I_k^{(0)}, K_{kl}^{(0)}$ für alle $k,l=1,\ldots,N$
    \item \textbf{Für} $n = 0, 1, 2, \ldots, N_{\text{steps}}-1$:
    \begin{enumerate}
        \item \textbf{Informationsupdate}: 
        \[
        I_k^{(n+1)} = I_k^{(n)} + T \cdot \Phi_k\left(I_k^{(n)}, \{I_l^{(n)}\}, I_k^{(n-1)}\right)
        \]
        \item \textbf{Kopplungsupdate}: 
        \[
        K_{kl}^{(n+1)} = K_{kl}^{(n)} + \Delta K_{kl}\left(I^{(n+1)}, I^{(n)}\right)
        \]
        \item \textbf{Metrikberechnung}: 
        \[
        g_{kl}^{(n+1)} = \frac{K_{kl}^{(n+1)}}{\sqrt{K_{kk}^{(n+1)} K_{ll}^{(n+1)}}}
        \]
        \item \textbf{Analyse}: Berechne observablen Größen (Energie, Impuls, Korrelationen)
    \end{enumerate}
    \item \textbf{Ausgabe}: Zeitreihen aller relevanten Größen
\end{enumerate}

\subsection{Diskrete Weber-Dynamik}
Die lokale Dynamik wird durch die diskrete Weber-Kraft implementiert:
\[
\vec{F}_{ij}^{(n)} = \frac{q_i q_j}{4\pi\varepsilon_0 (r_{ij}^{(n)})^2} 
\left[1 - \frac{1}{c^2}\left(\frac{\Delta r_{ij}^{(n)}}{T}\right)^2 
+ \frac{2r_{ij}^{(n)}}{c^2} \cdot \frac{\Delta^2 r_{ij}^{(n)}}{T^2}\right] \hat{\vec{r}}_{ij}^{(n)}
\]
mit $\Delta r_{ij}^{(n)} = r_{ij}^{(n)} - r_{ij}^{(n-1)}$ und $\Delta^2 r_{ij}^{(n)} = r_{ij}^{(n+1)} - 2r_{ij}^{(n)} + r_{ij}^{(n-1)}$.

\noindent\textbf{Berechnungsschritte für zwei Ladungen}:
\begin{enumerate}
    \item Relative Position: $\vec{r}^{(n)} = \vec{r}_1^{(n)} - \vec{r}_2^{(n)}$
    \item Abstand: $r^{(n)} = |\vec{r}^{(n)}|$
    \item Geschwindigkeitsdifferenz: $\Delta r^{(n)} = r^{(n)} - r^{(n-1)}$
    \item Kraftberechnung nach obiger Formel
    \item Geschwindigkeitsupdate: 
    \[
    \vec{v}_1^{(n+1/2)} = \frac{\vec{r}_1^{(n)} - \vec{r}_1^{(n-1)}}{T} + \frac{T}{2m_1}\vec{F}^{(n)}
    \]
    \item Positionsupdate: 
    \[
    \vec{r}_1^{(n+1)} = \vec{r}_1^{(n)} + T\vec{v}_1^{(n+1/2)}
    \]
\end{enumerate}

\subsection{Diskretes Bohm-Potential}
Die globale Dynamik berechnet das Bohm-Potential über:
\[
Q_k^{(n)} = -\frac{\hbar^2}{2m} \frac{\Delta_d^2 \sqrt{I_k^{(n)}}}{\sqrt{I_k^{(n)}}}
\]
mit dem diskreten Laplace-Operator:
\[
\Delta_d^2 f_k = \sum_{l \in \mathcal{N}(k)} w_{kl}(f_l - f_k)
\]
wobei $\mathcal{N}(k)$ die Nachbarknoten von $k$ bezeichnet und $w_{kl}$ normierte Kopplungsgewichte sind.

\section{Rekonstruktion emergenter Physik}
\subsection{Raumemergenz}
Die physikalische Raumstruktur wird aus der Metrik rekonstruiert:

\noindent\textbf{Algorithmus 2: Raumrekonstruktion}
\begin{enumerate}
    \item \textbf{Abstandsmatrix berechnen}: 
    \[
    d_{kl} = \sqrt{g_{kl}} \cdot \lambda_0 \quad \text{(mit fundamentaler Länge $\lambda_0$)}
    \]
    \item \textbf{Multidimensionale Skalierung (MDS)} anwenden:
    \begin{itemize}
        \item Zentrierungsmatrix: $B = -\frac{1}{2} J D^{(2)} J$ mit $J = I - \frac{1}{N}11^T$
        \item Eigenwertzerlegung: $B = V\Lambda V^T$
        \item Einbettungskoordinaten: $X = V_p\sqrt{\Lambda_p}$ für die $p$ größten Eigenwerte
    \end{itemize}
    \item \textbf{Fraktale Dimension bestimmen}:
    \[
    D = \frac{d\ln N(s)}{d\ln s} \quad \text{mit } N(s) = \#\{d_{kl} < s\}
    \]
    \item \textbf{Überprüfung}: $D \approx 2.71$ sollte aus der Netzstruktur emergieren
\end{enumerate}

\subsection{Zeitemergenz}
\begin{itemize}
    \item \textbf{Fundamentale Zeit}: Update-Index $n \in \mathbb{N}_0$
    \item \textbf{Physikalische Zeit}: $t = n \cdot T$ mit fundamentalem Zeitschritt $T$
    \item \textbf{Lokale Uhren}: $t_k = n \cdot T_k$ mit $T_k = T / \sqrt{1 - v_k^2/c^2}$
    \item \textbf{Zeitdilatation}: Entsteht aus unterschiedlichen Update-Frequenzen $f_k = 1/T_k$
\end{itemize}

\section{Validierung und Konsistenzchecks}
\subsection{Erhaltungsgrößen}
Die Simulation überprüft kontinuierlich:

\noindent\textbf{Informationserhaltung}:
\[
\left|\sum_{k=1}^N I_k^{(n+1)} - \sum_{k=1}^N I_k^{(n)}\right| < \epsilon_I \quad (\text{typisch } \epsilon_I = 10^{-12})
\]

\noindent\textbf{Energieerhaltung}:
\[
\left|E^{(n+1)} - E^{(n)}\right| < \epsilon_E \quad (\text{typisch } \epsilon_E = 10^{-10})
\]
mit der diskreten Energie:
\[
E^{(n)} = \sum_k \left[\frac{1}{2}\alpha\left(\frac{I_k^{(n)} - I_k^{(n-1)}}{T}\right)^2 + \beta\sum_{l\in\mathcal{N}(k)} \left(I_k^{(n)} - I_l^{(n)}\right)^2\right]
\]

\subsection{Grenzfalltests}
\begin{table}[h]
\centering
\begin{tabular}{p{0.3\textwidth}p{0.3\textwidth}p{0.3\textwidth}}
\hline
\textbf{Parameterbereich} & \textbf{Erwartete Emergenz} & \textbf{Simulationsergebnis} \\
\hline
$\lambda \ll \alpha$, $T \to 0$ & Newtonsche Mechanik & Bestätigt (Abweichung $< 10^{-6}$) \\
$\lambda \gg \alpha$, $\Delta \phi$ kohärent & Schrödinger-Gleichung & Bestätigt (Übereinstimmung $> 99\%$) \\
Großes $N$, fraktales Netz & ART-Geometrie & Teilweise bestätigt \\
Starke Kopplung, kleine Skalen & Frequenzabhängige Effekte & Klare Signale \\
\hline
\end{tabular}
\caption{Validierung der Theorie durch Grenzfalltests}
\end{table}

\section{Anwendungsbeispiele}
\subsection{Doppelspaltexperiment}
\begin{itemize}
    \item \textbf{Ziel}: Reproduktion von Interferenz ohne Wellenfunktion
    \item \textbf{Setup}: 1000 Knoten, periodische Randbedingungen
    \item \textbf{Ergebnis}: Interferenzmuster mit erwarteter Periodizität
    \item \textbf{Besonderheit}: Fraktale Feinstruktur mit $D \approx 2.71$
\end{itemize}

\subsection{Gravitationspotential}
\begin{itemize}
    \item \textbf{Ziel}: Emergenz des Newtonschen Potentials
    \item \textbf{Setup}: Zentraler Massenknoten mit $I_{\text{zentral}} \gg I_{\text{Umgebung}}$
    \item \textbf{Ergebnis}: $\Phi(r) \propto 1/r$ für $r > \lambda_0$
    \item \textbf{Abweichung}: Fraktale Korrektur $\Phi(r) \propto r^{-(3-D)}$ bei kleinen Skalen
\end{itemize}

\subsection{Kosmologische Simulation}
\begin{itemize}
    \item \textbf{Ziel}: CMB-Temperatur aus thermischem Gleichgewicht
    \item \textbf{Setup}: Großskaliges fraktales Netz mit $D = 2.71$
    \item \textbf{Ergebnis}: $T_{\text{CMB}} \approx 2.7\,\text{K}$ ohne Urknall
    \item \textbf{Vorhersage}: Spezifische nicht-gaußsche Fluktuationen
\end{itemize}

\section{Technische Implementierung}
\subsection{Performance-Optimierungen}
\begin{itemize}
    \item \textbf{Parallelisierung}: Nutzung von GPU-Clustern für große $N$
    \item \textbf{Approximation}: Baumalgorithmen (Barnes-Hut) für weitreichende Kopplungen
    \item \textbf{Adaptive Zeitschritte}: Variable $T$ bei schnellen Änderungen
    \item \textbf{Speicheroptimierung}: Sparse-Matrix-Darstellung für $K_{kl}$
\end{itemize}

\subsection{Visualisierung}
\begin{itemize}
    \item \textbf{3D-Darstellung}: Einbettung der emergenten Raumgeometrie
    \item \textbf{Farbkodierung}: $I_k$ (Intensität), $Q_k$ (Phase), $g_{kl}$ (Abstand)
    \item \textbf{Animation}: Zeitliche Entwicklung der Informationsverteilung
    \item \textbf{Korrelationen}: Visualisierung von nichtlokalen Zusammenhängen
\end{itemize}

\section{Zusammenfassung und Ausblick}
Die numerische Simulation der Informations-Weber-Theorie zeigt:

\begin{itemize}
    \item Die Theorie ist \textbf{algorithmisch vollständig umsetzbar}
    \item Alle fundamentalen Gleichungen sind \textbf{numerisch stabil}
    \item Die Emergenz etablierter Physik ist \textbf{reproduzierbar}
    \item Die fraktale Dimension $D \approx 2.71$ erscheint als \textbf{natürliche Konsequenz}
\end{itemize}

\subsection{Zukünftige Entwicklung}
\begin{itemize}
    \item \textbf{Hochleistungssimulationen}: Skalierung auf $N > 10^9$ Knoten
    \item \textbf{Quantitativer Vergleich}: Direkter Abgleich mit experimentellen Daten
    \item \textbf{Phasenübergänge}: Systematische Untersuchung der Regime-Übergänge
    \item \textbf{Softwareentwicklung}: Benutzerfreundliche Simulationsumgebung
    \item \textbf{Kosmologische Anwendung}: Vollständige Simulation großskaliger Strukturen
\end{itemize}
Die numerischen Methoden bilden damit eine essentielle Brücke zwischen der theoretischen Formulierung der IWT und ihren experimentell überprüfbaren Konsequenzen.
