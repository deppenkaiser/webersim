\chapter{Vollständige Lösungen exemplarischer Systeme}
\label{app:vollstaendige-loesungen}

\section{Einleitung}
Dieser Anhang präsentiert mathematisch vollständige Lösungen für charakteristische physikalische Systeme im Rahmen der \gls{iwt}. Während Anhang~C numerische Beispiele
enthält, konzentrieren sich diese analytischen Lösungen auf die Demonstration fundamentaler Prinzipien. Alle Lösungen werden sowohl in ihrer diskreten Grundform als auch im
kontinuierlichen Grenzfall dargestellt.

\section{Das Zwei-Körper-Problem mit Weber-Gravitation}
\label{sec:loesung-zweikoerper}

\subsection{Problemstellung}
Zwei Massen $m_1$ und $m_2$ mit $m_2 \ll m_1$ bewegen sich unter dem Einfluss der \gls{wg}.

\subsection{Diskrete Bewegungsgleichungen}
\[
m_2 \frac{\Delta^2 \vec{r}_n}{T^2} = -G \frac{m_1 m_2}{r_n^2} 
\left[ 1 - \frac{1}{c^2} \left( \frac{\Delta r_n}{T} \right)^2 + \beta \frac{r_n}{c^2} \cdot \frac{\Delta^2 r_n}{T^2} \right] \hat{\vec{r}}_n
\]

\subsection{Reduktion auf Ein-Körper-Problem}
Mit reduzierter Masse $\mu = \frac{m_1 m_2}{m_1 + m_2}$ und Relativkoordinate $\vec{r}_n = \vec{r}_{2,n} - \vec{r}_{1,n}$:
\[
\mu \frac{\Delta^2 \vec{r}_n}{T^2} = -G \frac{m_1 m_2}{r_n^2} 
\left[ 1 - \frac{1}{c^2} \left( \frac{\Delta r_n}{T} \right)^2 + \beta \frac{r_n}{c^2} \cdot \frac{\Delta^2 r_n}{T^2} \right] \hat{\vec{r}}_n
\]

\subsection{Polarkoordinaten und Drehimpulserhaltung}
In Polarkoordinaten $(r_n, \theta_n)$ mit Drehimpulserhaltung $h = r_n^2 \frac{\Delta \theta_n}{T}$:
\begin{align*}
\mu \left( \frac{\Delta^2 r_n}{T^2} - r_n \left( \frac{\Delta \theta_n}{T} \right)^2 \right) &= -G \frac{m_1 m_2}{r_n^2} \left[ 1 - \frac{1}{c^2} \left( \frac{\Delta r_n}{T} \right)^2 + \beta \frac{r_n}{c^2} \cdot \frac{\Delta^2 r_n}{T^2} \right] \\
\mu r_n^2 \frac{\Delta \theta_n}{T} &= h = \text{konstant}
\end{align*}

\subsection{Bahn-Gleichung}
Mit $u_n = 1/r_n$ und Entwicklung bis zur ersten Ordnung in $1/c^2$:
\[
\frac{\Delta^2 u_n}{\Delta \theta_n^2} + u_n = \frac{G(m_1 + m_2)}{h^2} \left[ 1 + \frac{3G(m_1 + m_2)}{c^2} u_n \right]
\]

\subsection{Analytische Lösung}
Die Lösung lautet:
\[
u_n = \frac{G(m_1 + m_2)}{h^2} \left[ 1 + e \cos\left( (1 - \delta) \theta_n \right) \right]
\]
mit
\[
\delta = \frac{3G(m_1 + m_2)}{a(1 - e^2)c^2}
\]

\subsection{Periheldrehung}
Pro Umlauf:
\[
\Delta \theta = 2\pi \delta = \frac{6\pi G(m_1 + m_2)}{a(1 - e^2)c^2}
\]
Für das System Sonne-Merkur:
\[
\Delta \theta \approx 42.98'' \text{ pro Jahrhundert}
\]

\section{Der harmonische Oszillator in diskreter Darstellung}
\label{sec:loesung-harmonischer-oszillator}

\subsection{Diskrete Schrödinger-Gleichung}
\[
i\hbar \frac{\psi_{n+1} - \psi_n}{T} = \left( -\frac{\hbar^2}{2m} \Delta_x^2 + \frac{1}{2} m\omega^2 x^2 \right) \psi_n
\]
mit dem diskreten Laplace-Operator $\Delta_x^2$.

\subsection{Stationäre Lösungen}
Ansatz: $\psi_{n,k} = \phi_k e^{-iE_k nT/\hbar}$
\[
\left( -\frac{\hbar^2}{2m} \Delta_x^2 + \frac{1}{2} m\omega^2 x^2 \right) \phi_k = E_k \phi_k
\]

\subsection{Diskrete Eigenfunktionen}
Auf einem äquidistanten Gitter $x_j = j\Delta x$:
\[
-\frac{\hbar^2}{2m(\Delta x)^2} (\phi_{j+1} - 2\phi_j + \phi_{j-1}) + \frac{1}{2} m\omega^2 (j\Delta x)^2 \phi_j = E \phi_j
\]

\subsection{Numerische Eigenwerte}
Für $\Delta x \to 0$ reproduziert die diskrete Gleichung die bekannten Energieniveaus:
\[
E_k = \left( k + \frac{1}{2} \right) \hbar \omega, \quad k = 0, 1, 2, \ldots
\]

\subsection{Informationsdichte}
\[
\rho_{I,j} = |\phi_j|^2 = \frac{1}{2^k k!} \sqrt{\frac{m\omega}{\pi\hbar}} H_k^2\left( \sqrt{\frac{m\omega}{\hbar}} x_j \right) e^{-m\omega x_j^2/\hbar}
\]

\section{Plasma-Oszillationen im diskreten Netzwerk}
\label{sec:loesung-plasma-oscillationen}

\subsection{Diskretes Plasma-Modell}
Elektronen auf diskreten Positionen $\vec{x}_{j,n}$ im Hintergrund positiver Ionen.

\subsection{Bewegungsgleichungen}
\[
m_e \frac{\Delta^2 \vec{x}_{j,n}}{T^2} = -e \vec{E}_{j,n}
\]
mit dem elektrischen Feld aus der \gls{wed}:
\[
\vec{E}_{j,n} = \frac{1}{4\pi\varepsilon_0} \sum_{k \neq j} \frac{e(\vec{x}_{j,n} - \vec{x}_{k,n})}{|\vec{x}_{j,n} - \vec{x}_{k,n}|^3} 
\left[ 1 - \frac{1}{c^2} \left( \frac{\Delta r_{jk,n}}{T} \right)^2 + \frac{2r_{jk,n}}{c^2} \cdot \frac{\Delta^2 r_{jk,n}}{T^2} \right]
\]

\subsection{Linearisierung}
Für kleine Auslenkungen $\vec{x}_{j,n} = \vec{x}_j^0 + \vec{\xi}_{j,n}$:
\[
m_e \frac{\Delta^2 \vec{\xi}_{j,n}}{T^2} = -\frac{e^2}{4\pi\varepsilon_0} \sum_{k \neq j} \frac{\vec{\xi}_{j,n} - \vec{\xi}_{k,n}}{|\vec{x}_j^0 - \vec{x}_k^0|^3}
\]

\subsection{Plasmafrequenz}
Im Kontinuumslimes:
\[
\frac{\Delta^2 \vec{\xi}_n}{T^2} = -\omega_p^2 \vec{\xi}_n
\]
mit
\[
\omega_p^2 = \frac{n_0 e^2}{\varepsilon_0 m_e}
\]

\subsection{Diskrete Lösung}
\[
\vec{\xi}_n = \vec{\xi}_0 \cos(\omega_p nT)
\]

\section{Lichtausbreitung in fraktaler Geometrie}
\label{sec:loesung-licht-fraktal}

\subsection{Wirkung für Photonen}
\[
S_\gamma = \sum_n \left[ \frac{E}{c^2} \left( \frac{\Delta \vec{x}_n}{T} \right)^2 - \frac{2GME}{c^4 r_n} \left( 1 - \frac{1}{c^2} \left( \frac{\Delta r_n}{T} \right)^2 + \frac{r_n}{c^2} \cdot \frac{\Delta^2 r_n}{T^2} \right) \right] T
\]

\subsection{Geodätengleichung}
\[
\frac{\Delta^2 x_n^\mu}{T^2} + \Gamma^\mu_{\alpha\beta} \frac{\Delta x_n^\alpha}{T} \frac{\Delta x_n^\beta}{T} = 0
\]
mit diskreten Christoffel-Symbolen in fraktaler Geometrie.

\subsection{Frequenzabhängige Lichtablenkung}
\[
\Delta \theta(\omega) = \frac{4GM}{c^2 b} \left[ 1 + \alpha(D) \left( \frac{\omega_0}{\omega} \right)^{3-D} \right]
\]
mit fraktaler Dimension $D \approx 2.71$ und
\[
\alpha(D) = \frac{\Gamma(4-D)}{(3-D)\Gamma(3-D)}
\]

\section{Nichtlineare Schrödinger-Gleichung aus IWT}
\label{sec:loesung-nls}

\subsection{Erweiterte Wirkung}
\[
S = \sum_n \left[ i\hbar \psi_n^* \frac{\psi_{n+1} - \psi_n}{T} - \frac{\hbar^2}{2m} |\Delta_x \psi_n|^2 - g |\psi_n|^4 \right] T
\]

\subsection{Variation}
\[
i\hbar \frac{\psi_{n+1} - \psi_n}{T} = -\frac{\hbar^2}{2m} \Delta_x^2 \psi_n + 2g |\psi_n|^2 \psi_n
\]

\subsection{Soliton-Lösung}
Im Kontinuumslimes für 1D:
\[
\psi(x,t) = A \operatorname{sech}\left( \frac{x - vt}{\xi} \right) e^{i(kx - \omega t)}
\]
mit
\[
A = \sqrt{\frac{\hbar^2}{2mg\xi^2}}, \quad \omega = \frac{\hbar k^2}{2m} - \frac{\hbar}{2m\xi^2}
\]

\section{Thermisches Gleichgewicht im diskreten Netzwerk}
\label{sec:loesung-thermisches-gleichgewicht}

\subsection{Informations-Hamiltonian}
\[
H = \sum_k \left[ \alpha (\Delta_t I_{k,n})^2 + \beta (\Delta_x I_{k,n})^2 + \gamma \frac{(\Delta_x I_{k,n})^2}{I_{k,n}} \right]
\]

\subsection{Canonische Verteilung}
\[
P(\{I_k\}) = \frac{1}{Z} \exp\left( -\frac{H}{k_B T} \right)
\]
mit Zustandssumme
\[
Z = \int \exp\left( -\frac{H}{k_B T} \right) \prod_k dI_k
\]

\subsection{Korrelationsfunktionen}
\[
\langle I_k I_l \rangle - \langle I_k \rangle \langle I_l \rangle = \frac{k_B T}{\beta} G_{kl}
\]
mit Greens-Funktion $G_{kl}$ des diskreten Laplace-Operators.

\section{Zusammenfassung der Lösungsmethoden}
\begin{table}[ht]
\centering
\begin{tabular}{p{0.25\textwidth}p{0.3\textwidth}p{0.3\textwidth}}
\hline
\textbf{System} & \textbf{Lösungsmethode} & \textbf{Charakteristika} \\
\hline
Zwei-Körper-Problem & Diskretes Variationsprinzip & Periheldrehung ohne Raumzeitkrümmung \\
\hline
Harmonischer Oszillator & Diskrete Eigenwertgleichung & Quantisierung aus globalem Funktional \\
\hline
Plasma-Oszillationen & Lineare Störungstheorie & Emergenz der Plasmafrequenz \\
\hline
Lichtausbreitung & Diskrete Geodätengleichung & Frequenzabhängige Ablenkung \\
\hline
Nichtlineare Wellen & Variation erweiterter Wirkung & Soliton-Lösungen \\
\hline
Thermisches Gleichgewicht & Statistische Mechanik & Korrelationsfunktionen \\
\hline
\end{tabular}
\caption{Übersicht der gelösten Systeme und Methoden}
\end{table}

\subsection{Allgemeine Lösungsstrategien}
\begin{enumerate}
    \item \textbf{Diskretisierung}: Übertragung des Problems auf diskretes Netzwerk
    \item \textbf{Variationsprinzip}: Ableitung der Bewegungsgleichungen aus diskreter Wirkung
    \item \textbf{Linearisierung}: Behandlung kleiner Störungen
    \item \textbf{Symmetrien}: Ausnutzung von Erhaltungsgrößen
    \item \textbf{Kontinuumslimes}: Übergang zu etablierten Gleichungen
\end{enumerate}

\subsection{Schlussfolgerungen}
\begin{itemize}
    \item Die \gls{iwt} bietet konsistente Lösungen für alle grundlegenden physikalischen Systeme
    \item Die diskrete Formulierung ist mathematisch wohldefiniert und lösbar
    \item Im entsprechenden Grenzfall werden alle bekannten Ergebnisse reproduziert
    \item Die Theorie macht darüber hinaus spezifische neue Vorhersagen
    \item Die Lösungsmethoden sind allgemein und auf komplexere Systeme übertragbar
\end{itemize}
Diese vollständigen analytischen Lösungen demonstrieren die mathematische Konsistenz und Anwendbarkeit der \gls{iwt} über das gesamte Spektrum physikalischer Phänomene
hinweg.
