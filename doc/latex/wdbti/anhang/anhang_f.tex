\chapter{Tabellen, Symbole und Definitionen}

\section{Symbolverzeichnis}
\begin{table}[ht]
\centering
\begin{tabular}{p{0.2\textwidth}p{0.7\textwidth}}
\textbf{Symbol} & \textbf{Bedeutung} \\
\hline
$I_k^{(n)}$ & Diskretes universelles Informationsfeld am Knoten $k$ zum Zeitindex $n$ \\
$K_{kl}^{(n)}$ & Kopplungsmatrix zwischen Knoten $k$ und $l$ \\
$g_{kl}^{(n)}$ & Diskrete Informationsmetrik (emergente Geometrie) \\
$\Delta_k I^{(n)}$ & Diskreter Informationsgradient am Knoten $k$ \\
$\Delta_{kl} I^{(n)}$ & Diskretes Informationsdifferenz zwischen Knoten $k$ und $l$ \\
$\Delta^2 I_k^{(n)}$ & Diskreter Laplace-Operator des Informationsfeldes \\
$\lambda$ & Stärke der globalen Bohm-Kopplung \\
$\mu$ & Stärke der fraktalen Skalierungskopplung \\
$\gamma_{\mathrm{eff}}$ & Effektive fraktale Kopplungskonstante \\
$\rho_{\mathrm{eff}}$ & Effektive Massendichte des Universums \\
$L$ & Kosmische Skala (Mach-Radius) \\
$D$ & Fraktale Dimension des Universums ($D \approx 2.71$) \\
$K_M(D)$ & Mach-Kopplungsfaktor der fraktalen Struktur \\
$\bar{\alpha}(L)$ & Kosmische Verlustkonstante \\
$u_\gamma$ & Energiedichte des Photonengases \\
$\varepsilon$ & Emissivität des kosmischen Plasmas \\
$A_{\mathrm{eff}}$ & Effektive Oberfläche pro Volumen \\
$X$ & Kombinierte Plasmaparametergröße $X = u_\gamma/(\varepsilon A_{\mathrm{eff}})$ \\
$c$ & Maximale Informationsflussrate (Lichtgeschwindigkeit) \\
$\hbar$ & Globale Informationsgranularität (Plancksches Wirkungsquantum) \\
$G$ & Gravitationskonstante (fraktale Kopplungsstärke) \\
$\alpha$ & Feinstrukturkonstante \\
$k_B$ & Informations-Temperatur-Skala (Boltzmann-Konstante) \\
$T_{\mathrm{CMB}}$ & Gleichgewichtstemperatur des kosmischen Plasmas \\
$z$ & Rotverschiebung \\
$T$ & Fundamentaler Zeitschritt (diskrete Zeit) \\
$n$ & Diskreter Zeitindex (Update-Schritt) \\
$\lambda_0$ & Fundamentale Netzwerklänge \\
$f_{\max}$ & Maximale Update-Frequenz ($f_{\max} = 1/T$) \\
$m_{\mathrm{eff},k}$ & Effektive Masse am Knoten $k$ \\
$E^{(n)}$ & Diskrete Gesamtenergie \\
$\vec{P}^{(n)}$ & Diskreter Gesamtimpuls \\
$\vec{L}^{(n)}$ & Diskreter Gesamtdrehimpuls \\
\hline
\end{tabular}
\caption{Symbolverzeichnis der diskreten Informations-Weber-Theorie}
\end{table}

\section{Glossar}
\begin{itemize}
    \item \textbf{Informationsfeld} — Fundamentale Größe der IWT; diskretes Skalarfeld $I_k^{(n)}$, dessen Struktur und Dynamik alle physikalischen Phänomene beschreibt.
    \item \textbf{Informationsmetrik} — Aus dem Informationsfeld emergente diskrete Geometrie $g_{kl}^{(n)}$; definiert Raum, Zeit und Dynamik im Netzwerk.
    \item \textbf{Informationsfluss} — Grundlegende Form der Dynamik; Energie ist eine abgeleitete Erhaltungsgröße des Informationsflusses zwischen Knoten.
    \item \textbf{Weber-Dynamik} — Lokale direkte Wechselwirkung zwischen Informationsknoten; erzeugt Trägheit, klassische Mechanik und elektromagnetische Effekte.
    \item \textbf{Bohm-Struktur} — Globale Organisationsdynamik des gesamten Netzwerks; erzeugt Wellenphänomene, Nichtlokalität und quantenartige Kohärenz.
    \item \textbf{Fraktale Informationsgeometrie} — Großskalige selbstähnliche Struktur des Universums mit effektiver Dimension $D \approx 2.71$.
    \item \textbf{Kosmische Verlustkonstante} — Logarithmische Kopplungsgröße $\bar{\alpha}(L)$, die die Rotverschiebungsheizung bestimmt.
    \item \textbf{Kombinierte Plasmaparametergröße $X$} — Zusammenfassung der mikrophysikalischen Plasmaeigenschaften: $X = u_\gamma/(\varepsilon A_{\mathrm{eff}})$.
    \item \textbf{Mach-Radius $L$} — Kosmische Skala, die die fraktale Kopplung bestimmt und mit der effektiven Dichte $\rho_{\mathrm{eff}}$ zusammenhängt.
    \item \textbf{Informationszeit} — Emergent aus der Dynamik der Informationsmetrik; physikalische Zeit ist Ordnungsstruktur der Informationsänderung ($t \approx nT$).
    \item \textbf{Diskretes Variationsprinzip} — Grundlage der Dynamik; minimiert das diskrete Informations-Lagrange-Funktional $\mathcal{L}_d[I_k^{(n)}]$.
    \item \textbf{Emergenz} — Prozess, durch den bekannte physikalische Theorien (klassische Mechanik, QM, ART) als Grenzfälle der diskreten IWT erscheinen.
\end{itemize}

\section{Wichtige Gleichungen}

\subsection{Dynamische Gleichung der Informationsmetrik (diskret)}
\[
\boxed{
g_{kl}^{(n+1)} = g_{kl}^{(n)} + T \cdot \left[
\Delta_{k} I^{(n)} \Delta_{l} I^{(n)}
-
\lambda\,\frac{\Delta_{kl}^2 I^{(n)}}{I_{kl}^{(n)}}
+
\mu\,g_{kl}^{(n)}\,\ln\!\left(1+\gamma_{\mathrm{eff}} G \rho_{\mathrm{eff}} L^2\right)
\right]
}
\]

\subsection{Dynamische Gleichung der Informationsmetrik (kontinuierlich)}
\[
\frac{d}{dt} g_{ij}
=
\partial_i I \partial_j I
-
\lambda\,\frac{\partial_i\partial_j I}{I}
+
\mu\,g_{ij}\,\ln\!\left(1+\gamma_{\mathrm{eff}} G \rho_{\mathrm{eff}} L^2\right).
\]

\subsection{Diskretes Informations-Lagrange-Funktional}
\[
\mathcal{L}_d[I_k^{(n)}] 
= 
\frac{1}{2} \sum_{k,l} K_{kl}^{(n)} \Delta_{kl} I^{(n)} \Delta_{kl} I^{(n)}
-
\frac{\lambda}{2}\sum_k \frac{\Delta^2 I_k^{(n)}}{I_k^{(n)}}
+
\mu \ln\!\left(1+\gamma_{\mathrm{eff}} G \rho_{\mathrm{eff}} L^2\right) \sum_k I_k^{(n)}.
\]

\subsection{Kontinuierliches Informations-Lagrange-Funktional}
\[
\mathcal{L}[I] 
= 
\frac{1}{2} g^{ij} \partial_i I \partial_j I
-
\frac{\lambda}{2}\frac{\nabla^2 I}{I}
+
\mu \ln\!\left(1+\gamma_{\mathrm{eff}} G \rho_{\mathrm{eff}} L^2\right).
\]

\subsection{Kosmische Verlustkonstante}
\[
\bar{\alpha}(L)
=
\frac{1}{L\,\gamma_{\mathrm{eff}} G \rho_{\mathrm{eff}}}
\ln\!\bigl(1+\gamma_{\mathrm{eff}} G \rho_{\mathrm{eff}} L^2\bigr).
\]

\subsection{CMB-Gleichgewichtstemperatur}
\[
T_{\mathrm{CMB}}
=
\left(
\frac{\bar{\alpha}(L)\,u_\gamma}
{\varepsilon\,A_{\mathrm{eff}}\,\sigma}
\right)^{1/4}.
\]

\subsection{Kombinierte Plasmaparametergröße}
\[
X = \frac{u_\gamma}{\varepsilon A_{\mathrm{eff}}}.
\]

\subsection{Diskrete Weber-Kraft (allgemeine Form)}
\[
F_{ij}^{(n)} = \frac{q_i q_j}{4\pi\varepsilon_0 (r_{ij}^{(n)})^2} \left[ 1 - \frac{1}{c^2} \left( \frac{\Delta r_{ij}^{(n)}}{T} \right)^2 + \frac{2r_{ij}^{(n)}}{c^2} \cdot \frac{\Delta^2 r_{ij}^{(n)}}{T^2} \right] \hat{r}_{ij}^{(n)}.
\]

\subsection{Weber-Gravitation (kontinuierlich)}
\[
F_{\rm WG} = -G\frac{Mm}{r^{2}}\left(1 - \frac{\dot{r}^{2}}{c^{2}} + \beta\,\frac{r\ddot{r}}{c^{2}}\right),
\]
wobei $\beta = 0.5$ für massive Körper, $\beta = 1$ für Photonen und $\beta = 2$ für elektromagnetische Wechselwirkungen.

\subsection{Diskretes Bohm-Potential}
\[
Q_k^{(n)} = -\frac{\hbar^2}{2m}\frac{\Delta^2 \sqrt{I_k^{(n)}}}{\sqrt{I_k^{(n)}}}.
\]

\subsection{Kontinuierliches Bohm-Potential}
\[
Q = -\frac{\hbar^2}{2m}\frac{\nabla^2 R}{R}, \quad \text{mit } R = \sqrt{\rho_I}.
\]

\subsection{Rotverschiebungsrelation (fraktale Kosmologie)}
\[
z(d) = \gamma_{\mathrm{eff}} G \rho_{\mathrm{eff}} d^2.
\]

\subsection{Fraktale Dimension}
\[
D = \frac{\ln 20}{\ln(2+\phi)} \approx 2.71, \quad \text{mit } \phi = \frac{1+\sqrt{5}}{2} \text{ (Goldener Schnitt)}.
\]

\section{Schlussbemerkung}
Dieser Anhang fasst die mathematischen und begrifflichen Grundlagen der \gls{iwt} in ihrer diskreten Formulierung zusammen. Er dient als Referenz für die
in den Teilen I–III entwickelten Strukturen und ermöglicht eine konsistente Anwendung der Theorie auf physikalische, kosmologische und mathematische Fragestellungen. Die
durchgehende Unterscheidung zwischen fundamentaler diskreter und emergenter kontinuierlicher Darstellung entspricht der konzeptionellen Struktur der gesamten Arbeit.
