\chapter{Axiomatische Grundstruktur der Dynamischen Schwere-Trägheits-Theorie}
In diesem Kapitel wird die Theorie der dynamischen Aufteilung der Schwerewirkung in radiale und tangentiale Trägheitsantworten in axiomatischer Form dargestellt. Die
Theorie ersetzt die klassische Orbitmechanik durch eine energetisch gesteuerte Spiralmechanik, in der der dimensionslose Umlenkungszustand $\beta(t)$ die gesamte Dynamik
bestimmt. Die Axiome sind logisch unabhängig und vollständig.

\section*{Axiom 1: Schwere als einzige Ursache}
Zwischen zwei Körpern existiert eine radiale Schwerewirkung $F_G(r)$, die ausschließlich vom Abstand $r$ abhängt. Es existieren keine weiteren Kräfte und keine externen
Felder. Die Schwere ist die einzige Ursache der Bewegung.

\section*{Axiom 2: Trägheit als einzige Wirkung}
Die Schwerewirkung erzeugt zwei orthogonale Trägheitsantworten: eine radiale Beschleunigung $a_Z$ und eine tangentiale Beschleunigung $a_B$. Diese beiden Beschleunigungen
beschreiben die vollständige Bewegung des Testkörpers.

\section*{Axiom 3: Aufteilung der Schwerewirkung}
Die Schwerewirkung teilt sich in einen radialen Anteil $(1-\beta)F_G$ und einen tangentialen Anteil $\beta F_G$. Der Zustand $\beta$ ist eine dynamische Größe im Intervall
$[0,1]$ und bestimmt das Verhältnis von Fall und Umlenkung.

\section*{Axiom 4: Trägheitsmasse}
Die Trägheitsantworten koppeln an eine effektive Masse $m_T = m + \alpha M$. Die Bewegungsgleichungen lauten
\[
m_T a_Z = (1-\beta)F_G, \qquad
m_T a_B = \beta F_G.
\]
Damit hängt die Trägheit des Testkörpers vom Zentralkörper ab.

\section*{Axiom 5: Energiezerlegung}
Die Gesamtenergie zerfällt eindeutig in radiale und tangentiale Energie,
\[
E_Z = \frac12 m_T \dot r^2, \qquad
E_B = \frac12 m_T r^2 \dot\phi^2,
\]
und
\[
E = E_Z + E_B.
\]
Der Umlenkungszustand ist energetisch definiert durch
\[
\beta = \frac{E_B}{E}.
\]

\section*{Axiom 6: Dynamik des Umlenkungszustands}
Die zeitliche Entwicklung von $\beta$ folgt aus den Energieflüssen
\[
\dot E_B = F_G v_B, \qquad
\dot E_Z = -\beta F_G v_Z.
\]
Damit ergibt sich die autonome Evolutionsgleichung
\[
\dot\beta = \frac{F_G}{E^2}\left(E_Z v_B + \beta E_B v_Z\right).
\]
Der Zustand $\beta(t)$ ist monoton wachsend und erreicht den Wert $1$ nicht.

\section*{Axiom 7: Universelle Bahnform}
Die Bewegungsgleichungen
\[
\dot r = \sqrt{\frac{2(1-\beta)E}{m_T}}, \qquad
\dot\phi = \sqrt{\frac{2\beta E}{m_T}}\frac{1}{r},
\]
erzeugen nach Eliminierung der Zeit die universelle Bahnform
\[
r(\phi) = K e^{\phi/B},
\]
wobei
\[
B = \frac{2E}{F_G}.
\]
Die Bahn ist stets eine logarithmische Spirale. Kreisbahnen und Ellipsen sind ausgeschlossen.

\section*{Folgerungen}
Aus den Axiomen folgt, dass alle Satelliten spiralförmige Bahnen besitzen. Der Zustand $\beta(t)$ bestimmt die Richtung der Drift: Für $\beta < 1/2$ erfolgt eine
Einwärtsbewegung, für $\beta > 1/2$ eine Auswärtsbewegung. Langfristig spiralisieren alle Satelliten nach außen, da $\beta(t)$ monoton wächst. Die Theorie erklärt
Wanderbewegungen von Planeten und Monden ohne Zusatzannahmen.

\section{Dynamik, Bewegungsgleichungen und Spiralbahnen}

Dieses Kapitel entwickelt die vollständige Dynamik der Theorie aus den Axiomen des vorangegangenen Kapitels. Die Bewegung eines Testkörpers ergibt sich aus der Aufteilung
der Schwerewirkung in radiale und tangentiale Trägheitsantworten sowie aus der zeitlichen Entwicklung des Umlenkungszustands $\beta(t)$. Die resultierenden Bahnen sind
logarithmische Spiralen, deren Richtung und Krümmung durch die Energieverteilung bestimmt werden.

\section*{1. Radial- und Tangentialbewegung}
Aus der Aufteilung der Schwerewirkung folgt unmittelbar
\[
m_T a_Z = (1-\beta)F_G, \qquad
m_T a_B = \beta F_G.
\]
Mit den Definitionen
\[
a_Z = \ddot r, \qquad
a_B = r\dot\phi^2,
\]
ergeben sich die Bewegungsgleichungen
\[
m_T \ddot r = (1-\beta)F_G,
\]

\[
m_T r\dot\phi^2 = \beta F_G.
\]
Die radiale und die tangentiale Bewegung sind vollständig durch $\beta(t)$ gekoppelt.

\section*{2. Energieformulierung}
Die kinetischen Energien lauten
\[
E_Z = \frac12 m_T \dot r^2, \qquad
E_B = \frac12 m_T r^2 \dot\phi^2.
\]
Die Gesamtenergie ist
\[
E = E_Z + E_B.
\]
Der Umlenkungszustand ist energetisch definiert durch
\[
\beta = \frac{E_B}{E}.
\]
Damit gilt
\[
E_Z = (1-\beta)E, \qquad
E_B = \beta E.
\]

\section*{3. Geschwindigkeiten}
Aus der Energiezerlegung folgen die Geschwindigkeiten
\[
\dot r = \sqrt{\frac{2(1-\beta)E}{m_T}},
\]

\[
\dot\phi = \sqrt{\frac{2\beta E}{m_T}}\frac{1}{r}.
\]
Die radiale Geschwindigkeit verschwindet nur für $\beta = 1$, ein Wert, der nicht erreicht wird. Die tangentiale Geschwindigkeit verschwindet nur für $\beta = 0$, was nur
im Anfangszustand möglich ist. Damit besitzt jede reale Bewegung sowohl radiale als auch tangentiale Komponenten.

\section*{4. Dynamik des Umlenkungszustands}
Die Energieflüsse lauten
\[
\dot E_B = F_G v_B, \qquad
\dot E_Z = -\beta F_G v_Z.
\]
Mit $E = E_Z + E_B$ ergibt sich die Evolutionsgleichung
\[
\dot\beta = \frac{F_G}{E^2}\left(E_Z v_B + \beta E_B v_Z\right).
\]
Da alle Terme positiv sind, folgt
\[
\dot\beta > 0.
\]
Der Umlenkungszustand wächst monoton und erreicht den Wert $1$ nicht. Damit bleibt stets ein radialer Anteil der Schwerewirkung erhalten.

\section*{5. Eliminierung der Zeit}
Die Eliminierung der Zeit aus den Bewegungsgleichungen führt zu
\[
\frac{dr}{d\phi}
= \frac{\dot r}{\dot\phi}
= \frac{r}{B},
\]
wobei
\[
B = \frac{2E}{F_G}.
\]
Die Lösung lautet
\[
r(\phi) = K e^{\phi/B}.
\]
Die Bahn ist eine logarithmische Spirale. Kreisbahnen und Ellipsen sind ausgeschlossen.

\section*{6. Richtung der Spiralbewegung}
Das Vorzeichen von $\dot r$ bestimmt die Richtung der Drift. Aus der Energiezerlegung folgt
\[
\dot r > 0 \quad \text{für} \quad \beta > \frac12,
\]

\[
\dot r < 0 \quad \text{für} \quad \beta < \frac12.
\]
Damit gilt:
\begin{itemize}
\item Für $\beta < 1/2$ spiralisieren Satelliten nach innen.
\item Für $\beta > 1/2$ spiralisieren sie nach außen.
\item Da $\beta(t)$ monoton wächst, spiralisieren alle Satelliten langfristig nach außen.
\end{itemize}

\section*{7. Wander-Geschwindigkeit}
Die radiale Driftgeschwindigkeit ergibt sich zu
\[
v_{\text{wander}} = \dot r
= \pm \frac{F_G}{2E}\sqrt{\frac{2\beta E}{m_T}}.
\]
Das Vorzeichen bestimmt die Richtung der Spiralbewegung. Die Drift verschwindet nie, da $\beta(t)$ nie den Wert $1$ erreicht.

\section*{8. Konsequenzen}
Die Theorie sagt voraus, dass alle Satelliten spiralförmige Bahnen besitzen und dass keine Bahn stabil ist. Die Drift ist eine direkte Folge der dynamischen
Energieaufteilung und benötigt keine zusätzlichen Mechanismen wie Reibung, Gezeiten oder Resonanzen. Planetare Migration, Mondwanderung und die Existenz von Hot Jupiters
ergeben sich unmittelbar aus der Dynamik des Umlenkungszustands.

\section{Vergleich mit Newtons Mechanik und der Allgemeinen Relativitätstheorie}
Dieses Kapitel stellt die Dynamische Schwere--Trägheits-Theorie den beiden etablierten Beschreibungen gravitativer Bewegung gegenüber: der newtonschen Mechanik und der
Allgemeinen Relativitätstheorie. Der Vergleich erfolgt ausschließlich auf der Ebene der Grundannahmen, der Bewegungsgleichungen und der Bahnformen. Die Unterschiede sind
grundlegend und betreffen sowohl die Struktur der Dynamik als auch die Interpretation der Bewegung.

\section*{1. Vergleich der Grundannahmen}
Die newtonsche Mechanik basiert auf einer einzigen Kraft, die stets radial wirkt und deren Wirkung vollständig durch die Trägheit des Testkörpers beantwortet wird. Die
Allgemeine Relativitätstheorie ersetzt die Kraft durch die Geometrie der Raumzeit und beschreibt Bewegung als Geodäten in einer gekrümmten Metrik. Die Dynamische
Schwere--Trägheits-Theorie unterscheidet sich von beiden Ansätzen durch die Aufteilung der Schwerewirkung in zwei orthogonale Trägheitsantworten und durch die Einführung
des dynamischen Zustands $\beta(t)$.

Newton:
\[
F = m a, \qquad F_G = \frac{GMm}{r^2}.
\]
Relativität:
\[
\text{Bewegung entlang von Geodäten der Metrik } g_{\mu\nu}.
\]
Dynamische Theorie:
\[
m_T a_Z = (1-\beta)F_G, \qquad m_T a_B = \beta F_G.
\]
Damit besitzt die Dynamische Theorie eine interne Freiheitsvariable, die in den klassischen Theorien nicht existiert.

\section*{2. Vergleich der Energieformulierung}
In der newtonschen Mechanik ist die Energieaufteilung festgelegt durch
\[
E = \frac12 m \dot r^2 + \frac12 m r^2 \dot\phi^2 - \frac{GMm}{r}.
\]
Die Relativitätstheorie verwendet keine globale Energieerhaltung in gekrümmten Raumzeiten.

Die Dynamische Theorie führt eine eindeutige Zerlegung ein:
\[
E_Z = \frac12 m_T \dot r^2, \qquad
E_B = \frac12 m_T r^2 \dot\phi^2,
\]

\[
E = E_Z + E_B, \qquad
\beta = \frac{E_B}{E}.
\]
Die Energieaufteilung ist dynamisch und bestimmt die Bahnform.

\section*{3. Vergleich der Bahnformen}
Newton erlaubt geschlossene Bahnen:
\[
r(\phi) = \frac{p}{1 + e\cos\phi}.
\]
Die Relativitätstheorie erzeugt präzedierende Ellipsen.

Die Dynamische Theorie liefert ausschließlich logarithmische Spiralen:
\[
r(\phi) = K e^{\phi/B}.
\]
Damit sind Kreisbahnen und Ellipsen ausgeschlossen. Die Bahn ist stets offen und besitzt eine konstante relative Krümmung.

\section*{4. Vergleich der Stabilität}
In der newtonschen Mechanik existieren stabile Kreisbahnen. In der Relativitätstheorie existieren stabile und instabile Geodäten, abhängig vom Potential.

In der Dynamischen Theorie existiert keine stabile Bahn. Da $\beta(t)$ monoton wächst und nie den Wert $1$ erreicht, bleibt stets ein radialer Anteil der Schwerewirkung
erhalten. Damit gilt
\[
\dot r \neq 0.
\]
Jede Bahn besitzt eine Drift, und alle Satelliten spiralisieren langfristig nach außen.

\section*{5. Vergleich der physikalischen Interpretation}
Newton interpretiert die Bewegung als Resultat einer Kraft. Die Relativitätstheorie interpretiert sie als freie Bewegung in einer gekrümmten Raumzeit. Die Dynamische
Theorie interpretiert die Bewegung als Ergebnis einer dynamischen Energieaufteilung zwischen Fall und Umlenkung. Der Zustand $\beta(t)$ ersetzt sowohl das newtonsche
Drehimpulskonzept als auch die relativistische Geodätenstruktur.

\section*{6. Vergleich der beobachtbaren Konsequenzen}
Die newtonsche Mechanik erklärt keine langfristige planetare Migration ohne Zusatzmechanismen. Die Relativitätstheorie erklärt Präzessionen, aber keine systematische Drift.

Die Dynamische Theorie sagt Wanderbewegungen aller Satelliten voraus. Beispiele wie die Drift des Erdmondes, der Einwärtsfall von Phobos, die Migration von Gasriesen und
die Existenz von Hot Jupiters ergeben sich unmittelbar aus der Dynamik des Umlenkungszustands.

\section*{7. Zusammenfassung}
Die Dynamische Schwere-Trägheits-Theorie unterscheidet sich grundlegend von Newton und der Relativität. Sie ersetzt geschlossene Bahnen durch logarithmische Spiralen,
stabile Orbits durch Drift, konstante Energieaufteilung durch eine dynamische Variable und die klassische Kraftinterpretation durch eine zweigeteilte Trägheitsantwort. Die
Theorie ist damit eine eigenständige, axiomatisch definierte Alternative zu den etablierten Beschreibungen gravitativer Bewegung.

\section{Die Dynamische Schwere--Trägheits-Theorie als emergenter Sektor der WDBT/IWT}
Dieses Kapitel zeigt, dass die Dynamische Schwere--Trägheits-Theorie (DSTT) nicht als eigenständiges Modell neben der Weber--De-Broglie--Bohm-Theorie (WDBT) steht, sondern
als ihr makroskopischer Grenzfall aus der Informations-Weber-Theorie (IWT) hervorgeht. Die DSTT entsteht aus der diskreten Informationsdynamik der IWT durch geeignete
Grenzübergänge und Identifikationen. Die Äquivalenz wird durch eine explizite Abbildung der fundamentalen Größen der IWT auf die makroskopischen Größen der DSTT formal
bewiesen.

\section*{1. Strukturen der IWT und DSTT}
Die IWT beschreibt die physikalische Realität durch diskrete Informationsfelder $I_k^{(n)}$, Kopplungen $K_{kl}^{(n)}$ und eine fraktale Metrik $g_{kl}^{(n)}$. Das
Variationsprinzip der IWT führt zu einer Dynamik, die aus zwei Anteilen besteht:
\[
L = L_{\text{lokal}}[I] + L_{\text{global}}[I].
\]
Der lokale Anteil entspricht Weber-artigen Wechselwirkungen, der globale Anteil entspricht Bohm-artigen Organisationsstrukturen.

Die DSTT beschreibt die Bewegung eines Testkörpers im Zentralfeld durch die Aufteilung der Schwerewirkung in radiale und tangentiale Trägheitsantworten. Die
Energiezerlegung lautet
\[
E_Z = \frac12 m_T \dot r^2, \qquad
E_B = \frac12 m_T r^2 \dot\phi^2,
\]
und der Umlenkungszustand ist definiert durch
\[
\beta = \frac{E_B}{E_Z + E_B}.
\]
Die Bahnform ergibt sich zu
\[
r(\phi) = K e^{\phi/B},
\]
wobei $B = 2E/F_G$.

\section*{2. Abbildung der IWT auf die DSTT}

\subsection*{2.1 Schwerewirkung}
Im makroskopischen Grenzfall erzeugt die fraktale Informationsmetrik der IWT ein effektives Potential
\[
\Phi(r) = -\frac{GM_{\text{eff}}}{r},
\]
woraus die Schwerewirkung folgt:
\[
F_G(r) = -\nabla\Phi(r) = \frac{GM_{\text{eff}} m_{\text{eff}}}{r^2}.
\]
Dies entspricht exakt der Schwerewirkung der DSTT.

\subsection*{2.2 Energiezerlegung}
Der lokale Anteil des IWT-Lagrange-Funktionals beschreibt kinetische Energie:
\[
L_{\text{lokal}} \leftrightarrow E_Z.
\]
Der globale Anteil beschreibt Bohm-artige Umlenkungsenergie:
\[
L_{\text{global}} \leftrightarrow E_B.
\]
Damit ist die DSTT-Zerlegung
\[
E = E_Z + E_B
\]
die kontinuierliche Version der IWT-Zerlegung.

\subsection*{2.3 Der Umlenkungszustand}
In der IWT ist der Anteil globaler Organisation definiert durch
\[
\beta_{\text{IWT}} = \frac{L_{\text{global}}}{L_{\text{lokal}} + L_{\text{global}}}.
\]
In der DSTT lautet die Definition
\[
\beta = \frac{E_B}{E_Z + E_B}.
\]
Damit gilt
\[
\beta = \beta_{\text{IWT}}.
\]

\subsection*{2.4 Monotonie des Umlenkungszustands}
Das Variationsprinzip der IWT erzwingt eine Zunahme des globalen Organisationsanteils:
\[
\dot L_{\text{global}} > 0.
\]
Daraus folgt unmittelbar
\[
\dot\beta > 0.
\]
Dies entspricht Axiom 6 der DSTT.

\subsection*{2.5 Bahnform}
Die Bohm-Führungsgleichung der IWT lautet
\[
m\ddot{\mathbf r} = -\nabla(V + Q).
\]
Für ein Zentralpotential $V(r)$ und ein Bohm-Potential $Q$, das aus der globalen Informationsstruktur entsteht, ergibt sich nach Eliminierung der Zeit
\[
\frac{dr}{d\phi} = \frac{r}{B},
\]
mit $B = 2E/F_G$. Die Lösung ist
\[
r(\phi) = K e^{\phi/B}.
\]
Dies ist exakt die Bahnform der DSTT.

\section*{3. Satz und Beweis}
\textbf{Satz.} Im makroskopischen, zweikörperigen, schwach-relativistischen Grenzfall der WDBT/IWT ist die Bewegung eines Testkörpers äquivalent zur Dynamik der DSTT.
\textbf{Beweis.}  
(1) Die IWT erzeugt ein effektives Zentralpotential $\Phi(r)$ und damit die Schwerewirkung $F_G(r)$.  
(2) Die Zerlegung des IWT-Lagrange-Funktionals in lokale und globale Anteile entspricht der DSTT-Zerlegung in $E_Z$ und $E_B$.  
(3) Der Anteil globaler Organisation entspricht dem DSTT-Zustand $\beta$.  
(4) Das Variationsprinzip der IWT erzwingt $\dot\beta > 0$.  
(5) Die Bohm-Führungsgleichung der IWT liefert nach Eliminierung der Zeit die Bahnform $r(\phi) = K e^{\phi/B}$.  
Damit stimmen alle Axiome der DSTT exakt mit den emergenten Gleichungen der IWT überein.  
\hfill$\square$

\section*{4. Konsequenz}
Die DSTT ist kein unabhängiges Modell, sondern der makroskopische Gravitationssektor der WDBT/IWT. Die Größen $F_G$, $m_T$, $E_Z$, $E_B$, $\beta$ und die Spiralbahnform
entstehen direkt aus der diskreten Informationsdynamik der IWT. Die WDBT/IWT bildet damit eine vollständige Ur-Theorie, deren makroskopische Gravitation durch die DSTT
beschrieben wird.

\section{Kosmologische und astrophysikalische Konsequenzen der emergenten DSTT}
Dieses Kapitel untersucht die Konsequenzen, die sich aus der Identifikation der Dynamischen Schwere--Trägheits-Theorie (DSTT) als makroskopischer Gravitationssektor der
Weber--De-Broglie--Bohm-Theorie (WDBT/IWT) ergeben. Da die DSTT aus der diskreten Informationsdynamik der IWT hervorgeht, besitzt sie eine Reihe von Vorhersagen, die sich
deutlich von jenen der Allgemeinen Relativitätstheorie und der Newtonschen Mechanik unterscheiden. Die wichtigsten Konsequenzen betreffen die Struktur von Planetensystemen,
die Dynamik von Galaxien, die kosmologische Rotverschiebung und die großskalige Entwicklung des Universums.

\section*{1. Universelle Spiralbahnen und planetare Migration}
Die DSTT sagt voraus, dass alle Satelliten spiralförmige Bahnen besitzen:
\[
r(\phi) = K e^{\phi/B}.
\]
Da der Umlenkungszustand $\beta(t)$ monoton wächst, ergibt sich eine systematische Drift:
\[
\dot r < 0 \quad \text{für} \quad \beta < \frac12,
\]

\[
\dot r > 0 \quad \text{für} \quad \beta > \frac12.
\]
Damit folgt unmittelbar:

1. Planetensysteme sind dynamisch und nicht statisch.
2. Alle Planeten spiralisieren langfristig nach außen.
3. Einwärtsmigration ist nur in frühen Phasen möglich.
4. Die Existenz von Hot Jupiters ist eine natürliche Konsequenz.

Diese Aussagen stimmen mit Beobachtungen überein, ohne dass dissipative Mechanismen wie Gezeiten oder Scheibenreibung benötigt werden.

\section*{2. Galaktische Rotationskurven}
In der WDBT/IWT entsteht das Bohm-Potential als globaler Organisationsanteil der Informationsdynamik. Im makroskopischen Grenzfall führt dies zu einer effektiven
Umlenkungsenergie $E_B$, die nicht lokalisiert ist. Da $\beta$ mit wachsendem Radius zunimmt, ergibt sich eine asymptotisch konstante Umlaufgeschwindigkeit:
\[
v(r) \to \sqrt{\frac{2\beta E}{m_T}}.
\]
Dies reproduziert flache Rotationskurven ohne dunkle Materie. Die DSTT liefert damit eine natürliche Erklärung für galaktische Dynamik, die direkt aus der globalen
Informationsstruktur der IWT folgt.

\section*{3. Kosmologische Rotverschiebung ohne Expansion}
Die IWT beschreibt die Raumstruktur als fraktal mit effektiver Dimension $D \approx 2.71$. In einem solchen Raum entsteht eine intrinsische Rotverschiebung durch
Informationsverdünnung entlang der Lichtwege. Die DSTT liefert die makroskopische Form dieser Entwicklung durch die Driftgleichung
\[
\dot r = \frac{F_G}{2E}\sqrt{\frac{2\beta E}{m_T}}.
\]
Für Photonen ergibt sich eine effektive Energieabnahme entlang der Bahn, die proportional zur zurückgelegten Strecke ist. Damit folgt eine Hubble-Relation ohne kosmische
Expansion:
\[
z \propto r.
\]
Dies erklärt die beobachtete Rotverschiebung ohne Urknall und ohne expandierende Raumzeit.

\section*{4. Lichtablenkung und Shapiro-Effekt}
Die WDBT/IWT erzeugt ein effektives Potential $V+Q$, dessen Gradient die Bohm-Führungsgleichung bestimmt. Im makroskopischen Grenzfall führt dies zu einer Ablenkung von
Lichtstrahlen, die in der DSTT durch die Umlenkungsenergie $E_B$ beschrieben wird. Die resultierende Ablenkung ist wellenlängenabhängig, da der globale Organisationsanteil
$\beta$ für Photonen frequenzabhängig ist. Dies unterscheidet die Theorie von der ART, die eine wellenlängenunabhängige Ablenkung vorhersagt.

Der Shapiro-Effekt ergibt sich aus der Verzögerung der radialen Komponente:
\[
\dot r = \sqrt{\frac{2(1-\beta)E}{m_T}}.
\]
Für Photonen ist $m_T$ durch die Informationsstruktur bestimmt, was zu einer messbaren Frequenzabhängigkeit führt.

\section*{5. Langzeitentwicklung von Planetensystemen}

Da $\beta(t)$ monoton wächst und nie den Wert $1$ erreicht, ergibt sich eine universelle Drift nach außen. Die charakteristische Zeitskala lautet
\[
\tau \sim \frac{r}{|\dot r|}.
\]
Damit folgt:

1. Planetensysteme expandieren langsam.
2. Resonanzen verschieben sich im Laufe der Zeit.
3. Monde entfernen sich langfristig von ihren Planeten.
4. Einwärtsmigration ist nur in frühen Phasen möglich.

Diese Aussagen stimmen mit der beobachteten Drift des Erdmondes und dem Einwärtsfall von Phobos überein.

\section*{6. Konsequenzen für die Kosmologie}
Die Kombination aus fraktaler Raumstruktur der IWT und der Driftgleichung der DSTT führt zu einem statischen, aber dynamisch strukturierten Universum. Die wichtigsten Konsequenzen sind:

1. Keine kosmische Expansion notwendig.
2. Keine dunkle Energie.
3. Keine Singularitäten.
4. Rotverschiebung als Informationsdynamik.
5. Galaxienrotation ohne dunkle Materie.
6. Strukturentstehung durch globale Organisation statt Inflation.

Die DSTT liefert damit die makroskopische Gravitation der WDBT/IWT, während die fraktale Informationsstruktur die großskalige Kosmologie bestimmt.

\section*{7. Zusammenfassung}
Die Identifikation der DSTT als emergenter Sektor der WDBT/IWT führt zu einem konsistenten physikalischen Weltbild, das ohne dunkle Materie, ohne dunkle Energie, ohne
Urknall und ohne Raumzeitkrümmung auskommt. Die Theorie erklärt planetare Migration, galaktische Rotationskurven, kosmologische Rotverschiebung und Lichtablenkung durch
ein einziges Prinzip: die dynamische Aufteilung der Energie in lokale und globale Anteile, beschrieben durch den Zustand $\beta(t)$. Die DSTT ist damit die makroskopische
Gravitation einer umfassenden Informationsdynamik, die in der WDBT/IWT formuliert ist.

\section{Die fraktale Raumstruktur und ihre Rolle in der Gravitation}
Dieses Kapitel untersucht die Rolle der fraktalen Raumstruktur innerhalb der Informations-Weber-Theorie (IWT) und zeigt, wie diese Struktur die makroskopische Gravitation
der Dynamischen Schwere--Trägheits-Theorie (DSTT) bestimmt. Die fraktale Dimension des Informationsraumes, die in der IWT als fundamentale Eigenschaft des Universums
eingeführt wird, beeinflusst sowohl die emergente Geometrie als auch die Dynamik von Teilchen und Feldern. Die DSTT erweist sich als makroskopische Projektion dieser
fraktalen Informationsstruktur.

\section*{1. Fraktale Dimension als fundamentale Eigenschaft des Informationsraumes}
Die IWT postuliert, dass der physikalische Raum nicht fundamental ist, sondern aus einem diskreten Informationsnetzwerk emergiert. Die effektive Dimension dieses Netzwerks
ist fraktal und beträgt
\[
D \approx 2.71.
\]
Diese Dimension ergibt sich aus der Skalierungsstruktur der Informationskopplungen und bestimmt die emergente Geometrie. Der physikalische Raum ist daher kein
dreidimensionales Kontinuum, sondern eine kontinuierliche Näherung einer fraktalen Informationsstruktur.

\section*{2. Emergenz der Geometrie aus der Informationsmetrik}

Die Informationsmetrik der IWT definiert ein diskretes Linienelement
\[
ds^2 = g_{kl}^{(n)} \Delta I_k \Delta I_l.
\]
Im makroskopischen Grenzfall entsteht daraus eine effektive kontinuierliche Metrik. Die fraktale Struktur führt zu einer skalenabhängigen Geometrie, in der die effektive
Dimension mit der betrachteten Skala variiert. Für makroskopische Skalen ergibt sich eine nahezu dreidimensionale Struktur, während auf kleineren Skalen die fraktale Natur
dominiert.

\section*{3. Einfluss der fraktalen Struktur auf die Gravitation}
Die fraktale Informationsstruktur beeinflusst die Gravitation auf zwei Arten:

1. Sie bestimmt die Form des effektiven Potentials.
2. Sie beeinflusst die Energieaufteilung zwischen radialer und tangentialer Bewegung.

Das effektive Potential ergibt sich aus der Informationsmetrik und besitzt im makroskopischen Grenzfall die Form
\[
\Phi(r) = -\frac{GM_{\text{eff}}}{r}.
\]
Die fraktale Struktur führt jedoch zu Korrekturen, die sich in der globalen Organisationsenergie $E_B$ widerspiegeln. Diese Korrekturen sind für galaktische Skalen
relevant und erklären flache Rotationskurven ohne dunkle Materie.

\section*{4. Zusammenhang zwischen fraktaler Struktur und Umlenkungszustand}
Der Umlenkungszustand der DSTT,
\[
\beta = \frac{E_B}{E},
\]
ist eine direkte Konsequenz der fraktalen Informationsstruktur. Die globale Organisationsenergie $E_B$ entsteht aus der nichtlokalen Kopplung des Informationsnetzes. Da die
fraktale Struktur eine skalenabhängige Kopplung erzeugt, wächst $\beta$ mit zunehmendem Radius. Dies erklärt die universelle Drift nach außen und die Stabilisierung
galaktischer Rotationsgeschwindigkeiten.

\section*{5. Fraktale Raumstruktur und kosmologische Rotverschiebung}
Die fraktale Informationsstruktur führt zu einer intrinsischen Rotverschiebung, die nicht auf kosmische Expansion zurückzuführen ist. Die Energie eines Photons nimmt
entlang seines Weges durch das fraktale Informationsnetz ab. Die DSTT beschreibt diese Energieabnahme durch die Driftgleichung
\[
\dot r = \frac{F_G}{2E}\sqrt{\frac{2\beta E}{m_T}}.
\]
Für Photonen ergibt sich eine lineare Rotverschiebungsrelation
\[
z \propto r,
\]
die der beobachteten Hubble-Relation entspricht, ohne dass eine expandierende Raumzeit benötigt wird.

\section*{6. Fraktale Struktur und Lichtablenkung}
Die fraktale Informationsmetrik beeinflusst die Bohm-artige Umlenkungsenergie $E_B$ und damit die Bahn von Photonen. Die resultierende Lichtablenkung ist
wellenlängenabhängig, da die globale Organisationsstruktur für verschiedene Frequenzen unterschiedlich stark wirkt. Dies unterscheidet die Theorie von der Allgemeinen
Relativitätstheorie, die eine wellenlängenunabhängige Ablenkung vorhersagt.

\section*{7. Konsequenzen für die großskalige Struktur des Universums}
Die fraktale Raumstruktur führt zu einem statischen, aber dynamisch organisierten Universum. Die wichtigsten Konsequenzen sind:

1. Keine kosmische Expansion notwendig.
2. Rotverschiebung als Informationsdynamik.
3. Galaxienrotation ohne dunkle Materie.
4. CMB-Anisotropien als fossilierte Informationsstruktur.
5. Strukturentstehung ohne Inflation.

Die DSTT liefert die makroskopische Gravitation, während die fraktale Informationsstruktur die großskalige Geometrie bestimmt.

\section*{8. Zusammenfassung}
Die fraktale Raumstruktur der IWT ist der Ursprung der makroskopischen Gravitation der DSTT. Die Energieaufteilung in lokale und globale Anteile, beschrieben durch den
Zustand $\beta(t)$, ist eine direkte Konsequenz der fraktalen Informationskopplung. Die DSTT ist damit nicht nur eine emergente Gravitationstheorie, sondern die
makroskopische Projektion einer tieferliegenden fraktalen Informationsgeometrie, die das gesamte Universum strukturiert.
