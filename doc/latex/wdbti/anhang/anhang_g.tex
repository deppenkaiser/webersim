\chapter{Axiomatische Grundstruktur der Dynamischen Schwere--Trägheits-Theorie}
In diesem Kapitel wird die Theorie der dynamischen Aufteilung der Schwerewirkung in radiale und tangentiale Trägheitsantworten in axiomatischer Form dargestellt. Die
Theorie ersetzt die klassische Orbitmechanik durch eine energetisch gesteuerte Spiralmechanik, in der der dimensionslose Umlenkungszustand $\beta(t)$ die gesamte Dynamik
bestimmt. Die Axiome sind logisch unabhängig und vollständig.

\section*{Axiom 1: Schwere als einzige Ursache}
Zwischen zwei Körpern existiert eine radiale Schwerewirkung $F_G(r)$, die ausschließlich vom Abstand $r$ abhängt. Es existieren keine weiteren Kräfte und keine externen
Felder. Die Schwere ist die einzige Ursache der Bewegung.

\section*{Axiom 2: Trägheit als einzige Wirkung}
Die Schwerewirkung erzeugt zwei orthogonale Trägheitsantworten: eine radiale Beschleunigung $a_Z$ und eine tangentiale Beschleunigung $a_B$. Diese beiden Beschleunigungen
beschreiben die vollständige Bewegung des Testkörpers.

\section*{Axiom 3: Aufteilung der Schwerewirkung}
Die Schwerewirkung teilt sich in einen radialen Anteil $(1-\beta)F_G$ und einen tangentialen Anteil $\beta F_G$. Der Zustand $\beta$ ist eine dynamische Größe im Intervall
$[0,1]$ und bestimmt das Verhältnis von Fall und Umlenkung.

\section*{Axiom 4: Trägheitsmasse}
Die Trägheitsantworten koppeln an eine effektive Masse $m_T = m + \alpha M$. Die Bewegungsgleichungen lauten
\[
m_T a_Z = (1-\beta)F_G, \qquad
m_T a_B = \beta F_G.
\]
Damit hängt die Trägheit des Testkörpers vom Zentralkörper ab.

\section*{Axiom 5: Energiezerlegung}
Die Gesamtenergie zerfällt eindeutig in radiale und tangentiale Energie,
\[
E_Z = \frac12 m_T \dot r^2, \qquad
E_B = \frac12 m_T r^2 \dot\phi^2,
\]
und
\[
E = E_Z + E_B.
\]
Der Umlenkungszustand ist energetisch definiert durch
\[
\beta = \frac{E_B}{E}.
\]

\section*{Axiom 6: Dynamik des Umlenkungszustands}
Die zeitliche Entwicklung von $\beta$ folgt aus den Energieflüssen
\[
\dot E_B = F_G v_B, \qquad
\dot E_Z = -\beta F_G v_Z.
\]
Damit ergibt sich die autonome Evolutionsgleichung
\[
\dot\beta = \frac{F_G}{E^2}\left(E_Z v_B + \beta E_B v_Z\right).
\]
Der Zustand $\beta(t)$ ist monoton wachsend und erreicht den Wert $1$ nicht.

\section*{Axiom 7: Universelle Bahnform}
Die Bewegungsgleichungen
\[
\dot r = \sqrt{\frac{2(1-\beta)E}{m_T}}, \qquad
\dot\phi = \sqrt{\frac{2\beta E}{m_T}}\frac{1}{r},
\]
erzeugen nach Eliminierung der Zeit die universelle Bahnform
\[
r(\phi) = K e^{\phi/B},
\]
wobei
\[
B = \frac{2E}{F_G}.
\]
Die Bahn ist stets eine logarithmische Spirale. Kreisbahnen und Ellipsen sind ausgeschlossen.

\section*{Folgerungen}
Aus den Axiomen folgt, dass alle Satelliten spiralförmige Bahnen besitzen. Der Zustand $\beta(t)$ bestimmt die Richtung der Drift: Für $\beta < 1/2$ erfolgt eine
Einwärtsbewegung, für $\beta > 1/2$ eine Auswärtsbewegung. Langfristig spiralisieren alle Satelliten nach außen, da $\beta(t)$ monoton wächst. Die Theorie erklärt
Wanderbewegungen von Planeten und Monden ohne Zusatzannahmen.

