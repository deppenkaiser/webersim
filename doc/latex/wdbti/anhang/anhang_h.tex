\chapter{Emergente Maxwell-Gravitation}

\section{Einleitung}
Die \gls{iwt} beschreibt physikalische Realität nicht durch Felder, sondern durch diskrete Informationskopplungen, deren lokale und globale Anteile eine organisierte Dynamik erzeugen. Obwohl die fundamentale Theorie keine Felder kennt, können im makroskopischen Grenzfall kontinuierliche Strukturen entstehen, die formal den Maxwell-Gleichungen entsprechen. Dieses Kapitel zeigt, wie aus der \gls{iwt} und ihrer effektiven Formulierung in der \gls{wdbt} eine Maxwell-artige Gravitationsdynamik emergiert, deren makroskopischer, rotationssymmetrischer Grenzfall durch die \gls{dstt} beschrieben wird.

\section{Diskrete Informationsdynamik und Kontinuumslimit}
Die \gls{iwt} basiert auf diskreten Informationsfeldern $I_k$ und Kopplungen $K_{ij}$, die lokale und globale Beiträge besitzen. Im Kontinuumslimit entsteht aus diesen Kopplungen eine effektive Geometrie, deren Gradienten- und Wirbelstrukturen den Differentialoperatoren $\nabla \cdot$ und $\nabla \times$ entsprechen. Jede gerichtete Kopplungsstruktur erzeugt im Grenzfall Quellen- und Wirbelterme, unabhängig davon, ob die fundamentale Theorie Felder postuliert.

Damit folgt:
\begin{itemize}
\item lokale Kopplungen $\rightarrow$ divergente Strukturen,
\item globale Kopplungen $\rightarrow$ rotierende Strukturen,
\item Energieflüsse $\rightarrow$ Poynting-artige Größen.
\end{itemize}
Diese mathematische Struktur ist universell und unabhängig von der physikalischen Interpretation.

\section{Lokale und globale Dynamik in der \gls{iwt}}
Die \gls{iwt} besitzt eine natürliche Zerlegung des Lagrange-Funktionals in lokale und globale Anteile:
\[
L = L_{\mathrm{lokal}} + L_{\mathrm{global}}.
\]
Der relative Anteil globaler Organisation ist definiert durch
\[
\beta = \frac{L_{\mathrm{global}}}{L_{\mathrm{lokal}} + L_{\mathrm{global}}}.
\]
Das Variationsprinzip der \gls{iwt} erzwingt eine monotone Zunahme dieses Anteils:
\[
\dot{\beta} > 0.
\]
Damit existiert ein intrinsischer Ordnungsparameter, der die Dynamik strukturiert und einen irreversiblen Pfeil der Zeit definiert.

\section{Emergente gravito-elektrische Struktur}
Im makroskopischen Grenzfall entsteht aus der effektiven \gls{iwt}-Metrik ein Potential
\[
\Phi_{\mathrm{eff}}(r) = -\frac{G M_{\mathrm{eff}}}{r} + Q_{\mathrm{\gls{iwt}}}(r),
\]
wobei $Q_{\mathrm{\gls{iwt}}}$ der globale Organisationsanteil ist, der im \gls{wdbt}-Grenzfall als Bohm-Potential erscheint.  
Das effektive gravito-elektrische Feld wird definiert durch
\[
\mathbf{E}_g = -\nabla \Phi_{\mathrm{eff}}.
\]
Die Divergenz dieses Feldes ergibt im statischen, rotationssymmetrischen Fall
\[
\nabla \cdot \mathbf{E}_g = -4\pi G \rho_{\mathrm{eff}},
\]
wobei $\rho_{\mathrm{eff}}$ die effektive Massen- und Organisationsdichte ist. Diese Gleichung ist formal identisch zur elektrischen Poisson-Gleichung, jedoch emergent und nicht fundamental.

\section{Emergente gravito-magnetische Struktur}
Im Kontinuumslimit der \gls{iwt} entspricht die tangentiale Trägheitsantwort einer rotationsartigen Bewegung. Definiert man das Geschwindigkeitsfeld $\mathbf{v} = \dot{r}\hat{r} + r\dot{\phi}\hat{\phi}$, so kann man ein effektives Vektorpotential
\[
\mathbf{A}_g = \frac{\beta}{c_g} \mathbf{v}
\]
einführen, wobei $c_g$ eine charakteristische Geschwindigkeit aus der IWT-Metrik ist. Das zugehörige gravito-magnetische Feld ist
\[
\mathbf{B}_g = \nabla \times \mathbf{A}_g.
\]
Aus der \gls{dstt}-Bewegungsgleichung $m_T r\dot{\phi}^2 = \beta F_G$ folgt, dass im stationären Fall $\mathbf{B}_g$ proportional zur Umlenkungsenergiedichte ist.  
Für zeitlich veränderliche Felder ergibt sich aus der Struktur der IWT die Induktionsgleichung
\[
\nabla \times \mathbf{E}_g = -\frac{\partial \mathbf{B}_g}{\partial t}.
\]

\section{Energiefluss und Poynting-Analogon}
Aus den \gls{dstt}-Energieflüssen $\dot{E}_B = F_G v_B$ und $\dot{E}_Z = -\beta F_G v_Z$ lässt sich im Kontinuumslimit eine Energieerhaltungsgleichung
\[
\frac{\partial u_g}{\partial t} + \nabla \cdot \mathbf{S}_g = 0
\]
herleiten, wobei $u_g = \tfrac12 \epsilon_g E_g^2 + \tfrac12 \mu_g^{-1} B_g^2$ die effektive Energiedichte und
\[
\mathbf{S}_g = \frac{1}{\mu_g} \mathbf{E}_g \times \mathbf{B}_g
\]
der gravitative Poynting-Vektor ist. Die Konstanten $\epsilon_g$ und $\mu_g$ sind durch die IWT-Metrik bestimmt und erfüllen $\epsilon_g \mu_g = c_g^{-2}$.  
Im makroskopischen, stationären Grenzfall reduziert sich dies auf den \gls{dstt}-Ausdruck $\mathbf{S}_g \approx E_B \mathbf{v}_\perp$.

\section{Emergente Maxwell-Gleichungen der Gravitation}
Die Struktur der \gls{iwt} führt im Kontinuumslimit zu effektiven Feldgleichungen, die formal den Maxwell-Gleichungen entsprechen:
\[
\begin{aligned}
\nabla \cdot \mathbf{E}_g &= -4\pi G \rho_{\mathrm{eff}}, \\
\nabla \times \mathbf{E}_g &= -\frac{\partial \mathbf{B}_g}{\partial t}, \\
\nabla \cdot \mathbf{B}_g &= 0, \\
\nabla \times \mathbf{B}_g &= \mu_g \mathbf{J}_g + \epsilon_g \frac{\partial \mathbf{E}_g}{\partial t},
\end{aligned}
\]
wobei $\mathbf{J}_g = \rho_{\mathrm{eff}} \mathbf{v}$ der effektive Massenstrom ist. Diese Gleichungen sind **emergent**; sie beschreiben die makroskopische Gravitation als effektive Feldtheorie, die aus der diskreten Informationsdynamik der \gls{iwt} hervorgeht.

\section{Makroskopischer Grenzfall: Die \gls{dstt}}
Die \gls{dstt} ist der makroskopische, rotationssymmetrische und stationäre Grenzfall der emergenten Maxwell-Gravitation. Im Zentralfeld vereinfachen sich die Feldgleichungen zu den \gls{dstt}-Bewegungsgleichungen
\[
m_T \ddot{r} = (1-\beta)F_G, \qquad m_T r\dot{\phi}^2 = \beta F_G.
\]
Die zugehörige Bahnform
\[
r(\varphi) = K e^{\varphi/B}
\]
ist die direkte Lösung dieser Gleichungen und folgt aus dem Wirbelcharakter des gravito-magnetischen Feldes $\mathbf{B}_g$ im stationären Fall.  
Damit ist die logarithmische Spirale keine separate Annahme, sondern die notwendige Konsequenz einer Rotation, die durch den wachsenden Ordnungszustand $\beta(t)$ gesteuert wird.

\section{Schlussfolgerung}
Obwohl die \gls{iwt} keine Felder postuliert, entstehen im makroskopischen Grenzfall Maxwell-artige Strukturen der Gravitation. Diese emergenten Felder sind keine fundamentalen Entitäten, sondern Kontinuumsnäherungen der diskreten Informationsdynamik. Die \gls{dstt} ist die makroskopische, rotationssymmetrische Manifestation dieser Struktur und liefert die universelle Spiralform gravitativer Bahnen. Die emergente Maxwell-Gravitation verbindet damit die \gls{iwt}, die \gls{wdbt} und die \gls{dstt} zu einem konsistenten, hierarchisch organisierten Gesamtbild der Gravitation.