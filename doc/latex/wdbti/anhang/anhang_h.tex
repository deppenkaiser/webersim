\chapter{Emergente Maxwell-Gravitation}

\section{Einleitung}
Die \gls{iwt} beschreibt physikalische Realität nicht durch Felder, sondern durch diskrete Informationskopplungen, deren lokale und globale Anteile
eine organisierte Dynamik erzeugen. Obwohl die fundamentale Theorie keine Felder kennt, können im makroskopischen Grenzfall kontinuierliche Strukturen entstehen, die formal
den Maxwell-Gleichungen ähneln. Dieses Kapitel zeigt, wie aus der \gls{iwt} und ihrer effektiven Formulierung in der \gls{wdbt} eine Maxwell-ähnliche
Gravitationsdynamik emergiert, deren makroskopischer Gravitationssektor durch die \gls{dstt} beschrieben wird.

\section{Diskrete Informationsdynamik und Kontinuumslimit}
Die \gls{iwt} basiert auf diskreten Informationsfeldern $I_k$ und Kopplungen $K_{ij}$, die lokale und globale Beiträge besitzen. Im Kontinuumslimit entsteht aus diesen Kopplungen
eine effektive Geometrie, deren Gradienten- und Wirbelstrukturen den Differentialoperatoren $\nabla \cdot$ und $\nabla \times$ entsprechen. Jede gerichtete Kopplungsstruktur
erzeugt im Grenzfall Quellen- und Wirbelterme, unabhängig davon, ob die fundamentale Theorie Felder postuliert.

Damit folgt:
\begin{itemize}
\item lokale Kopplungen $\rightarrow$ divergente Strukturen,
\item globale Kopplungen $\rightarrow$ rotierende Strukturen,
\item Energieflüsse $\rightarrow$ Poynting-ähnliche Größen.
\end{itemize}
Diese mathematische Struktur ist universell und unabhängig von der physikalischen Interpretation.

\section{Lokale und globale Dynamik in der \gls{iwt}}
Die \gls{iwt} besitzt eine natürliche Zerlegung des Lagrange-Funktionals in lokale und globale Anteile:
\[
L = L_{\mathrm{lokal}} + L_{\mathrm{global}}.
\]
Der relative Anteil globaler Organisation ist definiert durch
\[
\beta = \frac{L_{\mathrm{global}}}{L_{\mathrm{lokal}} + L_{\mathrm{global}}}.
\]
Das Variationsprinzip der \gls{iwt} erzwingt eine monotone Zunahme dieses Anteils:
\[
\dot{\beta} > 0.
\]
Damit existiert ein intrinsischer Ordnungsparameter, der die Dynamik strukturiert und einen irreversiblen Pfeil der Zeit definiert.

\section{Emergente gravito-elektrische Struktur}
Im makroskopischen Grenzfall entsteht aus der effektiven \gls{iwt}-Metrik ein Potential
\[
\Phi_{\mathrm{eff}}(r) = -\frac{G M_{\mathrm{eff}}}{r} + Q_{\mathrm{\gls{iwt}}}(r),
\]
wobei $Q_{\mathrm{\gls{iwt}}}$ der globale Organisationsanteil ist, der im \gls{wdbt}-Grenzfall als Bohm-Potential erscheint.

Das effektive gravito-elektrische Feld wird definiert durch
\[
\mathbf{E}_g = -\nabla \Phi_{\mathrm{eff}}.
\]
Die Divergenz dieses Feldes ergibt
\[
\nabla \cdot \mathbf{E}_g = -4\pi G \rho_{\mathrm{eff}},
\]
wobei $\rho_{\mathrm{eff}}$ die effektive Massen- und Organisationsdichte ist. Diese Gleichung ist formal identisch zur elektrischen Poisson-Gleichung, jedoch emergent und
nicht fundamental.

\section{Emergente gravito-magnetische Struktur}
Die \gls{dstt} beschreibt die Schwerewirkung durch zwei orthogonale Trägheitsantworten:
\[
a_z = (1-\beta)\frac{F_G}{m_T}, \qquad
a_B = \beta \frac{F_G}{m_T}.
\]
Die tangentiale Umlenkung $a_B$ besitzt Wirbelcharakter. Im Kontinuumslimit entsteht daraus ein effektives gravito-magnetisches Feld
\[
\mathbf{B}_g = \nabla \times \mathbf{A}_g,
\]
wobei das Potential proportional zur Umlenkungsenergie ist:
\[
\mathbf{A}_g \propto \beta\, \mathbf{v}.
\]
Damit folgt die rot-Gleichung
\[
\nabla \times \mathbf{E}_g = -\frac{\partial \mathbf{B}_g}{\partial t}.
\]

\section{Energiefluss und Poynting-Analogon}
Die \gls{dstt} besitzt eine eindeutige Energiezerlegung
\[
E = E_z + E_B,
\]
mit
\[
E_z = (1-\beta)E, \qquad E_B = \beta E.
\]
Die Energieflüsse lauten
\[
\dot{E}_B = F_G v_B, \qquad \dot{E}_z = -\beta F_G v_z.
\]
Daraus entsteht ein effektiver Energieflussvektor
\[
\mathbf{S}_g = E_B\, \mathbf{v}_\perp,
\]
der formal dem Poynting-Vektor der Elektrodynamik entspricht.

\section{Emergente Maxwell-Gleichungen der Gravitation}
Aus den vorherigen Abschnitten ergibt sich ein vollständiges Maxwell-Analogon der Gravitation:
\[
\begin{aligned}
\nabla \cdot \mathbf{E}_g &= -4\pi G \rho_{\mathrm{eff}}, \
\nabla \times \mathbf{E}_g &= -\frac{\partial \mathbf{B}_g}{\partial t}, \
\nabla \cdot \mathbf{B}_g &= 0, \
\nabla \times \mathbf{B}_g &= \mu_g \mathbf{J}_g + \epsilon_g \frac{\partial \mathbf{E}_g}{\partial t}.
\end{aligned}
\]
Dabei sind
\[
\mathbf{J}_g = \rho_{\mathrm{eff}}\, \mathbf{v}
\]
der Massenstrom und $\mu_g, \epsilon_g$ emergente Konstanten, die aus der \gls{iwt}-Metrik resultieren.

\section{Makroskopischer Grenzfall: Die \gls{dstt}}
Die \gls{dstt} ist der makroskopische Gravitationssektor der \gls{iwt}. Ihre Bahnform
\[
r(\varphi) = K e^{\varphi/B}
\]
ist die direkte Projektion des Wirbelprinzips:
\[
\dot{\varphi} = \omega, \qquad \dot{r} = \alpha(\beta) r.
\]
Damit ist die logarithmische Spirale keine Annahme, sondern die notwendige Konsequenz einer Rotation, die durch einen wachsenden Ordnungszustand beeinflusst wird.

\section{Schlussfolgerung}
Obwohl die \gls{iwt} keine Felder postuliert, entstehen im makroskopischen Grenzfall Maxwell-ähnliche Strukturen der Gravitation. Diese emergenten Felder sind keine
fundamentalen Entitäten, sondern Kontinuumsnäherungen der diskreten Informationsdynamik. Die \gls{dstt} ist die makroskopische Manifestation dieser Struktur und
liefert die universelle Spiralform gravitativer Bahnen. Die emergente Maxwell-Gravitation verbindet damit die \gls{iwt}, die \gls{wdbt} und die \gls{dstt} zu einem konsistenten, hierarchisch
organisierten Gesamtbild der Gravitation.
