\chapter{Mathematische Grundlagen der Informations-Weber-Theorie}
\label{anhang:mathematik}

\section{Variationsrechnung}
Die Informations-Weber-Theorie basiert auf einem Lagrange-Funktional
\[
    \mathcal{L}_I[\rho_I]
    =
    \int \mathcal{F}(\rho_I, \nabla \rho_I, \partial_t \rho_I)\, d^3x.
\]
Die Variation eines Funktionals der Form
\[
    \mathcal{S}[\rho_I] = \int \mathcal{F}(\rho_I, \partial_\mu \rho_I)\, d^4x
\]
ergibt
\[
    \delta \mathcal{S}
    =
    \int
    \left(
        \frac{\partial \mathcal{F}}{\partial \rho_I}
        -
        \partial_\mu
        \frac{\partial \mathcal{F}}{\partial (\partial_\mu \rho_I)}
    \right)
    \delta \rho_I
    \, d^4x.
\]
Der Ausdruck in Klammern verschwindet für stationäre Punkte des Funktionals.

\section{Euler-Lagrange-Gleichungen für Informationsfelder}
Für ein Informationsfeld $\rho_I(\vec{r},t)$ ergibt sich die Euler-Lagrange-Gleichung:
\[
    \frac{\partial}{\partial t}
    \left(
        \frac{\partial \mathcal{F}}{\partial (\partial_t \rho_I)}
    \right)
    +
    \nabla \cdot
    \left(
        \frac{\partial \mathcal{F}}{\partial (\nabla \rho_I)}
    \right)
    -
    \frac{\partial \mathcal{F}}{\partial \rho_I}
    = 0.
\]
Diese Gleichung ist die Grundlage der Informations-Weber-Dynamik.

\section{Noether-Theorem im Informationsraum}
Symmetrien des Informationsraums erzeugen Erhaltungsgrößen:

\begin{itemize}
    \item Zeitsymmetrie $\Rightarrow$ Energieerhaltung,
    \item Translationssymmetrie $\Rightarrow$ Impulserhaltung,
    \item Rotationssymmetrie $\Rightarrow$ Drehimpulserhaltung,
    \item Informationsinvarianz $\Rightarrow$ Erhaltung der Gesamtinformation.
\end{itemize}

\section{Informationsmetriken und fraktale Dimension}

Die informationsbasierte Metrik ergibt sich aus
\[
    g_{ij}
    =
    \frac{\partial^2 \mathcal{F}}{\partial (\partial_i \rho_I)\, \partial (\partial_j \rho_I)}.
\]
Die fraktale Dimension des Informationsraums ist definiert durch
\[
    D = \frac{\ln 20}{\ln(2+\phi)}.
\]
Sie bestimmt die effektive Geometrie des emergenten Raumes.
