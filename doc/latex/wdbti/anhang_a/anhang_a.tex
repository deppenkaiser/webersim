\chapter{Mathematische Grundlagen der Informations-Weber-Theorie}
\label{app:mathematik}

\paragraph{Hinweis zur mathematischen Darstellung}
Dieses Kapitel verwendet größtenteils die \emph{kontinuierliche Notation} für Kompaktheit. Die zugrundeliegende fundamentale Formulierung ist diskret rekursiv. Wo nötig
wird die diskrete Form explizit angegeben. Eine vollständige diskrete Darstellung findet sich in Kapitel X.

In diesem Anhang werden die mathematischen Werkzeuge zusammengestellt, auf denen die Informations-Weber-Theorie basiert. Ziel ist es, die verwendeten Methoden so
darzustellen, dass alle im Haupttext verwendeten Gleichungen nachvollzogen werden können, ohne auf externe Quellen angewiesen zu sein.

Der Schwerpunkt liegt auf:
\begin{itemize}
    \item der Variationsrechnung für kontinuierliche Informationsfelder,
    \item den Euler--Lagrange-Gleichungen im Informationsraum,
    \item dem Noether-Theorem und Erhaltungsgrößen,
    \item der Definition der Informationsmetrik und der fraktalen Dimension.
\end{itemize}

\section{Variationsrechnung für Informationsfunktionale}
\label{app:variation}
Die Informations-Weber-Theorie formuliert Dynamik über ein Lagrange-Funktional der Informationsdichte \(\rho_I(\vec{r},t)\). Wir beginnen daher mit der klassischen
Variationsrechnung für Funktionale vom Typ
\[
    S[\rho_I] = \int \mathcal{F}\big(\rho_I, \partial_\mu \rho_I\big)\, d^4x,
\]
wobei \(\partial_\mu\) mit \(\mu = 0,1,2,3\) für Zeit- und Raumableitungen steht.

\subsection{Allgemeine Formulierung}
Betrachte ein Funktional
\[
    S[\rho_I]
    =
    \int \mathcal{F}\big(\rho_I, \partial_\mu \rho_I\big)\, d^4x,
\]
wobei \(\mathcal{F}\) eine skalare Dichte ist, die von \(\rho_I\) und ihren Ableitungen abhängt.

Wir betrachten eine Variation
\[
    \rho_I \to \rho_I + \varepsilon\, \eta,
\]
wobei \(\eta(\vec{r},t)\) eine beliebige, glatte Testfunktion mit verschwindenden Randwerten sei und \(\varepsilon\) ein infinitesimaler Parameter.

Die Variation des Funktionals ist dann
\[
    \delta S
    =
    \left.\frac{d}{d\varepsilon} S[\rho_I + \varepsilon \eta]\right|_{\varepsilon=0}.
\]
Mit der Kettenregel erhält man
\[
    \delta S
    =
    \int
    \left(
        \frac{\partial \mathcal{F}}{\partial \rho_I}\, \delta \rho_I
        +
        \frac{\partial \mathcal{F}}{\partial (\partial_\mu \rho_I)}\, \delta(\partial_\mu \rho_I)
    \right)
    d^4x.
\]
Da \(\delta(\partial_\mu \rho_I) = \partial_\mu(\delta \rho_I)\), folgt
\[
    \delta S
    =
    \int
    \left(
        \frac{\partial \mathcal{F}}{\partial \rho_I}\, \delta \rho_I
        +
        \frac{\partial \mathcal{F}}{\partial (\partial_\mu \rho_I)}\, \partial_\mu(\delta \rho_I)
    \right)
    d^4x.
\]
Durch partielle Integration und unter der Annahme, dass Randterme verschwinden, erhält man
\[
    \delta S
    =
    \int
    \left[
        \frac{\partial \mathcal{F}}{\partial \rho_I}
        -
        \partial_\mu
        \left(
            \frac{\partial \mathcal{F}}{\partial (\partial_\mu \rho_I)}
        \right)
    \right]
    \delta \rho_I\, d^4x.
\]
Da \(\delta \rho_I\) beliebig ist, folgt die Bedingung für stationäre Punkte (\(\delta S = 0\)):
\[
    \frac{\partial \mathcal{F}}{\partial \rho_I}
    -
    \partial_\mu
    \left(
        \frac{\partial \mathcal{F}}{\partial (\partial_\mu \rho_I)}
    \right)
    = 0.
\]
Dies ist die Euler--Lagrange-Gleichung für das Informationsfeld \(\rho_I\).

\section{Euler--Lagrange-Gleichungen für Informationsfelder}
\label{app:euler_lagrange}

Für die Informations-Weber-Theorie schreiben wir das Lagrange-Funktional als
\[
    \mathcal{L}[\rho_I]
    =
    \int \mathcal{F}(\rho_I, \partial_t \rho_I, \nabla \rho_I)\, d^3x.
\]

\subsection{Zeitabhängiges Informationsfeld}
Wir betrachten
\[
    S[\rho_I]
    =
    \int dt \int d^3x\,
    \mathcal{F}(\rho_I, \partial_t \rho_I, \nabla \rho_I).
\]
Die Variation liefert
\[
    \frac{\partial \mathcal{F}}{\partial \rho_I}
    -
    \partial_t
    \left(
        \frac{\partial \mathcal{F}}{\partial (\partial_t \rho_I)}
    \right)
    -
    \nabla \cdot
    \left(
        \frac{\partial \mathcal{F}}{\partial (\nabla \rho_I)}
    \right)
    = 0.
\]
Dies ist die konkrete Form der Euler--Lagrange-Gleichung, die im Haupttext mehrfach verwendet wird.

\subsection{Beispiel: Lokaler Anteil des Informationsfunktionals}
Nehmen wir einen lokalen Anteil der Form
\[
    \mathcal{F}_{\text{lokal}}
    =
    \alpha\, (\partial_t \rho_I)^2
    +
    \beta\, (\nabla \rho_I)^2.
\]
Dann sind
\[
    \frac{\partial \mathcal{F}_{\text{lokal}}}{\partial \rho_I}
    = 0,
    \qquad
    \frac{\partial \mathcal{F}_{\text{lokal}}}{\partial (\partial_t \rho_I)}
    = 2\alpha\, \partial_t \rho_I,
    \qquad
    \frac{\partial \mathcal{F}_{\text{lokal}}}{\partial (\nabla \rho_I)}
    = 2\beta\, \nabla \rho_I.
\]
Die Euler--Lagrange-Gleichung wird zu
\[
    - \partial_t (2\alpha\, \partial_t \rho_I)
    - \nabla \cdot (2\beta\, \nabla \rho_I)
    = 0,
\]
also
\[
    \alpha\, \partial_t^2 \rho_I
    +
    \beta\, \nabla^2 \rho_I
    = 0.
\]
Dies ist eine Wellengleichung für die Informationsdichte \(\rho_I\). Sie illustriert, wie aus dem lokalen Funktional eine dynamische Gleichung entsteht.

\section{Noether-Theorem im Informationsraum}
\label{app:noether}

Das Noether-Theorem verbindet Symmetrien eines Lagrange-Funktionals mit Erhaltungsgrößen. Im Informationsraum bedeutet dies: Symmetrien der Informationsdichte und ihres
Funktionals erzeugen Erhaltungssätze.

\subsection{Allgemeine Formulierung}
Betrachte eine kontinuierliche Transformation
\[
    \rho_I(\vec{r},t)
    \to
    \rho_I'(\vec{r},t)
    =
    \rho_I(\vec{r},t) + \varepsilon\, \Delta \rho_I(\vec{r},t),
\]
bei der sich das Funktional nur um einen Randterm ändert:
\[
    \delta \mathcal{F}
    =
    \varepsilon\, \partial_\mu K^\mu.
\]
Dann existiert eine erhaltene Größe \(J^\mu\) mit
\[
    \partial_\mu J^\mu = 0.
\]

\subsection{Beispiele für Symmetrien}
\begin{itemize}
    \item \textbf{Zeitsymmetrie:}  
    Invarianz unter \(t \to t + \text{const}\)  
    \(\Rightarrow\) Energieerhaltung als abgeleitetes Informationsmaß.

    \item \textbf{Translationssymmetrie im Raum:}  
    Invarianz unter \(\vec{r} \to \vec{r} + \text{const}\)  
    \(\Rightarrow\) Impulserhaltung.

    \item \textbf{Rotationssymmetrie:}  
    Invarianz unter \(\vec{r} \to R\vec{r}\)  
    \(\Rightarrow\) Drehimpulserhaltung.

    \item \textbf{Informationsinvarianz:}  
    Invarianz der Gesamtinformation \(\int \rho_I\, d^3x\)  
    \(\Rightarrow\) Erhaltung der Gesamtinformation, aus der Energieerhaltung als
    Spezialfall folgt.
\end{itemize}
Damit werden klassische Erhaltungssätze als Konsequenz der Symmetrien des Informationsraums verstanden.

\section{Informationsmetriken und fraktale Dimension}
\label{app:infomatrik}

Die Informationsmetrik beschreibt, wie empfindlich das Informationsfunktional auf räumliche Änderungen der Informationsdichte reagiert.

\subsection{Definition der Informationsmetrik}
Ausgehend von
\[
    \mathcal{F}
    =
    \mathcal{F}\big(\rho_I, \partial_i \rho_I\big)
\]
definieren wir die Informationsmetrik als
\[
    g_{ij}
    =
    \frac{\partial^2 \mathcal{F}}{\partial (\partial_i \rho_I)\, \partial (\partial_j \rho_I)}.
\]
Interpretation:
\begin{itemize}
    \item Große \(g_{ij}\): kleine Änderungen von \(\partial_i \rho_I\) haben große Wirkung auf
    die Dynamik \(\Rightarrow\) „steife“ Informationsgeometrie.

    \item Kleine \(g_{ij}\): die Informationsstruktur ist „weich“, Änderungen von
    \(\partial_i \rho_I\) haben geringe dynamische Konsequenzen.
\end{itemize}

\subsection{Fraktale Dimension als Skalierungssignatur}
Die fraktale Dimension des Informationsnetzes ist definiert durch
\[
    D
    =
    \frac{\ln 20}{\ln(2+\phi)}.
\]
Sie ist kein Maß für die topologische Raumdimension, sondern charakterisiert die Skalierung der Kopplungsstruktur im Informationsnetz.

Wichtige Eigenschaften:
\begin{itemize}
    \item Auf kleinen Skalen beschreibt \(D\) die Feinstruktur der Informationsverzweigungen.
    \item Für große Skalen gilt \(D \to 3\), sodass ein scheinbar dreidimensionaler Raum
    emergiert.
    \item Die Skalierungsrelationen für Naturkonstanten (Kapitel~\ref{chap:naturkonstanten})
    beruhen direkt auf \(D\).
\end{itemize}

\section{Zusammenfassung von Anhang A}
In diesem Anhang wurden die mathematischen Grundlagen der Informations-Weber-Theorie ausführlich dargestellt:
\begin{itemize}
    \item Die Variationsrechnung liefert die Euler-Lagrange-Gleichungen für die
    Informationsdichte \(\rho_I\).
    \item Das Noether-Theorem verbindet Symmetrien mit Erhaltungsgrößen im Informationsraum.
    \item Die Informationsmetrik entsteht aus der Sensitivität des Funktionals gegenüber
    Gradienten von \(\rho_I\).
    \item Die fraktale Dimension \(D\) beschreibt die Skalierung der Kopplungsstruktur
    und ist die Grundlage der emergenten Geometrie und der Naturkonstanten.
\end{itemize}
Diese Struktur erlaubt es, alle im Haupttext verwendeten Gleichungen systematisch nachzuvollziehen und bildet die mathematische Basis für die weiteren Anhänge.
