\chapter{Herleitungen}
\label{app:herleitungen}

\paragraph{Hinweis zur mathematischen Darstellung}
Dieses Kapitel verwendet größtenteils die \emph{kontinuierliche Notation} für Kompaktheit. Die zugrundeliegende fundamentale Formulierung ist diskret rekursiv. Wo nötig
wird die diskrete Form explizit angegeben. Eine vollständige diskrete Darstellung findet sich in Kapitel X.

In diesem Anhang werden die wichtigsten Gleichungen der Informations-Weber-Theorie vollständig und Schritt für Schritt hergeleitet. Ziel ist es, die mathematischen
Grundlagen transparent zu machen und die Verbindung zwischen lokaler Weber-Dynamik, globaler Informationsorganisation und emergenter Geometrie klar herauszuarbeiten.

Behandelt werden:
\begin{itemize}
    \item die Weber-Gravitation,
    \item das Bohm-Potential,
    \item die Kontinuitätsgleichung,
    \item die effektive Masse als Informationsmaß.
\end{itemize}
Alle Herleitungen basieren auf dem Informations-Lagrange-Funktional.

\section{Herleitung der Weber-Gravitation}
\label{app:webergrav}

Die \gls{wg} ist der gravitative Analogon der \gls{wed}. Sie beschreibt die direkte Wechselwirkung zweier Massen ohne Felder und enthält
geschwindigkeits- und beschleunigungsabhängige Korrekturen.

Die allgemeine Form der Weber-Gravitationskraft lautet:
\begin{equation}
F_{\text{WG}}
=
-\,G\frac{m_1 m_2}{r^2}
\left(
1
- \frac{\dot r^2}{c^2}
+ \beta\,\frac{r\ddot r}{c^2}
\right),
\label{eq:wg_allgemein}
\end{equation}
wobei der Parameter \(\beta\) die physikalische Situation charakterisiert:
\[
\beta =
\begin{cases}
0.5 & \text{für massive Körper (Planetenbahnen)} \\
1   & \text{für Photonen (Lichtablenkung, Shapiro-Effekt)} \\
2   & \text{für elektromagnetische Weber-Wechselwirkung}
\end{cases}
\]
Diese Struktur ist zentral für die \gls{wdbt}, da sie die korrekte Periheldrehung, Lichtablenkung und Rotverschiebung liefert.

\subsection{Ausgangspunkt: Variationsprinzip für direkte Wechselwirkungen}
In der Informations-Weber-Theorie wird die Weber-Gravitation nicht aus einer klassischen Lagrange-Funktion hergeleitet, sondern ergibt sich als lokaler Grenzfall des informationsbasierten Variationsprinzips. Das zugrunde liegende Prinzip ist die Minimierung der gesamten Wechselwirkungsinformation zwischen zwei Massen.

Wir betrachten das folgende Wechselwirkungsfunktional für zwei Massen $m_1$ und $m_2$:
\begin{equation}
\mathcal{I}[r, \dot{r}, \ddot{r}] 
= \int \left[ 
\frac{1}{2}\mu \dot{r}^2 
+ G\frac{m_1 m_2}{r} 
\left( 
1 
- \frac{\dot{r}^2}{2c^2} 
+ \frac{r\ddot{r}}{2c^2} 
\right) 
\right] dt,
\label{eq:wg_wirkung}
\end{equation}
wobei $\mu = m_1 m_2/(m_1 + m_2)$ die reduzierte Masse ist.

\subsection{Variation und Bewegungsgleichung}
Die Variation des Funktionals \eqref{eq:wg_wirkung} nach dem Bahnparameter $r(t)$ unter Berücksichtigung der Randterme führt auf:
\begin{equation}
\mu \ddot{r} - \frac{\partial}{\partial r}\left[ G\frac{m_1 m_2}{r} \left(1 - \frac{\dot{r}^2}{2c^2}\right) \right]
+ \frac{d}{dt}\left( \frac{\partial}{\partial \dot{r}}\left[ G\frac{m_1 m_2}{r} \left(1 - \frac{\dot{r}^2}{2c^2}\right) \right] \right)
+ \frac{\partial}{\partial \ddot{r}}\left[ G\frac{m_1 m_2}{r} \cdot \frac{r\ddot{r}}{2c^2} \right] = 0.
\end{equation}

Die explizite Berechnung dieser Terme ergibt:
\begin{align}
\frac{\partial}{\partial r}\left[ G\frac{m_1 m_2}{r} \left(1 - \frac{\dot{r}^2}{2c^2}\right) \right] 
&= -G\frac{m_1 m_2}{r^2} \left(1 - \frac{\dot{r}^2}{2c^2}\right), \\
\frac{d}{dt}\left( \frac{\partial}{\partial \dot{r}}\left[ G\frac{m_1 m_2}{r} \left(1 - \frac{\dot{r}^2}{2c^2}\right) \right] \right)
&= -G\frac{m_1 m_2}{c^2} \frac{d}{dt}\left( \frac{\dot{r}}{r} \right) \\
&= -G\frac{m_1 m_2}{c^2} \left( \frac{\ddot{r}}{r} - \frac{\dot{r}^2}{r^2} \right), \\
\frac{\partial}{\partial \ddot{r}}\left[ G\frac{m_1 m_2}{r} \cdot \frac{r\ddot{r}}{2c^2} \right]
&= G\frac{m_1 m_2}{2c^2}.
\end{align}

\subsection{Resultierende Kraftgleichung}
Das Einsetzen aller Terme in die Variationsgleichung und Umstellung nach der Kraft $F = \mu \ddot{r}$ liefert:
\begin{equation}
F = -G\frac{m_1 m_2}{r^2} \left( 1 - \frac{\dot{r}^2}{c^2} + \frac{r\ddot{r}}{2c^2} \right).
\end{equation}

Um die allgemeine Form \eqref{eq:wg_allgemein} zu erhalten, führen wir den Parameter $\beta$ ein, der verschiedene physikalische Regime beschreibt:
\begin{equation}
F_{\text{WG}} = -G\frac{m_1 m_2}{r^2} \left( 1 - \frac{\dot{r}^2}{c^2} + \beta\,\frac{r\ddot r}{c^2} \right).
\end{equation}

\subsection{Interpretation der Parameterwahl}
\begin{itemize}
    \item \(\beta = 0.5\):  
    Entspricht der Herleitung aus dem Variationsprinzip \eqref{eq:wg_wirkung}. Liefert die korrekte Periheldrehung des Merkur und stabilisiert Planetenbahnen.
    
    \item \(\beta = 1\):  
    Beschreibt die Wechselwirkung mit masselosen Teilchen (Photonen). Ergibt die korrekte Lichtablenkung und Shapiro-Verzögerung, wie sie von der ART vorhergesagt wird.
    
    \item \(\beta = 2\):  
    Entspricht der elektromagnetischen Weber-Kraft für geladene Teilchen. Dieser Wert ergibt sich aus der Struktur der Maxwell-Gleichungen im Weber-Formalismus.
\end{itemize}

Die unterschiedlichen Werte von $\beta$ spiegeln wider, dass die Informationskopplung zwischen Teilchen von deren intrinsischen Eigenschaften (Masse, Ladung, Masse-losigkeit) abhängt. In der vollständigen Informations-Weber-Theorie ist $\beta$ kein freier Parameter, sondern eine Funktion der Informationsdichte und -struktur der wechselwirkenden Entitäten.

\subsection{Konsistenz mit beobachteten Effekten}
Für $\beta = 0.5$ ergibt sich die Periheldrehung pro Umlauf:
\[
\Delta\theta = \frac{6\pi GM}{a(1-e^2)c^2},
\]
was exakt mit dem beobachteten Wert für Merkur übereinstimmt.

Für $\beta = 1$ erhalten wir die Lichtablenkung:
\[
\delta\theta = \frac{4GM}{c^2b},
\]
identisch zur Vorhersage der ART.

Damit ist die \gls{wg} vollständig kompatibel mit den präzisesten Tests der \gls{art}, jedoch ohne das Konzept der Raumzeitkrümmung. Die Wechselwirkung bleibt direkt und nicht-lokal im Sinne der Informationsübertragung.

\section{Herleitung des Bohm-Potentials}
\label{app:bohm}

Das Bohm-Potential ist der globale Anteil des Informations-Lagrange-Funktionals. Es entsteht aus der Variation des globalen Terms:
\[
    \mathcal{F}_{\text{global}}
    =
    \gamma \frac{(\nabla \rho_I)^2}{\rho_I}.
\]

\subsection{Variation des Funktionals}
Wir berechnen:
\[
    \frac{\partial \mathcal{F}}{\partial \rho_I}
    =
    -\gamma \frac{(\nabla \rho_I)^2}{\rho_I^2},
\]

\[
    \frac{\partial \mathcal{F}}{\partial (\nabla \rho_I)}
    =
    2\gamma \frac{\nabla \rho_I}{\rho_I}.
\]
Dann:
\[
    \nabla \cdot
    \left(
        \frac{\partial \mathcal{F}}{\partial (\nabla \rho_I)}
    \right)
    =
    2\gamma
    \left[
        \frac{\nabla^2 \rho_I}{\rho_I}
        -
        \frac{(\nabla \rho_I)^2}{\rho_I^2}
    \right].
\]
Die Euler-Lagrange-Gleichung lautet:
\[
    \frac{\partial \mathcal{F}}{\partial \rho_I}
    -
    \nabla \cdot
    \left(
        \frac{\partial \mathcal{F}}{\partial (\nabla \rho_I)}
    \right)
    = 0.
\]
Einsetzen ergibt:
\[
    -\gamma \frac{(\nabla \rho_I)^2}{\rho_I^2}
    -
    2\gamma
    \left[
        \frac{\nabla^2 \rho_I}{\rho_I}
        -
        \frac{(\nabla \rho_I)^2}{\rho_I^2}
    \right]
    = 0.
\]
Nach Vereinfachung:
\[
    \frac{\nabla^2 \rho_I}{\rho_I}
    -
    \frac{1}{2}
    \frac{(\nabla \rho_I)^2}{\rho_I^2}
    = 0.
\]
Mit der Substitution \(\rho_I = |\Psi|^2\) erhält man:
\[
    Q
    =
    -\frac{\hbar^2}{2m}
    \frac{\nabla^2 |\Psi|}{|\Psi|}.
\]
Dies ist das Bohm-Potential.

\subsection{Interpretation}
\begin{itemize}
    \item \(Q\) beschreibt globale Informationsorganisation.
    \item Es erzeugt quantisierte Energieniveaus.
    \item Es ist nichtlokal, aber kausal unproblematisch.
\end{itemize}

\section{Herleitung der Kontinuitätsgleichung}
\label{app:kontinuitaet}

Die Kontinuitätsgleichung folgt aus der Erhaltung der Gesamtinformation:
\[
    \frac{d}{dt} \int \rho_I\, d^3x = 0.
\]

\subsection{Lokale Form}
Wir fordern:
\[
    \partial_t \rho_I + \nabla \cdot \vec{J}_I = 0.
\]

\subsection{Herleitung aus dem Funktional}
Der lokale Anteil des Funktionals enthält:
\[
    \mathcal{F}_{\text{lokal}}
    =
    \alpha (\partial_t \rho_I)^2
    +
    \beta (\nabla \rho_I)^2.
\]
Die Variation nach \(\partial_t \rho_I\) liefert:
\[
    \frac{\partial \mathcal{F}}{\partial (\partial_t \rho_I)}
    =
    2\alpha\, \partial_t \rho_I.
\]
Dies definiert den Informationsfluss:
\[
    \vec{J}_I
    =
    -2\beta\, \nabla \rho_I.
\]
Einsetzen ergibt die Kontinuitätsgleichung.

\subsection{Interpretation}
\begin{itemize}
    \item \(\rho_I\) ist die Informationsdichte.
    \item \(\vec{J}_I\) ist der Informationsfluss.
    \item Die Gleichung beschreibt lokale Informationsbilanz.
\end{itemize}

\section{Herleitung der effektiven Masse}
\label{app:effmasse}

Die effektive Masse entsteht aus der Trägheit der Informationsstruktur.

\subsection{Definition}
\[
    m_{\text{eff}}
    \propto
    \int (\partial_t \rho_I)^2\, d^3x.
\]

\subsection{Herleitung}
Die kinetische Energie eines Informationsfeldes ist:
\[
    T
    =
    \int \alpha (\partial_t \rho_I)^2\, d^3x.
\]
Vergleich mit der klassischen Form \(T = \frac{1}{2} m v^2\) zeigt:
\[
    m_{\text{eff}}
    \propto
    \int (\partial_t \rho_I)^2\, d^3x.
\]

\subsection{Interpretation}
\begin{itemize}
    \item Masse ist keine fundamentale Größe.
    \item Sie misst die Steifigkeit der Informationsstruktur.
    \item Trägheit ist ein emergentes Phänomen.
\end{itemize}

\section{Zusammenfassung}
In diesem Anhang wurden die zentralen Herleitungen der Informations-Weber-Theorie ausführlich dargestellt:
\begin{itemize}
    \item Die Weber-Kraft entsteht aus dem lokalen Anteil des Funktionals.
    \item Das Bohm-Potential entsteht aus dem globalen Anteil.
    \item Die Kontinuitätsgleichung folgt aus Informationsbilanz.
    \item Die effektive Masse misst die Trägheit der Informationsstruktur.
\end{itemize}
Damit sind alle mathematischen Grundlagen für die Kapitel 4–13 vollständig nachvollziehbar.
