\chapter{Herleitungen}
\label{app:herleitungen}

In diesem Anhang werden die wichtigsten Gleichungen der Informations-Weber-Theorie vollständig und Schritt für Schritt hergeleitet. Ziel ist es, die mathematischen
Grundlagen transparent zu machen und die Verbindung zwischen lokaler Weber-Dynamik, globaler Informationsorganisation und emergenter Geometrie klar herauszuarbeiten.

Behandelt werden:
\begin{itemize}
    \item die Weber-Kraft,
    \item das Bohm-Potential,
    \item die Kontinuitätsgleichung,
    \item die effektive Masse als Informationsmaß.
\end{itemize}
Alle Herleitungen basieren auf dem Informations-Lagrange-Funktional aus Kapitel~\ref{chap:lagrange}.

\section{Herleitung der Weber-Gravitation}
\label{app:webergrav}

Die \gls{wg} ist der gravitative Analogon der \gls{wed}. Sie beschreibt die direkte Wechselwirkung zweier Massen ohne Felder und enthält
geschwindigkeits- und beschleunigungsabhängige Korrekturen.

Die allgemeine Form der Weber-Gravitationskraft lautet:
\[
F_{\text{WG}}
=
-\,G\frac{m_1 m_2}{r^2}
\left(
1
- \frac{\dot r^2}{c^2}
+ \beta\,\frac{r\ddot r}{c^2}
\right),
\]
wobei der Parameter \(\beta\) die physikalische Situation charakterisiert:
\[
\beta =
\begin{cases}
0.5 & \text{für massive Körper (Planetenbahnen)} \\
1   & \text{für Photonen (Lichtablenkung, Shapiro-Effekt)} \\
2   & \text{für elektromagnetische Weber-Wechselwirkung}
\end{cases}
\]
Diese Struktur ist zentral für die \gls{wdbt}, da sie die korrekte Periheldrehung, Lichtablenkung und Rotverschiebung liefert.

\subsection{Ausgangspunkt: Weber-Lagrange-Funktional}
Analog zur \gls{wed} wird das gravitative Lagrange-Funktional angesetzt als:
\[
L
=
-\,G\frac{m_1 m_2}{r}
\left(
1
- \frac{\dot r^2}{c^2}
+ \beta\,\frac{r\ddot r}{c^2}
\right).
\]
Dabei ist:
\[
r = |\vec{r}_1 - \vec{r}_2|,\qquad
\dot r = \frac{dr}{dt},\qquad
\ddot r = \frac{d^2r}{dt^2}.
\]

\subsection{Euler--Lagrange-Gleichung}
Die Kraft ergibt sich aus:
\[
F = \frac{d}{dt}\left(\frac{\partial L}{\partial \dot r}\right)
    - \frac{\partial L}{\partial r}.
\]
Wir berechnen die Terme einzeln.

\subsubsection*{Ableitung nach \(\dot r\)}
\[
\frac{\partial L}{\partial \dot r}
=
-\,G\frac{m_1 m_2}{r}
\left(
-\frac{2\dot r}{c^2}
\right)
=
2G\frac{m_1 m_2}{r}\frac{\dot r}{c^2}.
\]

\[
\frac{d}{dt}
\left(
\frac{\partial L}{\partial \dot r}
\right)
=
2G\frac{m_1 m_2}{c^2}
\left(
\frac{\ddot r}{r}
- \frac{\dot r^2}{r^2}
\right).
\]

\subsubsection*{Ableitung nach \(r\)}
\[
\frac{\partial L}{\partial r}
=
G\frac{m_1 m_2}{r^2}
\left(
1
- \frac{\dot r^2}{c^2}
+ \beta\,\frac{r\ddot r}{c^2}
\right)
-
G\frac{m_1 m_2}{r}
\left(
\beta\,\frac{\ddot r}{c^2}
\right).
\]
Nach Vereinfachung:
\[
\frac{\partial L}{\partial r}
=
G\frac{m_1 m_2}{r^2}
\left(
1
- \frac{\dot r^2}{c^2}
\right).
\]

\subsection{Einsetzen in die Euler-Lagrange-Gleichung}
\[
F
=
-\,G\frac{m_1 m_2}{r^2}
\left(
1
- \frac{\dot r^2}{c^2}
+ \beta\,\frac{r\ddot r}{c^2}
\right).
\]
Dies ist die allgemeine Weber-Gravitationskraft.

\subsection{Interpretation der Parameterwahl}
\begin{itemize}
    \item \(\beta = 0.5\):  
    liefert die korrekte Periheldrehung des Merkur und stabilisiert Planetenbahnen.

    \item \(\beta = 1\):  
    ergibt die korrekte Lichtablenkung und Shapiro-Verzögerung.

    \item \(\beta = 2\):  
    entspricht der elektromagnetischen Weber-Kraft.
\end{itemize}
Damit ist die \gls{wg} vollständig kompatibel mit den beobachteten Effekten der \gls{art}, jedoch ohne Raumzeitkrümmung.

\section{Herleitung des Bohm-Potentials}
\label{app:bohm}

Das Bohm-Potential ist der globale Anteil des Informations-Lagrange-Funktionals. Es entsteht aus der Variation des globalen Terms:
\[
    \mathcal{F}_{\text{global}}
    =
    \gamma \frac{(\nabla \rho_I)^2}{\rho_I}.
\]

\subsection{Variation des Funktionals}
Wir berechnen:
\[
    \frac{\partial \mathcal{F}}{\partial \rho_I}
    =
    -\gamma \frac{(\nabla \rho_I)^2}{\rho_I^2},
\]

\[
    \frac{\partial \mathcal{F}}{\partial (\nabla \rho_I)}
    =
    2\gamma \frac{\nabla \rho_I}{\rho_I}.
\]
Dann:
\[
    \nabla \cdot
    \left(
        \frac{\partial \mathcal{F}}{\partial (\nabla \rho_I)}
    \right)
    =
    2\gamma
    \left[
        \frac{\nabla^2 \rho_I}{\rho_I}
        -
        \frac{(\nabla \rho_I)^2}{\rho_I^2}
    \right].
\]
Die Euler-Lagrange-Gleichung lautet:
\[
    \frac{\partial \mathcal{F}}{\partial \rho_I}
    -
    \nabla \cdot
    \left(
        \frac{\partial \mathcal{F}}{\partial (\nabla \rho_I)}
    \right)
    = 0.
\]
Einsetzen ergibt:
\[
    -\gamma \frac{(\nabla \rho_I)^2}{\rho_I^2}
    -
    2\gamma
    \left[
        \frac{\nabla^2 \rho_I}{\rho_I}
        -
        \frac{(\nabla \rho_I)^2}{\rho_I^2}
    \right]
    = 0.
\]
Nach Vereinfachung:
\[
    \frac{\nabla^2 \rho_I}{\rho_I}
    -
    \frac{1}{2}
    \frac{(\nabla \rho_I)^2}{\rho_I^2}
    = 0.
\]
Mit der Substitution \(\rho_I = |\Psi|^2\) erhält man:
\[
    Q
    =
    -\frac{\hbar^2}{2m}
    \frac{\nabla^2 |\Psi|}{|\Psi|}.
\]
Dies ist das Bohm-Potential.

\subsection{Interpretation}
\begin{itemize}
    \item \(Q\) beschreibt globale Informationsorganisation.
    \item Es erzeugt quantisierte Energieniveaus.
    \item Es ist nichtlokal, aber kausal unproblematisch.
\end{itemize}

\section{Herleitung der Kontinuitätsgleichung}
\label{app:kontinuitaet}

Die Kontinuitätsgleichung folgt aus der Erhaltung der Gesamtinformation:
\[
    \frac{d}{dt} \int \rho_I\, d^3x = 0.
\]

\subsection{Lokale Form}
Wir fordern:
\[
    \partial_t \rho_I + \nabla \cdot \vec{J}_I = 0.
\]

\subsection{Herleitung aus dem Funktional}
Der lokale Anteil des Funktionals enthält:
\[
    \mathcal{F}_{\text{lokal}}
    =
    \alpha (\partial_t \rho_I)^2
    +
    \beta (\nabla \rho_I)^2.
\]
Die Variation nach \(\partial_t \rho_I\) liefert:
\[
    \frac{\partial \mathcal{F}}{\partial (\partial_t \rho_I)}
    =
    2\alpha\, \partial_t \rho_I.
\]
Dies definiert den Informationsfluss:
\[
    \vec{J}_I
    =
    -2\beta\, \nabla \rho_I.
\]
Einsetzen ergibt die Kontinuitätsgleichung.

\subsection{Interpretation}
\begin{itemize}
    \item \(\rho_I\) ist die Informationsdichte.
    \item \(\vec{J}_I\) ist der Informationsfluss.
    \item Die Gleichung beschreibt lokale Informationsbilanz.
\end{itemize}

\section{Herleitung der effektiven Masse}
\label{app:effmasse}

Die effektive Masse entsteht aus der Trägheit der Informationsstruktur.

\subsection{Definition}
\[
    m_{\text{eff}}
    \propto
    \int (\partial_t \rho_I)^2\, d^3x.
\]

\subsection{Herleitung}
Die kinetische Energie eines Informationsfeldes ist:
\[
    T
    =
    \int \alpha (\partial_t \rho_I)^2\, d^3x.
\]
Vergleich mit der klassischen Form \(T = \frac{1}{2} m v^2\) zeigt:
\[
    m_{\text{eff}}
    \propto
    \int (\partial_t \rho_I)^2\, d^3x.
\]

\subsection{Interpretation}
\begin{itemize}
    \item Masse ist keine fundamentale Größe.
    \item Sie misst die Steifigkeit der Informationsstruktur.
    \item Trägheit ist ein emergentes Phänomen.
\end{itemize}

\section{Zusammenfassung}
In diesem Anhang wurden die zentralen Herleitungen der Informations-Weber-Theorie ausführlich dargestellt:
\begin{itemize}
    \item Die Weber-Kraft entsteht aus dem lokalen Anteil des Funktionals.
    \item Das Bohm-Potential entsteht aus dem globalen Anteil.
    \item Die Kontinuitätsgleichung folgt aus Informationsbilanz.
    \item Die effektive Masse misst die Trägheit der Informationsstruktur.
\end{itemize}
Damit sind alle mathematischen Grundlagen für die Kapitel 4–13 vollständig nachvollziehbar.
