\chapter{Herleitungen}
\label{anhang:herleitungen}

\section{Herleitung der Weber-Kraft}
Aus dem Weber-Lagrange-Funktional
\[
    L
    =
    \frac{1}{2} m \dot{r}^2
    -
    \frac{q_1 q_2}{4\pi\varepsilon_0 r}
    \left(
        1
        -
        \frac{\dot{r}^2}{2c^2}
        +
        \frac{r \ddot{r}}{c^2}
    \right)
\]
ergibt die Variation die Weber-Kraft
\[
    \vec{F}
    =
    \frac{q_1 q_2}{4\pi\varepsilon_0 r^2}
    \left[
        1
        -
        \frac{\dot{r}^2}{c^2}
        +
        \frac{2 r \ddot{r}}{c^2}
    \right]
    \hat{\vec{r}}.
\]

\section{Herleitung des Bohm-Potentials}
Aus dem globalen Informationsfunktional
\[
    \mathcal{F}_{\text{global}}
    =
    \gamma \frac{(\nabla \rho_I)^2}{\rho_I}
\]
ergibt die Variation
\[
    Q
    =
    -\frac{\hbar^2}{2m}
    \frac{\nabla^2 \sqrt{\rho_I}}{\sqrt{\rho_I}}.
\]

\section{Herleitung der Kontinuitätsgleichung}
Die Variation nach $\partial_t \rho_I$ ergibt
\[
    \vec{J}_I
    =
    \frac{\partial \mathcal{F}}{\partial (\nabla \rho_I)}.
\]
Damit folgt
\[
    \partial_t \rho_I + \nabla \cdot \vec{J}_I = 0.
\]

\section{Herleitung der effektiven Masse}

Die effektive Masse ergibt sich aus
\[
    m_{\text{eff}}
    =
    2\alpha \int \left(\frac{\partial \rho_I}{\partial v}\right)^2 d^3x.
\]
