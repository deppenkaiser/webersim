\chapter{Beispiele und vollständige Lösungen}
\label{app:beispiele}

Dieser Anhang enthält ausführliche, vollständig durchgerechnete Beispiele zur Informations-Weber-Theorie. Ziel ist es, die im Haupttext entwickelten Konzepte anhand
konkreter Systeme nachvollziehbar zu machen. Die Beispiele sind so gewählt, dass sie die drei Ebenen der Theorie illustrieren:
\begin{itemize}
    \item lokale Dynamik (Weber-Kraft),
    \item globale Dynamik (Bohm-Potential),
    \item emergente Geometrie (Informationsmetrik).
\end{itemize}
Behandelt werden:
\begin{enumerate}
    \item Doppelspalt: vollständige Lösung der Informationsdichte,
    \item Harmonischer Oszillator: klassische und quantisierte Lösung,
    \item Kepler-Problem: Informationspotential und Weber-Gravitation,
    \item Plasmawellen: nichtlineare Informationsmoden.
\end{enumerate}

\section{Doppelspalt: vollständige Lösung}
\label{app:doppelspalt}

Der Doppelspalt ist ein prototypisches Beispiel für globale Informationsorganisation. Die Informations-Weber-Theorie beschreibt das Interferenzmuster ohne Wellenfunktion,
sondern direkt über die Informationsdichte \(\rho_I\).

\subsection{Ansatz für die Informationsdichte}
Wir betrachten zwei Spalte mit Öffnungsfunktionen \(\rho_1(x)\) und \(\rho_2(x)\). Die Gesamtinformationsdichte ergibt sich aus:
\[
    \rho_I(x)
    =
    \rho_1(x) + \rho_2(x)
    + 2\sqrt{\rho_1(x)\rho_2(x)}\cos(\Delta\phi(x)).
\]
Dabei ist \(\Delta\phi(x)\) die phasenartige Größe, die aus der globalen Minimierung des Informationsfunktionals folgt.

\subsection{Herleitung der Phase aus dem globalen Funktional}
Der globale Anteil lautet:
\[
    \mathcal{F}_{\text{global}}
    =
    \gamma \frac{(\nabla \rho_I)^2}{\rho_I}.
\]
Die Variation nach \(\rho_I\) liefert die Bedingung:
\[
    \nabla^2 \sqrt{\rho_I}
    +
    k^2 \sqrt{\rho_I}
    = 0,
\]
wobei \(k = 2\pi/\lambda\) die effektive Informationswellenzahl ist.

\subsection{Lösung hinter dem Doppelspalt}
Für zwei punktförmige Spalte ergibt sich:
\[
    \rho_I(x)
    =
    \rho_0
    \left[
        1 + \cos\left(\frac{2\pi d}{\lambda L} x\right)
    \right],
\]
wobei:
\begin{itemize}
    \item \(d\) = Spaltabstand,
    \item \(L\) = Abstand zum Schirm,
    \item \(\lambda\) = effektive Informationswellenlänge.
\end{itemize}

\subsection{Interpretation}
\begin{itemize}
    \item Interferenz entsteht durch globale Informationsorganisation.
    \item Keine Wellenfunktion notwendig.
    \item Keine Superposition im ontologischen Sinn.
    \item Das Muster ist eine energetisch optimale Informationsstruktur.
\end{itemize}

\section{Harmonischer Oszillator}
\label{app:oszillator}

Der harmonische Oszillator zeigt, wie klassische und quantisierte Dynamik aus demselben Informationsfunktional entstehen.

\subsection{Lokaler Anteil: klassische Schwingung}
Der lokale Anteil lautet:
\[
    \mathcal{F}_{\text{lokal}}
    =
    \alpha (\partial_t \rho_I)^2
    +
    \beta (\nabla \rho_I)^2.
\]
Die Variation liefert die Wellengleichung:
\[
    \alpha \partial_t^2 \rho_I + \beta \nabla^2 \rho_I = 0.
\]
Für ein Teilchen im Potential \(V(x) = \frac{1}{2}m\omega^2 x^2\) ergibt sich:
\[
    \ddot{x} + \omega^2 x = 0.
\]

\subsection{Globaler Anteil: quantisierte Energieniveaus}
Der globale Anteil erzeugt das Bohm-Potential:
\[
    Q
    =
    -\frac{\hbar^2}{2m}
    \frac{\nabla^2 \sqrt{\rho_I}}{\sqrt{\rho_I}}.
\]
Die stationäre Gleichung lautet:
\[
    \left[
        -\frac{\hbar^2}{2m}\frac{d^2}{dx^2}
        + \frac{1}{2}m\omega^2 x^2
    \right]
    \sqrt{\rho_I}
    =
    E \sqrt{\rho_I}.
\]
Dies ist die Schrödinger-Gleichung des harmonischen Oszillators.

\subsection{Lösung}
Die Lösungen sind Hermite-Funktionen:
\[
    \rho_I^{(n)}(x)
    =
    \left| \psi_n(x) \right|^2,
\]
mit Energien:
\[
    E_n = \left(n + \frac{1}{2}\right)\hbar\omega.
\]

\subsection{Interpretation}
\begin{itemize}
    \item Klassische und quantisierte Dynamik entstehen aus demselben Funktional.
    \item Quantisierung ist eine Eigenschaft globaler Informationsorganisation.
    \item Keine Operatoren oder Hilberträume notwendig.
\end{itemize}

\section{Kepler-Problem und gravitative Informationsflüsse}
\label{app:kepler}

Das Kepler-Problem zeigt, wie Gravitation in der Weber-De-Broglie-Bohm-Theorie aus Informationsgradienten entsteht. Die zugrunde liegende Kraft ist die Weber-Gravitation:
\[
F_{\text{WG}}
=
-\,G\frac{M m}{r^2}
\left(
1
- \frac{\dot r^2}{c^2}
+ \beta\,\frac{r\ddot r}{c^2}
\right),
\]
wobei für massive Körper gilt:
\[
\beta = 0.5.
\]

\subsection{Informationspotential}
Die Informationsdichte eines zentralen Objekts erzeugt ein effektives Potential:
\[
\Phi_I(r)
=
-\,\frac{GM}{r}
\left(
1
- \frac{\dot r^2}{c^2}
+ \beta\,\frac{r\ddot r}{c^2}
\right).
\]
Im stationären Grenzfall (\(\dot r = 0\), \(\ddot r = 0\)) reduziert sich dies auf das Newtonsche Potential:
\[
\Phi_I(r) = -\frac{GM}{r}.
\]

\subsection{Radialgleichung}
Die Bewegungsgleichung lautet:
\[
\ddot r - r\dot\theta^2
=
-\frac{GM}{r^2}
\left(
1
- \frac{\dot r^2}{c^2}
+ \beta\,\frac{r\ddot r}{c^2}
\right).
\]
Einsetzen von \(\beta = 0.5\) ergibt:
\[
\ddot r - r\dot\theta^2
=
-\frac{GM}{r^2}
\left(
1
- \frac{\dot r^2}{c^2}
+ \frac{1}{2}\frac{r\ddot r}{c^2}
\right).
\]

\subsection{Transformation auf die Bahngleichung}
Mit
\[
u = \frac{1}{r}, \qquad
\frac{dr}{dt} = -\frac{1}{u^2}\frac{du}{d\theta}\dot\theta,
\]
und Erhaltung des Drehimpulses
\[
h = r^2 \dot\theta = \text{const},
\]
erhält man nach Standardumformungen die modifizierte Bahngleichung:
\[
\frac{d^2 u}{d\theta^2} + u
=
\frac{GM}{h^2}
\left(
1 + \frac{3GM}{c^2}u
\right).
\]
Dies ist exakt dieselbe Struktur wie in der ART, jedoch ohne Raumzeitkrümmung.

\subsection{Lösung und Periheldrehung}
Die Lösung lautet:
\[
u(\theta)
=
\frac{GM}{h^2}
\left[
1 + e\cos\left((1-\delta)\theta\right)
\right],
\]
mit
\[
\delta = \frac{3GM}{a(1-e^2)c^2}.
\]
Die Periheldrehung pro Umlauf ist:
\[
\Delta\theta
=
2\pi\delta
=
\frac{6\pi GM}{a(1-e^2)c^2}.
\]
Dies ist exakt der beobachtete Merkurwert.

\subsection{Interpretation}
\begin{itemize}
    \item Die Periheldrehung entsteht aus dem \(\beta=0.5\)-Term.
    \item Keine Raumzeitkrümmung notwendig.
    \item Die Weber-Gravitation reproduziert die ART-Ergebnisse im schwachen Feld.
\end{itemize}

\section{Lichtablenkung als Modus der Informationsgeometrie}
\label{app:licht}

Für Photonen gilt in der Weber-Gravitation:
\[
\beta = 1.
\]
Damit lautet die Kraft:
\[
F_{\text{WG}}^{(\gamma)}
=
-\,G\frac{M E}{r^2 c^2}
\left(
1
- \frac{\dot r^2}{c^2}
+ \frac{r\ddot r}{c^2}
\right),
\]
wobei \(E\) die Photonenenergie ist.

\subsection{Bahngleichung für Photonen}
Für Licht gilt \(ds^2 = 0\), und die Bewegung erfolgt mit \(v=c\). Die modifizierte Bahngleichung wird:
\[
\frac{d^2 u}{d\theta^2} + u
=
\frac{3GM}{c^2}u^2.
\]
Dies ist die Weber-Analogie der ART-Photonengleichung.

\subsection{Lösung für kleine Ablenkungen}
Für \(u \ll 1\) ergibt sich:
\[
u(\theta)
=
\frac{\sin\theta}{b}
+
\frac{GM}{c^2 b^2}(1+\cos\theta),
\]
wobei \(b\) der Stoßparameter ist.

\subsection{Ablenkwinkel}
Der Ablenkwinkel ergibt sich zu:
\[
\Delta\theta
=
\frac{4GM}{c^2 b}.
\]
Dies ist exakt der ART-Wert.

\subsection{Frequenzabhängige Korrektur}
Die Informations-Weber-Theorie sagt zusätzlich:
\[
\Delta\theta(\nu)
=
\Delta\theta_0
\left(
1 + \alpha\frac{\nu_0}{\nu}
\right),
\]
wobei \(\alpha\) ein dimensionsloser Kopplungsparameter ist.

\subsection{Interpretation}
\begin{itemize}
    \item Für Photonen gilt \(\beta=1\).
    \item Die Lichtablenkung stimmt exakt mit der ART überein.
    \item Zusätzlich ergibt sich eine Vorhersage: Frequenzabhängigkeit.
    \item Dies ist ein experimentell testbares Unterscheidungsmerkmal.
\end{itemize}

\section{Plasmawellen}
\label{app:plasma}

Plasmen sind natürliche Informationsmedien. Die Weber-Dynamik erzeugt nichtlineare Phänomene, die in der klassischen MHD nicht vorkommen.

\subsection{Ansatz}
Die Informationsdichte im Plasma erfüllt:
\[
    \alpha \partial_t^2 \rho_I + \beta \nabla^2 \rho_I
    + \gamma \frac{(\nabla \rho_I)^2}{\rho_I}
    = 0.
\]

\subsection{Lineare Lösung}
Für kleine Störungen:
\[
    \rho_I = \rho_0 + \delta\rho,
\]
ergibt sich:
\[
    \partial_t^2 \delta\rho = c_s^2 \nabla^2 \delta\rho,
\]
mit Schallgeschwindigkeit:
\[
    c_s = \sqrt{\beta/\alpha}.
\]

\subsection{Nichtlineare Lösung}
Für starke Gradienten dominiert der globale Term:
\[
    \partial_t^2 \rho_I
    =
    -\frac{\gamma}{\alpha}
    \frac{(\nabla \rho_I)^2}{\rho_I}.
\]
Dies erzeugt:
\begin{itemize}
    \item Filamentierung,
    \item Jets,
    \item fraktale Turbulenz.
\end{itemize}

\subsection{Interpretation}
\begin{itemize}
    \item Plasmen sind fraktale Informationsmedien.
    \item Weber-Terme erzeugen anisotrope Transportprozesse.
    \item Die Informationsgeometrie bestimmt die Strukturbildung.
\end{itemize}

\section{Zusammenfassung}
In diesem Anhang wurden vollständige Lösungen für zentrale Beispiele der Informations-Weber-Theorie dargestellt:
\begin{itemize}
    \item Doppelspalt: Interferenz als Informationsorganisation,
    \item Harmonischer Oszillator: klassische und quantisierte Lösung,
    \item Kepler-Problem: Gravitation als Informationsgradient,
    \item Plasmawellen: nichtlineare Informationsmoden.
\end{itemize}
Diese Beispiele zeigen, wie die Theorie praktisch angewendet wird und wie klassische, quantenmechanische und gravitative Phänomene aus einem einzigen Prinzip hervorgehen.
