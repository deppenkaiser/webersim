\chapter{Energieerhaltung, Rotverschiebung und die Gleichgewichtstemperatur des kosmischen Plasmas}
In diesem Anhang wird die energetische Struktur des stationären Universums der Informations-Weber-Theorie untersucht. Die Rotverschiebung kosmischer Photonen,
der Energiefluss im intergalaktischen Plasma und die thermische Gleichgewichtstemperatur stehen in einem direkten Zusammenhang, der sich aus der fraktalen Geometrie des
Universums und der Weber-Dynamik ergibt. Die folgenden Abschnitte entwickeln diese Zusammenhänge systematisch und führen zu einer vollständigen Kalibrierung der
Rotverschiebungsdynamik.

\section{Kalibrierung der Rotverschiebungsdynamik im fraktalen Universum}
Die Rotverschiebung eines Photons entlang einer kosmischen Strecke $d$ folgt in der Informations-Weber-Theorie aus der integrierten Wechselwirkung mit der fraktalen
Massenverteilung des Universums. Die allgemeine Form lautet
\begin{equation}
z(d) = \gamma_{\mathrm{eff}}\, G\, \rho_{\mathrm{eff}}\, d^2,
\end{equation}
wobei $\gamma_{\mathrm{eff}}$ die effektive Kopplungskonstante ist, die sowohl die fraktale Geometrie als auch die Weber-Dynamik umfasst. Die Größe $\rho_{\mathrm{eff}}$
ist die mittlere kosmische Dichte, die aus der fraktalen Struktur folgt.

Die Bestimmung von $\gamma_{\mathrm{eff}}$ ist der zentrale Schritt zur Festlegung der\\Entfernung–Rotverschiebungs-Relation. Dazu werden zunächst die fraktalen
Normierungen der Mach-Konstante und der Weber-Kopplung hergeleitet.

\subsection{Fraktale Normierung der Weber-Kopplung}
Die fraktale Massenverteilung des Universums wird durch
\begin{equation}
\rho(r) = \rho_0 \left(\frac{r}{R}\right)^{D-3}
\end{equation}
beschrieben, wobei $R$ der Mach-Radius und $D$ die fraktale Dimension ist. Aus dieser Verteilung ergibt sich das Mach-Potential
\begin{equation}
\Phi_M = \frac{4\pi}{D-1}\,G\rho_0 R^2.
\end{equation}
Vergleich mit der Mach-Relation
\begin{equation}
c^2 = 2\kappa_M G\rho_{\mathrm{eff}} R^2
\end{equation}
liefert die fraktale Mach-Konstante
\begin{equation}
\kappa_M(D) = \frac{2\pi D}{3(D-1)}.
\end{equation}
Für die fraktale Dimension $D = 2.71$ ergibt sich numerisch $\kappa_M \approx 3.3$.

Die Weber-Kopplung eines Photons mit der fraktalen Massenverteilung führt auf die dimensionslose Normierung
\begin{equation}
\gamma(D)
= C\,\frac{D}{3(D-2)}\,\frac{\eta^{D-4}}{c^2 R},
\end{equation}
wobei $C$ eine Weber-Normierungskonstante und $\eta = L/R$ das Verhältnis der kosmischen Kopplungslänge $L$ zum Mach-Radius ist. Die effektive Rotverschiebungskonstante
ergibt sich zu
\begin{equation}
\gamma_{\mathrm{eff}} = C\,\eta^{D-4}\,\gamma(D).
\end{equation}

\subsection{Konsequenz für die kosmische Rotverschiebungsskala}
Die beobachtete Rotverschiebung $z\approx 1$ bei einer Entfernung von
\begin{equation}
d_0 = 1\,\mathrm{Gpc}
\end{equation}
liefert die Bedingung
\begin{equation}
1 = \gamma_{\mathrm{eff}}\, G\rho_{\mathrm{eff}}\, d_0^2.
\end{equation}
Damit folgt
\begin{equation}
\gamma_{\mathrm{eff}} = \frac{1}{G\rho_{\mathrm{eff}} d_0^2}.
\end{equation}
Für $\rho_{\mathrm{eff}} = 4\times 10^{-28}\,\mathrm{kg/m^3}$ ergibt sich
\begin{equation}
\gamma_{\mathrm{eff}} \approx 4\times 10^{-14}.
\end{equation}

Die fraktale Struktur verlangt
\begin{equation}
C\,\eta^{-1.29} \approx 10^{30},
\end{equation}
wobei der Exponent $1.29$ aus $D-4$ für $D=2.71$ resultiert.

Eine physikalisch sinnvolle Wahl ist eine kosmische Kopplungslänge
\begin{equation}
L = 100\,\mathrm{Mpc},
\end{equation}
was dem Verhältnis
\begin{equation}
\eta = \frac{L}{R} \approx 4\times 10^{-3}
\end{equation}
entspricht. Damit ergibt sich
\begin{equation}
C \approx 10^{27}.
\end{equation}

Die effektive Rotverschiebungskonstante ist damit vollständig bestimmt:
\begin{equation}
\gamma_{\mathrm{eff}} = 4\times 10^{-14}.
\end{equation}

Mit dieser Kalibrierung folgt für alle kosmischen Distanzen
\begin{equation}
z(d) = \left(\frac{d}{1\,\mathrm{Gpc}}\right)^2.
\end{equation}
Damit ergeben sich für hohe Rotverschiebungen die Entfernungen
\begin{align}
z = 10 &\;\Rightarrow\; d \approx 3.2\,\mathrm{Gpc},\\
z = 20 &\;\Rightarrow\; d \approx 4.5\,\mathrm{Gpc}.
\end{align}
Die extremen JWST-Rotverschiebungen liegen somit in einem Bereich von wenigen Gigaparsec und erfordern weder eine kosmische Expansion noch eine thermische Frühzeit. Die
Rotverschiebung ist eine direkte Konsequenz der fraktalen Weber-Dynamik im stationären Universum der Informations-Weber-Theorie.

\section{Fraktale Herleitung der kosmischen Verlustkonstante}
Die fraktale Informationsstruktur des Universums liefert die effektive Kopplung \(\gamma_{\mathrm{eff}}\) sowie die effektive Massendichte \(\rho_{\mathrm{eff}}\) 
und die charakteristische Längenskala \(L\) nicht als freie Parameter, sondern als konsequente Resultate der fraktalen Geometrie. Aus diesen Größen ergibt sich die 
mittlere kosmische Verlustkonstante
\[
\bar{\alpha}(L)
=
\frac{1}{L\,\gamma_{\mathrm{eff}} G \rho_{\mathrm{eff}}}
\ln\!\bigl(1+\gamma_{\mathrm{eff}} G \rho_{\mathrm{eff}} L^2\bigr),
\]
die die mittlere Energierückführung von Photonen an das kosmische Medium beschreibt. Damit ist \(\bar{\alpha}(L)\) keine Annahme, sondern eine direkte Konsequenz der 
fraktalen Struktur und der Weber-artigen Kopplung zwischen Materie und Information. Die fraktale Kosmologie liefert somit eine vollständig theoretische Bestimmung der 
kosmischen Heizrate.

\section{Die kombinierte Plasmaparametergröße \texorpdfstring{$X$}{X}}
Die beobachtete CMB-Temperatur \(T_{\mathrm{CMB}}\) erfüllt die Gleichgewichtsbedingung
\[
T_{\mathrm{CMB}}^4
=
\frac{\bar{\alpha}(L)\,u_\gamma}
{\varepsilon\,A_{\mathrm{eff}}\,\sigma},
\]
wobei \(u_\gamma\) die Photonenenergiedichte, \(\varepsilon\) die Emissivität und \(A_{\mathrm{eff}}\) die effektive Oberfläche pro Volumen des kosmischen Plasmas ist. 
Da \(\bar{\alpha}(L)\) vollständig aus der fraktalen Struktur folgt, bestimmt die beobachtete Temperatur die kombinierte Plasmaparametergröße
\[
X := \frac{u_\gamma}{\varepsilon A_{\mathrm{eff}}}.
\]
Diese Größe fasst die mikrophysikalischen Eigenschaften des extrem dünnen intergalaktischen Plasmas zusammen. Die Theorie benötigt keine getrennte Bestimmung 
von \(u_\gamma\), \(\varepsilon\) oder \(A_{\mathrm{eff}}\); das Gleichgewicht legt lediglich ihre Kombination fest. Damit entsteht ein konsistentes Bild aus 
kosmischer Struktur (über \(\bar{\alpha}(L)\)) und Plasmaphysik (über \(X\)).

\section{Abgrenzung zu klassischen tired-light-Modellen}
Die hier betrachtete Energiebilanz stellt keine klassische Form der \emph{Lichtermüdung} dar, wie sie in der Standardkosmologie verworfen wird. Klassische
tired-light-Modelle postulieren einen linearen oder exponentiellen Energieverlust pro Weglänge, der zu spektralen Verzerrungen, fehlender Zeitdilatation oder
unphysikalischen Dämpfungsprofilen führt. Solche Modelle sind beobachtungswidrig und werden daher zurecht ausgeschlossen.

Der in diesem Anhang behandelte Energiefluss unterscheidet sich grundlegend davon:
\begin{itemize}
    \item Er ist nicht ad hoc, sondern folgt aus der fraktalen Informationsstruktur.
    \item Er ist nicht linear und nicht exponentiell, sondern logarithmisch in 
          \(\ln(1+\gamma_{\mathrm{eff}} G \rho_{\mathrm{eff}} L^2)\).
    \item Er erzeugt keine spektralen Verzerrungen, da die Kopplung 
          frequenzunabhängig ist.
    \item Er ist extrem schwach, aber über kosmologische Distanzen nicht verschwindend.
\end{itemize}
Damit handelt es sich nicht um ein tired-light-Modell, sondern um eine Weber-artige Energiebilanz, die aus der fraktalen Struktur des Universums folgt und mit allen
Beobachtungen vereinbar ist.

\section{Plasmafrequenz, optische Tiefe und Transparenz des kosmischen Mediums}
Das intergalaktische Plasma besitzt eine endliche Elektronendichte \(n_e\), woraus die Plasmafrequenz
\[
\omega_p^2 = \frac{n_e e^2}{\varepsilon_0 m_e}
\]
resultiert. Für CMB-Frequenzen gilt \(\omega_{\mathrm{CMB}} \gg \omega_p\), sodass das Plasma im relevanten Frequenzbereich nahezu transparent ist. Gleichzeitig ist die
optische Tiefe über kosmologische Distanzen
\[
\tau_{\mathrm{eff}} \sim \varepsilon A_{\mathrm{eff}} L
\]
nicht verschwindend, da die effektive Oberfläche \(A_{\mathrm{eff}}\) aufgrund der großen Zahl mikroskopischer Streu- und Emissionsprozesse sehr groß ist. Das kosmische
Plasma ist daher im CMB-Bereich transparent genug, um das Planck-Spektrum nicht zu verzerren, aber gekoppelt genug, um ein thermisches Gleichgewicht mit der durch
Rotverschiebung erzeugten Heizrate herzustellen. Die Kombination aus geringer Emissivität, großer effektiver Oberfläche und nicht-verschwindender optischer Tiefe erklärt
die beobachtete CMB-Temperatur als stationäres Gleichgewicht eines dünnen, nahezu durchsichtigen Plasmas.

\section{Fazit}
Die CMB-Temperatur ergibt sich als Gleichgewicht zwischen kosmischer Rotverschiebungsheizung und schwacher thermischer Abstrahlung eines dünnen, nahezu transparenten
Plasmas. Die fraktale Struktur liefert die kosmische Verlustkonstante \(\bar{\alpha}(L)\) rein theoretisch, während die beobachtete Temperatur die kombinierte
Plasmaparametergröße \(X = u_\gamma/(\varepsilon A_{\mathrm{eff}})\) festlegt. Beide Seiten ergeben ein konsistentes, vollständig physikalisches Bild, das ohne klassische
tired-light-Mechanismen auskommt und die CMB-Temperatur als stationäres thermodynamisches Resultat eines fraktal strukturierten Universums versteht.
