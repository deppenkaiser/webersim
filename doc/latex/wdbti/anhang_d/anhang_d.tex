\chapter{Energieerhaltung, Rotverschiebung und die Gleichgewichtstemperatur des kosmischen Plasmas}
In diesem Anhang wird die energetische Struktur des stationären Universums der Informations-Weber-Theorie untersucht. Die Rotverschiebung kosmischer Photonen,
der Energiefluss im intergalaktischen Plasma und die thermische Gleichgewichtstemperatur stehen in einem direkten Zusammenhang, der sich aus der fraktalen Geometrie des
Universums und der Weber-Dynamik ergibt. Die folgenden Abschnitte entwickeln diese Zusammenhänge systematisch und führen zu einer vollständigen Kalibrierung der
Rotverschiebungsdynamik.

\section{Kalibrierung der Rotverschiebungsdynamik im fraktalen Universum}
Die Rotverschiebung eines Photons entlang einer kosmischen Strecke $d$ folgt in der Informations-Weber-Theorie aus der integrierten Wechselwirkung mit der fraktalen
Massenverteilung des Universums. Die allgemeine Form lautet
\begin{equation}
z(d) = \gamma_{\mathrm{eff}}\, G\, \rho_{\mathrm{eff}}\, d^2,
\end{equation}
wobei $\gamma_{\mathrm{eff}}$ die effektive Kopplungskonstante ist, die sowohl die fraktale Geometrie als auch die Weber-Dynamik umfasst. Die Größe $\rho_{\mathrm{eff}}$
ist die mittlere kosmische Dichte, die aus der fraktalen Struktur folgt.

Die Bestimmung von $\gamma_{\mathrm{eff}}$ ist der zentrale Schritt zur Festlegung der\\Entfernung–Rotverschiebungs-Relation. Dazu werden zunächst die fraktalen
Normierungen der Mach-Konstante und der Weber-Kopplung hergeleitet.

\subsection{Fraktale Normierung der Weber-Kopplung}
Die fraktale Massenverteilung des Universums wird durch
\begin{equation}
\rho(r) = \rho_0 \left(\frac{r}{R}\right)^{D-3}
\end{equation}
beschrieben, wobei $R$ der Mach-Radius und $D$ die fraktale Dimension ist. Aus dieser Verteilung ergibt sich das Mach-Potential
\begin{equation}
\Phi_M = \frac{4\pi}{D-1}\,G\rho_0 R^2.
\end{equation}
Vergleich mit der Mach-Relation
\begin{equation}
c^2 = 2\kappa_M G\rho_{\mathrm{eff}} R^2
\end{equation}
liefert die fraktale Mach-Konstante
\begin{equation}
\kappa_M(D) = \frac{2\pi D}{3(D-1)}.
\end{equation}
Für die fraktale Dimension $D = 2.71$ ergibt sich numerisch $\kappa_M \approx 3.3$.

Die Weber-Kopplung eines Photons mit der fraktalen Massenverteilung führt auf die dimensionslose Normierung
\begin{equation}
\gamma(D)
= C\,\frac{D}{3(D-2)}\,\frac{\eta^{D-4}}{c^2 R},
\end{equation}
wobei $C$ eine Weber-Normierungskonstante und $\eta = L/R$ das Verhältnis der kosmischen Kopplungslänge $L$ zum Mach-Radius ist. Die effektive Rotverschiebungskonstante
ergibt sich zu
\begin{equation}
\gamma_{\mathrm{eff}} = C\,\eta^{D-4}\,\gamma(D).
\end{equation}

\subsection{Konsequenz für die kosmische Rotverschiebungsskala}
Die beobachtete Rotverschiebung $z\approx 1$ bei einer Entfernung von
\begin{equation}
d_0 = 1\,\mathrm{Gpc}
\end{equation}
liefert die Bedingung
\begin{equation}
1 = \gamma_{\mathrm{eff}}\, G\rho_{\mathrm{eff}}\, d_0^2.
\end{equation}
Damit folgt
\begin{equation}
\gamma_{\mathrm{eff}} = \frac{1}{G\rho_{\mathrm{eff}} d_0^2}.
\end{equation}
Für $\rho_{\mathrm{eff}} = 4\times 10^{-28}\,\mathrm{kg/m^3}$ ergibt sich
\begin{equation}
\gamma_{\mathrm{eff}} \approx 4\times 10^{-14}.
\end{equation}

Die fraktale Struktur verlangt
\begin{equation}
C\,\eta^{-1.29} \approx 10^{30},
\end{equation}
wobei der Exponent $1.29$ aus $D-4$ für $D=2.71$ resultiert.

Eine physikalisch sinnvolle Wahl ist eine kosmische Kopplungslänge
\begin{equation}
L = 100\,\mathrm{Mpc},
\end{equation}
was dem Verhältnis
\begin{equation}
\eta = \frac{L}{R} \approx 4\times 10^{-3}
\end{equation}
entspricht. Damit ergibt sich
\begin{equation}
C \approx 10^{27}.
\end{equation}

Die effektive Rotverschiebungskonstante ist damit vollständig bestimmt:
\begin{equation}
\gamma_{\mathrm{eff}} = 4\times 10^{-14}.
\end{equation}

Mit dieser Kalibrierung folgt für alle kosmischen Distanzen
\begin{equation}
z(d) = \left(\frac{d}{1\,\mathrm{Gpc}}\right)^2.
\end{equation}
Damit ergeben sich für hohe Rotverschiebungen die Entfernungen
\begin{align}
z = 10 &\;\Rightarrow\; d \approx 3.2\,\mathrm{Gpc},\\
z = 20 &\;\Rightarrow\; d \approx 4.5\,\mathrm{Gpc}.
\end{align}
Die extremen JWST-Rotverschiebungen liegen somit in einem Bereich von wenigen Gigaparsec und erfordern weder eine kosmische Expansion noch eine thermische Frühzeit. Die
Rotverschiebung ist eine direkte Konsequenz der fraktalen Weber-Dynamik im stationären Universum der Informations-Weber-Theorie.
