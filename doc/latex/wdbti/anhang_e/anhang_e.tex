\chapter{Kosmologie in der Informations-Weber-Theorie}
\label{chap:kosmologie_e}

\section{Einleitung: Kosmologie als Informationsdynamik}

Die moderne Kosmologie ist geprägt von tiefen konzeptionellen Spannungen. Während die \gls{art} ein dynamisches Raumzeitkontinuum postuliert, das sich ausdehnt, krümmt und 
Singularitäten bildet, zeigen Beobachtungen zunehmend Phänomene, die mit diesem Bild nur schwer vereinbar sind: extrem frühe Galaxien, massereiche Strukturen bei hohen 
Rotverschiebungen, eine nahezu fraktale Materieverteilung und eine kosmische Hintergrundstrahlung, deren Ursprung keineswegs eindeutig auf einen Urknall zurückgeführt
werden muss.

Die \textbf{Informations-Weber-Theorie} (IWT) bietet eine alternative Perspektive. Sie beschreibt das Universum nicht als expandierende Raumzeit, sondern als
\emph{statisches, informationsgetragenes Netzwerk}, in dem Gravitation, Trägheit und Rotverschiebung aus der Kopplungsstruktur eines fraktalen Informationsgitters
emergieren. Kosmologie wird damit nicht zur Frage der Geometrie, sondern der \emph{Informationsorganisation}.

In diesem Anhang wird gezeigt, dass die IWT eine vollständige, konsistente und empirisch überprüfbare Kosmologie liefert - ohne Expansion, ohne Dunkle Materie, ohne
Inflation und ohne Singularitäten. Die zentralen kosmologischen Größen wie die Hubble-Konstante $H_0$, der kosmische Radius $R$, die effektive Dichte $\rho_{\mathrm{eff}}$
und die Entfernung-Rotverschiebungs-Relation $d(z)$ ergeben sich direkt aus der Informationsarchitektur.

\section{Das Informationsgitter als Fundament der Kosmologie}

Die IWT postuliert, dass der physikalische Raum aus diskreten Informationszellen besteht. 
Jede Zelle besitzt:
\begin{itemize}
    \item eine fundamentale Länge $l_0$,
    \item eine fundamentale Masse-- bzw.\ Energieeinheit $m_0$,
    \item eine Kopplungsstruktur, die durch ein dodekaedrisches Netzwerk beschrieben wird,
    \item eine fraktale Dimension
	\[
        D = \frac{\ln 20}{\ln(2+\varphi)} \approx 2.71.
    \]
\end{itemize}
Diese Struktur definiert eine effektive mittlere Dichte des Universums:
\begin{equation}
    \rho_{\mathrm{eff}} = \frac{m_0}{l_0^D}.
    \label{eq:rho_eff}
\end{equation}
Diese Größe ist nicht frei wählbar, sondern folgt aus der Informationsarchitektur selbst. Die Naturkonstanten $c$, $\hbar$ und $G$ entstehen als emergente
Kopplungsparameter dieses Gitters. Damit ist die effektive Dichte des Universums keine kosmologische Annahme, sondern eine direkte Konsequenz der IWT.

\section{Trägheit, Mach--Prinzip und kosmische Skala}
Die IWT erfüllt das Mach'sche Prinzip in seiner stärksten Form: Trägheit entsteht durch die Kopplung eines Teilchens an die Gesamtheit aller Informationszellen des
Universums. Die effektive Trägheitsenergie eines Teilchens ergibt sich aus der Summe aller Wechselwirkungen:
\begin{equation}
    c^2 = k_M G \int \frac{\rho(r)}{r}\, dV.
    \label{eq:mach_integral}
\end{equation}
Für eine homogene effektive Dichte $\rho_{\mathrm{eff}}$ ergibt sich:
\begin{equation}
    c^2 = 2\pi k_M G \rho_{\mathrm{eff}} R^2,
    \label{eq:mach_radius}
\end{equation}
wobei $R$ der effektive Mach--Radius des Universums ist.

Setzt man \eqref{eq:rho_eff} ein und verwendet die IWT-Relation für die Gravitationskonstante
\begin{equation}
    G \sim \frac{l_0^{D-1}}{m_0},
    \label{eq:G_IWT}
\end{equation}
so folgt:
\begin{equation}
    R = \sqrt{\frac{c^2 l_0}{2\pi k_M}}.
    \label{eq:R_IWT}
\end{equation}
Damit ist der kosmische Radius vollständig durch die Informationszellenstruktur bestimmt.

\section{Rotverschiebung als integrierter Weber-Effekt}
In der IWT ist die kosmologische Rotverschiebung kein geometrisches Phänomen, sondern ein dynamischer Energieverlust eines Photons beim Durchlaufen des gravitativen 
Informationshintergrunds. Die Weber-Gravitation liefert für ein homogenes Medium:
\begin{equation}
    z = \gamma(D)\, G \rho_{\mathrm{eff}} d^2,
    \label{eq:z_IWT}
\end{equation}
wobei $\gamma(D)$ ein dimensionsloser Faktor ist, der die fraktale Struktur berücksichtigt.

Diese Relation ist quadratisch in der Entfernung $d$ und benötigt keine Expansion.

\section{Die Hubble--Konstante aus der Informationsarchitektur}
Vergleicht man \eqref{eq:z_IWT} mit der WG--Kosmologie
\begin{equation}
    z = \frac{3H_0^2 d^2}{2c^2},
    \label{eq:z_WG}
\end{equation}
so ergibt sich:
\begin{equation}
    H_0^2 = \frac{2\gamma}{3} G \rho_{\mathrm{eff}}.
    \label{eq:H0_basic}
\end{equation}

Setzt man \eqref{eq:rho_eff} und \eqref{eq:G_IWT} ein, so folgt:
\begin{equation}
    H_0^2 = \frac{2\gamma}{3} \frac{1}{l_0}.
\end{equation}

Damit ist:
\begin{equation}
    \boxed{
    H_0 = \sqrt{\frac{2\gamma(D)}{3}}\, l_0^{-1/2}
    }
    \label{eq:H0_IWT}
\end{equation}

Die Hubble--Konstante ist somit keine kosmologische Größe, sondern eine direkte Konsequenz der Informationszellenlänge $l_0$.

\section{Eliminieren der Dichte: Die Relation \texorpdfstring{$H_0 = \beta(D)\, c/R$}{H0 = beta(D) c/R}}
Setzt man \eqref{eq:R_IWT} in \eqref{eq:H0_IWT} ein, so ergibt sich:
\begin{equation}
    H_0 = \beta(D)\, \frac{c}{R},
    \label{eq:H0_R_relation}
\end{equation}
mit
\begin{equation}
    \beta(D) = \sqrt{\frac{\gamma(D)}{3\pi k_M(D)}}.
\end{equation}
Damit ist die Hubble--Konstante direkt proportional zur inversen kosmischen Skala.

\section{Entfernung--Rotverschiebungs--Relation}

Setzt man \eqref{eq:H0_R_relation} in \eqref{eq:z_WG} ein, so erhält man:
\begin{equation}
    z = \frac{\alpha(D)}{k_M(D)} \frac{d^2}{R^2},
\end{equation}
und damit:
\begin{equation}
    \boxed{
    d(z) = \sqrt{\frac{k_M(D)}{\alpha(D)}}\, R \sqrt{z}
    }
    \label{eq:d_z_IWT}
\end{equation}

Für $z = 25$:
\begin{equation}
    d(25) = 5 \sqrt{\frac{k_M}{\alpha}}\, R.
\end{equation}

Mit den aus der IWT folgenden Werten $\alpha/k_M \approx 16$ ergibt sich:
\begin{equation}
    d(25) \approx 1.25\, R.
\end{equation}

Für den Mach-Radius $R \approx 4.4 \times 10^{26}\,\mathrm{m}$:
\begin{equation}
    d(25) \approx 5.6 \times 10^{10}\,\mathrm{Lj}.
\end{equation}

\section{Konsequenzen für JWST und moderne Kosmologie}
Die IWT--Kosmologie liefert eine Reihe von Vorhersagen, die sich fundamental von der 
\gls{art} unterscheiden:
\begin{itemize}
    \item Keine kosmische Expansion.
    \item Kein Horizontproblem.
    \item Frühe Galaxien sind natürlich.
    \item $z = 25$ entspricht $\sim 56$ Milliarden Lichtjahren.
    \item Die \gls{cmb} ist ein thermisches Gleichgewicht, kein Urknall--Echo.
    \item $H_0$ ist fundamental und folgt aus $l_0$.
\end{itemize}
Diese Ergebnisse machen die IWT zu einer der wenigen Theorien, die die JWST-Daten ohne Ad-hoc-Erweiterungen erklären können.

\section{Zusammenfassung}

Die Informations-Weber-Theorie liefert eine vollständig konsistente Kosmologie, die auf folgenden Prinzipien beruht:
\begin{itemize}
    \item Der Raum ist ein fraktales Informationsgitter.
    \item Trägheit entsteht durch globale Kopplung (Mach).
    \item Rotverschiebung ist ein dynamischer Weber--Effekt.
    \item Die Hubble--Konstante folgt aus der Informationszellenlänge.
    \item Der kosmische Radius folgt aus dem Mach--Prinzip.
    \item Die Entfernung--Rotverschiebungs--Relation ist quadratisch.
    \item JWST--Beobachtungen bei hohen Rotverschiebungen sind natürlich.
\end{itemize}

Damit ist die IWT nicht nur eine Alternative zur \gls{art}-Kosmologie, sondern eine 
fundamentale Ur-Theorie, die kosmologische Phänomene aus erster Information ableitet.
