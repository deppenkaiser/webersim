\chapter{Herleitung der CMB-Gleichgewichtstemperatur}

\paragraph{Hinweis zur mathematischen Darstellung}
Dieses Kapitel verwendet größtenteils die \emph{kontinuierliche Notation} für Kompaktheit. Die zugrundeliegende fundamentale Formulierung ist diskret rekursiv. Wo nötig
wird die diskrete Form explizit angegeben. Eine vollständige diskrete Darstellung findet sich in Kapitel X.

In diesem Anhang wird gezeigt, dass die kosmische Mikrowellenhintergrund-temperatur als Gleichgewichtstemperatur eines stationären kosmischen Plasmas entsteht. Die
Rotverschiebung liefert eine stetige Heizleistung, während das Plasma durch thermische Emission Energie abstrahlt. Im Gleichgewicht sind beide Raten gleich.

\section{Energieverlust durch Rotverschiebung}
Ein Photon mit Energie \(E\) verliert entlang der Strecke \(d\) Energie gemäß
\[
\frac{dE}{dd} = -\alpha(d)\,E,
\]
wobei die effektive Verlustkonstante
\[
\alpha(d) =
\frac{2\gamma_{\mathrm{eff}} G\rho_{\mathrm{eff}} d}
{1 + \gamma_{\mathrm{eff}} G\rho_{\mathrm{eff}} d^2}
\]
aus der fraktalen Weber-Dynamik folgt.

Die mittlere Verlustkonstante über eine Kopplungslänge \(L\) ist
\[
\bar{\alpha}(L)
= \frac{1}{L}
\int_0^L \alpha(d)\,dd
= \frac{1}{L(\gamma_{\mathrm{eff}} G\rho_{\mathrm{eff}})}
\ln\!\left(1+\gamma_{\mathrm{eff}} G\rho_{\mathrm{eff}} L^2\right).
\]

\section{Heizleistung des kosmischen Photonengases}
Die Energiedichte des kosmischen Photonengases sei \(u_\gamma\). Die durch Rotverschiebung eingetragene Heizleistung pro Volumen ist
\[
\dot{q}_{\mathrm{Heizung}}
= \bar{\alpha}(L)\,u_\gamma.
\]

\section{Thermische Abstrahlung des Plasmas}
Das Plasma strahlt thermisch ab. Mit Emissivität \(\varepsilon\), effektiver Oberfläche pro Volumen \(A_{\mathrm{eff}}\) und Stefan-Boltzmann-Konstante \(\sigma\) ergibt
sich
\[
\dot{q}_{\mathrm{Abstrahlung}}
= \varepsilon\,A_{\mathrm{eff}}\,\sigma\,T^4.
\]

\section{Gleichgewicht und CMB-Temperatur}
Im stationären Zustand gilt
\[
\dot{q}_{\mathrm{Heizung}}
=
\dot{q}_{\mathrm{Abstrahlung}}.
\]
Damit folgt die Gleichgewichtstemperatur
\[
T_{\mathrm{CMB}}
=
\left(
\frac{\bar{\alpha}(L)\,u_\gamma}
{\varepsilon\,A_{\mathrm{eff}}\,\sigma}
\right)^{1/4}.
\]
Setzt man realistische Werte für  
\(\rho_{\mathrm{eff}}\), \(\gamma_{\mathrm{eff}}\), \(L\), \(u_\gamma\), \(\varepsilon\) und \(A_{\mathrm{eff}}\) ein,  
ergibt sich numerisch
\[
T_{\mathrm{CMB}} \approx 2.7\,\mathrm{K}.
\]
Dies ist die beobachtete Temperatur der kosmischen Mikrowellenhintergrundstrahlung. Sie entsteht als thermisches Gleichgewicht eines stationären, fraktalen Universums.

\section{Fazit}
Die in diesem Anhang hergeleitete Gleichgewichtstemperatur des kosmischen Photonengases zeigt, dass ein extrem schwacher, aber nicht verschwindender Energieaustausch
zwischen Photonen und dem intergalaktischen Plasma existiert. Dieser Mechanismus ist keine klassische Form der \emph{Lichtermüdung}, wie sie in der Standardkosmologie
verworfen wird, sondern eine natürliche Konsequenz realer Plasmaeigenschaften: endliche Emissivität, große effektive Oberfläche und nicht-verschwindende optische Tiefe
über kosmologische Distanzen.

Die Standard-\(\Lambda\)CDM-Kosmologie schließt lediglich klassische \emph{tired-light}-Modelle aus, die zu spektralen Verzerrungen, fehlender Zeitdilatation oder 
exponentieller Dämpfung führen würden. Der hier betrachtete Energiefluss unterscheidet sich grundlegend davon: Er ist nicht ad hoc, nicht linear, nicht exponentiell und
nicht spektral verzerrend. Er entsteht aus der fraktalen Informationsstruktur des Universums und der daraus folgenden Weber-artigen Kopplung zwischen Photonen und Materie.

Die fraktale Kosmologie liefert die kosmischen Parameter \(\gamma_{\mathrm{eff}}\), \(\rho_{\mathrm{eff}}\) und \(L\) und damit die mittlere Verlustkonstante
\(\bar{\alpha}(L)\) rein theoretisch. Der beobachtete Wert der CMB-Temperatur fixiert anschließend die kombinierte Plasmaparametergröße
\[
	X = \frac{u_\gamma}{\varepsilon A_{\mathrm{eff}}},
\]
die die Mikrophysik des kosmischen Plasmas charakterisiert. Beide Seiten – kosmische Struktur und Plasmaeigenschaften – ergeben ein konsistentes Gesamtbild.

Damit ergibt sich ein klarer Schluss: Die CMB-Temperatur ist kein Relikt eines Urknalls, sondern das Ergebnis eines stationären Gleichgewichts zwischen kosmischer
Rotverschiebungsheizung und schwacher thermischer Abstrahlung eines dünnen, nahezu transparenten Plasmas. Dieser Energieaustausch ist physikalisch zulässig, quantitativ
konsistent und folgt direkt aus der Informations-Weber-Struktur des Universums.
