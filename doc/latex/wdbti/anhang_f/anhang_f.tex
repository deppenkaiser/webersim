% ============================================================
% Anhang 1 – Informations-Lagrange-Funktional
% ============================================================
\section{Informations-Lagrange-Funktional}
\subsection{Grundstruktur}
Das Informations-Lagrange-Funktional $\mathcal{L}_I$ beschreibt die Dynamik eines Informationszustands $I(\mathbf{x},t)$ und setzt sich aus einem lokalen und einem 
globalen Anteil zusammen:
\[
\mathcal{L}_I = \mathcal{L}_{\text{lokal}} + \mathcal{L}_{\text{global}}.
\]
Der lokale Anteil erzeugt Weber-Dynamik, der globale Anteil Bohm-Dynamik.

\subsection{Lokaler Anteil (Weber-Struktur)}
Der lokale Informationsfluss wird durch Informationsdichte $\rho_I$ und Informationsgradienten bestimmt. Ein minimaler Ansatz lautet:
\[
\mathcal{L}_{\text{lokal}} =
\frac{1}{2}\,\alpha\, \rho_I 
\left( \frac{\partial I}{\partial t} \right)^2
-
\frac{1}{2}\,\beta\, \rho_I 
\left( \nabla I \right)^2.
\]

\subsection{Globaler Anteil (Bohm-Struktur)}
Der globale Anteil ist ein Funktional der Form:
\[
\mathcal{L}_{\text{global}} =
- \gamma \, \rho_I \,
\frac{\nabla^2 \sqrt{\rho_I}}{\sqrt{\rho_I}}.
\]
Dies entspricht strukturell dem Bohm-Potential, jedoch als Informationsoperator.

\subsection{Gesamtes Funktional}
\[
\boxed{
\mathcal{L}_I =
\frac{1}{2}\,\alpha\, \rho_I 
\left( \frac{\partial I}{\partial t} \right)^2
-
\frac{1}{2}\,\beta\, \rho_I 
\left( \nabla I \right)^2
-
\gamma \, \rho_I \,
\frac{\nabla^2 \sqrt{\rho_I}}{\sqrt{\rho_I}}
}
\]
Dieses Funktional liefert durch Variation die Weber-Kraft, das Bohm-Potential, die Kontinuitätsgleichung sowie emergente Energie- und Impulsgrößen.

% ============================================================
% Anhang 2 – Informationsmetrik und emergente Raumzeit
% ============================================================

\section{Informationsmetrik und emergente Raumzeit}
\subsection{Grundidee}
Raum entsteht als effektive Metrik der Kopplungsstruktur eines Informationsnetzes. Die Metrik ergibt sich aus der Kopplungsdichte $C(\mathbf{x})$:
\[
g_{ij}(\mathbf{x}) = f\!\left( C(\mathbf{x}) \right)\, \delta_{ij}.
\]

\subsection*{Fraktale Dimension}
Die fraktale Dimension $D_f$ folgt aus der Skalierung der Kopplungsdichte:
\[
C(\lambda \mathbf{x}) = \lambda^{D_f - 3} C(\mathbf{x}).
\]
Damit gilt:
\[
D_f = 3 \Rightarrow \text{klassischer Raum}, \qquad
D_f < 3 \Rightarrow \text{fraktale Geometrie}.
\]

\subsection{Zeit als Informationsparameter}
Zeit entsteht aus der invertierbaren Transformation des Informationszustands:
\[
t \equiv \tau(I), 
\qquad
\frac{dI}{dt} \neq 0.
\]
Zeit ist somit ein Ordnungsparameter der Informationsentwicklung.

\subsection{Effektive Raumzeit}
Die emergente Raumzeit besitzt die Metrik:
\[
ds^2 = c_{\text{eff}}^2(I)\, dt^2 - g_{ij}(I)\, dx^i dx^j,
\]
mit einer effektiven Lichtgeschwindigkeit:
\[
c_{\text{eff}} = \sqrt{\frac{\alpha}{\beta}}.
\]
Damit ist $c$ ein emergenter Parameter der Informationskopplung.

% ============================================================
% Anhang 3 – Kopplungsparameter und Naturkonstanten
% ============================================================

\section{Kopplungsparameter und Naturkonstanten}
\subsection{Lichtgeschwindigkeit}
\[
c = \sqrt{\frac{\alpha}{\beta}}.
\]
Die maximale Informationsflussrate ergibt sich aus dem Verhältnis der lokalen Kopplungsparameter.

\subsection{Planck-Konstante}
\[
h = k \cdot \gamma,
\]
wobei $k$ ein dimensionsloser Skalierungsfaktor der globalen Informationsorganisation ist.

\subsection{Gravitationskonstante}
\[
G = \frac{1}{4\pi} \frac{\beta}{C_0},
\]
mit $C_0$ als mittlerer Kopplungsdichte des Informationsnetzes.

\subsection{Feinstrukturkonstante}
\[
\alpha_{\text{fs}} = F(\alpha,\beta,\gamma,C_0),
\]
eine reine Funktion der Informationskopplung.

% ============================================================
% Anhang 4 – Numerische Simulation eines Informationsnetzes
% ============================================================
\section{Numerische Simulation eines Informationsnetzes}
\subsection{Diskretisierung}

Das Informationsfeld wird auf einem Gitter $I_{i,j,k}(t)$ definiert.  
Die Kopplungsstruktur ist ein Graph:
\[
C_{(i,j,k),(i',j',k')}.
\]

\subsection{Evolutionsgleichung}
Die Variation von $\mathcal{L}_I$ liefert:
\[
\alpha \rho_I \frac{\partial^2 I}{\partial t^2}
-
\beta \rho_I \nabla^2 I
+
\gamma 
\left( 
\frac{\nabla^2 \sqrt{\rho_I}}{\sqrt{\rho_I}}
\right)' 
= 0.
\]

\subsection{Algorithmus}

\begin{enumerate}
    \item Initialisierung von $I(\mathbf{x},0)$ und $\rho_I(\mathbf{x},0)$.
    \item Berechnung lokaler Gradienten.
    \item Berechnung des globalen Bohm-Terms.
    \item Aktualisierung von $I$ über einen Zeitschritt $\Delta t$.
    \item Wiederholung der Schritte 2–4.
\end{enumerate}

\subsection{Beobachtbare Größen}
\begin{itemize}
    \item emergente Raumzeit,
    \item effektive Lichtgeschwindigkeit,
    \item Gravitationspotential,
    \item Wellenphänomene,
    \item Nichtlokalität,
    \item Rotationskurven,
    \item CMB-Struktur.
\end{itemize}
