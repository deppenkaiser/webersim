\chapter{Die dynamische Gleichung der Informationsmetrik}

\paragraph{Hinweis zur mathematischen Darstellung}
Dieses Kapitel verwendet größtenteils die \emph{kontinuierliche Notation} für Kompaktheit. Die zugrundeliegende fundamentale Formulierung ist diskret rekursiv. Wo nötig
wird die diskrete Form explizit angegeben. Eine vollständige diskrete Darstellung findet sich in Kapitel X.

In diesem Anhang wird die fundamentale dynamische Gleichung der Informationsmetrik hergeleitet, die die Informations-Weber-Theorie (IWT) zu einer geschlossenen, 
selbstkonsistenten Urtheorie erweitert. Diese Gleichung vereinigt lokale Weber-Dynamik, globale Bohm-Struktur und fraktale Informationsgeometrie in einem einzigen Ausdruck. 
Sie bildet den dynamischen Kern der Theorie, aus dem Raum, Zeit, Energie, Gravitation, Quantenstruktur und kosmologische Skalierung emergieren.

\section{Ausgangspunkt: Das Informations-Lagrange-Funktional}
Die IWT basiert auf einem Informationsfeld \(I(x,t)\), dessen Dynamik durch ein Informations-Lagrange-Funktional beschrieben wird. Dieses setzt sich aus drei Beiträgen 
zusammen:
\[
\mathcal{L}[I] 
= 
\mathcal{L}_{\mathrm{lokal}}
+
\mathcal{L}_{\mathrm{global}}
+
\mathcal{L}_{\mathrm{fraktal}}.
\]

\subsection{Lokaler Anteil: Weber-Struktur}
Der lokale Anteil beschreibt die direkte Wechselwirkung im Sinne der Weber-Dynamik:
\[
\mathcal{L}_{\mathrm{lokal}}
=
\frac{1}{2} g^{ij} \partial_i I \partial_j I.
\]
Er erzeugt Trägheit, klassische Dynamik und lokale Informationsflüsse.

\subsection{Globaler Anteil: Bohm-Struktur}
Der globale Anteil beschreibt die nichtlokale Organisationsstruktur:
\[
\mathcal{L}_{\mathrm{global}}
=
-\frac{\lambda}{2}\frac{\nabla^2 I}{I}.
\]
Dieser Term erzeugt Wellenphänomene, Nichtlokalität und quantenartige Kohärenz.

\subsection{Fraktaler Anteil: Kosmische Skalierung}
Die fraktale Informationsarchitektur des Universums führt zu einer logarithmischen Skalierungsinvarianz:
\[
\mathcal{L}_{\mathrm{fraktal}}
=
\mu \ln\!\left(1+\gamma_{\mathrm{eff}} G \rho_{\mathrm{eff}} L^2\right).
\]
Dieser Term ist verantwortlich für kosmische Rotverschiebung, die Verlustkonstante \(\bar{\alpha}(L)\) und die CMB-Gleichgewichtstemperatur.

\section{Variation nach der Informationsmetrik}
Da die Metrik \(g_{ij}\) nicht vorgegeben ist, sondern emergent, folgt ihre Dynamik aus der Variation des Funktionals:
\[
\frac{\delta \mathcal{L}}{\delta g_{ij}} = 0.
\]
Die Variation der drei Beiträge ergibt:
\[
\frac{d}{dt} g_{ij}
=
\partial_i I \partial_j I
-
\lambda\,\frac{\partial_i\partial_j I}{I}
+
\mu\,g_{ij}\,\ln\!\left(1+\gamma_{\mathrm{eff}} G \rho_{\mathrm{eff}} L^2\right).
\]

\section{Die fundamentale Gleichung der Informationsmetrik}
Damit ergibt sich die dynamische Urgleichung der IWT:
\[
\boxed{
\frac{d}{dt} g_{ij}
=
\partial_i I \partial_j I
-
\lambda\,\frac{\partial_i\partial_j I}{I}
+
\mu\,g_{ij}\,\ln\!\left(1+\gamma_{\mathrm{eff}} G \rho_{\mathrm{eff}} L^2\right)
}
\]
Diese Gleichung vereinigt:
\begin{itemize}
    \item lokale Weber-Dynamik (\(\partial_i I \partial_j I\)),
    \item globale Bohm-Struktur (\(\partial_i\partial_j I / I\)),
    \item fraktale kosmische Skalierung (logarithmischer Term).
\end{itemize}
Sie ist die erste vollständig geschlossene dynamische Gleichung, aus der die gesamte\\Informations-Weber-Theorie emergiert.

\section{Grenzfälle und physikalische Interpretation}
\subsection{Klassischer Grenzfall}
Für schwache Informationsgradienten dominiert der lokale Term:
\[
\frac{d}{dt} g_{ij} \approx \partial_i I \partial_j I.
\]
Dies reproduziert klassische Mechanik und Weber-Dynamik.

\subsection{Quantenmechanischer Grenzfall}
Für stark gekrümmte Informationsfelder dominiert der globale Term:
\[
\frac{d}{dt} g_{ij} \approx -\lambda\,\frac{\partial_i\partial_j I}{I}.
\]
Dies reproduziert das Bohm-Potential und quantenartige Kohärenz.

\subsection{Kosmologischer Grenzfall}
Für große Skalen dominiert der fraktale Term:
\[
\frac{d}{dt} g_{ij} \approx 
\mu\,g_{ij}\,\ln\!\left(1+\gamma_{\mathrm{eff}} G \rho_{\mathrm{eff}} L^2\right).
\]
Dies erzeugt:
\begin{itemize}
    \item kosmische Rotverschiebung,
    \item die Verlustkonstante \(\bar{\alpha}(L)\),
    \item das CMB-Gleichgewicht.
\end{itemize}

\section{Diskrete Form: Dynamik des Informationsnetzes}
Für das diskrete Informationsnetz der WDBT+ ergibt sich die Update-Regel:
\[
g_{ij}(t+\Delta t)
=
g_{ij}(t)
+
\Delta t\,
\Bigl[
\partial_i I \partial_j I
-
\lambda\,\frac{\partial_i\partial_j I}{I}
+
\mu\,g_{ij}\,\ln\!\left(1+\gamma_{\mathrm{eff}} G \rho_{\mathrm{eff}} L^2\right)
\Bigr].
\]
Damit ist die Gleichung numerisch implementierbar.

\section{Konsequenzen für Naturkonstanten}
Die Urgleichung liefert die Grundlage für die Herleitung der Naturkonstanten:
\begin{itemize}
    \item \(c\) als maximale Informationsflussrate,
    \item \(\hbar\) als globale Informationsgranularität,
    \item \(G\) als Kopplungsparameter der Informationsmetrik,
    \item \(\alpha\) als fraktale Skalierungsinvariante.
\end{itemize}
Diese Konstanten sind keine Eingaben, sondern emergente Größen.

\section{Konsequenzen für Kosmologie}
Die Urgleichung reproduziert:
\begin{itemize}
    \item Rotverschiebung ohne Expansion,
    \item die kosmische Verlustkonstante,
    \item das CMB-Gleichgewicht,
    \item fraktale Struktur des Universums,
    \item galaktische Rotationskurven ohne Dunkle Materie.
\end{itemize}

\section{Schlussbemerkung}
Mit der dynamischen Gleichung der Informationsmetrik liegt erstmals eine vollständig geschlossene, selbstkonsistente Urgleichung vor, aus der Raum, Zeit, Energie, Gravitation, 
Quantenstruktur und kosmologische Phänomene emergieren. Die Informations-Weber-Theorie wird damit zu einer echten Urtheorie der Physik.
