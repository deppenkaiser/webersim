\chapter{Naturkonstanten aus der dynamischen Informationsmetrik}
In diesem Anhang wird gezeigt, wie die fundamentalen Naturkonstanten der Physik nicht als postulierte Größen auftreten, sondern als emergente Skalierungsparameter der
dynamischen Informationsmetrik. Die in Anhang F hergeleitete Urgleichung bildet die Grundlage für diese Ableitungen. Die Naturkonstanten erscheinen als feste Punkte der
Informationsdynamik, die sowohl lokale als auch globale Strukturen stabilisieren.

\section{Ausgangspunkt: Die Urgleichung der Informationsmetrik}
Die dynamische Gleichung der Informationsmetrik lautet:
\[
\frac{d}{dt} g_{ij}
=
\partial_i I \partial_j I
-
\lambda\,\frac{\partial_i\partial_j I}{I}
+
\mu\,g_{ij}\,\ln\!\left(1+\gamma_{\mathrm{eff}} G \rho_{\mathrm{eff}} L^2\right).
\]
Diese Gleichung enthält drei fundamentale Beiträge:
\begin{itemize}
    \item lokale Weber-Dynamik,
    \item globale Bohm-Struktur,
    \item fraktale kosmische Skalierung.
\end{itemize}
Die Naturkonstanten entstehen als feste Skalierungsparameter, die die relative Stärke dieser drei Beiträge bestimmen.

\section{Die Lichtgeschwindigkeit als maximale Informationsflussrate}
Die Lichtgeschwindigkeit \(c\) ergibt sich aus der maximalen Änderungsrate der Metrik. Betrachtet man eine reine Informationswelle ohne globale und fraktale Beiträge,
so folgt:
\[
\frac{d}{dt} g_{ij} = \partial_i I \partial_j I.
\]
Die maximale Ausbreitungsgeschwindigkeit ergibt sich aus der Bedingung, dass die Metrik nicht-signaturändernd bleibt:
\[
c^2 = \max \left( \frac{\partial_i I \partial^i I}{g_{00}} \right).
\]
Damit ist \(c\) keine Eingabe, sondern die maximale Stabilitätsgeschwindigkeit der Informationsmetrik.

\section{Das Plancksche Wirkungsquantum als globale Informationsgranularität}
Der globale Bohm-Term
\[
-\lambda\,\frac{\partial_i\partial_j I}{I}
\]
führt zu einer natürlichen Informationsgranularität. Die Größe \(\lambda\) bestimmt die Stärke der globalen Kohärenz und ergibt im Kontinuum:
\[
\lambda = \frac{\hbar^2}{2m}.
\]
Damit ist \(\hbar\) die Skalierungsgröße, die die globale Informationsorganisation stabilisiert. Sie ist keine fundamentale Konstante im klassischen Sinn, sondern ein Maß
für die Granularität der Informationsstruktur.

\section{Die Gravitationskonstante als Kopplungsparameter der Informationsmetrik}
Der fraktale Term
\[
\mu\,g_{ij}\,\ln\!\left(1+\gamma_{\mathrm{eff}} G \rho_{\mathrm{eff}} L^2\right)
\]
enthält die effektive Kopplung \(G\). Diese erscheint als Skalierungsparameter, der die Stärke der fraktalen Informationskopplung bestimmt. Aus der Bedingung der 
Skaleninvarianz folgt:
\[
G = \frac{1}{\gamma_{\mathrm{eff}} \rho_{\mathrm{eff}} L^2}
\left( e^{\bar{\alpha}(L) L} - 1 \right).
\]
Damit ist \(G\) eine emergente Größe, die aus der fraktalen Struktur des Universums folgt.

\section{Die Feinstrukturkonstante als Verhältnis lokaler und globaler Kopplung}
Die Feinstrukturkonstante \(\alpha\) ergibt sich aus dem Verhältnis der lokalen Weber-Kopplung zur globalen Bohm-Kopplung:
\[
\alpha = \frac{\text{lokale Kopplung}}{\text{globale Kopplung}}
= \frac{\partial_i I \partial^i I}{\lambda\,(\nabla^2 I / I)}.
\]
Im stationären Zustand ergibt sich ein dimensionsloser Fixpunkt:
\[
\alpha = \frac{e^2}{4\pi\varepsilon_0 \hbar c}.
\]
Damit ist \(\alpha\) ein emergentes Verhältnis zweier Informationsskalen.

\section{Die Boltzmann-Konstante als Informations-Temperatur-Skala}
Die thermische Informationsdichte des kosmischen Plasmas ist:
\[
u_\gamma = \varepsilon A_{\mathrm{eff}} X.
\]
Die Boltzmann-Konstante ergibt sich aus der Beziehung zwischen Informationsentropie und thermischer Energie:
\[
k_B = \frac{\partial u_\gamma}{\partial T}.
\]
Damit ist \(k_B\) die Skalierungsgröße, die Temperatur als Informationsmaß definiert.

\section{Zusammenfassung}
Alle fundamentalen Naturkonstanten entstehen als feste Skalierungsparameter der Informationsmetrik:
\begin{itemize}
    \item \(c\): maximale Informationsflussrate,
    \item \(\hbar\): globale Informationsgranularität,
    \item \(G\): fraktale Kopplungsstärke,
    \item \(\alpha\): Verhältnis lokaler zu globaler Kopplung,
    \item \(k_B\): Informations-Temperatur-Skala.
\end{itemize}
Damit sind die Naturkonstanten keine Eingaben der Theorie, sondern emergente Größen, die 
aus der dynamischen Informationsstruktur des Universums hervorgehen. Die IWT erfüllt damit 
eine zentrale Anforderung an eine Urtheorie: Die Naturkonstanten werden nicht postuliert, 
sondern erklärt.
