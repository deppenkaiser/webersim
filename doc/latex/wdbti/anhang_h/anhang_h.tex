\chapter{Axiome der Informations-Weber-Theorie}

\paragraph{Hinweis zur mathematischen Darstellung}
Dieses Kapitel verwendet größtenteils die \emph{kontinuierliche Notation} für Kompaktheit. Die zugrundeliegende fundamentale Formulierung ist diskret rekursiv. Wo nötig
wird die diskrete Form explizit angegeben. Eine vollständige diskrete Darstellung findet sich in Kapitel X.

Die Informations-Weber-Theorie (IWT) ist eine fundamentale Urtheorie, in der Raum, Zeit, Energie, Dynamik und Naturkonstanten als emergente Größen aus der Struktur und
Dynamik eines universellen Informationsfeldes hervorgehen. Dieser Anhang formuliert die Theorie in axiomatischer Form. Alle späteren Ergebnisse – klassische Mechanik,
Quantenmechanik, Gravitation, Kosmologie und Naturkonstanten – folgen aus diesen Axiomen.

\section{Axiom 1: Existenz eines universellen Informationsfeldes}
Es existiert ein skalare Informationsfeld
\[
I(x,t),
\]
das die vollständige physikalische Realität beschreibt. Materie, Energie, Wellen, Geometrie und Dynamik sind Manifestationen der Struktur und Veränderung dieses Feldes.

\section{Axiom 2: Informationsmetrik als emergente Geometrie}
Die physikalische Raumzeit ist nicht fundamental. Stattdessen entsteht eine effektive Metrik
\[
g_{ij}(x,t)
\]
aus der lokalen und globalen Struktur des Informationsfeldes. Die Metrik ist dynamisch und wird nicht vorgegeben, sondern durch die Informationsdynamik bestimmt.

\section{Axiom 3: Variationsprinzip der Informationsdynamik}
Die Dynamik des Informationsfeldes folgt aus einem universellen Informations-Lagrange-Funktional
\[
\mathcal{L}[I] 
= 
\mathcal{L}_{\mathrm{lokal}}
+
\mathcal{L}_{\mathrm{global}}
+
\mathcal{L}_{\mathrm{fraktal}},
\]
mit
\[
\mathcal{L}_{\mathrm{lokal}}
=
\frac{1}{2} g^{ij} \partial_i I \partial_j I,
\qquad
\mathcal{L}_{\mathrm{global}}
=
-\frac{\lambda}{2}\frac{\nabla^2 I}{I},
\qquad
\mathcal{L}_{\mathrm{fraktal}}
=
\mu \ln\!\left(1+\gamma_{\mathrm{eff}} G \rho_{\mathrm{eff}} L^2\right).
\]

\section{Axiom 4: Dynamische Gleichung der Informationsmetrik}
Die Metrik entsteht aus der Variation des Funktionals nach \(g_{ij}\). Die fundamentale Gleichung der Informationsmetrik lautet:
\[
\boxed{
\frac{d}{dt} g_{ij}
=
\partial_i I \partial_j I
-
\lambda\,\frac{\partial_i\partial_j I}{I}
+
\mu\,g_{ij}\,\ln\!\left(1+\gamma_{\mathrm{eff}} G \rho_{\mathrm{eff}} L^2\right)
}
\]
Sie vereint:
\begin{itemize}
    \item lokale Weber-Dynamik,
    \item globale Bohm-Struktur,
    \item fraktale kosmische Skalierung.
\end{itemize}

\section{Axiom 5: Energieerhaltung als Informationsfluss}
Energie ist keine fundamentale Größe. Sie entsteht als Erhaltungsgröße des Informationsflusses. Die Energieerhaltung folgt aus der Zeitinvarianz des 
Informations-Lagrange-Funktionals:
\[
\frac{d}{dt} E[I] = 0.
\]

\section{Axiom 6: Naturkonstanten als emergente Skalierungsparameter}
Die fundamentalen Naturkonstanten sind keine Eingaben der Theorie, sondern feste Punkte der Informationsdynamik:
\[
c = \text{maximale Informationsflussrate},
\qquad
\hbar = \text{globale Informationsgranularität},
\]

\[
G = \text{fraktale Kopplungsstärke},
\qquad
\alpha = \text{Verhältnis lokaler zu globaler Kopplung},
\qquad
k_B = \text{Informations-Temperatur-Skala}.
\]

\section{Axiom 7: Fraktale Skalierungsinvarianz des Universums}
Die großskalige Struktur des Universums ist fraktal mit effektiver Dimension
\[
D \approx 2.71.
\]
Diese fraktale Struktur bestimmt die kosmische Rotverschiebung, die Verlustkonstante \(\bar{\alpha}(L)\) und die CMB-Gleichgewichtstemperatur.

\section{Axiom 8: Emergenz von Raum, Zeit und Dynamik}
Raum, Zeit und Dynamik sind emergente Eigenschaften der Informationsmetrik. Die physikalische Zeit entsteht als Ordnungsstruktur der Informationsänderung:
\[
dt \propto \sqrt{g_{00}}\,d\tau.
\]
Klassische Mechanik, Quantenmechanik und Gravitation sind Grenzfälle der Informationsdynamik.

\section{Axiom 9: Universelle Gültigkeit}
Die Informations-Weber-Theorie gilt auf allen Skalen:
\begin{itemize}
    \item mikroskopisch (Quantenstruktur),
    \item mesokopisch (klassische Mechanik),
    \item makroskopisch (Gravitation),
    \item kosmologisch (Rotverschiebung, CMB, Struktur).
\end{itemize}

\section{Schlussbemerkung}
Diese neun Axiome definieren die Informations-Weber-Theorie vollständig. Alle physikalischen Phänomene – von der Quantenmechanik über die Gravitation bis zur
Kosmologie – folgen aus der Struktur und Dynamik des Informationsfeldes und der daraus emergierenden Metrik. Die IWT ist damit eine konsistente, geschlossene und
vollständig emergente Urtheorie.
