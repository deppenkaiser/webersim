\chapter{Tabellen, Symbole und Definitionen}

\section{Symbolverzeichnis}

\begin{tabular}{ll}
$\rho_I$ & Informationsdichte \\
$\vec{J}_I$ & Informationsfluss \\
$\mathcal{F}$ & Informationsdichtefunktional \\
$Q$ & Bohm-Potential \\
$\Phi_I$ & Informationspotential \\
$g_{ij}$ & Informationsmetrische Komponenten \\
$D$ & fraktale Dimension \\
\end{tabular}

\section{Glossar}

\textbf{Informationsraum:}  
Abstrakter Raum aller Informationsverteilungen.

\textbf{Informationsgeometrie:}  
Emergente Metrikstruktur des physikalischen Raumes.

\textbf{Systemische Kausalität:}  
Globale Informationsorganisation.

\section{Wichtige Gleichungen}
\[
    \partial_t \rho_I + \nabla \cdot \vec{J}_I = 0
\]

\[
    Q = -\frac{\hbar^2}{2m}
    \frac{\nabla^2 \sqrt{\rho_I}}{\sqrt{\rho_I}}
\]

\[
    \vec{F}_{\text{Weber}}
    =
    \frac{q_1 q_2}{4\pi\varepsilon_0 r^2}
    \left[
        1
        -
        \frac{\dot{r}^2}{c^2}
        +
        \frac{2 r \ddot{r}}{c^2}
    \right]
    \hat{\vec{r}}
\]
