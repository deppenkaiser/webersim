\chapter{Tabellen, Symbole und Definitionen}
Dieser Anhang fasst alle in der Informations-Weber-Theorie verwendeten Symbole, Begriffe und zentralen Gleichungen zusammen. Er dient als Referenz für die mathematische
Struktur der Theorie und erleichtert das Verständnis der in den vorhergehenden Kapiteln entwickelten Konzepte.

\section{Symbolverzeichnis}
\begin{table}[h!]
\centering
\begin{tabular}{ll}
\textbf{Symbol} & \textbf{Bedeutung} \\
\hline
$I(x,t)$ & Universelles Informationsfeld \\
$p_I$ & Informationsdichte \\
$g_{ij}$ & Informationsmetrik (emergente Geometrie) \\
$g^{ij}$ & Inverse Informationsmetrik \\
$\partial_i I$ & Lokaler Informationsgradient \\
$\nabla^2 I$ & Laplace-Operator des Informationsfeldes \\
$\lambda$ & Stärke der globalen Bohm-Kopplung \\
$\mu$ & Stärke der fraktalen Skalierungskopplung \\
$\gamma_{\mathrm{eff}}$ & Effektive fraktale Kopplung \\
$\rho_{\mathrm{eff}}$ & Effektive Massendichte des Universums \\
$L$ & Kosmische Skala (Mach-Radius) \\
$D$ & Fraktale Dimension des Universums \\
$K_M(D)$ & Mach-Kopplungsfaktor der fraktalen Struktur \\
$\bar{\alpha}(L)$ & Kosmische Verlustkonstante \\
$u_\gamma$ & Energiedichte des Photonengases \\
$\varepsilon$ & Emissivität des kosmischen Plasmas \\
$A_{\mathrm{eff}}$ & Effektive Oberfläche pro Volumen \\
$X$ & Kombinierte Plasmaparametergröße $u_\gamma/(\varepsilon A_{\mathrm{eff}})$ \\
$c$ & Maximale Informationsflussrate (Lichtgeschwindigkeit) \\
$\hbar$ & Globale Informationsgranularität \\
$G$ & Gravitationskonstante (fraktale Kopplungsstärke) \\
$\alpha$ & Feinstrukturkonstante \\
$k_B$ & Informations-Temperatur-Skala \\
$T_{\mathrm{CMB}}$ & Gleichgewichtstemperatur des kosmischen Plasmas \\
$z$ & Rotverschiebung \\
\hline
\end{tabular}
\end{table}

\section{Glossar}
\begin{itemize}
    \item \textbf{Informationsfeld} — Fundamentale Größe der IWT; alle physikalischen 
          Phänomene sind Manifestationen seiner Struktur und Dynamik.
    \item \textbf{Informationsmetrik} — Aus dem Informationsfeld emergente Geometrie; 
          definiert Raum, Zeit und Dynamik.
    \item \textbf{Informationsfluss} — Grundlegende Form der Dynamik; Energie ist eine 
          abgeleitete Erhaltungsgröße.
    \item \textbf{Weber-Dynamik} — Lokale direkte Wechselwirkung, die Trägheit und 
          klassische Mechanik erzeugt.
    \item \textbf{Bohm-Struktur} — Globale Organisationsdynamik, die Wellenphänomene 
          und Nichtlokalität hervorbringt.
    \item \textbf{Fraktale Informationsgeometrie} — Großskalige Struktur des Universums 
          mit effektiver Dimension $D \approx 2.71$.
    \item \textbf{Kosmische Verlustkonstante} — Logarithmische Kopplungsgröße, die die 
          Rotverschiebungsheizung bestimmt.
    \item \textbf{Kombinierte Plasmaparametergröße $X$} — Zusammenfassung der 
          mikrophysikalischen Plasmaeigenschaften.
    \item \textbf{Mach-Radius $L$} — Kosmische Skala, die die fraktale Kopplung bestimmt.
    \item \textbf{Informationszeit} — Emergent aus der Dynamik der Informationsmetrik; 
          physikalische Zeit ist Ordnungsstruktur der Informationsänderung.
\end{itemize}

\section{Wichtige Gleichungen}
\subsection{Dynamische Gleichung der Informationsmetrik}
\[
\frac{d}{dt} g_{ij}
=
\partial_i I \partial_j I
-
\lambda\,\frac{\partial_i\partial_j I}{I}
+
\mu\,g_{ij}\,\ln\!\left(1+\gamma_{\mathrm{eff}} G \rho_{\mathrm{eff}} L^2\right).
\]

\subsection{Informations-Lagrange-Funktional}
\[
\mathcal{L}[I] 
= 
\frac{1}{2} g^{ij} \partial_i I \partial_j I
-
\frac{\lambda}{2}\frac{\nabla^2 I}{I}
+
\mu \ln\!\left(1+\gamma_{\mathrm{eff}} G \rho_{\mathrm{eff}} L^2\right).
\]

\subsection{Kosmische Verlustkonstante}
\[
\bar{\alpha}(L)
=
\frac{1}{L\,\gamma_{\mathrm{eff}} G \rho_{\mathrm{eff}}}
\ln\!\bigl(1+\gamma_{\mathrm{eff}} G \rho_{\mathrm{eff}} L^2\bigr).
\]

\subsection{CMB-Gleichgewicht}
\[
T_{\mathrm{CMB}}^4
=
\frac{\bar{\alpha}(L)\,u_\gamma}
{\varepsilon\,A_{\mathrm{eff}}\,\sigma}.
\]

\subsection{Kombinierte Plasmaparametergröße}
\[
X = \frac{u_\gamma}{\varepsilon A_{\mathrm{eff}}}.
\]

\subsection{Weber-Gravitation (massive Körper)}
\[
F_{\mathrm{WG}}
=
-\frac{GMm}{r^2}
\left(
1 - \frac{\dot{r}^2}{2c^2} + \frac{r\ddot{r}}{c^2}
\right).
\]

\subsection{Bohm-Potential}
\[
Q = -\frac{\hbar^2}{2m}\frac{\nabla^2 R}{R}.
\]

\section{Schlussbemerkung}
Dieser Anhang fasst die mathematischen und begrifflichen Grundlagen der Informations-Weber-Theorie zusammen. Er dient als Referenz für die in den vorhergehenden 
Kapiteln entwickelten Strukturen und ermöglicht eine konsistente Anwendung der Theorie auf physikalische, kosmologische und mathematische Fragestellungen.
