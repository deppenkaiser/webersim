\chapter{Einleitung}
\label{chap:einleitung}

\section{Information als fundamentale Größe der Physik}
Die zentrale These dieses Buches lautet: \textbf{Information ist die grundlegende physikalische Größe, aus der Energie, Raum, Zeit und Dynamik emergieren}. Während die
klassische Physik Energie als fundamentale Erhaltungsgröße betrachtet, zeigt sich in modernen Theorien zunehmend, dass Energie selbst nur eine abgeleitete Form von
Information ist – eine Maßzahl für die Organisation, Struktur und Veränderbarkeit physikalischer Zustände. In der \gls{qm} beschreibt die Wellenfunktion keine
materielle Welle, sondern eine Informationsverteilung über mögliche Zustände. In der Thermodynamik ist Entropie ein Maß für fehlende Information. In der Relativitätstheorie
bestimmt die Energie-Impuls-Verteilung die Geometrie der Raumzeit – doch diese Verteilung ist letztlich eine Informationsstruktur. Selbst die Lichtgeschwindigkeit erscheint
weniger als fundamentale Konstante, sondern als Eigenschaft eines Informationsflusses in einem strukturierten Medium.

Die \gls{wdbt} bietet einen natürlichen Zugang zu dieser Sichtweise. Sie verbindet direkte Wechselwirkungen (Weber), wellenartige Informationsfelder (De Broglie) und
nichtlokale Organisationsprinzipien (Bohm). In dieser Synthese wird deutlich, dass die Dynamik physikalischer Systeme nicht primär durch Kräfte, Felder oder Geometrien
bestimmt wird, sondern durch die \emph{Transformation von Information}.

Die in diesem Buch entwickelte \textbf{Informations-Weber-Theorie} hebt diesen Gedanken auf eine fundamentale Ebene. Sie interpretiert die bekannten physikalischen Größen
als Informationsfunktionale und zeigt, dass die Erhaltung von Information die eigentliche Grundlage der Energieerhaltung ist. Die scheinbaren Widersprüche zwischen
\gls{qm} und Relativitätstheorie lösen sich auf, wenn man beide als unterschiedliche Manifestationen eines universellen Informationsprinzips versteht.

Diese Perspektive bildet den Ausgangspunkt für die folgenden Kapitel. Erst vor diesem Hintergrund wird verständlich, warum alternative Modelle – etwa die
\gls{wed}, die Bohmsche Mechanik oder fraktale Raumstrukturen – nicht exotische Randphänomene sind, sondern Hinweise auf eine tiefere, informationsbasierte
Ordnung der Natur.

\section{Motivation}
Die moderne Physik steht vor grundlegenden Widersprüchen: Während die \gls{art} die Gravitation als Krümmung der Raumzeit beschreibt, basiert die \gls{srt} auf
idealisierten Inertialsystemen, die in einer gekrümmten Raumzeit streng genommen nicht existieren können. Dieser Konflikt wirft Fragen auf – etwa zur Natur der
Lichtgeschwindigkeit, die in der \gls{srt} absolut ist, in der \gls{art} jedoch lokal variabel.

\begin{quote}
\enquote{Einstein's postulates contain inherent contradictions when applied to real gravitational systems, challenging the universality of special relativity.} \cite{Rubcic1998}
\end{quote}

Hinzu kommen ungelöste Probleme der \gls{qm}: der Welle-Teilchen-Dualismus, der \enquote{Kollaps} der Wellenfunktion bei Messungen und nichtlokale Verschränkung.
Selbst erfolgreiche Theorien wie die \gls{qed} postulieren scheinbar paradoxe Phänomene, etwa virtuelle Photonen mit Überlichtgeschwindigkeit im Pfadintegralformalismus.

Diese Spannungen deuten darauf hin, dass die etablierten Modelle möglicherweise nur Annäherungen an eine tiefere Realität sind. Statt Dogmen zu folgen, sollten wir
alternative Perspektiven prüfen – wie die \gls{wed} oder die \gls{dbt}, die in diesem Buch vorgestellt werden.

\begin{quote}
\enquote{The observer-dependent collapse of the wavefunction is not a fundamental feature of nature but a limitation of the standard interpretation.} \cite{bohm1952}
\end{quote}

\subsection{Dogmatismus und blinde Flecken der modernen Physik}
Die heutige Physik leidet unter einer paradoxen Situation: Einerseits werden etablierte Theorien wie die \gls{art} oder die Quantenfeldtheorie kaum hinterfragt, obwohl sie
fundamentale Schwächen aufweisen – insbesondere Singularitäten in starken Gravitationsfeldern, unendliche Selbstenergien von Teilchen oder die Notwendigkeit \enquote{dunkler}
Entitäten. Andererseits werden unorthodoxe Ansätze oft bereits im Peer-Review aussortiert, obwohl sie Lösungen für diese Probleme bieten könnten.

Ein Beispiel ist die Interpretation der \gls{cmb} als Beweis für den Urknall. Alternative Erklärungen – etwa thermische Gleichgewichtsprozesse in Plasmen – werden kaum
diskutiert, obwohl sie ohne Singularitäten auskommen. Ähnlich verhält es sich mit der Rotverschiebung von Galaxien, die nicht zwingend auf eine Expansion des Universums
hindeuten muss.

\begin{quote}
\enquote{Theoretical physics has become stuck in a paradigm that values mathematical elegance over empirical testability, leading to a stagnation of genuine progress.} \cite{Smolin2006}
\end{quote}

\subsection{Spekulation statt Fortschritt}
Seit den revolutionären Durchbrüchen der \gls{qm} und Relativitätstheorie vor einem Jahrhundert gab es kaum vergleichbare Fortschritte. Stattdessen dominieren
spekulative Konzepte wie höhere Dimensionen oder Multiversen, die empirisch kaum überprüfbar sind.

Doch Wissenschaft sollte sich auf beobachtbare Phänomene konzentrieren. Die Weber-Elektrodynamik zeigt, wie sich elektromagnetische Effekte ohne Felder beschreiben
lassen – durch direkte Wechselwirkungen zwischen Ladungen. Solche Ansätze könnten den Weg zu einer konsistenteren Physik ebnen.

\subsection{Alternative Theorien}
Ein zentrales Problem der modernen Physik liegt in ihrem übermäßigen Vertrauen in die Mathematik. Nur weil etwas mathematisch formulierbar ist, muss es noch lange nicht
der physikalischen Realität entsprechen. Doch statt diese Grenzen anzuerkennen, werden grundlegende Prinzipien der klassischen Physik – wie Energieerhaltung oder die
Gesetze der Thermodynamik – zugunsten abstrakter Gleichungen aufgegeben.

Die \gls{art} beispielsweise postuliert eine dynamische Raumzeit, die Gravitationswellen ermöglicht und im schwachen Feld äußerst erfolgreich ist. Gleichzeitig führt sie im
starken Feld zu echten Singularitäten, die physikalisch problematisch sind. Die Weber-Gravitation bleibt in allen Feldstärken regulär und liefert eine direkte dynamische
Beschreibung ohne geometrische Interpretation.

Konkrete Widersprüche zeigen sich in der Praxis: Nach der \gls{art} müssten Planeten durch die Abstrahlung von Gravitationswellen Energie verlieren – doch warum sind
Planetenbahnen dann über Milliarden Jahre stabil? Wenn die Raumzeit als elastisches Gebilde beschrieben wird, das sich verformen und bewegen lässt: Welche Kraft verrichtet
hier Arbeit, und woher kommt die Energie dafür?

Auch die vermeintlichen Beweise für den Urknall sind keineswegs so eindeutig, wie oft behauptet wird. Die kosmische \gls{cmb} wird automatisch als Echo des Urknalls
interpretiert – doch es gibt alternative Erklärungen, etwa thermische Gleichgewichtsprozesse oder Streuphänomene.

\begin{quote}
\enquote{The interpretation of cosmic microwave background as proof of the Big Bang ignores alternative explanations, such as intrinsic redshifts in plasma cosmology.} \cite{Arp1998}
\end{quote}

Ebenso könnte die Rotverschiebung von Galaxien nicht nur durch Expansion, sondern auch durch andere Mechanismen verursacht werden. Selbst Phänomene wie die Lichtablenkung
oder der Shapiro-Effekt lassen sich ohne \gls{art} erklären, wenn man alternative Gravitationsmodelle zulässt.

\begin{quote}
\enquote{Weber's formulation of electrodynamics provides a consistent framework for gravitational phenomena without invoking curved spacetime.} \cite{WeberElectrodynamics}
\end{quote}

In diesem Buch sollen solche alternativen Erklärungen aufgezeigt werden. Die Physik darf nicht bei mathematischen Dogmen stehen bleiben – sie muss sich wieder auf Logik,
Experiment und echte Kausalität besinnen.

\section{Abweichende Perspektiven in der Physik: Licht, Relativität und alternative Modelle}
\subsection{Feynmans Teilchenmodell des Lichts}
Richard Feynman argumentierte, dass selbst Interferenzphänomene durch Teilchen (Photonen) erklärbar sind – ohne Wellen. Dies wirft die Frage auf: Ist der
Welle-Teilchen-Dualismus wirklich notwendig, oder spiegelt er nur die Grenzen unserer Modelle wider?

\subsection{Widersprüche in der QED: Überlichtschnelle Photonen und Pfadintegrale}
Der Pfadintegralformalismus der \gls{qed} summiert über alle möglichen Photonenpfade – inklusive solcher mit Überlichtgeschwindigkeit. Mathematisch führt dies zu korrekten
Vorhersagen, doch physikalisch bleibt unklar:

\begin{itemize}
\item Können virtuelle Photonen schneller als Licht sein, ohne die \gls{srt} zu verletzen?
\item Ist die Lichtgeschwindigkeit wirklich eine absolute Grenze, oder nur ein makroskopischer Effekt?
\end{itemize}

\subsection{Energieabhängige Lichtgeschwindigkeit? Experimentelle Hinweise}
Einige alternative Theorien (z. B. Schleifenquantengravitation oder VSL-Modelle) schlagen vor, dass die Lichtgeschwindigkeit von der Photonenenergie abhängen könnte.

Mögliche Indizien:
\begin{itemize}
\item Gammablitze mit extrem hohen Energien zeigen minimale Laufzeitunterschiede.
\item Quantengravitationseffekte könnten bei hohen Energien zu Dispersion führen.
\end{itemize}

\begin{quote}
\enquote{The constancy of the speed of light is not an immutable law but a parameter that may vary under extreme conditions, offering solutions to cosmological puzzles.} \cite{Magueijo2003}
\end{quote}

\section{Die Entwicklung des Wellenkonzepts in der Physik}
Das Verständnis von Wellen in der Physik hat sich im Laufe der Zeit radikal gewandelt. Während klassische Wellen wie Schall oder Wasserwellen als Störungen eines
materiellen Mediums beschrieben werden konnten, führten elektromagnetische Wellen und Quantenphänomene zu grundlegenden Umbrüchen. Maxwell zeigte 1865, dass Licht sich als
elektromagnetische Welle auch ohne Äther ausbreitet – was die Frage aufwarf, wie Energie ohne Trägermedium transportiert wird. Die \gls{srt} etablierte die
Lichtgeschwindigkeit als absolute Grenze, während die \gls{art} sie als lokal variabel beschreibt – ein scheinbarer Widerspruch, den alternative Theorien wie die
Weber-Elektrodynamik zu lösen versuchen.

Die Quantenphysik revolutionierte das Wellenkonzept weiter: De Broglie verband Teilchen- und Welleneigenschaften, und die \gls{qed} beschreibt Photonen als Felder mit
überlichtschnellen Pfadintegral-Komponenten. Doch diese mathematische Eleganz wirft physikalische Deutungsprobleme auf – etwa die Rolle des Beobachters beim Kollaps der
Wellenfunktion oder die nicht-lokale Natur der Quantenverschränkung.

Auch Gravitationswellen in der \gls{art} bleiben rätselhaft: Wenn Raumzeit als schwingendes Medium gilt, woher stammt die Energie für ihre Verformung?

Diese Widersprüche zeigen, dass die etablierten Theorien möglicherweise nur Annäherungen an eine tiefere Wahrheit sind.

\section{Wellenphänomene: Die Dualität von instantaner Ganzheit und lokaler Ausbreitung}
Wellen besitzen eine einzigartige Doppelnatur: lokale Ausbreitung und instantane globale Struktur. Diese Dualität zeigt sich besonders deutlich in fundamentalen
Wechselwirkungen.

Die newtonsche Mechanik postuliert mit \enquote{actio = reactio} eine instantane Fernwirkung:
\begin{equation}
\vec{F}_{12} = -\vec{F}_{21}
\end{equation}

Das Coulombsche Gesetz zeigt dieselbe Struktur:
\begin{equation}
\vec{F} = \frac{1}{4\pi\epsilon_0}\frac{q_1 q_2}{r^2}\hat{\vec{r}}
\end{equation}

Interferenzphänomene wie das Doppelspaltexperiment zeigen, dass Wellen sich global so organisieren, dass die Gesamtenergie minimiert wird:
\begin{equation}
|\Psi(x)|^2 = |\psi_1(x) + \psi_2(x)|^2
\end{equation}

Die \gls{wed} erweitert das Coulombsche Gesetz um geschwindigkeits- und beschleunigungsabhängige Terme:
\begin{equation}
\vec{F} = \frac{q_1 q_2}{4\pi\epsilon_0 r^2}
\left[
1 - \frac{\dot{r}^2}{c^2} + \frac{2 r \ddot{r}}{c^2}
\right]
\hat{\vec{r}}
\end{equation}

\section{Das erweiterte Kausalitätskonzept}
Die Physik benötigt einen erweiterten Kausalitätsbegriff, der sowohl lokale Dynamik (Energietransport) als auch systemische Ganzheit (globale Organisation) umfasst.

Das Bohm’sche Quantenpotential
\begin{equation}
Q(\vec{r},t) = -\frac{\hbar^2}{2m}
\frac{\nabla^2 \sqrt{\rho(\vec{r},t)}}{\sqrt{\rho(\vec{r},t)}}
\end{equation}
wirkt instantan und global, während die Weber-Kraft lokale retardierte Effekte beschreibt.

Diese duale Struktur löst zahlreiche konzeptionelle Probleme der modernen Physik und bildet die Grundlage für die in diesem Buch entwickelte Theorie.

\section{Axiome der Informations-Weber-Theorie}
Die in diesem Buch entwickelte Informations-Weber-Theorie basiert auf einer kleinen Anzahl klar formulierter Grundannahmen, die als Axiome dienen. Sie ersetzen die
Vielzahl unvereinbarer Postulate der modernen Physik durch ein einheitliches, informationsbasiertes Fundament.

\subsection*{Axiom I: Der physikalische Zustand ist ein Informationszustand}
Jedes physikalische System wird durch eine Informationsverteilung beschrieben. Größen wie Energie, Impuls oder Ladung sind abgeleitete Funktionale dieser Verteilung.

\subsection*{Axiom II: Information ist eine Erhaltungsgröße}
Die Zeitentwicklung eines Systems ist eine invertierbare Transformation des Informationszustands. Nichts geht verloren, nichts entsteht aus dem Nichts. Energieerhaltung
ist ein Spezialfall dieses Prinzips.

\subsection*{Axiom III: Dynamik ist Informationsfluss}
Die Bewegung von Teilchen, Feldern oder Wellen ergibt sich aus der Umlagerung von Information. Die Weber-Kraft beschreibt lokale Informationsflüsse, das Bohmsche
Quantenpotential globale.

\subsection*{Axiom IV: Raum ist eine emergente Informationsgeometrie}
Der physikalische Raum ist keine Grundgröße, sondern die effektive Metrik der Kopplungsstruktur des Informationsnetzes. Seine fraktale Dimension ist eine Eigenschaft dieser Struktur.

\subsection*{Axiom V: Kausalität besitzt zwei Ebenen}
Lokale Kausalität beschreibt den Energietransport mit endlicher Geschwindigkeit. Systemische Kausalität beschreibt die instantane Organisation des Informationszustands.
Beide sind komplementär.

\section{Aufbau und Zielsetzung dieses Buches}
Dieses Buch verfolgt zwei zentrale Ziele. Erstens soll gezeigt werden, dass die \gls{wdbt} eine konsistente Grundlage für eine alternative Beschreibung von Gravitation und
\gls{qm} bildet, in der direkte Wechselwirkungen und nichtlokale Informationsstrukturen vereinigt werden. Zweitens entwickelt es eine informationsbasierte Urtheorie, in der
Energie, Raum, Zeit und Dynamik als abgeleitete Größen eines zugrunde liegenden Informationsnetzes erscheinen.

Der Aufbau des Buches folgt der endgültigen Struktur:
\begin{enumerate}
\item \textbf{Einleitung}  
Motivation, Kritik der modernen Physik, Einführung des Informationsbegriffs und Formulierung der Axiome der Informations-Weber-Theorie.
\item \textbf{Der Informationszustand}  
Definition der Informationsdichte, Informationsflüsse, Kontinuitätsgleichung und Informationsfunktionale.
\item \textbf{Die klassische Weber-Elektrodynamik}  
Historische Einordnung, Lagrange-Ansatz, Herleitung und Interpretation der Weber-Kraft als lokaler Grenzfall.
\item \textbf{Informations-Lagrange-Funktional}  
Mathematische Grundstruktur der Theorie: vollständiges Funktional, Variation, Euler-Lagrange-Gleichungen, Zerlegung in lokale und globale Beiträge.
\item \textbf{Informationsmetrik und emergente Raumzeit}  
Herleitung der effektiven Metrik, fraktale Dimension, emergente Zeit und diskrete Informationsgeometrie.
\item \textbf{Emergenz klassischer und quantenmechanischer Phänomene}  
Trägheit, Gravitation, Wellenphänomene, Nichtlokalität und systemische Kausalität als emergente Informationsdynamik.
\item \textbf{Vergleich mit etablierten Theorien}  
Einordnung von \gls{art}, \gls{srt}, \gls{qm}, \gls{qed} und \gls{qft} als Grenzfälle der Informations-Weber-Theorie.
\item \textbf{Naturkonstanten aus Informationsarchitektur}  
Herleitung von $c$, $\hbar$, $G$, $\alpha$ und weiteren Konstanten aus Informationskopplungen und fraktaler Struktur.
\item \textbf{Konsequenzen der Informations-Weber-Theorie}  
Kosmologie, Rotverschiebung, \gls{cmb}, Big Bounce und fraktale Struktur des Universums im informationsbasierten Rahmen.
\item \textbf{Numerische Simulation eines Informationsnetzes}  
Diskretisierung, Evolutionsgleichungen, Algorithmus und beobachtbare Größen in der digitalen \gls{wdbt}+.
\end{enumerate}

Diese Struktur ermöglicht es, die Informations-Weber-Theorie sowohl als Weiterentwicklung der klassischen \gls{wdbt} als auch als eigenständige fundamentale Theorie zu verstehen.

\section{Zusammenfassung der Einleitung}
Die Einleitung dieses Buches hat gezeigt, dass die moderne Physik trotz ihrer beeindruckenden Erfolge vor grundlegenden konzeptionellen Problemen steht. Widersprüche
zwischen \gls{srt} und \gls{art}, ungelöste Fragen der \gls{qm}, spekulative Erweiterungen wie höhere Dimensionen oder Multiversen sowie die Abhängigkeit von
mathematischen Konstruktionen ohne klare physikalische Interpretation deuten darauf hin, dass die etablierten Theorien nur Näherungen an eine tiefere Realität darstellen.

Der zentrale Gedanke dieses Werkes lautet, dass \textbf{Information die fundamentale physikalische Größe} ist, aus der Energie, Raum, Zeit und Dynamik emergieren. Energie
erscheint in dieser Sichtweise nicht als primäre Erhaltungsgröße, sondern als Ausdruck der Struktur und Organisation eines Informationszustands.
Die Weber–De-Broglie–Bohm-Theorie liefert hierfür einen natürlichen Ausgangspunkt, da sie direkte Wechselwirkungen, wellenartige Informationsfelder und nichtlokale
Organisationsprinzipien vereint.

Die formulierten Axiome der Informations-Weber-Theorie bilden das Fundament dieser neuen Perspektive. Sie beschreiben physikalische Systeme als Informationsverteilungen,
deren Zeitentwicklung durch invertierbare Transformationen bestimmt wird. Dynamik wird als Umlagerung von Information verstanden, während Raum als emergente Geometrie der
Kopplungsstruktur erscheint. Die Kausalität besitzt zwei Ebenen: eine lokale Dynamik des Energietransports und eine systemische Ganzheit, die die globale Organisation des
Informationszustands bestimmt.

Diese Einleitung bereitet damit den Boden für die folgenden Kapitel. Kapitel~\ref{chap:einleitung} verankert Motivation und Axiome, Kapitel~2 führt den Informationszustand
formal ein, Kapitel~3 zeigt die \gls{wed} als lokalen Grenzfall. Kapitel~4 und~5 etablieren die mathematische Grundstruktur der Theorie über
Informations-Lagrange-Funktional und Informationsmetrik. Kapitel~6 und~7 zeigen, wie klassische und quantenmechanische Phänomene als emergente Informationsdynamik
erscheinen und wie etablierte Theorien als Grenzfälle eingeordnet werden. Kapitel~8–10 widmen sich Naturkonstanten, kosmologischen Konsequenzen und numerischen
Simulationen des Informationsnetzes.

Damit ist der Rahmen gesetzt, in dem die Informations-Weber-Theorie im weiteren Verlauf systematisch entwickelt, mathematisch präzisiert und mit bestehenden Theorien
verglichen wird.
