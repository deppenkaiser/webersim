\chapter{Numerische Simulation der Informationsgeometrie}
\label{chap:simulation}

\paragraph{Hinweis zur mathematischen Darstellung}
Dieses Kapitel verwendet größtenteils die \emph{kontinuierliche Notation} für Kompaktheit. Die zugrundeliegende fundamentale Formulierung ist diskret rekursiv. Wo nötig
wird die diskrete Form explizit angegeben. Eine vollständige diskrete Darstellung findet sich in Kapitel X.

\section{Einleitung}
Die Informations-Weber-Theorie ist nicht nur analytisch formuliert, sondern auch algorithmisch implementierbar. Da Raum, Zeit und Dynamik aus einem diskreten
Informationsnetz emergieren, lässt sich die Theorie direkt numerisch simulieren.

Dieses Kapitel beschreibt die Grundprinzipien einer solchen Simulation:
\begin{itemize}
    \item die diskrete Informationsstruktur,
    \item die Aktualisierungsregeln,
    \item die Berechnung der Informationsmetrik,
    \item die Simulation lokaler und globaler Dynamik,
    \item die Rekonstruktion emergenter Größen wie Raum, Zeit und Wellen.
\end{itemize}
Damit wird die Informations-Weber-Theorie operational: Sie kann berechnet, visualisiert und experimentell getestet werden.

\section{Das diskrete Informationsnetz}
Die digitale Informations-Weber-Theorie beschreibt den physikalischen Zustand durch ein Netzwerk aus:
\begin{itemize}
    \item \textbf{Knoten} $i$ mit Informationswerten $\rho_I(i)$,
    \item \textbf{Kopplungen} $K_{ij}$ zwischen Knoten,
    \item \textbf{Aktualisierungsregeln} für $\rho_I$ und $K_{ij}$.
\end{itemize}
Dieses Netz ist kein Abbild des Raumes — es \emph{erzeugt} den Raum.

\subsection{Knoten}
Jeder Knoten repräsentiert eine elementare Informationszelle. Die Informationsdichte $\rho_I(i)$ ist die fundamentale Größe.

\subsection{Kopplungen}
Die Kopplungen $K_{ij}$ bestimmen:
\begin{itemize}
    \item die Stärke lokaler Informationsflüsse,
    \item die Reichweite globaler Informationsorganisation,
    \item die emergente Geometrie.
\end{itemize}

\subsection{Aktualisierungsregeln}
Die Dynamik besteht aus diskreten Aktualisierungsschritten:
\[
    \rho_I^{(n+1)} = T[\rho_I^{(n)}].
\]
Die Transformation $T$ ist die diskrete Version des Informations-Lagrange-Funktionals.

\section{Diskrete Form des Informations-Lagrange-Funktionals}
Das kontinuierliche Funktional aus Kapitel~\ref{chap:lagrange} wird diskretisiert:
\[
    \mathcal{L}_I
    =
    \sum_i
    \mathcal{F}
    \!\left(
        \rho_I(i),\,
        \Delta \rho_I(i),\,
        \delta_t \rho_I(i)
    \right).
\]
Dabei sind:
\begin{itemize}
    \item $\Delta \rho_I(i)$ diskrete Gradienten,
    \item $\delta_t \rho_I(i)$ zeitliche Differenzen.
\end{itemize}
Die Variation führt zu diskreten Euler-Lagrange-Gleichungen:
\[
    \delta_t
    \left(
        \frac{\partial \mathcal{F}}{\partial (\delta_t \rho_I)}
    \right)
    +
    \Delta
    \left(
        \frac{\partial \mathcal{F}}{\partial (\Delta \rho_I)}
    \right)
    -
    \frac{\partial \mathcal{F}}{\partial \rho_I}
    = 0.
\]
Diese Gleichungen bestimmen die Aktualisierungsregeln.

\section{Berechnung der Informationsmetrik}
Die Informationsmetrik aus Kapitel~\ref{chap:informationsmetrik} wird diskretisiert:
\[
    g_{ij}
    =
    \frac{\partial^2 \mathcal{F}}
    {\partial (\Delta_i \rho_I)\, \partial (\Delta_j \rho_I)}.
\]
Die Metrik entsteht also aus der Sensitivität der Informationsstruktur gegenüber räumlichen Änderungen.

\subsection{Rekonstruktion des Raumes}
Der emergente Raum entsteht durch:
\begin{enumerate}
    \item Berechnung der Metrik $g_{ij}$,
    \item Bestimmung der effektiven Abstände,
    \item Einbettung in eine dreidimensionale Darstellung.
\end{enumerate}
Damit wird der physikalische Raum aus der Informationsstruktur rekonstruiert.

\section{Simulation lokaler Dynamik}
Die lokale Dynamik entspricht der Weber-Struktur als lokaler Projektion der Informationsgeometrie:
\begin{itemize}
    \item geschwindigkeitsabhängige Kopplungen,
    \item beschleunigungsabhängige Reaktionskräfte,
    \item lokale Informationsflüsse.
\end{itemize}
Diese entstehen aus dem lokalen Anteil des diskreten Funktionals.

\section{Simulation globaler Dynamik}
Die globale Dynamik entspricht dem Bohm-Potential als systemischer Projektion:
\begin{itemize}
    \item globale Informationsorganisation,
    \item nichtlokale Kopplungen,
    \item Interferenzstrukturen.
\end{itemize}
Diese entstehen aus dem globalen Anteil des Funktionals.

\section{Emergenz von Wellen und Gravitationsmoden}
Wellen entstehen als kollektive Moden der Informationsgeometrie:
\begin{itemize}
    \item lokale Moden → elektromagnetische Wellen,
    \item globale Moden → Quantenwellen,
    \item geometrische Moden → Gravitationswellen.
\end{itemize}
Die Simulation zeigt:
\begin{itemize}
    \item Interferenzmuster,
    \item fraktale Skalierungsstrukturen,
    \item dispersive Effekte,
    \item frequenzabhängige Ablenkung.
\end{itemize}

\section{Numerische Stabilität und fraktale Skalierung}
Die fraktale Dimension
\[
    D = \frac{\ln 20}{\ln(2+\phi)}
\]
bestimmt:
\begin{itemize}
    \item die Skalierung der Kopplungen,
    \item die Stabilität der Simulation,
    \item die Übergänge zwischen Mikro- und Makrophysik.
\end{itemize}

\section{Beispiel: Simulation eines Doppelspalts}
Die Simulation reproduziert:
\begin{itemize}
    \item Interferenz ohne Wellenfunktion,
    \item Nichtlokalität ohne Kollaps,
    \item fraktale Feinstrukturen im Muster.
\end{itemize}

\section{Beispiel: Simulation eines Gravitationspotentials}
Die Simulation zeigt:
\begin{itemize}
    \item emergente Newton-Potentiale,
    \item fraktale Abweichungen bei kleinen Skalen,
    \item frequenzabhängige Lichtablenkung.
\end{itemize}

\section{Zusammenfassung}
Kapitel~\ref{chap:simulation} hat gezeigt:
\begin{itemize}
    \item Die Informations-Weber-Theorie ist numerisch simulierbar.
    \item Raum, Zeit und Dynamik entstehen aus einem diskreten Informationsnetz.
    \item Die Informationsmetrik ermöglicht die Rekonstruktion des Raumes.
    \item Lokale und globale Dynamik entstehen aus dem diskreten Lagrange-Funktional.
    \item Wellen, Gravitation und Nichtlokalität sind kollektive Moden der Informationsgeometrie.
\end{itemize}
Damit ist die Theorie nicht nur konzeptionell und analytisch, sondern auch algorithmisch vollständig formuliert.
