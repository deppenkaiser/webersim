\chapter{Kosmologie in der Informations-Weber-Theorie}
\label{chap:kosmologie}

\section{Einleitung}
Die Informations-Weber-Theorie liefert eine alternative kosmologische Struktur, die ohne fundamentale Annahmen wie Urknall-Singularität, Dunkle Materie oder Dunkle Energie
auskommt. Statt eines expandierenden Raumes beschreibt sie das Universum als dynamisches Informationsnetz, dessen fraktale Geometrie und Kopplungsstruktur die beobachteten
kosmologischen Phänomene hervorbringen.

Die zentrale Idee lautet:
\[
    \textbf{Kosmologie ist Informationsdynamik auf größten Skalen.}
\]
In diesem Kapitel werden die wichtigsten kosmologischen Konsequenzen der Theorie hergeleitet:

\begin{itemize}
    \item CMB-Struktur als fossilierte Informationsgeometrie,
    \item Rotverschiebung als informationsdynamischer Effekt,
    \item galaktische Rotationskurven ohne Dunkle Materie,
    \item großskalige Strukturbildung aus fraktaler Informationskopplung,
    \item Gravitationslinsen als Moden der Informationsgeometrie,
    \item Big Bounce statt Urknall.
\end{itemize}

\section{Das Universum als Informationsnetz}
In der digitalen Informations-Weber-Theorie besteht das Universum aus einem Netzwerk von Informationsknoten und Kopplungen. Die fraktale Dimension
\[
    D = \frac{\ln 20}{\ln(2+\phi)}
\]
bestimmt die Skalierungsstruktur dieses Netzes. Raum, Zeit und Dynamik sind emergente Eigenschaften dieser Informationsarchitektur.

\subsection{Keine fundamentale Raumzeit}
Die Theorie ersetzt:
\begin{itemize}
    \item die Raumzeit der ART durch eine emergente Informationsgeometrie,
    \item die Expansion des Raumes durch Informationsflüsse,
    \item Singularitäten durch fraktale Kernstrukturen.
\end{itemize}

\subsection{Kosmologie als fraktale Informationsdynamik}
Die großskalige Struktur des Universums entsteht aus:
\begin{itemize}
    \item fraktaler Kopplung,
    \item globaler Informationsorganisation,
    \item kollektiven Moden der Informationsgeometrie.
\end{itemize}

\section{CMB-Struktur als fossilierte Informationsgeometrie}
Die kosmische Hintergrundstrahlung (CMB) wird in der Standardkosmologie als thermisches Relikt des Urknalls interpretiert. In der Informations-Weber-Theorie kann sie als
fossilierte Struktur der frühen Informationsgeometrie eines großskaligen Plasmas verstanden werden.

\subsection{Fraktale Signaturen}
Die Theorie sagt voraus:
\begin{itemize}
    \item die CMB-Anisotropien besitzen fraktale Skalierungsgesetze,
    \item die Fluktuationen sind nicht rein thermisch,
    \item die Muster spiegeln die Kopplungsstruktur des frühen Informationsnetzes wider.
\end{itemize}

\subsection{Keine Inflation notwendig}
Die Informations-Weber-Theorie benötigt keine Inflation, da:
\begin{itemize}
    \item globale Informationskopplung instantan wirkt,
    \item Homogenität und Isotropie aus systemischer Ganzheit folgen,
    \item die fraktale Struktur die beobachteten Skalenrelationen erklärt.
\end{itemize}

\section{Rotverschiebung ohne Expansion}
Die Standardkosmologie interpretiert die Rotverschiebung als Folge der Expansion des Universums. Die Informations-Weber-Theorie bietet eine alternative Erklärung:

\subsection{Informationsdynamische Rotverschiebung}
Ein Photon verliert Energie durch:
\begin{itemize}
    \item fraktale Kopplungsprozesse,
    \item Informationsumstrukturierung entlang seines Weges,
    \item dispersive Effekte der Informationsgeometrie.
\end{itemize}
\[
    z = z_{\text{info}}(\text{Pfadlänge}, D, \text{Kopplungsstruktur})
\]

\subsection{Keine kosmische Expansion notwendig}
Die beobachtete Rotverschiebung kann ohne:
\begin{itemize}
    \item expandierenden Raum,
    \item Dunkle Energie,
    \item kosmologische Konstante
\end{itemize}
erklärt werden, wenn Informationsprozesse entlang des Photonenpfades berücksichtigt werden.

\section{Galaktische Rotationskurven ohne Dunkle Materie}
Die Informations-Weber-Theorie erklärt flache Rotationskurven durch:
\begin{itemize}
    \item fraktale Verstärkung der gravitativen Informationsflüsse,
    \item zusätzliche effektive Beschleunigungen aus der Informationsgeometrie,
    \item nichtlokale Kopplungen im galaktischen Plasma.
\end{itemize}
Damit wird Dunkle Materie als eigenständige Substanz überflüssig.

\section{Großskalige Strukturbildung}
Die fraktale Informationsarchitektur erzeugt:
\begin{itemize}
    \item Filamente,
    \item Voids,
    \item Cluster,
    \item Jets,
    \item selbstähnliche Muster über viele Skalen.
\end{itemize}
Diese Strukturen entstehen nicht primär aus Gravitationsinstabilitäten, sondern aus:
\begin{itemize}
    \item fraktaler Kopplung,
    \item Plasma-Informationsdynamik,
    \item globaler Organisation.
\end{itemize}

\section{Gravitationslinsen als Moden der Informationsgeometrie}
Gravitationslinsen entstehen in der Informations-Weber-Theorie nicht durch Krümmung eines ontologischen Raumes, sondern durch Moden der Informationsgeometrie.

\subsection{Vorhersage: Frequenzabhängige Lichtablenkung}
Die Theorie sagt:
\begin{itemize}
    \item hochfrequente Photonen werden stärker abgelenkt,
    \item niederfrequente Photonen werden schwächer abgelenkt.
\end{itemize}
Dies ist ein klarer Unterschied zur ART.

\section{Big Bounce statt Urknall}
Die Informations-Weber-Theorie kennt keine Singularitäten. Statt eines Urknalls entsteht ein zyklisches Modell:
\begin{itemize}
    \item Kontraktion des Informationsnetzes,
    \item fraktale Kernstruktur verhindert Singularität,
    \item Reorganisation der Informationskopplungen,
    \item Expansion der Informationsgeometrie.
\end{itemize}
Dies ist ein \emph{Big Bounce}, kein Urknall.

\section{Zusammenfassung}
Kapitel~\ref{chap:kosmologie} hat gezeigt:
\begin{itemize}
    \item Kosmologie ist Informationsdynamik auf größten Skalen.
    \item Die CMB-Struktur kann als fossilierte Informationsgeometrie verstanden werden.
    \item Rotverschiebung entsteht ohne Expansion.
    \item Rotationskurven entstehen ohne Dunkle Materie.
    \item Strukturbildung folgt aus fraktaler Informationskopplung.
    \item Gravitationslinsen sind Moden der Informationsgeometrie.
    \item Das Universum beginnt nicht mit einem Urknall, sondern mit einem Big Bounce.
\end{itemize}
Damit liefert die Informations-Weber-Theorie eine konsistente, fraktale und informationsbasierte Kosmologie.
