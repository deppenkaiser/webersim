\chapter{Kosmologie in der Informations-Weber-Theorie}
\label{chap:kosmologie}

\paragraph{Hinweis zur mathematischen Darstellung}
Dieses Kapitel verwendet größtenteils die \emph{kontinuierliche Notation} für Kompaktheit. Die zugrundeliegende fundamentale Formulierung ist diskret rekursiv. Wo nötig
wird die diskrete Form explizit angegeben. Eine vollständige diskrete Darstellung findet sich in Kapitel X.

\section{Einleitung}
Die Informations-Weber-Theorie (IWT) beschreibt das Universum nicht als expandierende Raumzeit, sondern als dynamisches Informationsnetz, dessen fraktale Struktur Raum,
Zeit, Gravitation und Materie hervorbringt. Die IWT ist eine Steady-State-Theorie: Materie entsteht kontinuierlich durch quantenmechanische Prozesse, während die
Gesamtenergie des Universums endlich und konstant bleibt. Das Universum ist beliebig alt und besitzt unendlich viel potenziellen Raum, jedoch nicht unendlich viel Energie.

Die kosmologische Struktur der IWT ergibt sich aus der Kopplung dreier Ebenen:
\begin{enumerate}
    \item der Weber-Gravitation als lokaler Informationsfluss,
    \item dem Bohm'schen Quantenpotential als globaler Organisationsmechanismus,
    \item der fraktalen Informationsarchitektur des Raumes.
\end{enumerate}
Diese Kombination führt zu einer Kosmologie ohne Urknall, ohne Expansion und ohne Singularitäten. Die beobachteten Phänomene entstehen aus der Informationsgeometrie und
den daraus emergierenden Wechselwirkungen.

\section{Das Universum als Informationsnetz}
Die IWT postuliert ein diskretes Informationsgitter, dessen Knoten und Kopplungen die Grundstruktur des Universums bilden. Die fraktale Dimension
\[
    D = \frac{\ln(2+\phi)}{\ln 2} \approx 2.71
\]
bestimmt die Skalierungsgesetze der Informationsgeometrie. Raum ist keine ontologische Entität, sondern die effektive Metrik der Kopplungsstruktur. Zeit ist ein
Ordnungsparameter der Aktualisierung des Informationsnetzes.

\subsection{Steady-State der Materie}
Materie entsteht kontinuierlich durch quantenmechanische Prozesse, die durch das Quantenpotential und die fraktale Informationsstruktur begünstigt werden. Es gibt keine
kosmische Anfangsphase, keine thermische Frühzeit und keine Expansion des Raumes. Die Gesamtenergie des Universums bleibt konstant.

\subsection{Zyklische Informationsarchitektur}
Während die materielle Ebene stationär ist, kann die Informationsarchitektur selbst zyklische Reorganisationen durchlaufen. Diese Zyklen sind nicht physikalisch sichtbar
und betreffen nicht die Materie, sondern die Struktur des Informationsnetzes. Sie entsprechen einem „informationalen Bounce“, der jedoch keine kosmologische Signatur
hinterlässt.

\section{CMB-Struktur als thermisches Gleichgewicht}
Die kosmische Mikrowellenstrahlung wird in der IWT nicht als Relikt einer Frühzeit interpretiert, sondern als thermisches Gleichgewicht eines unendlichen kosmischen Plasmas.
Die fraktale Informationsstruktur erzeugt charakteristische Anisotropien, die mit den beobachteten Mustern übereinstimmen.

\subsection{Fraktale Signaturen}
Die IWT sagt voraus:
\begin{itemize}
    \item fraktale Skalierung der Dichtefluktuationen,
    \item anisotrope Muster bei großen Winkeln,
    \item Abweichungen von rein thermischen Spektren.
\end{itemize}
Diese Eigenschaften ergeben sich aus der fraktalen Informationsgeometrie und der Wechselwirkung der Plasmaelemente über die Weber-Kraft.

\section{Rotverschiebung ohne Expansion}
Die kosmologische Rotverschiebung entsteht in der IWT nicht durch eine Expansion des Raumes, sondern durch einen informationsdynamischen Energieverlust von Photonen beim
Durchlaufen des kosmischen Hintergrundes. Die Weber-Gravitation führt zu einer quadratischen Entfernung–Rotverschiebungs-Relation:
\[
    z = \gamma(D)\, G\, \rho_{\text{eff}}\, d^2.
\]
Damit ist die Rotverschiebung ein integrierter Weber-Effekt und kein Hinweis auf eine kosmische Expansion.

\section{Die Hubble-Konstante aus der Informationsarchitektur}
Vergleicht man die informationsdynamische Rotverschiebung mit der klassischen Form
\[
    z = \frac{3H^2}{2c^2}\, d^2,
\]
so ergibt sich
\[
    H^2 = G\, \rho_{\text{eff}}.
\]
Die Hubble-Konstante ist somit kein kosmologischer Parameter, sondern ein emergenter Skalierungsfaktor der Informationsarchitektur. Sie hängt von der
Informationszellengröße $l_0$ und der fraktalen Dimension $D$ ab.

\section{Entfernung–Rotverschiebungs-Relation}
Die IWT liefert eine quadratische Entfernung–Rotverschiebungs-Relation:
\[
    d(z) = \sqrt{\frac{\kappa_M}{\alpha(D)}}\, R\, \sqrt{z},
\]
wobei $R$ der Mach-Radius des Universums ist. Für hohe Rotverschiebungen, etwa $z=25$, ergibt sich
\[
    d(25) \approx 1.25\, R.
\]
Damit sind die vom JWST beobachteten frühen Galaxien vollständig konsistent mit einer steady-state-Kosmologie.

\section{Galaktische Rotationskurven ohne Dunkle Materie}
Die IWT erklärt flache Rotationskurven durch:
\begin{itemize}
    \item fraktale Verstärkung gravitativer Informationsflüsse,
    \item nichtlokale Kopplungen im galaktischen Plasma,
    \item zusätzliche effektive Beschleunigungen aus der Informationsgeometrie.
\end{itemize}
Dunkle Materie ist nicht erforderlich.

\section{Mach-Prinzip und kosmische Skala}
Die IWT erfüllt das Mach-Prinzip in strenger Form: Trägheit entsteht durch die Kopplung eines Teilchens an alle Informationszellen des Universums. Die Trägheitsenergie
ergibt sich aus:
\[
    c^2 = 2 \kappa_M\, G\, \rho_{\text{eff}}\, R^2.
\]
Daraus folgt der kosmische Radius:
\[
    R = \sqrt{\frac{c^2}{2 \kappa_M G \rho_{\text{eff}}}}.
\]

\section{Konsequenzen für JWST und moderne Kosmologie}
Die IWT erklärt:
\begin{itemize}
    \item hohe Rotverschiebungen ohne Expansion,
    \item keine Notwendigkeit einer kosmischen Frühzeit,
    \item keine Inflation,
    \item die CMB als thermisches Gleichgewicht,
    \item die Hubble-Konstante als emergent,
    \item Rotverschiebung als Weber-Effekt.
\end{itemize}

\section{Zusammenfassung}
Die Informations-Weber-Theorie beschreibt das Universum als fraktales Informationsnetz, in dem Materie kontinuierlich entsteht und die Gesamtenergie konstant bleibt. Die
Rotverschiebung ist ein informationsdynamischer Weber-Effekt, die CMB ein thermisches Gleichgewicht eines unendlichen Plasmas, und die Hubble-Konstante ein emergenter
Skalierungsparameter. Die kosmologische Struktur ist steady-state auf der materiellen Ebene und kann zyklisch auf der Informations-Ebene sein, ohne dass diese Zyklen
physikalisch sichtbar werden. Damit liefert die IWT eine konsistente, fraktale und informationsbasierte Kosmologie ohne Urknall, ohne Expansion und ohne Dunkle Materie.
