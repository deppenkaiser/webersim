\chapter{Beispiele und Anwendungen}
\label{chap:beispiele}

\section{Einleitung}
Die Informations-Weber-Theorie ist nicht nur ein konzeptioneller Rahmen, sondern eine operationalisierbare physikalische Theorie. Dieses Kapitel zeigt anhand konkreter
Beispiele, wie klassische, quantenmechanische und gravitative Phänomene aus der Informationsdynamik hervorgehen. Die Beispiele dienen drei Zielen:
\begin{enumerate}
    \item Demonstration der praktischen Anwendbarkeit der Theorie,
    \item Vergleich mit etablierten Modellen,
    \item Vorbereitung numerischer Simulationen (Kapitel~\ref{chap:simulation}).
\end{enumerate}
Die Beispiele sind so gewählt, dass sie die drei Ebenen der Theorie illustrieren:
\begin{itemize}
    \item lokale Dynamik (Weber-Struktur),
    \item globale Dynamik (Bohm-Potential),
    \item emergente Geometrie (Informationsmetrik).
\end{itemize}

\section{Beispiel 1: Der Doppelspalt als Informationsprozess}
Der Doppelspalt ist ein klassisches Beispiel für Interferenz und Nichtlokalität. In der\\Informations-Weber-Theorie entsteht das Interferenzmuster aus der globalen
Optimierung des Informationsfunktionals:
\[
    \mathcal{F}_{\text{global}}
    =
    \gamma \frac{(\nabla \rho_I)^2}{\rho_I}.
\]

\subsection{Informationsdichte hinter dem Spalt}
Die Informationsdichte ergibt sich aus:
\[
    \rho_I = \rho_1 + \rho_2 + 2\sqrt{\rho_1 \rho_2}\cos(\Delta\phi).
\]
Interpretation:
\begin{itemize}
    \item keine Wellenfunktion als ontologisches Objekt,
    \item keine Superposition im klassischen Sinn,
    \item Interferenz als energetisch optimale Informationsorganisation.
\end{itemize}

\subsection{Nichtlokalität ohne Kollaps}
Die globale Informationsstruktur bestimmt das Muster instantan, ohne Energie zu übertragen. Damit entsteht:
\begin{itemize}
    \item Nichtlokalität ohne Kausalitätsverletzung,
    \item deterministische Trajektorien (Bohm-Pfade),
    \item vollständige Reproduzierbarkeit des Interferenzmusters.
\end{itemize}

\section{Beispiel 2: Harmonischer Oszillator}
Der harmonische Oszillator ist ein zentrales Modell der Physik. In der Informations-Weber-Theorie ergibt sich die Dynamik aus:
\[
    \frac{\delta \mathcal{F}}{\delta \rho_I} = 0.
\]

\subsection{Lokaler Anteil: klassische Schwingung}
Der lokale Anteil erzeugt:
\[
    \ddot{x} + \omega^2 x = 0.
\]

\subsection{Globaler Anteil: quantisierte Energieniveaus}
Der globale Anteil erzeugt:
\[
    Q = -\frac{\hbar^2}{2m}
    \frac{\nabla^2 \sqrt{\rho_I}}{\sqrt{\rho_I}},
\]
woraus die quantisierten Energieniveaus folgen:
\[
    E_n = \left(n + \frac{1}{2}\right)\hbar\omega.
\]
Interpretation:
\begin{itemize}
    \item Quantisierung ist eine Eigenschaft globaler Informationsorganisation,
    \item keine Operatoren, keine Hilberträume notwendig,
    \item klassische und quantisierte Dynamik entstehen aus demselben Funktional.
\end{itemize}

\section{Beispiel 3: Kepler-Problem und gravitative Informationsflüsse}
Das Kepler-Problem zeigt, wie Gravitation aus Informationsgradienten entsteht.

\subsection{Informationspotential}
\[
    \Phi_I(\vec{r})
    =
    \int \frac{\rho_I(\vec{r}')}{|\vec{r}-\vec{r}'|}\, d^3x'.
\]
Für schwache Gradienten ergibt sich:
\[
    \Phi_I \propto \frac{1}{r}.
\]

\subsection{Weber-Gravitation}
Die resultierende Kraft lautet:
\[
    \vec{F}_{\text{grav}} = -\nabla \Phi_I.
\]
Damit entstehen:
\begin{itemize}
    \item elliptische Bahnen,
    \item Periheldrehung,
    \item Abweichungen bei starken Informationsgradienten.
\end{itemize}

\section{Beispiel 4: Plasmawellen als Informationsmoden}
Plasmen sind natürliche Informationsmedien. Die Weber-Dynamik erzeugt:
\begin{itemize}
    \item geschwindigkeitsabhängige Kopplungen,
    \item beschleunigungsabhängige Reaktionskräfte,
    \item nichtlineare Wellenphänomene.
\end{itemize}

\subsection{Informationsgeometrische Interpretation}
Die fraktale Dimension
\[
    D = \frac{\ln 20}{\ln(2+\phi)}
\]
bestimmt:
\begin{itemize}
    \item Filamentierung,
    \item Jets,
    \item turbulente Skalenhierarchien.
\end{itemize}

\section{Beispiel 5: Gravitationslinsen als Moden der Informationsgeometrie}
Die Informations-Weber-Theorie sagt eine frequenzabhängige Lichtablenkung voraus:
\[
    \delta\theta(\nu)
    =
    \delta\theta_0
    \left(
        1 + \alpha \frac{\nu_0}{\nu}
    \right).
\]
Interpretation:
\begin{itemize}
    \item Gravitationslinsen sind Moden der Informationsgeometrie,
    \item keine Krümmung eines ontologischen Raumes notwendig,
    \item klare Abweichungen von der ART im starken Feld.
\end{itemize}

\section{Beispiel 6: Numerische Simulation eines Informationsnetzes}
Die digitale Informations-Weber-Theorie beschreibt den Raum als diskretes Netz. Eine Simulation zeigt:
\begin{itemize}
    \item Emergenz eines dreidimensionalen Raumes,
    \item Wellen als kollektive Moden,
    \item Gravitationspotentiale als Informationsgradienten,
    \item fraktale Strukturen über viele Skalen.
\end{itemize}

\section{Zusammenfassung}
Kapitel~\ref{chap:beispiele} hat gezeigt:
\begin{itemize}
    \item Die Informations-Weber-Theorie ist praktisch anwendbar.
    \item Klassische, quantenmechanische und gravitative Phänomene entstehen aus demselben Funktional.
    \item Plasmaprozesse, Interferenz, Gravitation und Lensing lassen sich informationsbasiert erklären.
    \item Numerische Simulationen machen die Theorie operational.
\end{itemize}
Damit bildet dieses Kapitel die Brücke zwischen der theoretischen Struktur der Kapitel 4–12 und den numerischen und konzeptionellen Erweiterungen der Kapitel 14 und der
Anhänge.
