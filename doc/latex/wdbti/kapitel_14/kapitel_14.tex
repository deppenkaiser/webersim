\chapter{Ausblick}
\label{chap:ausblick}

\section{Einleitung}
Die Informations-Weber-Theorie stellt eine neue fundamentale Sichtweise auf die Physik dar. Sie ersetzt Felder, Teilchen und Raumzeit durch Informationsdichten,
Informationsflüsse und eine emergente Informationsgeometrie. Die in diesem Werk entwickelte Struktur zeigt, dass klassische Mechanik, \gls{qm} und Gravitation keine
unabhängigen Theorien sind, sondern unterschiedliche Näherungen eines universellen Informationsprinzips.

Dieses Kapitel gibt einen Ausblick auf offene Fragen, zukünftige Forschungsrichtungen und die möglichen Konsequenzen einer informationsbasierten Physik.

\section{Eine neue Grundlage der Physik}
Die Informations-Weber-Theorie ist keine Erweiterung bestehender Modelle, sondern eine Urtheorie, aus der bekannte physikalische Gesetze als Grenzfälle hervorgehen. Sie 
liefert:
\begin{itemize}
    \item eine einheitliche Beschreibung lokaler und globaler Dynamik,
    \item eine natürliche Erklärung der Quantenstruktur,
    \item eine emergente Geometrie des Raumes,
    \item eine informationsbasierte Gravitation ohne Singularitäten,
    \item eine Herleitung der Naturkonstanten,
    \item eine konsistente kosmologische Dynamik ohne Urknall-Singularität.
\end{itemize}
Damit entsteht ein neues Fundament, das die Physik auf eine informationsbasierte Grundlage stellt.

\section{Offene Fragen und zukünftige Entwicklungen}
Obwohl die Informations-Weber-Theorie eine konsistente Struktur liefert, bleiben wichtige Fragen offen, die zukünftige Forschung leiten werden.

\subsection{Numerische Simulationen der Informationsgeometrie}
Die digitale \gls{wdbt} beschreibt den Raum als diskretes Informationsnetz. Eine zentrale Herausforderung besteht darin, diese Struktur numerisch zu simulieren:
\begin{itemize}
    \item Wie entwickelt sich die Informationsgeometrie in komplexen Systemen?
    \item Wie entstehen Wellen, Turbulenz und fraktale Muster im Informationsnetz?
    \item Wie lassen sich kosmologische Strukturen aus Informationsflüssen simulieren?
\end{itemize}
Solche Simulationen könnten die Theorie empirisch zugänglich machen und erlauben, die Emergenz von Raum, Zeit und Dynamik direkt zu beobachten.

\subsection{Quantitative Herleitung der Naturkonstanten}
Kapitel~\ref{chap:naturkonstanten} zeigt, dass Naturkonstanten aus der Informationsarchitektur emergieren. Eine zukünftige Aufgabe besteht darin, diese
Herleitungen quantitativ zu präzisieren:
\begin{itemize}
    \item exakte Abhängigkeiten von $c$, $\hbar$ und $G$,
    \item numerische Bestimmung der fraktalen Dimension,
    \item Zusammenhang zwischen Netzskala und physikalischen Skalen.
\end{itemize}
Dies würde die Theorie vollständig quantifizieren und experimentell überprüfbar machen.

\subsection{Informationsbasierte Kosmologie}
Die Informations-Weber-Theorie liefert eine konsistente Alternative zur Standardkosmologie. Zukünftige Arbeiten könnten folgende Fragen klären:
\begin{itemize}
    \item Wie genau verläuft der Big Bounce?
    \item Welche Signaturen hinterlässt er in der CMB?
    \item Wie entstehen Galaxien aus informationsbasierten Prozessen?
    \item Welche Rolle spielen Plasmen im frühen Universum?
\end{itemize}
Diese Fragen sind empirisch zugänglich und bieten klare Tests.

\subsection{Informationsbasierte Quantenphysik}
Die Theorie ersetzt die Wellenfunktion durch Informationsdichten. Offene Fragen sind:
\begin{itemize}
    \item Wie entstehen kohärente Informationsphasen in komplexen Systemen?
    \item Welche Rolle spielt das Bohm-Potential in Vielteilchensystemen?
    \item Wie lässt sich Quantenentropie informationsgeometrisch definieren?
\end{itemize}
Eine informationsbasierte Quantenphysik könnte neue Wege zur Quantenmetrologie, Quantenkommunikation und Quantenmaterialforschung eröffnen.

\section{Konsequenzen für Technologie und Wissenschaft}
Eine informationsbasierte Physik hat weitreichende Konsequenzen über die Grundlagenforschung hinaus.

\subsection{Neue Sicht auf Energie und Information}
Wenn Energie eine abgeleitete Größe der Information ist, ergeben sich neue Perspektiven:
\begin{itemize}
    \item Informationsoptimierung statt Energieoptimierung,
    \item neue Konzepte für Energieübertragung,
    \item informationsbasierte Materialwissenschaft.
\end{itemize}

\subsection{Informationsbasierte Messtechnik}
Die Theorie legt nahe, dass Messprozesse Informationsflüsse sind. Dies könnte zu neuen Messverfahren führen:
\begin{itemize}
    \item nichtlokale Messmethoden,
    \item fraktale Informationssensoren,
    \item neue Ansätze für Quantenmetrologie.
\end{itemize}

\subsection{Kosmologische Anwendungen}
Die informationsbasierte Sichtweise könnte neue Modelle für:
\begin{itemize}
    \item Rotverschiebung,
    \item Gravitationswellen,
    \item Strukturentstehung,
    \item Plasma-Kosmologie
\end{itemize}
liefern und damit die moderne Kosmologie erweitern.

\section{Schlussbemerkung}
Die Informations-Weber-Theorie zeigt, dass die Physik nicht auf Feldern, Teilchen oder Raumzeit beruhen muss, sondern auf Information. Raum, Zeit, Dynamik und Gravitation
emergieren aus der Struktur und Transformation von Information.

Dieses Werk bildet den Ausgangspunkt für eine informationsbasierte Physik, die die Grundlagen der Natur neu interpretiert und zukünftige Forschung in eine neue Richtung
lenkt. Die Informations-Weber-Theorie ist kein Abschluss, sondern ein Beginn: der Beginn einer Physik, die Information als fundamentale Größe versteht und die Natur aus
ihrer innersten Struktur heraus erklärt.
