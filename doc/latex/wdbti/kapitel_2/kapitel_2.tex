\chapter{Die Informations-Weber-Theorie}
\label{chap:informationstheorie}

\section{Der Informationszustand}
Die Informations-Weber-Theorie geht von der grundlegenden Annahme aus, dass jeder physikalische Zustand durch eine \emph{Informationsverteilung} beschrieben wird. Diese
Informationsverteilung sei durch eine skalare Dichtefunktion
\[
    \rho_I(\vec{r},t)
\]
gegeben, die angibt, wie viel strukturierte Information an einem Ort im Raum vorliegt.

Im Gegensatz zu klassischen Feldern besitzt $\rho_I$ keine direkte materielle Bedeutung. Sie beschreibt nicht Masse, Ladung oder Energie, sondern die \emph{Organisation}
des physikalischen Systems. Energie, Impuls und andere Größen ergeben sich als abgeleitete Funktionale dieser Informationsstruktur.

\subsection{Informationsdichte und Informationsfluss}
Analog zur Kontinuitätsgleichung der klassischen Physik wird der Informationsfluss durch einen Vektorstrom
\[
    \vec{J}_I(\vec{r},t)
\]
beschrieben. Die fundamentale Erhaltungsgleichung lautet:

\begin{equation}
    \frac{\partial \rho_I}{\partial t}
    + \nabla \cdot \vec{J}_I = 0.
    \label{eq:info_kontinuitaet}
\end{equation}

Diese Gleichung ist das Herzstück der Theorie:
Sie ersetzt die Energieerhaltung durch eine \emph{Informationserhaltung}, aus der die Energieerhaltung als Spezialfall folgt.

\section{Information als Ursprung physikalischer Größen}
Die bekannten physikalischen Größen entstehen als Funktionale der Informationsdichte. Für die Energie gilt:
\[
    E[\rho_I] = \int f(\rho_I, \nabla \rho_I, \ldots)\, d^3x,
\]
wobei $f$ eine noch zu bestimmende Funktion ist, die die Struktur des Systems beschreibt.

Impuls, Trägheit und sogar die geometrische Struktur des Raumes ergeben sich aus\\Transformations- und Symmetrieeigenschaften der Informationsverteilung.

\section{Dynamik als Informationsfluss}
Die Bewegungsgleichungen eines Systems ergeben sich aus der Umlagerung von Information. Die Weber-Kraft beschreibt lokale Informationsflüsse, während das Bohmsche
Quantenpotential globale, systemische Informationsorganisation repräsentiert.

\subsection{Lokale Dynamik: Weber-Kraft als Informationsfluss}
Die Weber-Kraft kann als lokaler Informationsflussoperator interpretiert werden:
\[
    \vec{F}_{\text{Weber}}
    = \mathcal{W}[\rho_I, \vec{J}_I].
\]
Die klassische Form der Weber-Kraft wird in Kapitel~\ref{chap:weberklassisch} hergeleitet und in den informationsbasierten Kontext eingeordnet.

\subsection{Globale Dynamik: Quantenpotential als Informationsoperator}
Das Bohmsche Quantenpotential
\[
    Q = -\frac{\hbar^2}{2m}
    \frac{\nabla^2 \sqrt{\rho}}{\sqrt{\rho}}
\]
beschreibt die systemische, nichtlokale Organisation des Informationszustands. In der\\Informations-Weber-Theorie wird $Q$ als globaler Informationsoperator verstanden,
der die gesamte Struktur des Systems berücksichtigt.

\subsection{Die analoge WDBT als Fernwirkungstheorie}
Die analoge Form der Weber–De-Broglie-Theorie beschreibt die Gesamtwirkung auf ein
physikalisches System durch die Überlagerung dreier Fernwirkungsbeiträge:
\[
    F = F_{\text{WED}} + F_{\text{WG}} + F_Q.
\]

\begin{itemize}
    \item $F_{\text{WED}}$ ist die Weber-Elektrodynamik als Fernwirkung zwischen Ladungen. Sie kommt ohne elektromagnetische Felder als ontologische Objekte aus
    und beschreibt die Wechselwirkung direkt über geschwindigkeits- und
    beschleunigungsabhängige Terme.

    \item $F_{\text{WG}}$ ist die Weber-Gravitationskraft als Fernwirkung zwischen Massen, strukturell analog zur Weber-Elektrodynamik. Die Einführung von $F_{\text{WG}}$
	war ein entscheidender Meilenstein: Erst dadurch wurde die WDBT zu einer konsistenten Theorie, die elektromagnetische, gravitative und quantenmechanische Effekte im
	selben formalen Rahmen behandeln kann.

    \item $F_Q$ ist die aus dem De-Broglie–Bohm-Quantenpotential resultierende Kraft. Auch sie ist eine Fernwirkung, jedoch nicht zwischen einzelnen Teilchen, sondern eine
	systemische, nichtlokale Wirkung auf die gesamte Informationsstruktur des Systems.
\end{itemize}

Entscheidend ist, dass alle drei Beiträge in dieser analogen Stufe als \emph{Fernwirkungstheorien} formuliert sind. Es werden weder klassische Felder noch eine dynamische
Raumzeit als Trägerobjekte benötigt. Die Dynamik ergibt sich vollständig aus direkten Wechselwirkungen und globalen Informationsbeziehungen.

Diese fernwirkungsbasierte Struktur hat jedoch eine klare Konsequenz:
\emph{In der analogen WDBT existiert kein explizites Raummodell.} Ohne ein solches Modell können keine propagierenden Störungen definiert werden. Daher gilt:

\begin{itemize}
    \item Gravitationswellen setzen ein dynamisches Raummodell voraus.
    \item Die analoge WDBT besitzt kein solches Modell.
    \item Daher kann sie Gravitationswellen nicht beschreiben.
\end{itemize}

Diese Einschränkung ist kein Defizit, sondern eine strukturelle Eigenschaft der analogen Theorie. Erst die digitale WDBT führt ein diskretes Informationsnetz ein, aus dem
der physikalische Raum als emergente Geometrie entsteht. In dieser digitalen Stufe werden Gravitationswellen zu kollektiven Moden der Informationsgeometrie, und die
großskalige Struktur der kosmischen Hintergrundstrahlung (CMB) sowie die Herleitung von Naturkonstanten werden möglich.

\section{Raum als emergente Informationsgeometrie}
Die analoge WDBT beschreibt physikalische Systeme ausschließlich durch Fernwirkungen:
\[
    F = F_{\text{WED}} + F_{\text{WG}} + F_Q.
\]
In dieser Stufe der Theorie existiert kein ontologischer physikalischer Raum. Die Dynamik ergibt sich vollständig aus direkten Wechselwirkungen und globalen
Informationsbeziehungen. Dies erklärt, warum die analoge WDBT keine Gravitationswellen kennt: Ohne ein explizites Raummodell können keine propagierenden Störungen
definiert werden.

Um Phänomene zu beschreiben, die ein dynamisches Raummodell erfordern – etwa Gravitationswellen, die Struktur der kosmischen Hintergrundstrahlung (CMB) oder die Herleitung
von Naturkonstanten – muss die Theorie auf eine diskrete, informationsbasierte Ebene erweitert werden. Diese Erweiterung bildet die \emph{digitale WDBT}.

\subsection{Warum Raum nicht fundamental sein kann}
Die Informations-Weber-Theorie geht davon aus, dass der physikalische Raum keine ontologische Grundgröße ist, sondern eine abgeleitete Eigenschaft der Informationsstruktur. Diese Sichtweise ergibt sich aus mehreren grundlegenden
Überlegungen:

\paragraph{(1) Fernwirkung ohne Trägerraum}
Die analoge WDBT beschreibt die Dynamik vollständig durch Fernwirkungen:
\[
    F = F_{\text{WED}} + F_{\text{WG}} + F_Q.
\]
Diese drei Beiträge benötigen keinen physikalischen Raum als Trägerobjekt. Die Wechselwirkungen sind relational definiert und beziehen sich ausschließlich auf
Informationszustände. Damit wird klar: Die Dynamik existiert logisch \emph{vor} dem Raum.

\paragraph{(2) Raum ist nicht notwendig für Kausalität}
Die Theorie unterscheidet zwei Kausalitätsebenen:
\begin{itemize}
    \item lokale Kausalität (Energietransport),
    \item systemische Kausalität (globale Informationsorganisation).
\end{itemize}
Beide Ebenen lassen sich ohne ein ontologisches Raumzeitkontinuum formulieren. Die Existenz eines Raumes ist daher keine Voraussetzung für Kausalität, sondern
eine Konsequenz der Informationsstruktur.

\paragraph{(3) Raum kann nicht fundamental sein, wenn er fraktal ist}
Die fraktale Dimension
\[
    D = \frac{\ln 20}{\ln(2+\phi)}
\]
ist eine Eigenschaft der Kopplungsstruktur des Informationsnetzes. Eine fraktale
Dimension ist jedoch unvereinbar mit einem fundamental glatten Kontinuum. Der
Raum kann daher nicht primitiv sein, sondern muss aus einer diskreten Struktur
emergieren.

\paragraph{(4) Kontinuumsmodelle erzeugen Paradoxien}
Kontinuumsmodelle wie die ART führen zu:
\begin{itemize}
    \item Singularitäten,
    \item Energieerzeugung aus dem Nichts,
    \item undefinierten Anfangsbedingungen (Urknall),
    \item unendlichen Selbstenergien.
\end{itemize}
Diese Probleme verschwinden, sobald Raum nicht als fundamental, sondern als
emergente Größe verstanden wird.

\paragraph{(5) Dynamische Raumzeit setzt ein Informationsnetz voraus}
Eine dynamische Raumzeit – wie sie in der ART angenommen wird – benötigt ein Trägerobjekt, das sich verändern kann. Ohne ein zugrunde liegendes Informationsnetz ist eine
solche Dynamik nicht definierbar. Erst die digitale WDBT liefert dieses Netz und damit ein konsistentes Raummodell.

\subsection*{Konsequenz}
Aus diesen Gründen kann der physikalische Raum nicht fundamental sein. Er ist die effektive Geometrie eines diskreten Informationsnetzes, das erst in der digitalen
WDBT explizit eingeführt wird. Die analoge WDBT arbeitet bewusst ohne Raum und beschreibt Fernwirkungen; die digitale WDBT lässt den Raum aus der Informationsarchitektur
emergieren und ermöglicht damit Phänomene wie Gravitationswellen, CMB-Struktur und die Herleitung von Naturkonstanten.

\subsection{Emergenz der Zeit}
Wenn der physikalische Raum keine fundamentale Größe ist, stellt sich unmittelbar die Frage nach dem Status der Zeit. In der Informations-Weber-Theorie ist auch die Zeit
keine ontologische Grundgröße, sondern eine emergente Eigenschaft der Transformationen des Informationszustands.

\paragraph{(1) Zeit als Ordnungsparameter der Information}
Die analoge WDBT beschreibt die Dynamik eines Systems durch eine invertierbare Transformation des Informationszustands:
\[
    I(t_2) = T[I(t_1)].
\]
Der Parameter $t$ besitzt in dieser Stufe keine physikalische Bedeutung, sondern dient als \emph{Ordnungsparameter}, der die Reihenfolge der Informationszustände festlegt.
Zeit ist damit keine Substanz, sondern eine Strukturierung der Veränderung.

\paragraph{(2) Zeit entsteht aus der Sequenz diskreter Informationszustände}
In der digitalen WDBT wird der Informationszustand durch ein diskretes Netzwerk von Knoten und Kopplungen beschrieben. Die Dynamik besteht aus elementaren Aktualisierungen
dieser Kopplungsstruktur. Die Zeit entsteht als Sequenz solcher Aktualisierungsschritte:
\[
    I_0 \rightarrow I_1 \rightarrow I_2 \rightarrow \cdots
\]
Die physikalische Zeit ist die effektive Kontinuumsbeschreibung dieser diskreten Sequenz. Damit ist Zeit eine emergente Größe, die aus der Ordnung der
Informationsveränderungen hervorgeht.

\paragraph{(3) Zwei Kausalitätsebenen erzeugen zwei Zeitstrukturen}
Die Informations-Weber-Theorie unterscheidet:
\begin{itemize}
    \item \textbf{lokale Kausalität}: Energie- und Impulsfluss mit endlicher Geschwindigkeit,
    \item \textbf{systemische Kausalität}: instantane Organisation des Informationszustands.
\end{itemize}
Diese beiden Ebenen erzeugen zwei komplementäre Zeitstrukturen:
\begin{itemize}
    \item eine \emph{lokale Zeit}, die durch Transportprozesse definiert ist,
    \item eine \emph{globale Zeit}, die durch die Ordnung der Informationsstruktur bestimmt wird.
\end{itemize}
Die beobachtete physikalische Zeit ist die Überlagerung beider Strukturen.

\paragraph{(4) Zeitdilatation als Informationsgeometrie}
In der digitalen WDBT entsteht die Zeitdilatation nicht aus einer ontologischen Raumzeit, sondern aus der Veränderung der Informationsgeometrie. Eine stärkere
Kopplungsdichte führt zu einer verlangsamten lokalen Aktualisierungsrate des Informationsnetzes. Damit ergibt sich die Zeitdilatation als emergente Eigenschaft der
Informationsarchitektur.

\paragraph{(5) Keine fundamentale Zeitrichtung}
Da die Transformationen des Informationszustands invertierbar sind, besitzt die Theorie keine fundamentale Zeitrichtung. Die beobachtete Zeitrichtung entsteht aus der
Informationsentropie:
\[
    \Delta S_I \ge 0.
\]
Die „Richtung der Zeit“ ist damit eine statistische Eigenschaft der Informationsdynamik, nicht ein fundamentales Gesetz.

\subsection*{Konsequenz}
Die Zeit ist in der Informations-Weber-Theorie keine primitive Größe. Sie entsteht aus der Ordnung der Informationszustände und aus der Aktualisierungsdynamik des diskreten
Informationsnetzes. Die analoge WDBT beschreibt Zeit als Ordnungsparameter, die digitale WDBT lässt sie als emergente Struktur der Informationsgeometrie entstehen. Damit
wird die Zeit – ebenso wie der Raum – zu einer abgeleiteten Größe einer tieferen Informationsordnung.

\subsection{Diskrete Informationsstruktur als Ursprung des Raumes}
Die digitale WDBT geht von einem Netzwerk aus Informationsknoten und Kopplungen aus. Dieses Netzwerk bildet keinen Raum ab, sondern \emph{erzeugt} ihn. Der physikalische
Raum ist die effektive Metrik der Kopplungsstruktur:
\[
    g_{ij} = g_{ij}[\text{Kopplungen}, \rho_I].
\]
Die geometrischen Eigenschaften des Raumes – Dimension, Krümmung, Metrik – sind keine Grundgrößen, sondern emergente Eigenschaften der Informationsarchitektur. Die fraktale
Dimension 
\[
    D = \frac{\ln 20}{\ln(2+\phi)}
\]
ist eine Eigenschaft dieser Kopplungsstruktur und bestimmt die effektive Geometrie des Raumes auf verschiedenen Skalen.

\subsection{Emergenz von Gravitationswellen}
In der digitalen WDBT entstehen Gravitationswellen als kollektive Moden der Informationsgeometrie. Sie sind keine Schwingungen eines ontologischen Kontinuums, sondern
Veränderungen der Kopplungsstruktur des Informationsnetzes. Damit wird klar:

\begin{itemize}
    \item Die analoge WDBT kann keine Gravitationswellen beschreiben.
    \item Die digitale WDBT erzeugt Gravitationswellen als emergente Phänomene.
\end{itemize}

\subsection{CMB-Struktur als fossilierte Informationsgeometrie}
Die großskalige Struktur der kosmischen Hintergrundstrahlung (CMB) ergibt sich in der digitalen WDBT aus frühen Zuständen der Informationsgeometrie. Die beobachteten
Anisotropien spiegeln die fraktale Kopplungsstruktur des Informationsnetzes wider und sind keine Signatur eines thermischen Urknalls.

\subsection{Herleitung von Naturkonstanten}
In der digitalen WDBT sind Naturkonstanten keine unabhängigen Eingabegrößen, sondern Konsequenzen der diskreten Informationsarchitektur. Größen wie $c$, $\hbar$ oder $G$
entstehen aus:

\begin{itemize}
    \item Kopplungsstärken des Informationsnetzes,
    \item Skalierungsrelationen der fraktalen Dimension,
    \item symmetrischen Transformationen des Informationsraums.
\end{itemize}

Damit wird die digitale WDBT zu einer echten Urtheorie: Sie beschreibt nicht nur Dynamik, sondern auch die Entstehung der fundamentalen Konstanten selbst.

\subsection{Zusammenfassung}
Die analoge WDBT arbeitet ohne Raum und beschreibt Fernwirkungen. Die digitale WDBT führt ein diskretes Informationsnetz ein, aus dem der physikalische Raum als emergente
Geometrie entsteht. Erst diese digitale Stufe ermöglicht die Beschreibung von Gravitationswellen, der CMB-Struktur und die Herleitung der Naturkonstanten.

\section{Informations-Lagrange-Funktional}
Um die Dynamik der Informations-Weber-Theorie formal zu beschreiben, wird ein Lagrange-Funktional eingeführt, das die zeitliche Entwicklung der Informationsdichte
$\rho_I(\vec{r},t)$ bestimmt. Dieses Funktional ersetzt die klassischen Lagrange-Ansätze der Mechanik und Feldtheorie durch eine informationsbasierte Struktur.

\subsection{Definition des Funktionals}
Das Informations-Lagrange-Funktional sei gegeben durch
\begin{equation}
    \mathcal{L}_I[\rho_I]
    =
    \int \mathcal{F}\!\left(
        \rho_I,\,
        \nabla \rho_I,\,
        \partial_t \rho_I
    \right) d^3x,
    \label{eq:info_lagrange}
\end{equation}
wobei $\mathcal{F}$ eine skalare Dichte ist, die die lokale Struktur des Informationsraums beschreibt. Sie hängt im Allgemeinen von der Informationsdichte, ihren
räumlichen Gradienten und ihrer zeitlichen Ableitung ab.

Die Form von $\mathcal{F}$ ist nicht a priori festgelegt, sondern ergibt sich aus den Axiomen der Informations-Weber-Theorie und den Symmetrien des Informationsraums.

\subsection{Variation und Euler-Lagrange-Gleichungen}
Die Dynamik folgt aus dem Variationsprinzip
\[
    \delta \mathcal{L}_I = 0.
\]
Die Variation nach $\rho_I$ führt zur Euler-Lagrange-Gleichung
\begin{equation}
    \frac{\partial}{\partial t}
    \left(
        \frac{\partial \mathcal{F}}{\partial (\partial_t \rho_I)}
    \right)
    +
    \nabla \cdot
    \left(
        \frac{\partial \mathcal{F}}{\partial (\nabla \rho_I)}
    \right)
    -
    \frac{\partial \mathcal{F}}{\partial \rho_I}
    = 0.
    \label{eq:euler_lagrange_info}
\end{equation}

Diese Gleichung ist die fundamentale Bewegungsgleichung der Informations-Weber-Theorie. Sie ersetzt die Newtonsche Bewegungsgleichung, die Maxwell-Gleichungen und die
Schrödinger-Gleichung durch eine einheitliche informationsbasierte Struktur.

\subsection{Lokale und globale Beiträge}
Die Struktur von $\mathcal{F}$ erlaubt eine natürliche Zerlegung in lokale und globale Anteile:
\begin{equation}
    \mathcal{F}
    =
    \mathcal{F}_{\text{lokal}}
    +
    \mathcal{F}_{\text{global}}.
\end{equation}

\begin{itemize}
    \item $\mathcal{F}_{\text{lokal}}$ beschreibt lokale Informationsflüsse und führt im
    Grenzfall zu den geschwindigkeits- und beschleunigungsabhängigen Termen der
    Weber-Kraft.

    \item $\mathcal{F}_{\text{global}}$ beschreibt die systemische Organisation des
    Informationszustands und führt im Grenzfall zum Bohmschen Quantenpotential.
\end{itemize}

Damit ergibt sich eine natürliche Interpretation:
\begin{itemize}
    \item Die Weber-Kraft ist der lokale Anteil der Informationsdynamik.
    \item Das Quantenpotential ist der globale Anteil der Informationsdynamik.
\end{itemize}

Diese Zerlegung ist nicht willkürlich, sondern folgt aus den Axiomen der Theorie:
lokale Kausalität (Axiom V) erzeugt lokale Terme, systemische Ganzheit erzeugt globale Terme.

\subsection{Informationsfluss als Bewegungsgleichung}
Aus der Euler-Lagrange-Gleichung \eqref{eq:euler_lagrange_info} ergibt sich eine Gleichung für den Informationsfluss
\[
    \vec{J}_I
    =
    \frac{\partial \mathcal{F}}{\partial (\nabla \rho_I)}.
\]
Damit wird die Kontinuitätsgleichung
\[
    \frac{\partial \rho_I}{\partial t}
    + \nabla \cdot \vec{J}_I = 0
\]
zu einer direkten Konsequenz des Variationsprinzips.

Die Informations-Weber-Theorie ist somit eine vollständig variationale Theorie, in der sowohl lokale als auch globale Dynamik aus einem einzigen Funktional hervorgehen.

\section{Emergenz klassischer Gleichungen}
Die Informations-Weber-Theorie ist so konstruiert, dass sie klassische physikalische Gleichungen als Grenzfälle reproduziert. Dies ist ein wesentliches Kriterium für die
physikalische Konsistenz der Theorie: Eine neue fundamentale Beschreibung muss die bewährten Modelle in geeigneten Näherungen enthalten. In diesem Abschnitt wird gezeigt,
wie die Weber-Kraft, das Bohmsche Quantenpotential und klassische Energie- und Impulsgrößen aus der informationsbasierten Struktur hervorgehen.

\subsection{Weber-Kraft als Grenzfall lokaler Informationsdynamik}
Die Weber-Kraft entsteht aus dem lokalen Anteil des Informations-Lagrange-Funktionals. Wird $\mathcal{F}_{\text{lokal}}$ auf Terme erster und zweiter Ordnung in den
zeitlichen Änderungen der Informationsdichte beschränkt, ergibt sich im Grenzfall schwacher Informationsgradienten eine Kraftform
\[
    \vec{F}_{\text{lokal}}
    =
    \mathcal{W}[\rho_I, \vec{J}_I]
    \;\longrightarrow\;
    \vec{F}_{\text{Weber}},
\]

wobei die klassische Weber-Kraft in Kapitel~\ref{chap:weberklassisch} hergeleitet wird. Die informationsbasierte Interpretation lautet:

\begin{itemize}
    \item Der Coulomb-Term beschreibt die statische Informationskopplung.
    \item Der geschwindigkeitsabhängige Term beschreibt lokale Änderungen des
    Informationsflusses.
    \item Der beschleunigungsabhängige Term beschreibt die Reaktion des Systems auf
    zeitliche Änderungen der Informationsstruktur.
\end{itemize}

Damit erscheint die Weber-Kraft nicht als exotische Modifikation der Elektrodynamik, sondern als \emph{lokale Näherung} einer tieferen informationsbasierten Dynamik.

\subsection{Bohm-Potential als globaler Informationsoperator}
Der globale Anteil des Informations-Lagrange-Funktionals führt im Grenzfall zu einem Term der Form
\[
    \mathcal{F}_{\text{global}}
    \propto
    \frac{(\nabla \rho_I)^2}{\rho_I},
\]

dessen Variation das Bohmsche Quantenpotential erzeugt:

\[
    Q
    =
    -\frac{\hbar^2}{2m}
    \frac{\nabla^2 \sqrt{\rho_I}}{\sqrt{\rho_I}}.
\]

Damit ergibt sich eine klare Interpretation:

\begin{itemize}
    \item $Q$ ist kein mysteriöser Zusatzterm der Quantenmechanik.
    \item $Q$ ist die systemische, globale Organisationsstruktur der Information.
    \item $Q$ entsteht aus der Minimierung des globalen Informationsfunktionals.
\end{itemize}

Die Nichtlokalität des Quantenpotentials ist somit keine Verletzung der Kausalität, sondern Ausdruck der systemischen Ganzheit des Informationsraums (Axiom~V).

\subsection{Energie und Impuls als Informationsfunktionale}
Die klassischen Größen Energie und Impuls entstehen aus Symmetrien des Informations-Lagrange-Funktionals. Nach dem Noether-Theorem gilt:

\begin{itemize}
    \item \textbf{Translationssymmetrie} $\Rightarrow$ Impulserhaltung
    \item \textbf{Zeitsymmetrie} $\Rightarrow$ Energieerhaltung
\end{itemize}

Die Energie eines Systems ergibt sich aus dem Funktional

\[
    E[\rho_I]
    =
    \int \mathcal{H}_I(\rho_I, \nabla \rho_I)\, d^3x,
\]

wobei $\mathcal{H}_I$ die informationsbasierte Hamilton-Dichte ist.

Der Impuls ergibt sich aus der Variation unter infinitesimalen Translationen:

\[
    \vec{p}
    =
    \int \rho_I(\vec{r},t)\, \vec{v}_I(\vec{r},t)\, d^3x,
\]

wobei $\vec{v}_I$ die effektive Informationsgeschwindigkeit ist, definiert durch

\[
    \vec{J}_I = \rho_I \vec{v}_I.
\]

Damit erscheinen Energie und Impuls nicht als primitive Größen, sondern als \emph{abgeleitete Eigenschaften der Informationsstruktur}.

\subsection{Zusammenführung der Grenzfälle}
Die Informations-Weber-Theorie reproduziert folgende klassische Gleichungen:

\begin{itemize}
    \item Die Weber-Kraft als Grenzfall lokaler Informationsdynamik.
    \item Das Bohm-Potential als Grenzfall globaler Informationsorganisation.
    \item Energie- und Impulserhaltung als Konsequenz der Symmetrien des
    Informationsraums.
\end{itemize}

Damit zeigt sich, dass die klassische Mechanik, die Weber-Elektrodynamik und die Bohmsche Mechanik nicht konkurrierende Modelle sind, sondern unterschiedliche
Näherungen einer einheitlichen informationsbasierten Theorie.

\section{Zusammenfassung von Kapitel 2}
In diesem Kapitel wurde die formale Grundlage der Informations-Weber-Theorie entwickelt. Ausgehend von der Annahme, dass jeder physikalische Zustand durch eine
Informationsdichte $\rho_I(\vec{r},t)$ beschrieben wird, wurde gezeigt, dass die Dynamik eines Systems aus der Umlagerung dieser Information hervorgeht. Die fundamentale
Erhaltungsgleichung
\[
    \frac{\partial \rho_I}{\partial t} + \nabla \cdot \vec{J}_I = 0
\]
ersetzt die klassische Energieerhaltung und bildet das Herzstück der Theorie.

Die physikalischen Größen Energie, Impuls und Trägheit erscheinen nicht als primitive Entitäten, sondern als Funktionale der Informationsstruktur. Ihre Erhaltung ergibt
sich aus den Symmetrien des Informationsraums, insbesondere aus Translations-, Rotations- und Zeitsymmetrie. Damit wird die klassische Mechanik als Konsequenz eines
tieferen informationsbasierten Prinzips verstanden.

Durch die Einführung eines informationsbasierten Lagrange-Funktionals
\[
    \mathcal{L}_I[\rho_I]
    =
    \int \mathcal{F}(\rho_I, \nabla \rho_I, \partial_t \rho_I)\, d^3x
\]
konnte gezeigt werden, dass sowohl lokale als auch globale Dynamik aus einem einzigen Variationsprinzip hervorgehen. Die Zerlegung von $\mathcal{F}$ in lokale und globale
Anteile führt im Grenzfall zu zwei bekannten Strukturen:

\begin{itemize}
    \item Die \textbf{Weber-Kraft} als Ausdruck lokaler Informationsflüsse.
    \item Das \textbf{Bohmsche Quantenpotential} als Ausdruck globaler, systemischer
    Informationsorganisation.
\end{itemize}

Damit wird deutlich, dass klassische und quantenmechanische Phänomene nicht im Widerspruch stehen, sondern unterschiedliche Näherungen einer einheitlichen
informationsbasierten Dynamik darstellen. Der physikalische Raum erscheint in dieser Sichtweise als emergente Informationsgeometrie, deren fraktale Struktur eine
fundamentale Rolle spielt.

Dieses Kapitel bildet die Grundlage für die folgenden Entwicklungen. Kapitel~3 widmet sich der klassischen Weber-Elektrodynamik, die als lokaler Grenzfall der
Informations-Weber-Theorie verstanden wird. Die dort hergeleiteten Gleichungen dienen als Referenzpunkt für die weiteren Kapitel, in denen gezeigt wird, wie klassische und
quantenmechanische Phänomene aus den Axiomen der Informationsdynamik hervorgehen.
