\chapter{Die Informations-Weber-Theorie}
\label{chap:informationstheorie}

\section{Der Informationszustand}
Die Informations-Weber-Theorie geht von der grundlegenden Annahme aus, dass jeder physikalische Zustand durch eine \emph{Informationsverteilung} beschrieben wird. Diese
wird durch eine skalare Dichtefunktion
\[
    \rho_I(\vec{r},t)
\]
repräsentiert, die angibt, wie viel strukturierte Information an einem Ort vorliegt. 

Im Gegensatz zu klassischen Feldern besitzt $\rho_I$ keine materielle Bedeutung. Sie beschreibt weder Masse noch Ladung oder Energie, sondern die \emph{Organisation} eines
physikalischen Systems. Energie, Impuls und andere Größen entstehen erst als abgeleitete Funktionale dieser Informationsstruktur.

\subsection{Informationsdichte und Informationsfluss}
Analog zur Kontinuitätsgleichung der klassischen Physik wird der Informationsfluss durch einen Vektorstrom
\[
    \vec{J}_I(\vec{r},t)
\]
beschrieben. Die fundamentale Erhaltungsgleichung lautet:
\[
    \frac{\partial \rho_I}{\partial t} + \nabla \cdot \vec{J}_I = 0.
\]
Diese Gleichung ist das Herzstück der Theorie: Sie ersetzt die Energieerhaltung durch eine \emph{Informationserhaltung}, aus der die Energieerhaltung als Spezialfall folgt. Die gesamte Dynamik ergibt sich aus der Umlagerung von Information.

\section{Information als Ursprung physikalischer Größen}
In der Informations-Weber-Theorie entstehen physikalische Größen als Funktionale der Informationsdichte. Energie, Impuls, Trägheit und sogar die geometrische Struktur des
Raumes ergeben sich aus Symmetrien und Transformationen der Informationsverteilung.

Damit wird die klassische Unterscheidung zwischen Materie, Feldern und Geometrie aufgehoben: Alles entsteht aus einer einzigen fundamentalen Größe – der Information.

\section{Dynamik als Informationsfluss}
Die Bewegungsgleichungen eines Systems ergeben sich aus der Umlagerung von Information. Die Theorie unterscheidet zwei komplementäre Dynamikformen:
\begin{itemize}
    \item \textbf{lokale Dynamik}: beschrieben durch die Weber-Kraft,
    \item \textbf{globale Dynamik}: beschrieben durch das Bohm’sche Quantenpotential.
\end{itemize}
Diese beiden Strukturen sind keine konkurrierenden Modelle, sondern zwei Projektionen derselben Informationsdynamik.

\subsection{Lokale Dynamik: Weber-Kraft}
Die Weber-Kraft beschreibt lokale Informationsflüsse. Sie ist der lokale Grenzfall der informationsbasierten Dynamik und wird in Kapitel~\ref{chap:weberklassisch}
hergeleitet.

\subsection{Globale Dynamik: Quantenpotential}
Das Bohm’sche Quantenpotential beschreibt die systemische, nichtlokale Organisation des Informationszustands. Es ist der globale Grenzfall der Informationsdynamik und wird
in Kapitel~4 aus dem Informations-Lagrange-Funktional abgeleitet.

\subsection{Die analoge WDBT als Fernwirkungstheorie}
Die analoge Weber–De-Broglie–Bohm-Theorie (WDBT) beschreibt die Gesamtwirkung auf ein System durch drei Fernwirkungsbeiträge:
\[
    F = F_{\text{WED}} + F_{\text{WG}} + F_Q.
\]
\begin{itemize}
    \item $F_{\text{WED}}$: Weber-Elektrodynamik (Ladungen),
    \item $F_{\text{WG}}$: Weber-Gravitation (Massen),
    \item $F_Q$: Bohm’sches Quantenpotential (Informationsstruktur).
\end{itemize}
Diese analoge Theorie besitzt \emph{kein Raummodell}. Sie arbeitet rein relational und kann daher keine propagierenden Störungen wie Gravitationswellen beschreiben. Dies ist kein Mangel, sondern eine Konsequenz der rein dynamischen Fernwirkungsstruktur ohne geometrische Interpretation.

\section{Raum als emergente Informationsgeometrie}
Die analoge WDBT arbeitet ohne ontologischen Raum. Erst die digitale WDBT führt ein diskretes Informationsnetz ein, aus dem der physikalische Raum als emergente Geometrie
entsteht.

\subsection{Warum Raum nicht fundamental sein kann}
Mehrere Argumente sprechen gegen einen fundamentalen Raum:
\begin{itemize}
    \item Fernwirkungen benötigen keinen Trägerraum.
    \item Kausalität kann ohne Raum formuliert werden.
    \item Die fraktale Dimension widerspricht einem glatten Kontinuum.
    \item Kontinuumsmodelle erzeugen Singularitäten und Paradoxien.
    \item Eine dynamische Raumzeit setzt ein Informationsnetz voraus.
\end{itemize}
Die Konsequenz lautet: Raum ist eine abgeleitete Größe, keine fundamentale.

\subsection{Emergenz der Zeit}
Auch die Zeit ist keine primitive Größe. Sie entsteht aus der Ordnung der Informationszustände und aus der Aktualisierungsdynamik des Informationsnetzes.
\begin{itemize}
    \item Zeit ist ein Ordnungsparameter der Informationsveränderung.
    \item In der digitalen WDBT entsteht Zeit aus diskreten Aktualisierungsschritten.
    \item Zeitdilatation ist eine Eigenschaft der Informationsgeometrie.
    \item Die Zeitrichtung entsteht aus Informationsentropie.
\end{itemize}

\subsection{Fraktale Dimension als geometrische Signatur}
Die fraktale Dimension
\[
    D = \frac{\ln 20}{\ln(2+\phi)}
\]
ist eine Eigenschaft der Kopplungsstruktur des Informationsnetzes. Sie beschreibt die Skalierungsstruktur der Informationsarchitektur und ist ein Hinweis darauf, dass der
Raum nicht fundamental sein kann.

\subsection{Diskrete Informationsstruktur als Ursprung des Raumes}
Die digitale WDBT beschreibt ein Netzwerk aus Informationsknoten und Kopplungen. Der physikalische Raum ist die effektive Metrik dieser Kopplungsstruktur:
\[
    g_{ij} = g_{ij}[\text{Kopplungen}, \rho_I].
\]

\subsection{Emergenz der Dynamik aus der Informationsgeometrie}
Wenn Raum und Zeit emergent sind, dann ist auch die Dynamik emergent. Bewegung, Kräfte und Wellen entstehen als Konsequenzen der Informationsgeometrie.
\begin{itemize}
    \item lokale Dynamik = Projektion der lokalen Informationsstruktur,
    \item globale Dynamik = Projektion der systemischen Informationsstruktur,
    \item Wellen = kollektive Moden der Informationsgeometrie.
\end{itemize}

\subsection{Emergenz von Gravitationswellen}
Die analoge WDBT kann keine Gravitationswellen beschreiben. Die digitale WDBT erzeugt Gravitationswellen als kollektive Moden der Informationsgeometrie. Damit wird die Stärke der ART – die Beschreibung dynamischer Geometrie – in einen informationsbasierten Rahmen überführt.

\subsection{CMB-Struktur als fossilierte Informationsgeometrie}
Die anisotrope Struktur der kosmischen Hintergrundstrahlung (CMB) spiegelt die fraktale Kopplungsstruktur des frühen Informationsnetzes wider.

\subsection{Herleitung von Naturkonstanten}
In der digitalen WDBT entstehen Naturkonstanten wie $c$, $\hbar$ und $G$ aus Skalierungsrelationen der Informationsarchitektur.

\subsection{Einordnung von WDBT, ART und ART+}

\begin{itemize}
    \item \textbf{analoge WDBT}: Fernwirkung, kein Raum, keine Wellen; direkte dynamische Struktur.
    \item \textbf{ART}: geometrisches Raummodell, das die Weber-Dynamik im schwachen Feld reproduziert, jedoch im starken Feld Singularitäten erzeugt.
    \item \textbf{ART+}: ART erweitert um informationsbasierte Struktur; keine echten Singularitäten.
    \item \textbf{digitale WDBT+}: vollständige informationsbasierte Urtheorie mit emergenter Geometrie und Gravitationswellen.
\end{itemize}

\section{Zusammenfassung}
Kapitel~2 hat die konzeptionellen Grundlagen der Informations-Weber-Theorie dargestellt:
\begin{itemize}
    \item Informationszustand als fundamentale Größe,
    \item lokale und globale Informationsdynamik,
    \item analoge WDBT als Fernwirkungstheorie,
    \item digitale WDBT als informationsbasierte Raumtheorie,
    \item Emergenz von Raum, Zeit, Dynamik und Naturkonstanten.
\end{itemize}

Die mathematische Formulierung erfolgt in Kapitel~4 (Informations-Lagrange-Funktional) und Kapitel~5 (Informationsmetrik).
