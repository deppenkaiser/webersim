\chapter{Die klassische Weber-Elektrodynamik}
\label{chap:weberklassisch}

\section{Motivation}
Die Weber-Elektrodynamik stellt einen der frühesten und konsequentesten Versuche dar, elektrische und magnetische Wechselwirkungen ohne Felder zu beschreiben. Statt eines
elektromagnetischen Feldes im Raum verwendet Weber ein Wirkungsprinzip, bei dem Ladungen direkt aufeinander einwirken.

Diese Sichtweise ist für die Informations-Weber-Theorie von zentraler Bedeutung:  
Sie zeigt, dass lokale Dynamik \emph{ohne} Feldkonzepte formuliert werden kann und dass Kräfte aus relationalen Größen entstehen können. Die Weber-Kraft bildet daher den
\textbf{lokalen Grenzfall} der informationsbasierten Dynamik, der entsteht, wenn globale Informationsstrukturen vernachlässigt werden.

Dieses Kapitel stellt die klassische Theorie dar, bevor in Kapitel~4 gezeigt wird, wie sie aus dem lokalen Anteil des Informations-Lagrange-Funktionals hervorgeht.

\section{Historischer Kontext}
Wilhelm Eduard Weber formulierte 1846 eine elektrodynamische Kraft, die sowohl die Coulomb-Wechselwirkung als auch geschwindigkeits- und beschleunigungsabhängige Terme
enthält. Diese Theorie war lange Zeit eine ernsthafte Alternative zu Maxwells Feldtheorie und wurde im 20. Jahrhundert durch Assis und andere rekonstruiert und präzisiert.

Die Weber-Kraft ist bemerkenswert, weil sie:
\begin{itemize}
    \item direkt zwischen Ladungen wirkt (keine Felder als ontologische Objekte),
    \item retardierte Effekte teilweise berücksichtigt,
    \item Energie- und Impulserhaltung strikt respektiert,
    \item magnetische und strahlungsähnliche Effekte aus rein mechanischen Prinzipien ableitet.
\end{itemize}
Diese Eigenschaften machen sie zu einem idealen lokalen Grenzfall der Informations-Weber-Theorie.

\section{Die Weber-Kraft als fundamentale Postulat}
Die Weber-Kraft wird als grundlegendes Postulat der direkten Teilchenwechselwirkung eingeführt. Für zwei Ladungen $q_1$ und $q_2$ mit Abstand $r = |\vec{r}_1 - \vec{r}_2|$, Relativgeschwindigkeit $\dot{r}$ und Relativbeschleunigung $\ddot{r}$ lautet sie:

\begin{equation}
    \vec{F}
    =
    \frac{q_1 q_2}{4\pi\varepsilon_0 r^2}
    \left[
        1
        -
        \frac{\dot{r}^2}{c^2}
        +
        \frac{2 r \ddot{r}}{c^2}
    \right]
    \hat{\vec{r}}.
    \label{eq:weber_kraft}
\end{equation}

\section{Interpretation der Terme}
Die drei Terme der Weber-Kraft haben klare physikalische Bedeutungen:
\begin{itemize}
    \item \textbf{Coulomb-Term ($1$):}  
    beschreibt die statische Fernwirkung zwischen Ladungen.
    
    \item \textbf{Geschwindigkeits-Term ($-\dot{r}^2/c^2$):}  
    erzeugt magnetische Effekte und ist proportional zum Quadrat der Relativgeschwindigkeit.
    
    \item \textbf{Beschleunigungs-Term ($2r\ddot{r}/c^2$):}  
    beschreibt die Reaktion des Systems auf zeitliche Änderungen der Bewegung
    und ist verantwortlich für strahlungsähnliche Widerstandseffekte sowie die partielle Berücksichtigung retardierter Wechselwirkungen.
\end{itemize}

Die Weber-Kraft ist damit eine vollständig mechanische Beschreibung elektromagnetischer Wechselwirkungen, die ohne Felder auskommt.

\section{Struktur der Wechselwirkung}
Die charakteristische Struktur der Weber-Kraft,
\[
F \propto \frac{1}{r^2}\left(1 - \frac{\dot{r}^2}{c^2} + \frac{2r\ddot{r}}{c^2}\right),
\]
zeigt, dass elektromagnetische Phänomene aus rein mechanischen Prinzipien entstehen können. Die Abhängigkeit von $\dot{r}^2$ und $r\ddot{r}$ ist das wesentliche Merkmal der Weber-Theorie und unterscheidet sie fundamental von feldtheoretischen Ansätzen.

\section{Bedeutung für die Informations-Weber-Theorie}
Die Weber-Kraft ist kein konkurrierendes Modell zur informationsbasierten Theorie, sondern ihr \textbf{lokaler Grenzfall}. In Kapitel~4 wird gezeigt, wie die Weber-Kraft
aus dem lokalen Anteil des Informations-Lagrange-Funktionals entsteht und wie das Bohm-Potential als globaler Anteil hinzukommt.

Damit bildet die klassische \gls{wed} die Brücke zwischen historischer Mechanik und moderner informationsbasierter Physik. Sie zeigt, dass lokale Dynamik ohne
Felder formuliert werden kann – ein zentrales Prinzip der Informations-Weber-Theorie.
