\chapter{Das Informations-Lagrange-Funktional}
\label{chap:lagrange}

\section{Einleitung}
Die Informations-Weber-Theorie beschreibt physikalische Systeme durch die Dynamik einer Informationsdichte $\rho_I(\vec{r},t)$. Um diese Dynamik mathematisch zu formulieren,
benötigt man ein Variationsprinzip, das die zeitliche Entwicklung von $\rho_I$ bestimmt. Dieses Kapitel führt das \emph{Informations-Lagrange-Funktional} ein, aus dem alle
Bewegungsgleichungen der Theorie folgen.

Das Informations-Lagrange-Funktional ersetzt im informationsbasierten Rahmen:
\begin{itemize}
    \item die Newtonsche Bewegungsgleichung,
    \item die Maxwell-Gleichungen,
    \item die Schrödinger-Gleichung,
    \item die geometrische Formulierung der Gravitation in der ART.
\end{itemize}
Es bildet die mathematische Grundlage der gesamten Theorie und zeigt, dass diese etablierten Gleichungen als Grenzfälle einer tieferen Informationsdynamik erscheinen.

\section{Grundidee des informationsbasierten Variationsprinzips}
Die Informations-Weber-Theorie geht von folgenden Prinzipien aus:
\begin{enumerate}
    \item Der physikalische Zustand ist eine Informationsverteilung $\rho_I$.
    \item Information ist eine Erhaltungsgröße.
    \item Dynamik ist Umlagerung von Information.
    \item Lokale und globale Dynamik sind komplementär.
\end{enumerate}
Daraus folgt, dass die Dynamik durch ein Funktional beschrieben werden muss, das sowohl lokale als auch globale Informationsstrukturen berücksichtigt.

\section{Definition des Informations-Lagrange-Funktionals}
Das Informations-Lagrange-Funktional lautet allgemein:
\begin{equation}
    \mathcal{L}_I[\rho_I]
    =
    \int
    \mathcal{F}
    \!\left(
        \rho_I,\,
        \nabla \rho_I,\,
        \partial_t \rho_I
    \right)
    d^3x.
    \label{eq:info_lagrange}
\end{equation}

Die skalare Dichte $\mathcal{F}$ beschreibt die lokale Struktur des Informationsraums. Sie hängt ab von:
\begin{itemize}
    \item der Informationsdichte $\rho_I$,
    \item ihren räumlichen Gradienten $\nabla \rho_I$,
    \item ihrer zeitlichen Ableitung $\partial_t \rho_I$.
\end{itemize}
Die Form von $\mathcal{F}$ ergibt sich aus den Axiomen der Theorie und den Symmetrien des Informationsraums.

\section{Variation und Euler-Lagrange-Gleichungen}
Die Dynamik folgt aus dem Variationsprinzip:
\[
    \delta \mathcal{L}_I = 0.
\]
Die Variation nach $\rho_I$ führt zur Euler-Lagrange-Gleichung:
\begin{equation}
    \frac{\partial}{\partial t}
    \left(
        \frac{\partial \mathcal{F}}{\partial (\partial_t \rho_I)}
    \right)
    +
    \nabla \cdot
    \left(
        \frac{\partial \mathcal{F}}{\partial (\nabla \rho_I)}
    \right)
    -
    \frac{\partial \mathcal{F}}{\partial \rho_I}
    = 0.
    \label{eq:euler_lagrange_info}
\end{equation}
Diese Gleichung ist die fundamentale Bewegungsgleichung der Informations-Weber-Theorie.

\section{Informationsfluss als natürliche Konsequenz}
Aus der Euler-Lagrange-Gleichung folgt unmittelbar der Informationsfluss:
\[
    \vec{J}_I
    =
    \frac{\partial \mathcal{F}}{\partial (\nabla \rho_I)}.
\]
Damit wird die Kontinuitätsgleichung
\[
    \frac{\partial \rho_I}{\partial t}
    + \nabla \cdot \vec{J}_I = 0
\]
zu einer direkten Konsequenz des Variationsprinzips.

\section{Zerlegung in lokale und globale Beiträge}
Die Struktur von $\mathcal{F}$ erlaubt eine natürliche Zerlegung:
\begin{equation}
    \mathcal{F}
    =
    \mathcal{F}_{\text{lokal}}
    +
    \mathcal{F}_{\text{global}}.
\end{equation}

\subsection{Lokaler Anteil}
Der lokale Anteil beschreibt lokale Informationsflüsse:
\[
    \mathcal{F}_{\text{lokal}}
    =
    \alpha\, (\partial_t \rho_I)^2
    +
    \beta\, (\nabla \rho_I)^2
    + \ldots
\]
Er führt im Grenzfall zu:
\begin{itemize}
    \item der Weber-Kraft (Kapitel~\ref{chap:weberklassisch}),
    \item klassischen Trägheits- und Energiebegriffen.
\end{itemize}
Damit entspricht der lokale Anteil der direkten, mechanischen Dynamik.

\subsection{Globaler Anteil}
Der globale Anteil beschreibt systemische Informationsorganisation:
\[
    \mathcal{F}_{\text{global}}
    =
    \gamma\, \frac{(\nabla \rho_I)^2}{\rho_I}
    + \ldots
\]
Die Variation dieses Terms führt zum Bohm’schen Quantenpotential:
\[
    Q
    =
    -\frac{\hbar^2}{2m}
    \frac{\nabla^2 \sqrt{\rho_I}}{\sqrt{\rho_I}}.
\]
Damit ist $Q$ kein Zusatz der Quantenmechanik, sondern der globale Anteil der Informationsdynamik.

\section{Weber-Kraft und Bohm-Potential als Grenzfälle}
Die Informations-Weber-Theorie reproduziert zwei klassische Strukturen:
\begin{itemize}
    \item \textbf{Weber-Kraft:}  
    lokaler Grenzfall der Informationsdynamik.

    \item \textbf{Bohm-Potential:}  
    globaler Grenzfall der Informationsdynamik.
\end{itemize}
Beide entstehen aus demselben Funktional — sie sind keine unabhängigen Modelle, sondern zwei Projektionen derselben Informationsstruktur.

\section{Symmetrien und Erhaltungsgrößen}
Nach dem Noether-Theorem ergeben sich:
\begin{itemize}
    \item Translationssymmetrie $\Rightarrow$ Impulserhaltung,
    \item Zeitsymmetrie $\Rightarrow$ Energieerhaltung,
    \item Rotationssymmetrie $\Rightarrow$ Drehimpulserhaltung.
\end{itemize}
Damit sind Energie und Impuls keine primitiven Größen, sondern abgeleitete Informationsfunktionale.

\section{Zusammenfassung}
Das Informations-Lagrange-Funktional bildet die mathematische Grundlage der Informations-Weber-Theorie:
\begin{itemize}
    \item Es beschreibt die Dynamik der Informationsdichte.
    \item Es erzeugt die Kontinuitätsgleichung.
    \item Es zerlegt sich in lokale und globale Beiträge.
    \item Es reproduziert Weber-Kraft und Bohm-Potential.
    \item Es liefert Energie- und Impulserhaltung aus Symmetrien.
\end{itemize}
Kapitel~5 entwickelt darauf aufbauend die Informationsmetrik und die emergente Raumzeit.
