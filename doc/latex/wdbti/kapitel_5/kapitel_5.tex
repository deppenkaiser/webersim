\chapter{Informationsmetrik und emergente Raumzeit}
\label{chap:informationsmetrik}

\section{Einleitung}
Die Informations-Weber-Theorie geht davon aus, dass Raum und Zeit keine fundamentalen Größen sind, sondern aus der Struktur der Informationsverteilung $\rho_I(\vec{r},t)$
emergieren. Während Kapitel~\ref{chap:lagrange} das Variationsprinzip der Informationsdynamik formuliert hat, entwickelt dieses Kapitel die geometrische Struktur, die aus
dieser Dynamik hervorgeht.

Die zentrale Idee lautet:
\[
    \textbf{Raum ist die effektive Metrik der Informationskopplung.}
\]
Damit wird die klassische Raumzeit der ART nicht verworfen, sondern als Grenzfall einer tieferen informationsbasierten Geometrie verstanden.

\section{Von der Informationsdynamik zur Geometrie}
Die Informations-Weber-Theorie unterscheidet zwei Ebenen:
\begin{itemize}
    \item \textbf{analoge WDBT:} Fernwirkung ohne Raummodell,
    \item \textbf{digitale WDBT:} diskretes Informationsnetz, aus dem Raum emergiert.
\end{itemize}
Die analoge Theorie beschreibt direkte Wechselwirkungen, benötigt aber kein ontologisches Raumzeitkontinuum. Erst die digitale Theorie führt ein Netzwerk aus
Informationsknoten ein, dessen Kopplungsstruktur eine effektive Geometrie erzeugt.

\section{Definition der Informationsmetrik}
Die Informationsmetrik entsteht aus der Sensitivität des Informations-Lagrange-Funktionals gegenüber räumlichen Änderungen der Informationsdichte. Formal ergibt sich die
Metrik aus:
\begin{equation}
    g_{ij}
    =
    \frac{\partial^2 \mathcal{F}}
    {\partial (\partial_i \rho_I)\, \partial (\partial_j \rho_I)}.
    \label{eq:info_metrik}
\end{equation}
Diese Definition ist analog zur Fisher-Informationsmetrik, jedoch nicht statistisch, sondern dynamisch motiviert.

\subsection{Interpretation}
\begin{itemize}
    \item Große Werte von $g_{ij}$ bedeuten, dass kleine Änderungen der Informationsstruktur
    große dynamische Effekte haben.
    \item Kleine Werte bedeuten, dass die Informationsstruktur „weich“ ist.
\end{itemize}
Die Metrik misst also die \emph{Steifigkeit der Informationsgeometrie}.

\section{Emergenz des physikalischen Raumes}
Der physikalische Raum entsteht als effektive Geometrie des Informationsnetzes:
\[
    ds^2 = g_{ij}\, dx^i dx^j.
\]
Damit ist Raum keine ontologische Entität, sondern eine Projektion der Informationskopplungen auf eine kontinuierliche Beschreibung.

\subsection{Diskrete Struktur}
In der digitalen WDBT besteht der Informationszustand aus:
\begin{itemize}
    \item Knoten (Informationspunkte),
    \item Kopplungen (Informationsflüsse),
    \item Aktualisierungsregeln (Dynamik).
\end{itemize}
Die Metrik ist die effektive Beschreibung dieser Kopplungsstruktur.

\section{Emergenz der Zeit}
Zeit entsteht aus der Ordnung der Aktualisierungsschritte des Informationsnetzes:
\[
    I_0 \rightarrow I_1 \rightarrow I_2 \rightarrow \cdots
\]
Die physikalische Zeit ist die Kontinuumsnäherung dieser Sequenz.

\subsection{Zwei Zeitstrukturen}
Die Theorie unterscheidet:
\begin{itemize}
    \item \textbf{lokale Zeit:} bestimmt durch Transportprozesse,
    \item \textbf{globale Zeit:} bestimmt durch systemische Informationsorganisation.
\end{itemize}
Die beobachtete Zeit ist die Überlagerung beider Strukturen.

\section{Fraktale Dimension als geometrische Signatur}
Die Kopplungsstruktur des Informationsnetzes besitzt eine fraktale Dimension:
\[
    D = \frac{\ln 20}{\ln(2+\phi)}.
\]
Diese Dimension ist keine Raumdimension, sondern eine Skalierungsstruktur der Informationskopplungen.

\subsection{Makroskopischer Grenzfall}
Für große Skalen gilt:
\[
    D \to 3,
\]
wodurch der klassische dreidimensionale Raum entsteht.

\section{Informationsgeometrie und Dynamik}
Die Dynamik eines Systems ergibt sich aus der Änderung der Informationsgeometrie:
\[
    \text{Dynamik} = \frac{d}{dt} g_{ij}[\rho_I].
\]

\subsection{Lokale Projektion}
Die Weber-Kraft ist die lokale Projektion der Informationsgeometrie:
\[
    F_{\text{lokal}} = F_{\text{WED}} + F_{\text{WG}}.
\]

\subsection{Globale Projektion}
Das Bohm-Potential ist die globale Projektion:
\[
    F_Q = -\nabla Q.
\]

\section{Gravitationswellen als Moden der Informationsgeometrie}
In der digitalen WDBT entstehen Gravitationswellen als kollektive Moden der Informationsgeometrie. Sie sind keine Schwingungen eines Kontinuums, sondern Muster der
Kopplungsänderungen im Informationsnetz.

\section{Vergleich mit der Allgemeinen Relativitätstheorie}
Die ART beschreibt Gravitation durch eine glatte Raumzeitmetrik $g_{\mu\nu}$. Die Informations-Weber-Theorie beschreibt Gravitation durch eine informationsbasierte
Metrik $g_{ij}$.

\subsection{Gemeinsamkeiten}
\begin{itemize}
    \item Beide Theorien verwenden eine Metrik.
    \item Beide Theorien beschreiben Geodäten als Bewegungsgleichungen.
\end{itemize}

\subsection{Unterschiede}
\begin{itemize}
    \item Die ART postuliert die Raumzeit als fundamental.
    \item Die Informations-Weber-Theorie lässt Raum und Zeit emergieren.
    \item Die ART kennt Singularitäten; die informationsbasierte Theorie nicht.
\end{itemize}

\section{Zusammenfassung}
Kapitel~5 hat gezeigt:
\begin{itemize}
    \item Die Informationsmetrik entsteht aus der Struktur des Informations-Lagrange-Funktionals.
    \item Raum ist die effektive Geometrie der Informationskopplung.
    \item Zeit entsteht aus der Ordnung der Informationsaktualisierungen.
    \item Die fraktale Dimension ist die Skalierungssignatur der Informationsarchitektur.
    \item Gravitationswellen sind Moden der Informationsgeometrie.
    \item Die ART ist der makroskopische Grenzfall der informationsbasierten Geometrie.
\end{itemize}

Damit ist die geometrische Grundlage der Informations-Weber-Theorie vollständig gelegt.
