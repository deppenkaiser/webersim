\chapter{Vergleich mit etablierten Theorien}
\label{chap:vergleich}

\section{Einleitung}
Die Informations-Weber-Theorie wurde in den vorangegangenen Kapiteln als eine fundamentale Urtheorie entwickelt, in der physikalische Größen und Dynamiken aus der Struktur
und Transformation von Information hervorgehen. In diesem Kapitel wird gezeigt, wie sich diese Theorie zu den etablierten physikalischen Modellen verhält. Ziel ist es
nicht, diese Modelle zu ersetzen, sondern ihre Gültigkeitsbereiche, Grenzen und emergenten Eigenschaften aus informationsbasierter Sicht zu verstehen.

Die etablierten Theorien der Physik lassen sich grob in vier Klassen einteilen:

\begin{enumerate}
    \item klassische Mechanik,
    \item Elektrodynamik (Maxwell, Lorentz, Weber),
    \item Quantenmechanik und Quantenfeldtheorie,
    \item Relativitätstheorie (SRT und ART).
\end{enumerate}

Jede dieser Theorien besitzt einen klar definierten Gültigkeitsbereich und liefert dort präzise Vorhersagen. Gleichzeitig weisen sie fundamentale Spannungen auf, die auf
eine tiefere, einheitliche Struktur hindeuten. Die Informations-Weber-Theorie bietet einen Rahmen, in dem diese Modelle als Grenzfälle einer universellen
Informationsdynamik verstanden werden können.

\section{Klassische Mechanik als lokaler Grenzfall}
Die klassische Mechanik basiert auf Newtons Axiomen, insbesondere auf dem zweiten Axiom
\[
    \vec{F} = m \vec{a}.
\]
In der Informations-Weber-Theorie entsteht diese Gleichung als Grenzfall lokaler Informationsdynamik. Die effektive Masse ist dabei keine fundamentale Größe, sondern
ein Maß für die Steifigkeit der Informationsstruktur:
\[
    m_{\text{eff}} \propto \int (\partial_t \rho_I)^2 d^3x.
\]
Damit wird die klassische Mechanik als Näherung einer tieferen Dynamik verstanden, die nur gültig ist, wenn:

\begin{itemize}
    \item Informationsgradienten klein sind,
    \item globale Informationsstrukturen vernachlässigt werden können,
    \item Geschwindigkeiten klein gegenüber $c$ sind.
\end{itemize}

Die klassische Mechanik ist somit eine lokale, niederenergetische Näherung der Informations-Weber-Theorie.

\section{Elektrodynamik: Maxwell, Lorentz und Weber}
Die Elektrodynamik existiert in drei historischen Formulierungen:

\begin{enumerate}
    \item \textbf{Maxwell-Felder} (kontinuierliche Felder im Raum),
    \item \textbf{Lorentz-Kraft} (Felder + Ladungen),
    \item \textbf{Weber-Kraft} (direkte Wechselwirkung).
\end{enumerate}

\subsection{Maxwell-Theorie als effektive Feldbeschreibung}
Die Maxwell-Gleichungen beschreiben elektromagnetische Felder als kontinuierliche Objekte im Raum. In der Informations-Weber-Theorie erscheinen diese Felder als
\emph{effektive makroskopische Beschreibungen} von Informationsflüssen.

Die Feldstärke $F_{\mu\nu}$ ist dabei kein fundamentales Objekt, sondern ein Mittelwert über lokale Informationsgradienten.

\subsection{Lorentz-Kraft als phänomenologische Näherung}
Die Lorentz-Kraft
\[
    \vec{F} = q(\vec{E} + \vec{v} \times \vec{B})
\]
entsteht als phänomenologische Näherung, wenn die Informationsstruktur durch Felder parametrisiert wird. Sie ist gültig, wenn:

\begin{itemize}
    \item retardierte Effekte klein sind,
    \item Beschleunigungen gering sind,
    \item globale Informationsstrukturen vernachlässigt werden.
\end{itemize}

\subsection{Weber-Kraft als lokaler Grenzfall}
Die Weber-Kraft
\[
    \vec{F}_{\text{Weber}}
    =
    \frac{q_1 q_2}{4\pi\varepsilon_0 r^2}
    \left[
        1
        -
        \frac{\dot{r}^2}{c^2}
        +
        \frac{2 r \ddot{r}}{c^2}
    \right]
    \hat{\vec{r}}
\]
ist der \emph{exakte lokale Grenzfall} der Informations-Weber-Theorie, wenn globale Informationsstrukturen vernachlässigt werden.

Damit ergibt sich eine klare Hierarchie:
\[
    \text{Informations-Weber-Theorie}
    \;\longrightarrow\;
    \text{Weber}
    \;\longrightarrow\;
    \text{Lorentz}
    \;\longrightarrow\;
    \text{Maxwell}.
\]

\section{Quantenmechanik als globale Informationsdynamik}
Die Quantenmechanik basiert auf der Schrödinger-Gleichung
\[
    i\hbar \partial_t \Psi = \hat{H} \Psi.
\]
In der Informations-Weber-Theorie ist die Wellenfunktion kein ontologisches Objekt, sondern eine parametrische Darstellung der Informationsdichte:
\[
    \rho_I = |\Psi|^2.
\]
Das Bohm-Potential
\[
    Q = -\frac{\hbar^2}{2m}
    \frac{\nabla^2 \sqrt{\rho_I}}{\sqrt{\rho_I}}
\]
entsteht als globaler Anteil des Informations-Lagrange-Funktionals.

Damit wird die Quantenmechanik als \emph{globaler Grenzfall} verstanden, der gültig ist, wenn:

\begin{itemize}
    \item globale Informationsstrukturen dominieren,
    \item lokale Dynamik vernachlässigt werden kann,
    \item kohärente Informationsphasen existieren.
\end{itemize}

\section{Relativitätstheorie als emergente Geometrie}
Die SRT und ART basieren auf der Idee, dass Raum und Zeit eine feste geometrische Struktur besitzen. In der Informations-Weber-Theorie ist diese Struktur nicht
fundamental, sondern emergent.

\subsection{SRT als Symmetrie des Informationsflusses}

Die Lorentz-Invarianz entsteht aus der Symmetrie des Informationsflusses bei maximaler Informationsgeschwindigkeit $c$. Sie ist gültig, wenn:

\begin{itemize}
    \item Informationsgradienten homogen sind,
    \item globale Strukturen vernachlässigt werden,
    \item keine fraktalen Effekte auftreten.
\end{itemize}

\subsection{ART als effektive Informationsgeometrie}
Die ART beschreibt Gravitation als Krümmung der Raumzeit. In der Informations-Weber-Theorie ist diese Krümmung eine effektive Beschreibung der Informationsmetriken:
\[
    g_{ij}
    =
    \frac{\partial^2 \mathcal{F}}{\partial (\partial_i \rho_I)\, \partial (\partial_j \rho_I)}.
\]
Die ART ist gültig, wenn:

\begin{itemize}
    \item Informationsdichten groß sind,
    \item globale Strukturen langsam variieren,
    \item fraktale Effekte vernachlässigt werden können.
\end{itemize}

\section{Zusammenfassung}
Die Informations-Weber-Theorie integriert die etablierten Theorien als Grenzfälle:

\begin{itemize}
    \item klassische Mechanik: lokale, niederenergetische Näherung,
    \item Weber-Elektrodynamik: exakter lokaler Grenzfall,
    \item Quantenmechanik: globaler Grenzfall,
    \item Relativitätstheorie: emergente Informationsgeometrie.
\end{itemize}

Damit entsteht ein einheitliches, reduktionistisches Bild der Physik, in dem alle bekannten Modelle als approximative Manifestationen eines fundamentalen
Informationsprinzips verstanden werden.

\section{Frequenzabhängige Lichtablenkung als Test der Theorie}
Ein zentraler Unterschied zwischen der Allgemeinen Relativitätstheorie und der Informations-Weber-Theorie betrifft die Ablenkung von Licht im Gravitationsfeld.
Während die ART eine frequenzunabhängige Ablenkung vorhersagt, ergibt sich in der Informations-Weber-Theorie eine explizite Frequenzabhängigkeit.

\subsection{Vorhersage der ART}
In der ART folgt Licht einer nullartigen Geodäte. Die Ablenkung am Sonnenrand beträgt
\[
    \delta\theta_{\text{ART}}
    =
    \frac{4GM}{c^2 b},
\]
unabhängig von Frequenz oder Energie des Photons.

\subsection{Vorhersage der Informations-Weber-Theorie}
In der Informations-Weber-Theorie besitzt ein Photon eine effektive Informationssteifigkeit, die von seiner Frequenz abhängt. Dadurch ergibt sich eine frequenzabhängige
Ablenkung:
\[
    \delta\theta(\nu)
    =
    \delta\theta_0
    \left(
        1 + \alpha \frac{\nu_0}{\nu}
    \right),
\]
wobei $\alpha$ eine dimensionslose Kopplungskonstante ist.

Damit gilt:
\[
    \delta\theta_{\text{blau}} < \delta\theta_{\text{rot}}.
\]

\subsection{Experimentelle Tests}
Die frequenzabhängige Ablenkung kann getestet werden durch:

\begin{itemize}
    \item spektral aufgelöste Sonnenrandmessungen,
    \item Gravitationslinsen im optischen, Röntgen- und Radiobereich,
    \item Pulsar-Timing und Fast Radio Bursts.
\end{itemize}

Eine nachgewiesene Frequenzabhängigkeit würde die ART falsifizieren und die
Informations-Weber-Theorie bestätigen.

\subsection{Von WDBT → ART → ART+ → WDBT+}
Die verschiedenen Ebenen der Gravitationstheorie lassen sich im Rahmen der Informations-Weber-Theorie klar hierarchisieren. Jede Ebene besitzt eine eigene
ontologische Struktur und einen eigenen Gültigkeitsbereich. Die folgende Abfolge beschreibt die logische Entwicklung von der fernwirkungsbasierten Dynamik bis zur
vollständigen informationsbasierten Raumgeometrie.

\paragraph{(1) WDBT: Fernwirkung ohne Raum, aber mit Quantenstruktur}
Die analoge Weber–De-Broglie-Theorie (WDBT) besitzt kein Raummodell. Sie beschreibt Gravitation und Elektrodynamik als Fernwirkungen
\[
    F = F_{\text{WED}} + F_{\text{WG}} + F_Q,
\]
wobei $F_Q$ das Bohm’sche Quantenpotential enthält. Damit verfügt die WDBT bereits über eine vollständige nichtlokale Quantenstruktur, jedoch ohne geometrische
Interpretation. Klassische Effekte wie die Periheldrehung ergeben sich direkt aus $F_{\text{WG}}$.

\paragraph{(2) ART: Geometrische Raumzeit ohne Quantenstruktur}
Die Allgemeine Relativitätstheorie (ART) steht zwischen der analogen WDBT und
der informationsbasierten WDBT+. Sie übernimmt die gravitative Dynamik der
WDBT, ersetzt jedoch die Fernwirkung durch ein geometrisches Raummodell. Die
ART ist mächtiger als die analoge WDBT, da sie Gravitationswellen und eine
vollständige Raumzeitgeometrie beschreibt. Gleichzeitig fehlt ihr die
nichtlokale Informationsstruktur des Quantenpotentials.

\paragraph{(3) ART+: ART erweitert um die Informationsstruktur der WDBT}
Wird die ART um das Bohm’sche Quantenpotential und das Prinzip der
Informationserhaltung ergänzt, entsteht eine erweiterte Theorie, die wir als
ART+ bezeichnen. In dieser Theorie werden echte Singularitäten vermieden, da
die Informationsdichte nicht divergieren kann. Der Urknall wird durch einen
\emph{Big Bounce} ersetzt:
\[
    \rho_I^{\text{min}} > 0 \quad \Rightarrow \quad \text{kein Big Bang}.
\]
ART+ verbindet geometrische Gravitation mit nichtlokaler Quantenorganisation, bleibt jedoch eine Kontinuumstheorie.

\paragraph{(4) WDBT+: Digitale Informations-Weber-Theorie als fundamentale Ebene}
Die WDBT+ führt ein diskretes Informationsnetz ein, aus dem Raum, Zeit und
Dynamik emergieren. Sie ist damit mächtiger als ART und ART+, da sie
Phänomene erklären kann, die in der ART nur postuliert oder nicht verstanden
werden:
\begin{itemize}
    \item Gravitationswellen als Moden der Informationsgeometrie,
    \item Naturkonstanten als Skalierungsparameter der Informationsarchitektur,
    \item Rotationskurven ohne Dunkle Materie,
    \item CMB-Struktur als fossilierte Informationsgeometrie,
    \item Big Bounce statt Big Bang,
    \item fraktale Raumdimension als emergente Eigenschaft.
\end{itemize}
ART und ART+ erscheinen in dieser Sichtweise als effektive Grenzfälle der
diskreten Informationsgeometrie.

\paragraph{Konsequenz}
Die Entwicklungslinie
\[
    \text{WDBT} \;\longrightarrow\; \text{ART} \;\longrightarrow\; \text{ART+}
    \;\longrightarrow\; \text{WDBT+}
\]
beschreibt den Übergang von einer rein dynamischen Fernwirkungstheorie über geometrische Kontinuumsmodelle hin zu einer fundamentalen informationsbasierten Urtheorie.
Erst die WDBT+ vereinigt Gravitation, Quantenstruktur und Raumgeometrie in einem konsistenten Rahmen.
