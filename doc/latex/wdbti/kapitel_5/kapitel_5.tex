\chapter{Informationsmetrik und emergente Raumzeit}
\label{chap:informationsmetrik}

\paragraph{Hinweis zur mathematischen Darstellung}
Dieses Kapitel verwendet größtenteils die \emph{kontinuierliche Notation} für Kompaktheit. Die zugrundeliegende fundamentale Formulierung ist diskret rekursiv. Wo nötig
wird die diskrete Form explizit angegeben. Eine vollständige diskrete Darstellung findet sich in Kapitel X.

\section{Einleitung}
Die Informations-Weber-Theorie geht davon aus, dass Raum und Zeit keine fundamentalen Größen sind, sondern aus der Struktur der Informationsverteilung $\rho_I(\vec{r},t)$
emergieren. Während Kapitel~\ref{chap:lagrange} das Variationsprinzip der Informationsdynamik formuliert hat, entwickelt dieses Kapitel die geometrische Struktur, die aus
dieser Dynamik hervorgeht.

Die zentrale Idee lautet:
\[
    \textbf{Raum ist die effektive Metrik der Informationskopplung.}
\]
Damit wird die klassische Raumzeit der ART nicht verworfen, sondern als makroskopischer Grenzfall einer tieferen informationsbasierten Geometrie verstanden.

\section{Von der Informationsdynamik zur Geometrie}
Die Informations-Weber-Theorie unterscheidet zwei Ebenen:
\begin{itemize}
    \item \textbf{analoge \gls{wdbt}:} Fernwirkung ohne Raummodell,
    \item \textbf{digitale \gls{wdbt}:} diskretes Informationsnetz, aus dem Raum emergiert.
\end{itemize}
Die analoge Theorie beschreibt direkte Wechselwirkungen, benötigt aber kein ontologisches Raumzeitkontinuum. Erst die digitale Theorie führt ein Netzwerk aus
Informationsknoten ein, dessen Kopplungsstruktur eine effektive Geometrie erzeugt.

\section{Definition der Informationsmetrik}
Die Informationsmetrik entsteht aus der Sensitivität des Informations-Lagrange-Funktionals gegenüber räumlichen Änderungen der Informationsdichte. Formal ergibt sich die
Metrik aus:
\begin{equation}
    g_{ij}
    =
    \frac{\partial^2 \mathcal{F}}
    {\partial (\partial_i \rho_I)\, \partial (\partial_j \rho_I)}.
    \label{eq:info_metrik}
\end{equation}
Diese Definition ist analog zur Fisher-Informationsmetrik, jedoch nicht statistisch, sondern dynamisch motiviert.

\subsection{Interpretation}
\begin{itemize}
    \item Große Werte von $g_{ij}$ bedeuten, dass kleine Änderungen der Informationsstruktur
    große dynamische Effekte haben.
    \item Kleine Werte bedeuten, dass die Informationsstruktur „weich“ ist.
\end{itemize}
Die Metrik misst also die \emph{Steifigkeit der Informationsgeometrie}.

\section{Emergenz des physikalischen Raumes}
Der physikalische Raum entsteht als effektive Geometrie des Informationsnetzes:
\[
    ds^2 = g_{ij}\, dx^i dx^j.
\]
Damit ist Raum keine ontologische Entität, sondern eine Projektion der Informationskopplungen auf eine kontinuierliche Beschreibung.

\subsection{Diskrete Struktur}
In der digitalen \gls{wdbt} besteht der Informationszustand aus:
\begin{itemize}
    \item Knoten (Informationspunkte),
    \item Kopplungen (Informationsflüsse),
    \item Aktualisierungsregeln (Dynamik).
\end{itemize}
Die Metrik ist die effektive Beschreibung dieser Kopplungsstruktur.

\section{Emergenz der Zeit}
Zeit entsteht aus der Ordnung der Aktualisierungsschritte des Informationsnetzes:
\[
    I_0 \rightarrow I_1 \rightarrow I_2 \rightarrow \cdots
\]
Die physikalische Zeit ist die Kontinuumsnäherung dieser Sequenz.

\subsection{Zwei Zeitstrukturen}
Die Theorie unterscheidet:
\begin{itemize}
    \item \textbf{lokale Zeit:} bestimmt durch Transportprozesse,
    \item \textbf{globale Zeit:} bestimmt durch systemische Informationsorganisation.
\end{itemize}
Die beobachtete Zeit ist die Überlagerung beider Strukturen.

\section{Fraktale Dimension als geometrische Signatur}
Die Kopplungsstruktur des Informationsnetzes besitzt eine fraktale Dimension:
\[
    D = \frac{\ln 20}{\ln(2+\phi)}.
\]
Diese Dimension ist keine Raumdimension, sondern eine Skalierungsstruktur der Informationskopplungen.

\subsection{Makroskopischer Grenzfall}
Für große Skalen gilt:
\[
    D \to 3,
\]
wodurch der klassische dreidimensionale Raum entsteht.

\section{Informationsgeometrie und Dynamik}
Die Dynamik eines Systems ergibt sich aus der Änderung der Informationsgeometrie:
\[
    \text{Dynamik} = \frac{d}{dt} g_{ij}[\rho_I].
\]

\subsection{Lokale Projektion}
Die Weber-Kraft ist die lokale Projektion der Informationsgeometrie:
\[
    F_{\text{lokal}} = F_{\text{WED}} + F_{\text{WG}}.
\]
Sie beschreibt die direkte, mechanische Dynamik, die im informationsbasierten Rahmen als lokale Struktur erscheint.

\subsection{Globale Projektion}
Das Bohm-Potential ist die globale Projektion:
\[
    F_Q = -\nabla Q.
\]
Es beschreibt die systemische, nichtlokale Organisation des Informationszustands.

\section{Gravitationswellen als Moden der Informationsgeometrie}
In der digitalen \gls{wdbt} entstehen Gravitationswellen als kollektive Moden der Informationsgeometrie. Sie sind keine Schwingungen eines Kontinuums, sondern Muster der
Kopplungsänderungen im Informationsnetz. Damit wird die Stärke der \gls{art} – die Beschreibung dynamischer Geometrie – in einen informationsbasierten Rahmen überführt.

\section{Vergleich mit der Allgemeinen Relativitätstheorie}
Die \gls{art} beschreibt Gravitation durch eine glatte Raumzeitmetrik $g_{\mu\nu}$. Die Informations-Weber-Theorie beschreibt Gravitation durch eine informationsbasierte
Metrik $g_{ij}$.

\subsection{Gemeinsamkeiten}
\begin{itemize}
    \item Beide Theorien verwenden eine Metrik.
    \item Beide Theorien beschreiben Geodäten als Bewegungsgleichungen.
\end{itemize}

\subsection{Unterschiede}
\begin{itemize}
    \item Die \gls{art} postuliert die Raumzeit als fundamental.
    \item Die Informations-Weber-Theorie lässt Raum und Zeit emergieren.
    \item Die \gls{art} kennt Singularitäten; die informationsbasierte Theorie nicht.
    \item Die \gls{art} ist eine makroskopische Näherung; die informationsbasierte Theorie ist fundamental.
\end{itemize}

\section{Zusammenfassung}
Kapitel~5 hat gezeigt:
\begin{itemize}
    \item Die Informationsmetrik entsteht aus der Struktur des Informations-Lagrange-Funktionals.
    \item Raum ist die effektive Geometrie der Informationskopplung.
    \item Zeit entsteht aus der Ordnung der Informationsaktualisierungen.
    \item Die fraktale Dimension ist die Skalierungssignatur der Informationsarchitektur.
    \item Gravitationswellen sind Moden der Informationsgeometrie.
    \item Die \gls{art} ist der makroskopische Grenzfall der informationsbasierten Geometrie.
\end{itemize}

Damit ist die geometrische Grundlage der Informations-Weber-Theorie vollständig gelegt.

\section{Dynamik des Informationsfeldes}
\subsection{Diskrete Update-Regel}
Für einen Informationsknoten $k$ zum Schritt $n$:
\[
I_k^{(n+1)} = I_k^{(n)} + T \cdot \Phi\left(
I_k^{(n)}, 
\{I_l^{(n)}\}_{l\in\mathcal{N}(k)}, 
I_k^{(n-1)},
\{\nabla_d^2 I^{(n)}\}
\right)
\]
mit dem diskreten Laplace-Operator $\nabla_d^2$.

\subsection{Kontinuierlicher Grenzfall}
Bei Mittelung über viele Knoten ergibt sich im Limes:
\[
\frac{\partial I}{\partial t} = \alpha \nabla^2 I + \beta \frac{(\nabla I)^2}{I} + \gamma I \ln(I/I_0)
\]

\subsection{Variationsprinzip}
Die kontinuierliche Form kann kompakt als Euler-Lagrange-Gleichung geschrieben werden:
\[
\frac{\delta \mathcal{L}}{\delta I} = 0, \quad 
\mathcal{L} = \int \left[ \frac{1}{2}(\partial_t I)^2 - \frac{1}{2}(\nabla I)^2 - V(I) \right] d^3x
\]
Dies ist jedoch eine \emph{effektive} Beschreibung, nicht die fundamentale.

\section{Dynamik der emergenten Metrik}

\subsection{Diskrete Kopplungsaktualisierung}
Die effektive Metrik zwischen Knotengruppen $A$ und $B$ entwickelt sich gemäß:
\[
g_{AB}^{(n+1)} = g_{AB}^{(n)} + T \cdot \left[
\frac{I_A^{(n)} - I_A^{(n-1)}}{T} \cdot \frac{I_B^{(n)} - I_B^{(n-1)}}{T}
- \lambda \frac{\nabla_d^2 I_{AB}^{(n)}}{I_{AB}^{(n)}}
+ \mu g_{AB}^{(n)} \ln\left(1 + \gamma G \rho L^2\right)
\right]
\]

\subsection{Kontinuierliche Notation}
Im makroskopischen Grenzfall:
\[
\frac{d}{dt}g_{ij} = \partial_i I \partial_j I - \lambda \frac{\partial_i\partial_j I}{I} + \mu g_{ij} \ln(1 + \gamma G\rho L^2)
\]

\subsection{Zeitbegriff}
Die Ableitung $d/dt$ bezieht sich hier auf die \emph{emergent kontinuierliche Zeit} $t = nT$. Fundamentaler ist der Schrittindex $n$.

\chapter{Mathematik der diskreten Dynamik}

\section{Diskrete vs. kontinuierliche Beschreibung}
Physik wurde historisch kontinuierlich formuliert, obwohl Messungen diskret sind. Die IWT macht diese Diskretion fundamental.

\section{Diskrete Ableitungen}
\begin{align}
\text{Diskret:} & \quad \frac{df}{dt} \rightarrow \frac{f^{(n)} - f^{(n-1)}}{T} \\
\text{Zentral:} & \quad \frac{d^2f}{dt^2} \rightarrow \frac{f^{(n+1)} - 2f^{(n)} + f^{(n-1)}}{T^2}
\end{align}

\section{Rekursive Gleichungen}
Viele fundamentale Gleichungen sind in diskreter Form \emph{explizit rekursiv}:
\[
x^{(n+1)} = F(x^{(n)}, x^{(n-1)}, \dots, \text{Parameter})
\]
Dies vermeidet die „Zirkularität“ kontinuierlicher impliziter Gleichungen.

\section{Informationserhaltung diskret}
Die diskrete Erhaltungsgleichung:
\[
\sum_k I_k^{(n)} = \text{const} \quad \forall n
\]
ist fundamentaler als die kontinuierliche $\partial_t \rho + \nabla\cdot\vec{J} = 0$.
