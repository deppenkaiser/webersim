\chapter{Emergenz klassischer und quantenmechanischer Phänomene}
\label{chap:emergenz}

\section{Einleitung}
Die Informations-Weber-Theorie beschreibt physikalische Systeme nicht durch Felder, materielle Substanzen oder eine ontologische Raumzeit, sondern durch die Struktur und
Dynamik einer Informationsverteilung. Die klassischen und quantenmechanischen Gesetze entstehen dabei nicht als fundamentale Postulate, sondern als
\emph{emergente Ordnungsprinzipien} der Informationsdynamik.

Die Emergenz erfolgt in zwei komplementären Schritten:
\begin{enumerate}
    \item \textbf{Lokale Dynamik}  
    erzeugt klassische Phänomene wie Trägheit, Newtonsche Gravitation und die Weber-Kraft.

    \item \textbf{Globale Dynamik}  
    erzeugt quantenmechanische Phänomene wie Interferenz, Nichtlokalität und das Bohm-Potential.
\end{enumerate}
Damit wird die traditionelle Trennung zwischen „klassisch“ und „quantum“ aufgehoben: Beide sind Manifestationen derselben informationsbasierten Struktur.

\section{Trägheit als emergente Informationsstruktur}
In der klassischen Physik ist Trägheit eine primitive Eigenschaft der Materie.\\In der Informations-Weber-Theorie entsteht Trägheit aus der Reaktion der
Informationsstruktur auf zeitliche Änderungen.
\begin{itemize}
    \item Eine homogene Informationsverteilung besitzt minimale interne Gradienten.
    \item Eine Beschleunigung verändert die Informationsstruktur.
    \item Diese Veränderung ist energetisch ungünstig.
\end{itemize}
Die resultierende Widerstandskraft ist die Trägheit. Sie ist keine ontologische Eigenschaft eines Körpers, sondern eine Konsequenz der lokalen Informationsdynamik.

\section{Gravitation als Informationsfluss}
Die Allgemeine Relativitätstheorie beschreibt Gravitation als Krümmung der Raumzeit. Die Informations-Weber-Theorie beschreibt Gravitation als \emph{Informationsfluss}.

Eine inhomogene Informationsverteilung erzeugt einen effektiven Informationsgradienten. Dieser führt zu einer gerichteten Umlagerung von Information, die im makroskopischen
Grenzfall als Newtonsche Gravitation erscheint.

Damit ist Gravitation keine geometrische Eigenschaft eines ontologischen Raumes, sondern eine Konsequenz der Informationskopplung, aus der der Raum erst emergiert.

\section{Wellenphänomene als energetische Informationsorganisation}
Wellenphänomene entstehen aus der Tendenz eines Systems, seine Informationsstruktur energetisch zu optimieren. Die Minimierung des globalen Informationsfunktionals führt zu
Interferenzmustern, die in der klassischen Physik als Wellenphänomene erscheinen.

Die Wahrscheinlichkeitsdichte eines quantenmechanischen Systems ist die Informationsdichte:
\[
    |\Psi|^2 = \rho_I.
\]
Die Interferenz zweier Informationsstrukturen ergibt:
\[
    \rho_I = \rho_1 + \rho_2 + 2\sqrt{\rho_1 \rho_2}\cos(\Delta \phi),
\]
wobei $\Delta \phi$ die relative Informationsphase ist.

Interferenz ist damit keine „Welle“, sondern eine energetisch optimale Informationsorganisation.

\section{Nichtlokalität als systemische Ganzheit}
Die Informations-Weber-Theorie besitzt zwei Kausalitätsebenen:
\begin{itemize}
    \item \textbf{lokale Kausalität}  
    beschreibt Energietransport und führt zur Weber-Kraft.

    \item \textbf{systemische Kausalität}  
    beschreibt die globale Organisation des Informationszustands und führt zum Bohm-Potential.
\end{itemize}
Die systemische Kausalität ist nicht durch Lichtgeschwindigkeit begrenzt, da sie keine Energie transportiert. Sie erzeugt die Nichtlokalität der Quantenmechanik, ohne die
Relativität zu verletzen.

\section{Zusammenführung der klassischen und quantenmechanischen Emergenz}
Die Informations-Weber-Theorie zeigt:
\begin{itemize}
    \item Trägheit entsteht aus lokalen Informationsänderungen.
    \item Gravitation entsteht aus Informationsgradienten.
    \item Wellenphänomene entstehen aus globaler Informationsoptimierung.
    \item Nichtlokalität entsteht aus systemischer Ganzheit.
\end{itemize}
Damit erscheinen klassische und quantenmechanische Phänomene als unterschiedliche Aspekte derselben fundamentalen Informationsdynamik.

\section{Emergenz der klassischen Mechanik}
Die klassische Mechanik entsteht als Grenzfall schwacher Informationsgradienten und dominanter lokaler Dynamik. In diesem Regime gilt:
\begin{itemize}
    \item Informationsflüsse sind lokal,
    \item globale Beiträge sind vernachlässigbar,
    \item die Dynamik ist durch lokale Projektionen der Informationsgeometrie bestimmt.
\end{itemize}
Die Newtonsche Bewegungsgleichung ist die makroskopische Projektion der lokalen Informationsstruktur.

\section{Emergenz der Quantenmechanik}
Die Quantenmechanik entsteht als Grenzfall starker globaler Informationsorganisation. In diesem Regime gilt:
\begin{itemize}
    \item globale Informationskopplungen dominieren,
    \item das Bohm-Potential bestimmt die Dynamik,
    \item Interferenz und Nichtlokalität sind natürliche Konsequenzen.
\end{itemize}
Die Schrödinger-Gleichung ist die makroskopische Projektion der globalen Informationsstruktur.

\section{Emergenz der Relativität}
Die Relativitätstheorie entsteht als Grenzfall der Informationsgeometrie:
\begin{itemize}
    \item Die SRT beschreibt Symmetrien des Informationsflusses.
    \item Die ART beschreibt die makroskopische Kontinuumsgeometrie großer Informationsnetze.
\end{itemize}
Die informationsbasierte Theorie geht über beide hinaus, da sie Raum und Zeit nicht postuliert, sondern emergieren lässt.

\section{Zusammenfassung}
Kapitel~\ref{chap:emergenz} hat gezeigt:
\begin{itemize}
    \item Klassische und quantenmechanische Phänomene sind emergente Eigenschaften der Informationsdynamik.
    \item Trägheit, Gravitation, Interferenz und Nichtlokalität entstehen aus lokalen und globalen Informationsstrukturen.
    \item Die klassische Mechanik, die Quantenmechanik und die Relativitätstheorie sind Grenzfälle derselben fundamentalen Informationsordnung.
\end{itemize}
Damit ist die physikalische Emergenzstruktur vollständig beschrieben. Kapitel~7 vergleicht diese emergenten Strukturen mit den etablierten Theorien.
