\chapter{Anwendungen und Beispiele}
\label{chap:anwendungen}

\section{Einleitung}
In diesem Kapitel werden konkrete physikalische Systeme aus der Perspektive der Informations-Weber-Theorie analysiert. Ziel ist es zu zeigen, wie klassische und
quantenmechanische Phänomene aus der Dynamik der Informationsdichte $\rho_I$ emergieren. Die Beispiele dienen nicht nur der Illustration, sondern demonstrieren
die Erklärungskraft der Theorie in Bereichen, die traditionell als grundlegend verschieden betrachtet werden: Mechanik, Gravitation, Quantenphysik und Plasmaphysik.

Jedes Beispiel folgt demselben Schema:

\begin{enumerate}
    \item Formulierung des physikalischen Systems,
    \item Darstellung der Informationsstruktur,
    \item Herleitung der Dynamik aus dem Informations-Lagrange-Funktional,
    \item Vergleich mit der etablierten Theorie,
    \item Interpretation der Ergebnisse.
\end{enumerate}

Die Beispiele sind so gewählt, dass sie sowohl klassische Grenzfälle als auch genuin quantenmechanische Phänomene abdecken.

\section{Der Doppelspalt: Interferenz als Informationsorganisation}
Das Doppelspaltexperiment gilt als eines der zentralen Phänomene der Quantenmechanik. In der Informations-Weber-Theorie entsteht das Interferenzmuster nicht durch eine
„Welle“, sondern durch die energetisch optimale Organisation der Informationsdichte.

\subsection{Informationsstruktur des Doppelspalts}
Die Informationsdichte hinter zwei Spalten ergibt sich aus der Überlagerung zweier Informationsquellen:
\[
    \rho_I = \rho_1 + \rho_2 + 2\sqrt{\rho_1 \rho_2}\cos(\Delta \phi).
\]
Die Phase $\Delta \phi$ ist eine Eigenschaft der globalen Informationsstruktur und ergibt sich aus der Minimierung des globalen Funktionals
\[
    \mathcal{F}_{\text{global}}
    =
    \gamma \frac{(\nabla \rho_I)^2}{\rho_I}.
\]

\subsection{Entstehung des Interferenzmusters}
Die Variation von $\mathcal{F}_{\text{global}}$ führt zu einer Gleichung, die das Interferenzmuster bestimmt:
\[
    \nabla^2 \sqrt{\rho_I}
    +
    k^2 \sqrt{\rho_I}
    = 0.
\]
Damit entsteht das Interferenzmuster als energetisch optimale Informationsverteilung.

\subsection{Vergleich zur Quantenmechanik}
Die Schrödinger-Gleichung liefert dasselbe Muster, jedoch mit einer ontologischen Wellenfunktion. Die Informations-Weber-Theorie zeigt, dass diese Wellenfunktion nicht
fundamental ist, sondern eine parametrische Darstellung der Informationsstruktur.

\section{Der harmonische Oszillator: Lokale vs. globale Dynamik}
Der harmonische Oszillator ist ein ideales Testsystem, um die Beziehung zwischen lokaler und globaler Informationsdynamik zu untersuchen.

\subsection{Informationspotential}
Das klassische Potential
\[
    V(x) = \frac{1}{2} m \omega^2 x^2
\]
entspricht einem Informationspotential
\[
    \Phi_I(x) = \kappa x^2,
\]
wobei $\kappa$ eine Konstante ist, die die Kopplungsstärke der Informationsstruktur beschreibt.

\subsection{Bewegungsgleichung}
Die Bewegungsgleichung ergibt sich aus
\[
    \frac{d}{dt} \left( \frac{\partial \mathcal{F}}{\partial (\partial_t \rho_I)} \right)
    =
    \frac{\partial \mathcal{F}}{\partial \rho_I}.
\]
Im Grenzfall lokaler Dynamik entsteht die klassische Oszillatorgleichung:
\[
    m \ddot{x} + m \omega^2 x = 0.
\]
Im globalen Grenzfall entsteht die quantisierte Energie:
\[
    E_n = \left(n + \frac{1}{2}\right)\hbar \omega.
\]

\subsection{Interpretation}
Die Quantisierung ist keine Eigenschaft des Potentials, sondern der globalen Informationsorganisation.

\section{Das Kepler-Problem: Gravitation als Informationsfluss}
Das Kepler-Problem beschreibt die Bewegung zweier Körper unter Newtonscher Gravitation. In der Informations-Weber-Theorie entsteht die Gravitationskraft aus
Informationsgradienten.

\subsection{Informationspotential}
Das Newton-Potential
\[
    V(r) = -\frac{GMm}{r}
\]
entspricht einem Informationspotential
\[
    \Phi_I(r)
    =
    \int \frac{\rho_I(\vec{r}')}{|\vec{r}-\vec{r}'|}\, d^3x'.
\]
Für punktförmige Informationsquellen ergibt sich
\[
    \Phi_I(r) \propto \frac{1}{r}.
\]

\subsection{Bewegungsgleichung}
Die Gravitationskraft ergibt sich aus
\[
    \vec{F}_{\text{grav}} = -\nabla \Phi_I.
\]
Damit entsteht die Kepler-Gleichung
\[
    \ddot{\vec{r}} = -\frac{GM}{r^3}\vec{r}.
\]

\subsection{Interpretation}
Gravitation ist ein Informationsfluss, nicht eine Eigenschaft der Raumzeit.

\section{Plasmawellen: Informationsgradienten in Medien}
Plasmen besitzen komplexe kollektive Dynamiken, die in der Informations-Weber-Theorie als Informationsflüsse in einem Medium beschrieben werden.

\subsection{Informationsdichte im Plasma}
Die Informationsdichte eines Plasmas ergibt sich aus der Ladungs- und Geschwindigkeitsverteilung:
\[
    \rho_I(\vec{r},t)
    =
    \rho_e(\vec{r},t)
    +
    \rho_i(\vec{r},t).
\]

\subsection{Plasmaoszillationen}
Die Variation des lokalen Funktionals führt zu einer Oszillationsgleichung
\[
    \partial_t^2 \rho_I + \omega_p^2 \rho_I = 0,
\]
wobei $\omega_p$ die Plasmafrequenz ist.

\subsection{Interpretation}
Plasmawellen sind kollektive Informationsflüsse, nicht elektromagnetische Felder.

\section{Zusammenfassung}
Die Beispiele dieses Kapitels zeigen, dass die Informations-Weber-Theorie in der Lage ist, klassische und quantenmechanische Phänomene einheitlich zu beschreiben:

\begin{itemize}
    \item Interferenz entsteht aus globaler Informationsorganisation.
    \item Der harmonische Oszillator zeigt die Dualität lokaler und globaler Dynamik.
    \item Gravitation ist ein Informationsfluss, nicht Raumzeitkrümmung.
    \item Plasmawellen sind kollektive Informationsgradienten.
\end{itemize}

Damit demonstriert die Theorie ihre Erklärungskraft über alle Skalen hinweg.
