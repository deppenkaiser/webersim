\chapter{Naturkonstanten aus Informationsarchitektur}
\label{chap:naturkonstanten}

Die Informations-Weber-Theorie beschreibt physikalische Systeme nicht durch Felder oder
Teilchen, sondern durch Informationsdichten und Informationsflüsse. In dieser Sichtweise
sind Naturkonstanten keine unabhängigen Eingabegrößen, sondern emergente Parameter der
Informationsarchitektur. Sie entstehen aus der fraktalen Struktur des Informationsnetzes,
aus Skalierungsrelationen und aus der Dynamik lokaler und globaler Informationsflüsse.

Dieses Kapitel zeigt, wie fundamentale Konstanten wie $c$, $\hbar$ und $G$ aus der
Informationsstruktur hervorgehen und warum sie in der Informations-Weber-Theorie nicht
fundamental sind.

\section{Einleitung: Warum Naturkonstanten nicht fundamental sind}

In klassischen Theorien erscheinen Naturkonstanten als unveränderliche Größen, die nicht
erklärt werden können. Die Informations-Weber-Theorie liefert eine alternative Sichtweise:

\begin{itemize}
    \item Naturkonstanten sind \emph{Skalierungsparameter} der Informationsgeometrie.
    \item Sie entstehen aus der Kopplungsstruktur des Informationsnetzes.
    \item Sie sind Konsequenzen der fraktalen Dimension
    

\[
        D = \frac{\ln 20}{\ln(2+\phi)}.
    \]


    \item Sie sind nicht fundamental, sondern emergent.
\end{itemize}

Damit wird die Frage nach dem Ursprung der Naturkonstanten zu einer Frage der
Informationsarchitektur.

\section{Die Lichtgeschwindigkeit $c$ als maximale Informationsflussrate}

In der Informations-Weber-Theorie ist die Lichtgeschwindigkeit keine ontologische Grenze,
sondern die maximale Geschwindigkeit, mit der lokale Informationsflüsse übertragen werden
können. Sie ergibt sich aus der Kopplungsdichte des Informationsnetzes.

\subsection{Informationsfluss und Kopplungsdichte}

Die lokale Informationsgeschwindigkeit ist definiert durch


\[
    \vec{J}_I = \rho_I \vec{v}_I.
\]


Die maximale Geschwindigkeit $\vec{v}_I^{\text{max}}$ ergibt sich aus der maximalen Rate,
mit der Kopplungen im Informationsnetz aktualisiert werden können.

In einem fraktalen Netz mit Dimension $D$ ergibt sich eine natürliche Skalierung:


\[
    c \propto \lambda^{D-1},
\]


wobei $\lambda$ die charakteristische Kopplungslänge ist.

\subsection{Interpretation}

\begin{itemize}
    \item $c$ ist die maximale Geschwindigkeit lokaler Informationsflüsse.
    \item $c$ ist keine fundamentale Konstante, sondern eine emergente Eigenschaft der
    Informationsarchitektur.
    \item In stark gekoppelten Regionen (z.\,B. Gravitation) kann die effektive
    Informationsgeschwindigkeit variieren.
\end{itemize}

Damit wird die Lichtgeschwindigkeit zu einer abgeleiteten Größe.

\section{Das Plancksche Wirkungsquantum $\hbar$ als Maß der Informationsgranularität}

Das Plancksche Wirkungsquantum $\hbar$ ist in der Informations-Weber-Theorie ein Maß für
die Granularität der globalen Informationsorganisation. Es entsteht aus der Struktur des
Bohm’schen Quantenpotentials.

\subsection{Informationsgranularität und globale Organisation}

Das Bohm-Potential


\[
    Q = -\frac{\hbar^2}{2m}
    \frac{\nabla^2 \sqrt{\rho_I}}{\sqrt{\rho_I}}
\]


beschreibt die systemische Organisation des Informationszustands. Die Größe $\hbar$
bestimmt die Stärke dieser globalen Kopplung.

In einem fraktalen Informationsnetz ergibt sich


\[
    \hbar \propto \lambda^{2-D},
\]


wobei $\lambda$ die charakteristische Netzskala ist.

\subsection{Interpretation}

\begin{itemize}
    \item $\hbar$ misst die Stärke globaler Informationsorganisation.
    \item $\hbar$ ist kein fundamentales Wirkungsquantum, sondern ein Skalierungsparameter.
    \item Die Quantenmechanik entsteht als Grenzfall globaler Informationsdynamik.
\end{itemize}

Damit wird die Quantenstruktur zu einer emergenten Eigenschaft des Informationsraums.

\section{Die Gravitationskonstante $G$ als Kopplungsparameter der Informationsgeometrie}

Die Gravitationskonstante $G$ entsteht aus der Kopplungsstruktur des Informationsnetzes,
insbesondere aus der fraktalen Geometrie der Masseverteilung.

\subsection{Weber-Gravitation und Informationskopplung}

Die Weber-Gravitationskraft hat die Form


\[
    F_{\text{WG}}
    =
    -G \frac{m_1 m_2}{r^2}
    \left(
        1 - \frac{\dot{r}^2}{2c^2} + \frac{r \ddot{r}}{c^2}
    \right).
\]



In der Informations-Weber-Theorie ist $G$ kein unabhängiger Parameter, sondern eine
Konsequenz der Kopplungsstärke des Informationsnetzes:


\[
    G \propto \lambda^{3-D}.
\]



\subsection{Interpretation}

\begin{itemize}
    \item $G$ misst die Stärke der geometrischen Kopplung im Informationsnetz.
    \item Gravitation ist eine emergente Eigenschaft der Informationsgeometrie.
    \item $G$ ist nicht fundamental, sondern skalenabhängig.
\end{itemize}

Damit wird die Gravitation zu einer Konsequenz der Informationsarchitektur.

\section{Weitere Naturkonstanten}

\subsection{Die Feinstrukturkonstante $\alpha$}
Die Feinstrukturkonstante
\[
    \alpha = \frac{e^2}{4\pi \varepsilon_0 \hbar c}
\]
ist in der Informations-Weber-Theorie ein Maß für die relative Stärke lokaler und globaler Informationsflüsse.

Da sowohl $c$ als auch $\hbar$ emergent sind, ist auch $\alpha$ eine emergente Größe:
\[
    \alpha \propto \lambda^{D-3}.
\]

\subsection{Die Elementarladung $e$}
Die Elementarladung ist ein Maß für die lokale Kopplungsstärke im Informationsnetz. Sie ergibt sich aus der minimalen Änderung der Informationsdichte, die eine lokale
Wechselwirkung erzeugen kann.

\subsection{Die Boltzmann-Konstante $k_B$}
Die Boltzmann-Konstante misst die Beziehung zwischen Informationsentropie und Energie. In der Informations-Weber-Theorie ist sie ein Maß für die Umrechnung zwischen lokaler
Informationsentropie und makroskopischer Energie.

\section{Zusammenfassung}
Die Informations-Weber-Theorie zeigt, dass Naturkonstanten keine fundamentalen Größen sind. Sie entstehen aus der fraktalen Struktur des Informationsnetzes und aus der Dynamik
lokaler und globaler Informationsflüsse. Die Lichtgeschwindigkeit $c$, das Wirkungsquantum $\hbar$, die Gravitationskonstante $G$ und weitere Konstanten sind Skalierungsparameter der
Informationsarchitektur. Damit wird die Physik zu einer Theorie der Information, in der Naturkonstanten nicht postuliert, sondern erklärt werden.
