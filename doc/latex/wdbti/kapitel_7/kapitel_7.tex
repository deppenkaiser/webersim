\chapter{Vergleich mit etablierten Theorien}
\label{chap:vergleich}

\section{Einleitung}
Die Informations-Weber-Theorie ist keine weitere konkurrierende Einzeltheorie, sondern eine \emph{Urtheorie}, aus der klassische Mechanik, Elektrodynamik, Quantenmechanik
und Relativitätstheorie als Grenzfälle hervorgehen. Dieses Kapitel vergleicht die emergenten Strukturen der Informationsdynamik mit den etablierten physikalischen Theorien
und zeigt, wie diese als Näherungen einer tieferen Informationsordnung erscheinen.

Der Vergleich erfolgt entlang der fundamentalen Fragen:
\begin{itemize}
    \item Was ist der physikalische Zustand?
    \item Was ist die Dynamik?
    \item Was ist Raum und Zeit?
    \item Was ist Kausalität?
    \item Welche Größen sind fundamental?
\end{itemize}
Die Antworten der etablierten Theorien werden den Antworten der Informations-Weber-Theorie gegenübergestellt.

\section{Klassische Mechanik als lokaler Grenzfall}
Die klassische Mechanik basiert auf:
\begin{itemize}
    \item Punktteilchen,
    \item Kräften,
    \item einer absoluten Zeit,
    \item einem euklidischen Raum.
\end{itemize}
In der Informations-Weber-Theorie entsteht die klassische Mechanik als Grenzfall:
\begin{itemize}
    \item schwacher Informationsgradienten,
    \item dominanter lokaler Dynamik,
    \item vernachlässigbarer globaler Informationskopplung.
\end{itemize}
Die Newtonsche Bewegungsgleichung ist die makroskopische Projektion der lokalen Informationsstruktur. Trägheit, Impuls und Energie sind abgeleitete Informationsfunktionale.

\section{Elektrodynamik: Maxwell, Lorentz und Weber}
Die klassische Elektrodynamik existiert in drei Formen:
\begin{enumerate}
    \item \textbf{Maxwell-Theorie (MT)}: Felder als ontologische Objekte.
    \item \textbf{Lorentz-Kraft}: phänomenologische Kraftformel.
    \item \textbf{Weber-Elektrodynamik (WED)}: direkte Wechselwirkung ohne Felder.
\end{enumerate}

\subsection{Maxwell-Theorie als effektive Feldbeschreibung}
Die Maxwell-Theorie beschreibt elektromagnetische Phänomene durch Felder im Raum. In der Informations-Weber-Theorie erscheinen diese Felder als:
\begin{itemize}
    \item effektive Kontinuumsnäherungen,
    \item makroskopische Projektionen lokaler Informationsflüsse,
    \item nicht fundamentale Objekte.
\end{itemize}

\subsection{Lorentz-Kraft als phänomenologische Näherung}
Die Lorentz-Kraft ist eine Näherung der Weber-Kraft für:
\begin{itemize}
    \item kleine Geschwindigkeiten,
    \item stationäre Ströme,
    \item schwache Beschleunigungen.
\end{itemize}
Sie ist kein fundamentales Gesetz, sondern eine Vereinfachung.

\subsection{Weber-Kraft als lokaler Grenzfall}
Die Weber-Kraft ist der lokale Grenzfall der Informationsdynamik:
\[
    F_{\text{lokal}} = F_{\text{WED}}.
\]
Sie entsteht aus dem lokalen Anteil des Informations-Lagrange-Funktionals und benötigt keine Felder als ontologische Objekte.

\section{Quantenmechanik als globale Informationsdynamik}
Die Quantenmechanik basiert auf:
\begin{itemize}
    \item Wellenfunktionen,
    \item Superposition,
    \item Interferenz,
    \item Nichtlokalität.
\end{itemize}
In der Informations-Weber-Theorie entstehen diese Phänomene aus:
\begin{itemize}
    \item globaler Informationsorganisation,
    \item Minimierung des globalen Informationsfunktionals,
    \item dem Bohm’schen Quantenpotential als globalem Informationsoperator.
\end{itemize}
Die Schrödinger-Gleichung ist die makroskopische Projektion der globalen Informationsstruktur.

\section{Relativitätstheorie als emergente Geometrie}
Die Relativitätstheorie basiert auf:
\begin{itemize}
    \item einer glatten Raumzeit,
    \item einer fundamentalen Lichtgeschwindigkeit,
    \item geodätischer Bewegung.
\end{itemize}
In der Informations-Weber-Theorie entstehen diese Strukturen aus der Informationsmetrik:
\begin{itemize}
    \item Die SRT beschreibt Symmetrien des Informationsflusses.
    \item Die ART beschreibt die makroskopische Kontinuumsgeometrie großer Informationsnetze.
    \item Die Lichtgeschwindigkeit ist keine fundamentale Konstante, sondern eine emergente
    maximale Informationsflussrate.
\end{itemize}

\section{Grenzfälle und Übergänge}
Die Informations-Weber-Theorie reproduziert die etablierten Theorien in folgenden Grenzfällen:
\begin{itemize}
    \item \textbf{klassische Mechanik:}  
    lokale Dynamik, schwache Gradienten.

    \item \textbf{Weber-Elektrodynamik:}  
    lokale Informationsflüsse.

    \item \textbf{Maxwell-Theorie:}  
    effektive Kontinuumsnäherung lokaler Flüsse.

    \item \textbf{Quantenmechanik:}  
    globale Informationsorganisation.

    \item \textbf{SRT:}  
    Symmetrie des Informationsflusses.

    \item \textbf{ART:}  
    makroskopische Informationsgeometrie.
\end{itemize}
Damit ist jede etablierte Theorie ein Spezialfall der informationsbasierten Struktur.

\section{Frequenzabhängige Lichtablenkung als Test der Theorie}
Ein zentraler Unterschied zwischen ART und Informations-Weber-Theorie ist die Vorhersage der Lichtablenkung:
\begin{itemize}
    \item Die ART sagt eine frequenzunabhängige Ablenkung voraus.
    \item Die Informations-Weber-Theorie sagt eine frequenzabhängige Ablenkung voraus.
\end{itemize}
Dies ist ein experimentell testbares Unterscheidungsmerkmal.

\section{Von WDBT → ART → ART+ → WDBT+}
Die Informations-Weber-Theorie ordnet die etablierten Theorien logisch ein:
\begin{enumerate}
    \item \textbf{WDBT (analog):}  
    Fernwirkung ohne Raum.

    \item \textbf{ART:}  
    geometrische Erweiterung der Weber-Struktur.

    \item \textbf{ART+:}  
    ART ergänzt um globale Informationsstruktur (kein Kollaps, keine Singularitäten).

    \item \textbf{WDBT+:}  
    vollständige informationsbasierte Urtheorie mit diskretem Informationsnetz.
\end{enumerate}

\section{Zusammenfassung}
Kapitel~\ref{chap:vergleich} hat gezeigt:
\begin{itemize}
    \item Die Informations-Weber-Theorie ist eine Urtheorie, aus der alle etablierten Theorien
    als Grenzfälle hervorgehen.
    \item Klassische Mechanik, Elektrodynamik, Quantenmechanik und Relativitätstheorie sind
    emergente Näherungen der Informationsdynamik.
    \item Die Theorie macht klare, testbare Vorhersagen, die über die etablierten Modelle
    hinausgehen.
\end{itemize}
Damit ist der Vergleich abgeschlossen. Kapitel~8 entwickelt die Konsequenzen der informationsbasierten Struktur für Naturkonstanten.
