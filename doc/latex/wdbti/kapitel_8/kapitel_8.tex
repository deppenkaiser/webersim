\chapter{Naturkonstanten aus Informationsarchitektur}
\label{chap:naturkonstanten}

\section{Einleitung: Naturkonstanten als emergente Skalierungsparameter}
In der klassischen Physik erscheinen Naturkonstanten wie $c$, $\hbar$ oder $G$ als fundamentale, unveränderliche Größen. Sie werden nicht erklärt, sondern als gegebene
Parameter vorausgesetzt. Die Informations-Weber-Theorie bricht mit dieser Sichtweise.

In ihr sind Naturkonstanten keine ontologischen Eingabegrößen, sondern \emph{emergente Skalierungsparameter} der Informationsarchitektur. Sie entstehen aus:
\begin{itemize}
    \item der fraktalen Dimension des Informationsnetzes,
    \item der Kopplungsstruktur lokaler und globaler Informationsflüsse,
    \item der Skalierung der Informationsmetriken,
    \item der Granularität globaler Organisation.
\end{itemize}
Damit wird die Frage nach dem Ursprung der Naturkonstanten zu einer Frage der Informationsgeometrie.

\section{Fraktale Informationsarchitektur als Ursprung der Skalen}
Die fraktale Dimension des Informationsnetzes lautet:
\[
    D = \frac{\ln 20}{\ln(2+\phi)}.
\]
Sie bestimmt die Skalierung aller emergenten Größen. Die charakteristische Kopplungslänge $\lambda$ des Informationsnetzes erzeugt natürliche Skalierungsrelationen:
\[
    X \propto \lambda^{\alpha(D)}.
\]
Die Exponenten $\alpha(D)$ ergeben sich aus der Struktur des Informations-Lagrange-Funktionals und der Informationsmetrik.

\section{Die Lichtgeschwindigkeit als maximale Informationsflussrate}
In der Informations-Weber-Theorie ist die Lichtgeschwindigkeit keine fundamentale Konstante, sondern die maximale Geschwindigkeit lokaler Informationsflüsse.

\subsection{Informationsfluss und Kopplungsdichte}
Die lokale Informationsgeschwindigkeit ergibt sich aus:
\[
    \vec{J}_I = \rho_I \vec{v}_I.
\]
Die maximale Geschwindigkeit $\vec{v}_I^{\text{max}}$ ist durch die maximale Aktualisierungsrate der Kopplungen begrenzt. In einem fraktalen Netz gilt:
\[
    c \propto \lambda^{D-1}.
\]

\subsection{Interpretation}
\begin{itemize}
    \item $c$ ist die maximale Geschwindigkeit lokaler Informationsflüsse.
    \item $c$ ist keine fundamentale Konstante, sondern ein emergenter Skalierungsparameter.
    \item In stark gekoppelten Regionen kann die effektive Informationsgeschwindigkeit variieren.
\end{itemize}
Damit wird die Lichtgeschwindigkeit zu einer abgeleiteten Größe der Informationsarchitektur.

\section{Das Plancksche Wirkungsquantum als Maß globaler Informationsgranularität}
Das Plancksche Wirkungsquantum $\hbar$ misst die Granularität globaler Informationsorganisation. Es entsteht aus der Struktur des Bohm’schen Quantenpotentials:
\[
    Q = -\frac{\hbar^2}{2m}
    \frac{\nabla^2 \sqrt{\rho_I}}{\sqrt{\rho_I}}.
\]

\subsection{Skalierung aus fraktaler Geometrie}
In einem fraktalen Informationsnetz ergibt sich:
\[
    \hbar \propto \lambda^{2-D}.
\]

\subsection{Interpretation}
\begin{itemize}
    \item $\hbar$ misst die Stärke globaler Informationskopplung.
    \item $\hbar$ ist kein fundamentales Wirkungsquantum, sondern ein Skalierungsparameter.
    \item Die \gls{qm} entsteht als globaler Grenzfall der Informationsdynamik.
\end{itemize}
Damit wird die Quantenstruktur zu einer emergenten Eigenschaft des Informationsraums.

\section{Die Gravitationskonstante als Kopplungsparameter der Informationsgeometrie}
Die Gravitationskonstante $G$ misst die Stärke der geometrischen Kopplung im Informationsnetz. Sie entsteht aus der fraktalen Struktur der Informationsmetriken.

\subsection{Skalierung aus Informationskopplung}
Die allgemeine Form der Weber-Gravitation (siehe Anhang~\ref{app:webergrav}) lautet:
\[
    F_{\text{WG}}
    =
    -G \frac{m_1 m_2}{r^2}
    \left(
        1 - \frac{\dot{r}^2}{c^2} + \beta\,\frac{r \ddot{r}}{c^2}
    \right),
\]
wobei der Parameter $\beta$ je nach physikalischer Situation die Werte
$\beta = 0.5$ (massive Körper), $\beta = 1$ (Photonen) oder $\beta = 2$ (elektromagnetische Wechselwirkung) annimmt.

In der Informations-Weber-Theorie ist $G$ kein unabhängiger Parameter, sondern ergibt sich aus der Skalierung der fraktalen Informationsarchitektur:
\[
    G \propto \lambda^{3-D}.
\]

\subsection{Interpretation}
\begin{itemize}
    \item $G$ misst die Stärke der emergenten Informationsgeometrie.
    \item Gravitation ist eine Konsequenz der fraktalen Kopplungsstruktur des Informationsnetzes.
    \item $G$ ist skalenabhängig und nicht fundamental; seine beobachtete Konstanz ist ein makroskopischer Grenzfall.
    \item Die Beziehung $G \propto \lambda^{3-D}$ spiegelt wider, wie die Gravitationskopplung mit der charakteristischen Längenskala $\lambda$ des Informationsnetzes skaliert.
\end{itemize}

\section{Weitere Naturkonstanten}
\subsection{Die Feinstrukturkonstante}
Die Feinstrukturkonstante
\[
    \alpha = \frac{e^2}{4\pi \varepsilon_0 \hbar c}
\]
misst das Verhältnis lokaler und globaler Informationsflüsse.  
Da sowohl $c$ als auch $\hbar$ emergent sind, gilt:
\[
    \alpha \propto \lambda^{D-3}.
\]

\subsection{Die Elementarladung}
Die Elementarladung $e$ ist ein Maß für die minimale lokale Kopplungsstärke im Informationsnetz. Sie entsteht aus der kleinsten Änderung der Informationsdichte, die eine
lokale Wechselwirkung erzeugen kann.

\subsection{Die Boltzmann-Konstante}
Die Boltzmann-Konstante $k_B$ ist ein Skalierungsparameter zwischen:
\begin{itemize}
    \item lokaler Informationsentropie,
    \item makroskopischer Energie.
\end{itemize}
Sie ist keine fundamentale Größe, sondern eine Umrechnungsrelation zwischen zwei Informationsmaßen.

\section{Zusammenfassung}
Kapitel~\ref{chap:naturkonstanten} hat gezeigt:
\begin{itemize}
    \item Naturkonstanten sind keine fundamentalen Größen.
    \item Sie entstehen aus der fraktalen Informationsarchitektur.
    \item $c$ ist die maximale lokale Informationsflussrate.
    \item $\hbar$ misst die Granularität globaler Informationsorganisation.
    \item $G$ misst die Kopplungsstärke der Informationsgeometrie.
    \item $\alpha$, $e$ und $k_B$ sind Skalierungsparameter lokaler und globaler Informationsflüsse.
\end{itemize}
Damit liefert die Informations-Weber-Theorie eine einheitliche, reduktionistische Erklärung der Naturkonstanten als emergente Eigenschaften der Informationsstruktur.
