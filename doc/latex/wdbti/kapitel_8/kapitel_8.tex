\chapter{Experimentelle Vorhersagen und Tests}
\label{chap:vorhersagen}

Eine fundamentale Theorie muss nicht nur konsistent und widerspruchsfrei sein, sondern auch überprüfbare Vorhersagen machen. Die Informations-Weber-Theorie erfüllt dieses
Kriterium in besonderem Maße: Sie liefert klare, quantitative und qualitative Aussagen, die sich von den Vorhersagen der Allgemeinen Relativitätstheorie (ART), der
Quantenmechanik (QM) und der Standardkosmologie unterscheiden. Dieses Kapitel fasst die wichtigsten experimentellen Konsequenzen zusammen und zeigt, wie die Theorie empirisch
getestet werden kann.

\section{Einleitung: Testbarkeit einer Informations-Urtheorie}
Die Informations-Weber-Theorie ist keine spekulative Erweiterung bestehender Modelle, sondern eine Urtheorie, aus der klassische und quantenmechanische Phänomene als
Grenzfälle hervorgehen. Ihre Vorhersagen betreffen:

\begin{itemize}
    \item kosmologische Beobachtungen,
    \item gravitative Effekte,
    \item quantenmechanische Phänomene,
    \item Plasmaprozesse,
    \item die Struktur der Naturkonstanten.
\end{itemize}

Viele dieser Vorhersagen unterscheiden sich deutlich von ART, QFT und Standardkosmologie und ermöglichen daher klare experimentelle Tests.

\section{Vorhersagen, die der ART widersprechen}
\subsection{Keine echten Singularitäten}
Die Informations-Weber-Theorie postuliert eine minimale Informationsdichte
\[
    \rho_I^{\text{min}} > 0,
\]
wodurch echte Singularitäten ausgeschlossen sind. Dies führt zu folgenden Vorhersagen:

\begin{itemize}
    \item Schwarze Löcher besitzen einen informationsbasierten Kern statt einer Singularität.
    \item Die Raumzeitkrümmung bleibt endlich.
    \item Der Urknall wird durch einen Big Bounce ersetzt.
\end{itemize}

\subsection{Abweichungen bei extremen Gravitationsfeldern}
In Bereichen hoher Kopplungsdichte (z.\,B. nahe kompakter Objekte) ergeben sich Abweichungen von der ART:

\begin{itemize}
    \item modifizierte Lichtablenkung,
    \item veränderte Gravitationsrotverschiebung,
    \item Abweichungen in der Bahnpräzession.
\end{itemize}

Diese Effekte sind messbar, sobald die Informationsgeometrie von der effektiven Kontinuumsgeometrie der ART abweicht.

\section{Vorhersagen, die der Quantenfeldtheorie widersprechen}
\subsection{Keine virtuellen Teilchen}

Die Informations-Weber-Theorie benötigt keine virtuellen Photonen oder Feldquanten. Stattdessen entstehen Wechselwirkungen durch Informationsflüsse. Daraus folgt:

\begin{itemize}
    \item keine divergenten Selbstenergien,
    \item keine Renormierung als fundamentales Prinzip,
    \item keine überlichtschnellen Pfadintegral-Komponenten.
\end{itemize}

\subsection{Nichtlokalität ohne Widerspruch zur Kausalität}
Das Bohm’sche Quantenpotential beschreibt globale Informationsorganisation. Die Theorie sagt daher:

\begin{itemize}
    \item EPR-Korrelationen sind Ausdruck systemischer Ganzheit,
    \item keine Signale werden überlichtschnell übertragen,
    \item die Kausalität bleibt auf lokaler Ebene erhalten.
\end{itemize}

\section{Kosmologische Tests}
\subsection{CMB-Fraktalität}
Die Informations-Weber-Theorie sagt voraus, dass die CMB-Anisotropien fraktale Korrelationen aufweisen, die aus der fraktalen Dimension
\[
    D = \frac{\ln 20}{\ln(2+\phi)}
\]
resultieren. Messbare Konsequenzen:

\begin{itemize}
    \item Abweichungen von rein statistisch-gaussianischen Fluktuationen,
    \item fraktale Korrelationslängen,
    \item keine akustischen Peaks im klassischen Sinn.
\end{itemize}

\subsection{Rotverschiebung ohne Expansion}
Die Theorie sagt voraus, dass Rotverschiebungen durch Informationsumstrukturierung entstehen. Testbare Konsequenzen:

\begin{itemize}
    \item Rotverschiebung ist nicht streng proportional zur Entfernung,
    \item Abweichungen bei sehr hohen Rotverschiebungen,
    \item mögliche Abhängigkeit von Plasma- und Informationsdichte.
\end{itemize}

\subsection{Galaktische Rotationskurven ohne Dunkle Materie}
Die fraktale Informationsgeometrie erzeugt effektive zusätzliche Beschleunigungen. Vorhersagen:

\begin{itemize}
    \item flache Rotationskurven ohne Dunkle Materie,
    \item Tully-Fisher-Relation als Informationsgesetz,
    \item Abweichungen in Zwerggalaxien und Low-Surface-Brightness-Galaxien.
\end{itemize}

\section{Labor- und Plasma-Experimente}
\subsection{Weber-Effekte in Laborplasmen}

Die geschwindigkeits- und beschleunigungsabhängigen Terme der Weber-Kraft führen zu messbaren Effekten:

\begin{itemize}
    \item anisotrope Transportprozesse,
    \item nichtlineare Plasmaoszillationen,
    \item Abweichungen von Maxwell-basierten Modellen.
\end{itemize}

\subsection{Informationsflüsse in turbulenten Plasmen}
Die Theorie sagt voraus:

\begin{itemize}
    \item fraktale Skalenhierarchien,
    \item selbstorganisierte Filamentstrukturen,
    \item Abweichungen von klassischer MHD.
\end{itemize}

\section{Zusammenfassung}
Die Informations-Weber-Theorie macht eine Vielzahl klarer, überprüfbarer Vorhersagen, die sich von ART, QFT und Standardkosmologie unterscheiden. Besonders relevant sind:

\begin{itemize}
    \item keine Singularitäten,
    \item Big Bounce statt Big Bang,
    \item fraktale CMB-Struktur,
    \item Rotverschiebung ohne Expansion,
    \item Rotationskurven ohne Dunkle Materie,
    \item Abweichungen in Laborplasmen,
    \item keine virtuellen Teilchen.
\end{itemize}

Diese Vorhersagen machen die Informations-Weber-Theorie zu einer empirisch testbaren Urtheorie, die klassische und quantenmechanische Phänomene in einem einheitlichen
informationsbasierten Rahmen beschreibt.
