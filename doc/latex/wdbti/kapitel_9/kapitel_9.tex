\chapter{Experimentelle Vorhersagen und Tests}
\label{chap:tests}

\paragraph{Hinweis zur mathematischen Darstellung}
Dieses Kapitel verwendet größtenteils die \emph{kontinuierliche Notation} für Kompaktheit. Die zugrundeliegende fundamentale Formulierung ist diskret rekursiv. Wo nötig
wird die diskrete Form explizit angegeben. Eine vollständige diskrete Darstellung findet sich in Kapitel X.

\section{Einleitung: Testbarkeit einer informationsbasierten Urtheorie}
Eine fundamentale Theorie muss nicht nur konzeptionell konsistent sein, sondern auch \emph{experimentell überprüfbare Vorhersagen} machen. Die Informations-Weber-Theorie
erfüllt dieses Kriterium in besonderem Maße: Sie liefert klare, quantitative und qualitative Vorhersagen, die sich von denen der \gls{art}, der \gls{qft} und der
Standardkosmologie unterscheiden.

Die Testbarkeit ergibt sich aus drei Ebenen:
\begin{enumerate}
    \item \textbf{lokale Dynamik}  
    (Weber-Kraft, Informationsflüsse, Plasmaeffekte),

    \item \textbf{globale Informationsorganisation}
    (Nichtlokalität, Quantenstruktur, Bohm-Potential),

    \item \textbf{Informationsgeometrie}  
    (fraktale Raumstruktur, emergente Metrik, Naturkonstanten).
\end{enumerate}
Diese drei Ebenen erzeugen experimentelle Signaturen, die in etablierten Theorien nicht auftreten.

\section{Vorhersagen, die der ART widersprechen}
\subsection{Keine echten Singularitäten}
Die Informations-Weber-Theorie postuliert eine minimale Informationsdichte:
\[
    \rho_I^{\text{min}} > 0,
\]
wodurch echte Singularitäten ausgeschlossen sind. Dies führt zu folgenden Vorhersagen:
\begin{itemize}
    \item Schwarze Löcher besitzen einen informationsbasierten Kern statt einer Singularität.
    \item Die effektive Krümmung bleibt endlich.
    \item Der Urknall wird durch einen Big Bounce ersetzt.
\end{itemize}

\subsection{Abweichungen bei extremen Gravitationsfeldern}
In Bereichen hoher Kopplungsdichte ergeben sich Abweichungen von der \gls{art}:
\begin{itemize}
    \item modifizierte Lichtablenkung,
    \item veränderte Gravitationsrotverschiebung,
    \item Abweichungen in der Bahnpräzession.
\end{itemize}
Diese Effekte treten auf, sobald die informationsbasierte Geometrie von der makroskopischen Kontinuumsgeometrie der \gls{art} abweicht.

\subsection{Frequenzabhängige Lichtablenkung}
Die \gls{art} sagt eine frequenzunabhängige Ablenkung voraus:
\[
    \delta\theta_{\text{ART}} = \frac{4GM}{c^2 b}.
\]
Die Informations-Weber-Theorie sagt dagegen:
\[
    \delta\theta(\nu)
    =
    \delta\theta_0
    \left(
        1 + \alpha \frac{\nu_0}{\nu}
    \right),
\]
wobei hochfrequente Photonen \emph{weniger} abgelenkt werden als niederfrequente.

Messmethoden:
\begin{itemize}
    \item spektral aufgelöste Sonnenrandmessungen,
    \item Gravitationslinsen im optischen, Röntgen- und Radiobereich,
    \item Pulsar-Timing und Fast Radio Bursts.
\end{itemize}

\section{Vorhersagen, die der Quantenfeldtheorie widersprechen}
\subsection{Keine virtuellen Teilchen}
Die Informations-Weber-Theorie benötigt keine virtuellen Photonen oder Feldquanten. Wechselwirkungen entstehen durch Informationsflüsse. Daraus folgt:
\begin{itemize}
    \item keine divergenten Selbstenergien,
    \item keine Renormierung als fundamentales Prinzip,
    \item keine überlichtschnellen Pfadintegral-Komponenten.
\end{itemize}

\subsection{Nichtlokalität ohne Verletzung der Kausalität}
Das Bohm-Potential beschreibt globale Informationsorganisation. Die Theorie sagt:
\begin{itemize}
    \item EPR-Korrelationen sind Ausdruck systemischer Ganzheit,
    \item keine Signale werden überlichtschnell übertragen,
    \item lokale Kausalität bleibt erhalten.
\end{itemize}

\section{Kosmologische Tests}
\subsection{CMB-Fraktalität}
Die Informations-Weber-Theorie sagt voraus, dass die CMB-Anisotropien fraktale Korrelationen aufweisen, die aus der fraktalen Dimension
\[
    D = \frac{\ln 20}{\ln(2+\phi)}
\]
resultieren.

Messbare Konsequenzen:
\begin{itemize}
    \item Abweichungen von rein gaussianischen Fluktuationen,
    \item fraktale Korrelationslängen,
    \item modifizierte akustische Strukturen.
\end{itemize}

\subsection{Rotverschiebung ohne Expansion}
Die Theorie sagt voraus, dass Rotverschiebungen durch Informationsumstrukturierung entstehen.

Testbare Konsequenzen:
\begin{itemize}
    \item Rotverschiebung ist nicht streng proportional zur Entfernung,
    \item Abweichungen bei sehr hohen Rotverschiebungen,
    \item mögliche Abhängigkeit von Plasma- und Informationsdichte.
\end{itemize}

\subsection{Galaktische Rotationskurven ohne Dunkle Materie}
Die fraktale Informationsgeometrie erzeugt effektive zusätzliche Beschleunigungen.

Vorhersagen:
\begin{itemize}
    \item flache Rotationskurven ohne Dunkle Materie,
    \item Tully-Fisher-Relation als Informationsgesetz,
    \item Abweichungen in Zwerggalaxien und Low-Surface-Brightness-Galaxien.
\end{itemize}

\section{Labor- und Plasma-Experimente}
\subsection{Weber-Effekte in Laborplasmen}
Die geschwindigkeits- und beschleunigungsabhängigen Terme der Weber-Kraft führen zu messbaren Effekten:
\begin{itemize}
    \item anisotrope Transportprozesse,
    \item nichtlineare Plasmaoszillationen,
    \item Abweichungen von Maxwell-basierten Modellen.
\end{itemize}

\subsection{Informationsflüsse in turbulenten Plasmen}
Die Theorie sagt voraus:
\begin{itemize}
    \item fraktale Skalenhierarchien,
    \item selbstorganisierte Filamentstrukturen,
    \item Abweichungen von klassischer MHD.
\end{itemize}

\section{Zusammenfassung}
Kapitel~\ref{chap:tests} hat gezeigt:
\begin{itemize}
    \item Die Informations-Weber-Theorie ist experimentell testbar.
    \item Sie macht klare Vorhersagen, die der \gls{art}, \gls{qft} und Standardkosmologie widersprechen.
    \item Sie erklärt kosmologische Phänomene ohne Dunkle Materie und ohne Expansion.
    \item Sie liefert neue Tests in Plasmaphysik, Lensing und CMB-Analyse.
    \item Sie benötigt keine virtuellen Teilchen und keine Renormierung.
\end{itemize}
Damit ist die Theorie nicht nur konzeptionell und analytisch, sondern auch empirisch überprüfbar.
