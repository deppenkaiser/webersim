% Datei: kapitel_1_1.tex
% Teil I: Fundamentale diskrete Theorie
% Kapitel 1: Einleitung

\chapter{Einleitung}
\label{chap:einleitung}

\section{Zur erkenntnistheoretischen Stellung dieser Theorie}
\label{sec:erkenntnistheorie}

Die Informations-Weber-Theorie (IWT) versteht sich als \textbf{fundamentale diskrete Urtheorie}, die physikalische Realität aus der Dynamik eines Informationsnetzes ableitet. Ihre erkenntnistheoretische Besonderheit liegt in der expliziten Trennung zwischen fundamentaler diskreter Ebene und emergenter kontinuierlicher Beschreibung.

\subsection{Empirische Anker und iterative Selbstkonsistenz}
\label{subsec:empirische-anker}

Die IWT folgt einem \textbf{iterativen, rekursiven Wissenschaftsmodell}:

\begin{enumerate}
    \item \textbf{Empirische Anker}: Ausgangspunkt sind messbare Größen wie die fraktale Dimension \( D \approx 2.71 \) oder die erfolgreichen Vorhersagen der
	Weber-de-Broglie-Bohm-Theorie (WDBT).
    \item \textbf{Diskrete Einbettung}: Diese Größen werden in das diskrete formale Rahmenwerk der IWT eingebettet, bestehend aus:
    \begin{itemize}
        \item Informationsfeld \( I_k^{(n)} \) (diskretes Skalarfeld)
        \item Informationskopplungen \( K_{kl}^{(n)} \)
        \item Rekursive Update-Regeln
    \end{itemize}
    \item \textbf{Vorhersagen}: Aus diesem Rahmen folgen neue, testbare Vorhersagen für Naturkonstanten, kosmische Hintergrundtemperatur und Lichtablenkung.
    \item \textbf{Vergleich}: Vergleich mit empirischer Realität.
    \item \textbf{Iterative Verfeinerung}: Bei Übereinstimmung zeigt sich interne Konsistenz; bei Abweichung wird der Rahmen rekursiv angepasst.
\end{enumerate}

Dieser Prozess strebt einer \textbf{Konvergenz} zu: Die Theorie nähert sich schrittweise einer vollständigen Beschreibung, ohne alle Parameter a priori herleiten zu müssen.

\subsection{Die fraktale Dimension D als diskrete Skalenstruktur}
\label{subsec:beispiel-D}

Die fraktale Dimension
\[
D = \frac{\ln 20}{\ln(2+\phi)} \approx 2.71
\]
wird in dieser Arbeit als gegebene Größe verwendet, deren vollständige Herleitung aus den IWT-Axiomen ein offenes Forschungsprogramm bleibt. Sie charakterisiert die
\textbf{Skalenstruktur des fundamentalen Informationsnetzes} und ist damit keine Raumdimension, sondern ein Maß für die Verzweigungskomplexität der Informationskopplungen.

Dies entspricht dem üblichen Vorgehen in der theoretischen Physik:
\begin{itemize}
    \item Die ART übernimmt \( G \) aus der Newton'schen Theorie.
    \item Die QFT verwendet \( \hbar \) und \( c \) als fundamentale Parameter.
    \item Der Fortschritt der IWT liegt darin, aus \emph{einem} empirischen Anker (\( D \)) eine Vielzahl weiterer Konstanten und kosmologischer Beziehungen abzuleiten.
\end{itemize}

\subsection{Abgrenzung gegen Zirkularitätsvorwürfe}
\label{subsec:zirkularitaet}

Die IWT arbeitet \textbf{kohärent}, nicht zirkulär:
\begin{enumerate}
    \item \( D \) wird als Input verwendet (aus WDBT oder direkter Beobachtung fraktaler Strukturen).
    \item Die IWT leitet daraus die Gleichgewichtstemperatur des kosmischen Mikrowellenhintergrunds ab:
    \[
    T_{\mathrm{CMB}} = \left( \frac{\bar{\alpha}(L) u_\gamma}{\varepsilon A_{\mathrm{eff}} \sigma} \right)^{1/4},
    \]
    wobei \( \bar{\alpha}(L) \) selbst von \( D \) abhängt.
    \item Diese Vorhersage wird mit \( T_{\mathrm{CMB}} \approx 2.7\,\mathrm{K} \) verglichen.
    \item Die Übereinstimmung bestätigt die \emph{Konsistenz} des Rahmens, ist aber kein Beweis für \( D \).
\end{enumerate}

Damit verbindet die IWT verschiedene empirische Domänen (Fraktalgeometrie, Plasmaphysik, Kosmologie) in einem einheitlichen diskreten Rahmen.

\subsection{Offene Herleitungen als Forschungsprogramm}
\label{subsec:forschungsprogramm}

Die vollständige Herleitung aller Parameter – insbesondere von \( D \) – ausschließlich aus den IWT-Axiomen bleibt ein \textbf{offenes Forschungsprogramm}. Dies schmälert nicht den Wert der hier vorgelegten Theorie, denn:

\begin{itemize}
    \item Die Vorhersagen sind unabhängig von dieser Herleitung testbar.
    \item Die Theorie bietet einen geschlossenen \textbf{diskreten mathematischen Rahmen}, der solche Herleitungen prinzipiell ermöglicht.
    \item Jede fundamentale Theorie beginnt mit einigen Postulaten oder empirischen Inputs.
\end{itemize}

Die IWT versteht sich als \textbf{arbeitende Urtheorie im Aufbau}, die zur Überprüfung und Erweiterung einlädt.

\subsection{Vergleich mit etablierten Paradigmen}
\label{subsec:vergleich}

Die methodische Legitimität dieses Vorgehens wird durch historische Beispiele gestützt:
\begin{enumerate}
    \item \textbf{Newton's Mechanik} begann mit Keplerschen Gesetzen.
    \item \textbf{Maxwell's Elektrodynamik} übernahm die Lichtgeschwindigkeit aus Messungen.
    \item \textbf{Einstein's ART} verwendete das Äquivalenzprinzip als empirischen Ausgangspunkt.
\end{enumerate}

In allen Fällen wurden empirische Anker in einen theoretischen Rahmen eingebettet – genau dieses Vorgehen wählt auch die IWT.

\section{Information als fundamentale Größe der diskreten Physik}
\label{sec:information-als-fundamental}

Die zentrale These der IWT lautet: \textbf{Information ist die grundlegende physikalische Größe, aus der Energie, Raum, Zeit und Dynamik emergieren.}

In der diskreten Formulierung bedeutet dies:
\begin{itemize}
    \item Der physikalische Zustand wird durch eine diskrete Informationsverteilung \( I_k^{(n)} \) beschrieben.
    \item Energie ist ein abgeleitetes Funktional \( E[I^{(n)}] \).
    \item Raum entsteht als emergente Metrik aus der Kopplungsstruktur \( K_{kl}^{(n)} \).
    \item Zeit ist die Ordnungsstruktur der Update-Sequenz \( n = 0,1,2,\dots \).
\end{itemize}

Die Weber-de-Broglie-Bohm-Theorie (WDBT) bietet einen natürlichen Zugang zu dieser Sichtweise, da sie:
\begin{itemize}
    \item direkte Wechselwirkungen (Weber) in diskreter Form beschreibt,
    \item wellenartige Informationsfelder (De Broglie) als globale Struktur einführt,
    \item nichtlokale Organisationsprinzipien (Bohm) als systemische Kausalität versteht.
\end{itemize}

Die IWT hebt diesen Gedanken auf eine fundamentale Ebene: Sie interpretiert alle physikalischen Größen als Informationsfunktionale und zeigt, dass
\textbf{Informationserhaltung} die eigentliche Grundlage der Energieerhaltung ist.

\section{Motivation: Warum eine diskrete fundamentale Theorie?}
\label{sec:motivation}

Die moderne Physik steht vor grundlegenden Problemen, die auf die kontinuierliche mathematische Formulierung zurückgeführt werden können:

\subsection{Widersprüche zwischen SRT und ART}
\begin{itemize}
    \item Die SRT basiert auf idealisierten Inertialsystemen, die in einer gekrümmten Raumzeit nicht existieren.
    \item Die Lichtgeschwindigkeit ist in der SRT absolut, in der ART lokal variabel.
    \item Zitat: \enquote{Einstein's postulates contain inherent contradictions when applied to real gravitational systems} \cite{Rubcic1998}.
\end{itemize}

\subsection{Ungelöste Probleme der Quantenmechanik}
\begin{itemize}
    \item Welle-Teilchen-Dualismus als konzeptionelle Schwäche.
    \item Kollaps der Wellenfunktion bei Messungen.
    \item Nichtlokale Verschränkung.
    \item Virtuelle Teilchen mit Überlichtgeschwindigkeit in der QED.
    \item Zitat: \enquote{The observer-dependent collapse is not a fundamental feature of nature} \cite{bohm1952}.
\end{itemize}

\subsection{Dogmatismus und blinde Flecken}
\begin{itemize}
    \item Etablierte Theorien werden kaum hinterfragt, obwohl sie fundamentale Schwächen aufweisen (Singularitäten, unendliche Selbstenergien, dunkle Entitäten).
    \item Alternative Ansätze werden systematisch ausgegrenzt.
    \item Zitat: \enquote{Theoretical physics has become stuck in a paradigm that values mathematical elegance over empirical testability} \cite{Smolin2006}.
\end{itemize}

\subsection{Die Notwendigkeit eines diskreten Ansatzes}
Viele dieser Probleme entstehen aus der \textbf{unnötigen Kontinuitätsannahme}:
\begin{itemize}
    \item Kontinuumsmodelle erzeugen Singularitäten.
    \item Differentialgleichungen verbergen die rekursive Natur der Dynamik.
    \item Die Weber-Kraft zeigt: Physik kann \textbf{direkt und diskret} formuliert werden.
\end{itemize}

\section{Die diskrete Natur der Weber-Kraft}
\label{sec:diskrete-weber-kraft}

Die Weber-Kraft in ihrer fundamentalen diskreten Form lautet für zwei wechselwirkende Informationsknoten:
\[
\vec{F}^{(n)} = \frac{q_1 q_2}{4\pi\varepsilon_0 (r^{(n)})^2} 
\left[
1 - \frac{1}{c^2} \left( \frac{r^{(n)} - r^{(n-1)}}{T} \right)^2 
+ \frac{2r^{(n)}}{c^2} \cdot \frac{r^{(n+1)} - 2r^{(n)} + r^{(n-1)}}{T^2}
\right] \hat{\vec{r}}^{(n)}
\]
Diese Form ist:
\begin{enumerate}
    \item \textbf{Explizit rekursiv}: \( r^{(n+1)} \) wird aus \( r^{(n)} \) und \( r^{(n-1)} \) berechnet.
    \item \textbf{Nicht-zirkulär}: Im Gegensatz zur kontinuierlichen Form \( F = f(\ddot{r}) \) mit \( \ddot{r} = F/m \).
    \item \textbf{Stabil}: Numerisch gut konditioniert als diskreter IIR-Filter.
    \item \textbf{Feldlos}: Benötigt keine elektromagnetischen Felder als ontologische Objekte.
\end{enumerate}
Die kontinuierliche Notation
\[
\vec{F} = \frac{q_1 q_2}{4\pi\varepsilon_0 r^2} 
\left( 1 - \frac{\dot{r}^2}{c^2} + 2\frac{r\ddot{r}}{c^2} \right) \hat{\vec{r}}
\]
ist nur eine \textbf{emergente Näherung} für \( T \to 0 \).

\section{Axiome der Informations-Weber-Theorie (diskrete Form)}
\label{sec:axiome}

\subsection*{Axiom I: Diskreter Informationszustand}
Jedes physikalische System wird durch eine diskrete Informationsverteilung \( I_k^{(n)} \) beschrieben. Energie, Impuls und Ladung sind abgeleitete Funktionale.

\subsection*{Axiom II: Diskrete Informationserhaltung}
Die Zeitentwicklung ist eine invertierbare Transformation:
\[
I_k^{(n+1)} = \mathcal{T}[I_k^{(n)}, \{I_l^{(n)}\}, I_k^{(n-1)}]
\]
Die Gesamtinformation \( \sum_k I_k^{(n)} \) ist erhalten.

\subsection*{Axiom III: Diskrete Dynamik als Informationsfluss}
Bewegung entsteht aus der Umlagerung von Information. Lokale Dynamik: Weber-Kraft (rekursiv). Globale Dynamik: Bohm-Potential (systemisch).

\subsection*{Axiom IV: Raum als emergente diskrete Geometrie}
Raum ist keine Grundgröße. Die physikalische Metrik emergiert aus der Kopplungsmatrix \( K_{kl}^{(n)} \):
\[
g_{ij}^{(n)} = \mathcal{G}[K_{kl}^{(n)}]
\]
Die fraktale Dimension \( D \) charakterisiert die Skalenstruktur von \( K_{kl}^{(n)} \).

\subsection*{Axiom V: Zwei-Ebenen-Kausalität}
\begin{itemize}
    \item \textbf{Lokale Kausalität}: Energietransport mit maximaler Geschwindigkeit \( c \) (Weber-Dynamik).
    \item \textbf{Systemische Kausalität}: Instantane globale Organisation (Bohm-Potential).
\end{itemize}
