\chapter{Axiome der \gls{iwt}}

\section{Einleitung}
Die \gls{iwt} ist eine fundamentale Urtheorie, in der Raum, Zeit, Energie, Dynamik und Naturkonstanten als emergente Größen aus der Struktur und Dynamik eines diskreten
universellen Informationsfeldes hervorgehen. Dieser Abschnitt formuliert die Theorie in axiomatischer Form. Alle späteren Ergebnisse – klassische Mechanik, Quantenmechanik,
Gravitation, Kosmologie und Naturkonstanten – folgen aus diesen Axiomen.

\section{Axiom 1: Existenz eines diskreten universellen Informationsfeldes}
Es existiert ein skalares diskretes Informationsfeld
\[
I_k^{(n)},
\]
das die vollständige physikalische Realität beschreibt. Hierbei ist:
- \(k = 1, \ldots, N\): Index der diskreten Informationsknoten
- \(n \in \mathbb{N}_0\): Diskreter Zeitindex (Update-Schritt)
- \(I_k^{(n)} \in \mathbb{R}^+\): Informationswert (skalar, dimensionslos)

Materie, Energie, Wellen, Geometrie und Dynamik sind Manifestationen der Struktur und Veränderung dieses Feldes.

Weber war der erste, der eine feldlose, rein interaktionistische Physik formuliert hat. Axiom 1 behauptet genau das: keine Felder, sondern direkte Informationskopplung.
Das ist historisch exakt Webers Ansatz. \cite{Weber1846}

\section{Axiom 2: Informationsmetrik als emergente diskrete Geometrie}
Die physikalische Raumzeit ist nicht fundamental. Stattdessen entsteht eine effektive diskrete Metrik
\[
g_{kl}^{(n)}
\]
aus der lokalen und globalen Struktur des Informationsfeldes und seiner Kopplungsmatrix \(K_{kl}^{(n)}\). Die Metrik ist dynamisch und wird nicht vorgegeben, sondern durch
die diskrete Informationsdynamik bestimmt:
\[
g_{kl}^{(n)} = \mathcal{G}[I_k^{(n)}, K_{kl}^{(n)}].
\]

\section{Axiom 3: Variationsprinzip der diskreten Informationsdynamik}
Die Dynamik des Informationsfeldes folgt aus einem universellen diskreten Informations-Lagrange-Funktional
\[
\mathcal{L}_d[I_k^{(n)}] 
= 
\mathcal{L}_{\mathrm{lokal}}
+
\mathcal{L}_{\mathrm{global}}
+
\mathcal{L}_{\mathrm{fraktal}},
\]
mit den drei fundamentalen Beiträgen:

\subsection{Lokaler Anteil (Weber-Struktur)}
\[
\mathcal{L}_{\mathrm{lokal}}
=
\frac{1}{2} \sum_{k,l} K_{kl}^{(n)} \Delta_{kl} I^{(n)} \Delta_{kl} I^{(n)}.
\]
Er beschreibt direkte Informationsflüsse zwischen benachbarten Knoten. Tisserand \cite{tisserand1894} zeigt, dass die Weber-Kraft aus einem Variationsprinzip abgeleitet
werden kann.

\subsection{Globaler Anteil (Bohm-Struktur)}
\[
\mathcal{L}_{\mathrm{global}}
=
-\frac{\lambda}{2}\sum_k \frac{\Delta^2 I_k^{(n)}}{I_k^{(n)}}.
\]
Er beschreibt nichtlokale Organisationsstrukturen über das gesamte Netzwerk.

\subsection{Fraktaler Anteil (Kosmische Skalierung)}
\[
\mathcal{L}_{\mathrm{fraktal}}
=
\mu \ln\!\left(1+\gamma_{\mathrm{eff}} G \rho_{\mathrm{eff}} L^2\right) \sum_k I_k^{(n)}.
\]
Er beschreibt die skaleninvariante Struktur des Universums.

\section{Axiom 4: Dynamische Gleichung der Informationsmetrik}
Die Metrik entsteht aus der Variation des diskreten Funktionals nach \(g_{kl}^{(n)}\). Die fundamentale Gleichung der Informationsmetrik in diskreter Form lautet:
\[
\boxed{
g_{kl}^{(n+1)} = g_{kl}^{(n)} + T \cdot \left[
\Delta_{k} I^{(n)} \Delta_{l} I^{(n)}
-
\lambda\,\frac{\Delta_{kl}^2 I^{(n)}}{I_{kl}^{(n)}}
+
\mu\,g_{kl}^{(n)}\,\ln\!\left(1+\gamma_{\mathrm{eff}} G \rho_{\mathrm{eff}} L^2\right)
\right]
}
\]
Sie vereint:
\begin{itemize}
    \item \textbf{Lokale Weber-Dynamik}: Direkte Informationsflüsse zwischen Knoten
    \item \textbf{Globale Bohm-Struktur}: Nichtlokale Organisation des Gesamtnetzwerks
    \item \textbf{Fraktale kosmische Skalierung}: Skaleninvariante Struktur des Universums
\end{itemize}

\section{Axiom 5: Energieerhaltung als diskreter Informationsfluss}
Energie ist keine fundamentale Größe. Sie entsteht als Erhaltungsgröße des diskreten Informationsflusses. Die diskrete Energieerhaltung folgt aus der Zeitinvarianz des
Informations-Lagrange-Funktionals:
\[
E^{(n+1)} = E^{(n)} \quad \text{für alle } n,
\]
mit der diskreten Energie:
\[
E^{(n)} = \sum_k \left[ \frac{\partial \mathcal{L}_d}{\partial (\Delta_t I_k^{(n)})} \Delta_t I_k^{(n)} - \mathcal{L}_d \right].
\]

\section{Axiom 6: Naturkonstanten als emergente Skalierungsparameter}
Die fundamentalen Naturkonstanten sind keine Eingaben der Theorie, sondern feste Punkte der diskreten Informationsdynamik:
\begin{itemize}
    \item \(c\): Maximale Informationsflussrate (\(c = \lambda_{\max}/T\))
    \item \(\hbar\): Globale Informationsgranularität (\(\hbar = \alpha \Delta I_{\min} \lambda_0^2\))
    \item \(G\): Fraktale Kopplungsstärke (\(G = \beta \lambda_0^{3-D}/f_{\max}^2\))
    \item \(\alpha\): Verhältnis lokaler zu globaler Kopplung
    \item \(k_B\): Informations-Temperatur-Skala
\end{itemize}

\section{Axiom 7: Fraktale Skalierungsinvarianz des Universums}
Die großskalige Struktur des Universums ist fraktal mit effektiver Dimension
\[
D = \frac{\ln 20}{\ln(2+\phi)} \approx 2.71.
\]
Diese fraktale Struktur bestimmt:
\begin{itemize}
    \item Die kosmische Rotverschiebung: \(z(d) = \gamma_{\mathrm{eff}} G \rho_{\mathrm{eff}} d^2\)
    \item Die Verlustkonstante: \(\bar{\alpha}(L) = \frac{1}{L\gamma_{\mathrm{eff}} G \rho_{\mathrm{eff}}} \ln(1+\gamma_{\mathrm{eff}} G \rho_{\mathrm{eff}} L^2)\)
    \item Die CMB-Gleichgewichtstemperatur: \(T_{\mathrm{CMB}} = \left( \frac{\bar{\alpha}(L) u_\gamma}{\varepsilon A_{\mathrm{eff}} \sigma} \right)^{1/4}\)
\end{itemize}

\section{Axiom 8: Emergenz von Raum, Zeit und Dynamik}
Raum, Zeit und Dynamik sind emergente Eigenschaften der Informationsmetrik:

\subsection{Emergenz des Raumes}
Der physikalische Raum entsteht aus der diskreten Metrik:
\[
d_{kl}^{(n)} = \sqrt{g_{kl}^{(n)}} \cdot \lambda_0,
\]
mit fundamentaler Länge \(\lambda_0\).

\subsection{Emergenz der Zeit}
Die physikalische Zeit entsteht als Ordnungsstruktur der Informationsänderung:
\[
t \approx n \cdot T,
\]
wobei \(n\) der Update-Index und \(T\) der fundamentale Zeitschritt ist. Lokale Zeitdilatation entsteht durch unterschiedliche Update-Frequenzen:
\[
T_k = \frac{T}{\sqrt{1 - v_k^2/c^2}}.
\]

\subsection{Emergenz der Dynamik}
Klassische Mechanik, \gls{qm} und Gravitation sind Grenzfälle der diskreten Informationsdynamik:
\begin{itemize}
    \item \textbf{Klassischer Grenzfall}: Schwache Gradienten, dominante lokale Dynamik
    \item \textbf{Quantenmechanischer Grenzfall}: Starke globale Kopplung, Phasenkohärenz
    \item \textbf{Relativitätsgrenzfall}: Geometrie aus großem diskreten Netz
\end{itemize}

\section{Axiom 9: Fraktale Informationsdimension}
Der universelle Informationsraum besitzt eine fundamentale, skaleninvariante fraktale Dimension
\[
D = \frac{\ln(20)}{\ln(2+\phi)},
\]
wobei $\phi$ die Goldene Zahl bezeichnet. Diese Dimension ist eine primäre Eigenschaft des Informationsfeldes und bestimmt die Skalierung der Informationsmetrik, die
Kopplung lokaler und globaler Informationsflüsse, die Form der Weber-Kraft im makroskopischen Grenzfall, die Struktur des Bohm-Potentials sowie die emergenten
Naturkonstanten. Die fraktale Dimension $D$ ist damit ein fundamentaler Parameter der IWT und bildet die topologische Grundlage des Informationsuniversums.

Amelino-Camelia \cite{AmelinoCamelia2013} untersucht fraktale, skalenabhängige Raumzeitstrukturen in der Quantum-Spacetime-Phenomenology.

\section{Axiom 10: Universelle Gültigkeit}
Die \gls{iwt} gilt auf allen Skalen:
\begin{itemize}
    \item \textbf{Mikroskopisch}: Quantenstruktur, Elementarteilchen, Wellenphänomene
    \item \textbf{Mesoskopisch}: Klassische Mechanik, Elektrodynamik, Thermodynamik
    \item \textbf{Makroskopisch}: Gravitation, Astrophysik, Planetenbewegung
    \item \textbf{Kosmologisch}: Rotverschiebung, CMB, großskalige Struktur
\end{itemize}

\section{Zusammenfassung}
Diese zehn Axiome definieren die \gls{iwt} vollständig in ihrer fundamentalen diskreten Formulierung. Alle physikalischen Phänomene – von der Quantenmechanik über die
Gravitation bis zur Kosmologie – folgen aus der Struktur und Dynamik des diskreten Informationsfeldes und der daraus emergierenden Metrik. 

Die \gls{iwt} erfüllt damit die Kriterien einer konsistenten, geschlossenen und vollständig emergenten Urtheorie:
\begin{itemize}
    \item \textbf{Fundamental diskret}: Keine Kontinuitätsannahmen, rekursive Update-Regeln
    \item \textbf{Emergent kontinuierlich}: Kontinuierliche Theorien erscheinen als Grenzfälle
    \item \textbf{Vereinheitlicht}: Alle physikalischen Wechselwirkungen aus einem Prinzip
    \item \textbf{Testbar}: Spezifische Vorhersagen abweichend von Standardtheorien
    \item \textbf{Paradigmenwechsel}: Information als fundamentale Substanz, nicht Energie oder Masse
\end{itemize}
