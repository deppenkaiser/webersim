% Datei: kapitel_1_2.tex
% Teil I: Fundamentale diskrete Theorie
% Kapitel 2: Die Informations-Weber-Theorie

\chapter{Die Informations-Weber-Theorie}
\label{chap:informationstheorie}

\section{Der diskrete Informationszustand}
Die \gls{iwt} geht von der grundlegenden Annahme aus, dass jeder physikalische Zustand durch eine \emph{diskrete Informationsverteilung} beschrieben wird.
Diese wird durch eine skalare Dichtesequenz
\[
I_k^{(n)}
\]
repräsentiert, die angibt, wie viel strukturierte Information am Netzwerkknoten \( k \) zum Zeitschritt \( n \) vorliegt.

Hierbei ist:
\begin{itemize}
    \item \( k \in \{1,2,\dots,N\} \): Index der diskreten Informationszelle (Knoten)
    \item \( n \in \mathbb{N}_0 \): Diskreter Zeitindex (Zeitschritt)
    \item \( I_k^{(n)} \in \mathbb{R}^+ \): Informationswert (skalar, dimensionslos)
\end{itemize}
Im Gegensatz zu klassischen Feldern besitzt \( I_k^{(n)} \) keine materielle Bedeutung. Sie beschreibt weder Masse noch Ladung oder Energie, sondern die \emph{Organisation}
eines physikalischen Systems. Energie, Impuls und andere Größen entstehen erst als abgeleitete Funktionale dieser Informationsstruktur.

\subsection{Diskrete Informationserhaltung}
Analog zur Kontinuitätsgleichung der klassischen Physik wird der Informationsfluss durch diskrete Differenzengleichungen beschrieben. Die fundamentale Erhaltungsgleichung im
diskreten Netz lautet:
\[
\sum_{k \in \mathcal{N}(l)} \left( I_k^{(n+1)} - I_k^{(n)} \right) + \sum_{m \in \partial\mathcal{N}(l)} J_{lm}^{(n)} = 0
\]
mit:
\begin{itemize}
    \item \( \mathcal{N}(l) \): Menge der Nachbarknoten von \( l \)
    \item \( \partial\mathcal{N}(l) \): Rand des Nachbarschaftsbereichs
    \item \( J_{lm}^{(n)} \): Informationsfluss von Knoten \( l \) zu \( m \) in Zeitschritt \( n \)
\end{itemize}
Diese Gleichung ist das diskrete Herzstück der Theorie: Sie ersetzt die kontinuierliche Energieerhaltung durch eine \emph{diskrete Informationserhaltung}. Die gesamte
Dynamik ergibt sich aus der rekursiven Umlagerung von Information zwischen Netzwerkknoten.

\section{Information als Ursprung physikalischer Größen (diskret)}
In der diskreten Informations-Weber-Theorie entstehen physikalische Größen als Funktionale der Informationsverteilung \( I_k^{(n)} \).

\subsection{Diskrete Energie}
Die Energie am Knoten \( k \) zum Zeitpunkt \( n \) ist:
\[
E_k^{(n)} = \alpha \left( I_k^{(n)} - I_k^{(n-1)} \right)^2 + \beta \sum_{l \in \mathcal{N}(k)} \left( I_k^{(n)} - I_l^{(n)} \right)^2
\]
mit Kopplungskonstanten \( \alpha, \beta \).

\subsection{Diskreter Impuls}
Der Impulsfluss zwischen Knoten \( k \) und \( l \) ist:
\[
p_{kl}^{(n)} = \gamma \left( I_k^{(n)} - I_k^{(n-1)} \right) \left( I_l^{(n)} - I_l^{(n-1)} \right)
\]

\subsection{Diskrete Masse (Trägheit)}
Die effektive Masse am Knoten \( k \) ist:
\[
m_k^{(n)} = \delta \sum_{l \in \mathcal{N}(k)} \left( I_k^{(n)} - I_l^{(n)} \right)^2
\]
Sie misst den Widerstand gegen Änderungen der lokalen Informationsstruktur.

Damit wird die klassische Unterscheidung zwischen Materie, Feldern und Geometrie aufgehoben: Alles entsteht aus einer einzigen fundamentalen diskreten Größe – der
Information.

\section{Dynamik als diskreter Informationsfluss}
Die Bewegungsgleichungen eines Systems ergeben sich aus der rekursiven Umlagerung von Information. Die Theorie unterscheidet zwei komplementäre diskrete Dynamikformen:
\begin{itemize}
    \item \textbf{Lokale diskrete Dynamik}: beschrieben durch die diskrete Weber-Kraft
    \item \textbf{Globale diskrete Dynamik}: beschrieben durch das diskrete Bohm-Potential
\end{itemize}

Diese beiden Strukturen sind keine konkurrierenden Modelle, sondern zwei Projektionen derselben diskreten Informationsdynamik.

\subsection{Lokale diskrete Dynamik: Diskrete Weber-Kraft}
Die diskrete Weber-Kraft beschreibt lokale Informationsflüsse zwischen benachbarten Knoten. Für zwei wechselwirkende Knotengruppen \( A \) und \( B \) lautet sie:
\[
\vec{F}_{AB}^{(n)} = \frac{q_A q_B}{4\pi\varepsilon_0 (r_{AB}^{(n)})^2}
\left[
1 - \frac{1}{c^2} \left( \frac{r_{AB}^{(n)} - r_{AB}^{(n-1)}}{T} \right)^2 
+ \frac{2r_{AB}^{(n)}}{c^2} \cdot \frac{r_{AB}^{(n+1)} - 2r_{AB}^{(n)} + r_{AB}^{(n-1)}}{T^2}
\right] \hat{\vec{r}}_{AB}^{(n)}
\]

Eigenschaften:
\begin{enumerate}
    \item \textbf{Explizit rekursiv}: Berechnet \( r^{(n+1)} \) aus \( r^{(n)} \) und \( r^{(n-1)} \)
    \item \textbf{Nicht-zirkulär}: Keine implizite Abhängigkeit \( F = f(\ddot{r}) \) mit \( \ddot{r} = F/m \)
    \item \textbf{Lokal}: Nur Nachbarkopplungen
    \item \textbf{Feldlos}: Benötigt keine kontinuierlichen Felder
\end{enumerate}

\subsection{Globale diskrete Dynamik: Diskrete Bohm-Struktur}
Das diskrete Bohm-Potential beschreibt die systemische, nichtlokale Organisation des Informationszustands. Im diskreten Netz lautet es:
\[
Q_k^{(n)} = -\frac{\hbar^2}{2m} \frac{\Delta_d^2 \sqrt{I_k^{(n)}}}{\sqrt{I_k^{(n)}}}
\]
mit dem diskreten Laplace-Operator:
\[
\Delta_d^2 f_k = \sum_{l \in \mathcal{N}(k)} (f_l - f_k)
\]

Eigenschaften:
\begin{enumerate}
    \item \textbf{Global}: Wirkt über das gesamte Netz
    \item \textbf{Instantan}: Keine Retardierung
    \item \textbf{Nicht-energetisch}: Transportiert keine Energie
    \item \textbf{Organisierend}: Optimiert die Gesamtstruktur
\end{enumerate}

\subsection{Die digitale WDBT als fundamentales Netzwerk}
Die digitale \gls{wdbt} beschreibt die Gesamtwirkung auf einen Knoten \( k \) durch drei diskrete Beiträge:
\[
F_k^{(n)} = F_{\text{WED},k}^{(n)} + F_{\text{WG},k}^{(n)} + F_{Q,k}^{(n)}
\]
\begin{itemize}
    \item \( F_{\text{WED},k}^{(n)} \): Diskrete \gls{wed} (Ladungen)
    \item \( F_{\text{WG},k}^{(n)} \): Diskrete \gls{wg} (Massen)
    \item \( F_{Q,k}^{(n)} \): Diskrete Bohm-Struktur (globale Organisation)
\end{itemize}
Diese digitale Theorie besitzt \emph{kein vorgegebenes Raummodell}. Der Raum emergiert aus der Kopplungsstruktur \( K_{kl}^{(n)} \).

\section{Raum als emergente diskrete Informationsgeometrie}
Die analoge \gls{wdbt} arbeitet ohne ontologischen Raum. Die digitale \gls{wdbt} führt ein diskretes Informationsnetz ein, aus dem der physikalische Raum als emergente
diskrete Geometrie entsteht.

\subsection{Warum Raum nicht fundamental sein kann (diskret)}
Mehrere diskrete Argumente sprechen gegen einen fundamentalen Raum:
\begin{enumerate}
    \item \textbf{Fernwirkungen benötigen keinen Trägerraum}: Die Weber-Kraft wirkt direkt zwischen Knoten.
    \item \textbf{Diskrete Kausalität}: Ursache-Wirkung lässt sich über Update-Regeln definieren.
    \item \textbf{Fraktale Dimension}: \( D \approx 2.71 \) widerspricht einem glatten 3D-Kontinuum.
    \item \textbf{Singularitätenfreiheit}: Diskrete Systeme haben keine echten Singularitäten.
    \item \textbf{Dynamik vor Geometrie}: Die Metrik \( g_{ij}^{(n)} \) wird berechnet, nicht postuliert.
\end{enumerate}
Die Konsequenz: Raum ist eine abgeleitete diskrete Größe, keine fundamentale.

\subsection{Emergenz der diskreten Zeit}
Auch die Zeit ist keine primitive Größe. Sie entsteht aus:
\begin{itemize}
    \item \textbf{Update-Ordnung}: Die Sequenz \( n = 0,1,2,\dots \) definiert die Zeitrichtung.
    \item \textbf{Lokale Takte}: Jeder Knoten hat eine eigene Update-Frequenz \( f_k \).
    \item \textbf{Zeitdilatation}: Unterschiedliche \( f_k \) erzeugen effektive Zeitdehnung.
    \item \textbf{Entropie-Richtung}: Die Zunahme der Informationsentropie definiert den Zeitpfeil.
\end{itemize}
Die physikalische Zeit \( t \) ist eine kontinuierliche Näherung:
\[
t \approx n \cdot T_{\text{avg}}
\]
mit mittlerer Update-Periode \( T_{\text{avg}} \).

\subsection{Fraktale Dimension als diskrete Skalenstruktur}
Die fraktale Dimension
\[
D = \frac{\ln 20}{\ln(2+\phi)} \approx 2.71
\]
ist eine Eigenschaft der Kopplungsmatrix \( K_{kl}^{(n)} \). Sie beschreibt, wie die Anzahl der effektiven Nachbarn mit der Skala \( s \) skaliert:
\[
N(s) \sim s^D
\]

\subsection{Die diskrete Informationsstruktur als Ursprung des Raumes}
Die digitale \gls{wdbt} beschreibt ein Netzwerk aus:
\begin{itemize}
    \item \textbf{Knoten}: \( k = 1,\dots,N \) mit Informationswerten \( I_k^{(n)} \)
    \item \textbf{Kopplungen}: \( K_{kl}^{(n)} \) (Adjazenzmatrix mit Gewichten)
    \item \textbf{Update-Regeln}: \( I_k^{(n+1)} = \mathcal{U}(I_k^{(n)}, \{I_l^{(n)}\}, K_{kl}^{(n)}) \)
\end{itemize}
Die effektive diskrete Metrik zwischen Knoten \( k \) und \( l \) ist:
\[
g_{kl}^{(n)} = \frac{\partial^2 \mathcal{F}}{\partial I_k^{(n)} \partial I_l^{(n)}}
\]
wobei \( \mathcal{F} \) das diskrete Informationsfunktional ist.

\subsection{Emergenz der Dynamik aus der diskreten Informationsgeometrie}
Wenn Raum und Zeit emergent sind, dann ist auch die Dynamik emergent:
\begin{itemize}
    \item \textbf{Lokale Dynamik}: Projektion der lokalen Kopplungsstruktur \( K_{kl}^{(n)} \)
    \item \textbf{Globale Dynamik}: Projektion der Eigenvektoren von \( K_{kl}^{(n)} \)
    \item \textbf{Wellen}: Kollektive Moden der Kopplungsmatrix
\end{itemize}

\subsection{Emergenz von Gravitationswellen im diskreten Netz}
Die analoge \gls{wdbt} kann keine Gravitationswellen beschreiben. Die digitale \gls{wdbt} erzeugt Gravitationswellen als kollektive Schwingungsmoden der Kopplungsmatrix:
\[
K_{kl}^{(n+1)} = K_{kl}^{(n)} + \epsilon \cdot \text{Mode}_{kl}^{(n)}
\]

Diese Moden:
\begin{enumerate}
    \item Sind \textbf{dispersiv} (Frequenzabhängigkeit)
    \item Transportieren \textbf{Information}, aber keine Energie im klassischen Sinn
    \item Entstehen aus \textbf{nichtlokalen Korrelationen}
    \item Haben \textbf{endliche Ausbreitungsgeschwindigkeit} \( v_g \leq c \)
\end{enumerate}

LIGO \cite{LIGO2016} liefert die experimentelle Bestätigung, dass solche Moden real sind. LIGO 2023 \cite{LIGO2023} untersucht sogar frequenzabhängige Dispersion.

\subsection{CMB-Struktur als fossilierte diskrete Informationsgeometrie}
Die anisotrope Struktur der \gls{cmb} spiegelt die fraktale Kopplungsstruktur \( K_{kl}^{(n_0)} \) zu einem frühen Zeitpunkt \( n_0 \) wider. Die Temperaturfluktuationen
sind:
\[
\frac{\Delta T}{T}(\theta,\phi) \propto \sum_{k,l} K_{kl}^{(n_0)} \cdot Y_{lm}(\theta,\phi)
\]
\gls{cmb} als stationäres thermisches Gleichgewicht. Lerner \cite{Lerner2018} und Arp \cite{Arp1998} sind die einzigen, die nicht-expansive \gls{cmb}-Interpretationen
vertreten

\subsection{Herleitung von Naturkonstanten aus diskreter Skalierung}
In der digitalen \gls{wdbt} entstehen Naturkonstanten aus Skalierungsrelationen:
\begin{align*}
c &\sim \lambda \cdot f_{\text{max}} \quad &\text{(maximale Informationsflussrate)} \\
\hbar &\sim \Delta I_{\text{min}} \cdot \lambda^2 \quad &\text{(globale Granularität)} \\
G &\sim \frac{\lambda^{3-D}}{f_{\text{max}}^2} \quad &\text{(Kopplungsstärke)}
\end{align*}
mit charakteristischer Länge \( \lambda \) und maximaler Update-Frequenz \( f_{\text{max}} \).

\section{Einordnung: Diskrete vs. kontinuierliche Theorien}
\label{sec:einordnung}

\begin{table}[ht]
\centering
\begin{tabular}{p{0.25\textwidth}|p{0.3\textwidth}|p{0.35\textwidth}}
\textbf{Theorie} & \textbf{Raumkonzept} & \textbf{Dynamik} \\
\hline
\textbf{Analoge \gls{wdbt}} & Kein Raummodell & Direkte Fernwirkung, keine Wellen \\
\textbf{\gls{art}} & Glattes Kontinuum \( g_{\mu\nu}(x) \) & Geometrische Krümmung, Singularitäten \\
\textbf{Digitale \gls{wdbt}} & Emergente diskrete Metrik \( g_{kl}^{(n)} \) & Rekursive Update-Regeln, keine Singularitäten \\
\end{tabular}
\caption{Vergleich der Raumkonzepte}
\end{table}

\section{Zusammenfassung}
Kapitel~2 hat die konzeptionellen Grundlagen der Informations-Weber-Theorie in ihrer \textbf{fundamentalen diskreten Formulierung} dargestellt:

\begin{itemize}
    \item \textbf{Diskreter Informationszustand}: \( I_k^{(n)} \) als Grundgröße
    \item \textbf{Diskrete Erhaltung}: Summe der Information erhalten
    \item \textbf{Zweistufige Dynamik}: 
    \begin{itemize}
        \item Lokal: Diskrete Weber-Kraft (rekursiv)
        \item Global: Diskrete Bohm-Struktur (organisierend)
    \end{itemize}
    \item \textbf{Emergenter Raum}: Metrik \( g_{kl}^{(n)} \) aus Kopplungsmatrix \( K_{kl}^{(n)} \)
    \item \textbf{Emergente Zeit}: Update-Sequenz \( n = 0,1,2,\dots \)
    \item \textbf{Fraktale Skalierung}: \( D \approx 2.71 \) charakterisiert Netzwerkstruktur
    \item \textbf{Naturkonstanten}: Entstehen aus Skalierungsrelationen des Netzes
\end{itemize}

Die mathematische Formulierung der diskreten Dynamik erfolgt in Kapitel~3 (Diskrete Weber-Dynamik) und Kapitel~4 (Diskretes Informationsfunktional). Die emergente
kontinuierliche Physik wird in Teil~II behandelt.
