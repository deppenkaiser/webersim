% Datei: kapitel_1_3.tex
% Teil I: Fundamentale diskrete Theorie
% Kapitel 3: Die diskrete Weber-Elektrodynamik

\chapter{Die diskrete \gls{wed}}
\label{chap:weber-diskret}

\section{Motivation: Feldlose diskrete Wechselwirkungen}
Die diskrete \gls{wed} beschreibt elektrische und magnetische Wechselwirkungen \textbf{direkt zwischen Ladungen} ohne elektromagnetische Felder als
ontologische Objekte. Statt eines kontinuierlichen Feldes verwendet sie eine \textbf{rekursive Update-Regel}, die auf vergangenen Zuständen basiert.

Diese Sichtweise ist für die \gls{iwt} fundamental:
\begin{itemize}
    \item Sie zeigt, dass lokale Dynamik ohne Feldkonzepte formuliert werden kann.
    \item Sie demonstriert, wie Kräfte aus relationalen Größen entstehen.
    \item Die diskrete Weber-Kraft ist der \textbf{lokale Grenzfall} der informationsbasierten Dynamik.
\end{itemize}

\section{Historischer Kontext: Von Weber zur diskreten Dynamik}
Wilhelm Eduard Weber formulierte 1846 \cite{Weber1846} eine elektrodynamische Kraft mit\\geschwindigkeits- und beschleunigungsabhängigen Termen. Im 20. Jahrhundert
\cite{Assis1999} wurde diese Theorie rekonstruiert und als Alternative zur Maxwell'schen Feldtheorie \cite{Maxwell1865,Einstein1905} erkannt.

Die diskrete Reformulierung der Weber-Kraft ist bemerkenswert, weil sie:
\begin{itemize}
    \item \textbf{Direkte Wechselwirkung}: Keine Felder als ontologische Objekte
    \item \textbf{Rekursive Struktur}: Basierend auf vergangenen Zuständen
    \item \textbf{Energieerhaltung}: Streng erhalten im diskreten Schema
    \item \textbf{Magnetische Effekte}: Aus rein mechanischen Prinzipien
    \item \textbf{Strahlungseffekte}: Durch Beschleunigungsterme beschrieben
\end{itemize}

\section{Die diskrete Weber-Kraft als fundamentale Update-Regel}
Die diskrete Weber-Kraft wird als fundamentale rekursive Regel der direkten Teilchenwechselwirkung eingeführt.

\subsection{Diskrete Größen}
Für zwei Ladungen \( q_1, q_2 \) im diskreten Netz:
\begin{itemize}
    \item \( r^{(n)} = |\vec{r}_1^{(n)} - \vec{r}_2^{(n)}| \): Diskret er Abstand zum Zeitindex \( n \)
    \item \( \Delta r^{(n)} = r^{(n)} - r^{(n-1)} \): Erste zeitliche Differenz
    \item \( \Delta^2 r^{(n)} = r^{(n+1)} - 2r^{(n)} + r^{(n-1)} \): Zweite zeitliche Differenz
    \item \( T \): Fundamentaler Zeitschritt (konstant)
\end{itemize}

\subsection{Diskrete Weber-Kraft (fundamentale Form)}
\begin{equation}
\vec{F}^{(n)} = \frac{q_1 q_2}{4\pi\varepsilon_0 \left(r^{(n)}\right)^2}
\left[
1 - \frac{1}{c^2} \left( \frac{\Delta r^{(n)}}{T} \right)^2 
+ \frac{2r^{(n)}}{c^2} \cdot \frac{\Delta^2 r^{(n)}}{T^2}
\right] \hat{\vec{r}}^{\,(n)}
\label{eq:weber_diskret}
\end{equation}

Diese Form ist:
\begin{enumerate}
    \item \textbf{Explizit rekursiv}: \( \vec{F}^{(n)} \) hängt von \( r^{(n)} \), \( r^{(n-1)} \), \( r^{(n+1)} \) ab
    \item \textbf{Nicht-zirkulär}: Keine implizite Gleichung \( F = f(\ddot{r}) \) mit \( \ddot{r} = F/m \)
    \item \textbf{Numerisch stabil}: Gut konditioniert als IIR-Filter
\end{enumerate}

\subsection{Berechnung des nächsten Zustands}
Aus Gleichung \eqref{eq:weber_diskret} kann \( r^{(n+1)} \) explizit berechnet werden:
\[
r^{(n+1)} = 2r^{(n)} - r^{(n-1)} + \frac{T^2 c^2}{2r^{(n)}} 
\left[
\frac{4\pi\varepsilon_0 \left(r^{(n)}\right)^2}{q_1 q_2} \vec{F}^{(n)} \cdot \hat{\vec{r}}^{\,(n)}
- 1 + \frac{1}{c^2} \left( \frac{\Delta r^{(n)}}{T} \right)^2
\right]
\]

\section{Interpretation der diskreten Terme}
Die drei Terme in Gleichung \eqref{eq:weber_diskret} haben klare physikalische Bedeutungen:

\subsection{Coulomb-Term (1)}
\begin{itemize}
    \item Beschreibt die statische Fernwirkung
    \item Unabhängig von Bewegung
    \item Analog zum Coulomb-Gesetz
\end{itemize}

\subsection*{Geschwindigkeits-Term \(-\left(\frac{\Delta r^{(n)}}{T}\right)^2/c^2\)}
\begin{itemize}
    \item Erzeugt magnetische Effekte
    \item Proportional zum Quadrat der Relativgeschwindigkeit
    \item Führt zu geschwindigkeitsabhängiger Abschirmung
\end{itemize}

\subsection*{Beschleunigungs-Term \(2r^{(n)}\frac{\Delta^2 r^{(n)}}{T^2}/c^2\)}
\begin{itemize}
    \item Beschreibt Reaktion auf Bewegungsänderungen
    \item Verantwortlich für Strahlungswiderstand
    \item Implementiert partielle Retardierung
    \item Stabilisiert die numerische Integration
\end{itemize}

\section{Struktur der diskreten Wechselwirkung}
Die charakteristische Struktur der diskreten Weber-Kraft,
\[
F^{(n)} \propto \frac{1}{\left(r^{(n)}\right)^2}
\left[
1 - \frac{1}{c^2} \left( \frac{\Delta r^{(n)}}{T} \right)^2 
+ \frac{2r^{(n)}}{c^2} \cdot \frac{\Delta^2 r^{(n)}}{T^2}
\right],
\]
zeigt, dass elektromagnetische Phänomene aus rein mechanischen Prinzipien entstehen können. Die Abhängigkeit von \( \left(\Delta r^{(n)}\right)^2 \) und
\( r^{(n)}\Delta^2 r^{(n)} \) ist das wesentliche Merkmal.

\section{Energie- und Impulserhaltung im diskreten Schema}

\subsection{Diskrete Energieerhaltung}
Die Gesamtenergie im Zwei-Körper-System ist:
\[
E^{(n)} = \frac{1}{2} m_1 \left(\frac{\Delta \vec{r}_1^{(n)}}{T}\right)^2 
+ \frac{1}{2} m_2 \left(\frac{\Delta \vec{r}_2^{(n)}}{T}\right)^2 
+ \frac{q_1 q_j}{4\pi\varepsilon_0 r^{(n)}}
\left[ 1 - \frac{1}{2c^2} \left( \frac{\Delta r^{(n)}}{T} \right)^2 \right]
\]

Es gilt: \( E^{(n+1)} = E^{(n)} \) bis auf Rundungsfehler.

\subsection{Diskrete Impulserhaltung}
Der Gesamtimpuls ist:
\[
\vec{P}^{(n)} = m_1 \frac{\Delta \vec{r}_1^{(n)}}{T} + m_2 \frac{\Delta \vec{r}_2^{(n)}}{T}
\]
Es gilt: \( \vec{P}^{(n+1)} = \vec{P}^{(n)} \) exakt.

\section{Implementierung als diskreter Algorithmus}

\subsection{Update-Schritt für zwei Ladungen}
Der Update-Schritt für zwei Ladungen erfolgt in folgenden Schritten:

\begin{enumerate}
    \item \textbf{Eingabe}: \( \vec{r}_1^{(n)}, \vec{r}_1^{(n-1)}, \vec{r}_2^{(n)}, \vec{r}_2^{(n-1)} \)
    \item \textbf{Relative Position}: \( \vec{r}^{(n)} = \vec{r}_1^{(n)} - \vec{r}_2^{(n)} \)
    \item \textbf{Abstand}: \( r^{(n)} = |\vec{r}^{(n)}| \)
    \item \textbf{Geschwindigkeitsdifferenz}: \( \Delta r^{(n)} = r^{(n)} - r^{(n-1)} \)
    \item \textbf{Kraftberechnung}: \( \vec{F}^{(n)} \) nach Gleichung \eqref{eq:weber_diskret}
    \item \textbf{Geschwindigkeitsupdate (Mittelung)}:
    \[
    \vec{v}_1^{(n+1/2)} = \frac{\vec{r}_1^{(n)} - \vec{r}_1^{(n-1)}}{T} + \frac{T}{2m_1} \vec{F}^{(n)}
    \]
    \item \textbf{Positionsupdate}:
    \[
    \vec{r}_1^{(n+1)} = \vec{r}_1^{(n)} + T \vec{v}_1^{(n+1/2)}
    \]
    \item \textbf{Analog für Teilchen 2} mit \( -\vec{F}^{(n)} \)
\end{enumerate}

\subsection{Erweiterung auf N Ladungen}
Für \( N \) Ladungen mit Positionen \( \vec{r}_i^{(n)} \):
\[
\vec{F}_i^{(n)} = \sum_{j \neq i} \frac{q_i q_j}{4\pi\varepsilon_0 \left(r_{ij}^{(n)}\right)^2}
\left[
1 - \frac{1}{c^2} \left( \frac{\Delta r_{ij}^{(n)}}{T} \right)^2 
+ \frac{2r_{ij}^{(n)}}{c^2} \cdot \frac{\Delta^2 r_{ij}^{(n)}}{T^2}
\right] \hat{\vec{r}}_{ij}^{\,(n)}
\]
mit \( r_{ij}^{(n)} = |\vec{r}_i^{(n)} - \vec{r}_j^{(n)}| \).

\section{Vergleich mit der kontinuierlichen Notation}

Feynman \cite{Feynman1963} erklärt, wie kontinuierliche Felder als Näherung entstehen.

\subsection{Kontinuierliche Form (emergente Näherung)}
Für \( T \to 0 \) ergibt sich als Grenzfall:
\[
\vec{F}(t) = \frac{q_1 q_2}{4\pi\varepsilon_0 r(t)^2}
\left[
1 - \frac{\dot{r}(t)^2}{c^2} + \frac{2r(t)\ddot{r}(t)}{c^2}
\right] \hat{\vec{r}}(t)
\]
Diese Form ist kompakt, aber problematisch:
\begin{itemize}
    \item \textbf{Zirkularität}: \( F \) hängt von \( \ddot{r} \) ab, aber \( \ddot{r} = F/m \)
    \item \textbf{Implizit}: Erfordert iterative Lösung
    \item \textbf{Instabil}: Bei numerischer Integration
\end{itemize}

\subsection{Vorteile der diskreten Form}
\begin{table}[ht]
\centering
\begin{tabular}{p{0.45\textwidth}|p{0.45\textwidth}}
\textbf{Diskrete Form} & \textbf{Kontinuierliche Form} \\
\hline
Explizit rekursiv & Implizit zirkulär \\
Numerisch stabil & Numerisch problematisch \\
Direkt implementierbar & Erfordert Iteration \\
Keine Felder benötigt & Felder als Hilfskonstrukte \\
Natürliche Retardierung & Retardierung künstlich \\
\end{tabular}
\caption{Vergleich der Formulierungen}
\end{table}

\section{Physikalische Interpretation im Informationsnetz}
In der Informations-Weber-Theorie hat die diskrete Weber-Kraft eine tiefere Bedeutung:

\subsection{Als lokaler Informationsfluss}
Die Weber-Kraft beschreibt den \textbf{lokalen Informationsfluss} zwischen benachbarten Knoten:
\[
\vec{F}^{(n)} \propto \nabla_d I^{(n)}
\]
mit diskretem Gradienten \( \nabla_d \).

\subsection{Als rekursive Filterung}
Die diskrete Form implementiert einen \textbf{IIR-Filter} (Infinite Impulse Response):
\[
r^{(n+1)} = a_1 r^{(n)} + a_2 r^{(n-1)} + b_0 F^{(n)}
\]
Diese Filterung:
\begin{itemize}
    \item Glättet hochfrequente Fluktuationen
    \item Erhält niederfrequente Signale
    \item Ist kausal (nur vergangene Zustände)
\end{itemize}

\subsection{Als fundamentale Update-Regel}
Jeder Zeitschritt \( n \to n+1 \) entspricht einer \textbf{globalen Synchronisation} des Informationsnetzes. Die Weber-Kraft ist die Projektion dieser globalen
Aktualisierung auf die lokale Dynamik zweier Knoten.

\section{Bedeutung für die \gls{iwt}}
Die diskrete Weber-Kraft ist kein konkurrierendes Modell zur informationsbasierten Theorie, sondern ihr \textbf{lokaler Grenzfall}:
\begin{itemize}
    \item Sie entsteht aus dem lokalen Anteil des diskreten Informationsfunktionals.
    \item Sie beschreibt Informationsflüsse zwischen benachbarten Knoten.
    \item Sie benötigt keine globalen Strukturen (Bohm-Potential).
    \item Sie ist vollständig mit der diskreten Informationserhaltung kompatibel.
\end{itemize}

\section{Zusammenfassung}
Kapitel~3 hat die diskrete \gls{wed} in ihrer fundamentalen Form entwickelt:

\begin{itemize}
    \item \textbf{Diskrete Weber-Kraft}: Explizit rekursive Update-Regel \eqref{eq:weber_diskret}
    \item \textbf{Nicht-zirkulär}: Löst das Problem der kontinuierlichen Notation
    \item \textbf{Energieerhaltend}: Streng erhalten im diskreten Schema
    \item \textbf{Feldlos}: Direkte Wechselwirkung ohne elektromagnetische Felder
    \item \textbf{Algorithmisch}: Direkt als Update-Algorithmus implementierbar
    \item \textbf{Lokaler Grenzfall}: Der informationsbasierten Dynamik
\end{itemize}

Die diskrete Formulierung zeigt, dass elektromagnetische Phänomene – einschließlich magnetischer und Strahlungseffekte – aus rein mechanischen, rekursiven Prinzipien
entstehen können, ohne auf Feldkonzepte zurückzugreifen.

Im nächsten Kapitel wird diese Struktur in den allgemeineren Rahmen des diskreten Informationsfunktionals eingebettet.
