% Datei: kapitel_1_4.tex
% Teil I: Fundamentale diskrete Theorie
% Kapitel 4: Das diskrete Informations-Lagrange-Funktional

\chapter{Das diskrete Informations-Lagrange-Funktional}
\label{chap:lagrange-diskret}

\paragraph{Fundamentale Darstellung}
Dieses Kapitel entwickelt das \textbf{diskrete Informations-Lagrange-Funktional} als mathematische Grundlage der \gls{iwt}. Alle Größen sind diskrete Sequenzen mit Zeitindex
\( n \) und Raumindex \( k \). Die Variation erfolgt über diskrete Ableitungen, nicht über Differentiale.

\section{Einleitung: Variationsprinzip im diskreten Netz}
Die diskrete \gls{iwt} beschreibt physikalische Systeme durch die Dynamik der diskreten Informationsverteilung \( I_k^{(n)} \). Um diese Dynamik mathematisch
zu formulieren, benötigen wir ein \textbf{diskretes Variationsprinzip}, das die zeitliche Entwicklung von \( I_k^{(n)} \) bestimmt.

Das diskrete Informations-Lagrange-Funktional ersetzt im informationsbasierten Rahmen:
\begin{itemize}
    \item die Newtonsche Bewegungsgleichung durch rekursive Update-Regeln,
    \item die Maxwell-Gleichungen durch diskrete Weber-Kopplungen,
    \item die Schrödinger-Gleichung durch diskrete Bohm-Struktur,
    \item die geometrische Gravitation durch emergente diskrete Metrik.
\end{itemize}
Es bildet die mathematische Grundlage der gesamten Theorie und zeigt, dass kontinuierliche Gleichungen als Grenzfälle einer tieferen diskreten Informationsdynamik erscheinen.

\section{Grundidee des diskreten informationsbasierten Variationsprinzips}
Die diskrete \gls{iwt} geht von folgenden Prinzipien aus:
\begin{enumerate}
    \item Der physikalische Zustand ist eine diskrete Informationsverteilung \( I_k^{(n)} \).
    \item Information ist eine diskret erhaltene Größe.
    \item Dynamik ist rekursive Umlagerung von Information.
    \item Lokale und globale Dynamik sind komplementär.
\end{enumerate}

Daraus folgt, dass die Dynamik durch ein diskretes Funktional beschrieben werden muss, das sowohl lokale als auch globale Informationsstrukturen berücksichtigt.

\section{Das diskrete Informations-Lagrange-Funktional}
Das diskrete Informations-Lagrange-Funktional lautet allgemein:
\begin{equation}
    L_I\left[\{I_k^{(n)}\}\right]
    =
    \sum_{k=1}^{N} 
    \mathcal{F}_k
    \!\left(
        I_k^{(n)},\,
        \Delta I_k^{(n)},\,
        \delta_t I_k^{(n)}
    \right)
    \Delta V_k.
    \label{eq:info_lagrange_diskret}
\end{equation}

Hierbei ist:
\begin{itemize}
    \item \( k = 1,\dots,N \): Index der diskreten Informationszelle (Knoten)
    \item \( I_k^{(n)} \in \mathbb{R}^+ \): Informationswert am Knoten \( k \) zum Zeitindex \( n \)
    \item \( \Delta I_k^{(n)} \): Diskreter räumlicher Gradient (Nachbarschaftsdifferenzen)
    \item \( \delta_t I_k^{(n)} = I_k^{(n)} - I_k^{(n-1)} \): Diskrete zeitliche Änderung
    \item \( \Delta V_k \): Diskretes Volumenelement (knotenspezifisch)
    \item \( \mathcal{F}_k \): Diskrete Lagrangedichte am Knoten \( k \)
\end{itemize}
Die Form von \( \mathcal{F}_k \) ergibt sich aus den Axiomen der Theorie und den Symmetrien des diskreten Informationsnetzes.

\subsection{Diskrete räumliche Gradienten}
Der diskrete Gradient am Knoten \( k \) ist definiert als:
\[
\Delta I_k^{(n)} = \sum_{l \in \mathcal{N}(k)} w_{kl} \left( I_l^{(n)} - I_k^{(n)} \right)
\]
mit:
\begin{itemize}
    \item \( \mathcal{N}(k) \): Nachbarknoten von \( k \)
    \item \( w_{kl} \): Kopplungsgewichte (normalisiert: \( \sum_{l} w_{kl} = 1 \))
\end{itemize}

\subsection{Diskrete zeitliche Änderung}
Die diskrete zeitliche Ableitung ist:
\[
\delta_t I_k^{(n)} = \frac{I_k^{(n)} - I_k^{(n-1)}}{T}
\]
mit fundamentalem Zeitschritt \( T \).

\section{Diskrete Variation und Euler-Lagrange-Gleichungen}
Die Dynamik folgt aus dem diskreten Variationsprinzip:
\[
\delta L_I = 0 \quad \text{für alle } \delta I_k^{(n)}.
\]

Die Variation nach \( I_k^{(n)} \) führt zur diskreten Euler-Lagrange-Gleichung:
\begin{equation}
\delta_t\left( \frac{\partial \mathcal{F}_k}{\partial (\delta_t I_k^{(n)})} \right)
+ \Delta\left( \frac{\partial \mathcal{F}_k}{\partial (\Delta I_k^{(n)})} \right)
- \frac{\partial \mathcal{F}_k}{\partial I_k^{(n)}}
= 0.
\label{eq:euler_lagrange_diskret}
\end{equation}

Hierbei sind:
\begin{itemize}
    \item \( \delta_t(\cdot) \): Diskrete zeitliche Vorwärtsdifferenz
    \item \( \Delta(\cdot) \): Diskrete räumliche Divergenz
\end{itemize}

\subsection{Diskrete zeitliche Vorwärtsdifferenz}
\[
\delta_t f^{(n)} = \frac{f^{(n+1)} - f^{(n)}}{T}
\]

\subsection{Diskrete räumliche Divergenz}
\[
\Delta f_k = \sum_{l \in \mathcal{N}(k)} w_{kl} (f_l - f_k)
\]

\section{Diskreter Informationsfluss als natürliche Konsequenz}
Aus der diskreten Euler-Lagrange-Gleichung folgt unmittelbar der diskrete Informationsfluss:
\[
J_{kl}^{(n)} = \frac{\partial \mathcal{F}_k}{\partial (\Delta_{kl} I^{(n)})}
\]
mit \( \Delta_{kl} I^{(n)} = I_l^{(n)} - I_k^{(n)} \).

Damit wird die diskrete Kontinuitätsgleichung
\[
\delta_t I_k^{(n)} + \sum_{l \in \mathcal{N}(k)} J_{kl}^{(n)} = 0
\]
zu einer direkten Konsequenz des diskreten Variationsprinzips.

\section{Zerlegung in lokale und globale diskrete Beiträge}
Die Struktur von \( \mathcal{F}_k \) erlaubt eine natürliche Zerlegung:
\begin{equation}
\mathcal{F}_k = \mathcal{F}_k^{\text{lokal}} + \mathcal{F}_k^{\text{global}}.
\label{eq:zerlegung_diskret}
\end{equation}

\subsection{Lokaler diskreter Anteil}
Der lokale Anteil beschreibt lokale Informationsflüsse:
\begin{equation}
\mathcal{F}_k^{\text{lokal}} = \alpha \left( \delta_t I_k^{(n)} \right)^2 + \beta \left( \Delta I_k^{(n)} \right)^2 + \gamma I_k^{(n)} \ln\left( \frac{I_k^{(n)}}{I_0} \right).
\label{eq:lokal_diskret}
\end{equation}

Er führt im diskreten Grenzfall zu:
\begin{itemize}
    \item der diskreten Weber-Kraft (Kapitel~\ref{chap:weber-diskret}),
    \item klassischen Trägheits- und Energiebegriffen,
    \item stabiler numerischer Integration.
\end{itemize}

Die Variation von \( \mathcal{F}_k^{\text{lokal}} \) ergibt:
\[
\frac{\partial \mathcal{F}_k^{\text{lokal}}}{\partial I_k^{(n)}} = \gamma \left( 1 + \ln\left( \frac{I_k^{(n)}}{I_0} \right) \right)
\]
\[
\frac{\partial \mathcal{F}_k^{\text{lokal}}}{\partial (\delta_t I_k^{(n)})} = 2\alpha \delta_t I_k^{(n)}
\]
\[
\frac{\partial \mathcal{F}_k^{\text{lokal}}}{\partial (\Delta I_k^{(n)})} = 2\beta \Delta I_k^{(n)}
\]

\subsection{Globaler diskreter Anteil}
Der globale Anteil beschreibt systemische Informationsorganisation:
\begin{equation}
\mathcal{F}_k^{\text{global}} = \lambda \frac{\left( \Delta I_k^{(n)} \right)^2}{I_k^{(n)}} + \mu \frac{\Delta^2 I_k^{(n)}}{I_k^{(n)}}
\label{eq:global_diskret}
\end{equation}
mit diskretem Laplace-Operator:
\[
\Delta^2 I_k^{(n)} = \sum_{l \in \mathcal{N}(k)} w_{kl} \left( I_l^{(n)} - I_k^{(n)} \right)
\]

Die Variation dieses Terms führt zum diskreten Bohm-Potential:
\[
Q_k^{(n)} = -\frac{\hbar^2}{2m} \frac{\Delta^2 \sqrt{I_k^{(n)}}}{\sqrt{I_k^{(n)}}}.
\]

\section{Diskrete Weber-Kraft und Bohm-Potential als Grenzfälle}
Die diskrete \gls{iwt} reproduziert zwei fundamentale Strukturen:
\begin{itemize}
    \item \textbf{Diskrete Weber-Kraft}: Lokaler Grenzfall der diskreten Informationsdynamik
    \item \textbf{Diskretes Bohm-Potential}: Globaler Grenzfall der diskreten Informationsdynamik
\end{itemize}

Beide entstehen aus demselben diskreten Funktional — sie sind keine unabhängigen Modelle, sondern zwei Projektionen derselben diskreten Informationsstruktur.

\section{Symmetrien und Erhaltungsgrößen im diskreten Netz}
Nach dem diskreten Noether-Theorem ergeben sich:

\subsection{Translationsinvarianz}
Wenn \( \mathcal{F}_k \) nur von Differenzen \( I_l^{(n)} - I_k^{(n)} \) abhängt, dann ist der diskrete Gesamtimpuls erhalten:
\[
P^{(n)} = \sum_k \frac{\partial \mathcal{F}_k}{\partial (\delta_t I_k^{(n)})} = \text{konstant}
\]

\subsection{Zeitinvarianz}
Wenn \( \mathcal{F}_k \) nicht explizit von \( n \) abhängt, dann ist die diskrete Energie erhalten:
\[
E^{(n)} = \sum_k \left[ \delta_t I_k^{(n)} \frac{\partial \mathcal{F}_k}{\partial (\delta_t I_k^{(n)})} - \mathcal{F}_k \right] = \text{konstant}
\]

\subsection{Rotationsinvarianz}
Wenn das Netz rotationssymmetrisch ist, dann ist der diskrete Drehimpuls erhalten.

\section{Implementierung als diskreter Update-Algorithmus}
Aus Gleichung \eqref{eq:euler_lagrange_diskret} ergibt sich eine explizite Update-Regel für \( I_k^{(n+1)} \):

\subsection{Allgemeine Update-Form}
\[
I_k^{(n+1)} = I_k^{(n)} + T \cdot \Phi_k\left( I_k^{(n)}, \{I_l^{(n)}\}, I_k^{(n-1)} \right)
\]
mit
\[
\Phi_k = -\frac{1}{2\alpha} \left[ \Delta\left( 2\beta \Delta I_k^{(n)} \right) - \frac{\partial \mathcal{F}_k}{\partial I_k^{(n)}} \right]
\]

\subsection{Konkrete Update-Schritte}
\begin{enumerate}
    \item Berechne räumliche Gradienten: \( \Delta I_k^{(n)} \), \( \Delta^2 I_k^{(n)} \)
    \item Berechne partielle Ableitungen: \( \frac{\partial \mathcal{F}_k}{\partial I_k^{(n)}} \), \( \frac{\partial \mathcal{F}_k}{\partial (\Delta I_k^{(n)})} \), etc.
    \item Löse \eqref{eq:euler_lagrange_diskret} nach \( I_k^{(n+1)} \) auf
    \item Update aller Knoten gleichzeitig (globale Synchronisation)
\end{enumerate}

\section{Grenzfall zur kontinuierlichen Form}
Für \( T \to 0 \) und \( N \to \infty \) mit \( \Delta V_k \to 0 \) ergibt sich der kontinuierliche Grenzfall:
\[
L_I[\rho_I] = \int \mathcal{F}(\rho_I, \nabla\rho_I, \partial_t\rho_I) \, d^3x
\]
mit \( \rho_I(\vec{r},t) = \lim I_k^{(n)} \).

Die diskrete Euler-Lagrange-Gleichung \eqref{eq:euler_lagrange_diskret} geht über in:
\[
\frac{\partial}{\partial t} \left( \frac{\partial\mathcal{F}}{\partial(\partial_t\rho_I)} \right) + \nabla \cdot \left( \frac{\partial\mathcal{F}}{\partial(\nabla\rho_I)} \right) - \frac{\partial\mathcal{F}}{\partial\rho_I} = 0.
\]

\section{Zusammenfassung}
Das diskrete Informations-Lagrange-Funktional bildet die mathematische Grundlage der diskreten \gls{iwt}:
\begin{itemize}
    \item Es beschreibt die Dynamik der diskreten Informationsverteilung \( I_k^{(n)} \).
    \item Es erzeugt die diskrete Kontinuitätsgleichung als natürliche Konsequenz.
    \item Es zerlegt sich in lokale und globale diskrete Beiträge.
    \item Es reproduziert die diskrete Weber-Kraft und das diskrete Bohm-Potential.
    \item Es liefert diskrete Erhaltungssätze aus diskreten Symmetrien.
    \item Es ist algorithmisch direkt implementierbar.
    \item Der kontinuierliche Grenzfall emergiert für feine Diskretisierung.
\end{itemize}
Kapitel~5 entwickelt darauf aufbauend die diskrete Informationsmetrik und die emergente diskrete Raumzeit.
