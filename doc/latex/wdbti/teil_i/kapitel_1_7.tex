\chapter{Emergente Informationsmetrik}

\section{Die dynamische Gleichung der Informationsmetrik}

\subsection{Ausgangspunkt: Das Informations-Lagrange-Funktional}
Die \gls{iwt} basiert auf einem diskreten Informationsfeld \(I_k^{(n)}\), dessen Dynamik durch ein diskretes Informations-Lagrange-Funktional beschrieben
wird. Dieses setzt sich aus drei fundamentalen Beiträgen zusammen:
\[
\mathcal{L}_d[I_k^{(n)}] 
= 
\mathcal{L}_{\mathrm{lokal}}
+
\mathcal{L}_{\mathrm{global}}
+
\mathcal{L}_{\mathrm{fraktal}}.
\]

\subsubsection{Lokaler Anteil: Weber-Struktur}
Der lokale Anteil beschreibt die direkte Wechselwirkung im Sinne der diskreten Weber-Dynamik:
\[
\mathcal{L}_{\mathrm{lokal}}
=
\frac{1}{2} \sum_{k,l} K_{kl}^{(n)} \Delta_{kl} I^{(n)} \Delta_{kl} I^{(n)}.
\]
Er erzeugt Trägheit, klassische Dynamik und lokale Informationsflüsse zwischen benachbarten Knoten.

\subsubsection{Globaler Anteil: Bohm-Struktur}
Der globale Anteil beschreibt die nichtlokale Organisationsstruktur:
\[
\mathcal{L}_{\mathrm{global}}
=
-\frac{\lambda}{2}\sum_k \frac{\Delta^2 I_k^{(n)}}{I_k^{(n)}}.
\]
Dieser Term erzeugt Wellenphänomene, Nichtlokalität und quantenartige Kohärenz über das gesamte Netzwerk.

\subsubsection{Fraktaler Anteil: Kosmische Skalierung}
Die fraktale Informationsarchitektur des Universums führt zu einer logarithmischen Skalierungsinvarianz:
\[
\mathcal{L}_{\mathrm{fraktal}}
=
\mu \ln\!\left(1+\gamma_{\mathrm{eff}} G \rho_{\mathrm{eff}} L^2\right) \sum_k I_k^{(n)}.
\]
Dieser Term ist verantwortlich für kosmische Rotverschiebung, die Verlustkonstante \(\bar{\alpha}(L)\) und die CMB-Gleichgewichtstemperatur auf großen Skalen.

\section{Variation nach der Informationsmetrik}
Da die Metrik \(g_{kl}^{(n)}\) nicht vorgegeben ist, sondern emergent aus der Kopplungsstruktur \(K_{kl}^{(n)}\), folgt ihre Dynamik aus der Variation des diskreten
Funktionals:
\[
\frac{\delta \mathcal{L}_d}{\delta g_{kl}^{(n)}} = 0.
\]
Die Variation der drei Beiträge ergibt die diskrete dynamische Gleichung:
\[
g_{kl}^{(n+1)} = g_{kl}^{(n)} + T \cdot \left[
\Delta_{k} I^{(n)} \Delta_{l} I^{(n)}
-
\lambda\,\frac{\Delta_{kl}^2 I^{(n)}}{I_{kl}^{(n)}}
+
\mu\,g_{kl}^{(n)}\,\ln\!\left(1+\gamma_{\mathrm{eff}} G \rho_{\mathrm{eff}} L^2\right)
\right],
\]
wobei:
- \(\Delta_k I^{(n)}\) der diskrete Gradient am Knoten \(k\) ist,
- \(\Delta_{kl}^2 I^{(n)}\) der diskrete Laplace-Operator zwischen \(k\) und \(l\) ist,
- \(I_{kl}^{(n)} = \frac{1}{2}(I_k^{(n)} + I_l^{(n)})\) der gemittelte Informationswert ist,
- \(T\) der fundamentale Zeitschritt ist.

\section{Die fundamentale Gleichung der Informationsmetrik}
Damit ergibt sich die fundamentale dynamische Gleichung der \gls{iwt} in diskreter Form:
\[
\boxed{
g_{kl}^{(n+1)} = g_{kl}^{(n)} + T \cdot \left[
\Delta_{k} I^{(n)} \Delta_{l} I^{(n)}
-
\lambda\,\frac{\Delta_{kl}^2 I^{(n)}}{I_{kl}^{(n)}}
+
\mu\,g_{kl}^{(n)}\,\ln\!\left(1+\gamma_{\mathrm{eff}} G \rho_{\mathrm{eff}} L^2\right)
\right]
}
\]
Diese Gleichung vereinigt:
\begin{itemize}
    \item \textbf{Lokale Weber-Dynamik} (\(\Delta_{k} I^{(n)} \Delta_{l} I^{(n)}\)): Direkte Informationsflüsse zwischen Knoten
    \item \textbf{Globale Bohm-Struktur} (\(\Delta_{kl}^2 I^{(n)} / I_{kl}^{(n)}\)): Nichtlokale Organisation des Gesamtnetzwerks
    \item \textbf{Fraktale kosmische Skalierung} (logarithmischer Term): Skaleninvariante Struktur des Universums
\end{itemize}
Sie bildet den dynamischen Kern der Theorie, aus dem Raum, Zeit, Energie, Gravitation, Quantenstruktur und kosmologische Skalierung emergieren.

\section{Kontinuierlicher Grenzfall}
Für \(T \to 0\) und feine Diskretisierung ergibt sich der kontinuierliche Grenzfall:
\[
\frac{d}{dt} g_{ij}
=
\partial_i I \partial_j I
-
\lambda\,\frac{\partial_i\partial_j I}{I}
+
\mu\,g_{ij}\,\ln\!\left(1+\gamma_{\mathrm{eff}} G \rho_{\mathrm{eff}} L^2\right).
\]
Diese kontinuierliche Form ist kompakt und für analytische Berechnungen nützlich, aber die fundamentale Beschreibung bleibt die diskrete rekursive Form.

\section{Grenzfälle und physikalische Interpretation}

\subsection{Klassischer Grenzfall}
Für schwache Informationsgradienten (\(\Delta_k I^{(n)} \ll I_k^{(n)}\)) dominiert der lokale Term:
\[
g_{kl}^{(n+1)} \approx g_{kl}^{(n)} + T \cdot \Delta_{k} I^{(n)} \Delta_{l} I^{(n)}.
\]
Dies reproduziert klassische Mechanik und Weber-Dynamik im emergenten Grenzfall.

\subsection{Quantenmechanischer Grenzfall}
Für stark gekrümmte Informationsfelder (\(\Delta_{kl}^2 I^{(n)} \gg I_{kl}^{(n)}\)) dominiert der globale Term:
\[
g_{kl}^{(n+1)} \approx g_{kl}^{(n)} - T \cdot \lambda\,\frac{\Delta_{kl}^2 I^{(n)}}{I_{kl}^{(n)}}.
\]
Dies reproduziert das Bohm-Potential und quantenartige Kohärenz.

\subsection{Kosmologischer Grenzfall}
Für große Skalen (\(L \gg \lambda_0\), mit \(\lambda_0\) als fundamentaler Länge) dominiert der fraktale Term:
\[
g_{kl}^{(n+1)} \approx g_{kl}^{(n)} + T \cdot \mu\,g_{kl}^{(n)}\,\ln\!\left(1+\gamma_{\mathrm{eff}} G \rho_{\mathrm{eff}} L^2\right).
\]
Dies erzeugt:
\begin{itemize}
    \item Kosmische Rotverschiebung
    \item Die Verlustkonstante \(\bar{\alpha}(L)\)
    \item Das CMB-Gleichgewicht
    \item Fraktale Struktur des Universums
\end{itemize}

\section{Implementierung als diskreter Update-Algorithmus}
Die fundamentale Gleichung ist direkt als Update-Algorithmus implementierbar:
\textbf{Schritte zur Berechnung von \(g_{kl}^{(n+1)}\):}
\begin{enumerate}
    \item Berechne diskrete Gradienten: $\Delta_k I^{(n)}$, $\Delta_{kl}^2 I^{(n)}$
    \item Berechne gemittelte Information: $I_{kl}^{(n)} = \frac{1}{2}(I_k^{(n)} + I_l^{(n)})$
    \item Update Metrik gemäß:
    \[
    g_{kl}^{(n+1)} = g_{kl}^{(n)} + T \cdot \left[
    \Delta_{k} I^{(n)} \Delta_{l} I^{(n)}
    -
    \lambda\,\frac{\Delta_{kl}^2 I^{(n)}}{I_{kl}^{(n)}}
    +
    \mu\,g_{kl}^{(n)}\,\ln\!\left(1+\gamma_{\mathrm{eff}} G \rho_{\mathrm{eff}} L^2\right)
    \right]
    \]
\end{enumerate}
Diese rekursive Update-Regel ist numerisch stabil und erfordert nur vergangene Zustände ($g_{kl}^{(n)}$, $I_k^{(n)}$, $I_k^{(n-1)}$).

\section{Konsequenzen für Naturkonstanten}
Die Urgleichung liefert die Grundlage für die Herleitung der Naturkonstanten als emergente Skalierungsparameter:
\begin{itemize}
    \item \(c\) als maximale Informationsflussrate: $c = \frac{\lambda_{\max}}{T}$
    \item \(\hbar\) als globale Informationsgranularität: $\hbar = \alpha \Delta I_{\min} \lambda_0^2$
    \item \(G\) als Kopplungsparameter der Informationsmetrik: $G = \beta \frac{\lambda_0^{3-D}}{f_{\max}^2}$
    \item \(\alpha\) (Feinstrukturkonstante) als Verhältnis lokaler und globaler Kopplung
\end{itemize}
Diese Konstanten sind keine fundamentalen Eingaben, sondern emergente Größen aus der Netzwerkstruktur.

\section{Zusammenfassung}
Mit der dynamischen Gleichung der Informationsmetrik liegt erstmals eine vollständig geschlossene, selbstkonsistente Urgleichung vor, aus der Raum, Zeit, Energie,
Gravitation, Quantenstruktur und kosmologische Phänomene emergieren. Die Informations-Weber-Theorie wird damit zu einer echten Urtheorie der Physik, die in ihrer
fundamentalen Form diskret und rekursiv formuliert ist, während die bekannten kontinuierlichen Theorien als emergente Grenzfälle erscheinen.
