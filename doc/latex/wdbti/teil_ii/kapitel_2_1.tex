% Datei: kapitel_2_1.tex
% Teil II: Emergenz kontinuierlicher Physik
% Kapitel 1: Vergleich mit etablierten Theorien

\chapter{Vergleich mit etablierten Theorien}
\label{chap:vergleich}

\section{Einleitung: Die IWT als Urtheorie}
Die Informations-Weber-Theorie (IWT) ist keine weitere konkurrierende Einzeltheorie, sondern eine \emph{Urtheorie}, aus der klassische Mechanik, Elektrodynamik,
Quantenmechanik und Relativitätstheorie als Grenzfälle hervorgehen. Dieses Kapitel vergleicht die emergenten Strukturen der Informationsdynamik mit den etablierten
physikalischen Theorien und zeigt, wie diese als Näherungen einer tieferen Informationsordnung erscheinen.

Der Vergleich erfolgt entlang fünf fundamentaler Fragen:

\begin{table}[ht]
\centering
\begin{tabular}{p{0.3\textwidth}|p{0.3\textwidth}|p{0.3\textwidth}}
\textbf{Frage} & \textbf{Etablierte Theorien} & \textbf{IWT (Urtheorie)} \\
\hline
Physikalischer Zustand? & Teilchen, Felder, Wellenfunktionen & Diskrete Information \( I_k^{(n)} \) \\
Dynamik? & Kräfte, Feldgleichungen, Schrödinger-Gleichung & Rekursive Update-Regeln \\
Raum und Zeit? & Fundamentales Kontinuum & Emergente Metrik \( g_{kl}^{(n)} \) \\
Kausalität? & Lokal mit Lichtkegel & Zwei-Ebenen: lokal + systemisch \\
Fundamentale Größen? & Energie, Ladung, Masse & Information \( I_k^{(n)} \), Kopplungen \( K_{kl}^{(n)} \) \\
\end{tabular}
\caption{Fundamentale Fragen im Vergleich}
\end{table}

\section{Klassische Mechanik als lokaler Grenzfall}

\subsection{Klassische Mechanik: Fundamentale Konzepte}
Die klassische Mechanik basiert auf:
\begin{itemize}
    \item Punktteilchen mit Masse \( m \)
    \item Kräfte \( \vec{F} \)
    \item Absolute Zeit \( t \)
    \item Euklidischer Raum \( \mathbb{R}^3 \)
    \item Newtonsche Bewegungsgleichung: \( m\ddot{\vec{r}} = \vec{F} \)
\end{itemize}

\subsection{Emergenz aus der IWT}
In der IWT entsteht die klassische Mechanik als Grenzfall:
\begin{enumerate}
    \item \textbf{Schwache Informationsgradienten}: \( |\Delta I_k^{(n)}| \ll I_k^{(n)} \)
    \item \textbf{Dominante lokale Dynamik}: \( \mathcal{F}_k^{\text{lokal}} \gg \mathcal{F}_k^{\text{global}} \)
    \item \textbf{Kontinuumsnäherung}: \( T \to 0 \), \( N \to \infty \)
\end{enumerate}

\subsection{Emergente Gleichungen}
Unter diesen Bedingungen ergibt sich aus dem diskreten Euler-Lagrange-Formalismus:
\[
m_i \frac{\Delta^2 \vec{r}_i^{(n)}}{T^2} = \sum_{j \neq i} \vec{F}_{ij}^{(n)}
\]
was im Grenzfall \( T \to 0 \) zur Newtonschen Gleichung \( m_i \ddot{\vec{r}}_i = \sum_j \vec{F}_{ij} \) wird.

\subsection{Fundamentaler vs. emergenter Status}
\begin{table}[ht]
\centering
\begin{tabular}{p{0.45\textwidth}|p{0.45\textwidth}}
\textbf{In klassischer Mechanik (fundamental)} & \textbf{In IWT (emergent)} \\
\hline
Masse \( m \) ist primitiv & \( m_{\text{eff},k} = \alpha \sum_l (I_k - I_l)^2 \) \\
Kraft \( \vec{F} \) ist primitiv & \( \vec{F}_{kl} \propto \Delta I_{kl} \) \\
Zeit \( t \) ist absolut & \( t \approx nT \) (Update-Index) \\
Raum \( \mathbb{R}^3 \) ist gegeben & \( g_{kl} \) emergiert aus \( K_{kl} \) \\
\end{tabular}
\end{table}

\section{Elektrodynamik: Maxwell, Lorentz und Weber}

\subsection{Drei Formen der Elektrodynamik}
\begin{enumerate}
    \item \textbf{Maxwell-Theorie (MT)}: Felder \( \vec{E}, \vec{B} \) als ontologische Objekte
    \item \textbf{Lorentz-Kraft}: Phänomenologische Formel \( \vec{F} = q(\vec{E} + \vec{v} \times \vec{B}) \)
    \item \textbf{Weber-Elektrodynamik (WED)}: Direkte Wechselwirkung ohne Felder
\end{enumerate}

\subsection{Maxwell-Theorie als effektive Feldbeschreibung}
In der IWT erscheinen die Maxwell-Felder \cite{Maxwell1865} als effektive Kontinuumsnäherungen:
\[
\vec{E}(\vec{r},t) = \lim_{\text{Feindiskretisierung}} \langle \vec{F}_{kl}^{(n)} \rangle
\]
\[
\vec{B}(\vec{r},t) = \lim_{\text{Feindiskretisierung}} \langle \vec{F}_{kl}^{(n)} \times \hat{\vec{r}}_{kl} \rangle
\]
Die Maxwell-Gleichungen emergieren als Kontinuitätsbedingungen für die Informationsflüsse.

\subsection{Lorentz-Kraft als phänomenologische Näherung}
Die Lorentz-Kraft ist eine Näherung der Weber-Kraft \cite{Weber1846} für:
\begin{itemize}
    \item Kleine Geschwindigkeiten: \( v \ll c \)
    \item Stationäre Ströme
    \item Schwache Beschleunigungen
\end{itemize}

\subsection{Weber-Kraft als lokaler Grenzfall der IWT}
Die Weber-Kraft ist der lokale Grenzfall der IWT-Dynamik:
\[
F_{\text{lokal}} = F_{\text{WED}}
\]
Sie entsteht aus dem lokalen Anteil \( \mathcal{F}_k^{\text{lokal}} \) des diskreten Informationsfunktionals.

\section{Quantenmechanik als globale Informationsdynamik}
Die QM entsteht aus globalen Informationsflüssen. Bohm, Cramer und Valentini \cite{bohm1952,Cramer1986,Valentini2010} sind genau die Theorien, die QM als
Informationsdynamik interpretieren.

\subsection{QM: Fundamentale Konzepte}
Die Quantenmechanik basiert auf:
\begin{itemize}
    \item Wellenfunktion \( \psi(\vec{r},t) \)
    \item Superposition: \( \psi = \alpha\psi_1 + \beta\psi_2 \)
    \item Interferenz: \( |\psi|^2 = |\psi_1 + \psi_2|^2 \)
    \item Nichtlokalität: Verschränkung
    \item Schrödinger-Gleichung: \( i\hbar\partial_t\psi = \hat{H}\psi \)
\end{itemize}

\subsection{Emergenz aus der IWT}
In der IWT entstehen diese Phänomene aus:
\begin{enumerate}
    \item \textbf{Globale Informationsorganisation}: Dominanz von \( \mathcal{F}_k^{\text{global}} \)
    \item \textbf{Phasenkohärenz}: Wohldefinierte \( \phi_k^{(n)} \)
    \item \textbf{Systemische Kausalität}: Instantanes Bohm-Potential
\end{enumerate}

\subsection{Emergente Schrödinger-Gleichung}
Aus der diskreten Euler-Lagrange-Gleichung für \( \mathcal{F}_k^{\text{global}} \) ergibt sich:
\[
i\hbar \frac{\psi_k^{(n+1)} - \psi_k^{(n)}}{T} = -\frac{\hbar^2}{2m} \Delta^2 \psi_k^{(n)} + V_k \psi_k^{(n)}
\]
mit \( \psi_k^{(n)} = \sqrt{I_k^{(n)}} e^{i\phi_k^{(n)}} \).

Im Kontinuumslimes \( T \to 0 \):
\[
i\hbar \partial_t \psi = -\frac{\hbar^2}{2m} \nabla^2 \psi + V\psi
\]

\section{Relativitätstheorie als emergente Geometrie}

\subsection{Spezielle Relativitätstheorie (SRT)}
\begin{itemize}
    \item Fundamentale Lichtgeschwindigkeit \( c \)
    \item Lorentz-Transformationen
    \item Raumzeit \( \mathcal{M} = \mathbb{R}^{3,1} \) \cite{Einstein1905}
\end{itemize}

In der IWT emergiert die SRT aus:
\begin{itemize}
    \item Maximaler Informationsflussrate: \( c = d_{\text{max}}/T \)
    \item Invarianz der diskreten Wirkung unter Netz-Transformationen
    \item Relativitätsprinzip als Symmetrie des Informationsflusses
\end{itemize}

\subsection{Allgemeine Relativitätstheorie (ART)}
\begin{itemize}
    \item Dynamische Raumzeit-Metrik \( g_{\mu\nu}(x) \)
    \item Einstein-Gleichungen: \( G_{\mu\nu} = 8\pi G T_{\mu\nu} \) \cite{einstein1915}
    \item Geodätische Bewegung
\end{itemize}

In der IWT emergiert die ART aus:
\begin{itemize}
    \item Dynamischer diskreter Metrik: \( g_{kl}^{(n+1)} = f(g_{kl}^{(n)}, I_k^{(n)}) \)
    \item Große Informationsnetze: \( N \to \infty \)
    \item Kontinuumsnäherung der Netzgeometrie
\end{itemize}

\subsection{Emergente Einstein-Gleichungen}
Aus der Update-Regel für die diskrete Metrik \eqref{eq:metrik_update} ergibt sich im Kontinuumslimes:
\[
\frac{d}{dt}g_{ij} = \partial_i I \partial_j I - \lambda \frac{\partial_i\partial_j I}{I} + \mu g_{ij} \ln(1 + \gamma G\rho L^2)
\]

Für schwache Felder und geeignete Parametrisierung kann dies in die Form der Einstein-Gleichungen gebracht werden.

\section{Grenzfälle und Übergänge}
Die IWT reproduziert die etablierten Theorien in folgenden Grenzfällen:

\begin{table}[ht]
\centering
\begin{tabular}{p{0.3\textwidth}|p{0.3\textwidth}|p{0.3\textwidth}}
\textbf{Theorie} & \textbf{IWT-Grenzfall} & \textbf{Emergenzbedingungen} \\
\hline
Klassische Mechanik & \( \lambda \to 0 \) & Schwache Gradienten \\
Weber-Elektrodynamik & Lokales \( \mathcal{F}^{\text{lokal}} \) & Keine globalen Beiträge \\
Maxwell-Theorie & Kontinuumslimes von WED & \( T \to 0 \), \( N \to \infty \) \\
Quantenmechanik & \( \lambda \to \infty \) & Starke globale Kopplung \\
Spezielle Relativität & Symmetrie des Flusses & Invarianz unter Netz-Transformationen \\
Allgemeine Relativität & Großes Netz, Kontinuum & \( N \gg 1 \), feine Diskretisierung \\
\end{tabular}
\caption{Emergenz etablierter Theorien aus der IWT}
\end{table}

\section{Frequenzabhängige Lichtablenkung als Test}

\subsection{Vorhersage der ART}
Die Allgemeine Relativitätstheorie sagt \textbf{keine} frequenzunabhängige Lichtablenkung voraus:
\[
\Delta\theta_{\text{ART}} = \frac{4GM}{c^2 b}
\]
mit Stoßparameter \( b \).

\subsection{Vorhersage der IWT}
Die Informations-Weber-Theorie sagt dagegen \textbf{eine} frequenzabhängige Ablenkung voraus:
\[
\Delta\theta(\nu) = \Delta\theta_0 \left( 1 + \alpha \frac{\nu_0}{\nu} \right)
\]
wobei hochfrequente Photonen weniger abgelenkt werden als niederfrequente.

\subsection{Experimentelle Tests}
\begin{itemize}
    \item \textbf{Sonnenrandbeobachtungen}: Spektral aufgelöste Messungen
    \item \textbf{Gravitationslinsen}: Vergleich optischer, Röntgen- und Radioquellen
    \item \textbf{Pulsar-Timing}: Frequenzabhängige Laufzeitunterschiede
    \item \textbf{Fast Radio Bursts}: Breitbandige Messungen
\end{itemize}

\section{Theorie-Evolution: Von WDBT zu WDBT+}

\subsection{WDBT (analog)}
\begin{itemize}
    \item Fernwirkung ohne Raummodell
    \item Direkte Weber-Kräfte
    \item Keine Gravitationswellen
    \item Reine Dynamik, keine Geometrie
\end{itemize}

\subsection{ART (geometrisch)}
\begin{itemize}
    \item Raumzeit als fundamentales Kontinuum
    \item Geometrische Interpretation der Gravitation
    \item Gravitationswellen als Raumzeit-Krümmungen
    \item Singularitäten in starken Feldern
\end{itemize}

\subsection{ART+ (erweitert)}
\begin{itemize}
    \item ART plus globale Informationsstruktur
    \item Keine echten Singularitäten
    \item Informationsbasierte Regularisierung
    \item Übergangstheorie zur vollen IWT
\end{itemize}

\subsection{WDBT+ (digitale Urtheorie)}
\begin{itemize}
    \item Fundamentales diskretes Informationsnetz
    \item Raum und Zeit emergent
    \item Vereinheitlichung aller Wechselwirkungen
    \item Naturkonstanten als emergente Skalierungsparameter
    \item Vollständige informationsbasierte Physik
\end{itemize}

\section{Zusammenfassung}
Kapitel~1 von Teil II hat gezeigt:

\begin{itemize}
    \item Die IWT ist eine \textbf{Urtheorie}, aus der alle etablierten Theorien als Grenzfälle hervorgehen.
    \item \textbf{Klassische Mechanik} emergiert bei schwachen Informationsgradienten und dominanter lokaler Dynamik.
    \item \textbf{Elektrodynamik} erscheint in drei Stufen: WED (fundamental), Maxwell (Kontinuumsnäherung), Lorentz (phänomenologisch).
    \item \textbf{Quantenmechanik} entsteht bei starker globaler Informationsorganisation und Phasenkohärenz.
    \item \textbf{Relativitätstheorie} emergiert aus der Geometrie großer Informationsnetze und der Symmetrie des Informationsflusses.
    \item Die Theorie macht \textbf{testbare Vorhersagen} wie frequenzabhängige Lichtablenkung.
    \item Die \textbf{Theorie-Evolution} zeigt den Weg von der analogen WDBT zur digitalen WDBT+ als vollständiger Urtheorie.
\end{itemize}

Damit ist der Vergleich mit etablierten Theorien abgeschlossen. Das nächste Kapitel entwickelt die Konsequenzen für Naturkonstanten.