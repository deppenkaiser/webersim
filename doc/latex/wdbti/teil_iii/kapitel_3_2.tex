\chapter{Kosmologie ohne Urknall}
\label{chap:kosmologie-ohne-urknull}

\section{Einleitung}
\label{sec:kosmologie-einleitung}
Die Informations-Weber-Theorie (IWT) etabliert eine radikal alternative kosmologische Perspektive, die ohne die Grundannahmen der Standard-$\lambda$CDM-Kosmologie auskommt.
Anstelle eines expandierenden Universums mit einem heißen Urknall beschreibt die IWT ein stationäres, fraktal strukturiertes Informationsnetzwerk, in dem alle beobachteten
Phänomene aus der Dynamik und Geometrie der Information emergieren.

Die IWT-Kosmologie verzichtet auf:
\begin{itemize}
    \item Einen Anfangszustand (Urknall-Singularität)
    \item Raumzeitexpansion
    \item Inflationäre Phase
    \item Dunkle Energie
    \item Dunkle Materie als separate Entität
\end{itemize}
Stattdessen erklärt sie:
\begin{itemize}
    \item Kosmologische Rotverschiebung als informationsdynamischen Effekt
    \item CMB als thermisches Gleichgewicht eines unendlichen Plasmas
    \item Galaktische Rotation durch fraktale Geometrie
    \item Hubble-Konstante als emergenten Skalierungsparameter
\end{itemize}

\section{Das fraktale Universum als Informationsnetzwerk}
\label{sec:fraktales-universum}

\subsection{Grundstruktur}
Das Universum wird beschrieben als diskretes Informationsnetzwerk mit fraktaler Skaleninvarianz:
\[
\mathcal{U} = \left\{ I_k^{(n)},\, K_{kl}^{(n)},\, g_{kl}^{(n)} \mid k,l \in \mathbb{N} \right\}
\]
Die fraktale Dimension charakterisiert die Skalierungsgesetze:
\[
D = \frac{\ln 20}{\ln(2+\phi)} \approx 2.71
\]

\subsection{Raumemergenz}
Der physikalische Raum ist keine fundamentale Entität, sondern emergiert aus der Metrik des Netzwerks:
\[
d_{kl} = \sqrt{g_{kl}} \cdot \lambda_0, \quad g_{kl} = \frac{K_{kl}}{\sqrt{K_{kk}K_{ll}}}
\]
Auf großen Skalen zeigt die emergente Geometrie die beobachtete dreidimensionale Struktur mit fraktalen Abweichungen auf kleinen Skalen.

\subsection{Stationarität und Energieerhaltung}
Das IWT-Universum ist stationär im Sinne eines Steady-State:
\[
\frac{d}{dt} \sum_{k} I_k = 0, \quad \frac{d}{dt} E_{\text{ges}} = 0
\]
Materie entsteht kontinuierlich durch quantenmechanische Prozesse, während die Gesamtenergie konstant bleibt.

\section{Kosmologische Rotverschiebung ohne Expansion}
\label{sec:rotverschiebung-ohne-expansion-kosmologie}

\subsection{Mechanismus}
Die Rotverschiebung entsteht nicht durch Dopplereffekt in einem expandierenden Raum, sondern durch Energieübertrag von Photonen auf das intergalaktische Plasma:
\[
\frac{dE}{dd} = -\alpha(d) E
\]
mit der ortsabhängigen Verlustrate:
\[
\alpha(d) = \frac{2\gamma_{\text{eff}} G \rho_{\text{eff}} d}{1 + \gamma_{\text{eff}} G \rho_{\text{eff}} d^2}
\]

\subsection{Entfernungs-Rotverschiebungs-Relation}
Aus der Integration folgt die quadratische Relation:
\[
z(d) = \gamma_{\text{eff}} G \rho_{\text{eff}} d^2
\]
Für typische kosmische Dichten $\rho_{\text{eff}} \approx 4 \times 10^{-28}\,\text{kg/m}^3$ und $\gamma_{\text{eff}} \approx 4 \times 10^{-14}$ ergibt sich:
\[
z \approx \left( \frac{d}{1\,\text{Gpc}} \right)^2
\]

\subsection{Testbare Vorhersagen}
\begin{enumerate}
    \item \textbf{Nichtlinearität}: Die $z$-$d$-Relation ist quadratisch, nicht linear
    \item \textbf{Zeitdilatation}: Abweichungen von der bei Supernovae erwarteten Zeitdilatation
    \item \textbf{Evolutionseffekte}: Unterschiedliche $z$-$d$-Relationen für verschiedene Objekttypen
\end{enumerate}

\section{CMB als thermisches Plasma-Gleichgewicht}
\label{sec:cmb-thermisches-gleichgewicht}

\subsection{Gleichgewichtsmechanismus}
Die kosmische Mikrowellenhintergrundstrahlung entsteht aus dem thermischen Gleichgewicht zwischen:
\begin{itemize}
    \item \textbf{Heizung}: Energieeintrag durch Rotverschiebungseffekte
    \item \textbf{Abstrahlung}: Thermische Emission des intergalaktischen Plasmas
\end{itemize}

\subsection{Temperaturberechnung}
Die Gleichgewichtstemperatur folgt aus:
\[
T_{\text{CMB}} = \left( \frac{\bar{\alpha}(L) u_\gamma}{\varepsilon A_{\text{eff}} \sigma} \right)^{1/4}
\]
mit der mittleren Verlustkonstante:
\[
\bar{\alpha}(L) = \frac{1}{L \gamma_{\text{eff}} G \rho_{\text{eff}}} \ln\left(1 + \gamma_{\text{eff}} G \rho_{\text{eff}} L^2\right)
\]
Einsetzen realistischer Parameter liefert:
\[
T_{\text{CMB}} \approx 2.7\,\text{K}
\]

\subsection{Anisotropiestruktur}
Die beobachteten CMB-Fluktuationen spiegeln die fraktale Netzwerkstruktur wider:
\[
\frac{\Delta T}{T}(\theta) \propto \theta^{-(3-D)/2} \quad \text{mit } D \approx 2.71
\]

Spezifische Vorhersagen:
\begin{itemize}
    \item Fraktale nicht-gaußsche Statistik
    \item Anisotropien bei großen Winkeln
    \item Korrelationsfunktion mit fraktaler Skalierung
\end{itemize}

\section{Galaktische Dynamik ohne Dunkle Materie}
\label{sec:galaxien-ohne-dunkle-materie}

\subsection{Rotationskurven}
Die flachen Rotationskurven von Galaxien ergeben sich aus der fraktalen Geometrie:
\[
v_{\text{circ}}(r) = \sqrt{\frac{GM(r)}{r}} \cdot f(r), \quad f(r) = \left( \frac{r}{r_0} \right)^{(3-D)/2}
\]
Für $r \gg r_0$ und $D \approx 2.71$:
\[
v_{\text{circ}}(r) \approx \text{konstant}
\]

\subsection{Tully-Fisher-Relation}
Die beobachtete Relation $L \propto v_{\text{max}}^4$ folgt aus:
\[
L \propto M_{\text{baryon}} \propto \left( \sum_k I_k \right)^2 \propto v_{\text{max}}^{2(3-D)} \approx v_{\text{max}}^{4}
\]

\subsection{Masse-zu-Licht-Verhältnis}
Das scheinbar erhöhte Masse-zu-Licht-Verhältnis in Galaxienhaufen entsteht durch:
\[
\left( \frac{M}{L} \right)_{\text{eff}} = \left( \frac{M}{L} \right)_{\text{baryon}} \cdot \left( 1 + \beta(D) \right)
\]
mit $\beta(D) \approx 0.5$ für $D \approx 2.71$.

\section{Hubble-Konstante als emergenter Parameter}
\label{sec:hubble-emergent}

\subsection{Herleitung}
Vergleich der IWT-Rotverschiebungsformel mit der konventionellen Näherung $z \approx H_0 d/c$ liefert:
\[
H_0 = c \sqrt{\gamma_{\text{eff}} G \rho_{\text{eff}}}
\]
Einsetzen der Werte ergibt:
\[
H_0 \approx 70\,\text{km/s/Mpc}
\]

\subsection{Natürliche Erklärung der Hubble-Spannung}
Die beobachtete Diskrepanz zwischen lokalen und kosmologischen $H_0$-Messungen erklärt sich durch:
\begin{itemize}
    \item \textbf{Lokale Messungen}: Sensitiv auf $\rho_{\text{eff}}$ in Galaxienhaufen
    \item \textbf{Kosmologische Messungen}: Sensitiv auf $\rho_{\text{eff}}$ im großskaligen Netzwerk
    \item \textbf{Diskrepanz}: Unterschiedliche effektive Dichten in verschiedenen Skalenbereichen
\end{itemize}

\section{Mach-Prinzip und kosmische Skalen}
\label{sec:mach-prinzip}

\subsection{Trägheit aus Informationskopplung}
Die träge Masse eines Teilchens entsteht durch seine Kopplung an das gesamte kosmische Informationsnetz:
\[
m c^2 = \sum_{l \neq k} K_{kl} I_k I_l
\]

\subsection{Mach-Radius}
Der charakteristische kosmische Skalenradius folgt aus:
\[
R = \sqrt{\frac{c^2}{2 \kappa_M G \rho_{\text{eff}}}}
\]
mit der fraktalen Mach-Konstante:
\[
\kappa_M(D) = \frac{2\pi D}{3(D-1)} \approx 3.3 \quad \text{für } D \approx 2.71
\]

\section{Konsequenzen für JWST-Beobachtungen}
\label{sec:jwst-konsequenzen}

\subsection{Hohe Rotverschiebungen ohne frühes Universum}
Die vom James Webb Space Telescope beobachteten Galaxien mit $z > 10$ sind vollständig kompatibel mit der IWT:
\[
d(z=20) = \sqrt{\frac{z}{\gamma_{\text{eff}} G \rho_{\text{eff}}}} \approx 4.5\,\text{Gpc}
\]
Diese Entfernungen erfordern weder ein extrem junges Universum noch unphysikalisch schnelle Sternentstehungsraten.

\subsection{Strukturbildung}
Die beobachtete frühe Strukturbildung erklärt sich durch:
\begin{itemize}
    \item Fraktale Anfangsbedingungen des Informationsnetzwerks
    \item Schnelle gravitative Kollapszeiten im stationären Universum
    \item Natürliche Skalenhierarchie durch fraktale Dimension $D \approx 2.71$
\end{itemize}

\section{Testbare Vorhersagen und Falsifizierungsmöglichkeiten}
\label{sec:kosmologie-tests}

\subsection{Quantitative Vorhersagen}
\begin{table}[ht]
\centering
\begin{tabular}{p{0.35\textwidth}p{0.3\textwidth}p{0.25\textwidth}}
\hline
\textbf{Phänomen} & \textbf{IWT-Vorhersage} & \textbf{Standard-$\lambda$CDM} \\
\hline
CMB-Temperatur & $T = \left( \frac{\bar{\alpha} u_\gamma}{\varepsilon A_{\text{eff}}\sigma} \right)^{1/4}$ & Urknall-Relikt \\
\hline
$z$-$d$-Relation & $z \propto d^2$ & $z \propto d$ (linear) \\
\hline
Rotationskurven & $v(r) \propto r^{-(3-D)/2}$ & $v(r) \propto r^{-1/2}$ (mit DM) \\
\hline
CMB-Anisotropien & Fraktale Skalierung & Gaußsche Statistik \\
\hline
Hubble-Konstante & $H_0 = c\sqrt{\gamma G\rho}$ & Freier Parameter \\
\hline
\end{tabular}
\caption{Vergleich kosmologischer Vorhersagen}
\end{table}

\subsection{Experimentelle Tests}
\begin{enumerate}
    \item \textbf{Präzisions-CMB-Messungen}: Test fraktaler nicht-gaußscher Statistik
    \item \textbf{Supernovae-Beobachtungen}: Test der quadratischen $z$-$d$-Relation
    \item \textbf{Galaxien-Rotationskurven}: Vergleich mit fraktaler Vorhersage
    \item \textbf{JWST-Beobachtungen}: Test der Entfernungs-Rotverschiebungs-Relation
    \item \textbf{21-cm-Kosmologie}: Test des intergalaktischen Mediums
\end{enumerate}

\section{Zusammenfassung und Ausblick}
\label{sec:kosmologie-zusammenfassung}
Die IWT-Kosmologie bietet eine vollständige alternative Beschreibung des Universums:

\subsection{Kernaussagen}
\begin{itemize}
    \item Das Universum ist ein stationäres, fraktales Informationsnetzwerk
    \item Raum und Zeit emergieren aus der Netzwerkstruktur
    \item Rotverschiebung entsteht durch Plasma-Energieübertrag, nicht Expansion
    \item CMB ist thermisches Gleichgewicht, nicht Urknall-Relikt
    \item Galaktische Dynamik erklärt sich durch fraktale Geometrie, nicht Dunkle Materie
    \item Hubble-Konstante ist emergent, kein fundamentaler Parameter
\end{itemize}

\subsection{Vorteile gegenüber der Standardkosmologie}
\begin{itemize}
    \item \textbf{Einfachheit}: Weniger freie Parameter
    \item \textbf{Natürlichkeit}: Keine feinabgestimmten Anfangsbedingungen
    \item \textbf{Testbarkeit}: Spezifische, falsifizierbare Vorhersagen
    \item \textbf{Konsistenz}: Vereinheitlichung mit Quantenphysik und Informationstheorie
\end{itemize}

\subsection{Offene Forschungsfragen}
\begin{enumerate}
    \item Quantitative Herleitung aller kosmologischen Parameter aus Netzwerkeigenschaften
    \item Vollständige numerische Simulation der fraktalen Kosmologie
    \item Präzise Berechnung der CMB-Anisotropiestatistik
    \item Entwicklung spezifischer Tests zur Unterscheidung von $\lambda$CDM
\end{enumerate}

\subsection{Abschließende Bemerkung}
Die IWT-Kosmologie stellt nicht nur eine Alternative zum $\lambda$CDM-Modell dar, sondern zeigt, wie kosmologische Phänomene natürlich aus einer fundamentalen
Informationsdynamik emergieren können. Ihre Vorhersagen sind klar, testbar und bieten die Möglichkeit, die Kosmologie auf eine neue, informationsbasierte Grundlage zu
stellen.