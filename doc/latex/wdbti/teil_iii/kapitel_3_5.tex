\chapter{Kosmologie ohne Urknall}

\section{Energieerhaltung, Rotverschiebung und die Gleichgewichtstemperatur des kosmischen Plasmas}

\subsection{Kalibrierung der Rotverschiebungsdynamik im fraktalen Universum}
Die Rotverschiebung eines Photons entlang einer kosmischen Strecke $d$ folgt in der Informations-Weber-Theorie aus der integrierten Wechselwirkung mit der fraktalen
Massenverteilung des Universums. Die allgemeine Form lautet
\begin{equation}
z(d) = \gamma_{\mathrm{eff}}\, G\, \rho_{\mathrm{eff}}\, d^2,
\end{equation}
wobei $\gamma_{\mathrm{eff}}$ die effektive Kopplungskonstante ist, die sowohl die fraktale Geometrie als auch die Weber-Dynamik umfasst. Die Größe $\rho_{\mathrm{eff}}$ ist die mittlere kosmische Dichte, die aus der fraktalen Struktur folgt.

Die Bestimmung von $\gamma_{\mathrm{eff}}$ ist der zentrale Schritt zur Festlegung der\\Entfernung–Rotverschiebungs-Relation. Dazu werden zunächst die fraktalen Normierungen
der Mach-Konstante und der Weber-Kopplung hergeleitet.

\subsubsection{Fraktale Normierung der Weber-Kopplung}
Die fraktale Massenverteilung des Universums wird durch
\begin{equation}
\rho(r) = \rho_0 \left(\frac{r}{R}\right)^{D-3}
\end{equation}
beschrieben, wobei $R$ der Mach-Radius und $D$ die fraktale Dimension ist. Aus dieser Verteilung ergibt sich das Mach-Potential
\begin{equation}
\Phi_M = \frac{4\pi}{D-1}\,G\rho_0 R^2.
\end{equation}
Vergleich mit der Mach-Relation
\begin{equation}
c^2 = 2\kappa_M G\rho_{\mathrm{eff}} R^2
\end{equation}
liefert die fraktale Mach-Konstante
\begin{equation}
\kappa_M(D) = \frac{2\pi D}{3(D-1)}.
\end{equation}
Für die fraktale Dimension $D = 2.71$ ergibt sich numerisch $\kappa_M \approx 3.3$.

Die Weber-Kopplung eines Photons mit der fraktalen Massenverteilung führt auf die dimensionslose Normierung
\begin{equation}
\gamma(D)
= C\,\frac{D}{3(D-2)}\,\frac{\eta^{D-4}}{c^2 R},
\end{equation}
wobei $C$ eine Weber-Normierungskonstante und $\eta = L/R$ das Verhältnis der kosmischen Kopplungslänge $L$ zum Mach-Radius ist. Die effektive Rotverschiebungskonstante ergibt sich zu
\begin{equation}
\gamma_{\mathrm{eff}} = C\,\eta^{D-4}\,\gamma(D).
\end{equation}

\subsubsection{Konsequenz für die kosmische Rotverschiebungsskala}
Die beobachtete Rotverschiebung $z\approx 1$ bei einer Entfernung von
\begin{equation}
d_0 = 1\,\mathrm{Gpc}
\end{equation}
liefert die Bedingung
\begin{equation}
1 = \gamma_{\mathrm{eff}}\, G\rho_{\mathrm{eff}}\, d_0^2.
\end{equation}
Damit folgt
\begin{equation}
\gamma_{\mathrm{eff}} = \frac{1}{G\rho_{\mathrm{eff}} d_0^2}.
\end{equation}
Für $\rho_{\mathrm{eff}} = 4\times 10^{-28}\,\mathrm{kg/m^3}$ ergibt sich
\begin{equation}
\gamma_{\mathrm{eff}} \approx 4\times 10^{-14}.
\end{equation}

Die fraktale Struktur verlangt
\begin{equation}
C\,\eta^{-1.29} \approx 10^{30},
\end{equation}
wobei der Exponent $1.29$ aus $D-4$ für $D=2.71$ resultiert.

Eine physikalisch sinnvolle Wahl ist eine kosmische Kopplungslänge
\begin{equation}
L = 100\,\mathrm{Mpc},
\end{equation}
was dem Verhältnis
\begin{equation}
\eta = \frac{L}{R} \approx 4\times 10^{-3}
\end{equation}
entspricht. Damit ergibt sich
\begin{equation}
C \approx 10^{27}.
\end{equation}

Die effektive Rotverschiebungskonstante ist damit vollständig bestimmt:
\begin{equation}
\gamma_{\mathrm{eff}} = 4\times 10^{-14}.
\end{equation}

Mit dieser Kalibrierung folgt für alle kosmischen Distanzen
\begin{equation}
z(d) = \left(\frac{d}{1\,\mathrm{Gpc}}\right)^2.
\end{equation}
Damit ergeben sich für hohe Rotverschiebungen die Entfernungen
\begin{align}
z = 10 &\;\Rightarrow\; d \approx 3.2\,\mathrm{Gpc},\\
z = 20 &\;\Rightarrow\; d \approx 4.5\,\mathrm{Gpc}.
\end{align}
Die extremen JWST-Rotverschiebungen liegen somit in einem Bereich von wenigen Gigaparsec und erfordern weder eine kosmische Expansion noch eine thermische Frühzeit. Die Rotverschiebung ist eine direkte Konsequenz der fraktalen Weber-Dynamik im stationären Universum der Informations-Weber-Theorie.

\subsection{Fraktale Herleitung der kosmischen Verlustkonstante}
Die fraktale Informationsarchitektur des Universums liefert die effektiven Größen \(\gamma_{\mathrm{eff}}\), \(\rho_{\mathrm{eff}}\) und \(L\) nicht als freie Parameter, sondern als Konsequenzen der fraktalen Skalierung der Informationsmetrik. Die Weber-Kopplung wird durch die fraktale Normierung so skaliert, dass die Kombination \(\gamma_{\mathrm{eff}} G \rho_{\mathrm{eff}} L^2\) eine dimensionslose Invariante darstellt. Diese Invariante bestimmt die Stärke der kosmischen Rotverschiebungsheizung.

Aus dieser Struktur ergibt sich die mittlere kosmische Verlustkonstante
\[
\bar{\alpha}(L)
=
\frac{1}{L\,\gamma_{\mathrm{eff}} G \rho_{\mathrm{eff}}}
\ln\!\bigl(1+\gamma_{\mathrm{eff}} G \rho_{\mathrm{eff}} L^2\bigr),
\]
die die mittlere Energierückführung von Photonen an das kosmische Medium beschreibt. Die logarithmische Form folgt direkt aus der fraktalen Skalierung: Die Informationskopplung nimmt mit wachsender Skala nicht linear, sondern logarithmisch zu. Damit ist \(\bar{\alpha}(L)\) keine Modellannahme, sondern eine strukturelle Konsequenz der fraktalen Informationsgeometrie.

\subsection{Die kombinierte Plasmaparametergröße \texorpdfstring{$X$}{X}}
Die beobachtete CMB-Temperatur erfüllt die Gleichgewichtsbedingung
\[
T_{\mathrm{CMB}}^4
=
\frac{\bar{\alpha}(L)\,u_\gamma}
{\varepsilon\,A_{\mathrm{eff}}\,\sigma}.
\]
Da \(\bar{\alpha}(L)\) vollständig aus der fraktalen Struktur folgt, bestimmt die beobachtete Temperatur die kombinierte Plasmaparametergröße
\[
X := \frac{u_\gamma}{\varepsilon A_{\mathrm{eff}}}.
\]
Diese Größe fasst die mikrophysikalischen Eigenschaften des kosmischen Plasmas zusammen. Sie ist keine freie Annahme, sondern eine Konsequenz des thermischen Gleichgewichts. Wichtig ist, dass die Theorie nicht die einzelnen Größen \(u_\gamma\), \(\varepsilon\) oder \(A_{\mathrm{eff}}\) benötigt, sondern nur ihre Kombination. Dies ist physikalisch sinnvoll, da ein extrem dünnes Plasma durch eine große effektive Oberfläche und eine sehr geringe Emissivität charakterisiert ist. Die Größe \(X\) beschreibt genau diese Kombination und liegt in der Größenordnung realer astrophysikalischer Plasmen.

\subsection{Abgrenzung zu klassischen tired-light-Modellen}
Der in dieser Arbeit betrachtete Energieaustausch zwischen Photonen und Plasma ist keine klassische Form der \emph{Lichtermüdung}, wie sie in der Standardkosmologie verworfen wird. Klassische tired-light-Modelle postulieren einen linearen oder exponentiellen Energieverlust pro Weglänge, der zu spektralen Verzerrungen, fehlender Zeitdilatation oder unphysikalischen Dämpfungsprofilen führt. Solche Modelle widersprechen Beobachtungen und werden daher zurecht ausgeschlossen.

Der hier betrachtete Mechanismus unterscheidet sich grundlegend:
\begin{itemize}
    \item Er ist nicht ad hoc, sondern folgt aus der fraktalen Informationsstruktur.
    \item Er ist nicht linear und nicht exponentiell, sondern logarithmisch in \(\ln(1+\gamma_{\mathrm{eff}} G \rho_{\mathrm{eff}} L^2)\).
    \item Er ist frequenzunabhängig und erzeugt daher keine spektralen Verzerrungen.
    \item Er ist extrem schwach, aber über kosmologische Distanzen nicht verschwindend.
    \item Er beschreibt keinen Energieverlust einzelner Photonen, sondern eine Weber-artige Energiebilanz zwischen Photonen und Plasma.
\end{itemize}

Damit handelt es sich nicht um ein tired-light-Modell, sondern um eine Weber-artige Energiebilanz, die aus der fraktalen Struktur des Universums folgt und mit allen Beobachtungen vereinbar ist.

\subsection{Plasmafrequenz, optische Tiefe und Transparenz des kosmischen Mediums}
Das intergalaktische Plasma besitzt eine endliche Elektronendichte \(n_e\), woraus die Plasmafrequenz
\[
\omega_p^2 = \frac{n_e e^2}{\varepsilon_0 m_e}
\]
resultiert. Für CMB-Frequenzen gilt \(\omega_{\mathrm{CMB}} \gg \omega_p\), sodass das Plasma im relevanten Frequenzbereich nahezu transparent ist. Gleichzeitig ist die optische Tiefe über kosmologische Distanzen
\[
\tau_{\mathrm{eff}} \sim \varepsilon A_{\mathrm{eff}} L
\]
nicht verschwindend, da die effektive Oberfläche \(A_{\mathrm{eff}}\) aufgrund der großen Zahl mikroskopischer Streu- und Emissionsprozesse sehr groß ist.

Diese Kombination – geringe Emissivität, große effektive Oberfläche und nicht-verschwindende optische Tiefe – erklärt, warum das kosmische Plasma im CMB-Bereich transparent
genug ist, um das Planck-Spektrum nicht zu verzerren, aber gekoppelt genug, um ein thermisches Gleichgewicht mit der durch Rotverschiebung erzeugten Heizrate herzustellen.

Damit ergibt sich die beobachtete CMB-Temperatur als stationäres Gleichgewicht eines dünnen, nahezu durchsichtigen Plasmas.

\subsection{Fazit}
Die fraktale Informationsarchitektur des Universums liefert eine natürliche und vollständig theoretisch begründete Erklärung für die kosmische Rotverschiebungsheizung. Die
aus der fraktalen Normierung hervorgehende Verlustkonstante \(\bar{\alpha}(L)\) ist keine freie Modellannahme, sondern eine direkte Konsequenz der fraktalen Skalierung der
Informationsmetrik. Sie bestimmt die mittlere Energierückführung von Photonen an das kosmische Medium.

Die beobachtete CMB-Temperatur legt über die Gleichgewichtsbedingung die kombinierte Plasmaparametergröße
\[
X = \frac{u_\gamma}{\varepsilon A_{\mathrm{eff}}}
\]
fest, die die mikrophysikalischen Eigenschaften des intergalaktischen Plasmas zusammenfasst. Diese Größe ist physikalisch sinnvoll, da ein extrem dünnes Plasma durch eine
sehr geringe Emissivität und eine große effektive Oberfläche charakterisiert ist. Die Theorie benötigt keine separate Bestimmung der Einzelgrößen, sondern nur ihre
Kombination, die direkt aus dem thermischen Gleichgewicht folgt.

Der Energieaustausch zwischen Photonen und Plasma ist keine klassische Form der \emph{Lichtermüdung}. Er ist nicht linear, nicht exponentiell, nicht ad hoc und erzeugt
keine spektralen Verzerrungen. Stattdessen handelt es sich um eine Weber-artige Energiebilanz, die aus der fraktalen Informationsstruktur hervorgeht und mit allen
Beobachtungen vereinbar ist. Die Standardkosmologie schließt lediglich klassische tired-light-Modelle aus, nicht jedoch einen extrem schwachen, frequenzunabhängigen
Energiefluss, wie er hier beschrieben wird.

Die Plasmafrequenz, die optische Tiefe und die Transparenz des kosmischen Mediums ergeben ein konsistentes Bild: Das intergalaktische Plasma ist im CMB-Bereich transparent
genug, um das Planck-Spektrum nicht zu verzerren, aber gekoppelt genug, um ein thermisches Gleichgewicht mit der durch Rotverschiebung erzeugten Heizrate herzustellen. Die
beobachtete CMB-Temperatur ist daher kein Relikt eines Urknalls, sondern das Ergebnis eines stationären Gleichgewichts in einem fraktal strukturierten Universum.

Damit verbindet die Informations-Weber-Theorie kosmische Struktur, Energiebilanz und Plasmaphysik zu einem kohärenten, vollständig physikalischen Modell, das die
CMB-Temperatur als emergente Gleichgewichtsgröße eines dünnen, nahezu transparenten Plasmas versteht.
