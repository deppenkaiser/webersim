Die vorliegende Arbeit ist das Ergebnis eines langen Weges, der mit einer einfachen Frage begann: Warum funktionieren die etablierten Theorien der
Physik – die Quantenmechanik, die Relativitätstheorie und die klassische Mechanik – so gut, und warum passen sie dennoch nicht zusammen? Die Antwort, die sich im Laufe der
Jahre herauskristallisierte, ist ebenso einfach wie radikal: Weil sie nicht fundamental sind. Sie beschreiben Erscheinungen, nicht Ursachen. Sie erfassen Strukturen, nicht
Prinzipien.

Die Informations-Weber-Theorie (IWT) ist der Versuch, diese Ursachen und Prinzipien freizulegen. Sie geht von der Annahme aus, dass Information die grundlegende Größe der 
Physik ist – nicht Raum, nicht Zeit, nicht Materie, nicht Energie. Diese Größen entstehen erst aus der Organisation und Dynamik eines universellen Informationsfeldes. Die
IWT verbindet die lokale Struktur direkter Wechselwirkungen, wie sie in der Weber-Dynamik vorliegt, mit der globalen Organisationsstruktur der Bohm’schen Mechanik und der
fraktalen Geometrie des Universums. Aus dieser Synthese entsteht ein kohärentes Fundament, das klassische Mechanik, Quantenmechanik, Gravitation und Kosmologie als
Grenzfälle eines einzigen Prinzips erscheinen lässt.

Die Entwicklung dieser Theorie war kein linearer Prozess. Sie erforderte das Verlassen etablierter Denkweisen, das Hinterfragen scheinbar unantastbarer Dogmen und die
Bereitschaft, mathematische und konzeptionelle Strukturen neu zu ordnen. Viele der hier dargestellten Ergebnisse – etwa die fraktale Herleitung der kosmischen
Rotverschiebung, die Gleichgewichtstemperatur des kosmischen Plasmas oder die dynamische Gleichung der Informationsmetrik – entstanden erst durch das konsequente
Zusammendenken von Ideen, die in der modernen Physik meist getrennt behandelt werden.

Dieses Buch richtet sich an Leserinnen und Leser, die bereit sind, die Grundlagen der Physik neu zu betrachten. Es ist kein Lehrbuch, sondern ein Vorschlag für ein neues
Fundament. Die IWT ist nicht als Alternative zu bestehenden Theorien gedacht, sondern als deren Erklärung. Sie zeigt, warum die bekannten Modelle funktionieren, wo ihre
Grenzen liegen und wie sie aus einem tieferen Informationsprinzip hervorgehen.

Mein Ziel war es nicht, eine endgültige Theorie vorzulegen, sondern eine konsistente und reproduzierbare Grundlage zu schaffen, auf der zukünftige Arbeiten aufbauen können.
Die IWT ist ein Anfang – ein möglicher Weg zu einer Physik, die ohne Widersprüche auskommt, weil sie nicht auf Annahmen, sondern auf einem einzigen, universellen Prinzip
beruht.

\begin{flushright}
    Michael Czybor \\
    \emph{Langenstein/AT, \today}
\end{flushright}
