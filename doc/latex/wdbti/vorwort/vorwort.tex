Dieses Werk stellt die konsequente Weiterentwicklung der Weber--De-Broglie--Bohm-Theorie dar und hebt sie auf eine neue fundamentale Ebene: die Formulierung der Physik als 
\emph{Informationsdynamik}. Information wird als primäre physikalische Größe verstanden, deren Erhaltung und Transformation die bekannten Gesetze der Mechanik, der Quantenphysik 
und der Gravitation hervorbringen.

Die klassische Weber-Theorie liefert die lokale Struktur direkter Wechselwirkungen. Die Bohm’sche Mechanik ergänzt diese Struktur um eine globale, nichtlokale 
Organisationsdynamik. Die analoge WDBT vereinigt beide Ansätze zu einer Fernwirkungstheorie ohne Raummodell, die bereits wesentliche Phänomene wie die 
Periheldrehung oder die Quantenstruktur korrekt beschreibt.

Die Allgemeine Relativitätstheorie (ART) steht historisch und logisch zwischen dieser analogen WDBT und der informationsbasierten Theorie: Sie übernimmt die gravitative 
Dynamik der Weber-Struktur, ersetzt jedoch die Fernwirkung durch eine geometrische Raumzeit und ermöglicht damit Gravitationswellen. Gleichzeitig fehlt der ART die 
nichtlokale Informationsstruktur des Bohm-Potentials.

Wird die ART um diese informationsbasierte Quantenstruktur ergänzt, entsteht eine erweiterte Theorie -- \emph{ART+} -- in der echte Singularitäten verschwinden und der 
Urknall durch einen \emph{Big Bounce} ersetzt wird. Die vollständige informationsbasierte Theorie, die in diesem Buch entwickelt wird, geht jedoch über ART und ART+ hinaus: 
Die digitale Informations-Weber-Theorie (WDBT+) führt ein diskretes Informationsnetz ein, aus dem Raum, Zeit, Dynamik und Naturkonstanten emergieren. In dieser Struktur werden 
Gravitationswellen, Rotationskurven, die CMB-Anisotropien und die Werte der Naturkonstanten nicht postuliert, sondern als Konsequenzen der Informationsarchitektur 
verstanden.

Dieses Buch verfolgt daher zwei Ziele: Erstens zeigt es, dass die Weber--De-Broglie--Bohm-Theorie eine konsistente Grundlage für Gravitation und Quantenmechanik bildet. Zweitens entwickelt 
es eine informationsbasierte Urtheorie, in der Energie, Raum, Zeit und Dynamik als abgeleitete Größen erscheinen. Die Informations-Weber-Theorie verbindet direkte 
Wechselwirkungen, nichtlokale Quantenstruktur und fraktale Raumgeometrie zu einem kohärenten, widerspruchsfreien Fundament.

\begin{flushright}
    Michael Czybor \\
    \emph{Langenstein/AT, \today}
\end{flushright}
