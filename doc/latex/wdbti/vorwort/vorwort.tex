Dieses Werk stellt die konsequente Weiterentwicklung der Weber--De-Broglie--Bohm-Theorie dar und hebt sie auf eine neue fundamentale Ebene: die Formulierung der Physik als 
\emph{Informationsdynamik}. Information wird als primäre physikalische Größe verstanden, deren Erhaltung und Transformation die bekannten Gesetze der Mechanik, der
Quantenphysik und der Gravitation hervorbringen.

Die klassische Weber-Theorie liefert die lokale Struktur direkter Wechselwirkungen. Die Bohm’sche Mechanik ergänzt diese Struktur um eine globale, nichtlokale
Organisationsdynamik. Die analoge WDBT vereinigt beide Ansätze zu einer Fernwirkungstheorie ohne Raummodell, die bereits wesentliche Phänomene wie die Periheldrehung oder
die Quantenstruktur korrekt beschreibt.

Die Allgemeine Relativitätstheorie (ART) steht historisch und logisch zwischen dieser analogen WDBT und der informationsbasierten Theorie: Sie übernimmt die gravitative
Dynamik der Weber-Struktur, ersetzt jedoch die Fernwirkung durch eine geometrische Raumzeit und ermöglicht damit Gravitationswellen. Gleichzeitig fehlt der ART die
nichtlokale Informationsstruktur des Bohm-Potentials. Die ART ist damit eine geometrische Näherung, die im schwachen Feld ausgezeichnet funktioniert, im starken Feld
jedoch zu Singularitäten führt.

Wird die ART um diese informationsbasierte Quantenstruktur ergänzt, entsteht eine erweiterte Theorie -- \emph{ART+} -- in der echte Singularitäten verschwinden und der
Urknall durch einen \emph{Big Bounce} ersetzt wird. Die vollständige informationsbasierte Theorie, die in diesem Buch entwickelt wird, geht jedoch über ART und ART+
hinaus: Die digitale Informations-Weber-Theorie (WDBT+) führt ein diskretes Informationsnetz ein, aus dem Raum, Zeit, Dynamik und Naturkonstanten emergieren. In dieser
Struktur werden Gravitationswellen, Rotationskurven, die CMB-Anisotropien und die Werte der Naturkonstanten nicht postuliert, sondern als Konsequenzen der
Informationsarchitektur verstanden.

Dieses Buch verfolgt daher zwei Ziele: Erstens zeigt es, dass die Weber--De-Broglie--Bohm-Theorie eine konsistente Grundlage für Gravitation und \gls{qm} bildet.
Zweitens entwickelt es eine informationsbasierte Urtheorie, in der Energie, Raum, Zeit und Dynamik als abgeleitete Größen erscheinen. Die Informations-Weber-Theorie
verbindet direkte Wechselwirkungen, nichtlokale Quantenstruktur und fraktale Raumgeometrie zu einem kohärenten, widerspruchsfreien Fundament.

Ein besonderer Wendepunkt in der Entwicklung dieser Theorie war die Entdeckung des \(\beta\)-Parameters in der Weber-Gravitation. Die Erkenntnis, dass der Wert
\(\beta = 0.5\) nicht nur mathematisch konsistent ist, sondern die korrekten astronomischen Beobachtungen (Periheldrehung, Bahnstabilität, Rotverschiebung) ohne
Raumzeitkrümmung reproduziert, bildet das Fundament der gesamten Arbeit. Ohne diesen Parameter wäre weder die Synthese von Weber-Gravitation und
De-Broglie-Bohm-Theorie möglich, noch die informationsbasierte Erweiterung, die in diesem Buch entwickelt wird. Die \(\beta = 0.5\)-Struktur ist keine freie Modellannahme,
sondern eine strukturelle Invariante der Natur, die die Schwächen der ART im starken Feld überwindet und eine konsistente, singularitätsfreie Gravitation ermöglicht. Diese 
Entdeckung war der Ausgangspunkt für das gesamte hier vorliegende Werk.

\begin{flushright}
    Michael Czybor \\
    \emph{Langenstein/AT, \today}
\end{flushright}
