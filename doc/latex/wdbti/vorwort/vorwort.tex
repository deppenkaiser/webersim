Die moderne Physik befindet sich in einer paradoxen Lage. Einerseits verfügt sie über zwei der erfolgreichsten Theorien der Wissenschaftsgeschichte: die \gls{qm} und die
\gls{art}. Beide liefern präzise Vorhersagen, beide sind experimentell bestätigt, und doch stehen sie in einem fundamentalen Widerspruch zueinander. Die eine beschreibt die
Welt als nichtlokale, konfigurationsraumweite Wellenlandschaft, die andere als lokale Geometrie einer gekrümmten Raumzeit. Beide funktionieren, aber sie können nicht
gleichzeitig fundamental sein. Seit über einem Jahrhundert versucht die Physik, diese Spannung zu überbrücken, doch die vorgeschlagenen Lösungen – Felder, Krümmungen,
Renormierungen, Zusatzdimensionen, Strings, Schleifen, Inflationsfelder, dunkle Materie und dunkle Energie – haben die Lage eher verkompliziert als geklärt. Je mehr
Zusatzannahmen eingeführt werden, desto deutlicher wird, dass die etablierten Theorien nicht falsch sind, aber nicht fundamental.

Dieses Werk stellt eine alternative Sichtweise vor: die \gls{iwt}, die zusammen mit der \gls{wdbt} und der \gls{dstt} ein konsistentes, skalenübergreifendes Fundament bildet.
Diese Theorie ist kein weiterer Versuch, bestehende Modelle zu modifizieren oder zu erweitern. Sie ist ein Neuansatz, der die Grundannahmen der modernen Physik hinterfragt
und durch ein einfacheres, klareres und konzeptionell einheitliches Fundament ersetzt. Die \gls{iwt} geht von einem radikal anderen Prinzip aus: Information ist fundamental.
Nicht Raum, nicht Zeit, nicht Materie, nicht Energie, sondern Information und ihre Kopplung bilden die Grundlage aller physikalischen Prozesse.

Aus dieser diskreten Informationsdynamik emergieren Raum als fraktale Geometrie, Zeit als Schrittindex eines globalen Update-Prozesses, Trägheit als lokale
Informationsstruktur, Gravitation als globaler Informationsfluss, Quantenphänomene als nichtlokale Organisation und Naturkonstanten als Skalierungsparameter. Die
beobachtbare Physik ist eine kontinuierliche Näherung dieser tieferen Struktur. Die \gls{wdbt} bildet die mittlere Ebene dieser Architektur. Sie zeigt,
dass direkte Wechselwirkungen und Bohm'sche Nichtlokalität keine Gegensätze sind, sondern zwei Aspekte desselben Informationsprozesses. Das Quantenpotential ist kein
mysteriöser Zusatzterm, sondern der globale Anteil der Informationsdynamik. Die Weber-Kraft ist der lokale Anteil derselben Struktur. Zusammen ergeben sie eine kohärente,
deterministische, nichtlokale Theorie, die sowohl die \gls{qm} als auch die klassische Elektrodynamik umfasst.

Die \gls{dstt} schließlich ist der makroskopische Gravitationssektor dieser Ur-Theorie. Sie zeigt, dass Gravitation nicht als Krümmung der Raumzeit verstanden werden muss,
sondern als dynamische Aufteilung der Schwerewirkung in radiale und tangentiale Trägheitsantworten. Der zentrale Parameter dieser Theorie, der Umlenkungszustand $\beta(t)$,
ist die makroskopische Version des Bohm'schen Quantenpotentials. Seine monotone Entwicklung erklärt planetare Migration, Spiralbahnen, Lichtablenkung, Rotverschiebung und
galaktische Rotationskurven ohne dunkle Materie, ohne dunkle Energie und ohne expandierende Raumzeit.

Diese drei Ebenen – \gls{iwt}, \gls{wdbt} und \gls{dstt} – bilden zusammen eine Ur-Theorie, die ohne Widersprüche auskommt, ohne Singularitäten, ohne Renormierung und ohne
unphysikalische Zusatzannahmen. Sie ist deterministisch, nichtlokal, fraktal, informationsbasiert und skalenübergreifend konsistent. Sie erklärt nicht nur die bekannten
Phänomene der Physik, sondern auch, warum die etablierten Theorien funktionieren und wo ihre Grenzen liegen. Die Stärke dieser Theorie liegt nicht in mathematischer Eleganz
um ihrer selbst willen, sondern in konzeptioneller Klarheit. Sie zeigt, dass die Natur nicht aus Feldern besteht, die sich im leeren Raum ausbreiten, sondern aus
Informationsflüssen, die ein fraktales Netzwerk strukturieren. Sie zeigt, dass Quantenphänomene keine Paradoxien sind, sondern Ausdruck globaler Organisation. Sie zeigt,
dass Gravitation keine Geometrie ist, sondern ein Energiefluss zwischen lokalen und globalen Informationsanteilen. Und sie zeigt, dass das Universum weder expandiert noch
aus dem Nichts entstanden ist, sondern ein dynamisches, stationäres Informationssystem darstellt.

Dieses Werk ist daher nicht nur ein Beitrag zur theoretischen Physik, sondern ein Vorschlag für einen Paradigmenwechsel. Es fordert dazu auf, die Grundannahmen der
modernen Physik zu hinterfragen und die Möglichkeit zuzulassen, dass Raum, Zeit, Materie und Energie nicht fundamental sind, sondern emergente Erscheinungen einer tieferen,
diskreten Informationsdynamik. Die kommenden Jahre werden zeigen, ob die Physik bereit ist, diesen Schritt zu gehen. Doch unabhängig davon steht fest: Eine Theorie, die
Quantenmechanik, Elektrodynamik, Gravitation und Kosmologie in einem einzigen, widerspruchsfreien Rahmen vereint, verdient es, ernst genommen zu werden. Sie verdient es,
geprüft, diskutiert, weiterentwickelt und experimentell getestet zu werden.

Dieses Buch ist ein erster Schritt auf diesem Weg.

\begin{flushright}
    Der Autor, \\
    Michael Czybor \\
    \textit{Langenstein/AT, \date{\today}}
\end{flushright}
