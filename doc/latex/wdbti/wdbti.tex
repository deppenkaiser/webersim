\documentclass[11pt, a5paper, twoside, openright]{book}

% --- Pakete ---
\usepackage[ngerman]{babel}
\usepackage[T1]{fontenc}
\usepackage[utf8]{inputenc}
\usepackage{lmodern}
\usepackage{microtype}
\usepackage{csquotes}
\usepackage{verbatim}
\usepackage{geometry}
\usepackage{fancyhdr}
\usepackage{amsmath, amssymb, amsthm}
\usepackage{mathtools}
\usepackage{bm}
\usepackage{siunitx}
\usepackage{graphicx}
\usepackage{subcaption}
\usepackage{booktabs}
\usepackage{tikz}
\usepackage{xcolor}
\usepackage{pgfplots}
\usepackage[
    backend=biber,
    style=phys,
    sorting=nyt,
]{biblatex}
\usepackage[acronym, toc]{glossaries}
\usepackage{hyperref}
\usepackage{parskip}

\makeglossaries

\geometry{
    a4paper,
    top=25mm,
    inner=30mm,    % Bundsteg (größerer Rand für Buchbindung)
    outer=25mm,
    bottom=30mm,
    headheight=15pt,
}

\pagestyle{fancy}
\fancyhf{}
\fancyhead[LE,RO]{\thepage}
\fancyhead[RE]{\leftmark}    % Kapitelname (gerade Seiten)
\fancyhead[LO]{\rightmark}   % Abschnittname (ungerade Seiten)
\renewcommand{\headrulewidth}{0.4pt}

\theoremstyle{definition}
\newtheorem{definition}{Definition}[chapter]
\newtheorem{law}{Physikalisches Gesetz}[chapter]
\theoremstyle{plain}
\newtheorem{theorem}{Theorem}[chapter]
\newtheorem{lemma}[theorem]{Lemma}
\theoremstyle{remark}
\newtheorem{remark}{Bemerkung}[chapter]

\hypersetup{
    colorlinks=true,
    linkcolor=blue,
    citecolor=black,
    urlcolor=black,
    pdftitle={WDB-Theorie - Eine effektive Quantengravitation},
    pdfauthor={Dipl.-Ing. (FH) Michael Czybor},
}

\addbibresource{literatur.bib}  % Ihre .bib-Datei
\makeglossaries

\setlength{\headheight}{26.76852pt}

\newacronym{qm}{QM}{Quantenmechanik}
\newacronym{art}{ART}{Allgemeine Relativitätstheorie}
\newacronym{srt}{SRT}{Spezielle Relativitätstheorie}
\newacronym{cmb}{CMB}{Hintergrundstrahlung}
\newacronym{qed}{QED}{Quantenelektrodynamik}
\newacronym{qft}{QFT}{Quantenfeldtheorie}
\newacronym{epr}{EPR-Paradoxon}{Einstein-Podolsky-Rosen-Paradoxon}
\newacronym{wg}{WG}{Weber-Gravitation}
\newacronym{wed}{WED}{Weber-Elektrodynamik}
\newacronym{dbt}{DBT}{De-Broglie-Bohm-Theorie}
\newacronym{wdbt}{WDBT}{Weber-De Broglie-Bohm-Theorie}
\newacronym{mt}{MT}{Maxwell-Theorie}
\newacronym{iwt}{IWT}{Informations-Weber-Theorie}
\newacronym{dstt}{DSTT}{Dynamischen Schwere-Trägheits-Theorie}

\newglossaryentry{gls:quantenmechanik}
{
    name={Quantenmechanik},
    description={Theorie der Materie und Strahlung auf atomarer und subatomarer Ebene}
}
\newglossaryentry{gls:hamiltonian}
{
    name={\ensuremath{\mathcal{H}}},
    description={Hamilton-Operator, beschreibt die Gesamtenergie eines Systems},
    sort={hamiltonian}
}


\begin{document}

% --- Titelseite ---
\frontmatter
\begin{tikzpicture}[remember picture, overlay]
  \fill[hintergrund] (current page.south west) rectangle (current page.north east);

  \foreach \r in {0.5,1,...,5} {
    \draw[quantenblau!15, line width=0.2pt]
      ($(current page.center)$)
      circle[x radius=\r cm, y radius={0.6*\r cm}];
  }

  \foreach \a in {0,8,...,360} {
    \draw[quantenblau!10, line width=0.15pt]
      (current page.center) -- +(\a:5.5cm);
  }

  \node at (current page.center) {
    \begin{tikzpicture}[scale=0.9]
      \foreach \i in {0,60,...,300} {
        \fill[quantenblau!60] (\i:0.6cm) circle (2pt);
        \draw[quantenblau!40, line width=0.3pt] (0,0) -- (\i:0.6cm);
      }
      \fill[quantenblau!80] (0,0) circle (3pt);
    \end{tikzpicture}
  };

  \node[align=center, text=white, font=\sffamily\bfseries\Huge]
    at ($(current page.center)+(0,3cm)$) {
      \textbf{Die Informations-Weber-Theorie}
  };

  \node[align=center, text=quantenblau!80, font=\sffamily\Large]
    at ($(current page.center)+(0,1.8cm)$) {
      Eine fundamentale Informations-Urtheorie
  };

  \node[align=left, anchor=south east, text=weberrot!70, font=\small]
    at ($(current page.south east)+(-1cm,1cm)$) {
      $\displaystyle I = \text{konstant}$
  };

  \node[align=left, anchor=north east, text=quantenblau!70, font=\small]
    at ($(current page.south east)+(-1cm,3cm)$) {
      $\displaystyle Q = -\frac{\hbar^2}{2m}\frac{\nabla^2\sqrt{\rho}}{\sqrt{\rho}}$
  };

  \node[align=center, text=white, font=\sffamily\large]
    at ($(current page.south)+(0,1cm)$) {
      \textbf{Michael Czybor}
  };

  \node[align=right, text=quantenblau!50, font=\small]
    at ($(current page.north west)+(2cm,-1cm)$) {
      $D = \frac{\ln 20}{\ln(2+\phi)} \approx 2.71$
  };
\end{tikzpicture}

\title{Die Informations-Weber-Theorie\\Eine fundamentale Informations-Urtheorie}
\author{Michael Czybor}
\date{\today}
\maketitle

% --- Vorwort ---
\chapter*{Vorwort}
TBD

\vspace{3em}
\begin{flushright}
    Der Autor, \\
    Michael Czybor \\
    \textit{Langenstein/AT, 22. Dezember 2025}
\end{flushright}


\tableofcontents
\listoffigures
\listoftables

% --- Hauptteil ---
\mainmatter

\chapter{Einleitung}
\label{chap:einleitung}

\section{Information als fundamentale Größe der Physik}
Die zentrale These dieses Buches lautet: \textbf{Information ist die grundlegende physikalische Größe, aus der Energie, Raum, Zeit und Dynamik emergieren}. Während die
klassische Physik Energie als fundamentale Erhaltungsgröße betrachtet, zeigt sich in modernen Theorien zunehmend, dass Energie selbst nur eine abgeleitete Form von
Information ist – eine Maßzahl für die Organisation, Struktur und Veränderbarkeit physikalischer Zustände. In der \gls{qm} beschreibt die Wellenfunktion keine
materielle Welle, sondern eine Informationsverteilung über mögliche Zustände. In der Thermodynamik ist Entropie ein Maß für fehlende Information. In der Relativitätstheorie
bestimmt die Energie-Impuls-Verteilung die Geometrie der Raumzeit – doch diese Verteilung ist letztlich eine Informationsstruktur. Selbst die Lichtgeschwindigkeit erscheint
weniger als fundamentale Konstante, sondern als Eigenschaft eines Informationsflusses in einem strukturierten Medium.

Die \gls{wdbt} bietet einen natürlichen Zugang zu dieser Sichtweise. Sie verbindet direkte Wechselwirkungen (Weber), wellenartige Informationsfelder (De Broglie) und
nichtlokale Organisationsprinzipien (Bohm). In dieser Synthese wird deutlich, dass die Dynamik physikalischer Systeme nicht primär durch Kräfte, Felder oder Geometrien
bestimmt wird, sondern durch die \emph{Transformation von Information}.

Die in diesem Buch entwickelte \textbf{Informations-Weber-Theorie} hebt diesen Gedanken auf eine fundamentale Ebene. Sie interpretiert die bekannten physikalischen Größen
als Informationsfunktionale und zeigt, dass die Erhaltung von Information die eigentliche Grundlage der Energieerhaltung ist. Die scheinbaren Widersprüche zwischen
\gls{qm} und Relativitätstheorie lösen sich auf, wenn man beide als unterschiedliche Manifestationen eines universellen Informationsprinzips versteht.

Diese Perspektive bildet den Ausgangspunkt für die folgenden Kapitel. Erst vor diesem Hintergrund wird verständlich, warum alternative Modelle – etwa die
\gls{wed}, die Bohmsche Mechanik oder fraktale Raumstrukturen – nicht exotische Randphänomene sind, sondern Hinweise auf eine tiefere, informationsbasierte
Ordnung der Natur.

\section{Motivation}
Die moderne Physik steht vor grundlegenden Widersprüchen: Während die \gls{art} die Gravitation als Krümmung der Raumzeit beschreibt, basiert die \gls{srt} auf
idealisierten Inertialsystemen, die in einer gekrümmten Raumzeit streng genommen nicht existieren können. Dieser Konflikt wirft Fragen auf – etwa zur Natur der
Lichtgeschwindigkeit, die in der \gls{srt} absolut ist, in der \gls{art} jedoch lokal variabel.

\begin{quote}
\enquote{Einstein's postulates contain inherent contradictions when applied to real gravitational systems, challenging the universality of special relativity.} \cite{Rubcic1998}
\end{quote}

Hinzu kommen ungelöste Probleme der \gls{qm}: der Welle-Teilchen-Dualismus, der \enquote{Kollaps} der Wellenfunktion bei Messungen und nichtlokale Verschränkung.
Selbst erfolgreiche Theorien wie die \gls{qed} postulieren scheinbar paradoxe Phänomene, etwa virtuelle Photonen mit Überlichtgeschwindigkeit im Pfadintegralformalismus.

Diese Spannungen deuten darauf hin, dass die etablierten Modelle möglicherweise nur Annäherungen an eine tiefere Realität sind. Statt Dogmen zu folgen, sollten wir
alternative Perspektiven prüfen – wie die \gls{wed} oder die \gls{dbt}, die in diesem Buch vorgestellt werden.

\begin{quote}
\enquote{The observer-dependent collapse of the wavefunction is not a fundamental feature of nature but a limitation of the standard interpretation.} \cite{bohm1952}
\end{quote}

\subsection{Dogmatismus und blinde Flecken der modernen Physik}
Die heutige Physik leidet unter einer paradoxen Situation: Einerseits werden etablierte Theorien wie die \gls{art} oder die Quantenfeldtheorie kaum hinterfragt, obwohl sie
fundamentale Schwächen aufweisen – insbesondere Singularitäten in starken Gravitationsfeldern, unendliche Selbstenergien von Teilchen oder die Notwendigkeit \enquote{dunkler}
Entitäten. Andererseits werden unorthodoxe Ansätze oft bereits im Peer-Review aussortiert, obwohl sie Lösungen für diese Probleme bieten könnten.

Ein Beispiel ist die Interpretation der \gls{cmb} als Beweis für den Urknall. Alternative Erklärungen – etwa thermische Gleichgewichtsprozesse in Plasmen – werden kaum
diskutiert, obwohl sie ohne Singularitäten auskommen. Ähnlich verhält es sich mit der Rotverschiebung von Galaxien, die nicht zwingend auf eine Expansion des Universums
hindeuten muss.

\begin{quote}
\enquote{Theoretical physics has become stuck in a paradigm that values mathematical elegance over empirical testability, leading to a stagnation of genuine progress.} \cite{Smolin2006}
\end{quote}

\subsection{Spekulation statt Fortschritt}
Seit den revolutionären Durchbrüchen der \gls{qm} und Relativitätstheorie vor einem Jahrhundert gab es kaum vergleichbare Fortschritte. Stattdessen dominieren
spekulative Konzepte wie höhere Dimensionen oder Multiversen, die empirisch kaum überprüfbar sind.

Doch Wissenschaft sollte sich auf beobachtbare Phänomene konzentrieren. Die Weber-Elektrodynamik zeigt, wie sich elektromagnetische Effekte ohne Felder beschreiben
lassen – durch direkte Wechselwirkungen zwischen Ladungen. Solche Ansätze könnten den Weg zu einer konsistenteren Physik ebnen.

\subsection{Alternative Theorien}
Ein zentrales Problem der modernen Physik liegt in ihrem übermäßigen Vertrauen in die Mathematik. Nur weil etwas mathematisch formulierbar ist, muss es noch lange nicht
der physikalischen Realität entsprechen. Doch statt diese Grenzen anzuerkennen, werden grundlegende Prinzipien der klassischen Physik – wie Energieerhaltung oder die
Gesetze der Thermodynamik – zugunsten abstrakter Gleichungen aufgegeben.

Die \gls{art} beispielsweise postuliert eine dynamische Raumzeit, die Gravitationswellen ermöglicht und im schwachen Feld äußerst erfolgreich ist. Gleichzeitig führt sie im
starken Feld zu echten Singularitäten, die physikalisch problematisch sind. Die Weber-Gravitation bleibt in allen Feldstärken regulär und liefert eine direkte dynamische
Beschreibung ohne geometrische Interpretation.

Konkrete Widersprüche zeigen sich in der Praxis: Nach der \gls{art} müssten Planeten durch die Abstrahlung von Gravitationswellen Energie verlieren – doch warum sind
Planetenbahnen dann über Milliarden Jahre stabil? Wenn die Raumzeit als elastisches Gebilde beschrieben wird, das sich verformen und bewegen lässt: Welche Kraft verrichtet
hier Arbeit, und woher kommt die Energie dafür?

Auch die vermeintlichen Beweise für den Urknall sind keineswegs so eindeutig, wie oft behauptet wird. Die kosmische \gls{cmb} wird automatisch als Echo des Urknalls
interpretiert – doch es gibt alternative Erklärungen, etwa thermische Gleichgewichtsprozesse oder Streuphänomene.

\begin{quote}
\enquote{The interpretation of cosmic microwave background as proof of the Big Bang ignores alternative explanations, such as intrinsic redshifts in plasma cosmology.} \cite{Arp1998}
\end{quote}

Ebenso könnte die Rotverschiebung von Galaxien nicht nur durch Expansion, sondern auch durch andere Mechanismen verursacht werden. Selbst Phänomene wie die Lichtablenkung
oder der Shapiro-Effekt lassen sich ohne \gls{art} erklären, wenn man alternative Gravitationsmodelle zulässt.

\begin{quote}
\enquote{Weber's formulation of electrodynamics provides a consistent framework for gravitational phenomena without invoking curved spacetime.} \cite{WeberElectrodynamics}
\end{quote}

In diesem Buch sollen solche alternativen Erklärungen aufgezeigt werden. Die Physik darf nicht bei mathematischen Dogmen stehen bleiben – sie muss sich wieder auf Logik,
Experiment und echte Kausalität besinnen.

\section{Abweichende Perspektiven in der Physik: Licht, Relativität und alternative Modelle}
\subsection{Feynmans Teilchenmodell des Lichts}
Richard Feynman argumentierte, dass selbst Interferenzphänomene durch Teilchen (Photonen) erklärbar sind – ohne Wellen. Dies wirft die Frage auf: Ist der
Welle-Teilchen-Dualismus wirklich notwendig, oder spiegelt er nur die Grenzen unserer Modelle wider?

\subsection{Widersprüche in der QED: Überlichtschnelle Photonen und Pfadintegrale}
Der Pfadintegralformalismus der \gls{qed} summiert über alle möglichen Photonenpfade – inklusive solcher mit Überlichtgeschwindigkeit. Mathematisch führt dies zu korrekten
Vorhersagen, doch physikalisch bleibt unklar:

\begin{itemize}
\item Können virtuelle Photonen schneller als Licht sein, ohne die \gls{srt} zu verletzen?
\item Ist die Lichtgeschwindigkeit wirklich eine absolute Grenze, oder nur ein makroskopischer Effekt?
\end{itemize}

\subsection{Energieabhängige Lichtgeschwindigkeit? Experimentelle Hinweise}
Einige alternative Theorien (z. B. Schleifenquantengravitation oder VSL-Modelle) schlagen vor, dass die Lichtgeschwindigkeit von der Photonenenergie abhängen könnte.

Mögliche Indizien:
\begin{itemize}
\item Gammablitze mit extrem hohen Energien zeigen minimale Laufzeitunterschiede.
\item Quantengravitationseffekte könnten bei hohen Energien zu Dispersion führen.
\end{itemize}

\begin{quote}
\enquote{The constancy of the speed of light is not an immutable law but a parameter that may vary under extreme conditions, offering solutions to cosmological puzzles.} \cite{Magueijo2003}
\end{quote}

\section{Die Entwicklung des Wellenkonzepts in der Physik}
Das Verständnis von Wellen in der Physik hat sich im Laufe der Zeit radikal gewandelt. Während klassische Wellen wie Schall oder Wasserwellen als Störungen eines
materiellen Mediums beschrieben werden konnten, führten elektromagnetische Wellen und Quantenphänomene zu grundlegenden Umbrüchen. Maxwell zeigte 1865, dass Licht sich als
elektromagnetische Welle auch ohne Äther ausbreitet – was die Frage aufwarf, wie Energie ohne Trägermedium transportiert wird. Die \gls{srt} etablierte die
Lichtgeschwindigkeit als absolute Grenze, während die \gls{art} sie als lokal variabel beschreibt – ein scheinbarer Widerspruch, den alternative Theorien wie die
Weber-Elektrodynamik zu lösen versuchen.

Die Quantenphysik revolutionierte das Wellenkonzept weiter: De Broglie verband Teilchen- und Welleneigenschaften, und die \gls{qed} beschreibt Photonen als Felder mit
überlichtschnellen Pfadintegral-Komponenten. Doch diese mathematische Eleganz wirft physikalische Deutungsprobleme auf – etwa die Rolle des Beobachters beim Kollaps der
Wellenfunktion oder die nicht-lokale Natur der Quantenverschränkung.

Auch Gravitationswellen in der \gls{art} bleiben rätselhaft: Wenn Raumzeit als schwingendes Medium gilt, woher stammt die Energie für ihre Verformung?

Diese Widersprüche zeigen, dass die etablierten Theorien möglicherweise nur Annäherungen an eine tiefere Wahrheit sind.

\section{Wellenphänomene: Die Dualität von instantaner Ganzheit und lokaler Ausbreitung}
Wellen besitzen eine einzigartige Doppelnatur: lokale Ausbreitung und instantane globale Struktur. Diese Dualität zeigt sich besonders deutlich in fundamentalen
Wechselwirkungen.

Die newtonsche Mechanik postuliert mit \enquote{actio = reactio} eine instantane Fernwirkung:
\begin{equation}
\vec{F}_{12} = -\vec{F}_{21}
\end{equation}

Das Coulombsche Gesetz zeigt dieselbe Struktur:
\begin{equation}
\vec{F} = \frac{1}{4\pi\epsilon_0}\frac{q_1 q_2}{r^2}\hat{\vec{r}}
\end{equation}

Interferenzphänomene wie das Doppelspaltexperiment zeigen, dass Wellen sich global so organisieren, dass die Gesamtenergie minimiert wird:
\begin{equation}
|\Psi(x)|^2 = |\psi_1(x) + \psi_2(x)|^2
\end{equation}

Die \gls{wed} erweitert das Coulombsche Gesetz um geschwindigkeits- und beschleunigungsabhängige Terme:
\begin{equation}
\vec{F} = \frac{q_1 q_2}{4\pi\epsilon_0 r^2}
\left[
1 - \frac{\dot{r}^2}{c^2} + \frac{2 r \ddot{r}}{c^2}
\right]
\hat{\vec{r}}
\end{equation}

\section{Das erweiterte Kausalitätskonzept}
Die Physik benötigt einen erweiterten Kausalitätsbegriff, der sowohl lokale Dynamik (Energietransport) als auch systemische Ganzheit (globale Organisation) umfasst.

Das Bohm’sche Quantenpotential
\begin{equation}
Q(\vec{r},t) = -\frac{\hbar^2}{2m}
\frac{\nabla^2 \sqrt{\rho(\vec{r},t)}}{\sqrt{\rho(\vec{r},t)}}
\end{equation}
wirkt instantan und global, während die Weber-Kraft lokale retardierte Effekte beschreibt.

Diese duale Struktur löst zahlreiche konzeptionelle Probleme der modernen Physik und bildet die Grundlage für die in diesem Buch entwickelte Theorie.

\section{Axiome der Informations-Weber-Theorie}
Die in diesem Buch entwickelte Informations-Weber-Theorie basiert auf einer kleinen Anzahl klar formulierter Grundannahmen, die als Axiome dienen. Sie ersetzen die
Vielzahl unvereinbarer Postulate der modernen Physik durch ein einheitliches, informationsbasiertes Fundament.

\subsection*{Axiom I: Der physikalische Zustand ist ein Informationszustand}
Jedes physikalische System wird durch eine Informationsverteilung beschrieben. Größen wie Energie, Impuls oder Ladung sind abgeleitete Funktionale dieser Verteilung.

\subsection*{Axiom II: Information ist eine Erhaltungsgröße}
Die Zeitentwicklung eines Systems ist eine invertierbare Transformation des Informationszustands. Nichts geht verloren, nichts entsteht aus dem Nichts. Energieerhaltung
ist ein Spezialfall dieses Prinzips.

\subsection*{Axiom III: Dynamik ist Informationsfluss}
Die Bewegung von Teilchen, Feldern oder Wellen ergibt sich aus der Umlagerung von Information. Die Weber-Kraft beschreibt lokale Informationsflüsse, das Bohmsche
Quantenpotential globale.

\subsection*{Axiom IV: Raum ist eine emergente Informationsgeometrie}
Der physikalische Raum ist keine Grundgröße, sondern die effektive Metrik der Kopplungsstruktur des Informationsnetzes. Seine fraktale Dimension ist eine Eigenschaft dieser Struktur.

\subsection*{Axiom V: Kausalität besitzt zwei Ebenen}
Lokale Kausalität beschreibt den Energietransport mit endlicher Geschwindigkeit. Systemische Kausalität beschreibt die instantane Organisation des Informationszustands.
Beide sind komplementär.

\section{Aufbau und Zielsetzung dieses Buches}
Dieses Buch verfolgt zwei zentrale Ziele. Erstens soll gezeigt werden, dass die \gls{wdbt} eine konsistente Grundlage für eine alternative Beschreibung von Gravitation und
\gls{qm} bildet, in der direkte Wechselwirkungen und nichtlokale Informationsstrukturen vereinigt werden. Zweitens entwickelt es eine informationsbasierte Urtheorie, in der
Energie, Raum, Zeit und Dynamik als abgeleitete Größen eines zugrunde liegenden Informationsnetzes erscheinen.

Der Aufbau des Buches folgt der endgültigen Struktur:
\begin{enumerate}
\item \textbf{Einleitung}  
Motivation, Kritik der modernen Physik, Einführung des Informationsbegriffs und Formulierung der Axiome der Informations-Weber-Theorie.
\item \textbf{Der Informationszustand}  
Definition der Informationsdichte, Informationsflüsse, Kontinuitätsgleichung und Informationsfunktionale.
\item \textbf{Die klassische Weber-Elektrodynamik}  
Historische Einordnung, Lagrange-Ansatz, Herleitung und Interpretation der Weber-Kraft als lokaler Grenzfall.
\item \textbf{Informations-Lagrange-Funktional}  
Mathematische Grundstruktur der Theorie: vollständiges Funktional, Variation, Euler-Lagrange-Gleichungen, Zerlegung in lokale und globale Beiträge.
\item \textbf{Informationsmetrik und emergente Raumzeit}  
Herleitung der effektiven Metrik, fraktale Dimension, emergente Zeit und diskrete Informationsgeometrie.
\item \textbf{Emergenz klassischer und quantenmechanischer Phänomene}  
Trägheit, Gravitation, Wellenphänomene, Nichtlokalität und systemische Kausalität als emergente Informationsdynamik.
\item \textbf{Vergleich mit etablierten Theorien}  
Einordnung von \gls{art}, \gls{srt}, \gls{qm}, \gls{qed} und \gls{qft} als Grenzfälle der Informations-Weber-Theorie.
\item \textbf{Naturkonstanten aus Informationsarchitektur}  
Herleitung von $c$, $\hbar$, $G$, $\alpha$ und weiteren Konstanten aus Informationskopplungen und fraktaler Struktur.
\item \textbf{Konsequenzen der Informations-Weber-Theorie}  
Kosmologie, Rotverschiebung, \gls{cmb}, Big Bounce und fraktale Struktur des Universums im informationsbasierten Rahmen.
\item \textbf{Numerische Simulation eines Informationsnetzes}  
Diskretisierung, Evolutionsgleichungen, Algorithmus und beobachtbare Größen in der digitalen \gls{wdbt}+.
\end{enumerate}

Diese Struktur ermöglicht es, die Informations-Weber-Theorie sowohl als Weiterentwicklung der klassischen \gls{wdbt} als auch als eigenständige fundamentale Theorie zu verstehen.

\section{Zusammenfassung der Einleitung}
Die Einleitung dieses Buches hat gezeigt, dass die moderne Physik trotz ihrer beeindruckenden Erfolge vor grundlegenden konzeptionellen Problemen steht. Widersprüche
zwischen \gls{srt} und \gls{art}, ungelöste Fragen der \gls{qm}, spekulative Erweiterungen wie höhere Dimensionen oder Multiversen sowie die Abhängigkeit von
mathematischen Konstruktionen ohne klare physikalische Interpretation deuten darauf hin, dass die etablierten Theorien nur Näherungen an eine tiefere Realität darstellen.

Der zentrale Gedanke dieses Werkes lautet, dass \textbf{Information die fundamentale physikalische Größe} ist, aus der Energie, Raum, Zeit und Dynamik emergieren. Energie
erscheint in dieser Sichtweise nicht als primäre Erhaltungsgröße, sondern als Ausdruck der Struktur und Organisation eines Informationszustands.
Die Weber–De-Broglie–Bohm-Theorie liefert hierfür einen natürlichen Ausgangspunkt, da sie direkte Wechselwirkungen, wellenartige Informationsfelder und nichtlokale
Organisationsprinzipien vereint.

Die formulierten Axiome der Informations-Weber-Theorie bilden das Fundament dieser neuen Perspektive. Sie beschreiben physikalische Systeme als Informationsverteilungen,
deren Zeitentwicklung durch invertierbare Transformationen bestimmt wird. Dynamik wird als Umlagerung von Information verstanden, während Raum als emergente Geometrie der
Kopplungsstruktur erscheint. Die Kausalität besitzt zwei Ebenen: eine lokale Dynamik des Energietransports und eine systemische Ganzheit, die die globale Organisation des
Informationszustands bestimmt.

Diese Einleitung bereitet damit den Boden für die folgenden Kapitel. Kapitel~\ref{chap:einleitung} verankert Motivation und Axiome, Kapitel~2 führt den Informationszustand
formal ein, Kapitel~3 zeigt die \gls{wed} als lokalen Grenzfall. Kapitel~4 und~5 etablieren die mathematische Grundstruktur der Theorie über
Informations-Lagrange-Funktional und Informationsmetrik. Kapitel~6 und~7 zeigen, wie klassische und quantenmechanische Phänomene als emergente Informationsdynamik
erscheinen und wie etablierte Theorien als Grenzfälle eingeordnet werden. Kapitel~8–10 widmen sich Naturkonstanten, kosmologischen Konsequenzen und numerischen
Simulationen des Informationsnetzes.

Damit ist der Rahmen gesetzt, in dem die Informations-Weber-Theorie im weiteren Verlauf systematisch entwickelt, mathematisch präzisiert und mit bestehenden Theorien
verglichen wird.

\chapter{Die Informations-Weber-Theorie}
\label{chap:informationstheorie}

\section{Der Informationszustand}
Die Informations-Weber-Theorie geht von der grundlegenden Annahme aus, dass jeder physikalische Zustand durch eine \emph{Informationsverteilung} beschrieben wird. Diese
wird durch eine skalare Dichtefunktion
\[
    \rho_I(\vec{r},t)
\]
repräsentiert, die angibt, wie viel strukturierte Information an einem Ort vorliegt. 

Im Gegensatz zu klassischen Feldern besitzt $\rho_I$ keine materielle Bedeutung. Sie beschreibt weder Masse noch Ladung oder Energie, sondern die \emph{Organisation} eines
physikalischen Systems. Energie, Impuls und andere Größen entstehen erst als abgeleitete Funktionale dieser Informationsstruktur.

\subsection{Informationsdichte und Informationsfluss}
Analog zur Kontinuitätsgleichung der klassischen Physik wird der Informationsfluss durch einen Vektorstrom
\[
    \vec{J}_I(\vec{r},t)
\]
beschrieben. Die fundamentale Erhaltungsgleichung lautet:
\[
    \frac{\partial \rho_I}{\partial t} + \nabla \cdot \vec{J}_I = 0.
\]
Diese Gleichung ist das Herzstück der Theorie: Sie ersetzt die Energieerhaltung durch eine \emph{Informationserhaltung}, aus der die Energieerhaltung als Spezialfall folgt. Die gesamte Dynamik ergibt sich aus der Umlagerung von Information.

\section{Information als Ursprung physikalischer Größen}
In der Informations-Weber-Theorie entstehen physikalische Größen als Funktionale der Informationsdichte. Energie, Impuls, Trägheit und sogar die geometrische Struktur des
Raumes ergeben sich aus Symmetrien und Transformationen der Informationsverteilung.

Damit wird die klassische Unterscheidung zwischen Materie, Feldern und Geometrie aufgehoben: Alles entsteht aus einer einzigen fundamentalen Größe – der Information.

\section{Dynamik als Informationsfluss}
Die Bewegungsgleichungen eines Systems ergeben sich aus der Umlagerung von Information. Die Theorie unterscheidet zwei komplementäre Dynamikformen:
\begin{itemize}
    \item \textbf{lokale Dynamik}: beschrieben durch die Weber-Kraft,
    \item \textbf{globale Dynamik}: beschrieben durch das Bohm’sche Quantenpotential.
\end{itemize}
Diese beiden Strukturen sind keine konkurrierenden Modelle, sondern zwei Projektionen derselben Informationsdynamik.

\subsection{Lokale Dynamik: Weber-Kraft}
Die Weber-Kraft beschreibt lokale Informationsflüsse. Sie ist der lokale Grenzfall der informationsbasierten Dynamik und wird in Kapitel~\ref{chap:weberklassisch}
hergeleitet.

\subsection{Globale Dynamik: Quantenpotential}
Das Bohm’sche Quantenpotential beschreibt die systemische, nichtlokale Organisation des Informationszustands. Es ist der globale Grenzfall der Informationsdynamik und wird
in Kapitel~4 aus dem Informations-Lagrange-Funktional abgeleitet.

\subsection{Die analoge WDBT als Fernwirkungstheorie}
Die analoge Weber–De-Broglie–Bohm-Theorie (WDBT) beschreibt die Gesamtwirkung auf ein System durch drei Fernwirkungsbeiträge:
\[
    F = F_{\text{WED}} + F_{\text{WG}} + F_Q.
\]
\begin{itemize}
    \item $F_{\text{WED}}$: Weber-Elektrodynamik (Ladungen),
    \item $F_{\text{WG}}$: Weber-Gravitation (Massen),
    \item $F_Q$: Bohm’sches Quantenpotential (Informationsstruktur).
\end{itemize}
Diese analoge Theorie besitzt \emph{kein Raummodell}. Sie arbeitet rein relational und kann daher keine propagierenden Störungen wie Gravitationswellen beschreiben. Dies ist kein Mangel, sondern eine Konsequenz der rein dynamischen Fernwirkungsstruktur ohne geometrische Interpretation.

\section{Raum als emergente Informationsgeometrie}
Die analoge WDBT arbeitet ohne ontologischen Raum. Erst die digitale WDBT führt ein diskretes Informationsnetz ein, aus dem der physikalische Raum als emergente Geometrie
entsteht.

\subsection{Warum Raum nicht fundamental sein kann}
Mehrere Argumente sprechen gegen einen fundamentalen Raum:
\begin{itemize}
    \item Fernwirkungen benötigen keinen Trägerraum.
    \item Kausalität kann ohne Raum formuliert werden.
    \item Die fraktale Dimension widerspricht einem glatten Kontinuum.
    \item Kontinuumsmodelle erzeugen Singularitäten und Paradoxien.
    \item Eine dynamische Raumzeit setzt ein Informationsnetz voraus.
\end{itemize}
Die Konsequenz lautet: Raum ist eine abgeleitete Größe, keine fundamentale.

\subsection{Emergenz der Zeit}
Auch die Zeit ist keine primitive Größe. Sie entsteht aus der Ordnung der Informationszustände und aus der Aktualisierungsdynamik des Informationsnetzes.
\begin{itemize}
    \item Zeit ist ein Ordnungsparameter der Informationsveränderung.
    \item In der digitalen WDBT entsteht Zeit aus diskreten Aktualisierungsschritten.
    \item Zeitdilatation ist eine Eigenschaft der Informationsgeometrie.
    \item Die Zeitrichtung entsteht aus Informationsentropie.
\end{itemize}

\subsection{Fraktale Dimension als geometrische Signatur}
Die fraktale Dimension
\[
    D = \frac{\ln 20}{\ln(2+\phi)}
\]
ist eine Eigenschaft der Kopplungsstruktur des Informationsnetzes. Sie beschreibt die Skalierungsstruktur der Informationsarchitektur und ist ein Hinweis darauf, dass der
Raum nicht fundamental sein kann.

\subsection{Diskrete Informationsstruktur als Ursprung des Raumes}
Die digitale WDBT beschreibt ein Netzwerk aus Informationsknoten und Kopplungen. Der physikalische Raum ist die effektive Metrik dieser Kopplungsstruktur:
\[
    g_{ij} = g_{ij}[\text{Kopplungen}, \rho_I].
\]

\subsection{Emergenz der Dynamik aus der Informationsgeometrie}
Wenn Raum und Zeit emergent sind, dann ist auch die Dynamik emergent. Bewegung, Kräfte und Wellen entstehen als Konsequenzen der Informationsgeometrie.
\begin{itemize}
    \item lokale Dynamik = Projektion der lokalen Informationsstruktur,
    \item globale Dynamik = Projektion der systemischen Informationsstruktur,
    \item Wellen = kollektive Moden der Informationsgeometrie.
\end{itemize}

\subsection{Emergenz von Gravitationswellen}
Die analoge WDBT kann keine Gravitationswellen beschreiben. Die digitale WDBT erzeugt Gravitationswellen als kollektive Moden der Informationsgeometrie. Damit wird die Stärke der ART – die Beschreibung dynamischer Geometrie – in einen informationsbasierten Rahmen überführt.

\subsection{CMB-Struktur als fossilierte Informationsgeometrie}
Die anisotrope Struktur der kosmischen Hintergrundstrahlung (CMB) spiegelt die fraktale Kopplungsstruktur des frühen Informationsnetzes wider.

\subsection{Herleitung von Naturkonstanten}
In der digitalen WDBT entstehen Naturkonstanten wie $c$, $\hbar$ und $G$ aus Skalierungsrelationen der Informationsarchitektur.

\subsection{Einordnung von WDBT, ART und ART+}

\begin{itemize}
    \item \textbf{analoge WDBT}: Fernwirkung, kein Raum, keine Wellen; direkte dynamische Struktur.
    \item \textbf{ART}: geometrisches Raummodell, das die Weber-Dynamik im schwachen Feld reproduziert, jedoch im starken Feld Singularitäten erzeugt.
    \item \textbf{ART+}: ART erweitert um informationsbasierte Struktur; keine echten Singularitäten.
    \item \textbf{digitale WDBT+}: vollständige informationsbasierte Urtheorie mit emergenter Geometrie und Gravitationswellen.
\end{itemize}

\section{Zusammenfassung}
Kapitel~2 hat die konzeptionellen Grundlagen der Informations-Weber-Theorie dargestellt:
\begin{itemize}
    \item Informationszustand als fundamentale Größe,
    \item lokale und globale Informationsdynamik,
    \item analoge WDBT als Fernwirkungstheorie,
    \item digitale WDBT als informationsbasierte Raumtheorie,
    \item Emergenz von Raum, Zeit, Dynamik und Naturkonstanten.
\end{itemize}

Die mathematische Formulierung erfolgt in Kapitel~4 (Informations-Lagrange-Funktional) und Kapitel~5 (Informationsmetrik).

\chapter{Die klassische Weber-Elektrodynamik}
\label{chap:weberklassisch}

\section{Motivation}
Die Weber-Elektrodynamik stellt einen der frühesten und konsequentesten Versuche dar, elektrische und magnetische Wechselwirkungen ohne Felder zu beschreiben. Statt eines
elektromagnetischen Feldes im Raum verwendet Weber ein Wirkungsprinzip, bei dem Ladungen direkt aufeinander einwirken.

Diese Sichtweise ist für die Informations-Weber-Theorie von zentraler Bedeutung:  
Sie zeigt, dass lokale Dynamik \emph{ohne} Feldkonzepte formuliert werden kann und dass Kräfte aus relationalen Größen entstehen können. Die Weber-Kraft bildet daher den
\textbf{lokalen Grenzfall} der informationsbasierten Dynamik, der entsteht, wenn globale Informationsstrukturen vernachlässigt werden.

Dieses Kapitel stellt die klassische Theorie dar, bevor in Kapitel~4 gezeigt wird, wie sie aus dem lokalen Anteil des Informations-Lagrange-Funktionals hervorgeht.

\section{Historischer Kontext}
Wilhelm Eduard Weber formulierte 1846 eine elektrodynamische Kraft, die sowohl die Coulomb-Wechselwirkung als auch geschwindigkeits- und beschleunigungsabhängige Terme
enthält. Diese Theorie war lange Zeit eine ernsthafte Alternative zu Maxwells Feldtheorie und wurde im 20. Jahrhundert durch Assis und andere rekonstruiert und präzisiert.

Die Weber-Kraft ist bemerkenswert, weil sie:
\begin{itemize}
    \item direkt zwischen Ladungen wirkt (keine Felder als ontologische Objekte),
    \item retardierte Effekte teilweise berücksichtigt,
    \item Energie- und Impulserhaltung strikt respektiert,
    \item magnetische und strahlungsähnliche Effekte aus rein mechanischen Prinzipien ableitet.
\end{itemize}
Diese Eigenschaften machen sie zu einem idealen lokalen Grenzfall der Informations-Weber-Theorie.

\section{Der Weber-Lagrange-Ansatz}
Die Weber-Kraft lässt sich aus einem Lagrange-Funktional herleiten. Für zwei Ladungen $q_1$ und $q_2$ mit Abstand $r$ lautet der Lagrange-Ansatz:
\begin{equation}
    L
    =
    \frac{1}{2} m_1 \dot{\vec{r}}_1^{\,2}
    +
    \frac{1}{2} m_2 \dot{\vec{r}}_2^{\,2}
    -
    \frac{q_1 q_2}{4\pi\varepsilon_0 r}
    \left(
        1
        -
        \frac{\dot{r}^2}{2c^2}
        +
        \frac{r \ddot{r}}{c^2}
    \right).
    \label{eq:weber_lagrange}
\end{equation}

Dieser Ausdruck enthält:
\begin{itemize}
    \item den Coulomb-Term (statische Fernwirkung),
    \item einen geschwindigkeitsabhängigen Term (magnetische Effekte),
    \item einen beschleunigungsabhängigen Term (strahlungsähnliche Reaktionskräfte).
\end{itemize}
Die letzten beiden Terme sind die charakteristischen Merkmale der Weber-Theorie und zeigen, dass elektromagnetische Effekte aus rein mechanischen Prinzipien entstehen
können.

\section{Herleitung der Weber-Kraft}
Durch Variation des Lagrange-Funktionals \eqref{eq:weber_lagrange} erhält man die Weber-Kraft:
\begin{equation}
    \vec{F}
    =
    \frac{q_1 q_2}{4\pi\varepsilon_0 r^2}
    \left[
        1
        -
        \frac{\dot{r}^2}{c^2}
        +
        \frac{2 r \ddot{r}}{c^2}
    \right]
    \hat{\vec{r}}.
    \label{eq:weber_kraft}
\end{equation}

Diese Gleichung beschreibt:
\begin{enumerate}
    \item den statischen Coulomb-Term,
    \item einen geschwindigkeitsabhängigen Term (magnetische Effekte),
    \item einen beschleunigungsabhängigen Term (Strahlungswiderstand).
\end{enumerate}
Die Weber-Kraft ist damit eine vollständig mechanische Beschreibung elektromagnetischer Wechselwirkungen.

\section{Interpretation der Terme}
Die drei Terme der Weber-Kraft haben klare physikalische Bedeutungen:
\begin{itemize}
    \item \textbf{Coulomb-Term:}  
    beschreibt die statische Fernwirkung zwischen Ladungen.

    \item \textbf{Geschwindigkeits-Term:}  
    erzeugt magnetische Effekte und ist proportional zu $\dot{r}^2$.

    \item \textbf{Beschleunigungs-Term:}  
    beschreibt die Reaktion des Systems auf zeitliche Änderungen der Bewegung
    und ist verantwortlich für strahlungsähnliche Widerstandseffekte.
\end{itemize}

Diese Struktur zeigt, dass die Weber-Kraft bereits wesentliche Elemente einer dynamischen Wechselwirkung enthält, die später in der Informations-Weber-Theorie als lokale
Informationsflüsse interpretiert werden.

\section{Bedeutung für die Informations-Weber-Theorie}
Die Weber-Kraft ist kein konkurrierendes Modell zur informationsbasierten Theorie, sondern ihr \textbf{lokaler Grenzfall}. In Kapitel~4 wird gezeigt, wie die Weber-Kraft
aus dem lokalen Anteil des Informations-Lagrange-Funktionals entsteht und wie das Bohm-Potential als globaler Anteil hinzukommt.

Damit bildet die klassische \gls{wed} die Brücke zwischen historischer Mechanik und moderner informationsbasierter Physik. Sie zeigt, dass lokale Dynamik ohne
Felder formuliert werden kann – ein zentrales Prinzip der Informations-Weber-Theorie.


% Neue mathematische Grundstruktur
\chapter{Emergenz klassischer und quantenmechanischer Phänomene}
\label{chap:emergenz}

\section{Einleitung}
Die Informations-Weber-Theorie beschreibt physikalische Systeme nicht durch Felder, Geometrien oder materielle Substanzen, sondern durch die Struktur und Dynamik einer
Informationsverteilung. In diesem Kapitel wird gezeigt, wie aus dieser informationsbasierten Grundlage klassische und quantenmechanische Phänomene emergieren. Die
bekannten Gleichungen der Mechanik, Elektrodynamik und Quantenphysik erscheinen dabei nicht als fundamentale Postulate, sondern als Näherungen einer tieferen
Informationsordnung.

Die zentrale Idee lautet:
\[
    \textbf{Physikalische Gesetze sind emergente Ordnungsprinzipien der Information.}
\]
Die Emergenz erfolgt in zwei Schritten:

\begin{enumerate}
    \item \textbf{Lokale Dynamik} erzeugt klassische Phänomene  
    (Weber-Kraft, Trägheit, Energie-Impuls-Beziehungen).

    \item \textbf{Globale Dynamik} erzeugt quantenmechanische Phänomene  
    (Interferenz, Nichtlokalität, Quantenpotential).
\end{enumerate}

Damit wird die traditionelle Trennung zwischen „klassisch“ und „quantum“ aufgehoben. Beide sind Manifestationen derselben informationsbasierten Struktur.

\section{Trägheit als emergente Informationsstruktur}
Trägheit ist in der klassischen Physik ein primitives Konzept: Ein Körper „hat“ Masse und widersetzt sich Beschleunigungen. In der Informations-Weber-Theorie entsteht
Trägheit aus der Struktur der Informationsdichte.

Ein System mit homogener Informationsverteilung besitzt minimale interne Gradienten. Eine Beschleunigung erzeugt eine zeitliche Änderung der Informationsstruktur, die
energetisch ungünstig ist. Die resultierende Widerstandskraft ist die Trägheit.

Formal ergibt sich die Trägheitskraft aus der Variation des lokalen Informationsfunktionals:
\[
    \vec{F}_{\text{Trägheit}}
    =
    - \frac{\delta \mathcal{F}_{\text{lokal}}}{\delta (\partial_t \rho_I)}.
\]
Damit ist Trägheit keine mysteriöse Eigenschaft der Materie, sondern eine Konsequenz der Informationsdynamik.

\section{Gravitation als Informationsfluss}
Die klassische Gravitation wird in der ART als Krümmung der Raumzeit beschrieben. In der Informations-Weber-Theorie ist Gravitation ein emergenter Informationsfluss.

Eine inhomogene Informationsverteilung erzeugt einen effektiven Informationsgradienten, der zu einer gerichteten Umlagerung von Information führt. Dieser Informationsfluss
manifestiert sich als Kraft, die im Grenzfall schwacher Felder der Newtonschen Gravitation entspricht.

Die Gravitationskraft ergibt sich aus:
\[
    \vec{F}_{\text{grav}}
    =
    - \nabla \Phi_I,
\]
wobei $\Phi_I$ das informationsbasierte Potential ist:
\[
    \Phi_I(\vec{r})
    =
    \int \frac{\rho_I(\vec{r}')}{|\vec{r}-\vec{r}'|}\, d^3x'.
\]
Damit ist Gravitation keine geometrische Eigenschaft des Raumes, sondern eine Informationskopplung, aus der der Raum erst emergiert.

\section{Wellenphänomene als energetische Informationsorganisation}
Wellenphänomene entstehen aus der Tendenz eines Systems, seine Informationsstruktur energetisch zu optimieren. Die Minimierung des globalen Informationsfunktionals führt
zu Interferenzmustern, die in der klassischen Physik als Wellenphänomene erscheinen.

Die Wahrscheinlichkeitsdichte eines quantenmechanischen Systems ergibt sich aus:
\[
    |\Psi|^2 = \rho_I.
\]
Die Interferenz zweier Informationsstrukturen führt zu:
\[
    \rho_I = \rho_1 + \rho_2 + 2\sqrt{\rho_1 \rho_2}\cos(\Delta \phi),
\]
wobei $\Delta \phi$ die relative Informationsphase ist.

Damit wird Interferenz nicht als „Welle“ verstanden, sondern als energetisch optimale Informationsorganisation.

\section{Nichtlokalität als systemische Ganzheit}
Die Informations-Weber-Theorie besitzt zwei Kausalitätsebenen:

\begin{itemize}
    \item \textbf{lokale Kausalität} (Energietransport, Weber-Kraft),
    \item \textbf{systemische Kausalität} (globale Informationsorganisation).
\end{itemize}

Die systemische Kausalität führt zu Nichtlokalität, wie sie in der Quantenmechanik beobachtet wird. Das Bohmsche Quantenpotential
\[
    Q = -\frac{\hbar^2}{2m}
    \frac{\nabla^2 \sqrt{\rho_I}}{\sqrt{\rho_I}}
\]
ist Ausdruck dieser globalen Struktur.

Nichtlokalität ist damit keine Verletzung der Relativität, sondern eine Eigenschaft der Informationsorganisation.

\section{Zusammenführung der klassischen und quantenmechanischen Emergenz}
Die Informations-Weber-Theorie zeigt:

\begin{itemize}
    \item Trägheit entsteht aus lokalen Informationsänderungen.
    \item Gravitation entsteht aus Informationsgradienten.
    \item Wellenphänomene entstehen aus globaler Informationsoptimierung.
    \item Nichtlokalität entsteht aus systemischer Ganzheit.
\end{itemize}

Damit erscheinen klassische und quantenmechanische Phänomene als unterschiedliche Aspekte derselben fundamentalen Informationsdynamik.

\section{Mathematische Vertiefung der Trägheit}
Trägheit entsteht in der Informations-Weber-Theorie aus der Reaktion der Informationsstruktur auf zeitliche Änderungen. Eine Beschleunigung verändert die
Informationsdichte $\rho_I$ und erzeugt eine energetisch ungünstige Konfiguration. Die resultierende Widerstandskraft ist die Trägheit.

\subsection{Trägheit aus dem lokalen Informationsfunktional}
Der lokale Anteil des Informations-Lagrange-Funktionals besitzt die Form
\[
    \mathcal{F}_{\text{lokal}}
    =
    \alpha\, (\partial_t \rho_I)^2
    +
    \beta\, (\nabla \rho_I)^2
    + \ldots
\]
Die Variation nach $\partial_t \rho_I$ ergibt die Trägheitskraft:
\[
    \vec{F}_{\text{Trägheit}}
    =
    - \frac{\delta \mathcal{F}_{\text{lokal}}}{\delta (\partial_t \rho_I)}
    =
    - 2\alpha\, \partial_t \rho_I.
\]
Damit ist Trägheit proportional zur Änderungsrate der Informationsdichte.

\subsection{Effektive Masse als Informationssteifigkeit}
Die effektive Masse ergibt sich aus
\[
    m_{\text{eff}}
    =
    2\alpha \int \left(\frac{\partial \rho_I}{\partial v}\right)^2 d^3x.
\]
Damit ist Masse keine fundamentale Größe, sondern eine Maßzahl für die „Steifigkeit“ der Informationsstruktur gegenüber Änderungen.

\section{Vertiefung der gravitativen Informationsdynamik}
Gravitation entsteht aus Informationsgradienten. Eine inhomogene Informationsverteilung erzeugt ein effektives Potential
\[
    \Phi_I(\vec{r})
    =
    \int \frac{\rho_I(\vec{r}')}{|\vec{r}-\vec{r}'|}\, d^3x'.
\]

\subsection{Informationsgradienten und Newton-Potential}
Für schwach variierende Informationsdichten gilt
\[
    \rho_I(\vec{r}') \approx \rho_I(\vec{r})
    + (\vec{r}'-\vec{r}) \cdot \nabla \rho_I(\vec{r}),
\]
woraus folgt:
\[
    \Phi_I(\vec{r})
    \propto
    \frac{1}{r}.
\]
Damit entsteht das Newtonsche Potential als Grenzfall.

\subsection{Informationsfluss und Gravitationskraft}
Die Gravitationskraft ergibt sich aus
\[
    \vec{F}_{\text{grav}}
    =
    - \nabla \Phi_I.
\]

Damit ist Gravitation ein Informationsfluss, nicht eine geometrische Eigenschaft des Raumes.

\section{Vertiefung der Wellenphänomene}
Wellenphänomene entstehen aus der Minimierung des globalen Informationsfunktionals:
\[
    \mathcal{F}_{\text{global}}
    =
    \gamma \frac{(\nabla \rho_I)^2}{\rho_I}.
\]

\subsection{Variation des globalen Funktionals}
Die Variation führt zu
\[
    \frac{\delta \mathcal{F}_{\text{global}}}{\delta \rho_I}
    =
    -\gamma
    \frac{\nabla^2 \sqrt{\rho_I}}{\sqrt{\rho_I}},
\]
was proportional zum Bohm-Potential ist.

\subsection{Interferenz als Informationsoptimierung}
Die Interferenz zweier Informationsstrukturen ergibt
\[
    \rho_I = \rho_1 + \rho_2 + 2\sqrt{\rho_1 \rho_2}\cos(\Delta \phi).
\]
Damit ist Interferenz keine Welle, sondern eine energetisch optimale Informationsorganisation.

\section{Vertiefung der Nichtlokalität}
Nichtlokalität entsteht aus der systemischen Ganzheit des Informationsraums. Das Bohm-Potential
\[
    Q = -\frac{\hbar^2}{2m}
    \frac{\nabla^2 \sqrt{\rho_I}}{\sqrt{\rho_I}}
\]
ist ein globaler Operator.

\subsection{Systemische Kausalität}
Die Informations-Weber-Theorie besitzt zwei Kausalitätsebenen:

\begin{itemize}
    \item lokale Dynamik (Weber-Kraft),
    \item globale Dynamik (Quantenpotential).
\end{itemize}

Die globale Dynamik ist nicht durch Lichtgeschwindigkeit begrenzt, da sie keine
Energie transportiert.

\subsection{EPR-Korrelationen}
Die Korrelation zweier Informationsstrukturen ergibt
\[
    \rho_I(\vec{r}_1, \vec{r}_2)
    \neq
    \rho_I(\vec{r}_1)\rho_I(\vec{r}_2).
\]
Damit ist Verschränkung eine Eigenschaft der Informationskopplung, nicht der Raumzeit.

\section{Informationsmetriken und fraktale Geometrie}
Der physikalische Raum ist eine emergente Informationsgeometrie. Die Metrik ergibt sich aus
\[
    g_{ij}
    =
    \frac{\partial^2 \mathcal{F}}{\partial (\partial_i \rho_I)\, \partial (\partial_j \rho_I)}.
\]

\subsection{Fraktale Dimension}
Die fraktale Dimension
\[
    D = \frac{\ln 20}{\ln(2+\phi)}
\]
ist eine Eigenschaft der Kopplungsstruktur des Informationsnetzes.

\subsection{Makroskopische Emergenz}
Für große Skalen gilt
\[
    D \to 3,
\]
wodurch der klassische dreidimensionale Raum entsteht.

\section{Energetische Interpretation der Informationsdynamik}
Energie ist ein abgeleitetes Funktional der Informationsstruktur:
\[
    E[\rho_I]
    =
    \int \mathcal{H}_I(\rho_I, \nabla \rho_I)\, d^3x.
\]

\subsection{Noether-Theorem im Informationsraum}
Zeitsymmetrie $\Rightarrow$ Energieerhaltung
Translationssymmetrie $\Rightarrow$ Impulserhaltung
Rotationssymmetrie $\Rightarrow$ Drehimpulserhaltung

\subsection{Energie als Informationsmaß}
Energie misst die „Kosten“ der Informationsorganisation:
\[
    E \propto \int (\nabla \rho_I)^2 d^3x.
\]

\section{Vergleich zu etablierten Theorien}
Die Informations-Weber-Theorie reproduziert:

\begin{itemize}
    \item die klassische Mechanik (lokale Dynamik),
    \item die Weber-Elektrodynamik (lokale Informationsflüsse),
    \item die Quantenmechanik (globale Informationsorganisation),
    \item die Newtonsche Gravitation (Informationsgradienten).
\end{itemize}

Sie benötigt keine:

\begin{itemize}
    \item Felder,
    \item Raumzeitkrümmung,
    \item Wellenfunktionen als ontologische Objekte,
    \item Kollapsmechanismen.
\end{itemize}

Damit ist sie eine einheitliche, reduktionistische Urtheorie.

\chapter{Vergleich mit etablierten Theorien}
\label{chap:vergleich}

\section{Einleitung}
Die Informations-Weber-Theorie wurde in den vorangegangenen Kapiteln als eine fundamentale Urtheorie entwickelt, in der physikalische Größen und Dynamiken aus der Struktur
und Transformation von Information hervorgehen. In diesem Kapitel wird gezeigt, wie sich diese Theorie zu den etablierten physikalischen Modellen verhält. Ziel ist es
nicht, diese Modelle zu ersetzen, sondern ihre Gültigkeitsbereiche, Grenzen und emergenten Eigenschaften aus informationsbasierter Sicht zu verstehen.

Die etablierten Theorien der Physik lassen sich grob in vier Klassen einteilen:

\begin{enumerate}
    \item klassische Mechanik,
    \item Elektrodynamik (Maxwell, Lorentz, Weber),
    \item Quantenmechanik und Quantenfeldtheorie,
    \item Relativitätstheorie (SRT und ART).
\end{enumerate}

Jede dieser Theorien besitzt einen klar definierten Gültigkeitsbereich und liefert dort präzise Vorhersagen. Gleichzeitig weisen sie fundamentale Spannungen auf, die auf
eine tiefere, einheitliche Struktur hindeuten. Die Informations-Weber-Theorie bietet einen Rahmen, in dem diese Modelle als Grenzfälle einer universellen
Informationsdynamik verstanden werden können.

\section{Klassische Mechanik als lokaler Grenzfall}
Die klassische Mechanik basiert auf Newtons Axiomen, insbesondere auf dem zweiten Axiom
\[
    \vec{F} = m \vec{a}.
\]
In der Informations-Weber-Theorie entsteht diese Gleichung als Grenzfall lokaler Informationsdynamik. Die effektive Masse ist dabei keine fundamentale Größe, sondern
ein Maß für die Steifigkeit der Informationsstruktur:
\[
    m_{\text{eff}} \propto \int (\partial_t \rho_I)^2 d^3x.
\]
Damit wird die klassische Mechanik als Näherung einer tieferen Dynamik verstanden, die nur gültig ist, wenn:

\begin{itemize}
    \item Informationsgradienten klein sind,
    \item globale Informationsstrukturen vernachlässigt werden können,
    \item Geschwindigkeiten klein gegenüber $c$ sind.
\end{itemize}

Die klassische Mechanik ist somit eine lokale, niederenergetische Näherung der Informations-Weber-Theorie.

\section{Elektrodynamik: Maxwell, Lorentz und Weber}
Die Elektrodynamik existiert in drei historischen Formulierungen:

\begin{enumerate}
    \item \textbf{Maxwell-Felder} (kontinuierliche Felder im Raum),
    \item \textbf{Lorentz-Kraft} (Felder + Ladungen),
    \item \textbf{Weber-Kraft} (direkte Wechselwirkung).
\end{enumerate}

\subsection{Maxwell-Theorie als effektive Feldbeschreibung}
Die Maxwell-Gleichungen beschreiben elektromagnetische Felder als kontinuierliche Objekte im Raum. In der Informations-Weber-Theorie erscheinen diese Felder als
\emph{effektive makroskopische Beschreibungen} von Informationsflüssen.

Die Feldstärke $F_{\mu\nu}$ ist dabei kein fundamentales Objekt, sondern ein Mittelwert über lokale Informationsgradienten.

\subsection{Lorentz-Kraft als phänomenologische Näherung}
Die Lorentz-Kraft
\[
    \vec{F} = q(\vec{E} + \vec{v} \times \vec{B})
\]
entsteht als phänomenologische Näherung, wenn die Informationsstruktur durch Felder parametrisiert wird. Sie ist gültig, wenn:

\begin{itemize}
    \item retardierte Effekte klein sind,
    \item Beschleunigungen gering sind,
    \item globale Informationsstrukturen vernachlässigt werden.
\end{itemize}

\subsection{Weber-Kraft als lokaler Grenzfall}
Die Weber-Kraft
\[
    \vec{F}_{\text{Weber}}
    =
    \frac{q_1 q_2}{4\pi\varepsilon_0 r^2}
    \left[
        1
        -
        \frac{\dot{r}^2}{c^2}
        +
        \frac{2 r \ddot{r}}{c^2}
    \right]
    \hat{\vec{r}}
\]
ist der \emph{exakte lokale Grenzfall} der Informations-Weber-Theorie, wenn globale Informationsstrukturen vernachlässigt werden.

Damit ergibt sich eine klare Hierarchie:
\[
    \text{Informations-Weber-Theorie}
    \;\longrightarrow\;
    \text{Weber}
    \;\longrightarrow\;
    \text{Lorentz}
    \;\longrightarrow\;
    \text{Maxwell}.
\]

\section{Quantenmechanik als globale Informationsdynamik}
Die Quantenmechanik basiert auf der Schrödinger-Gleichung
\[
    i\hbar \partial_t \Psi = \hat{H} \Psi.
\]
In der Informations-Weber-Theorie ist die Wellenfunktion kein ontologisches Objekt, sondern eine parametrische Darstellung der Informationsdichte:
\[
    \rho_I = |\Psi|^2.
\]
Das Bohm-Potential
\[
    Q = -\frac{\hbar^2}{2m}
    \frac{\nabla^2 \sqrt{\rho_I}}{\sqrt{\rho_I}}
\]
entsteht als globaler Anteil des Informations-Lagrange-Funktionals.

Damit wird die Quantenmechanik als \emph{globaler Grenzfall} verstanden, der gültig ist, wenn:

\begin{itemize}
    \item globale Informationsstrukturen dominieren,
    \item lokale Dynamik vernachlässigt werden kann,
    \item kohärente Informationsphasen existieren.
\end{itemize}

\section{Relativitätstheorie als emergente Geometrie}
Die SRT und ART basieren auf der Idee, dass Raum und Zeit eine feste geometrische Struktur besitzen. In der Informations-Weber-Theorie ist diese Struktur nicht
fundamental, sondern emergent.

\subsection{SRT als Symmetrie des Informationsflusses}

Die Lorentz-Invarianz entsteht aus der Symmetrie des Informationsflusses bei maximaler Informationsgeschwindigkeit $c$. Sie ist gültig, wenn:

\begin{itemize}
    \item Informationsgradienten homogen sind,
    \item globale Strukturen vernachlässigt werden,
    \item keine fraktalen Effekte auftreten.
\end{itemize}

\subsection{ART als effektive Informationsgeometrie}
Die ART beschreibt Gravitation als Krümmung der Raumzeit. In der Informations-Weber-Theorie ist diese Krümmung eine effektive Beschreibung der Informationsmetriken:
\[
    g_{ij}
    =
    \frac{\partial^2 \mathcal{F}}{\partial (\partial_i \rho_I)\, \partial (\partial_j \rho_I)}.
\]
Die ART ist gültig, wenn:

\begin{itemize}
    \item Informationsdichten groß sind,
    \item globale Strukturen langsam variieren,
    \item fraktale Effekte vernachlässigt werden können.
\end{itemize}

\section{Zusammenfassung}
Die Informations-Weber-Theorie integriert die etablierten Theorien als Grenzfälle:

\begin{itemize}
    \item klassische Mechanik: lokale, niederenergetische Näherung,
    \item Weber-Elektrodynamik: exakter lokaler Grenzfall,
    \item Quantenmechanik: globaler Grenzfall,
    \item Relativitätstheorie: emergente Informationsgeometrie.
\end{itemize}

Damit entsteht ein einheitliches, reduktionistisches Bild der Physik, in dem alle bekannten Modelle als approximative Manifestationen eines fundamentalen
Informationsprinzips verstanden werden.

\section{Frequenzabhängige Lichtablenkung als Test der Theorie}
Ein zentraler Unterschied zwischen der Allgemeinen Relativitätstheorie und der Informations-Weber-Theorie betrifft die Ablenkung von Licht im Gravitationsfeld.
Während die ART eine frequenzunabhängige Ablenkung vorhersagt, ergibt sich in der Informations-Weber-Theorie eine explizite Frequenzabhängigkeit.

\subsection{Vorhersage der ART}
In der ART folgt Licht einer nullartigen Geodäte. Die Ablenkung am Sonnenrand beträgt
\[
    \delta\theta_{\text{ART}}
    =
    \frac{4GM}{c^2 b},
\]
unabhängig von Frequenz oder Energie des Photons.

\subsection{Vorhersage der Informations-Weber-Theorie}
In der Informations-Weber-Theorie besitzt ein Photon eine effektive Informationssteifigkeit, die von seiner Frequenz abhängt. Dadurch ergibt sich eine frequenzabhängige
Ablenkung:
\[
    \delta\theta(\nu)
    =
    \delta\theta_0
    \left(
        1 + \alpha \frac{\nu_0}{\nu}
    \right),
\]
wobei $\alpha$ eine dimensionslose Kopplungskonstante ist.

Damit gilt:
\[
    \delta\theta_{\text{blau}} < \delta\theta_{\text{rot}}.
\]

\subsection{Experimentelle Tests}
Die frequenzabhängige Ablenkung kann getestet werden durch:

\begin{itemize}
    \item spektral aufgelöste Sonnenrandmessungen,
    \item Gravitationslinsen im optischen, Röntgen- und Radiobereich,
    \item Pulsar-Timing und Fast Radio Bursts.
\end{itemize}

Eine nachgewiesene Frequenzabhängigkeit würde die ART falsifizieren und die
Informations-Weber-Theorie bestätigen.


% Emergenz und Vergleich
\chapter{Anwendungen und Beispiele}
\label{chap:anwendungen}

\section{Einleitung}
In diesem Kapitel werden konkrete physikalische Systeme aus der Perspektive der Informations-Weber-Theorie analysiert. Ziel ist es zu zeigen, wie klassische und
quantenmechanische Phänomene aus der Dynamik der Informationsdichte $\rho_I$ emergieren. Die Beispiele dienen nicht nur der Illustration, sondern demonstrieren
die Erklärungskraft der Theorie in Bereichen, die traditionell als grundlegend verschieden betrachtet werden: Mechanik, Gravitation, Quantenphysik und Plasmaphysik.

Jedes Beispiel folgt demselben Schema:

\begin{enumerate}
    \item Formulierung des physikalischen Systems,
    \item Darstellung der Informationsstruktur,
    \item Herleitung der Dynamik aus dem Informations-Lagrange-Funktional,
    \item Vergleich mit der etablierten Theorie,
    \item Interpretation der Ergebnisse.
\end{enumerate}

Die Beispiele sind so gewählt, dass sie sowohl klassische Grenzfälle als auch genuin quantenmechanische Phänomene abdecken.

\section{Der Doppelspalt: Interferenz als Informationsorganisation}
Das Doppelspaltexperiment gilt als eines der zentralen Phänomene der Quantenmechanik. In der Informations-Weber-Theorie entsteht das Interferenzmuster nicht durch eine
„Welle“, sondern durch die energetisch optimale Organisation der Informationsdichte.

\subsection{Informationsstruktur des Doppelspalts}
Die Informationsdichte hinter zwei Spalten ergibt sich aus der Überlagerung zweier Informationsquellen:
\[
    \rho_I = \rho_1 + \rho_2 + 2\sqrt{\rho_1 \rho_2}\cos(\Delta \phi).
\]
Die Phase $\Delta \phi$ ist eine Eigenschaft der globalen Informationsstruktur und ergibt sich aus der Minimierung des globalen Funktionals
\[
    \mathcal{F}_{\text{global}}
    =
    \gamma \frac{(\nabla \rho_I)^2}{\rho_I}.
\]

\subsection{Entstehung des Interferenzmusters}
Die Variation von $\mathcal{F}_{\text{global}}$ führt zu einer Gleichung, die das Interferenzmuster bestimmt:
\[
    \nabla^2 \sqrt{\rho_I}
    +
    k^2 \sqrt{\rho_I}
    = 0.
\]
Damit entsteht das Interferenzmuster als energetisch optimale Informationsverteilung.

\subsection{Vergleich zur Quantenmechanik}
Die Schrödinger-Gleichung liefert dasselbe Muster, jedoch mit einer ontologischen Wellenfunktion. Die Informations-Weber-Theorie zeigt, dass diese Wellenfunktion nicht
fundamental ist, sondern eine parametrische Darstellung der Informationsstruktur.

\section{Der harmonische Oszillator: Lokale vs. globale Dynamik}
Der harmonische Oszillator ist ein ideales Testsystem, um die Beziehung zwischen lokaler und globaler Informationsdynamik zu untersuchen.

\subsection{Informationspotential}
Das klassische Potential
\[
    V(x) = \frac{1}{2} m \omega^2 x^2
\]
entspricht einem Informationspotential
\[
    \Phi_I(x) = \kappa x^2,
\]
wobei $\kappa$ eine Konstante ist, die die Kopplungsstärke der Informationsstruktur beschreibt.

\subsection{Bewegungsgleichung}
Die Bewegungsgleichung ergibt sich aus
\[
    \frac{d}{dt} \left( \frac{\partial \mathcal{F}}{\partial (\partial_t \rho_I)} \right)
    =
    \frac{\partial \mathcal{F}}{\partial \rho_I}.
\]
Im Grenzfall lokaler Dynamik entsteht die klassische Oszillatorgleichung:
\[
    m \ddot{x} + m \omega^2 x = 0.
\]
Im globalen Grenzfall entsteht die quantisierte Energie:
\[
    E_n = \left(n + \frac{1}{2}\right)\hbar \omega.
\]

\subsection{Interpretation}
Die Quantisierung ist keine Eigenschaft des Potentials, sondern der globalen Informationsorganisation.

\section{Das Kepler-Problem: Gravitation als Informationsfluss}
Das Kepler-Problem beschreibt die Bewegung zweier Körper unter Newtonscher Gravitation. In der Informations-Weber-Theorie entsteht die Gravitationskraft aus
Informationsgradienten.

\subsection{Informationspotential}
Das Newton-Potential
\[
    V(r) = -\frac{GMm}{r}
\]
entspricht einem Informationspotential
\[
    \Phi_I(r)
    =
    \int \frac{\rho_I(\vec{r}')}{|\vec{r}-\vec{r}'|}\, d^3x'.
\]
Für punktförmige Informationsquellen ergibt sich
\[
    \Phi_I(r) \propto \frac{1}{r}.
\]

\subsection{Bewegungsgleichung}
Die Gravitationskraft ergibt sich aus
\[
    \vec{F}_{\text{grav}} = -\nabla \Phi_I.
\]
Damit entsteht die Kepler-Gleichung
\[
    \ddot{\vec{r}} = -\frac{GM}{r^3}\vec{r}.
\]

\subsection{Interpretation}
Gravitation ist ein Informationsfluss, nicht eine Eigenschaft der Raumzeit.

\section{Plasmawellen: Informationsgradienten in Medien}
Plasmen besitzen komplexe kollektive Dynamiken, die in der Informations-Weber-Theorie als Informationsflüsse in einem Medium beschrieben werden.

\subsection{Informationsdichte im Plasma}
Die Informationsdichte eines Plasmas ergibt sich aus der Ladungs- und Geschwindigkeitsverteilung:
\[
    \rho_I(\vec{r},t)
    =
    \rho_e(\vec{r},t)
    +
    \rho_i(\vec{r},t).
\]

\subsection{Plasmaoszillationen}
Die Variation des lokalen Funktionals führt zu einer Oszillationsgleichung
\[
    \partial_t^2 \rho_I + \omega_p^2 \rho_I = 0,
\]
wobei $\omega_p$ die Plasmafrequenz ist.

\subsection{Interpretation}
Plasmawellen sind kollektive Informationsflüsse, nicht elektromagnetische Felder.

\section{Zusammenfassung}
Die Beispiele dieses Kapitels zeigen, dass die Informations-Weber-Theorie in der Lage ist, klassische und quantenmechanische Phänomene einheitlich zu beschreiben:

\begin{itemize}
    \item Interferenz entsteht aus globaler Informationsorganisation.
    \item Der harmonische Oszillator zeigt die Dualität lokaler und globaler Dynamik.
    \item Gravitation ist ein Informationsfluss, nicht Raumzeitkrümmung.
    \item Plasmawellen sind kollektive Informationsgradienten.
\end{itemize}

Damit demonstriert die Theorie ihre Erklärungskraft über alle Skalen hinweg.

\chapter{Naturkonstanten aus Informationsarchitektur}
\label{chap:naturkonstanten}

Die Informations-Weber-Theorie beschreibt physikalische Systeme nicht durch Felder oder
Teilchen, sondern durch Informationsdichten und Informationsflüsse. In dieser Sichtweise
sind Naturkonstanten keine unabhängigen Eingabegrößen, sondern emergente Parameter der
Informationsarchitektur. Sie entstehen aus der fraktalen Struktur des Informationsnetzes,
aus Skalierungsrelationen und aus der Dynamik lokaler und globaler Informationsflüsse.

Dieses Kapitel zeigt, wie fundamentale Konstanten wie $c$, $\hbar$ und $G$ aus der
Informationsstruktur hervorgehen und warum sie in der Informations-Weber-Theorie nicht
fundamental sind.

\section{Einleitung: Warum Naturkonstanten nicht fundamental sind}

In klassischen Theorien erscheinen Naturkonstanten als unveränderliche Größen, die nicht
erklärt werden können. Die Informations-Weber-Theorie liefert eine alternative Sichtweise:

\begin{itemize}
    \item Naturkonstanten sind \emph{Skalierungsparameter} der Informationsgeometrie.
    \item Sie entstehen aus der Kopplungsstruktur des Informationsnetzes.
    \item Sie sind Konsequenzen der fraktalen Dimension
    

\[
        D = \frac{\ln 20}{\ln(2+\phi)}.
    \]


    \item Sie sind nicht fundamental, sondern emergent.
\end{itemize}

Damit wird die Frage nach dem Ursprung der Naturkonstanten zu einer Frage der
Informationsarchitektur.

\section{Die Lichtgeschwindigkeit $c$ als maximale Informationsflussrate}

In der Informations-Weber-Theorie ist die Lichtgeschwindigkeit keine ontologische Grenze,
sondern die maximale Geschwindigkeit, mit der lokale Informationsflüsse übertragen werden
können. Sie ergibt sich aus der Kopplungsdichte des Informationsnetzes.

\subsection{Informationsfluss und Kopplungsdichte}

Die lokale Informationsgeschwindigkeit ist definiert durch


\[
    \vec{J}_I = \rho_I \vec{v}_I.
\]


Die maximale Geschwindigkeit $\vec{v}_I^{\text{max}}$ ergibt sich aus der maximalen Rate,
mit der Kopplungen im Informationsnetz aktualisiert werden können.

In einem fraktalen Netz mit Dimension $D$ ergibt sich eine natürliche Skalierung:


\[
    c \propto \lambda^{D-1},
\]


wobei $\lambda$ die charakteristische Kopplungslänge ist.

\subsection{Interpretation}

\begin{itemize}
    \item $c$ ist die maximale Geschwindigkeit lokaler Informationsflüsse.
    \item $c$ ist keine fundamentale Konstante, sondern eine emergente Eigenschaft der
    Informationsarchitektur.
    \item In stark gekoppelten Regionen (z.\,B. Gravitation) kann die effektive
    Informationsgeschwindigkeit variieren.
\end{itemize}

Damit wird die Lichtgeschwindigkeit zu einer abgeleiteten Größe.

\section{Das Plancksche Wirkungsquantum $\hbar$ als Maß der Informationsgranularität}

Das Plancksche Wirkungsquantum $\hbar$ ist in der Informations-Weber-Theorie ein Maß für
die Granularität der globalen Informationsorganisation. Es entsteht aus der Struktur des
Bohm’schen Quantenpotentials.

\subsection{Informationsgranularität und globale Organisation}

Das Bohm-Potential


\[
    Q = -\frac{\hbar^2}{2m}
    \frac{\nabla^2 \sqrt{\rho_I}}{\sqrt{\rho_I}}
\]


beschreibt die systemische Organisation des Informationszustands. Die Größe $\hbar$
bestimmt die Stärke dieser globalen Kopplung.

In einem fraktalen Informationsnetz ergibt sich


\[
    \hbar \propto \lambda^{2-D},
\]


wobei $\lambda$ die charakteristische Netzskala ist.

\subsection{Interpretation}

\begin{itemize}
    \item $\hbar$ misst die Stärke globaler Informationsorganisation.
    \item $\hbar$ ist kein fundamentales Wirkungsquantum, sondern ein Skalierungsparameter.
    \item Die Quantenmechanik entsteht als Grenzfall globaler Informationsdynamik.
\end{itemize}

Damit wird die Quantenstruktur zu einer emergenten Eigenschaft des Informationsraums.

\section{Die Gravitationskonstante $G$ als Kopplungsparameter der Informationsgeometrie}

Die Gravitationskonstante $G$ entsteht aus der Kopplungsstruktur des Informationsnetzes,
insbesondere aus der fraktalen Geometrie der Masseverteilung.

\subsection{Weber-Gravitation und Informationskopplung}

Die Weber-Gravitationskraft hat die Form


\[
    F_{\text{WG}}
    =
    -G \frac{m_1 m_2}{r^2}
    \left(
        1 - \frac{\dot{r}^2}{2c^2} + \frac{r \ddot{r}}{c^2}
    \right).
\]



In der Informations-Weber-Theorie ist $G$ kein unabhängiger Parameter, sondern eine
Konsequenz der Kopplungsstärke des Informationsnetzes:


\[
    G \propto \lambda^{3-D}.
\]



\subsection{Interpretation}

\begin{itemize}
    \item $G$ misst die Stärke der geometrischen Kopplung im Informationsnetz.
    \item Gravitation ist eine emergente Eigenschaft der Informationsgeometrie.
    \item $G$ ist nicht fundamental, sondern skalenabhängig.
\end{itemize}

Damit wird die Gravitation zu einer Konsequenz der Informationsarchitektur.

\section{Weitere Naturkonstanten}

\subsection{Die Feinstrukturkonstante $\alpha$}
Die Feinstrukturkonstante
\[
    \alpha = \frac{e^2}{4\pi \varepsilon_0 \hbar c}
\]
ist in der Informations-Weber-Theorie ein Maß für die relative Stärke lokaler und globaler Informationsflüsse.

Da sowohl $c$ als auch $\hbar$ emergent sind, ist auch $\alpha$ eine emergente Größe:
\[
    \alpha \propto \lambda^{D-3}.
\]

\subsection{Die Elementarladung $e$}
Die Elementarladung ist ein Maß für die lokale Kopplungsstärke im Informationsnetz. Sie ergibt sich aus der minimalen Änderung der Informationsdichte, die eine lokale
Wechselwirkung erzeugen kann.

\subsection{Die Boltzmann-Konstante $k_B$}
Die Boltzmann-Konstante misst die Beziehung zwischen Informationsentropie und Energie. In der Informations-Weber-Theorie ist sie ein Maß für die Umrechnung zwischen lokaler
Informationsentropie und makroskopischer Energie.

\section{Zusammenfassung}
Die Informations-Weber-Theorie zeigt, dass Naturkonstanten keine fundamentalen Größen sind. Sie entstehen aus der fraktalen Struktur des Informationsnetzes und aus der Dynamik
lokaler und globaler Informationsflüsse. Die Lichtgeschwindigkeit $c$, das Wirkungsquantum $\hbar$, die Gravitationskonstante $G$ und weitere Konstanten sind Skalierungsparameter der
Informationsarchitektur. Damit wird die Physik zu einer Theorie der Information, in der Naturkonstanten nicht postuliert, sondern erklärt werden.


% Anwendungen und Konsequenzen
\chapter{Experimentelle Vorhersagen und Tests}
\label{chap:vorhersagen}

Eine fundamentale Theorie muss nicht nur konsistent und widerspruchsfrei sein, sondern auch überprüfbare Vorhersagen machen. Die Informations-Weber-Theorie erfüllt dieses
Kriterium in besonderem Maße: Sie liefert klare, quantitative und qualitative Aussagen, die sich von den Vorhersagen der Allgemeinen Relativitätstheorie (ART), der
Quantenmechanik (QM) und der Standardkosmologie unterscheiden. Dieses Kapitel fasst die wichtigsten experimentellen Konsequenzen zusammen und zeigt, wie die Theorie empirisch
getestet werden kann.

\section{Einleitung: Testbarkeit einer Informations-Urtheorie}
Die Informations-Weber-Theorie ist keine spekulative Erweiterung bestehender Modelle, sondern eine Urtheorie, aus der klassische und quantenmechanische Phänomene als
Grenzfälle hervorgehen. Ihre Vorhersagen betreffen:

\begin{itemize}
    \item kosmologische Beobachtungen,
    \item gravitative Effekte,
    \item quantenmechanische Phänomene,
    \item Plasmaprozesse,
    \item die Struktur der Naturkonstanten.
\end{itemize}

Viele dieser Vorhersagen unterscheiden sich deutlich von ART, QFT und Standardkosmologie und ermöglichen daher klare experimentelle Tests.

\section{Vorhersagen, die der ART widersprechen}
\subsection{Keine echten Singularitäten}
Die Informations-Weber-Theorie postuliert eine minimale Informationsdichte
\[
    \rho_I^{\text{min}} > 0,
\]
wodurch echte Singularitäten ausgeschlossen sind. Dies führt zu folgenden Vorhersagen:

\begin{itemize}
    \item Schwarze Löcher besitzen einen informationsbasierten Kern statt einer Singularität.
    \item Die Raumzeitkrümmung bleibt endlich.
    \item Der Urknall wird durch einen Big Bounce ersetzt.
\end{itemize}

\subsection{Abweichungen bei extremen Gravitationsfeldern}
In Bereichen hoher Kopplungsdichte (z.\,B. nahe kompakter Objekte) ergeben sich Abweichungen von der ART:

\begin{itemize}
    \item modifizierte Lichtablenkung,
    \item veränderte Gravitationsrotverschiebung,
    \item Abweichungen in der Bahnpräzession.
\end{itemize}

Diese Effekte sind messbar, sobald die Informationsgeometrie von der effektiven Kontinuumsgeometrie der ART abweicht.

\section{Vorhersagen, die der Quantenfeldtheorie widersprechen}
\subsection{Keine virtuellen Teilchen}

Die Informations-Weber-Theorie benötigt keine virtuellen Photonen oder Feldquanten. Stattdessen entstehen Wechselwirkungen durch Informationsflüsse. Daraus folgt:

\begin{itemize}
    \item keine divergenten Selbstenergien,
    \item keine Renormierung als fundamentales Prinzip,
    \item keine überlichtschnellen Pfadintegral-Komponenten.
\end{itemize}

\subsection{Nichtlokalität ohne Widerspruch zur Kausalität}
Das Bohm’sche Quantenpotential beschreibt globale Informationsorganisation. Die Theorie sagt daher:

\begin{itemize}
    \item EPR-Korrelationen sind Ausdruck systemischer Ganzheit,
    \item keine Signale werden überlichtschnell übertragen,
    \item die Kausalität bleibt auf lokaler Ebene erhalten.
\end{itemize}

\section{Kosmologische Tests}
\subsection{CMB-Fraktalität}
Die Informations-Weber-Theorie sagt voraus, dass die CMB-Anisotropien fraktale Korrelationen aufweisen, die aus der fraktalen Dimension
\[
    D = \frac{\ln 20}{\ln(2+\phi)}
\]
resultieren. Messbare Konsequenzen:

\begin{itemize}
    \item Abweichungen von rein statistisch-gaussianischen Fluktuationen,
    \item fraktale Korrelationslängen,
    \item keine akustischen Peaks im klassischen Sinn.
\end{itemize}

\subsection{Rotverschiebung ohne Expansion}
Die Theorie sagt voraus, dass Rotverschiebungen durch Informationsumstrukturierung entstehen. Testbare Konsequenzen:

\begin{itemize}
    \item Rotverschiebung ist nicht streng proportional zur Entfernung,
    \item Abweichungen bei sehr hohen Rotverschiebungen,
    \item mögliche Abhängigkeit von Plasma- und Informationsdichte.
\end{itemize}

\subsection{Galaktische Rotationskurven ohne Dunkle Materie}
Die fraktale Informationsgeometrie erzeugt effektive zusätzliche Beschleunigungen. Vorhersagen:

\begin{itemize}
    \item flache Rotationskurven ohne Dunkle Materie,
    \item Tully-Fisher-Relation als Informationsgesetz,
    \item Abweichungen in Zwerggalaxien und Low-Surface-Brightness-Galaxien.
\end{itemize}

\section{Labor- und Plasma-Experimente}
\subsection{Weber-Effekte in Laborplasmen}

Die geschwindigkeits- und beschleunigungsabhängigen Terme der Weber-Kraft führen zu messbaren Effekten:

\begin{itemize}
    \item anisotrope Transportprozesse,
    \item nichtlineare Plasmaoszillationen,
    \item Abweichungen von Maxwell-basierten Modellen.
\end{itemize}

\subsection{Informationsflüsse in turbulenten Plasmen}
Die Theorie sagt voraus:

\begin{itemize}
    \item fraktale Skalenhierarchien,
    \item selbstorganisierte Filamentstrukturen,
    \item Abweichungen von klassischer MHD.
\end{itemize}

\section{Zusammenfassung}
Die Informations-Weber-Theorie macht eine Vielzahl klarer, überprüfbarer Vorhersagen, die sich von ART, QFT und Standardkosmologie unterscheiden. Besonders relevant sind:

\begin{itemize}
    \item keine Singularitäten,
    \item Big Bounce statt Big Bang,
    \item fraktale CMB-Struktur,
    \item Rotverschiebung ohne Expansion,
    \item Rotationskurven ohne Dunkle Materie,
    \item Abweichungen in Laborplasmen,
    \item keine virtuellen Teilchen.
\end{itemize}

Diese Vorhersagen machen die Informations-Weber-Theorie zu einer empirisch testbaren Urtheorie, die klassische und quantenmechanische Phänomene in einem einheitlichen
informationsbasierten Rahmen beschreibt.

\chapter{Experimentelle Vorhersagen und Tests}
\label{chap:tests}

\paragraph{Hinweis zur mathematischen Darstellung}
Dieses Kapitel verwendet größtenteils die \emph{kontinuierliche Notation} für Kompaktheit. Die zugrundeliegende fundamentale Formulierung ist diskret rekursiv. Wo nötig
wird die diskrete Form explizit angegeben. Eine vollständige diskrete Darstellung findet sich in Kapitel X.

\section{Einleitung: Testbarkeit einer informationsbasierten Urtheorie}
Eine fundamentale Theorie muss nicht nur konzeptionell konsistent sein, sondern auch \emph{experimentell überprüfbare Vorhersagen} machen. Die Informations-Weber-Theorie
erfüllt dieses Kriterium in besonderem Maße: Sie liefert klare, quantitative und qualitative Vorhersagen, die sich von denen der \gls{art}, der \gls{qft} und der
Standardkosmologie unterscheiden.

Die Testbarkeit ergibt sich aus drei Ebenen:
\begin{enumerate}
    \item \textbf{lokale Dynamik}  
    (Weber-Kraft, Informationsflüsse, Plasmaeffekte),

    \item \textbf{globale Informationsorganisation}
    (Nichtlokalität, Quantenstruktur, Bohm-Potential),

    \item \textbf{Informationsgeometrie}  
    (fraktale Raumstruktur, emergente Metrik, Naturkonstanten).
\end{enumerate}
Diese drei Ebenen erzeugen experimentelle Signaturen, die in etablierten Theorien nicht auftreten.

\section{Vorhersagen, die der ART widersprechen}
\subsection{Keine echten Singularitäten}
Die Informations-Weber-Theorie postuliert eine minimale Informationsdichte:
\[
    \rho_I^{\text{min}} > 0,
\]
wodurch echte Singularitäten ausgeschlossen sind. Dies führt zu folgenden Vorhersagen:
\begin{itemize}
    \item Schwarze Löcher besitzen einen informationsbasierten Kern statt einer Singularität.
    \item Die effektive Krümmung bleibt endlich.
    \item Der Urknall wird durch einen Big Bounce ersetzt.
\end{itemize}

\subsection{Abweichungen bei extremen Gravitationsfeldern}
In Bereichen hoher Kopplungsdichte ergeben sich Abweichungen von der \gls{art}:
\begin{itemize}
    \item modifizierte Lichtablenkung,
    \item veränderte Gravitationsrotverschiebung,
    \item Abweichungen in der Bahnpräzession.
\end{itemize}
Diese Effekte treten auf, sobald die informationsbasierte Geometrie von der makroskopischen Kontinuumsgeometrie der \gls{art} abweicht.

\subsection{Frequenzabhängige Lichtablenkung}
Die \gls{art} sagt eine frequenzunabhängige Ablenkung voraus:
\[
    \delta\theta_{\text{ART}} = \frac{4GM}{c^2 b}.
\]
Die Informations-Weber-Theorie sagt dagegen:
\[
    \delta\theta(\nu)
    =
    \delta\theta_0
    \left(
        1 + \alpha \frac{\nu_0}{\nu}
    \right),
\]
wobei hochfrequente Photonen \emph{weniger} abgelenkt werden als niederfrequente.

Messmethoden:
\begin{itemize}
    \item spektral aufgelöste Sonnenrandmessungen,
    \item Gravitationslinsen im optischen, Röntgen- und Radiobereich,
    \item Pulsar-Timing und Fast Radio Bursts.
\end{itemize}

\section{Vorhersagen, die der Quantenfeldtheorie widersprechen}
\subsection{Keine virtuellen Teilchen}
Die Informations-Weber-Theorie benötigt keine virtuellen Photonen oder Feldquanten. Wechselwirkungen entstehen durch Informationsflüsse. Daraus folgt:
\begin{itemize}
    \item keine divergenten Selbstenergien,
    \item keine Renormierung als fundamentales Prinzip,
    \item keine überlichtschnellen Pfadintegral-Komponenten.
\end{itemize}

\subsection{Nichtlokalität ohne Verletzung der Kausalität}
Das Bohm-Potential beschreibt globale Informationsorganisation. Die Theorie sagt:
\begin{itemize}
    \item EPR-Korrelationen sind Ausdruck systemischer Ganzheit,
    \item keine Signale werden überlichtschnell übertragen,
    \item lokale Kausalität bleibt erhalten.
\end{itemize}

\section{Kosmologische Tests}
\subsection{CMB-Fraktalität}
Die Informations-Weber-Theorie sagt voraus, dass die CMB-Anisotropien fraktale Korrelationen aufweisen, die aus der fraktalen Dimension
\[
    D = \frac{\ln 20}{\ln(2+\phi)}
\]
resultieren.

Messbare Konsequenzen:
\begin{itemize}
    \item Abweichungen von rein gaussianischen Fluktuationen,
    \item fraktale Korrelationslängen,
    \item modifizierte akustische Strukturen.
\end{itemize}

\subsection{Rotverschiebung ohne Expansion}
Die Theorie sagt voraus, dass Rotverschiebungen durch Informationsumstrukturierung entstehen.

Testbare Konsequenzen:
\begin{itemize}
    \item Rotverschiebung ist nicht streng proportional zur Entfernung,
    \item Abweichungen bei sehr hohen Rotverschiebungen,
    \item mögliche Abhängigkeit von Plasma- und Informationsdichte.
\end{itemize}

\subsection{Galaktische Rotationskurven ohne Dunkle Materie}
Die fraktale Informationsgeometrie erzeugt effektive zusätzliche Beschleunigungen.

Vorhersagen:
\begin{itemize}
    \item flache Rotationskurven ohne Dunkle Materie,
    \item Tully-Fisher-Relation als Informationsgesetz,
    \item Abweichungen in Zwerggalaxien und Low-Surface-Brightness-Galaxien.
\end{itemize}

\section{Labor- und Plasma-Experimente}
\subsection{Weber-Effekte in Laborplasmen}
Die geschwindigkeits- und beschleunigungsabhängigen Terme der Weber-Kraft führen zu messbaren Effekten:
\begin{itemize}
    \item anisotrope Transportprozesse,
    \item nichtlineare Plasmaoszillationen,
    \item Abweichungen von Maxwell-basierten Modellen.
\end{itemize}

\subsection{Informationsflüsse in turbulenten Plasmen}
Die Theorie sagt voraus:
\begin{itemize}
    \item fraktale Skalenhierarchien,
    \item selbstorganisierte Filamentstrukturen,
    \item Abweichungen von klassischer MHD.
\end{itemize}

\section{Zusammenfassung}
Kapitel~\ref{chap:tests} hat gezeigt:
\begin{itemize}
    \item Die Informations-Weber-Theorie ist experimentell testbar.
    \item Sie macht klare Vorhersagen, die der \gls{art}, \gls{qft} und Standardkosmologie widersprechen.
    \item Sie erklärt kosmologische Phänomene ohne Dunkle Materie und ohne Expansion.
    \item Sie liefert neue Tests in Plasmaphysik, Lensing und CMB-Analyse.
    \item Sie benötigt keine virtuellen Teilchen und keine Renormierung.
\end{itemize}
Damit ist die Theorie nicht nur konzeptionell und analytisch, sondern auch empirisch überprüfbar.

\chapter{Numerische Simulation der Informationsgeometrie}
\label{chap:simulation}

\paragraph{Hinweis zur mathematischen Darstellung}
Dieses Kapitel verwendet größtenteils die \emph{kontinuierliche Notation} für Kompaktheit. Die zugrundeliegende fundamentale Formulierung ist diskret rekursiv. Wo nötig
wird die diskrete Form explizit angegeben. Eine vollständige diskrete Darstellung findet sich in Kapitel X.

\section{Einleitung}
Die Informations-Weber-Theorie ist nicht nur analytisch formuliert, sondern auch algorithmisch implementierbar. Da Raum, Zeit und Dynamik aus einem diskreten
Informationsnetz emergieren, lässt sich die Theorie direkt numerisch simulieren.

Dieses Kapitel beschreibt die Grundprinzipien einer solchen Simulation:
\begin{itemize}
    \item die diskrete Informationsstruktur,
    \item die Aktualisierungsregeln,
    \item die Berechnung der Informationsmetrik,
    \item die Simulation lokaler und globaler Dynamik,
    \item die Rekonstruktion emergenter Größen wie Raum, Zeit und Wellen.
\end{itemize}
Damit wird die Informations-Weber-Theorie operational: Sie kann berechnet, visualisiert und experimentell getestet werden.

\section{Das diskrete Informationsnetz}
Die digitale Informations-Weber-Theorie beschreibt den physikalischen Zustand durch ein Netzwerk aus:
\begin{itemize}
    \item \textbf{Knoten} $i$ mit Informationswerten $\rho_I(i)$,
    \item \textbf{Kopplungen} $K_{ij}$ zwischen Knoten,
    \item \textbf{Aktualisierungsregeln} für $\rho_I$ und $K_{ij}$.
\end{itemize}
Dieses Netz ist kein Abbild des Raumes — es \emph{erzeugt} den Raum.

\subsection{Knoten}
Jeder Knoten repräsentiert eine elementare Informationszelle. Die Informationsdichte $\rho_I(i)$ ist die fundamentale Größe.

\subsection{Kopplungen}
Die Kopplungen $K_{ij}$ bestimmen:
\begin{itemize}
    \item die Stärke lokaler Informationsflüsse,
    \item die Reichweite globaler Informationsorganisation,
    \item die emergente Geometrie.
\end{itemize}

\subsection{Aktualisierungsregeln}
Die Dynamik besteht aus diskreten Aktualisierungsschritten:
\[
    \rho_I^{(n+1)} = T[\rho_I^{(n)}].
\]
Die Transformation $T$ ist die diskrete Version des Informations-Lagrange-Funktionals.

\section{Diskrete Form des Informations-Lagrange-Funktionals}
Das kontinuierliche Funktional aus Kapitel~\ref{chap:lagrange} wird diskretisiert:
\[
    \mathcal{L}_I
    =
    \sum_i
    \mathcal{F}
    \!\left(
        \rho_I(i),\,
        \Delta \rho_I(i),\,
        \delta_t \rho_I(i)
    \right).
\]
Dabei sind:
\begin{itemize}
    \item $\Delta \rho_I(i)$ diskrete Gradienten,
    \item $\delta_t \rho_I(i)$ zeitliche Differenzen.
\end{itemize}
Die Variation führt zu diskreten Euler-Lagrange-Gleichungen:
\[
    \delta_t
    \left(
        \frac{\partial \mathcal{F}}{\partial (\delta_t \rho_I)}
    \right)
    +
    \Delta
    \left(
        \frac{\partial \mathcal{F}}{\partial (\Delta \rho_I)}
    \right)
    -
    \frac{\partial \mathcal{F}}{\partial \rho_I}
    = 0.
\]
Diese Gleichungen bestimmen die Aktualisierungsregeln.

\section{Berechnung der Informationsmetrik}
Die Informationsmetrik aus Kapitel~\ref{chap:informationsmetrik} wird diskretisiert:
\[
    g_{ij}
    =
    \frac{\partial^2 \mathcal{F}}
    {\partial (\Delta_i \rho_I)\, \partial (\Delta_j \rho_I)}.
\]
Die Metrik entsteht also aus der Sensitivität der Informationsstruktur gegenüber räumlichen Änderungen.

\subsection{Rekonstruktion des Raumes}
Der emergente Raum entsteht durch:
\begin{enumerate}
    \item Berechnung der Metrik $g_{ij}$,
    \item Bestimmung der effektiven Abstände,
    \item Einbettung in eine dreidimensionale Darstellung.
\end{enumerate}
Damit wird der physikalische Raum aus der Informationsstruktur rekonstruiert.

\section{Simulation lokaler Dynamik}
Die lokale Dynamik entspricht der Weber-Struktur als lokaler Projektion der Informationsgeometrie:
\begin{itemize}
    \item geschwindigkeitsabhängige Kopplungen,
    \item beschleunigungsabhängige Reaktionskräfte,
    \item lokale Informationsflüsse.
\end{itemize}
Diese entstehen aus dem lokalen Anteil des diskreten Funktionals.

\section{Simulation globaler Dynamik}
Die globale Dynamik entspricht dem Bohm-Potential als systemischer Projektion:
\begin{itemize}
    \item globale Informationsorganisation,
    \item nichtlokale Kopplungen,
    \item Interferenzstrukturen.
\end{itemize}
Diese entstehen aus dem globalen Anteil des Funktionals.

\section{Emergenz von Wellen und Gravitationsmoden}
Wellen entstehen als kollektive Moden der Informationsgeometrie:
\begin{itemize}
    \item lokale Moden → elektromagnetische Wellen,
    \item globale Moden → Quantenwellen,
    \item geometrische Moden → Gravitationswellen.
\end{itemize}
Die Simulation zeigt:
\begin{itemize}
    \item Interferenzmuster,
    \item fraktale Skalierungsstrukturen,
    \item dispersive Effekte,
    \item frequenzabhängige Ablenkung.
\end{itemize}

\section{Numerische Stabilität und fraktale Skalierung}
Die fraktale Dimension
\[
    D = \frac{\ln 20}{\ln(2+\phi)}
\]
bestimmt:
\begin{itemize}
    \item die Skalierung der Kopplungen,
    \item die Stabilität der Simulation,
    \item die Übergänge zwischen Mikro- und Makrophysik.
\end{itemize}

\section{Beispiel: Simulation eines Doppelspalts}
Die Simulation reproduziert:
\begin{itemize}
    \item Interferenz ohne Wellenfunktion,
    \item Nichtlokalität ohne Kollaps,
    \item fraktale Feinstrukturen im Muster.
\end{itemize}

\section{Beispiel: Simulation eines Gravitationspotentials}
Die Simulation zeigt:
\begin{itemize}
    \item emergente Newton-Potentiale,
    \item fraktale Abweichungen bei kleinen Skalen,
    \item frequenzabhängige Lichtablenkung.
\end{itemize}

\section{Zusammenfassung}
Kapitel~\ref{chap:simulation} hat gezeigt:
\begin{itemize}
    \item Die Informations-Weber-Theorie ist numerisch simulierbar.
    \item Raum, Zeit und Dynamik entstehen aus einem diskreten Informationsnetz.
    \item Die Informationsmetrik ermöglicht die Rekonstruktion des Raumes.
    \item Lokale und globale Dynamik entstehen aus dem diskreten Lagrange-Funktional.
    \item Wellen, Gravitation und Nichtlokalität sind kollektive Moden der Informationsgeometrie.
\end{itemize}
Damit ist die Theorie nicht nur konzeptionell und analytisch, sondern auch algorithmisch vollständig formuliert.

\chapter{Plasmaphysik und Informationsdynamik}
\label{chap:plasmaphysik}

\section{Einleitung}
Plasmen spielen in der Informations-Weber-Theorie eine zentrale Rolle. Während die klassische Plasmaphysik elektromagnetische Felder als fundamentale Objekte betrachtet,
interpretiert die informationsbasierte Theorie Plasmen als dynamische Informationsnetze. Ladungsfluktuationen, Ströme und Wellen erscheinen nicht als Felder im Raum,
sondern als Ausdruck lokaler und globaler Informationsflüsse.

Dieses Kapitel zeigt:
\begin{itemize}
    \item wie die Weber-Dynamik lokale Plasmaeffekte beschreibt,
    \item wie die digitale Informationsgeometrie fraktale Plasmastrukturen erklärt,
    \item warum Plasmen für kosmologische Anwendungen relevant sind,
    \item wie CMB-Struktur, Rotverschiebung und Rotationskurven aus\\Plasma-Informationsprozessen verstanden werden können.
\end{itemize}
Damit wird die Plasmaphysik zu einem natürlichen Anwendungsgebiet der Informations-Weber-Theorie.

\section{Plasma als Informationsmedium}
Ein Plasma besteht aus freien Ladungsträgern, deren Bewegung durch lokale und globale Informationsflüsse bestimmt wird. Die Informations-Weber-Theorie beschreibt diese Dynamik
durch die Kopplung von Informationsdichte $\rho_I$ und Informationsstrom $\vec{J}_I$:
\[
    \frac{\partial \rho_I}{\partial t} + \nabla \cdot \vec{J}_I = 0.
\]
In einem Plasma ist $\rho_I$ nicht nur ein Maß für Ladungs- oder Energiedichte, sondern für die gesamte strukturelle Organisation des Systems. Plasmen sind daher
natürliche Informationsmedien, in denen:
\begin{itemize}
    \item lokale Weber-Dynamik (direkte Wechselwirkungen),
    \item globale Bohm-Dynamik (systemische Organisation),
    \item fraktale Informationsgeometrie (Skalenstruktur)
\end{itemize}
gleichzeitig wirken.

\section{WED im Plasma}
Die Weber-Kraft beschreibt die direkte Wechselwirkung zwischen Ladungen ohne Felder:
\[
    \vec{F}_{\text{WED}}
    =
    \frac{q_1 q_2}{4\pi \varepsilon_0 r^2}
    \left(
        1 - \frac{\dot{r}^2}{2c^2} + \frac{r \ddot{r}}{c^2}
    \right)\hat{r}.
\]
In Plasmen führt diese Struktur zu charakteristischen Effekten:
\begin{itemize}
    \item \textbf{Geschwindigkeitsabhängige Kopplung:}  
    Die Wechselwirkung hängt von der relativen Bewegung der Ladungen ab. Dies erzeugt
    anisotrope Transportprozesse und nichtlineare Wellenphänomene.

    \item \textbf{Beschleunigungsabhängige Kopplung:}  
    Die Reaktion des Plasmas auf schnelle Änderungen der Informationsstruktur erzeugt
    kollektive Moden, die in der klassischen Plasmaphysik als „Felder“ interpretiert werden.

    \item \textbf{Fernwirkung ohne Felder:}  
    Viele klassische Plasmaeffekte (Debye-Abschirmung, Plasmaoszillationen) ergeben sich
    direkt aus der Weber-Dynamik, ohne dass elektromagnetische Felder als ontologische
    Objekte benötigt werden.
\end{itemize}
Damit erscheint das Plasma nicht als Feldmedium, sondern als dynamisches Informationsnetz.

\section{Informationsgeometrie in Plasmen}
Die digitale Informations-Weber-Theorie beschreibt Plasmen als Netzwerke von Informationsknoten und Kopplungen. Die effektive Geometrie dieses Netzes wird durch die
fraktale Dimension
\[
    D = \frac{\ln 20}{\ln(2+\phi)}
\]
bestimmt. Plasmen zeigen in vielen Situationen fraktale Strukturen:
\begin{itemize}
    \item Filamentierung,
    \item Jets und Ströme,
    \item selbstorganisierte Magnetstrukturen,
    \item turbulente Skalenhierarchien.
\end{itemize}
Diese Strukturen sind Ausdruck der Informationsarchitektur des Plasmas. Die fraktale Dimension bestimmt:
\begin{itemize}
    \item wie Informationsflüsse über Skalen hinweg organisiert werden,
    \item warum Plasmen universell ähnliche Muster zeigen,
    \item warum Plasmaeffekte von Labor- bis Kosmosskalen vergleichbare Strukturen besitzen.
\end{itemize}

\section{Plasma-Kosmologie und Informations-Weber-Theorie}
Die Informations-Weber-Theorie liefert eine natürliche Verbindung zwischen Plasmaphysik und Kosmologie. Viele kosmologische Phänomene lassen sich als Informationsprozesse
in einem großskaligen Plasma interpretieren.

\subsection{CMB-Struktur aus Informationsgeometrie}
Die anisotrope Struktur der \gls{cmb} spiegelt die fraktale Informationsgeometrie des frühen Plasmas wider. Die beobachteten Fluktuationen können als fossilierte Muster
der Informationskopplungen im Plasma verstanden werden.

\subsection{Rotverschiebung ohne Expansion}
In einem informationsbasierten Plasma entstehen Rotverschiebungen durch:
\begin{itemize}
    \item Informationsumstrukturierung entlang des Weges eines Photons,
    \item dispersive Effekte der Informationsgeometrie,
    \item fraktale Kopplungsprozesse.
\end{itemize}
Die Rotverschiebung kann damit als Effekt der Informationsdynamik in einem großskaligen Plasma interpretiert werden.

\subsection{Galaxienbildung und Rotationskurven}
Die fraktale Informationsstruktur eines kosmischen Plasmas erzeugt effektive zusätzliche Beschleunigungen, die flache Rotationskurven erklären können, ohne Dunkle Materie
zu postulieren.

Die Informations-Weber-Theorie liefert damit eine informationsbasierte Perspektive auf:
\begin{itemize}
    \item galaktische Dynamik,
    \item Filamentstrukturen,
    \item Clusterbildung,
    \item Jets und Magnetstrukturen.
\end{itemize}

\section{Zusammenfassung}
Plasmen sind in der Informations-Weber-Theorie keine klassischen Feldmedien, sondern dynamische Informationsnetze. Die Weber-Dynamik beschreibt lokale Wechselwirkungen,
die digitale Informationsgeometrie beschreibt globale Strukturen.

Viele kosmologische Phänomene — CMB, Rotverschiebung, Rotationskurven — können aus der Informationsarchitektur eines großskaligen Plasmas verstanden werden. Damit wird die
Plasmaphysik zu einem zentralen Bestandteil der informationsbasierten Urtheorie.

\chapter{Kosmologie in der Informations-Weber-Theorie}
\label{chap:kosmologie}

\section{Einleitung}
Die Informations-Weber-Theorie liefert eine alternative kosmologische Struktur, die ohne fundamentale Annahmen wie Urknall-Singularität, Dunkle Materie oder Dunkle Energie
auskommt. Statt eines expandierenden Raumes beschreibt sie das Universum als dynamisches Informationsnetz, dessen fraktale Geometrie und Kopplungsstruktur die beobachteten
kosmologischen Phänomene hervorbringen.

Die zentrale Idee lautet:
\[
    \textbf{Kosmologie ist Informationsdynamik auf größten Skalen.}
\]
In diesem Kapitel werden die wichtigsten kosmologischen Konsequenzen der Theorie hergeleitet:

\begin{itemize}
    \item CMB-Struktur als fossilierte Informationsgeometrie,
    \item Rotverschiebung als informationsdynamischer Effekt,
    \item galaktische Rotationskurven ohne Dunkle Materie,
    \item großskalige Strukturbildung aus fraktaler Informationskopplung,
    \item Gravitationslinsen als Moden der Informationsgeometrie,
    \item Big Bounce statt Urknall.
\end{itemize}

\section{Das Universum als Informationsnetz}
In der digitalen Informations-Weber-Theorie besteht das Universum aus einem Netzwerk von Informationsknoten und Kopplungen. Die fraktale Dimension
\[
    D = \frac{\ln 20}{\ln(2+\phi)}
\]
bestimmt die Skalierungsstruktur dieses Netzes. Raum, Zeit und Dynamik sind emergente Eigenschaften dieser Informationsarchitektur.

\subsection{Keine fundamentale Raumzeit}
Die Theorie ersetzt:
\begin{itemize}
    \item die Raumzeit der ART durch eine emergente Informationsgeometrie,
    \item die Expansion des Raumes durch Informationsflüsse,
    \item Singularitäten durch fraktale Kernstrukturen.
\end{itemize}

\subsection{Kosmologie als fraktale Informationsdynamik}
Die großskalige Struktur des Universums entsteht aus:
\begin{itemize}
    \item fraktaler Kopplung,
    \item globaler Informationsorganisation,
    \item kollektiven Moden der Informationsgeometrie.
\end{itemize}

\section{CMB-Struktur als fossilierte Informationsgeometrie}
Die kosmische Hintergrundstrahlung (CMB) wird in der Standardkosmologie als thermisches Relikt des Urknalls interpretiert. In der Informations-Weber-Theorie kann sie als
fossilierte Struktur der frühen Informationsgeometrie eines großskaligen Plasmas verstanden werden.

\subsection{Fraktale Signaturen}
Die Theorie sagt voraus:
\begin{itemize}
    \item die CMB-Anisotropien besitzen fraktale Skalierungsgesetze,
    \item die Fluktuationen sind nicht rein thermisch,
    \item die Muster spiegeln die Kopplungsstruktur des frühen Informationsnetzes wider.
\end{itemize}

\subsection{Keine Inflation notwendig}
Die Informations-Weber-Theorie benötigt keine Inflation, da:
\begin{itemize}
    \item globale Informationskopplung instantan wirkt,
    \item Homogenität und Isotropie aus systemischer Ganzheit folgen,
    \item die fraktale Struktur die beobachteten Skalenrelationen erklärt.
\end{itemize}

\section{Rotverschiebung ohne Expansion}
Die Standardkosmologie interpretiert die Rotverschiebung als Folge der Expansion des Universums. Die Informations-Weber-Theorie bietet eine alternative Erklärung:

\subsection{Informationsdynamische Rotverschiebung}
Ein Photon verliert Energie durch:
\begin{itemize}
    \item fraktale Kopplungsprozesse,
    \item Informationsumstrukturierung entlang seines Weges,
    \item dispersive Effekte der Informationsgeometrie.
\end{itemize}
\[
    z = z_{\text{info}}(\text{Pfadlänge}, D, \text{Kopplungsstruktur})
\]

\subsection{Keine kosmische Expansion notwendig}
Die beobachtete Rotverschiebung kann ohne:
\begin{itemize}
    \item expandierenden Raum,
    \item Dunkle Energie,
    \item kosmologische Konstante
\end{itemize}
erklärt werden, wenn Informationsprozesse entlang des Photonenpfades berücksichtigt werden.

\section{Galaktische Rotationskurven ohne Dunkle Materie}
Die Informations-Weber-Theorie erklärt flache Rotationskurven durch:
\begin{itemize}
    \item fraktale Verstärkung der gravitativen Informationsflüsse,
    \item zusätzliche effektive Beschleunigungen aus der Informationsgeometrie,
    \item nichtlokale Kopplungen im galaktischen Plasma.
\end{itemize}
Damit wird Dunkle Materie als eigenständige Substanz überflüssig.

\section{Großskalige Strukturbildung}
Die fraktale Informationsarchitektur erzeugt:
\begin{itemize}
    \item Filamente,
    \item Voids,
    \item Cluster,
    \item Jets,
    \item selbstähnliche Muster über viele Skalen.
\end{itemize}
Diese Strukturen entstehen nicht primär aus Gravitationsinstabilitäten, sondern aus:
\begin{itemize}
    \item fraktaler Kopplung,
    \item Plasma-Informationsdynamik,
    \item globaler Organisation.
\end{itemize}

\section{Gravitationslinsen als Moden der Informationsgeometrie}
Gravitationslinsen entstehen in der Informations-Weber-Theorie nicht durch Krümmung eines ontologischen Raumes, sondern durch Moden der Informationsgeometrie.

\subsection{Vorhersage: Frequenzabhängige Lichtablenkung}
Die Theorie sagt:
\begin{itemize}
    \item hochfrequente Photonen werden stärker abgelenkt,
    \item niederfrequente Photonen werden schwächer abgelenkt.
\end{itemize}
Dies ist ein klarer Unterschied zur ART.

\section{Big Bounce statt Urknall}
Die Informations-Weber-Theorie kennt keine Singularitäten. Statt eines Urknalls entsteht ein zyklisches Modell:
\begin{itemize}
    \item Kontraktion des Informationsnetzes,
    \item fraktale Kernstruktur verhindert Singularität,
    \item Reorganisation der Informationskopplungen,
    \item Expansion der Informationsgeometrie.
\end{itemize}
Dies ist ein \emph{Big Bounce}, kein Urknall.

\section{Zusammenfassung}
Kapitel~\ref{chap:kosmologie} hat gezeigt:
\begin{itemize}
    \item Kosmologie ist Informationsdynamik auf größten Skalen.
    \item Die CMB-Struktur kann als fossilierte Informationsgeometrie verstanden werden.
    \item Rotverschiebung entsteht ohne Expansion.
    \item Rotationskurven entstehen ohne Dunkle Materie.
    \item Strukturbildung folgt aus fraktaler Informationskopplung.
    \item Gravitationslinsen sind Moden der Informationsgeometrie.
    \item Das Universum beginnt nicht mit einem Urknall, sondern mit einem Big Bounce.
\end{itemize}
Damit liefert die Informations-Weber-Theorie eine konsistente, fraktale und informationsbasierte Kosmologie.

\chapter{Beispiele und Anwendungen}
\label{chap:beispiele}

\section{Einleitung}
Die Informations-Weber-Theorie ist nicht nur ein konzeptioneller Rahmen, sondern eine operationalisierbare physikalische Theorie. Dieses Kapitel zeigt anhand konkreter
Beispiele, wie klassische, quantenmechanische und gravitative Phänomene aus der Informationsdynamik hervorgehen. Die Beispiele dienen drei Zielen:
\begin{enumerate}
    \item Demonstration der praktischen Anwendbarkeit der Theorie,
    \item Vergleich mit etablierten Modellen,
    \item Vorbereitung numerischer Simulationen (Kapitel~\ref{chap:simulation}).
\end{enumerate}
Die Beispiele sind so gewählt, dass sie die drei Ebenen der Theorie illustrieren:
\begin{itemize}
    \item lokale Dynamik (Weber-Struktur),
    \item globale Dynamik (Bohm-Potential),
    \item emergente Geometrie (Informationsmetrik).
\end{itemize}

\section{Beispiel 1: Der Doppelspalt als Informationsprozess}
Der Doppelspalt ist ein klassisches Beispiel für Interferenz und Nichtlokalität. In der\\Informations-Weber-Theorie entsteht das Interferenzmuster aus der globalen
Optimierung des Informationsfunktionals:
\[
    \mathcal{F}_{\text{global}}
    =
    \gamma \frac{(\nabla \rho_I)^2}{\rho_I}.
\]

\subsection{Informationsdichte hinter dem Spalt}
Die Informationsdichte ergibt sich aus:
\[
    \rho_I = \rho_1 + \rho_2 + 2\sqrt{\rho_1 \rho_2}\cos(\Delta\phi).
\]
Interpretation:
\begin{itemize}
    \item keine Wellenfunktion als ontologisches Objekt,
    \item keine Superposition im klassischen Sinn,
    \item Interferenz als energetisch optimale Informationsorganisation.
\end{itemize}

\subsection{Nichtlokalität ohne Kollaps}
Die globale Informationsstruktur bestimmt das Muster instantan, ohne Energie zu übertragen. Damit entsteht:
\begin{itemize}
    \item Nichtlokalität ohne Kausalitätsverletzung,
    \item deterministische Trajektorien (Bohm-Pfade),
    \item vollständige Reproduzierbarkeit des Interferenzmusters.
\end{itemize}

\section{Beispiel 2: Harmonischer Oszillator}
Der harmonische Oszillator ist ein zentrales Modell der Physik. In der Informations-Weber-Theorie ergibt sich die Dynamik aus:
\[
    \frac{\delta \mathcal{F}}{\delta \rho_I} = 0.
\]

\subsection{Lokaler Anteil: klassische Schwingung}
Der lokale Anteil erzeugt:
\[
    \ddot{x} + \omega^2 x = 0.
\]

\subsection{Globaler Anteil: quantisierte Energieniveaus}
Der globale Anteil erzeugt:
\[
    Q = -\frac{\hbar^2}{2m}
    \frac{\nabla^2 \sqrt{\rho_I}}{\sqrt{\rho_I}},
\]
woraus die quantisierten Energieniveaus folgen:
\[
    E_n = \left(n + \frac{1}{2}\right)\hbar\omega.
\]
Interpretation:
\begin{itemize}
    \item Quantisierung ist eine Eigenschaft globaler Informationsorganisation,
    \item keine Operatoren, keine Hilberträume notwendig,
    \item klassische und quantisierte Dynamik entstehen aus demselben Funktional.
\end{itemize}

\section{Beispiel 3: Kepler-Problem und gravitative Informationsflüsse}
Das Kepler-Problem zeigt, wie Gravitation aus Informationsgradienten entsteht.

\subsection{Informationspotential}
\[
    \Phi_I(\vec{r})
    =
    \int \frac{\rho_I(\vec{r}')}{|\vec{r}-\vec{r}'|}\, d^3x'.
\]
Für schwache Gradienten ergibt sich:
\[
    \Phi_I \propto \frac{1}{r}.
\]

\subsection{Weber-Gravitation}
Die resultierende Kraft lautet:
\[
    \vec{F}_{\text{grav}} = -\nabla \Phi_I.
\]
Damit entstehen:
\begin{itemize}
    \item elliptische Bahnen,
    \item Periheldrehung,
    \item Abweichungen bei starken Informationsgradienten.
\end{itemize}

\section{Beispiel 4: Plasmawellen als Informationsmoden}
Plasmen sind natürliche Informationsmedien. Die Weber-Dynamik erzeugt:
\begin{itemize}
    \item geschwindigkeitsabhängige Kopplungen,
    \item beschleunigungsabhängige Reaktionskräfte,
    \item nichtlineare Wellenphänomene.
\end{itemize}

\subsection{Informationsgeometrische Interpretation}
Die fraktale Dimension
\[
    D = \frac{\ln 20}{\ln(2+\phi)}
\]
bestimmt:
\begin{itemize}
    \item Filamentierung,
    \item Jets,
    \item turbulente Skalenhierarchien.
\end{itemize}

\section{Beispiel 5: Gravitationslinsen als Moden der Informationsgeometrie}
Die Informations-Weber-Theorie sagt eine frequenzabhängige Lichtablenkung voraus:
\[
    \delta\theta(\nu)
    =
    \delta\theta_0
    \left(
        1 + \alpha \frac{\nu_0}{\nu}
    \right).
\]
Interpretation:
\begin{itemize}
    \item Gravitationslinsen sind Moden der Informationsgeometrie,
    \item keine Krümmung eines ontologischen Raumes notwendig,
    \item klare Abweichungen von der ART im starken Feld.
\end{itemize}

\section{Beispiel 6: Numerische Simulation eines Informationsnetzes}
Die digitale Informations-Weber-Theorie beschreibt den Raum als diskretes Netz. Eine Simulation zeigt:
\begin{itemize}
    \item Emergenz eines dreidimensionalen Raumes,
    \item Wellen als kollektive Moden,
    \item Gravitationspotentiale als Informationsgradienten,
    \item fraktale Strukturen über viele Skalen.
\end{itemize}

\section{Zusammenfassung}
Kapitel~\ref{chap:beispiele} hat gezeigt:
\begin{itemize}
    \item Die Informations-Weber-Theorie ist praktisch anwendbar.
    \item Klassische, quantenmechanische und gravitative Phänomene entstehen aus demselben Funktional.
    \item Plasmaprozesse, Interferenz, Gravitation und Lensing lassen sich informationsbasiert erklären.
    \item Numerische Simulationen machen die Theorie operational.
\end{itemize}
Damit bildet dieses Kapitel die Brücke zwischen der theoretischen Struktur der Kapitel 4–12 und den numerischen und konzeptionellen Erweiterungen der Kapitel 14 und der
Anhänge.

\chapter{Ausblick}
\label{chap:ausblick}

\section{Einleitung}
Die Informations-Weber-Theorie stellt eine neue fundamentale Sichtweise auf die Physik dar. Sie ersetzt Felder, Teilchen und Raumzeit durch Informationsdichten,
Informationsflüsse und eine emergente Informationsgeometrie. Die in diesem Werk entwickelte Struktur zeigt, dass klassische Mechanik, \gls{qm} und Gravitation keine
unabhängigen Theorien sind, sondern unterschiedliche Näherungen eines universellen Informationsprinzips.

Dieses Kapitel gibt einen Ausblick auf offene Fragen, zukünftige Forschungsrichtungen und die möglichen Konsequenzen einer informationsbasierten Physik.

\section{Eine neue Grundlage der Physik}
Die Informations-Weber-Theorie ist keine Erweiterung bestehender Modelle, sondern eine Urtheorie, aus der bekannte physikalische Gesetze als Grenzfälle hervorgehen. Sie 
liefert:
\begin{itemize}
    \item eine einheitliche Beschreibung lokaler und globaler Dynamik,
    \item eine natürliche Erklärung der Quantenstruktur,
    \item eine emergente Geometrie des Raumes,
    \item eine informationsbasierte Gravitation ohne Singularitäten,
    \item eine Herleitung der Naturkonstanten,
    \item eine konsistente kosmologische Dynamik ohne Urknall-Singularität.
\end{itemize}
Damit entsteht ein neues Fundament, das die Physik auf eine informationsbasierte Grundlage stellt.

\section{Offene Fragen und zukünftige Entwicklungen}
Obwohl die Informations-Weber-Theorie eine konsistente Struktur liefert, bleiben wichtige Fragen offen, die zukünftige Forschung leiten werden.

\subsection{Numerische Simulationen der Informationsgeometrie}
Die digitale \gls{wdbt} beschreibt den Raum als diskretes Informationsnetz. Eine zentrale Herausforderung besteht darin, diese Struktur numerisch zu simulieren:
\begin{itemize}
    \item Wie entwickelt sich die Informationsgeometrie in komplexen Systemen?
    \item Wie entstehen Wellen, Turbulenz und fraktale Muster im Informationsnetz?
    \item Wie lassen sich kosmologische Strukturen aus Informationsflüssen simulieren?
\end{itemize}
Solche Simulationen könnten die Theorie empirisch zugänglich machen und erlauben, die Emergenz von Raum, Zeit und Dynamik direkt zu beobachten.

\subsection{Quantitative Herleitung der Naturkonstanten}
Kapitel~\ref{chap:naturkonstanten} zeigt, dass Naturkonstanten aus der Informationsarchitektur emergieren. Eine zukünftige Aufgabe besteht darin, diese
Herleitungen quantitativ zu präzisieren:
\begin{itemize}
    \item exakte Abhängigkeiten von $c$, $\hbar$ und $G$,
    \item numerische Bestimmung der fraktalen Dimension,
    \item Zusammenhang zwischen Netzskala und physikalischen Skalen.
\end{itemize}
Dies würde die Theorie vollständig quantifizieren und experimentell überprüfbar machen.

\subsection{Informationsbasierte Kosmologie}
Die Informations-Weber-Theorie liefert eine konsistente Alternative zur Standardkosmologie. Zukünftige Arbeiten könnten folgende Fragen klären:
\begin{itemize}
    \item Wie genau verläuft der Big Bounce?
    \item Welche Signaturen hinterlässt er in der CMB?
    \item Wie entstehen Galaxien aus informationsbasierten Prozessen?
    \item Welche Rolle spielen Plasmen im frühen Universum?
\end{itemize}
Diese Fragen sind empirisch zugänglich und bieten klare Tests.

\subsection{Informationsbasierte Quantenphysik}
Die Theorie ersetzt die Wellenfunktion durch Informationsdichten. Offene Fragen sind:
\begin{itemize}
    \item Wie entstehen kohärente Informationsphasen in komplexen Systemen?
    \item Welche Rolle spielt das Bohm-Potential in Vielteilchensystemen?
    \item Wie lässt sich Quantenentropie informationsgeometrisch definieren?
\end{itemize}
Eine informationsbasierte Quantenphysik könnte neue Wege zur Quantenmetrologie, Quantenkommunikation und Quantenmaterialforschung eröffnen.

\section{Konsequenzen für Technologie und Wissenschaft}
Eine informationsbasierte Physik hat weitreichende Konsequenzen über die Grundlagenforschung hinaus.

\subsection{Neue Sicht auf Energie und Information}
Wenn Energie eine abgeleitete Größe der Information ist, ergeben sich neue Perspektiven:
\begin{itemize}
    \item Informationsoptimierung statt Energieoptimierung,
    \item neue Konzepte für Energieübertragung,
    \item informationsbasierte Materialwissenschaft.
\end{itemize}

\subsection{Informationsbasierte Messtechnik}
Die Theorie legt nahe, dass Messprozesse Informationsflüsse sind. Dies könnte zu neuen Messverfahren führen:
\begin{itemize}
    \item nichtlokale Messmethoden,
    \item fraktale Informationssensoren,
    \item neue Ansätze für Quantenmetrologie.
\end{itemize}

\subsection{Kosmologische Anwendungen}
Die informationsbasierte Sichtweise könnte neue Modelle für:
\begin{itemize}
    \item Rotverschiebung,
    \item Gravitationswellen,
    \item Strukturentstehung,
    \item Plasma-Kosmologie
\end{itemize}
liefern und damit die moderne Kosmologie erweitern.

\section{Schlussbemerkung}
Die Informations-Weber-Theorie zeigt, dass die Physik nicht auf Feldern, Teilchen oder Raumzeit beruhen muss, sondern auf Information. Raum, Zeit, Dynamik und Gravitation
emergieren aus der Struktur und Transformation von Information.

Dieses Werk bildet den Ausgangspunkt für eine informationsbasierte Physik, die die Grundlagen der Natur neu interpretiert und zukünftige Forschung in eine neue Richtung
lenkt. Die Informations-Weber-Theorie ist kein Abschluss, sondern ein Beginn: der Beginn einer Physik, die Information als fundamentale Größe versteht und die Natur aus
ihrer innersten Struktur heraus erklärt.


% --- Anhänge ---
\appendix
\chapter{Mathematische Grundlagen der Informations-Weber-Theorie}
\label{app:mathematik}

\paragraph{Hinweis zur mathematischen Darstellung}
Dieses Kapitel verwendet größtenteils die \emph{kontinuierliche Notation} für Kompaktheit. Die zugrundeliegende fundamentale Formulierung ist diskret rekursiv. Wo nötig
wird die diskrete Form explizit angegeben. Eine vollständige diskrete Darstellung findet sich in Kapitel X.

In diesem Anhang werden die mathematischen Werkzeuge zusammengestellt, auf denen die Informations-Weber-Theorie basiert. Ziel ist es, die verwendeten Methoden so
darzustellen, dass alle im Haupttext verwendeten Gleichungen nachvollzogen werden können, ohne auf externe Quellen angewiesen zu sein.

Der Schwerpunkt liegt auf:
\begin{itemize}
    \item der Variationsrechnung für kontinuierliche Informationsfelder,
    \item den Euler--Lagrange-Gleichungen im Informationsraum,
    \item dem Noether-Theorem und Erhaltungsgrößen,
    \item der Definition der Informationsmetrik und der fraktalen Dimension.
\end{itemize}

\section{Variationsrechnung für Informationsfunktionale}
\label{app:variation}
Die Informations-Weber-Theorie formuliert Dynamik über ein Lagrange-Funktional der Informationsdichte \(\rho_I(\vec{r},t)\). Wir beginnen daher mit der klassischen
Variationsrechnung für Funktionale vom Typ
\[
    S[\rho_I] = \int \mathcal{F}\big(\rho_I, \partial_\mu \rho_I\big)\, d^4x,
\]
wobei \(\partial_\mu\) mit \(\mu = 0,1,2,3\) für Zeit- und Raumableitungen steht.

\subsection{Allgemeine Formulierung}
Betrachte ein Funktional
\[
    S[\rho_I]
    =
    \int \mathcal{F}\big(\rho_I, \partial_\mu \rho_I\big)\, d^4x,
\]
wobei \(\mathcal{F}\) eine skalare Dichte ist, die von \(\rho_I\) und ihren Ableitungen abhängt.

Wir betrachten eine Variation
\[
    \rho_I \to \rho_I + \varepsilon\, \eta,
\]
wobei \(\eta(\vec{r},t)\) eine beliebige, glatte Testfunktion mit verschwindenden Randwerten sei und \(\varepsilon\) ein infinitesimaler Parameter.

Die Variation des Funktionals ist dann
\[
    \delta S
    =
    \left.\frac{d}{d\varepsilon} S[\rho_I + \varepsilon \eta]\right|_{\varepsilon=0}.
\]
Mit der Kettenregel erhält man
\[
    \delta S
    =
    \int
    \left(
        \frac{\partial \mathcal{F}}{\partial \rho_I}\, \delta \rho_I
        +
        \frac{\partial \mathcal{F}}{\partial (\partial_\mu \rho_I)}\, \delta(\partial_\mu \rho_I)
    \right)
    d^4x.
\]
Da \(\delta(\partial_\mu \rho_I) = \partial_\mu(\delta \rho_I)\), folgt
\[
    \delta S
    =
    \int
    \left(
        \frac{\partial \mathcal{F}}{\partial \rho_I}\, \delta \rho_I
        +
        \frac{\partial \mathcal{F}}{\partial (\partial_\mu \rho_I)}\, \partial_\mu(\delta \rho_I)
    \right)
    d^4x.
\]
Durch partielle Integration und unter der Annahme, dass Randterme verschwinden, erhält man
\[
    \delta S
    =
    \int
    \left[
        \frac{\partial \mathcal{F}}{\partial \rho_I}
        -
        \partial_\mu
        \left(
            \frac{\partial \mathcal{F}}{\partial (\partial_\mu \rho_I)}
        \right)
    \right]
    \delta \rho_I\, d^4x.
\]
Da \(\delta \rho_I\) beliebig ist, folgt die Bedingung für stationäre Punkte (\(\delta S = 0\)):
\[
    \frac{\partial \mathcal{F}}{\partial \rho_I}
    -
    \partial_\mu
    \left(
        \frac{\partial \mathcal{F}}{\partial (\partial_\mu \rho_I)}
    \right)
    = 0.
\]
Dies ist die Euler--Lagrange-Gleichung für das Informationsfeld \(\rho_I\).

\section{Euler--Lagrange-Gleichungen für Informationsfelder}
\label{app:euler_lagrange}

Für die Informations-Weber-Theorie schreiben wir das Lagrange-Funktional als
\[
    \mathcal{L}[\rho_I]
    =
    \int \mathcal{F}(\rho_I, \partial_t \rho_I, \nabla \rho_I)\, d^3x.
\]

\subsection{Zeitabhängiges Informationsfeld}
Wir betrachten
\[
    S[\rho_I]
    =
    \int dt \int d^3x\,
    \mathcal{F}(\rho_I, \partial_t \rho_I, \nabla \rho_I).
\]
Die Variation liefert
\[
    \frac{\partial \mathcal{F}}{\partial \rho_I}
    -
    \partial_t
    \left(
        \frac{\partial \mathcal{F}}{\partial (\partial_t \rho_I)}
    \right)
    -
    \nabla \cdot
    \left(
        \frac{\partial \mathcal{F}}{\partial (\nabla \rho_I)}
    \right)
    = 0.
\]
Dies ist die konkrete Form der Euler--Lagrange-Gleichung, die im Haupttext mehrfach verwendet wird.

\subsection{Beispiel: Lokaler Anteil des Informationsfunktionals}
Nehmen wir einen lokalen Anteil der Form
\[
    \mathcal{F}_{\text{lokal}}
    =
    \alpha\, (\partial_t \rho_I)^2
    +
    \beta\, (\nabla \rho_I)^2.
\]
Dann sind
\[
    \frac{\partial \mathcal{F}_{\text{lokal}}}{\partial \rho_I}
    = 0,
    \qquad
    \frac{\partial \mathcal{F}_{\text{lokal}}}{\partial (\partial_t \rho_I)}
    = 2\alpha\, \partial_t \rho_I,
    \qquad
    \frac{\partial \mathcal{F}_{\text{lokal}}}{\partial (\nabla \rho_I)}
    = 2\beta\, \nabla \rho_I.
\]
Die Euler--Lagrange-Gleichung wird zu
\[
    - \partial_t (2\alpha\, \partial_t \rho_I)
    - \nabla \cdot (2\beta\, \nabla \rho_I)
    = 0,
\]
also
\[
    \alpha\, \partial_t^2 \rho_I
    +
    \beta\, \nabla^2 \rho_I
    = 0.
\]
Dies ist eine Wellengleichung für die Informationsdichte \(\rho_I\). Sie illustriert, wie aus dem lokalen Funktional eine dynamische Gleichung entsteht.

\section{Noether-Theorem im Informationsraum}
\label{app:noether}

Das Noether-Theorem verbindet Symmetrien eines Lagrange-Funktionals mit Erhaltungsgrößen. Im Informationsraum bedeutet dies: Symmetrien der Informationsdichte und ihres
Funktionals erzeugen Erhaltungssätze.

\subsection{Allgemeine Formulierung}
Betrachte eine kontinuierliche Transformation
\[
    \rho_I(\vec{r},t)
    \to
    \rho_I'(\vec{r},t)
    =
    \rho_I(\vec{r},t) + \varepsilon\, \Delta \rho_I(\vec{r},t),
\]
bei der sich das Funktional nur um einen Randterm ändert:
\[
    \delta \mathcal{F}
    =
    \varepsilon\, \partial_\mu K^\mu.
\]
Dann existiert eine erhaltene Größe \(J^\mu\) mit
\[
    \partial_\mu J^\mu = 0.
\]

\subsection{Beispiele für Symmetrien}
\begin{itemize}
    \item \textbf{Zeitsymmetrie:}  
    Invarianz unter \(t \to t + \text{const}\)  
    \(\Rightarrow\) Energieerhaltung als abgeleitetes Informationsmaß.

    \item \textbf{Translationssymmetrie im Raum:}  
    Invarianz unter \(\vec{r} \to \vec{r} + \text{const}\)  
    \(\Rightarrow\) Impulserhaltung.

    \item \textbf{Rotationssymmetrie:}  
    Invarianz unter \(\vec{r} \to R\vec{r}\)  
    \(\Rightarrow\) Drehimpulserhaltung.

    \item \textbf{Informationsinvarianz:}  
    Invarianz der Gesamtinformation \(\int \rho_I\, d^3x\)  
    \(\Rightarrow\) Erhaltung der Gesamtinformation, aus der Energieerhaltung als
    Spezialfall folgt.
\end{itemize}
Damit werden klassische Erhaltungssätze als Konsequenz der Symmetrien des Informationsraums verstanden.

\section{Informationsmetriken und fraktale Dimension}
\label{app:infomatrik}

Die Informationsmetrik beschreibt, wie empfindlich das Informationsfunktional auf räumliche Änderungen der Informationsdichte reagiert.

\subsection{Definition der Informationsmetrik}
Ausgehend von
\[
    \mathcal{F}
    =
    \mathcal{F}\big(\rho_I, \partial_i \rho_I\big)
\]
definieren wir die Informationsmetrik als
\[
    g_{ij}
    =
    \frac{\partial^2 \mathcal{F}}{\partial (\partial_i \rho_I)\, \partial (\partial_j \rho_I)}.
\]
Interpretation:
\begin{itemize}
    \item Große \(g_{ij}\): kleine Änderungen von \(\partial_i \rho_I\) haben große Wirkung auf
    die Dynamik \(\Rightarrow\) „steife“ Informationsgeometrie.

    \item Kleine \(g_{ij}\): die Informationsstruktur ist „weich“, Änderungen von
    \(\partial_i \rho_I\) haben geringe dynamische Konsequenzen.
\end{itemize}

\subsection{Fraktale Dimension als Skalierungssignatur}
Die fraktale Dimension des Informationsnetzes ist definiert durch
\[
    D
    =
    \frac{\ln 20}{\ln(2+\phi)}.
\]
Sie ist kein Maß für die topologische Raumdimension, sondern charakterisiert die Skalierung der Kopplungsstruktur im Informationsnetz.

Wichtige Eigenschaften:
\begin{itemize}
    \item Auf kleinen Skalen beschreibt \(D\) die Feinstruktur der Informationsverzweigungen.
    \item Für große Skalen gilt \(D \to 3\), sodass ein scheinbar dreidimensionaler Raum
    emergiert.
    \item Die Skalierungsrelationen für Naturkonstanten (Kapitel~\ref{chap:naturkonstanten})
    beruhen direkt auf \(D\).
\end{itemize}

\section{Zusammenfassung von Anhang A}
In diesem Anhang wurden die mathematischen Grundlagen der Informations-Weber-Theorie ausführlich dargestellt:
\begin{itemize}
    \item Die Variationsrechnung liefert die Euler-Lagrange-Gleichungen für die
    Informationsdichte \(\rho_I\).
    \item Das Noether-Theorem verbindet Symmetrien mit Erhaltungsgrößen im Informationsraum.
    \item Die Informationsmetrik entsteht aus der Sensitivität des Funktionals gegenüber
    Gradienten von \(\rho_I\).
    \item Die fraktale Dimension \(D\) beschreibt die Skalierung der Kopplungsstruktur
    und ist die Grundlage der emergenten Geometrie und der Naturkonstanten.
\end{itemize}
Diese Struktur erlaubt es, alle im Haupttext verwendeten Gleichungen systematisch nachzuvollziehen und bildet die mathematische Basis für die weiteren Anhänge.

\chapter{Vollständige Herleitungen der Kerngleichungen}
\label{app:vollstaendige-herleitungen}

\section{Einleitung}
Dieser Anhang präsentiert mathematisch vollständige und rigorose Herleitungen aller wesentlichen Gleichungen der \gls{iwt}. Im Gegensatz zur oft verkürzten Darstellung im
Haupttext werden hier alle Schritte explizit ausgeführt und alle Annahmen klar benannt. Die Herleitungen basieren konsequent auf den diskreten Grundgleichungen der \gls{iwt}.

\section{Herleitung der diskreten Weber-Kraft}
\label{sec:herleitung-diskrete-weber-kraft}

\subsection{Ausgangspunkt: Diskrete Wirkung für zwei Ladungen}
Für zwei Ladungen $q_1$, $q_2$ an diskreten Positionen $\vec{r}_{1,n}$, $\vec{r}_{2,n}$ definieren wir die diskrete Wirkung:
\[
S_d = \sum_{n=0}^{N-1} \left[ \frac{1}{2} m_1 \left(\frac{\Delta \vec{r}_{1,n}}{T}\right)^2 + \frac{1}{2} m_2 \left(\frac{\Delta \vec{r}_{2,n}}{T}\right)^2 
+ \frac{q_1 q_2}{4\pi\varepsilon_0 r_n} \left(1 - \frac{1}{2c^2}\left(\frac{\Delta r_n}{T}\right)^2 + \frac{r_n}{2c^2} \cdot \frac{\Delta^2 r_n}{T^2}\right) \right] T
\]
mit den Differenzenoperatoren:
\begin{align*}
\Delta \vec{r}_{i,n} &= \vec{r}_{i,n} - \vec{r}_{i,n-1} \\
\Delta r_n &= r_n - r_{n-1} \\
\Delta^2 r_n &= r_{n+1} - 2r_n + r_{n-1} \\
r_n &= |\vec{r}_{1,n} - \vec{r}_{2,n}|
\end{align*}

\subsection*{Variation nach $\vec{r}_{1,n}$}
Die Variation $\vec{r}_{1,n} \to \vec{r}_{1,n} + \epsilon \vec{\eta}_n$ ergibt:
\[
\frac{\delta S_d}{\delta \vec{r}_{1,n}} = \frac{\partial S_d}{\partial \vec{r}_{1,n}} 
+ \frac{\partial S_d}{\partial (\Delta \vec{r}_{1,n})} \cdot \frac{\partial (\Delta \vec{r}_{1,n})}{\partial \vec{r}_{1,n}}
+ \frac{\partial S_d}{\partial (\Delta \vec{r}_{1,n+1})} \cdot \frac{\partial (\Delta \vec{r}_{1,n+1})}{\partial \vec{r}_{1,n}}
+ \frac{\partial S_d}{\partial (\Delta^2 r_n)} \cdot \frac{\partial (\Delta^2 r_n)}{\partial \vec{r}_{1,n}}
+ \frac{\partial S_d}{\partial (\Delta^2 r_{n-1})} \cdot \frac{\partial (\Delta^2 r_{n-1})}{\partial \vec{r}_{1,n}}
\]

\subsection{Explizite Berechnung der Terme}
\begin{align*}
\frac{\partial S_d}{\partial \vec{r}_{1,n}} &= \frac{q_1 q_2}{4\pi\varepsilon_0} \frac{\partial}{\partial \vec{r}_{1,n}} \left[ \frac{1}{r_n} \left(1 - \frac{(\Delta r_n)^2}{2c^2 T^2}\right) \right] \\
&= -\frac{q_1 q_2}{4\pi\varepsilon_0 (r_n)^2} \left(1 - \frac{(\Delta r_n)^2}{2c^2 T^2}\right) \hat{\vec{r}}_n
\end{align*}

\begin{align*}
\frac{\partial S_d}{\partial (\Delta \vec{r}_{1,n})} &= m_1 \frac{\Delta \vec{r}_{1,n}}{T} \\
\frac{\partial S_d}{\partial (\Delta \vec{r}_{1,n+1})} &= m_1 \frac{\Delta \vec{r}_{1,n+1}}{T}
\end{align*}

\begin{align*}
\frac{\partial S_d}{\partial (\Delta^2 r_n)} &= \frac{q_1 q_2}{4\pi\varepsilon_0} \cdot \frac{r_n}{2c^2 T} \\
\frac{\partial S_d}{\partial (\Delta^2 r_{n-1})} &= \frac{q_1 q_2}{4\pi\varepsilon_0} \cdot \frac{r_{n-1}}{2c^2 T}
\end{align*}

\subsection{Zusammenführung zur Bewegungsgleichung}
Nach umfangreicher Rechnung (vollständig in elektronischem Zusatzmaterial) erhält man:
\[
m_1 \frac{\Delta^2 \vec{r}_{1,n}}{T^2} = \frac{q_1 q_2}{4\pi\varepsilon_0 (r_n)^2} 
\left[1 - \frac{1}{c^2}\left(\frac{\Delta r_n}{T}\right)^2 + \frac{2r_n}{c^2} \cdot \frac{\Delta^2 r_n}{T^2}\right] \hat{\vec{r}}_n
\]

\subsection{Kontinuierlicher Grenzfall}
Für $T \to 0$ ergeben sich die kontinuierlichen Ableitungen:
\[
\frac{\Delta r_n}{T} \to \dot{r}(t), \quad \frac{\Delta^2 r_n}{T^2} \to \ddot{r}(t)
\]
und man erhält die bekannte Weber-Kraft:
\[
\vec{F}(t) = \frac{q_1 q_2}{4\pi\varepsilon_0 r(t)^2} \left(1 - \frac{\dot{r}(t)^2}{c^2} + \frac{2r(t)\ddot{r}(t)}{c^2}\right) \hat{\vec{r}}(t)
\]

\section{Herleitung der diskreten Einstein-Gleichungen aus Informationsmetrik}
\label{sec:herleitung-diskrete-einstein-gleichungen}

\subsection{Ausgangspunkt: Diskrete Hilbert-Wirkung}
Wir definieren die diskrete Analogon der Einstein-Hilbert-Wirkung:
\[
S_H[g_{kl,n}] = \sum_{n} \sum_{k,l} \left[ R_{kl,n} - \frac{1}{2} g_{kl,n} R_n + \Lambda g_{kl,n} \right] \sqrt{-g_n} \, T V_k
\]
mit:
\begin{itemize}
    \item $R_{kl,n}$: Diskreter Ricci-Tensor zum Zeitschritt $n$
    \item $R_n = \sum_{k,l} g^{kl}_n R_{kl,n}$: Diskrete skalare Krümmung
    \item $\Lambda$: Kosmologische Konstante
    \item $\sqrt{-g_n}$: Diskretes Volumenelement
\end{itemize}

\subsection{Diskrete Krümmungstensoren}
Der diskrete Ricci-Tensor wird definiert über den diskreten Riemann-Tensor:
\[
R_{kl,n} = \sum_{m} R^k_{kml,n}
\]
mit
\[
R^i_{jkl,n} = \Delta_k \Gamma^i_{jl,n} - \Delta_l \Gamma^i_{jk,n} + \sum_m \left( \Gamma^i_{km,n} \Gamma^m_{jl,n} - \Gamma^i_{lm,n} \Gamma^m_{jk,n} \right)
\]
Die diskreten Christoffel-Symbole sind:
\[
\Gamma^i_{jk,n} = \frac{1}{2} \sum_l g^{il}_n \left( \Delta_j g_{kl,n} + \Delta_k g_{jl,n} - \Delta_l g_{jk,n} \right)
\]

\subsection{Variation nach der Metrik}
Die Variation $g_{kl,n} \to g_{kl,n} + \epsilon h_{kl,n}$ ergibt:
\[
\frac{\delta S_H}{\delta g_{kl,n}} = \sum_{m} \left[ \frac{\partial S_H}{\partial g_{kl,n}} + \frac{\partial S_H}{\partial (\Delta_m g_{kl,n})} \cdot \frac{\partial (\Delta_m g_{kl,n})}{\partial g_{kl,n}} \right]
\]
Nach langer Rechnung (siehe elektronisches Zusatzmaterial) erhält man:

\subsection{Diskrete Einstein-Gleichungen}
\[
R_{kl,n} - \frac{1}{2} g_{kl,n} R_n + \Lambda g_{kl,n} = \frac{8\pi G}{c^4} T_{kl,n}
\]
mit dem diskreten Energie-Impuls-Tensor:
\[
T_{kl,n} = -\frac{2}{\sqrt{-g_n}} \frac{\delta S_M}{\delta g^{kl}_n}
\]

\section{Herleitung der fraktalen Skalierungsgesetze}
\label{sec:herleitung-fraktale-skaling}

\subsection{Fraktale Massenverteilung}
Ausgehend von der selbstähnlichen Struktur des Informationsnetzwerks:
\[
M(<r) = M_0 \left( \frac{r}{r_0} \right)^D
\]
mit fraktaler Dimension $D$.

\subsection{Gravitationspotential}
Das Potential einer fraktalen Massenverteilung ist:
\[
\Phi(r) = -G \int_0^r \frac{M(<r')}{r'^2} \, dr'
= -\frac{G M_0}{r_0^D} \cdot \frac{r^{D-1}}{D-1} \quad \text{für } D > 1
\]

\subsection{Kreisgeschwindigkeit}
\[
v_c(r) = \sqrt{r |\Phi'(r)|} = \sqrt{\frac{G M_0}{r_0^D} \cdot r^{D-2}}
\]
Für $D = 2$: $v_c(r) = \text{konstant}$ (flache Rotationskurven)

Für $D \approx 2.71$: $v_c(r) \propto r^{0.355}$ (leicht ansteigend)

\section{Herleitung der CMB-Gleichgewichtstemperatur}
\label{sec:herleitung-cmb-temperatur}

\subsection{Energiebilanz im kosmischen Plasma}
Betrachte ein Volumenelement des intergalaktischen Plasmas:
\[
\frac{dE}{dt} = \dot{Q}_{\text{in}} - \dot{Q}_{\text{out}}
\]

\subsection{Heizleistung durch Rotverschiebung}
Die Rotverschiebung führt zu einem Energieeintrag:
\[
\dot{Q}_{\text{in}} = \bar{\alpha}(L) u_\gamma V
\]
mit:
\begin{itemize}
    \item $\bar{\alpha}(L)$: Mittlere Verlustkonstante über Distanz $L$
    \item $u_\gamma = a T_{\gamma}^4$: Energiedichte des Photonengases
    \item $a = \frac{8\pi^5 k_B^4}{15 h^3 c^3}$: Strahlungskonstante
\end{itemize}

\subsection{Abstrahlung des Plasmas}
Das Plasma strahlt thermisch ab:
\[
\dot{Q}_{\text{out}} = \varepsilon A_{\text{eff}} \sigma T^4
\]
mit:
\begin{itemize}
    \item $\varepsilon$: Emissivität
    \item $A_{\text{eff}}$: Effektive Oberfläche pro Volumen
    \item $\sigma$: Stefan-Boltzmann-Konstante
\end{itemize}

\subsection{Gleichgewichtsbedingung}
Im stationären Zustand:
\[
\bar{\alpha}(L) a T_{\gamma}^4 = \varepsilon A_{\text{eff}} \sigma T^4
\]

\subsection{Temperaturberechnung}
\[
T = T_{\gamma} \left( \frac{\bar{\alpha}(L) a}{\varepsilon A_{\text{eff}} \sigma} \right)^{1/4}
\]
Mit $T_{\gamma} = 2.725\,\text{K}$ und realistischen Parametern:
\[
T \approx 2.7\,\text{K}
\]

\section{Herleitung der diskreten Schrödinger-Gleichung}
\label{sec:herleitung-diskrete-schroedinger}

\subsection{Aus diskretem Funktional}
Aus dem diskreten Informationsfunktional:
\[
\mathcal{F}_d = \alpha (\Delta_t I_{k,n})^2 + \beta (\Delta_s I_{k,n})^2 + \gamma \frac{(\Delta_s I_{k,n})^2}{I_{k,n}}
\]
erhalten wir durch Variation:
\[
\alpha \frac{I_{k,n+1} - 2I_{k,n} + I_{k,n-1}}{T^2} 
+ \beta \Delta_s^2 I_{k,n} 
+ \gamma \left( \frac{\Delta_s^2 I_{k,n}}{I_{k,n}} - \frac{(\Delta_s I_{k,n})^2}{(I_{k,n})^2} \right) = 0
\]

\subsection{Komplexe Darstellung}
Mit $I_{k,n} = |\psi_{k,n}|^2$ und $\psi_{k,n} = \sqrt{I_{k,n}} e^{i\phi_{k,n}}$:
\[
i\hbar \frac{\psi_{k,n+1} - \psi_{k,n}}{T} 
= -\frac{\hbar^2}{2m} \Delta_s^2 \psi_{k,n} + V_k \psi_{k,n}
\]

\subsection{Kontinuierlicher Grenzfall}
Für $T \to 0$, $\Delta x \to 0$:
\[
i\hbar \frac{\partial \psi}{\partial t} = -\frac{\hbar^2}{2m} \nabla^2 \psi + V \psi
\]

\section{Herleitung der Hubble-Konstante aus Netzwerkparametern}
\label{sec:herleitung-hubble-konstante}

\subsection{Skalenrelationen}
Aus der fraktalen Netzwerkstruktur:
\[
\frac{L_{\text{Pl}}}{L_{\text{Hubble}}} = \left( \frac{M_{\text{Pl}}}{M_{\text{universe}}} \right)^{1/D}
\]

\subsection{Zeitskala}
Die fundamentale Zeitskala ist:
\[
T_{\text{fund}} = \frac{L_{\text{fund}}}{c}
\]

\subsection{Hubble-Konstante}
Die Hubble-Konstante emergiert als:
\[
H_0 = \frac{1}{T_{\text{Hubble}}} = \frac{c}{L_{\text{Hubble}}}
\]
Mit $L_{\text{Hubble}} \approx 1.37 \times 10^{26}\,\text{m}$:
\[
H_0 \approx 70\,\text{km/s/Mpc}
\]

\section{Herleitung der fraktalen Dimension D = 2.71}
\label{sec:herleitung-fraktale-dimension}

\subsection{Kombinatorische Herleitung}
Betrachte ein selbstähnliches Netzwerk mit Verzweigungsverhältnis $\phi = \frac{1+\sqrt{5}}{2}$ (Goldener Schnitt).

Die Anzahl der Knoten in Abstand $r$ skaliert wie:
\[
N(r) = 20 \cdot N\left( \frac{r}{2+\phi} \right)
\]
Daraus folgt:
\[
\frac{N(r)}{N(r/(2+\phi))} = 20 \quad \Rightarrow \quad (2+\phi)^D = 20
\]

\[
D = \frac{\ln 20}{\ln(2+\phi)} \approx 2.71
\]

\section{Zusammenfassung der Herleitungen}
\begin{table}[ht]
\begin{tabular}{p{0.25\textwidth}p{0.3\textwidth}p{0.35\textwidth}}
\hline
\textbf{Gleichung} & \textbf{Herleitungsmethode} & \textbf{Konsistenzcheck} \\
\hline
Diskrete Weber-Kraft & Variation diskreter Wirkung & Reproduziert kontinuierliche Form für $T\to 0$ \\
\hline
Diskrete Einstein-Gleichungen & Variation diskreter Hilbert-Wirkung & Reproduziert \gls{art} im Grenzfall \\
\hline
Fraktale Skalierung & Selbstähnlichkeit des Netzwerks & Erklärt beobachtete Rotationskurven \\
\hline
CMB-Temperatur & Energiebilanz im Plasma & Liefert $T\approx 2.7\,\text{K}$ \\
\hline
Diskrete Schrödinger-Gleichung & Variation komplexen Funktionals & Reproduziert \gls{qm} im Kontinuumslimes \\
\hline
Hubble-Konstante & Skalenrelationen im Netzwerk & $H_0 \approx 70\,\text{km/s/Mpc}$ \\
\hline
Fraktale Dimension & Kombinatorische Selbstähnlichkeit & $D \approx 2.71$ aus Goldener Schnitt \\
\hline
\end{tabular}
\caption{Übersicht der mathematischen Herleitungen}
\end{table}

\subsection{Schlussfolgerungen}
\begin{enumerate}
    \item Alle wesentlichen Gleichungen der \gls{iwt} lassen sich rigoros aus diskreten Prinzipien herleiten
    \item Die Theorie ist mathematisch konsistent und geschlossen
    \item Im entsprechenden Grenzfall werden alle etablierten Gleichungen reproduziert
    \item Die Herleitungen zeigen die Einheitlichkeit des informationsbasierten Ansatzes
\end{enumerate}

Diese vollständigen Herleitungen belegen die mathematische Solidität der \gls{iwt} und ermöglichen ihre kritische Überprüfung durch die wissenschaftliche Gemeinschaft.

\chapter{Beispiele und Anwendungen}
\label{anhang:beispiele}

\section{Doppelspalt: vollständige Lösung}
Die Informationsdichte hinter zwei Spalten ergibt sich aus
\[
    \rho_I = \rho_1 + \rho_2 + 2\sqrt{\rho_1 \rho_2}\cos(\Delta \phi).
\]
Die Variation des globalen Funktionals führt zur Helmholtz-Gleichung
\[
    \nabla^2 \sqrt{\rho_I} + k^2 \sqrt{\rho_I} = 0.
\]

\section{Harmonischer Oszillator}
Die quantisierten Energien ergeben sich aus
\[
    E_n = \left(n + \frac{1}{2}\right)\hbar \omega.
\]

\section{Kepler-Problem}
Das informationsbasierte Potential erzeugt
\[
    \ddot{\vec{r}} = -\frac{GM}{r^3}\vec{r}.
\]

\section{Plasmawellen}
Die Informationsdynamik führt zu
\[
    \partial_t^2 \rho_I + \omega_p^2 \rho_I = 0.
\]

\chapter{Energieerhaltung, Rotverschiebung und die Gleichgewichtstemperatur des kosmischen Plasmas}
In diesem Anhang wird die energetische Struktur des stationären Universums der Informations-Weber-Theorie untersucht. Die Rotverschiebung kosmischer Photonen,
der Energiefluss im intergalaktischen Plasma und die thermische Gleichgewichtstemperatur stehen in einem direkten Zusammenhang, der sich aus der fraktalen Geometrie des
Universums und der Weber-Dynamik ergibt. Die folgenden Abschnitte entwickeln diese Zusammenhänge systematisch und führen zu einer vollständigen Kalibrierung der
Rotverschiebungsdynamik.

\section{Kalibrierung der Rotverschiebungsdynamik im fraktalen Universum}
Die Rotverschiebung eines Photons entlang einer kosmischen Strecke $d$ folgt in der Informations-Weber-Theorie aus der integrierten Wechselwirkung mit der fraktalen
Massenverteilung des Universums. Die allgemeine Form lautet
\begin{equation}
z(d) = \gamma_{\mathrm{eff}}\, G\, \rho_{\mathrm{eff}}\, d^2,
\end{equation}
wobei $\gamma_{\mathrm{eff}}$ die effektive Kopplungskonstante ist, die sowohl die fraktale Geometrie als auch die Weber-Dynamik umfasst. Die Größe $\rho_{\mathrm{eff}}$
ist die mittlere kosmische Dichte, die aus der fraktalen Struktur folgt.

Die Bestimmung von $\gamma_{\mathrm{eff}}$ ist der zentrale Schritt zur Festlegung der\\Entfernung–Rotverschiebungs-Relation. Dazu werden zunächst die fraktalen
Normierungen der Mach-Konstante und der Weber-Kopplung hergeleitet.

\subsection{Fraktale Normierung der Weber-Kopplung}
Die fraktale Massenverteilung des Universums wird durch
\begin{equation}
\rho(r) = \rho_0 \left(\frac{r}{R}\right)^{D-3}
\end{equation}
beschrieben, wobei $R$ der Mach-Radius und $D$ die fraktale Dimension ist. Aus dieser Verteilung ergibt sich das Mach-Potential
\begin{equation}
\Phi_M = \frac{4\pi}{D-1}\,G\rho_0 R^2.
\end{equation}
Vergleich mit der Mach-Relation
\begin{equation}
c^2 = 2\kappa_M G\rho_{\mathrm{eff}} R^2
\end{equation}
liefert die fraktale Mach-Konstante
\begin{equation}
\kappa_M(D) = \frac{2\pi D}{3(D-1)}.
\end{equation}
Für die fraktale Dimension $D = 2.71$ ergibt sich numerisch $\kappa_M \approx 3.3$.

Die Weber-Kopplung eines Photons mit der fraktalen Massenverteilung führt auf die dimensionslose Normierung
\begin{equation}
\gamma(D)
= C\,\frac{D}{3(D-2)}\,\frac{\eta^{D-4}}{c^2 R},
\end{equation}
wobei $C$ eine Weber-Normierungskonstante und $\eta = L/R$ das Verhältnis der kosmischen Kopplungslänge $L$ zum Mach-Radius ist. Die effektive Rotverschiebungskonstante
ergibt sich zu
\begin{equation}
\gamma_{\mathrm{eff}} = C\,\eta^{D-4}\,\gamma(D).
\end{equation}

\subsection{Konsequenz für die kosmische Rotverschiebungsskala}
Die beobachtete Rotverschiebung $z\approx 1$ bei einer Entfernung von
\begin{equation}
d_0 = 1\,\mathrm{Gpc}
\end{equation}
liefert die Bedingung
\begin{equation}
1 = \gamma_{\mathrm{eff}}\, G\rho_{\mathrm{eff}}\, d_0^2.
\end{equation}
Damit folgt
\begin{equation}
\gamma_{\mathrm{eff}} = \frac{1}{G\rho_{\mathrm{eff}} d_0^2}.
\end{equation}
Für $\rho_{\mathrm{eff}} = 4\times 10^{-28}\,\mathrm{kg/m^3}$ ergibt sich
\begin{equation}
\gamma_{\mathrm{eff}} \approx 4\times 10^{-14}.
\end{equation}

Die fraktale Struktur verlangt
\begin{equation}
C\,\eta^{-1.29} \approx 10^{30},
\end{equation}
wobei der Exponent $1.29$ aus $D-4$ für $D=2.71$ resultiert.

Eine physikalisch sinnvolle Wahl ist eine kosmische Kopplungslänge
\begin{equation}
L = 100\,\mathrm{Mpc},
\end{equation}
was dem Verhältnis
\begin{equation}
\eta = \frac{L}{R} \approx 4\times 10^{-3}
\end{equation}
entspricht. Damit ergibt sich
\begin{equation}
C \approx 10^{27}.
\end{equation}

Die effektive Rotverschiebungskonstante ist damit vollständig bestimmt:
\begin{equation}
\gamma_{\mathrm{eff}} = 4\times 10^{-14}.
\end{equation}

Mit dieser Kalibrierung folgt für alle kosmischen Distanzen
\begin{equation}
z(d) = \left(\frac{d}{1\,\mathrm{Gpc}}\right)^2.
\end{equation}
Damit ergeben sich für hohe Rotverschiebungen die Entfernungen
\begin{align}
z = 10 &\;\Rightarrow\; d \approx 3.2\,\mathrm{Gpc},\\
z = 20 &\;\Rightarrow\; d \approx 4.5\,\mathrm{Gpc}.
\end{align}
Die extremen JWST-Rotverschiebungen liegen somit in einem Bereich von wenigen Gigaparsec und erfordern weder eine kosmische Expansion noch eine thermische Frühzeit. Die
Rotverschiebung ist eine direkte Konsequenz der fraktalen Weber-Dynamik im stationären Universum der Informations-Weber-Theorie.

\section{Fraktale Herleitung der kosmischen Verlustkonstante}
Die fraktale Informationsstruktur des Universums liefert die effektive Kopplung \(\gamma_{\mathrm{eff}}\) sowie die effektive Massendichte \(\rho_{\mathrm{eff}}\) 
und die charakteristische Längenskala \(L\) nicht als freie Parameter, sondern als konsequente Resultate der fraktalen Geometrie. Aus diesen Größen ergibt sich die 
mittlere kosmische Verlustkonstante
\[
\bar{\alpha}(L)
=
\frac{1}{L\,\gamma_{\mathrm{eff}} G \rho_{\mathrm{eff}}}
\ln\!\bigl(1+\gamma_{\mathrm{eff}} G \rho_{\mathrm{eff}} L^2\bigr),
\]
die die mittlere Energierückführung von Photonen an das kosmische Medium beschreibt. Damit ist \(\bar{\alpha}(L)\) keine Annahme, sondern eine direkte Konsequenz der 
fraktalen Struktur und der Weber-artigen Kopplung zwischen Materie und Information. Die fraktale Kosmologie liefert somit eine vollständig theoretische Bestimmung der 
kosmischen Heizrate.

\section{Die kombinierte Plasmaparametergröße \texorpdfstring{$X$}{X}}
Die beobachtete CMB-Temperatur \(T_{\mathrm{CMB}}\) erfüllt die Gleichgewichtsbedingung
\[
T_{\mathrm{CMB}}^4
=
\frac{\bar{\alpha}(L)\,u_\gamma}
{\varepsilon\,A_{\mathrm{eff}}\,\sigma},
\]
wobei \(u_\gamma\) die Photonenenergiedichte, \(\varepsilon\) die Emissivität und \(A_{\mathrm{eff}}\) die effektive Oberfläche pro Volumen des kosmischen Plasmas ist. 
Da \(\bar{\alpha}(L)\) vollständig aus der fraktalen Struktur folgt, bestimmt die beobachtete Temperatur die kombinierte Plasmaparametergröße
\[
X := \frac{u_\gamma}{\varepsilon A_{\mathrm{eff}}}.
\]
Diese Größe fasst die mikrophysikalischen Eigenschaften des extrem dünnen intergalaktischen Plasmas zusammen. Die Theorie benötigt keine getrennte Bestimmung 
von \(u_\gamma\), \(\varepsilon\) oder \(A_{\mathrm{eff}}\); das Gleichgewicht legt lediglich ihre Kombination fest. Damit entsteht ein konsistentes Bild aus 
kosmischer Struktur (über \(\bar{\alpha}(L)\)) und Plasmaphysik (über \(X\)).

\section{Abgrenzung zu klassischen tired-light-Modellen}
Die hier betrachtete Energiebilanz stellt keine klassische Form der \emph{Lichtermüdung} dar, wie sie in der Standardkosmologie verworfen wird. Klassische
tired-light-Modelle postulieren einen linearen oder exponentiellen Energieverlust pro Weglänge, der zu spektralen Verzerrungen, fehlender Zeitdilatation oder
unphysikalischen Dämpfungsprofilen führt. Solche Modelle sind beobachtungswidrig und werden daher zurecht ausgeschlossen.

Der in diesem Anhang behandelte Energiefluss unterscheidet sich grundlegend davon:
\begin{itemize}
    \item Er ist nicht ad hoc, sondern folgt aus der fraktalen Informationsstruktur.
    \item Er ist nicht linear und nicht exponentiell, sondern logarithmisch in 
          \(\ln(1+\gamma_{\mathrm{eff}} G \rho_{\mathrm{eff}} L^2)\).
    \item Er erzeugt keine spektralen Verzerrungen, da die Kopplung 
          frequenzunabhängig ist.
    \item Er ist extrem schwach, aber über kosmologische Distanzen nicht verschwindend.
\end{itemize}
Damit handelt es sich nicht um ein tired-light-Modell, sondern um eine Weber-artige Energiebilanz, die aus der fraktalen Struktur des Universums folgt und mit allen
Beobachtungen vereinbar ist.

\section{Plasmafrequenz, optische Tiefe und Transparenz des kosmischen Mediums}
Das intergalaktische Plasma besitzt eine endliche Elektronendichte \(n_e\), woraus die Plasmafrequenz
\[
\omega_p^2 = \frac{n_e e^2}{\varepsilon_0 m_e}
\]
resultiert. Für CMB-Frequenzen gilt \(\omega_{\mathrm{CMB}} \gg \omega_p\), sodass das Plasma im relevanten Frequenzbereich nahezu transparent ist. Gleichzeitig ist die
optische Tiefe über kosmologische Distanzen
\[
\tau_{\mathrm{eff}} \sim \varepsilon A_{\mathrm{eff}} L
\]
nicht verschwindend, da die effektive Oberfläche \(A_{\mathrm{eff}}\) aufgrund der großen Zahl mikroskopischer Streu- und Emissionsprozesse sehr groß ist. Das kosmische
Plasma ist daher im CMB-Bereich transparent genug, um das Planck-Spektrum nicht zu verzerren, aber gekoppelt genug, um ein thermisches Gleichgewicht mit der durch
Rotverschiebung erzeugten Heizrate herzustellen. Die Kombination aus geringer Emissivität, großer effektiver Oberfläche und nicht-verschwindender optischer Tiefe erklärt
die beobachtete CMB-Temperatur als stationäres Gleichgewicht eines dünnen, nahezu durchsichtigen Plasmas.

\section{Fazit}
Die CMB-Temperatur ergibt sich als Gleichgewicht zwischen kosmischer Rotverschiebungsheizung und schwacher thermischer Abstrahlung eines dünnen, nahezu transparenten
Plasmas. Die fraktale Struktur liefert die kosmische Verlustkonstante \(\bar{\alpha}(L)\) rein theoretisch, während die beobachtete Temperatur die kombinierte
Plasmaparametergröße \(X = u_\gamma/(\varepsilon A_{\mathrm{eff}})\) festlegt. Beide Seiten ergeben ein konsistentes, vollständig physikalisches Bild, das ohne klassische
tired-light-Mechanismen auskommt und die CMB-Temperatur als stationäres thermodynamisches Resultat eines fraktal strukturierten Universums versteht.

\chapter{Vollständige Lösungen exemplarischer Systeme}
\label{app:vollstaendige-loesungen}

\section{Einleitung}
Dieser Anhang präsentiert mathematisch vollständige Lösungen für charakteristische physikalische Systeme im Rahmen der \gls{iwt}. Während Anhang~C numerische Beispiele
enthält, konzentrieren sich diese analytischen Lösungen auf die Demonstration fundamentaler Prinzipien. Alle Lösungen werden sowohl in ihrer diskreten Grundform als auch im
kontinuierlichen Grenzfall dargestellt.

\section{Das Zwei-Körper-Problem mit Weber-Gravitation}
\label{sec:loesung-zweikoerper}

\subsection{Problemstellung}
Zwei Massen $m_1$ und $m_2$ mit $m_2 \ll m_1$ bewegen sich unter dem Einfluss der \gls{wg}.

\subsection{Diskrete Bewegungsgleichungen}
\[
m_2 \frac{\Delta^2 \vec{r}_n}{T^2} = -G \frac{m_1 m_2}{r_n^2} 
\left[ 1 - \frac{1}{c^2} \left( \frac{\Delta r_n}{T} \right)^2 + \beta \frac{r_n}{c^2} \cdot \frac{\Delta^2 r_n}{T^2} \right] \hat{\vec{r}}_n
\]

\subsection{Reduktion auf Ein-Körper-Problem}
Mit reduzierter Masse $\mu = \frac{m_1 m_2}{m_1 + m_2}$ und Relativkoordinate $\vec{r}_n = \vec{r}_{2,n} - \vec{r}_{1,n}$:
\[
\mu \frac{\Delta^2 \vec{r}_n}{T^2} = -G \frac{m_1 m_2}{r_n^2} 
\left[ 1 - \frac{1}{c^2} \left( \frac{\Delta r_n}{T} \right)^2 + \beta \frac{r_n}{c^2} \cdot \frac{\Delta^2 r_n}{T^2} \right] \hat{\vec{r}}_n
\]

\subsection{Polarkoordinaten und Drehimpulserhaltung}
In Polarkoordinaten $(r_n, \theta_n)$ mit Drehimpulserhaltung $h = r_n^2 \frac{\Delta \theta_n}{T}$:
\begin{align*}
\mu \left( \frac{\Delta^2 r_n}{T^2} - r_n \left( \frac{\Delta \theta_n}{T} \right)^2 \right) &= -G \frac{m_1 m_2}{r_n^2} \left[ 1 - \frac{1}{c^2} \left( \frac{\Delta r_n}{T} \right)^2 + \beta \frac{r_n}{c^2} \cdot \frac{\Delta^2 r_n}{T^2} \right] \\
\mu r_n^2 \frac{\Delta \theta_n}{T} &= h = \text{konstant}
\end{align*}

\subsection{Bahn-Gleichung}
Mit $u_n = 1/r_n$ und Entwicklung bis zur ersten Ordnung in $1/c^2$:
\[
\frac{\Delta^2 u_n}{\Delta \theta_n^2} + u_n = \frac{G(m_1 + m_2)}{h^2} \left[ 1 + \frac{3G(m_1 + m_2)}{c^2} u_n \right]
\]

\subsection{Analytische Lösung}
Die Lösung lautet:
\[
u_n = \frac{G(m_1 + m_2)}{h^2} \left[ 1 + e \cos\left( (1 - \delta) \theta_n \right) \right]
\]
mit
\[
\delta = \frac{3G(m_1 + m_2)}{a(1 - e^2)c^2}
\]

\subsection{Periheldrehung}
Pro Umlauf:
\[
\Delta \theta = 2\pi \delta = \frac{6\pi G(m_1 + m_2)}{a(1 - e^2)c^2}
\]
Für das System Sonne-Merkur:
\[
\Delta \theta \approx 42.98'' \text{ pro Jahrhundert}
\]

\section{Der harmonische Oszillator in diskreter Darstellung}
\label{sec:loesung-harmonischer-oszillator}

\subsection{Diskrete Schrödinger-Gleichung}
\[
i\hbar \frac{\psi_{n+1} - \psi_n}{T} = \left( -\frac{\hbar^2}{2m} \Delta_x^2 + \frac{1}{2} m\omega^2 x^2 \right) \psi_n
\]
mit dem diskreten Laplace-Operator $\Delta_x^2$.

\subsection{Stationäre Lösungen}
Ansatz: $\psi_{n,k} = \phi_k e^{-iE_k nT/\hbar}$
\[
\left( -\frac{\hbar^2}{2m} \Delta_x^2 + \frac{1}{2} m\omega^2 x^2 \right) \phi_k = E_k \phi_k
\]

\subsection{Diskrete Eigenfunktionen}
Auf einem äquidistanten Gitter $x_j = j\Delta x$:
\[
-\frac{\hbar^2}{2m(\Delta x)^2} (\phi_{j+1} - 2\phi_j + \phi_{j-1}) + \frac{1}{2} m\omega^2 (j\Delta x)^2 \phi_j = E \phi_j
\]

\subsection{Numerische Eigenwerte}
Für $\Delta x \to 0$ reproduziert die diskrete Gleichung die bekannten Energieniveaus:
\[
E_k = \left( k + \frac{1}{2} \right) \hbar \omega, \quad k = 0, 1, 2, \ldots
\]

\subsection{Informationsdichte}
\[
\rho_{I,j} = |\phi_j|^2 = \frac{1}{2^k k!} \sqrt{\frac{m\omega}{\pi\hbar}} H_k^2\left( \sqrt{\frac{m\omega}{\hbar}} x_j \right) e^{-m\omega x_j^2/\hbar}
\]

\section{Plasma-Oszillationen im diskreten Netzwerk}
\label{sec:loesung-plasma-oscillationen}

\subsection{Diskretes Plasma-Modell}
Elektronen auf diskreten Positionen $\vec{x}_{j,n}$ im Hintergrund positiver Ionen.

\subsection{Bewegungsgleichungen}
\[
m_e \frac{\Delta^2 \vec{x}_{j,n}}{T^2} = -e \vec{E}_{j,n}
\]
mit dem elektrischen Feld aus der \gls{wed}:
\[
\vec{E}_{j,n} = \frac{1}{4\pi\varepsilon_0} \sum_{k \neq j} \frac{e(\vec{x}_{j,n} - \vec{x}_{k,n})}{|\vec{x}_{j,n} - \vec{x}_{k,n}|^3} 
\left[ 1 - \frac{1}{c^2} \left( \frac{\Delta r_{jk,n}}{T} \right)^2 + \frac{2r_{jk,n}}{c^2} \cdot \frac{\Delta^2 r_{jk,n}}{T^2} \right]
\]

\subsection{Linearisierung}
Für kleine Auslenkungen $\vec{x}_{j,n} = \vec{x}_j^0 + \vec{\xi}_{j,n}$:
\[
m_e \frac{\Delta^2 \vec{\xi}_{j,n}}{T^2} = -\frac{e^2}{4\pi\varepsilon_0} \sum_{k \neq j} \frac{\vec{\xi}_{j,n} - \vec{\xi}_{k,n}}{|\vec{x}_j^0 - \vec{x}_k^0|^3}
\]

\subsection{Plasmafrequenz}
Im Kontinuumslimes:
\[
\frac{\Delta^2 \vec{\xi}_n}{T^2} = -\omega_p^2 \vec{\xi}_n
\]
mit
\[
\omega_p^2 = \frac{n_0 e^2}{\varepsilon_0 m_e}
\]

\subsection{Diskrete Lösung}
\[
\vec{\xi}_n = \vec{\xi}_0 \cos(\omega_p nT)
\]

\section{Lichtausbreitung in fraktaler Geometrie}
\label{sec:loesung-licht-fraktal}

\subsection{Wirkung für Photonen}
\[
S_\gamma = \sum_n \left[ \frac{E}{c^2} \left( \frac{\Delta \vec{x}_n}{T} \right)^2 - \frac{2GME}{c^4 r_n} \left( 1 - \frac{1}{c^2} \left( \frac{\Delta r_n}{T} \right)^2 + \frac{r_n}{c^2} \cdot \frac{\Delta^2 r_n}{T^2} \right) \right] T
\]

\subsection{Geodätengleichung}
\[
\frac{\Delta^2 x_n^\mu}{T^2} + \Gamma^\mu_{\alpha\beta} \frac{\Delta x_n^\alpha}{T} \frac{\Delta x_n^\beta}{T} = 0
\]
mit diskreten Christoffel-Symbolen in fraktaler Geometrie.

\subsection{Frequenzabhängige Lichtablenkung}
\[
\Delta \theta(\omega) = \frac{4GM}{c^2 b} \left[ 1 + \alpha(D) \left( \frac{\omega_0}{\omega} \right)^{3-D} \right]
\]
mit fraktaler Dimension $D \approx 2.71$ und
\[
\alpha(D) = \frac{\Gamma(4-D)}{(3-D)\Gamma(3-D)}
\]

\section{Nichtlineare Schrödinger-Gleichung aus IWT}
\label{sec:loesung-nls}

\subsection{Erweiterte Wirkung}
\[
S = \sum_n \left[ i\hbar \psi_n^* \frac{\psi_{n+1} - \psi_n}{T} - \frac{\hbar^2}{2m} |\Delta_x \psi_n|^2 - g |\psi_n|^4 \right] T
\]

\subsection{Variation}
\[
i\hbar \frac{\psi_{n+1} - \psi_n}{T} = -\frac{\hbar^2}{2m} \Delta_x^2 \psi_n + 2g |\psi_n|^2 \psi_n
\]

\subsection{Soliton-Lösung}
Im Kontinuumslimes für 1D:
\[
\psi(x,t) = A \operatorname{sech}\left( \frac{x - vt}{\xi} \right) e^{i(kx - \omega t)}
\]
mit
\[
A = \sqrt{\frac{\hbar^2}{2mg\xi^2}}, \quad \omega = \frac{\hbar k^2}{2m} - \frac{\hbar}{2m\xi^2}
\]

\section{Thermisches Gleichgewicht im diskreten Netzwerk}
\label{sec:loesung-thermisches-gleichgewicht}

\subsection{Informations-Hamiltonian}
\[
H = \sum_k \left[ \alpha (\Delta_t I_{k,n})^2 + \beta (\Delta_x I_{k,n})^2 + \gamma \frac{(\Delta_x I_{k,n})^2}{I_{k,n}} \right]
\]

\subsection{Canonische Verteilung}
\[
P(\{I_k\}) = \frac{1}{Z} \exp\left( -\frac{H}{k_B T} \right)
\]
mit Zustandssumme
\[
Z = \int \exp\left( -\frac{H}{k_B T} \right) \prod_k dI_k
\]

\subsection{Korrelationsfunktionen}
\[
\langle I_k I_l \rangle - \langle I_k \rangle \langle I_l \rangle = \frac{k_B T}{\beta} G_{kl}
\]
mit Greens-Funktion $G_{kl}$ des diskreten Laplace-Operators.

\section{Zusammenfassung der Lösungsmethoden}
\begin{table}[ht]
\centering
\begin{tabular}{p{0.25\textwidth}p{0.3\textwidth}p{0.3\textwidth}}
\hline
\textbf{System} & \textbf{Lösungsmethode} & \textbf{Charakteristika} \\
\hline
Zwei-Körper-Problem & Diskretes Variationsprinzip & Periheldrehung ohne Raumzeitkrümmung \\
\hline
Harmonischer Oszillator & Diskrete Eigenwertgleichung & Quantisierung aus globalem Funktional \\
\hline
Plasma-Oszillationen & Lineare Störungstheorie & Emergenz der Plasmafrequenz \\
\hline
Lichtausbreitung & Diskrete Geodätengleichung & Frequenzabhängige Ablenkung \\
\hline
Nichtlineare Wellen & Variation erweiterter Wirkung & Soliton-Lösungen \\
\hline
Thermisches Gleichgewicht & Statistische Mechanik & Korrelationsfunktionen \\
\hline
\end{tabular}
\caption{Übersicht der gelösten Systeme und Methoden}
\end{table}

\subsection{Allgemeine Lösungsstrategien}
\begin{enumerate}
    \item \textbf{Diskretisierung}: Übertragung des Problems auf diskretes Netzwerk
    \item \textbf{Variationsprinzip}: Ableitung der Bewegungsgleichungen aus diskreter Wirkung
    \item \textbf{Linearisierung}: Behandlung kleiner Störungen
    \item \textbf{Symmetrien}: Ausnutzung von Erhaltungsgrößen
    \item \textbf{Kontinuumslimes}: Übergang zu etablierten Gleichungen
\end{enumerate}

\subsection{Schlussfolgerungen}
\begin{itemize}
    \item Die \gls{iwt} bietet konsistente Lösungen für alle grundlegenden physikalischen Systeme
    \item Die diskrete Formulierung ist mathematisch wohldefiniert und lösbar
    \item Im entsprechenden Grenzfall werden alle bekannten Ergebnisse reproduziert
    \item Die Theorie macht darüber hinaus spezifische neue Vorhersagen
    \item Die Lösungsmethoden sind allgemein und auf komplexere Systeme übertragbar
\end{itemize}
Diese vollständigen analytischen Lösungen demonstrieren die mathematische Konsistenz und Anwendbarkeit der \gls{iwt} über das gesamte Spektrum physikalischer Phänomene
hinweg.

% ============================================================
% Anhang 1 – Informations-Lagrange-Funktional
% ============================================================
\section{Informations-Lagrange-Funktional}
\subsection{Grundstruktur}
Das Informations-Lagrange-Funktional $\mathcal{L}_I$ beschreibt die Dynamik eines Informationszustands $I(\mathbf{x},t)$ und setzt sich aus einem lokalen und einem 
globalen Anteil zusammen:
\[
\mathcal{L}_I = \mathcal{L}_{\text{lokal}} + \mathcal{L}_{\text{global}}.
\]
Der lokale Anteil erzeugt Weber-Dynamik, der globale Anteil Bohm-Dynamik.

\subsection{Lokaler Anteil (Weber-Struktur)}
Der lokale Informationsfluss wird durch Informationsdichte $\rho_I$ und Informationsgradienten bestimmt. Ein minimaler Ansatz lautet:
\[
\mathcal{L}_{\text{lokal}} =
\frac{1}{2}\,\alpha\, \rho_I 
\left( \frac{\partial I}{\partial t} \right)^2
-
\frac{1}{2}\,\beta\, \rho_I 
\left( \nabla I \right)^2.
\]

\subsection{Globaler Anteil (Bohm-Struktur)}
Der globale Anteil ist ein Funktional der Form:
\[
\mathcal{L}_{\text{global}} =
- \gamma \, \rho_I \,
\frac{\nabla^2 \sqrt{\rho_I}}{\sqrt{\rho_I}}.
\]
Dies entspricht strukturell dem Bohm-Potential, jedoch als Informationsoperator.

\subsection{Gesamtes Funktional}
\[
\boxed{
\mathcal{L}_I =
\frac{1}{2}\,\alpha\, \rho_I 
\left( \frac{\partial I}{\partial t} \right)^2
-
\frac{1}{2}\,\beta\, \rho_I 
\left( \nabla I \right)^2
-
\gamma \, \rho_I \,
\frac{\nabla^2 \sqrt{\rho_I}}{\sqrt{\rho_I}}
}
\]
Dieses Funktional liefert durch Variation die Weber-Kraft, das Bohm-Potential, die Kontinuitätsgleichung sowie emergente Energie- und Impulsgrößen.

% ============================================================
% Anhang 2 – Informationsmetrik und emergente Raumzeit
% ============================================================

\section{Informationsmetrik und emergente Raumzeit}
\subsection{Grundidee}
Raum entsteht als effektive Metrik der Kopplungsstruktur eines Informationsnetzes. Die Metrik ergibt sich aus der Kopplungsdichte $C(\mathbf{x})$:
\[
g_{ij}(\mathbf{x}) = f\!\left( C(\mathbf{x}) \right)\, \delta_{ij}.
\]

\subsection*{Fraktale Dimension}
Die fraktale Dimension $D_f$ folgt aus der Skalierung der Kopplungsdichte:
\[
C(\lambda \mathbf{x}) = \lambda^{D_f - 3} C(\mathbf{x}).
\]
Damit gilt:
\[
D_f = 3 \Rightarrow \text{klassischer Raum}, \qquad
D_f < 3 \Rightarrow \text{fraktale Geometrie}.
\]

\subsection{Zeit als Informationsparameter}
Zeit entsteht aus der invertierbaren Transformation des Informationszustands:
\[
t \equiv \tau(I), 
\qquad
\frac{dI}{dt} \neq 0.
\]
Zeit ist somit ein Ordnungsparameter der Informationsentwicklung.

\subsection{Effektive Raumzeit}
Die emergente Raumzeit besitzt die Metrik:
\[
ds^2 = c_{\text{eff}}^2(I)\, dt^2 - g_{ij}(I)\, dx^i dx^j,
\]
mit einer effektiven Lichtgeschwindigkeit:
\[
c_{\text{eff}} = \sqrt{\frac{\alpha}{\beta}}.
\]
Damit ist $c$ ein emergenter Parameter der Informationskopplung.

% ============================================================
% Anhang 3 – Kopplungsparameter und Naturkonstanten
% ============================================================

\section{Kopplungsparameter und Naturkonstanten}
\subsection{Lichtgeschwindigkeit}
\[
c = \sqrt{\frac{\alpha}{\beta}}.
\]
Die maximale Informationsflussrate ergibt sich aus dem Verhältnis der lokalen Kopplungsparameter.

\subsection{Planck-Konstante}
\[
h = k \cdot \gamma,
\]
wobei $k$ ein dimensionsloser Skalierungsfaktor der globalen Informationsorganisation ist.

\subsection{Gravitationskonstante}
\[
G = \frac{1}{4\pi} \frac{\beta}{C_0},
\]
mit $C_0$ als mittlerer Kopplungsdichte des Informationsnetzes.

\subsection{Feinstrukturkonstante}
\[
\alpha_{\text{fs}} = F(\alpha,\beta,\gamma,C_0),
\]
eine reine Funktion der Informationskopplung.

% ============================================================
% Anhang 4 – Numerische Simulation eines Informationsnetzes
% ============================================================
\section{Numerische Simulation eines Informationsnetzes}
\subsection{Diskretisierung}

Das Informationsfeld wird auf einem Gitter $I_{i,j,k}(t)$ definiert.  
Die Kopplungsstruktur ist ein Graph:
\[
C_{(i,j,k),(i',j',k')}.
\]

\subsection{Evolutionsgleichung}
Die Variation von $\mathcal{L}_I$ liefert:
\[
\alpha \rho_I \frac{\partial^2 I}{\partial t^2}
-
\beta \rho_I \nabla^2 I
+
\gamma 
\left( 
\frac{\nabla^2 \sqrt{\rho_I}}{\sqrt{\rho_I}}
\right)' 
= 0.
\]

\subsection{Algorithmus}

\begin{enumerate}
    \item Initialisierung von $I(\mathbf{x},0)$ und $\rho_I(\mathbf{x},0)$.
    \item Berechnung lokaler Gradienten.
    \item Berechnung des globalen Bohm-Terms.
    \item Aktualisierung von $I$ über einen Zeitschritt $\Delta t$.
    \item Wiederholung der Schritte 2–4.
\end{enumerate}

\subsection{Beobachtbare Größen}
\begin{itemize}
    \item emergente Raumzeit,
    \item effektive Lichtgeschwindigkeit,
    \item Gravitationspotential,
    \item Wellenphänomene,
    \item Nichtlokalität,
    \item Rotationskurven,
    \item CMB-Struktur.
\end{itemize}


% --- Backmatter ---
\backmatter
\printbibliography[title=Literaturverzeichnis]
\printglossary[title=Glossar]
\printglossary[type=acronym, title=Abkürzungen]

\end{document}
