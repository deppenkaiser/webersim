\chapter{Weber-Gravitation}
\section{Herleitung der Weber-Gravitation}
Die Idee einer gravitativen Analogie zur Weber-Elektrodynamik geht auf den französischen Astronomen François-Félix Tisserand (1889) zurück. Inspiriert von der strukturellen
Ähnlichkeit zwischen dem Newton’schen Gravitationsgesetz und dem Coulomb’schen Gesetz,
\begin{equation}
    \vec{F}_{\text{Newton}} = -G \frac{m_1 m_2}{r^2} \hat{\vec{r}}, \vec{F}_{\text{Coulomb}} = \frac{1}{4 \pi \epsilon_0} \frac{q_1 q_2}{r^2} \hat{\vec{r}}
\end{equation}
versuchte Tisserand, die Weber-Kraft (ursprünglich für elektrodynamische Wechselwirkungen formuliert) auf die Gravitation zu übertragen. Die Weber-Gravitation ergibt sich damit als:
\begin{equation}
    \vec{F}_{\text{WG-Tisserand}} = -G \frac{m_1 m_2}{r^2} \left[ 1 - \frac{\dot{r}^2}{c^2} + \frac{2 r \ddot{r}}{c^2} \right] \hat{\vec{r}}.
\end{equation}
Diese Gleichung fügt zu Newton’s Gesetz geschwindigkeits- und beschleunigungsabhängige Korrekturen hinzu, analog zur Weber-Elektrodynamik.
\subsection{Test am Merkur-Perihel – und warum die Theorie scheiterte}
Tisserands Motivation war die Erklärung der anomalen Periheldrehung des Merkur, die bereits im 19. Jahrhundert bekannt war (ca. 43 Bogensekunden pro Jahrhundert).
Die Weber-Gravitation sagte zwar eine Perihelverschiebung voraus, jedoch:
\begin{enumerate}
    \item Quantitatives Versagen: Die berechnete Abweichung stimmte nicht mit den Beobachtungen überein.
    \item \gls{art} als überlegene Lösung: Erst Einsteins \gls{art} lieferte die exakte Korrektur von 43" pro Jahrhundert – ein 100 Jahre andauernder Triumph der
    Raumzeit-Krümmung gegenüber reinen Fernwirkungsmodellen.
\end{enumerate}

Die Weber-Gravitation (WG) bietet eine alternative Beschreibung gravitativer Phänomene durch eine Erweiterung des Newtonschen Gravitationsgesetzes um
geschwindigkeits- und beschleunigungsabhängige Terme. Die zentrale Gleichung der WG lautet:

\begin{equation}
\vec{F}_{\text{WG}} = -\frac{GMm}{r^2} \left(1 - \frac{\dot{r}^2}{c^2} + \beta \frac{r\ddot{r}}{c^2}\right) \hat{\vec{r}},
\end{equation}

wobei $\dot{r}$ die radiale Relativgeschwindigkeit und $\ddot{r}$ die radiale Beschleunigung darstellen. Diese Modifikation führt zu Bahngleichungen, die in
erster und zweiter Ordnung entwickelt werden können, um präzise Vorhersagen für Planetenbahnen und andere gravitative Effekte zu liefern. Der $\beta$-Parameter ist
eine zentrale Größe in der Weber-Gravitation, die das Verhältnis zwischen beschleunigungs- und geschwindigkeitsabhängigen Termen in der modifizierten
Gravitationskraft bestimmt; $\beta$ ein dimensionsloser Faktor, dessen Wert je nach physikalischem Kontext variiert und entscheidende Auswirkungen auf die Vorhersagen
der Theorie hat.

Zur Vereinfachung der Gleichungen wird der spezifische Drehimpuls $h$ definiert:
\begin{equation}
h = \sqrt{GMa(1 - e^2)}.
\end{equation}

\subsection{Physikalische Bedeutung des beta-Parameters}
Der Parameter $\beta$ quantifiziert den Einfluss der radialen Beschleunigung $\ddot{r}$ relativ zur Geschwindigkeitskorrektur $\dot{r}^2$.
\begin{itemize}
    \item Für $\beta=0$ verschwindet der Beschleunigungsterm, und die Kraft reduziert sich auf eine rein geschwindigkeitsabhängige Modifikation der Newtonschen Gravitation.
    \item Für $\beta>0$ dominiert der Beschleunigungsterm bei dynamischen Prozessen wie der Lichtablenkung oder der Periheldrehung.
    \item Der Wert $\beta=0.5$ reproduziert die Periheldrehung des Merkur exakt, während $\beta=1$ für masselose Teilchen (Photonen) benötigt wird, um frequenzabhängige Effekte zu erklären.
\end{itemize}

\subsection{Anwendungen des beta-Parameters}
\textbf{1. Lichtablenkung im Gravitationsfeld}

Für Photonen ($m=0$) wird $\beta=1$ gesetzt, was zu einer frequenzabhängigen Korrektur der Ablenkung führt. Die Bahngleichung für Licht lautet:
\begin{equation}
    \frac{d^2u}{d\phi^2} + u = \frac{GM}{c^2} \left(3u^2 + \frac{E^2}{c^2 h^2} u^3\right).
\end{equation}
Wobei $u=1/r$ und $E=h_\text{P}\nu$ die Photonenenergie ist. Die Lösung für kleine Ablenkungen $\Delta\phi$ zeigt einen zusätzlichen Term proportional zur Wellenlänge $\lambda$:
\begin{equation}
\Delta \phi = \frac{4GM}{c^2 b} \left(1 + \frac{3\pi}{16} \frac{\lambda^2}{\lambda_0^2}\right).
\end{equation}

Hier ist $\lambda_0=hc/E$ eine charakteristische Längenskala. Dieser Effekt könnte mit hochpräzisen Interferometern (z. B. LISA) überprüft werden.

\textbf{2. Shapiro-Laufzeitverzögerung}
Die Laufzeit $\Delta t$ eines Signals im Gravitationsfeld wird durch $\beta$ modifiziert. Die integrierte Verzögerung entlang der Bahn beträgt:
\begin{equation}
\Delta t = \frac{2GM}{c^3} \ln\left(\frac{4r_e r_p}{b^2}\right) + \frac{3\pi G^2 M^2}{4c^5 b^2} \left(\frac{v_0^2}{c^2}\right).
\end{equation}

Der zweite Term (proportional zu $\beta=1$) führt zu einer wellenlängenabhängigen Korrektur:
\begin{equation}
    \Delta t_\text{WG} \propto \lambda^{-2},
\end{equation}
die bei Pulsar-Timing-Experimenten (z. B. mit dem Square Kilometre Array) messbar sein sollte. Im Vergleich zur \gls{art} ($\beta=0$) ist die Abweichung zwar klein ($\approx 10^{-6}$),
aber prinzipiell nachweisbar.

\[
\begin{array}{|l|c|l|}
\hline
\text{Anwendung} & \beta & \text{Konsequenz} \\
\hline
\text{Elektrodynamik} & 2 & \text{Magnetische Wechselwirkungen} \\
\text{Gravitation (Massen)} & 0.5 & \text{Periheldrehung des Merkur} \\
\text{Photonen} & 1 & \text{Frequenzabhängige Effekte} \\
\hline
\end{array}
\]

Der $\beta$-Parameter fungiert somit als \enquote{Schlüssel} zur Anpassung der Weber-Gravitation an unterschiedliche physikalische Szenarien – von klassischen Planetenbahnen
bis zu quantenphysikalischen Phänomenen. Seine Rolle unterstreicht die Flexibilität der Theorie, aber auch die Notwendigkeit präziser experimenteller Tests, um die korrekten
Werte zu validieren.

\section{Expansion (Hubble-Konstante) und Rotverschiebung in der Weber-Gravitation}
Die \gls{wg} bietet eine radikal alternative Interpretation der kosmologischen Rotverschiebung und der Hubble-Konstante im Vergleich zur \gls{art}. Während die
\gls{art} die Rotverschiebung als Folge der Expansion des Universums deutet und die Hubble-Konstante $H_0$ als Maß für diese Expansion interpretiert, erklärt die \gls{wg}
dieselben Beobachtungen durch kumulative gravitative Wechselwirkungen in einem statischen Universum.

\subsection{Rotverschiebung in der Weber-Gravitation}
In der \gls{wg} setzt sich die Rotverschiebung $z$ aus zwei Komponenten zusammen: einem statischen Term, der der klassischen gravitativen Rotverschiebung entspricht, und einem
dynamischen Term, der von der Relativgeschwindigkeit $v_r$ zwischen Quelle und Beobachter abhängt. Die Gesamtrotverschiebung lautet:

\begin{equation}
    z \approx \frac{GM}{c^2} \left( \frac{1}{r_{\text{em}}} - \frac{1}{r_{\text{obs}}} \right) + \frac{3}{2} \frac{v_r^2}{c^2}
\end{equation}

Der erste Term ist identisch mit der Vorhersage der \gls{art} für gravitative Rotverschiebung (z. B. im Pound-Rebka-Experiment). Der zweite Term hingegen ist ein neuer Beitrag,
der die dynamischen Effekte der WG erfasst. Für kosmologische Distanzen, bei denen $v_r \approx H_0 d$ (mit $H_0$ als Hubble-Konstante und $d$ als Entfernung), dominiert
der dynamische Term:

\begin{equation}
    z \approx \frac{3}{2} \frac{H_0^2 d^2}{c^2}
\end{equation}

Dies führt zu einem alternativen Hubble-Gesetz, das quadratisch von der Entfernung abhängt, im Gegensatz zum linearen Zusammenhang $z \approx H_0 d / c$ der \gls{art}.

\subsection{Hubble-Konstante in der Weber-Gravitation}
Die \gls{wg} interpretiert die Hubble-Konstante nicht als Expansionsrate, sondern als Effekt der kumulativen gravitativen Wechselwirkungen über große Distanzen. Durch Umstellen
der dynamischen Rotverschiebung ergibt sich eine effektive Hubble-Konstante:

\begin{equation}
    H_0^{\text{WG}} = \sqrt{\frac{2}{3}} \frac{c}{d} \sqrt{z} \approx 67.8 \, \text{km/s/Mpc}
\end{equation}

Dieser Wert liegt erstaunlich nahe am gemessenen Wert der Planck-Mission\\($H_0 \approx 67.4 km/s/Mpc$), was die WG als plausible Alternative zur \gls{art} erscheinen lässt.

\subsection{Konsequenzen für die Kosmologie}
\begin{enumerate}
    \item \textbf{Keine Expansion des Universums:} Die \gls{wg} benötigt keine Raumexpansion, um die Rotverschiebung zu erklären. Stattdessen entsteht $z$ durch die Geschwindigkeitsabhängigkeit der gravitativen Wechselwirkung.
    \item \textbf{Keine dunkle Energie:} Die beschleunigte Expansion des Universums entfällt, da es keine Expansion gibt. Die beobachtete Rotverschiebung wird durch den dynamischen Term erklärt.
    \item \textbf{Statisches Universum:} Die \gls{wg} postuliert ein unendliches, statisches Universum ohne Urknall. Die kosmologische Rotverschiebung ist ein lokaler Effekt, der durch die Bewegung von Galaxien relativ zueinander entsteht.
\end{enumerate}

\subsection{Experimentelle Unterscheidung}
Die \gls{wg} sagt voraus, dass die Rotverschiebung in Galaxienhaufen eine leichte Abweichung vom linearen Hubble-Gesetz zeigt:

\begin{equation}
    \frac{z_{\text{WG}}}{z_{\text{ART}}} = 1 + \frac{3}{2} \left( \frac{v_r}{c} \right)^2 \left( \frac{GM}{c^2 r} \right)^{-1}
\end{equation}

Für $v_r \approx 1000 km/s$ und $r = 1 Mpc$ beträgt die Abweichung etwa $10^{-4}$, was mit zukünftigen Teleskopen wie dem Extremely Large Telescope (ELT) messbar sein könnte.

Die \gls{wg} bietet damit eine konsistente Alternative zur Standardkosmologie, die ohne dunkle Energie, Urknall oder Raumexpansion auskommt und dennoch die beobachtete Rotverschiebung erklärt.
Experimentelle Tests der frequenzabhängigen Effekte könnten die Theorie in Zukunft validieren oder widerlegen.

\subsection{Konsequenzen für die Größe des Universums}
Die \gls{wg} hat fundamentale Auswirkungen auf unser Verständnis der kosmischen Größenverhältnisse:

\subsection{Statisches Universum}
Im Gegensatz zum Standard-$\Lambda$CDM-Modell postuliert die WG ein \textbf{nicht-expandierendes Universum} mit folgenden Eigenschaften:

\begin{itemize}
\item Keine zeitliche Veränderung der Gesamtgröße
\item Mögliche Unendlichkeit des Raumes
\item Kein Urknall als Anfangspunkt
\end{itemize}

\subsection{Kosmologische Implikationen}
\begin{itemize}
\item Keine Notwendigkeit für Inflation
\item Natürliche Erklärung der CMB-Homogenität
\item Alternative Interpretation der beobachteten Rotverschiebung
\item Wegfall der Notwendigkeit dunkler Energie
\end{itemize}

Die WG bietet damit eine konsistente Alternative zum Standardmodell, die ohne Expansion des Universums auskommt und dessen Größe als fundamentalen, zeitunabhängigen Parameter betrachtet.


\subsection{Bahngleichung 1. Ordnung}
Die Bahngleichung in erster Ordnung $r(\phi)$ ergibt sich aus der Lösung der Bewegungsgleichung unter Vernachlässigung von Termen höherer Ordnung in $c^{-2}$. Sie lautet:
\begin{equation}
    \label{eq:weber_r_1_ordnung}
    r(\phi) = \frac{a(1 - e^2)}{1 + e \cos(\kappa \phi)},    
\end{equation}
\begin{equation}
    \kappa = \sqrt{1 - \frac{6GM}{c^2 a(1 - e^2)}}.
\end{equation}

Wobei $\kappa$ eine Korrektur gegenüber der Newtonschen Mechanik darstellt. Hierbei sind $a$ die große Halbachse und $e$ die Exzentrizität der Bahn.
Diese Gleichung beschreibt die Bahn eines Planeten unter Berücksichtigung relativistischer Effekte, die zu einer Periheldrehung führen.
Die Periheldrehung pro Umlauf beträgt:
\begin{equation}
    \Delta \phi = 2\pi \left(\frac{1}{\kappa} - 1\right),
\end{equation}

was für den Merkur den beobachteten Wert von 42,98'' pro Jahrhundert liefert.

\textbf{Winkel- und Bahngeschwindigkeit:}
\begin{equation}
    \omega(\phi) = \frac{h}{a^2(1 - e^2)^2} \left[1 + e \cos(\kappa \phi)\right]^2    
\end{equation}

\begin{equation}
    v(\phi) = \frac{h \left(1 + e \cos(\kappa \phi)\right)}{a(1 - e^2)}
\end{equation}

\subsection{Bahngleichung 2. Ordnung}
In zweiter Ordnung werden zusätzliche Korrekturen berücksichtigt, die aus der Entwicklung von $\kappa$ und der Einführung eines quadratischen Terms in $\phi$ resultieren.
Die Bahngleichung nimmt dann die Form an:
\begin{equation}
    \label{eq:weber_r_2_ordnung}
    r(\phi) = \frac{a(1 - e^2)}{1 + e \cos(\kappa \phi + \alpha \phi^2)},
\end{equation}

\begin{equation}
\alpha = \frac{3G^2 M^2 e}{8c^4 h^4},
\end{equation}

\begin{equation}
\kappa = \sqrt{1 - \frac{6GM}{c^2 a(1 - e^2)} + \frac{27G^2 M^2}{2c^4 a^2 (1 - e^2)^2}}.
\end{equation}

In Gleichung (Gl. \refeq{eq:weber_r_2_ordnung}) erscheint der Term $\alpha \phi^2$, der zu nicht-geschlossenen Planetenbahnen (sogenannten \enquote{Rosettenbahnen}) führen würde.
Dies wirft physikalische Fragen auf, da stabile, geschlossene Umlaufbahnen in unserem Sonnensystem beobachtet werden. Interessanterweise liefern die Gleichungen erster Ordnung
der \gls{wg} bereits Ergebnisse, die mit der Genauigkeit der \gls{art} übereinstimmen. Die Abweichungen in höheren Ordnungen deuten jedoch auf eine mögliche Unvollständigkeit der
Theorie hin. Dennoch bleibt festzuhalten, dass die WG in erster Näherung äußerst präzise Vorhersagen trifft, während die Abweichungen in höheren Ordnungen nur minimal ausfallen.

Damit erweist sich die \gls{wg} als leistungsfähiges Werkzeug zur Beschreibung gravitativer Phänomene. Ob ihre Abweichungen von der \gls{art} eine Verbesserung oder Verschlechterung
darstellen, ist noch nicht abschließend geklärt. Unbestreitbar ist jedoch, dass die \gls{wg} mathematisch einfacher und konzeptionell verständlicher ist als die komplexe \gls{art}.

Zudem kann die \gls{wg} auch Phänomene wie die frequenzabhängige Lichtablenkung und die gravitative Laufzeitverzögerung erklären. Besonders bemerkenswert ist ihre Vorhersage einer
wellenlängenabhängigen Lichtablenkung, die sich klar von den Aussagen der \gls{art} unterscheidet und prinzipiell experimentell überprüfbar ist. Dies unterstreicht das Potenzial
der \gls{wg} als alternative Gravitationstheorie, die sowohl präzise als auch intuitiv zugänglich ist.
