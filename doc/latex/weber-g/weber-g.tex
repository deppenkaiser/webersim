\chapter{Weber-Gravitation}
\section{Herleitung der Weber-Gravitation}
Die Idee einer gravitativen Analogie zur Weber-Elektrodynamik geht auf den französischen Astronomen François-Félix Tisserand (1889) zurück. Inspiriert von der strukturellen
Ähnlichkeit zwischen dem Newton’schen Gravitationsgesetz und dem Coulomb’schen Gesetz,
\[
\vec{F}_{\text{Newton}} = -G \frac{m_1 m_2}{r^2} \hat{r}, \vec{F}_{\text{Coulomb}} = \frac{1}{4 \pi \epsilon_0} \frac{q_1 q_2}{r^2} \hat{r}
\]
versuchte Tisserand, die Weber-Kraft (ursprünglich für elektrodynamische Wechselwirkungen formuliert) auf die Gravitation zu übertragen. Die Weber-Gravitation ergibt sich damit als:
\[
\vec{F}_{\text{WG-Tisserand}} = -G \frac{m_1 m_2}{r^2} \left[ 1 - \frac{\dot{r}^2}{c^2} + \frac{2 r \ddot{r}}{c^2} \right] \hat{r}
\]
\begin{itemize}
    \item $G$: Gravitationskonstante
    \item $m_1, m_2$: Wechselwirkende Massen
    \item $r$: Abstand zwischen den Massen
    \item $\dot{r}$: Relative Radialgeschwindigkeit ($dr/dt$)
    \item $\ddot{r}$: Relative Radialbeschleunigung ($d^2r/dt^2$)
    \item $c$: Lichtgeschwindigkeit
\end{itemize}

Diese Gleichung fügt zu Newton’s Gesetz geschwindigkeits- und beschleunigungsabhängige Korrekturen hinzu, analog zur Weber-Elektrodynamik.
\subsection{Test am Merkur-Perihel – und warum die Theorie scheiterte}
Tisserands Motivation war die Erklärung der anomalen Periheldrehung des Merkur, die bereits im 19. Jahrhundert bekannt war (ca. 43 Bogensekunden pro Jahrhundert).
Die Weber-Gravitation sagte zwar eine Perihelverschiebung voraus, jedoch:
\begin{enumerate}
    \item Quantitatives Versagen: Die berechnete Abweichung stimmte nicht mit den Beobachtungen überein.
    \item ART als überlegene Lösung: Erst Einsteins \gls{art} lieferte die exakte Korrektur von 43" pro Jahrhundert – ein 100 Jahre andauernder Triumph der
    Raumzeit-Krümmung gegenüber reinen Fernwirkungsmodellen.
\end{enumerate}

Die Weber-Gravitation (WG) bietet eine alternative Beschreibung gravitativer Phänomene durch eine Erweiterung des Newtonschen Gravitationsgesetzes um
geschwindigkeits- und beschleunigungsabhängige Terme. Die zentrale Gleichung der WG lautet:
\[
\vec{F}_{\text{WG}} = -\frac{GMm}{r^2} \left(1 - \frac{\dot{r}^2}{c^2} + \beta \frac{r\ddot{r}}{c^2}\right) \hat{r}.
\]
wobei $\dot{r}$ die radiale Relativgeschwindigkeit und $\ddot{r}$ die radiale Beschleunigung darstellen. Diese Modifikation führt zu Bahngleichungen, die in
erster und zweiter Ordnung entwickelt werden können, um präzise Vorhersagen für Planetenbahnen und andere gravitative Effekte zu liefern. Der $\beta$-Parameter ist
eine zentrale Größe in der Weber-Gravitation, die das Verhältnis zwischen beschleunigungs- und geschwindigkeitsabhängigen Termen in der modifizierten
Gravitationskraft bestimmt; $\beta$ ein dimensionsloser Faktor, dessen Wert je nach physikalischem Kontext variiert und entscheidende Auswirkungen auf die Vorhersagen
der Theorie hat.

\subsection{Physikalische Bedeutung des beta-Parameters}
Der Parameter $\beta$ quantifiziert den Einfluss der radialen Beschleunigung $\ddot{r}$ relativ zur Geschwindigkeitskorrektur $\dot{r}^2$.
\begin{itemize}
    \item Für $\beta=0$ verschwindet der Beschleunigungsterm, und die Kraft reduziert sich auf eine rein geschwindigkeitsabhängige Modifikation der Newtonschen Gravitation.
    \item Für $\beta>0$ dominiert der Beschleunigungsterm bei dynamischen Prozessen wie der Lichtablenkung oder der Periheldrehung.
    \item Der Wert $\beta=0.5$ reproduziert die Periheldrehung des Merkur exakt, während $\beta=1$ für masselose Teilchen (Photonen) benötigt wird, um frequenzabhängige Effekte zu erklären.
\end{itemize}

\subsection{Anwendungen des beta-Parameters}
\textbf{1. Lichtablenkung im Gravitationsfeld}

Für Photonen ($m=0$) wird $\beta=1$ gesetzt, was zu einer frequenzabhängigen Korrektur der Ablenkung führt. Die Bahngleichung für Licht lautet:
\[
\frac{d^2u}{d\phi^2} + u = \frac{GM}{c^2} \left(3u^2 + \frac{E^2}{c^2 h^2} u^3\right).
\]
Wobei $u=1/r$ und $E=h\nu$ die Photonenenergie ist. Die Lösung für kleine Ablenkungen $\Delta\phi$ zeigt einen zusätzlichen Term proportional zur Wellenlänge $\lambda$:
\[
\Delta \phi = \frac{4GM}{c^2 b} \left(1 + \frac{3\pi}{16} \frac{\lambda^2}{\lambda_0^2}\right).
\]
Hier ist $\lambda_0=hc/E$ eine charakteristische Längenskala. Dieser Effekt könnte mit hochpräzisen Interferometern (z. B. LISA) überprüft werden.

\subsection{Bahngleichung 1. Ordnung}
Die Bahngleichung in erster Ordnung $r(\phi)$ ergibt sich aus der Lösung der Bewegungsgleichung unter Vernachlässigung von Termen höherer Ordnung in $c^{-2}$. Sie lautet:
\[
r(\phi) = \frac{a(1 - e^2)}{1 + e \cos(\kappa \phi)},
\]
\[
\kappa = \sqrt{1 - \frac{6GM}{c^2 a(1 - e^2)}}.
\]
Wobei $\kappa$ eine Korrektur gegenüber der Newtonschen Mechanik darstellt. Hierbei sind $a$ die große Halbachse und $e$ die Exzentrizität der Bahn.
Diese Gleichung beschreibt die Bahn eines Planeten unter Berücksichtigung relativistischer Effekte, die zu einer Periheldrehung führen.
Die Periheldrehung pro Umlauf beträgt:
\[
\Delta \phi = 2\pi \left(\frac{1}{\kappa} - 1\right).
\]
