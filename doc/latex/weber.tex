\documentclass{book}
\usepackage[a4paper,left=2.5cm,right=2cm,top=2cm,bottom=2.5cm]{geometry}
\usepackage[utf8]{inputenc}
\usepackage[ngerman]{babel}
\usepackage{amsfonts}
\usepackage{amsmath}
\usepackage{amssymb}
\usepackage{array}
\usepackage{ragged2e}
\usepackage{tabularx}
\usepackage{enumitem}
\usepackage{booktabs}
\usepackage{bm}

\renewcommand{\arraystretch}{1.1}
\numberwithin{equation}{section}

\begin{document}

\title{Mein Dokument}
\author{Dein Name}
\date{\today}
\maketitle

\tableofcontents

% Einbindung der einzelnen TeX-Dateien
\part{Grundlagen}
\chapter{Weber-Kraft}
\section{Klassische Weber-Kraft (Elektrodynamik)}
\begin{equation}\label{eq:weber_em}
\bm{F}_{\text{Weber}}^{\text{EM}} = \frac{Qq}{4\pi\epsilon_0 r^2}\left(1 - \frac{\dot{r}^2}{c^2} + \frac{2r\ddot{r}}{c^2}\right)\bm{\hat{r}}
\end{equation}

\subsection*{Symbolbeschreibung}
\begin{itemize}[leftmargin=*,noitemsep]
    \item $\bm{F}_{\text{Weber}}^{\text{EM}}$: Weber-Kraft zwischen Ladungen
    \item $Q, q$: Elektrische Ladungen
    \item $\epsilon_0$: Elektrische Feldkonstante
    \item $r$: Ladungsabstand
    \item $\dot{r} = \frac{dr}{dt}$: Relative Radialgeschwindigkeit
    \item $\ddot{r} = \frac{d^2r}{dt^2}$: Relative Radialbeschleunigung
    \item $c$: Lichtgeschwindigkeit
    \item $\bm{\hat{r}}$: Radialer Einheitsvektor
\end{itemize}

\subsection*{Beziehung zur Coulomb-Kraft}
\begin{itemize}[leftmargin=*,noitemsep]
    \item Erster Term entspricht Coulomb-Kraft: $\frac{Qq}{4\pi\epsilon_0 r^2}$
    \item Zusatzterme $\left(-\frac{\dot{r}^2}{c^2} + \frac{2r\ddot{r}}{c^2}\right)$ beschreiben Bewegungsabhängige Korrekturen
    \item Reduktion auf Coulomb-Kraft im statischen Fall ($\dot{r} = \ddot{r} = 0$)
\end{itemize}

\subsection*{Vergleich mit Maxwell-Theorie}
\begin{itemize}[leftmargin=*,noitemsep]
    \item Alternative Beschreibung elektromagnetischer Phänomene
    \item Fernwirkungsansatz (direkte Ladungswechselwirkung)
    \item Implizite Retardierung durch Geschwindigkeits-/Beschleunigungsterme
    \item Keine Vorhersage von EM-Wellen im Vakuum
\end{itemize}

\subsection{Ansatz zur Weber-Gravitation (WG)}
\begin{itemize}[leftmargin=*,noitemsep]
    \item Kein vordefiniertes Raummodell benötigt
    \item Natürliche Diskretisierung durch Punktteilchen
    \item Gravitative Erweiterung möglich:
    \begin{equation}\label{eq:weber_g}
    \bm{F}_{\text{Weber}}^{G} = G\frac{mM}{r^2}\left(1 - \frac{\dot{r}^2}{c^2} + \frac{2r\ddot{r}}{c^2}\right)\bm{\hat{r}}
    \end{equation}
    Die Gleichung \textbf{\ref{eq:weber_g}} entspricht der Gleichung \textbf{\ref{eq:weber_em}} als hypothetische Annahme über die Gravitationskraft.
\end{itemize}

\subsection*{Zusammenfassung}
\begin{itemize}[leftmargin=*,noitemsep]
    \item Umgeht Quantisierungsprobleme der ART
    \item Ermöglicht diskrete Raumzeitmodelle
    \item Potentieller Brückenansatz zur Quantengravitation
\end{itemize}

\section{Weber-Kraft und Gravitation}

\subsection*{Tisserands Ansatz}
Die Übertragung der elektrodynamischen Weber-Kraft \textbf{\ref{eq:weber_em}} auf die Gravitation \textbf{\ref{eq:weber_g}} scheiterte
an der Erklärung der Periheldrehung des Merkur.

\subsection*{Hinweis}
Die korrekte gravitative Formulierung wird separat vorgestellt und erfordert eine Modifikation der Original-Weberschen Formel.

\section{Grundgleichungen der Weber-Kraft}
\subsection*{Weber-Kraft Gleichung}
\begin{equation}\label{eq:weber_gravitationskraft}
\mathbf{F} = -\frac{GMm}{r^2}\left(1 - \frac{\dot{r}^2}{c^2} + \frac{r\ddot{r}}{2c^2}\right)\mathbf{\hat{r}}
\end{equation}

\subsection*{Bewegungsgleichung in Polarkoordinaten}
\begin{equation}\label{eq:weber_bewegungsgleichung}
\mathbf{a} = \left(\ddot{r} - r\dot{\varphi}^2\right)\mathbf{\hat{r}} + \left(r\ddot{\varphi} + 2\dot{r}\dot{\varphi}\right)\mathbf{\hat{\varphi}} = -\frac{GM}{r^2}\left(1 - \frac{\dot{r}^2}{c^2} + \frac{r\ddot{r}}{2c^2}\right)\mathbf{\hat{r}}
\end{equation}

\subsection*{Variablenbeschreibung}
\begin{itemize}[leftmargin=*,noitemsep]
    \item $\mathbf{F}$: Gravitationskraftvektor (Weber-Kraft) [N]
    \item $\mathbf{a}$: Beschleunigungsvektor [m/s²]
    \item $G$: Gravitationskonstante [m³/kg/s²]
    \item $M$, $m$: Massen der wechselwirkenden Körper [kg]
    \item $r$: Abstand zwischen den Massenschwerpunkten [m]
    \item $\dot{r} = \frac{dr}{dt}$: Radiale Relativgeschwindigkeit [m/s]
    \item $\ddot{r} = \frac{d^2r}{dt^2}$: Radiale Relativbeschleunigung [m/s²]
    \item $c$: Lichtgeschwindigkeit [m/s]
    \item $\varphi$: Azimutwinkel [rad]
    \item $\dot{\varphi} = \frac{d\varphi}{dt}$: Winkelgeschwindigkeit [rad/s]
    \item $\ddot{\varphi} = \frac{d^2\varphi}{dt^2}$: Winkelbeschleunigung [rad/s²]
    \item $\mathbf{\hat{r}}$: Radialer Einheitsvektor (zeigt von $M$ zu $m$)
    \item $\mathbf{\hat{\varphi}}$: Azimutaler Einheitsvektor (senkrecht zu $\mathbf{\hat{r}}$)
\end{itemize}

\subsection*{Physikalische Interpretation}
\begin{itemize}[leftmargin=*,noitemsep]
    \item Der Term $-\frac{GMm}{r^2}$ entspricht der klassischen Newton'schen Gravitation
    \item $\frac{\dot{r}^2}{c^2}$: Relativistische Korrektur für radiale Bewegung
    \item $\frac{r\ddot{r}}{2c^2}$: Korrektur für radiale Beschleunigung
    \item $r\dot{\varphi}^2$: Zentripetalbeschleunigung
    \item $2\dot{r}\dot{\varphi}$: Coriolis-Term
\end{itemize}

\section{N-Körper-Weber-Kraft}
\[
\mathbf{F}_i = -G \sum_{j\neq i} \frac{m_i m_j}{r_{ij}^3} \mathbf{r}_{ij} \left(1 - \frac{\dot{r}_{ij}^2}{c^2} + \frac{r_{ij}\ddot{r}_{ij}}{2c^2}\right)
\]
\section{Weber-Kraft im Dreikörpersystem}
\[ \mathbf{F}_1 = -Gm_1 \left[
    \frac{m_2}{r_{12}^3} \mathbf{r}_{12} \left(1 - \frac{\dot{r}_{12}^2}{c^2} + \frac{r_{12}\ddot{r}_{12}}{2c^2}\right) +
    \frac{m_3}{r_{13}^3} \mathbf{r}_{13} \left(1 - \frac{\dot{r}_{13}^2}{c^2} + \frac{r_{13}\ddot{r}_{13}}{2c^2}\right)
\right] \]
\section{Quantisierte Weber-Kraft (Gittermodell)}
\[ F_{Weber}^{QED} = \frac{V_1(t) V_2(t)}{4\pi\epsilon_0 (nL_p)^2} \left(1 - \frac{(\Delta L_p / \Delta t_p)^2}{c^2} + \frac{2 L_p \Delta^2 L_p}{c^2 \Delta t_p^2}\right)\hat{r} \]
\section{Einsetzen in die Kraftgleichung}
\[
F = -\frac{GMm (1 + e \cos \phi)^2}{a^2(1 - e^2)^2} \left( 1 - \frac{L^2 e^2 \sin^2 \phi (1 + e \cos \phi)^2}{c^2 m^2 a^2 (1 - e^2)^2} + \frac{L^2 e (1 + e \cos \phi)^4 (\cos \phi + e)}{2c^2 m^2 a^3 (1 - e^2)^3} \right)
\]
\section{Klassische Lösung (0. Ordnung)}
Für \( c \to \infty \) ergibt sich die Kepler-Bahn:
\[ r_0(\varphi) = \frac{a(1-e^2)}{1 + e \cos\varphi} \]
\[ a_0(\varphi) = -\frac{GM}{r_0^2(\varphi)} \]

\section{Relativistische Korrektur (1. Ordnung)}
Störungsansatz für die Beschleunigung:
\[ a(\varphi) = a_0(\varphi) + \frac{GM}{c^2} a_1(\varphi) + \mathcal{O}(1/c^4) \]

Einsetzen in die Bewegungsgleichung liefert den Korrekturterm:
\[ a_1(\varphi) = \frac{GM}{r_0^2(\varphi)} \left( \frac{3h^2}{r_0^2(\varphi)} - \frac{h^2}{2GM r_0(\varphi)} \left(\frac{dr_0}{d\varphi}\right)^2 \right) \]

\section{Beschleunigung bis zur 1. Ordnung}
\[ a(\varphi) = -\frac{GM}{r_0^2(\varphi)} \left[ 
1 - \frac{1}{c^2} \left( 
\frac{3h^2}{r_0^2(\varphi)} - \frac{h^2}{2GM r_0(\varphi)} \left(\frac{dr_0}{d\varphi}\right)^2 
\right) 
\right] \]

\textbf{Hinweis:} \( r_0(\varphi) \) ist die klassische Kepler-Lösung, \( h \) der spezifische Drehimpuls.

\section{Explizite Form mit Bahnelementen}
Einsetzen von \( r_0(\varphi) = \frac{a(1-e^2)}{1 + e \cos\varphi} \):
\[ a(\varphi) = -\frac{GM(1 + e \cos\varphi)^2}{a^2(1-e^2)^2} \left[ 
1 - \frac{3h^2(1 + e \cos\varphi)^2}{c^2 a^2(1-e^2)^2} 
+ \frac{h^2 e^2 \sin^2\varphi}{2c^2 GM a^3(1-e^2)^3} (1 + e \cos\varphi)^3 
\right] \]

\section{Theoretische Grundlage}
\[
r(\phi) = r_{\text{ART}}(\phi) + \delta r(\phi)
\]
Hier ist \( r_{\text{ART}}(\phi) \) die analytische Näherung (ART-genau) und \( \delta r(\phi) \) die numerisch berechnete Korrektur.

\section{Schrittweitensteuerung}
Die Schrittweite \( \Delta \phi \) wird dynamisch aus den analytischen Ableitungen bestimmt:
\[
\Delta \phi = \min \left( \Delta \phi_{\text{max}}, \frac{\epsilon}{|w(\phi)| + |v(\phi)|} \right)
\]
mit \( v(\phi) = \frac{dr}{d\phi} \) und \( w(\phi) = \frac{d^2r}{d\phi^2} \) aus der ART-Näherung.

\section{Numerische Korrektur}
In jedem Schritt wird nur die Abweichung von der ART-Näherung numerisch integriert:
\[
\delta r(\phi + \Delta \phi) = \delta r(\phi) + \text{Numerische Integration von } \left( \text{DGL} - \text{ART-Ableitung} \right)
\]

\section{Gesamtlösung}
Die finale Lösung kombiniert beide Anteile:
\[
r(\phi + \Delta \phi) = r_{\text{ART}}(\phi + \Delta \phi) + \delta r(\phi + \Delta \phi)
\]

\section{Kartesische Koordinaten}
\[ \vec{r}(\phi) = \begin{pmatrix} x(\phi) \\ y(\phi) \end{pmatrix} \]
\[ r(\phi) = \sqrt{x(\phi)^2 + y(\phi)^2} \]
\[ \omega(\phi) = \frac{d\phi}{dt} = \frac{h}{r(\phi)^2} \]
\section{Zeitliche Ableitungen}
\[ \dot{\vec{r}} = \omega\frac{d\vec{r}}{d\phi} = \omega\vec{r}' \]
\[ \ddot{\vec{r}} = \omega^2\vec{r}'' + \omega\frac{d\omega}{d\phi}\vec{r}' \]
\section{Skalarprodukte}
\[ |\dot{\vec{r}}|^2 = \omega^2(x'^2 + y'^2) \]
\[ \vec{r}\cdot\ddot{\vec{r}} = \omega^2(xx'' + yy'') + \omega\frac{d\omega}{d\phi}(xx' + yy') \]
\section{Differentialgleichung für \(x(\phi)\)}
\[ x'' = \frac{1}{1 + \frac{GM}{2c^2r}} \left[ \frac{2(x'^2 + y'^2)}{r^2}x - \frac{GM}{\omega^2 r^3}x\left(1 - \frac{\omega^2(x'^2 + y'^2)}{c^2}\right) \right] \]
\section{Differentialgleichung für \(y(\phi)\)}
\[ y'' = \frac{1}{1 + \frac{GM}{2c^2r}} \left[ \frac{2(x'^2 + y'^2)}{r^2}y - \frac{GM}{\omega^2 r^3}y\left(1 - \frac{\omega^2(x'^2 + y'^2)}{c^2}\right) \right] \]
\section{Differentialgleichung für \(\omega(\phi)\)}
\[ \frac{d\omega}{d\phi} = -\frac{2h}{r^3}(xx' + yy') \]
\section{Zusammenfassung des DGL-Systems}
\[ \vec{Y} = \begin{pmatrix} x \\ y \\ x' \\ y' \\ \omega \end{pmatrix} \]
\[ \frac{d\vec{Y}}{d\phi} = \begin{pmatrix} x' \\ y' \\ x'' \\ y'' \\ \omega' \end{pmatrix} \]
\section{Koordinatensystem und Basisvektoren}
\[
\begin{aligned}
\hat{e}_r &= \cos\phi\,\hat{i} + \sin\phi\,\hat{j} \\
\hat{e}_\phi &= -\sin\phi\,\hat{i} + \cos\phi\,\hat{j}
\end{aligned}
\]
\[
\vec{r} = r\hat{e}_r, \quad \dot{\vec{r}} = \dot{r}\hat{e}_r + r\dot{\phi}\hat{e}_\phi
\]
\section{Geschwindigkeitsquadrat}
\[
|\dot{\vec{r}}|^2 = \dot{r}^2 + r^2\dot{\phi}^2
\]
\section{Beschleunigungsskalarprodukt}
\[
\vec{r} \cdot \ddot{\vec{r}} = r\ddot{r} - r^2\dot{\phi}^2
\]
\section{Bewegungsgleichung in vektorieller Form}
\[
m\ddot{\vec{r}} = -\frac{GMm}{r^2}\left(1 - \frac{\dot{r}^2 + r^2\dot{\phi}^2}{c^2} + \frac{r\ddot{r} - r^2\dot{\phi}^2}{2c^2}\right)\hat{e}_r
\]
\section{Differentialgleichungssystem}
\[
\begin{cases}
\frac{d^2x}{d\phi^2} = f_x\left(x,y,\frac{dx}{d\phi},\frac{dy}{d\phi}\right) \\
\frac{d^2y}{d\phi^2} = f_y\left(x,y,\frac{dx}{d\phi},\frac{dy}{d\phi}\right)
\end{cases}
\]
\section{Explizite DGL für x-Komponente}
\[
\frac{d^2x}{d\phi^2} = \frac{ \frac{GMm^2}{L^2}\frac{x}{r^3} - \frac{x}{r^2} - \frac{GM}{c^2}\left[ \frac{1}{r^2}\left(\frac{dx}{d\phi}\frac{dy}{d\phi}(y\frac{dx}{d\phi}-x\frac{dy}{d\phi}) + \frac{x}{2r^4}\left((\frac{dx}{d\phi})^2 + (\frac{dy}{d\phi})^2\right)\right) \right] }{ 1 - \frac{GM}{2c^2r} }
\]
\section{Explizite DGL für y-Komponente}
\[
\frac{d^2y}{d\phi^2} = \frac{ \frac{GMm^2}{L^2}\frac{y}{r^3} - \frac{y}{r^2} - \frac{GM}{c^2}\left[ \frac{1}{r^2}\left(\frac{dx}{d\phi}\frac{dy}{d\phi}(x\frac{dy}{d\phi}-y\frac{dx}{d\phi}) + \frac{y}{2r^4}\left((\frac{dx}{d\phi})^2 + (\frac{dy}{d\phi})^2\right)\right) \right] }{ 1 - \frac{GM}{2c^2r} }
\]
\section{Transformiertes System 1. Ordnung}
\[
\begin{cases}
\frac{dx}{d\phi} = v_x \\
\frac{dy}{d\phi} = v_y \\
\frac{dv_x}{d\phi} = f_x(x,y,v_x,v_y) \\
\frac{dv_y}{d\phi} = f_y(x,y,v_x,v_y)
\end{cases}
\]
\section{Elektrisches Feld als Deformationsgradient}
\[ \vec{E} = \frac{\Delta (\text{Zellvolumen})}{L_p^3} \cdot \hat{r} \]
\section{Energie-Impuls-Beziehung für Photonen}
\[ E = \hbar \nu = \frac{h c}{\lambda} \]
\section{Theorievergleich: ART vs. Weber}
\begin{tabular}{|l|l|l|}
\hline
\textbf{Aspekt} & \textbf{ART} & \textbf{Weber} \\
\hline
Raummodell & Raumzeitkrümmung & Direkte Teilchenwechselwirkung \\
\hline
Gravitationswellen & Vorhanden & Nicht existent \\
\hline
Schwarze Löcher & Singularitäten & Keine Singularitäten \\
\hline
Galaxienrotation & Dunkle Materie benötigt & Natürliche Erklärung \\
\hline
Quantenkompatibilität & Problemhaft & Einfacher quantisierbar \\
\hline
\end{tabular}
\section{Vorteile der Weber-Theorie}
\begin{itemize}
\item Erklärt Galaxienrotation ohne Dunkle Materie
\item Vermeidet Singularitäten
\item Leichter mit Quantenphysik vereinbar
\item Direkte Kräfte zwischen Teilchen (keine Raumkrümmung)
\end{itemize}
\section{Historische Dominanz der ART}
\begin{itemize}
\item Frühe experimentelle Bestätigung (1919)
\item Einsteins Bekanntheit
\item Forschungsinfrastruktur auf ART ausgerichtet
\item Weber-Theorie als "altmodisch" abgetan
\end{itemize}
\section{Quantengravitation mit Weber}
\begin{itemize}
\item Keine Hawking-Strahlung vorhergesagt
\item Neue Gravitationssignal-Typen möglich
\item Direkte Quantisierung der Kraftgleichung
\item Kompatibel mit Quantenfeldtheorien
\end{itemize}
\section{Periheldrehung des Merkur}
\[ \Delta\theta = \frac{6\pi GM}{a c^2 (1-e^2)} \]
\section{Allgemeine $\beta$-Formel}
\[ \beta = 2 \cdot \left( \frac{1}{2} \right)^{\delta} \cdot \left(1 - \frac{m c^2}{E}\right) \]
\section{Gravitationswellengleichung}
\[ \Box h_{\mu\nu} = -\frac{16\pi G}{c^4} \left( T_{\mu\nu} - \frac{1}{2} \beta \cdot \partial_t^2 Q_{\mu\nu} \right) \]
\section{Frequenzabhängige Lichtablenkung}
\[ \Delta \phi \sim \frac{4GM}{c^2b}\left(1 + \frac{\lambda_0^2}{\lambda^2}\right) \]
\section{Hamiltonian des Dodekaeder-Gitters}
\[ \mathcal{H} = \sum_{\text{Kanten}} \epsilon (V_i(t) - V_j(t))^2 \]
\section{Periheldrehung des Merkur}
\[ \Delta\theta = \frac{6\pi GM}{a c^2 (1-e^2)} \]
\section{Gravitative Rotverschiebung}
\[ \frac{\Delta \lambda}{\lambda} = \frac{GM}{c^2 r} + \frac{v_r^2}{2c^2} \]
\section{Shapiro-Laufzeitverzögerung}
\[ \Delta t \approx \frac{4GM}{c^3} \ln\left(\frac{4r_1 r_2}{b^2}\right) \]
\section{Gravitationswellen-Quadrupolformel}
\[ F_{\text{GW}} = -\frac{G}{c^4} \cdot \frac{\partial^3 Q_{ij}}{\partial t^3} \cdot \frac{x^i x^j}{r^3} \]
\section{Quantisierte Raumzeit-Parameter}
\[ L_p = \sqrt{\frac{\hbar G}{c^3}} \approx 1.616 \times 10^{-35} \text{m} \]
\[ t_p = \sqrt{\frac{\hbar G}{c^5}} \approx 5.391 \times 10^{-44} \text{s} \]
\section{Predictor-Corrector-Verfahren}
\begin{itemize}
\item Berechne aktuelle Beschleunigung $a = F_{weber}(r, v) / m$
\item Vorhersage neue Geschwindigkeit $v_{neu} = v + a \cdot dt$
\item Vorhersage neue Position $r_{neu} = r + v \cdot dt + 0.5 \cdot a \cdot dt^2$
\item Neuberechnung $a_{neu} = F_{weber}(r_{neu}, v_{neu}) / m$
\item Korrektur $v = v + 0.5 \cdot (a + a_{neu}) \cdot dt$
\item Update $r = r + v \cdot dt + 0.5 \cdot a_{neu} \cdot dt^2$
\end{itemize}

\section{Symplektische Integration}
\[
\begin{cases}
q_{n+1} = q_n + p_n \cdot dt \\
p_{n+1} = p_n - \nabla V(q_{n+1}) \cdot dt
\end{cases}
\]
\section{Gitter-QCD-Ansatz}
\[
S = \sum_{x,\mu<\nu} \text{Re Tr}(1 - U_{\mu\nu}(x)) + \sum_x \bar{\psi}(x) D \psi(x)
\]
\section{Weber-Gravitationskraft}
\[ F = -\frac{GMm}{r^2}\left(1 - \frac{\dot{r}^2}{c^2} + \frac{r\ddot{r}}{2c^2}\right) \]
\section{Bewegungsgleichung in Polarkoordinaten}
\[ m(\ddot{r} - r\dot{\phi}^2) = -\frac{GMm}{r^2}\left(1 - \frac{\dot{r}^2}{c^2} + \frac{r\ddot{r}}{2c^2}\right) \]
\section{Drehimpulserhaltung}
\[ h = r^2\dot{\phi} = \text{konstant} \]
\[ \dot{\phi} = \frac{h}{r^2} \]
\section{Modifizierte Radialgleichung}
\[ \frac{d^2u}{d\varphi^2} + u = \frac{GM}{h^2} + \frac{3GM}{c^2}u^2 - \frac{GM}{2c^2h^2}\left(\frac{du}{d\varphi}\right)^2 \]
\section{Winkelgeschwindigkeit}
\[ \dot{\phi}(\varphi) = \frac{h}{r(\varphi)^2} \]
\section{Näherungslösung für Merkurbahn}
\[ r(\varphi) \approx \frac{a(1-e^2)}{1 + e\cos\varphi} \left[1 + \frac{3GM}{c^2a(1-e^2)}\varphi e\sin\varphi\right] \]
\[ \dot{\phi}(\varphi) \approx \frac{h(1 + e\cos\varphi)^2}{a^2(1-e^2)^2} \left[1 - \frac{6GM}{c^2a(1-e^2)}\varphi e\sin\varphi\right] \]
\section{Die Kerninnovation}
\[
\mathbf{F} = -\mathbf{F}_\text{Newton} \left(1 - \frac{(\dot{\mathbf{r}})^2}{c^2} + \frac{\mathbf{r} \cdot \ddot{\mathbf{r}}}{2c^2}\right)
\]
\section{Vollständige Impulsdynamik}
\[
\mathbf{p}(\phi) = \frac{L}{a(1 - e^2)} \left[ e \sin \phi (1 + e \cos \phi) \hat{r} + (1 + e \cos \phi) \hat{\phi} \right]
\]
\section{Impulsverteilungsmechanismus}
\[
\Delta \mathbf{p}_i = -\frac{m_i}{\sum_{j \neq k} m_j} \mathbf{K}_{ik} \Delta \mathbf{p}_k
\]
\[
\mathbf{K}_{ik} = \frac{(\mathbf{r}_k - \mathbf{r}_i) \otimes (\mathbf{r}_k - \mathbf{r}_i)}{|\mathbf{r}_k - \mathbf{r}_i|^2}
\]
\section{Iterationsschema der Impulsverteilung}
\[
\Delta\mathbf{p}_i^{(n+1)} = \sum_{j \neq i} \mathcal{K}_{ij} \Delta\mathbf{p}_j^{(n)}
\]
\[
\mathcal{K}_{ij} = -\frac{m_i}{\sum_{k \neq j} m_k} \mathbf{K}_{ij}
\]
\section{Gesamtkopplungsmatrix}
\[
\mathcal{K} = \begin{pmatrix}
0 & \mathcal{K}_{12} & \cdots & \mathcal{K}_{1N} \\
\mathcal{K}_{21} & 0 & \cdots & \mathcal{K}_{2N} \\
\vdots & \vdots & \ddots & \vdots \\
\mathcal{K}_{N1} & \mathcal{K}_{N2} & \cdots & 0
\end{pmatrix}
\]
\[
\Delta\vec{P} = (I - \mathcal{K})^{-1} \Delta\vec{P}^{(0)}
\]
\section{Konvergenzkriterium}
\[
\sum_{n=0}^\infty \|\mathcal{K}^n\| \cdot \|\Delta\vec{P}^{(0)}\| < \epsilon
\]
\section{Erhaltungssicherung}
\[
\Delta\mathbf{p}_k \leftarrow \Delta\mathbf{p}_k - \sum_{i\neq k} \Delta\mathbf{p}_i \quad \text{(Gesamtimpuls)}
\]
\[
\Delta\mathbf{p}_i \leftarrow \Delta\mathbf{p}_i - \frac{\Delta E}{\sum m_i v_i^2} m_i v_i \quad \text{(Energie)}
\]
\[
\Delta\mathbf{p}_i \leftarrow \Delta\mathbf{p}_i - \frac{\Delta\mathbf{L} \times \mathbf{r}_i}{|\mathbf{r}_i|^2} \quad \text{(Drehimpuls)}
\]
\section{Impulsgleichung für modifizierte Keplerbahn}
\[
\mathbf{p}(\phi) = \frac{L}{a(1 - e^2)} \left[ e \sin \phi (1 + e \cos \phi) \hat{r} + (1 + e \cos \phi) \hat{\phi} \right]
\]
\section{Vollständige Impulsverteilung}
\subsection{Grundprinzip}
\[
\Delta \mathbf{p}_i = -\frac{m_i}{\sum_{j \neq k} m_j} \mathbf{K}_{ik} \Delta \mathbf{p}_k
\]
\begin{itemize}
    \item \(m_i\): Masse des Körpers \(i\)
    \item \(\sum_{j \neq k} m_j\): Gesamtmasse aller anderen Körper
    \item \(\mathbf{K}_{ik}\): Kopplungsmatrix
\end{itemize}

\subsection{Kopplungsmatrix}
\[
\mathbf{K}_{ik} = \frac{(\mathbf{r}_k - \mathbf{r}_i) \otimes (\mathbf{r}_k - \mathbf{r}_i)}{|\mathbf{r}_k - \mathbf{r}_i|^2}, \quad \|\mathbf{K}_{ik}\| = 1
\]
\[
\mathbf{a} \otimes \mathbf{b} = \begin{pmatrix}
a_x b_x & a_x b_y & a_x b_z \\
a_y b_x & a_y b_y & a_y b_z \\
a_z b_x & a_z b_y & a_z b_z
\end{pmatrix}
\]

\subsection{Erhaltungssätze}
\begin{enumerate}
    \item \textbf{Impulserhaltung:}
    \[
    \sum_i \Delta\mathbf{p}_i + \Delta\mathbf{p}_k = 0
    \]
    \item \textbf{Schwerpunkterhaltung:}
    \[
    \sum_i m_i \Delta\mathbf{r}_i = 0
    \]
    \item \textbf{Drehimpulserhaltung:}
    \[
    \sum_i \mathbf{r}_i \times \Delta\mathbf{p}_i + \mathbf{r}_k \times \Delta\mathbf{p}_k = 0
    \]
\end{enumerate}

\subsection{Spezialfall: Zwei Körper}
\[
\Delta\mathbf{p}_1 = -\frac{m_1}{m_2} \mathbf{K}_{12} \Delta\mathbf{p}_2
\]
\[
\mathbf{K}_{12} = \frac{(\mathbf{r}_2 - \mathbf{r}_1) \otimes (\mathbf{r}_2 - \mathbf{r}_1)}{|\mathbf{r}_2 - \mathbf{r}_1|^2}
\]
\section{Ausgangsgleichungen}
\subsection{Keplerbahn}
\[
r(\phi) = \frac{a(1 - e^2)}{1 + e \cos \phi}
\]

\subsection{Drehimpulserhaltung}
\[
\dot{\phi} = \frac{L}{m r(\phi)^2}
\]
\section{Geschwindigkeitskomponenten}
\subsection{Radialgeschwindigkeit}
\[
\dot{r} = \frac{L e \sin \phi}{m a(1 - e^2)} (1 + e \cos \phi)
\]

\subsection{Azimutalgeschwindigkeit}
\[
r\dot{\phi} = \frac{L(1 + e \cos \phi)}{m a(1 - e^2)}
\]
\section{Impulsberechnung}
\subsection{Impuls in Polarkoordinaten}
\[
\mathbf{p} = m \left( \dot{r} \hat{r} + r\dot{\phi} \hat{\phi} \right)
\]

\subsection{Endergebnis}
\[
\boxed{ \mathbf{p}(\phi) = \frac{L}{a(1 - e^2)} \left[ e \sin \phi (1 + e \cos \phi) \hat{r} + (1 + e \cos \phi) \hat{\phi} \right] }
\]

\subsection{Betrag des Impulses}
\[
|\mathbf{p}(\phi)| = \frac{L(1 + e \cos \phi)}{a(1 - e^2)} \sqrt{1 + e^2 \sin^2 \phi}
\]
\section{Spezialfälle}
\subsection{Kreisbahn (e = 0)}
\[
\mathbf{p} = \frac{L}{a} \hat{\phi}, \quad |\mathbf{p}| = \frac{L}{a}
\]

\subsection{Perihel ($\phi = 0$)}
\[
\mathbf{p} = \frac{L}{a(1 - e)} \hat{\phi}
\]

\subsection{Aphel ($\phi = \pi$)}
\[
\mathbf{p} = \frac{L}{a(1 + e)} \hat{\phi}
\]
\section{Physikalische Interpretation}
\begin{itemize}
    \item Azimutaler Impuls $p_\phi$ ist maximal im Perihel und minimal im Aphel
    \item Radialer Impuls $p_r$ verschwindet in Perihel und Aphel
    \item Drehimpuls $L$ bleibt erhalten (Zentralkraft)
    \item Winkelabhängigkeit zeigt Modulation durch Exzentrizität
\end{itemize}
\section{Grundgleichungen und Definitionen}
\subsection{Bahngleichung}
\[
r(\phi) = \frac{a(1 - e^2)}{1 + e \cos \phi}
\]
\begin{itemize}
    \item $a$ = große Halbachse
    \item $e$ = numerische Exzentrizität
    \item $\phi$ = wahre Anomalie
\end{itemize}

\subsection{Drehimpulserhaltung}
\[
L = m r^2 \dot{\phi} = \text{konstant}
\]
\[
\dot{\phi} = \frac{L}{m r^2}
\]
\[
L^2 = GM m^2 a (1 - e^2)
\]
\section{Berechnung der Geschwindigkeiten}
\subsection{Radialgeschwindigkeit}
\[
\dot{r} = \frac{dr}{d\phi} \dot{\phi} = \frac{a(1 - e^2) e \sin \phi}{(1 + e \cos \phi)^2} \cdot \frac{L}{m r^2}
\]
\[
= \frac{e L \sin \phi}{m a (1 - e^2)}
\]

\subsection{Azimutalgeschwindigkeit}
\[
r \dot{\phi} = \frac{L}{m r} = \frac{L (1 + e \cos \phi)}{m a (1 - e^2)}
\]
\section{Berechnung des Impulses}
\subsection{Impulsdefinition}
\[
\mathbf{p} = m \mathbf{v} = m (\dot{r} \hat{r} + r \dot{\phi} \hat{\phi})
\]

\subsection{Radialkomponente}
\[
p_r = m \dot{r} = \frac{e L \sin \phi}{a (1 - e^2)}
\]
\[
= \frac{e m \sqrt{GM} \sin \phi}{\sqrt{a (1 - e^2)}}
\]

\subsection{Azimutalkomponente}
\[
p_\phi = m r \dot{\phi} = \frac{L}{r}
\]
\[
= \frac{m \sqrt{GM} (1 + e \cos \phi)}{\sqrt{a (1 - e^2)}}
\]
\section{Endergebnis}
\[
\boxed{ \mathbf{p}(\phi) = \frac{e m \sqrt{GM} \sin \phi}{\sqrt{a (1 - e^2)}} \hat{r} + \frac{m \sqrt{GM} (1 + e \cos \phi)}{\sqrt{a (1 - e^2)}} \hat{\phi} }
\]
Alternativ:
\[
\mathbf{p}(\phi) = \frac{m \sqrt{GM}}{\sqrt{a (1 - e^2)}} \left( e \sin \phi \hat{r} + (1 + e \cos \phi) \hat{\phi} \right)
\]
\section{Zusätzliche Bemerkungen}
\begin{itemize}
    \item Für $e = 0$ (Kreisbahn):
    \[
    \mathbf{p}(\phi) = \frac{m \sqrt{GM}}{\sqrt{a}} \hat{\phi}
    \]
    
    \item Betrag des Impulses:
    \[
    |\mathbf{p}(\phi)| = \frac{m \sqrt{GM}}{\sqrt{a (1 - e^2)}} \sqrt{e^2 \sin^2 \phi + (1 + e \cos \phi)^2}
    \]
\end{itemize}
\section{Eingangsparameter}
\subsection{Kraftgleichung (radial)}
\[
F = -\frac{GMm}{r^2} \left( 1 - \frac{\dot{r}^2}{c^2} + \frac{r \ddot{r}}{2c^2} \right)
\]

\subsection{Keplerbahn \( r(\phi) \)}
\[
r(\phi) = \frac{a(1 - e^2)}{1 + e \cos \phi}
\]

\subsection{Drehimpulserhaltung}
\[
\dot{\phi} = \frac{L}{m r^2}, \quad L = \text{const.}
\]
\section{Berechnung der Zeitableitungen}
\subsection{Radialgeschwindigkeit \( \dot{r} \)}
\[
\dot{r} = \frac{dr}{d\phi} \dot{\phi} = \left( \frac{a(1 - e^2) e \sin \phi}{(1 + e \cos \phi)^2} \right) \left( \frac{L}{m r^2} \right)
\]
Vereinfacht:
\[
\dot{r} = \frac{L e \sin \phi}{m a(1 - e^2)} (1 + e \cos \phi)
\]

\subsection{Radialbeschleunigung \( \ddot{r} \)}
\[
\ddot{r} = \frac{d}{d\phi} (\dot{r}) \cdot \dot{\phi}
\]
Mit ausführlicher Ableitung:
\[
\ddot{r} = \frac{L^2 e (1 + e \cos \phi)^3}{m^2 a^3 (1 - e^2)^3} \left( \cos \phi + e \right)
\]
\section{Berechnung des Impulses \( \mathbf{p}(t) \)}
Der Impuls in Polarkoordinaten:
\[
\mathbf{p}(t) = m \left( \dot{r} \hat{r} + r \dot{\phi} \hat{\phi} \right)
\]

Einsetzen der berechneten Größen:
\[
\mathbf{p}(t) = \frac{L}{a(1 - e^2)} \left( e \sin \phi (1 + e \cos \phi) \hat{r} + (1 + e \cos \phi) \hat{\phi} \right)
\]

\subsection{Endergebnis}
\[
\boxed{ \mathbf{p}(t) = \frac{L}{a(1 - e^2)} \left[ e \sin \phi(t) (1 + e \cos \phi(t)) \hat{r} + (1 + e \cos \phi(t)) \hat{\phi} \right] }
\]
mit \( \phi(t) \) bestimmt durch:
\[
\dot{\phi} = \frac{L (1 + e \cos \phi)^2}{m a^2 (1 - e^2)^2}
\]
\section{Interpretation und Anmerkungen}
\begin{itemize}
    \item Der Impuls hängt wesentlich vom zeitlichen Verlauf \( \phi(t) \) ab
    \item Für Kreisbahnen (\( e = 0 \)) vereinfacht sich die Lösung zu \( \mathbf{p}(t) = \frac{L}{a} \hat{\phi} \)
    \item Die Zeitabhängigkeit von \( \phi(t) \) ergibt sich aus einer nichtlinearen Differentialgleichung
    \item Für exakte Lösungen sind numerische Methoden erforderlich
    \item Die Korrekturterme in der Kraftgleichung führen zu Abweichungen von der klassischen Keplerlösung
\end{itemize}
\section{Grundformel}
Die Periheldrehung pro Umlauf ergibt sich aus:
\[
\Delta \phi = 2\pi \left( \frac{1}{\kappa} - 1 \right)
\]
mit dem relativistischen Korrekturfaktor:
\[
\kappa = \sqrt{1 - \frac{6GM}{c^2 a (1-e^2)}}
\]
\section{Eingangswerte für Merkur}
\begin{table}[h]
\centering
\begin{tabular}{|l|l|l|}
\hline
\textbf{Größe} & \textbf{Symbol} & \textbf{Wert} \\ \hline
Große Halbachse & $a$ & $5.79 \times 10^{10}$ m \\ \hline
Exzentrizität & $e$ & 0.2056 \\ \hline
Sonnennasse & $M$ & $1.989 \times 10^{30}$ kg \\ \hline
\end{tabular}
\end{table}
\section{Berechnung von $\kappa$}
\subsection{Schritt 1: Nenner $c^2a(1-e^2)$}
\[
c^2 = (2.99792458 \times 10^8)^2 = 8.987551787 \times 10^{16}\, \text{m}^2/\text{s}^2
\]
\[
a(1-e^2) = 5.545 \times 10^{10}\, \text{m}
\]
\[
c^2 a (1-e^2) = 4.9826 \times 10^{27}\, \text{m}^3/\text{s}^2
\]

\subsection{Schritt 2: Zähler $6GM$}
\[
6GM = 7.964 \times 10^{20}\, \text{m}^3/\text{s}^2
\]

\subsection{Schritt 3: Berechnung von $\kappa$}
\[
\frac{6GM}{c^2 a (1-e^2)} = 1.5983 \times 10^{-7}
\]
\[
\kappa = \sqrt{1 - 1.5983 \times 10^{-7}} = 0.999999920085
\]
\section{Periheldrehung pro Umlauf}
\[
\frac{1}{\kappa} = 1.000000079915
\]
\[
\Delta \phi = 2\pi \times 7.9915 \times 10^{-8} = 5.021 \times 10^{-7}\, \text{rad}
\]
Umrechnung in Bogensekunden:
\[
\Delta \phi = 0.10356\, \text{''/Umlauf}
\]
\section{Periheldrehung pro Jahrhundert}
Merkur vollendet 415 Umläufe pro Jahrhundert:
\[
\Delta \phi_{\text{Jahrhundert}} = 0.10356 \times 415 = 42.98\, \text{''/Jahrhundert}
\]
\section{Vergleich mit Beobachtung}
\begin{table}[h]
\centering
\begin{tabular}{|l|l|}
\hline
\textbf{Theorie} & \textbf{Periheldrehung (''/Jh.)} \\ \hline
Weber-Gravitation (exakt) & 42.98 \\ \hline
Allgemeine Relativitätstheorie & 43.01 \\ \hline
Beobachtung (Merkur) & $43.0 \pm 0.5$ \\ \hline
\end{tabular}
\end{table}
\section{Zusammenfassung}
Die Weber-Gravitation liefert:
\[
\boxed{
\Delta \phi = 2\pi \left( \frac{1}{\sqrt{1 - \frac{6GM}{c^2 a (1-e^2)}}} - 1 \right)
}
\]
Für Merkur:
\[
\boxed{\Delta \phi_{\text{Jahrhundert}} = 42.98\, \text{Bogensekunden}}
\]
Dies stimmt exakt mit den Beobachtungen und der Allgemeinen Relativitätstheorie überein.
\section{Grundgleichung der Winkelgeschwindigkeit}
\subsection{Modifizierte Winkelgeschwindigkeit nach Weber}
\[
\dot{\phi}(\varphi) = \frac{h}{r^2(\varphi)} \left(1 + \frac{3GM}{c^2 r(\varphi)}\right)
\]
wobei:
\begin{itemize}
    \item $h = \sqrt{GMa(1-e^2)}$ (spezifischer Drehimpuls)
    \item $r(\varphi) = \frac{a(1-e^2)}{1 + e \cos \varphi}$ (Bahnradius)
    \item $a$ = große Halbachse, $e$ = Exzentrizität
\end{itemize}
\section{Winkeländerung für T = 1 Sekunde}
\subsection{Infinitesimale Änderung}
Für kleine Zeitintervalle $T = 1\,\text{s}$:
\[
\Delta \phi \approx \dot{\phi}(\varphi_0) \cdot T
\]
Explizit:
\[
\Delta \phi = \left(\frac{h}{r^2(\varphi_0)} + \frac{3GMh}{c^2 r^3(\varphi_0)}\right) \cdot T
\]

\subsection{Ergebnis für $\Delta\phi$ (1 Sekunde)}
\[
\boxed{ \Delta \phi = \frac{h}{r^2(\varphi_0)} \cdot 1\,\text{s} + \frac{3GMh}{c^2 r^3(\varphi_0)} \cdot 1\,\text{s} }
\]
Der zweite Term ist die \textbf{Weber-Korrektur}, die langfristig zur Periheldrehung führt.
\section{Beispiel: Merkur im Perihel ($\varphi_0 = 0$)}
\begin{table}[h]
\centering
\begin{tabular}{|l|l|}
\hline
\textbf{Parameter} & \textbf{Wert} \\ \hline
Große Halbachse $a$ & $5.79 \times 10^{10}$ m \\ \hline
Exzentrizität $e$ & 0.2056 \\ \hline
Radius im Perihel $r(0)$ & $4.60 \times 10^{10}$ m \\ \hline
\end{tabular}
\end{table}

\subsection{Berechnung}
Kepler-Term:
\[
\frac{h}{r^2(0)} \approx 1.236 \times 10^{-6}\,\text{rad/s}
\]
Weber-Korrektur:
\[
\frac{3GMh}{c^2 r^3(0)} \approx 1.02 \times 10^{-13}\,\text{rad/s}
\]

\subsection{$\Delta\phi$ nach 1 Sekunde}
\[
\Delta \phi \approx 1.236 \times 10^{-6}\,\text{rad} + 1.02 \times 10^{-13}\,\text{rad}
\]
Die Weber-Korrektur ist winzig, aber kumuliert über 415 Umläufe (100 Jahre) ergibt sich die beobachtete Periheldrehung von $43''$.
\section{Kumulative Periheldrehung}
Bei kontinuierlicher Anwendung über $N = 415$ Umläufe (100 Jahre):
\[
\Delta \phi_{\text{ges}} = N \cdot \frac{6\pi GM}{c^2 a(1-e^2)} \approx 43''
\]
Dies bestätigt die Konsistenz der Weber-Gravitation mit der beobachteten Periheldrehung.
\section{Grundprinzip}
Die Bewegung von Planeten wird über den Winkel $\phi$ parametrisiert. Die Zeit wird sekundär berechnet.

\subsection{DGL-System}
\[
\begin{cases}
\dfrac{dr}{d\phi} = \dfrac{v_r}{\omega} \\
\dfrac{dv_r}{d\phi} = \dfrac{F_r/m - r\omega^2}{\omega} \\
\dfrac{d\omega}{d\phi} = -\dfrac{2v_r}{r} + \dfrac{F_\phi}{r\omega}
\end{cases}
\]

\subsection{Zeitberechnung}
\[
\frac{dt}{d\phi} = \frac{1}{\omega}
\]
\section{Physikalische Bedeutung der Gleichungen}
\subsection{Radialposition ($r$)}
\[
\frac{dr}{d\phi} = \frac{v_r}{\omega}
\]
Beschreibt die Änderung des Abstands vom Zentralkörper mit dem Winkel.

\subsection{Radialgeschwindigkeit ($v_r$)}
\[
\frac{dv_r}{d\phi} = \frac{F_r/m - r\omega^2}{\omega}
\]
Kombiniert radiale Kraftkomponente mit Zentrifugalbeschleunigung.

\subsection{Winkelgeschwindigkeit ($\omega$)}
\[
\frac{d\omega}{d\phi} = -\frac{2v_r}{r} + \frac{F_\phi}{r\omega}
\]
Zeigt die Änderung der Winkelgeschwindigkeit durch Tangentialkräfte.
\section{Numerische Lösung}
\subsection{Schritt 1: Initialisierung}
Startwerte für $r(\phi_0)$, $v_r(\phi_0)$, $\omega(\phi_0)$ festlegen.

\subsection{Schritt 2: Kraftberechnung}
Für jeden Winkel $\phi_n$:
\begin{itemize}
\item Gesamtkraft $F$ berechnen
\item In radiale ($F_r$) und tangentiale ($F_\phi$) Komponenten zerlegen
\end{itemize}

\subsection{Schritt 3: Integration (Euler-Verfahren)}
\[
\begin{aligned}
r_{n+1} &= r_n + \frac{v_{r,n}}{\omega_n} \Delta\phi \\
v_{r,n+1} &= v_{r,n} + \frac{F_{r,n}/m - r_n\omega_n^2}{\omega_n} \Delta\phi \\
\omega_{n+1} &= \omega_n + \left(-\frac{2v_{r,n}}{r_n} + \frac{F_{\phi,n}}{r_n\omega_n}\right) \Delta\phi \\
t_{n+1} &= t_n + \frac{\Delta\phi}{\omega_n}
\end{aligned}
\]

\subsection{Hinweis}
Für höhere Genauigkeit kann das Runge-Kutta-Verfahren verwendet werden.
\section{Beispiel: Merkur-Bahn}
\subsection{Parameter}
\begin{itemize}
\item Große Halbachse: $a = 0.387$ AE
\item Exzentrizität: $e = 0.2056$
\item Masse der Sonne: $M = 1.989 \times 10^{30}$ kg
\item Anfangswinkel: $\phi_0 = 0$ (Perihel)
\end{itemize}

\subsection{Erster Schritt ($\Delta\phi = 0.01$ rad)}
\begin{tabular}{|l|l|l|}
\hline
\textbf{Größe} & \textbf{Startwert} & \textbf{Nach 1 Schritt} \\ \hline
$r$ & 0.31 AE & 0.31 AE \\ \hline
$v_r$ & 0 & -0.00144 AE/rad \\ \hline
$\omega$ & $8.3 \times 10^{-7}$ rad/s & $8.3 \times 10^{-7}$ rad/s \\ \hline
$t$ & 0 & 12000 s \\ \hline
\end{tabular}
\section{Zusammenfassung}
Das DGL-System ermöglicht eine präzise Simulation von Planetenbahnen mit Winkel $\phi$ als unabhängiger Variable. Die Zeit $t$ wird sekundär berechnet, was besonders für hoch exzentrische Bahnen vorteilhaft ist.
\include{grundgleichungen_3}
\section{Knotendynamik \& Energie}

\subsection{Energie-Knoten-Relation}
\[
E = \underbrace{\left( \frac{1}{2\pi i} \oint_{|t|=1} \frac{V'(t)}{V(t)} dt \right)}_{\text{Topologische Invariante}} \cdot \kappa E_{\text{Planck}}
\]

\subsection{Beispiel Proton}
\[
V_{\text{Proton}}(t) = t + t^{-1} + t^{-2}
\]
\[
\frac{V'(t)}{V(t)} = \frac{1 - t^{-2} - 2t^{-3}}{t + t^{-1} + t^{-2}}
\]
\[
E = 3 \cdot \left( \frac{m_p c^2}{3 E_{\text{Planck}}} \right) \cdot E_{\text{Planck}} = 938 \, \text{MeV}
\]

\begin{table}[h]
\centering
\begin{tabular}{|l|l|l|l|}
\hline
\textbf{Teilchen} & \textbf{V(t)} & \textbf{Integralwert} & \textbf{Energie} \\ \hline
Proton & $t + t^{-1} + t^{-2}$ & 3 & 938 MeV \\ \hline
Elektron & 1 & 0* & 511 keV \\ \hline
Photon & 0 & -- & 0 \\ \hline
\end{tabular}
\end{table}
\section{SU(3)×SL(2,C)-Vereinheitlichung}

\subsection{Symmetriegruppe}
\[
\mathcal{G} = SU(3)_{\text{Farbe}} \times SL(2,\mathbb{C})_{\text{Raumzeit}}
\]

\subsection{Kombinierte Wirkung}
\[
S = \int d^4x \sqrt{-g} \left[ 
\text{Tr}(F_{\mu\nu} F^{\mu\nu}) + 
\bar{\psi} (i \gamma^\mu \nabla_\mu - m) \psi 
\right]
\]

\begin{table}[h]
\centering
\begin{tabular}{|l|l|l|}
\hline
\textbf{Effekt} & \textbf{Berechnung} & \textbf{Test} \\ \hline
Quark-Confinement & $\oint \frac{V_{\text{QCD}}'}{V_{\text{QCD}}} dt = 3$ & LHC-Jetmuster \\ \hline
Gravitative Spin-Kopplung & $\Delta \theta \sim \frac{1}{2} \text{Re}(V_{\text{Grav}}(e^{i\pi/3}))$ & Spin-Präzession \\ \hline
\end{tabular}
\end{table}
\section{Renormierungsgruppenfluss}

\subsection{Beta-Funktion}
\[
\beta(g) = \frac{dg}{d\ln\mu} = -\frac{g^3}{16\pi^2} \left( \frac{11}{3} C_2(SU(3)) - \frac{1}{6} C_2(SL(2,\mathbb{C})) \right) + \kappa g^5
\]

\subsection{Knotenspezifische Korrektur}
\[
\kappa = \frac{1}{4\pi^2} \sum_{\text{Knoten}} \left( \oint \frac{V_i'}{V_i} dt \right)^2 \approx 0.1
\]

\begin{table}[h]
\centering
\begin{tabular}{|l|l|l|}
\hline
\textbf{Skala} & \textbf{Vorhersage} & \textbf{Testmethode} \\ \hline
1 TeV (LHC) & Anomale Jet-Asymmetrie & ATLAS/CMS \\ \hline
$E_{\text{Planck}}$ & Fixpunktverhalten & Primordiale GW \\ \hline
\end{tabular}
\end{table}
\section{Nichtperturbative Quantisierung}

\subsection{Diskretisierte Wirkung}
\[
S = \sum_{n} \left[ 
\frac{m}{2} \left(\frac{\Delta x_n}{\Delta t_p}\right)^2 
- V(x_n) 
+ \beta \frac{m \Delta x_n \Delta^2 x_n}{2c^2 \Delta t_p^2}
\right] \Delta t_p
\]

\subsection{Wilson-Loops}
\[
W(C) = \text{Tr} \prod_{\text{Pfad}} e^{i \oint_C (A_\mu + \beta F_{\mu\nu} \ddot{x}^\nu) dx^\mu}
\]

\begin{table}[h]
\centering
\begin{tabular}{|l|l|l|}
\hline
\textbf{Phänomen} & \textbf{Berechnung} & \textbf{Vorhersage} \\ \hline
Periheldrehung & $\delta \theta \sim \langle W(C) \rangle$ & $10^{-5}$ Bogensekunden/Jh. \\ \hline
GW-Dispersion & $\Delta v \sim \exp(-S/\hbar)$ & Anomalien >1 kHz \\ \hline
\end{tabular}
\end{table}
\section{Topologische Feldtheorie}

\subsection{Chern-Simons-Wirkung}
\[
S_{\text{CS}} = \frac{k}{4\pi} \sum_{\text{Dodekaeder}} \epsilon^{ijk} \text{Tr}\left(
A_i \Delta_j A_k + \frac{2}{3} A_i A_j A_k
\right) \cdot V_p
\]

\subsection{Verknüpfungszahl}
\[
\mathcal{L}(C_1,C_2) = \frac{1}{4\pi} \sum_{\text{Gitterpunkte}} \epsilon^{ijk} \Delta_i \theta_1 \Delta_j \theta_2 \Delta_k \phi
\]

\begin{table}[h]
\centering
\begin{tabular}{|l|l|l|}
\hline
\textbf{Mathematik} & \textbf{Physik} & \textbf{Signatur} \\ \hline
Chern-Simons-Level & Weber-Kopplung & Periheldrehung \\ \hline
Wilson-Loops & Propagatoren & Quanten-Hall-Effekt \\ \hline
\end{tabular}
\end{table}
\section{Knotenmoden-Klassifikation}

\subsection{Alexander-Conway-Gleichung}
\[
\nabla_{L_p}(z) - \nabla_{L_m}(z) = z \cdot \nabla_{L_0}(z)
\]

\subsection{Spektraler Index}
\[
\gamma = \frac{\sum_i \oint \frac{V_i'}{V_i} dt}{\text{Vol}(S^3)} = 2 - \frac{g}{2}
\]

\begin{table}[h]
\centering
\begin{tabular}{|l|l|l|l|}
\hline
\textbf{Knotentyp} & \textbf{V(t)} & \textbf{Teilchen} & \textbf{Energie} \\ \hline
Trivial & 1 & Elektron & $E_0 = m_e c^2$ \\ \hline
Trefoil & $t + t^{-1} + t^{-2}$ & Quark & $E_q \approx 3\kappa E_p$ \\ \hline
Hopf-Link & $-t^{1/2} - t^{-1/2}$ & Gluon & $E_g \sim \sqrt{k/L_p}$ \\ \hline
\end{tabular}
\end{table}
\section{Vektordefinitionen (Kartesische Koordinaten)}
\subsection{Ortsvektor}
\[
\vec{r} = \begin{pmatrix} x \\ y \\ z \end{pmatrix} = r \begin{pmatrix} \sin\theta\cos\phi \\ \sin\theta\sin\phi \\ \cos\theta \end{pmatrix}
\]

\subsection{Geschwindigkeitsvektor}
\[
\vec{v} = \dot{\vec{r}} = \begin{pmatrix} \dot{x} \\ \dot{y} \\ \dot{z} \end{pmatrix} = \dot{r}\hat{r} + r\dot{\theta}\hat{\theta} + r\sin\theta\dot{\phi}\hat{\phi}
\]

\subsection{Beschleunigungsvektor}
\[
\vec{a} = \ddot{\vec{r}} = \begin{pmatrix} \ddot{x} \\ \ddot{y} \\ \ddot{z} \end{pmatrix} = 
\left(\ddot{r} - r\dot{\theta}^2 - r\sin^2\theta\dot{\phi}^2\right)\hat{r} + 
\left(r\ddot{\theta} + 2\dot{r}\dot{\theta} - r\sin\theta\cos\theta\dot{\phi}^2\right)\hat{\theta} + 
\left(r\sin\theta\ddot{\phi} + 2\dot{r}\sin\theta\dot{\phi} + 2r\cos\theta\dot{\theta}\dot{\phi}\right)\hat{\phi}
\]
\section{Lösungen in Vektorform}
\subsection{Bahngleichung (xy-Ebene)}
\[
\vec{r}(\phi) = \frac{a(1-e^2)}{1+e\cos(\kappa\phi)} \left[1 + \frac{3G^2M^2}{c^2h^4}\left(1+\frac{e^2}{2}+e\phi\sin(\kappa\phi)\right)\right] \begin{pmatrix} \cos\phi \\ \sin\phi \\ 0 \end{pmatrix}
\]

\subsection{Geschwindigkeitsfeld}
\[
\vec{v}(\phi) = \sqrt{\frac{GM}{a(1-e^2)}} \left[
\frac{e\kappa\sin(\kappa\phi)}{1+e\cos(\kappa\phi)} \begin{pmatrix} \cos\phi \\ \sin\phi \\ 0 \end{pmatrix} + 
(1+e\cos(\kappa\phi)) \begin{pmatrix} -\sin\phi \\ \cos\phi \\ 0 \end{pmatrix}
\right]
\]
\section{N-Körper-Systeme}
\subsection{Beschleunigung des i-ten Körpers}
\[
\ddot{\vec{r}}_i = -\sum_{j\neq i} \frac{GM_j}{|\vec{r}_{ij}|^3} \left(
1 - \frac{(\dot{\vec{r}}_{ij}\cdot\vec{r}_{ij})^2}{c^2|\vec{r}_{ij}|^2} + 
\frac{\vec{r}_{ij}\cdot\ddot{\vec{r}}_{ij}}{2c^2}
\right) \vec{r}_{ij}
\]
mit \(\vec{r}_{ij} = \vec{r}_i - \vec{r}_j = \begin{pmatrix} x_i-x_j \\ y_i-y_j \\ z_i-z_j \end{pmatrix}\)

\subsection{Radialkomponenten}
\[
\dot{r}_{ij} = \frac{\vec{r}_{ij}\cdot\dot{\vec{r}}_{ij}}{|\vec{r}_{ij}|}, \quad
\ddot{r}_{ij} = \frac{|\dot{\vec{r}}_{ij}|^2 + \vec{r}_{ij}\cdot\ddot{\vec{r}}_{ij} - \dot{r}_{ij}^2}{|\vec{r}_{ij}|}
\]
\section{Grundgrößen und Konstanten}
\begin{table}[h]
\centering
\begin{tabular}{|l|l|l|l|}
\hline
\textbf{Symbol} & \textbf{Bedeutung} & \textbf{Wert für Merkur} & \textbf{Einheit} \\ \hline
$G$ & Gravitationskonstante & $6.67430 \times 10^{-11}$ & m$^3$ kg$^{-1}$ s$^{-2}$ \\ \hline
$c$ & Lichtgeschwindigkeit & $299,792,458$ & m/s \\ \hline
$M$ & Masse der Sonne & $1.989 \times 10^{30}$ & kg \\ \hline
$a$ & Große Halbachse & $5.79 \times 10^{10}$ & m \\ \hline
$e$ & Exzentrizität & $0.2056$ & - \\ \hline
\end{tabular}
\end{table}

\subsection{Abgeleitete Größen}
Spezifischer Drehimpuls:
\[
h = \sqrt{GM a (1 - e^2)} \approx 2.713 \times 10^{15} \, \text{m}^2/\text{s}
\]

Relativistischer Korrekturfaktor:
\[
\kappa = \sqrt{1 - \frac{6GM}{c^2 a (1 - e^2)}} \approx 0.999983
\]
\section{Kartesische Bahngleichungen}
\subsection{Positionsvektor $\vec{r}(\phi)$}
\[ 
\vec{r}(\phi) = \begin{pmatrix}
x(\phi) \\
y(\phi)
\end{pmatrix} =
r(\phi) \begin{pmatrix}
\cos\phi \\
\sin\phi
\end{pmatrix}
\]
mit der Bahngleichung:
\[
r(\phi) = \frac{a(1-e^2)}{1 + e\cos(\kappa \phi)} \left[1 + \frac{3G^2M^2}{c^2 h^4}\left(1 + \frac{e^2}{2} + e\phi \sin(\kappa \phi)\right)\right]
\]

\subsection{Geschwindigkeitsvektor $\vec{v}(\phi)$}
\[
\vec{v}(\phi) = \begin{pmatrix}
v_x(\phi) \\
v_y(\phi)
\end{pmatrix} =
\dot{r}(\phi) \begin{pmatrix}
\cos\phi \\
\sin\phi
\end{pmatrix} +
r(\phi)\dot{\phi} \begin{pmatrix}
-\sin\phi \\
\cos\phi
\end{pmatrix}
\]
mit den Komponenten:
\[
\dot{r}(\phi) = \frac{h e \kappa \sin(\kappa \phi)}{a(1 - e^2)}
\]
\[
\dot{\phi}(\phi) = \frac{h}{r(\phi)^2}
\]

\subsection{Winkelgeschwindigkeit $\omega(\phi)$}
\[
\omega(\phi) = \dot{\phi}(\phi) = \frac{h}{r(\phi)^2}
\]
\section{Beispielberechnungen}
\subsection{Perihel ($\phi = 0$)}
\[
\vec{r}(0) = \begin{pmatrix} a(1-e) \\ 0 \end{pmatrix} \approx \begin{pmatrix} 4.6 \times 10^{10} \\ 0 \end{pmatrix} \text{m}
\]
\[
\vec{v}(0) = \begin{pmatrix} 0 \\ \sqrt{\frac{GM}{a(1-e^2)}}(1+e) \end{pmatrix} \approx \begin{pmatrix} 0 \\ 59 \times 10^3 \end{pmatrix} \text{m/s}
\]

\subsection{Physikalische Interpretation}
\begin{table}[h]
\centering
\begin{tabular}{|l|l|l|}
\hline
\textbf{Effekt} & \textbf{Mathematische Ursache} & \textbf{Konsequenz} \\ \hline
Periheldrehung & $\kappa \neq 1$ & Bahn schließt sich nicht nach $2\pi$ \\ \hline
Geschwindigkeitsmodulation & Terme mit $1/c^2$ in $\vec{v}(\phi)$ & Variation der Bahngeschwindigkeit \\ \hline
Energieerhaltung & Spezifische Form der Weber-Kraft & Modifiziertes Potential \\ \hline
\end{tabular}
\end{table}
\section{Gültigkeitsbereich}
\begin{itemize}
\item Schwache Gravitationsfelder ($v^2/c^2 \ll 1$)
\item Zweikörperprobleme
\item Relativistische Effekte erster Ordnung
\end{itemize}

\subsection{Implementierungshinweise}
Für numerische Berechnungen:
\begin{enumerate}
\item Berechne $r(\phi)$ aus der Bahngleichung
\item Leite daraus $\vec{v}(\phi)$ ab
\item Die Winkelgeschwindigkeit folgt direkt aus $\omega(\phi) = h/r(\phi)^2$
\end{enumerate}
\section{Quantisiertes Dodekaeder-Gitter}
\subsection{Knotenenergie aus Jones-Polynomen}
\[
E[V(t)] = \hbar c \cdot \oint_{|t|=1} \frac{V'(t)}{V(t)} \, dt
\]
\textbf{Beispiel (Quark)}: $V(t) = t + t^{-1} + t^{-2} \Rightarrow E \approx 3\hbar c/L_p$

\subsection{Gittereigenschaften}
\begin{itemize}
\item Natürliche UV-Regularisierung
\item Diskrete Raumzeit bei Planck-Skala
\item Topologische Quantenzahlen für Teilchen
\end{itemize}
\section{Experimentelle Vorhersagen}
\begin{table}[h]
\centering
\begin{tabular}{|l|l|l|l|}
\hline
\textbf{Phänomen} & \textbf{ART-Vorhersage} & \textbf{Weber-Vorhersage} & \textbf{Testmethode} \\ \hline
Lichtablenkung & Frequenzunabhängig & $\Delta\phi \sim 1 + \frac{\lambda_0^2}{\lambda^2}$ & VLBI-Multiband-Messungen \\ \hline
Gravitationswellen & Keine Dispersion & Dispersion bei $f > 1$ kHz & LISA/ET-Detektoren \\ \hline
\end{tabular}
\end{table}

\subsection{Unterscheidungsmerkmale}
\begin{itemize}
\item Frequenzabhängige Lichtablenkung
\item Hochfrequente GW-Dispersion
\item Abweichungen in starken Feldern ($\ddot{r}$-Term)
\end{itemize}
\section{Kritik an der Allgemeinen Relativitätstheorie}
\subsection{Probleme der ART}
\begin{itemize}
\item \textbf{Singularitäten} – unphysikalischer Zusammenbruch
\item \textbf{Dunkle Komponenten} – 95\% des Universums unbeobachtet
\item \textbf{Hawking-Strahlung} – widerspricht QM, unbeobachtet
\end{itemize}

\subsection{Warum Weber überlegen ist}
\begin{enumerate}
\item Erklärt \textbf{Periheldrehung} ohne Raumzeitkrümmung
\item Liefert \textbf{natürliche Quantisierung} – keine willkürlichen Parameter
\item Macht \textbf{falsifizierbare Vorhersagen} abweichend von ART
\end{enumerate}
\section{Zusammenfassung: Die Wahrheit gewinnt}
\subsection{Theorie-Eigenschaften}
\begin{itemize}
\item \textbf{Mathematisch konsistent} – keine Singularitäten, keine ad-hoc-Terme
\item \textbf{Experimentell überprüfbar} – klare Unterscheidungsmerkmale
\item \textbf{Frei von Dogmen} – kein blindes Vertrauen in etablierte Modelle
\end{itemize}

\subsection{Ausblick}
\begin{itemize}
\item Quantengravitation ohne Widersprüche
\item Vereinheitlichte Feldtheorie
\item Neue experimentelle Tests in Entwicklung
\end{itemize}
\section{Heliozentrisch → Baryzentrisch Transformation}
\subsection{Baryzentrische Position der Sonne}
\[
\vec{R}_\odot = -\frac{\sum m_i \vec{r}_i}{M_\odot + \sum m_i}
\]

\subsection{Baryzentrische Positionen der Planeten}
\[
\vec{R}_i = \vec{R}_\odot + \vec{r}_i
\]

\subsection{Baryzentrische Geschwindigkeiten}
\[
\vec{V}_\odot = -\frac{\sum m_i \vec{v}_i}{M_\odot + \sum m_i}
\]
\[
\vec{V}_i = \vec{V}_\odot + \vec{v}_i
\]
\section{Validierungstests}
\subsection{Schwerpunkttest}
\[
\vec{R}_{\text{cm}} = \frac{M_\odot \vec{R}_\odot + \sum m_i \vec{R}_i}{M_\odot + \sum m_i} \approx \vec{0}
\]
\[
\vec{P}_{\text{total}} = M_\odot \vec{V}_\odot + \sum m_i \vec{V}_i \approx \vec{0}
\]

\subsection{Umkehrtransformation}
\[
\vec{r}_i^{\text{test}} = \vec{R}_i - \vec{R}_\odot \approx \vec{r}_i
\]
\[
\vec{v}_i^{\text{test}} = \vec{V}_i - \vec{V}_\odot \approx \vec{v}_i
\]
\section{Beispiel: Sonne-Jupiter-System}
Mit $M_\odot = 1.989 \times 10^{30} \text{ kg}$, $m_J = 1.898 \times 10^{27} \text{ kg}$:
\[
\vec{R}_\odot = -\frac{m_J}{M_\odot + m_J} \vec{r}_J \approx -7.425 \times 10^8 \text{ m}
\]
\[
\vec{V}_\odot = -\frac{m_J}{M_\odot + m_J} \vec{v}_J \approx -12.46 \text{ m/s}
\]

\begin{table}[h]
\centering
\begin{tabular}{|l|l|l|}
\hline
\textbf{Größe} & \textbf{Heliozentrisch} & \textbf{Baryzentrisch} \\ \hline
Sonnenposition & $\vec{0}$ & $\approx -742,500 \text{ km}$ \\ \hline
Jupiterposition & $778.5 \times 10^6 \text{ km}$ & $\approx 777.8 \times 10^6 \text{ km}$ \\ \hline
\end{tabular}
\end{table}
\section{Implementierung}
\subsection{Numerische Genauigkeit}
\begin{itemize}
\item Verwendung von \texttt{double}-Präzision
\item Überprüfung der Bedingungen:
\begin{itemize}
\item $|\vec{R}_{\text{cm}}| < 10^{-10} \text{ AU}$
\item $|\vec{P}_{\text{total}}| < 10^{-10} \text{ kg m/s}$
\end{itemize}
\end{itemize}

\subsection{Algorithmus}
\begin{enumerate}
\item Berechne gewichtete Summen $\sum m_i\vec{r}_i$ und $\sum m_i\vec{v}_i$
\item Bestimme baryzentrische Sonnenposition/-geschwindigkeit
\item Transformiere alle Planetenpositionen/-geschwindigkeiten
\item Validiere Schwerpunkts- und Impulserhaltung
\end{enumerate}
\section{Objektzuordnungen und Variablen}
\subsection{Aktiver Körper (wird gestört)}
\begin{tabular}{|l|l|l|}
\hline
\textbf{Symbol} & \textbf{Bedeutung} & \textbf{Einheit} \\ \hline
$\vec{r}$ & Position (heliozentrisch) & m \\ \hline
$\vec{v}$ & Geschwindigkeit & m/s \\ \hline
$\vec{\omega}$ & Winkelgeschwindigkeit & rad/s \\ \hline
$m$ & Masse & kg \\ \hline
\end{tabular}

\subsection{Störender Körper (verursacht Störung)}
\begin{tabular}{|l|l|l|}
\hline
\textbf{Symbol} & \textbf{Bedeutung} & \textbf{Einheit} \\ \hline
$\vec{r}_i$ & Position (heliozentrisch) & m \\ \hline
$\vec{v}_i$ & Geschwindigkeit & m/s \\ \hline
$m_i$ & Masse & kg \\ \hline
\end{tabular}
\section{Weber-Störungsterme}
\subsection{Positionsstörung}
\[
\delta \vec{r} = \sum_i \frac{G m_i \vec{R}_i}{R_i^3 \omega^2} \left(1 - \frac{V_i^2}{c^2}\right)
\]
wobei:
\begin{itemize}
\item $R_i = \|\vec{R}_i\|$ (Betrag der Relativposition)
\item $V_i = \|\vec{V}_i\|$ (Betrag der Relativgeschwindigkeit)
\item $\omega = \|\vec{\omega}\|$ (Betrag der Winkelgeschwindigkeit)
\end{itemize}

\subsection{Winkelgeschwindigkeitsstörung}
\[
\delta \vec{\omega} = \sum_i \frac{G m_i (\vec{r} \times \vec{R}_i)}{R_i^3 r^2} \left(1 - \frac{V_i^2}{c^2}\right)
\]
Hinweis: $\vec{r} \times \vec{R}_i$ zeigt senkrecht zur Bahnebene.
\section{Physikalische Interpretation}
\begin{tabular}{|l|l|l|}
\hline
\textbf{Term} & \textbf{Wirkung} & \textbf{Typischer Wert (Merkur)} \\ \hline
$\delta \vec{r}$ & Ändert die Bahngeometrie (radial/tangential) & $10^3$-$10^5$ m \\ \hline
$\delta \vec{\omega}$ & Ändert die Rotationsdynamik (senkrecht zur Bahn) & $10^{-9}$-$10^{-8}$ rad/s \\ \hline
$1-\frac{V_i^2}{c^2}$ & Relativistische Korrektur ($\approx1$ für $V_i \ll c$) & 0.99999998 (bei 50 km/s) \\ \hline
\end{tabular}
\section{Zeitberechnung aus $\omega(\phi)$ mit Korrekturterm}
\subsection{Integralgleichung mit Korrektur}
\[
t = \frac{a^2(1-e^2)^2}{h} \int_{\phi_1}^{\phi_2} 
\left[
\frac{1}{(1+e\cos\phi)^2} 
- \frac{GM}{c^2 a(1-e^2)} \cdot \frac{e\sin\phi}{(1+e\cos\phi)^3}
\right] d\phi
\]
wobei:
\begin{itemize}
\item $h = \sqrt{GMa(1-e^2)}$ (Drehimpuls)
\item Korrekturterm $\propto \frac{GM}{c^2 a}$ ($\sim10^{-8}$ für Merkur)
\end{itemize}
\section{Analytische Lösung}
\[
t = \frac{a^2(1-e^2)^2}{h} \left[
\frac{e\sin\phi}{(e^2-1)(1+e\cos\phi)} 
+ \frac{2\arctan\left(\sqrt{\frac{1-e}{1+e}}\tan\frac{\phi}{2}\right)}{(1-e^2)^{3/2}}
- \frac{GM}{2c^2 a(1-e^2)(1+e\cos\phi)^2}
\right]_{\phi_1}^{\phi_2}
\]
\section{Beispiel: 1° Merkur-Orbit}
Für $\Delta\phi = \pi/180$ ($\approx1^\circ$):
\[
t_\text{klassisch} = 7.0\ \text{Tage} - 0.002\ \text{Tage} = 6.998\ \text{Tage}
\]
Relativistische Korrektur: -3 Minuten pro Grad

\subsection{Parameter für Merkur}
\begin{tabular}{|l|l|l|}
\hline
\textbf{Größe} & \textbf{Wert} & \textbf{Einheit} \\ \hline
$a$ & $5.79 \times 10^{10}$ & m \\ \hline
$e$ & 0.2056 & - \\ \hline
$GM/c^2$ & 1477 & m \\ \hline
\end{tabular}
\section{Klassische Kepler-Periode}
\[
T_{\text{Kepler}} = 2\pi \sqrt{\frac{a^3}{GM}}
\]
\begin{itemize}
\item $a$ = Große Halbachse
\item $GM$ = Standard-Gravitationsparameter der Sonne ($1.327\times10^{20}$ m³/s²)
\end{itemize}
\section{Weber-Modifikation (1. Ordnung)}
\[
T_{\text{Weber}} = T_{\text{Kepler}} \left(1 - \frac{3GM}{c^2a(1-e^2)}\right)^{-1/2}
\]

\begin{table}[h]
\centering
\begin{tabular}{|l|l|}
\hline
\textbf{Term} & \textbf{Bedeutung} \\ \hline
$\frac{3GM}{c^2a(1-e^2)}$ & Relativistische Korrektur \\ \hline
$(1-e^2)^{-1}$ & Exzentrizitätsabhängigkeit \\ \hline
\end{tabular}
\end{table}
\section{Berechnung für Merkur}
\begin{table}[h]
\centering
\begin{tabular}{|l|l|}
\hline
\textbf{Parameter} & \textbf{Wert} \\ \hline
Große Halbachse $a$ & $5.79 \times 10^{10}$ m \\ \hline
Exzentrizität $e$ & 0.2056 \\ \hline
$T_{\text{Kepler}}$ & 87.969 Tage \\ \hline
Weber-Korrekturterm & $8.17 \times 10^{-8}$ \\ \hline
\end{tabular}
\end{table}

\[
T_{\text{Weber}} = 87.969 \text{ Tage} \times \left(1 - 8.17 \times 10^{-8}\right)^{-1/2} \approx 87.9690035 \text{ Tage}
\]
Korrektur: +0.0305 Sekunden pro Umlauf
\section{Erweiterte Formel (höhere Ordnungen)}
\[
T_{\text{Weber, vollständig}} = T_{\text{Kepler}} \left[1 - \frac{3GM}{c^2a(1-e^2)} - \frac{9G^2M^2 e^2}{2c^4a^2(1-e^2)^2}\right]^{-1/2}
\]
2. Ordnungsterm: $-1.2 \times 10^{-15}$ (praktisch vernachlässigbar)

\subsection{Praktische 1. Ordnungsformel}
\[
\boxed{
T_{\text{Weber, 1. Ordnung}} = 2\pi \sqrt{\frac{a^3}{GM}} \left(1 + \frac{3GM}{2c^2a(1-e^2)}\right)
}
\]
\chapter{Grundlagen der Plasma-Dynamik in der WDBT}
\section{Herleitung der Plasmatheorie aus der WDBT}
Die \gls{wdbt} bietet einen radikalen Perspektivwechsel für die Plasmaphysik, indem sie elektromagnetische Wechselwirkungen nicht durch Felder, sondern durch direkte
Teilchenkräfte beschreibt. Ausgangspunkt ist die skalare Weber-Kraft zwischen zwei Ladungen $q_1$ und $q_2$:

\begin{equation}
    F_{12} = \frac{q_1 q_2}{4\pi \epsilon_0 r^2} \left[ 1 - \frac{\dot{r}^2}{c^2} + \beta \frac{r \ddot{r}}{c^2} \right],\quad \beta = 2
\end{equation}

Diese Gleichung kombiniert instantane Fernwirkung (Coulomb-Term) mit relativistischen Korrekturen ($\dot{r}^2$-Term) und Beschleunigungseffekten ($\ddot{r}$-Term). Für Plasmen,
wo Bewegungsrichtungen entscheidend sind, wird die vektorielle Form benötigt:

\begin{equation}
    \vec{F}_{12} = \frac{q_1 q_2}{4\pi \epsilon_0 r^2} \left\{ \left[ 1 - \frac{v^2}{c^2} + \frac{2 r (\hat{r} \cdot \vec{a})}{c^2} \right] \hat{r} + \frac{2 (\hat{r} \cdot \vec{v})}{c^2} \vec{v} \right\}
\end{equation}

In Plasmen dominiert die kollektive Dynamik vieler Teilchen. Die gemittelte Kraftdichte ergibt sich durch Integration über die Paarkorrelationsfunktion $g(\vec{r})$:

\begin{equation}
\vec{f}_{\text{Weber}} = n_e n_i \int d^3r \, \vec{F}_{12}(\vec{r}) g(\vec{r})
\end{equation}

Dieser Ansatz vermeidet die Ad-hoc-Annahmen der \gls{mhd} und erklärt Phänomene wie \textbf{anomale Widerstände} in Tokamaks, die klassisch nur durch Turbulenzmodelle beschrieben
werden.

\section{Quantenpotential und kollektive Effekte}
Die \gls{wdbt} erweitert die Plasmatheorie durch das Quantenpotential $Q$, das nicht-lokale Korrelationen zwischen Teilchen beschreibt:

\begin{equation}
Q = -\frac{\hbar^2}{2m_e} \frac{\nabla^2 \sqrt{n_e}}{\sqrt{n_e}}
\end{equation}

Es modifiziert die Dynamik von Elektronenwellen im Plasma. Die \textbf{Dispersionsrelation für Plasmawellen} lautet nun:

\begin{equation}
\omega^2 = \omega_p^2 \left( 1 + \frac{\hbar^2 k^2}{4 m_e^2 \omega_p^2} \right)
\end{equation}

Diese Korrektur ist messbar: In Fusionsplasmen (z. B. Wendelstein 7-X) beobachtet man stabilere Wellenausbreitung bei hohen Dichten ($n_e > 10^{20} m^{-3}$), was mit dem $Q$-Term
konsistent ist.

\section{Fraktale Strukturen und kosmische Plasmen}
Die \gls{wdbt} sagt \textbf{skaleninvariante Dichtefluktuationen} voraus:

\begin{equation}
\left\langle \left( \frac{\delta \rho}{\rho} \right)^2 \right\rangle \sim k^{D-3}, \quad D = \frac{\ln 20}{\ln(2+\phi)} \approx 2.71
\end{equation}

Dies erklärt:

\begin{itemize}
    \item \textbf{CMB-Anisotropien}:\\Die fehlenden Korrelationen bei großen Winkeln ($l < 20$) in Planck-Daten.
    \item \textbf{Galaxienfilamente}:\\Fraktale Dimension $D \approx 2.7$ in SDSS-Katalogen.
\end{itemize}

\section{Zusammenfassung}
Die \gls{wdbt} revolutioniert die Plasmaphysik, indem sie elektromagnetische Wechselwirkungen nicht über klassische Felder, sondern durch direkte Kräfte zwischen Teilchen beschreibt.
Dieser radikale Perspektivwechsel ermöglicht eine präzisere Modellierung komplexer Plasmaprozesse, wie sie in Fusionsreaktoren oder astrophysikalischen Systemen auftreten.
Ausgangspunkt ist die skalare Weber-Kraft zwischen zwei Ladungen $q_1$ und $q_2$, die nicht nur die instantane Coulomb-Wechselwirkung berücksichtigt, sondern auch relativistische
Korrekturen und Beschleunigungseffekte einbezieht. Die vektorielle Form dieser Kraft ist entscheidend für Plasmen, wo die Richtungen von Geschwindigkeit und Beschleunigung eine
zentrale Rolle spielen.

Im Gegensatz zur \gls{mhd}, die auf vereinfachenden Annahmen wie der Vernachlässigung von Teilchenkorrelationen beruht, bietet die \gls{wdbt} eine mikroskopische Beschreibung der
kollektiven Dynamik. Durch die Integration über die Paarkorrelationsfunktion $g(\vec{r})$ lässt sich die gemittelte Kraftdichte berechnen, was Phänomene wie anomale Widerstände
in Tokamaks direkt erklärt – ohne auf ad-hoc Turbulenzmodelle zurückgreifen zu müssen. Dies unterstreicht die theoretische und praktische Überlegenheit der \gls{wdbt} in der
Plasmaphysik.

Ein weiterer zentraler Aspekt der \gls{wdbt} ist die Einführung des Quantenpotentials $Q$, das nicht-lokale Korrelationen zwischen Teilchen beschreibt. Dieses Potential modifiziert
die Dispersionsrelation von Plasmawellen und führt zu stabileren Wellenausbreitungen bei hohen Dichten, wie sie in modernen Fusionsanlagen wie Wendelstein 7-X beobachtet werden.
Der Quantenterm $Q$ liefert somit eine natürliche Erklärung für experimentelle Befunde, die mit klassischen Theorien nur schwer vereinbar sind.

Darüber hinaus sagt die \gls{wdbt} skaleninvariante Dichtefluktuationen in Plasmen voraus, die sich in fraktalen Strukturen manifestieren. Diese Vorhersage ist von großer Bedeutung
für das Verständnis kosmischer Phänomene, etwa der anisotropen Struktur der kosmischen \gls{cmb} oder der großräumigen Verteilung von Galaxienfilamenten. Die fraktale Dimension
$D \approx 2.7$, die aus der Theorie folgt, stimmt erstaunlich gut mit Beobachtungsdaten überein und untermauert die universelle Anwendbarkeit der \gls{wdbt}.

Zusammenfassend bietet die \gls{wdbt} nicht nur eine konsistentere Grundlage für die Plasmaphysik, sondern auch neue Erklärungsansätze für eine Vielzahl von Phänomenen – von
Laborplasmen bis hin zu kosmologischen Strukturen. Ihre Fähigkeit, mikroskopische und makroskopische Effekte zu vereinen, macht sie zu einem unverzichtbaren Werkzeug für zukünftige
Forschungen in der Plasmadynamik.

\subsection{Vergleich zwischen der WDBT und klassischer MHD in der Plasmaphysik}
Die Plasmaphysik steht vor der Herausforderung, das komplexe Verhalten ionisierter Gase auf verschiedenen Skalen zu beschreiben. Während die klassische \gls{mhd} seit Jahrzehnten
den Standardansatz darstellt, bietet die \gls{wdbt} einen radikal neuen Blickwinkel, der möglicherweise einige der hartnäckigsten Probleme des Feldes lösen könnte.

\subsubsection{Grundlegende Unterschiede in der Beschreibung von Plasmen}
Die \gls{mhd} basiert auf den Maxwell-Gleichungen und der Hydrodynamik, behandelt Plasmen also als kontinuierliche, leitfähige Fluide, die durch elektromagnetische Felder
beeinflusst werden. Dieser Ansatz hat sich zwar in vielen Fällen als nützlich erwiesen, stößt jedoch an Grenzen, wenn mikroskopische Effekte oder nicht-lokale Wechselwirkungen
eine Rolle spielen. Die \gls{wdbt} hingegen geht von direkten Teilchenwechselwirkungen aus, beschrieben durch die Weber-Kraft, und integriert zudem Quanteneffekte über das
Bohm'sche Quantenpotential. Während die \gls{mhd} mit der Lorentzkraft arbeitet, berechnet die \gls{wdbt} die Kraftdichte durch Integration über Paarkorrelationen, was eine
natürlichere Beschreibung kollektiver Phänomene ermöglicht.

\subsubsection{Stabilität und Wellenausbreitung in Plasmen}
Ein zentraler Unterschied zeigt sich in der Beschreibung von Plasmawellen und Instabilitäten. Die klassische \gls{mhd} sagt Alfvén-Wellen vorher, deren Dispersionrelation durch
Magnetfelder und Plasmadruck bestimmt wird. Die \gls{wdbt} führt dagegen eine Quantenkorrektur ein, die besonders bei hohen Dichten relevant wird - ein Effekt, der tatsächlich in
Experimenten wie Wendelstein 7-X beobachtet wurde. Während die \gls{mhd} auf externe Magnetfelder angewiesen ist, um Plasmen zu stabilisieren, erklärt die \gls{wdbt}
Stabilisierungseffekte durch das Quantenpotential, was völlig neue Möglichkeiten für Fusionsreaktoren eröffnen könnte.

\subsubsection{Kosmologische Implikationen und großskalige Phänomene}
Besonders bemerkenswert sind die Unterschiede bei der Erklärung kosmologischer Phänomene. Die \gls{mhd}-basierte Astrophysik benötigt Konzepte wie dunkle Materie, um die
Rotationskurven von Galaxien zu erklären. Die \gls{wdbt} hingegen bietet eine elegante Alternative durch ihre fraktale Beschreibung der Dichteverteilung, die ohne solche
Zusatzannahmen auskommt. Ähnlich verhält es sich mit den Anisotropien der kosmischen Hintergrundstrahlung: Während das Standardmodell die Inflationstheorie benötigt, ergibt sich
die Skaleninvarianz in der \gls{wdbt} natürlich aus den grundlegenden Gleichungen.

\subsubsection{Experimentelle Konsequenzen und zukünftige Entwicklungen}
Die \gls{wdbt} sagt mehrere messbare Abweichungen von \gls{mhd}-Vorhersagen voraus, etwa bei der Lamb-Verschiebung oder der Lichtausbreitung in Plasmen. Diese Effekte könnten in
modernen Experimenten überprüft werden und würden im Erfolgsfall die Plasmaphysik revolutionieren. Besonders vielversprechend ist das Potential der \gls{wdbt} in der
Fusionsforschung, wo sie zu stabileren und effizienteren Reaktordesigns führen könnte.

\subsubsection{Fazit: Paradigmenwechsel in der Plasmaphysik?}
Während die \gls{mhd} nach wie vor ein unverzichtbares Werkzeug für viele praktische Anwendungen bleibt, deutet vieles darauf hin, dass die \gls{wdbt} eine tiefere und umfassendere
Theorie der Plasmadynamik bietet. Ihre Fähigkeit, mikroskopische und makroskopische Phänomene konsistent zu beschreiben, ohne auf ad-hoc-Annahmen zurückgreifen zu müssen, macht
sie zu einem vielversprechenden Kandidaten für den nächsten großen Schritt in unserem Verständnis ionisierter Materie - von Laborplasmen bis hin zur großräumigen Struktur des
Universums.

\section{Mathematische Herleitung}
\subsection{Integralformulierung}
\[
t = \int \frac{r^2(\phi)}{h} \left(1 - \frac{GM}{c^2 r(\phi)} \cdot \frac{e \sin \phi}{1 + e \cos \phi}\right) d\phi
\]

\subsection{Substitution der Bahnkurve}
\[
t = \frac{a^2 (1-e^2)^2}{h} \int \frac{d\phi}{(1 + e \cos \phi)^2} - \frac{GM a (1-e^2)}{c^2 h} \int \frac{e \sin \phi}{(1 + e \cos \phi)^3} d\phi
\]

\subsection{Lösung der Integrale}
\subsubsection{Hauptterm (klassisch)}
\[
\int \frac{d\phi}{(1 + e \cos \phi)^2} = \frac{e \sin \phi}{(e^2-1)(1 + e \cos \phi)} + \frac{2}{(1-e^2)^{3/2}} \arctan\left(\sqrt{\frac{1-e}{1+e}} \tan \frac{\phi}{2}\right)
\]

\subsubsection{Relativistischer Korrekturterm}
\[
\int \frac{e \sin \phi}{(1 + e \cos \phi)^3} d\phi = \frac{1}{2(1 + e \cos \phi)^2}
\]
\section{Anwendungsbeispiel: Merkur-Orbit}
\subsection{Berechnung für 1° Bahnsegment ($\Delta\phi = \pi/180$)}
\begin{table}[h]
\centering
\begin{tabular}{|l|l|}
\hline
\textbf{Term} & \textbf{Beitrag zur Zeit t} \\ \hline
Klassisch (Kepler) & $\approx 7.0$ Tage \\ \hline
Relativistische Korrektur & $\approx -0.002$ Tage ($\approx -3$ Minuten) \\ \hline
\textbf{Gesamt} & \textbf{$\approx 6.998$ Tage} \\ \hline
\end{tabular}
\end{table}

\subsection{Physikalische Interpretation}
Die negative Korrektur zeigt, dass der Merkur schneller als klassisch vorhergesagt läuft -- dies erklärt die beobachtete Periheldrehung von $43''$ pro Jahrhundert.
\section{Vergleich mit der ART}
Ihre Theorie liefert für schwache Felder ($GM/rc^2 \ll 1$) dieselbe Zeitberechnung wie die 1. post-newtonsche Näherung der ART:
\[
t_{\text{ART}} = t_{\text{klassisch}} \left(1 - \frac{3GM}{c^2 a(1-e^2)}\right)
\]

\subsection{Vorteile der Formulierung}
\begin{itemize}
\item Zeitberechnung direkt aus der Bahngeometrie $r(\phi)$
\item Kein Metriktensor benötigt
\item Ideal für numerische Simulationen
\end{itemize}
\section{Zusammenfassung}
\begin{itemize}
\item Die Zeitintegration aus $\omega(\phi)$ ist \textbf{analytisch näherbar} und \textbf{GPU-freundlich} implementierbar
\item Die relativistischen Korrekturen reproduzieren die \textbf{Periheldrehung des Merkur}
\item Der Formalismus kommt \textbf{ohne Raumzeitkrümmung} aus und vermeidet Singularitäten
\end{itemize}
\section{Universelle Knoten-Gitter-Dynamik}

\subsection{Grundform der Theorie}
\begin{equation}
\mathcal{S} = \sum_{\text{alle Knoten } i} \left[ 
\frac{E[V_i(t)]}{c^2} \left( 
1 - \frac{|\Delta \vec{x}_i|^2}{L_p^2} + \frac{\vec{x}_i \cdot \Delta^2 \vec{x}_i}{2 L_p^2} 
\right) 
+ \lambda \oint \frac{V_i'(t)}{V_i(t)} \, dt 
\right]
\end{equation}

\subsection{Symbolerklärungen}
\begin{tabular}{lll}
    \( E[V_i(t)] \) & Knotenenergie & Jones-Polynom \\
    \( \Delta \vec{x}_i \) & Diskrete Ableitung & Gittergeometrie \\
    \( L_p \) & Planck-Länge & Fundamentale Skala \\
    \( \lambda \) & Topologische Kopplung & Universelle Konstante \\
\end{tabular}
\section{Vollständige analytische Lösung für $\vec{v}(\phi)$ mit Weber-Kraft}

\subsection{Definition der Variablen}
\begin{itemize}
    \item $G = 6.67430 \times 10^{-11}\,\text{m}^3\,\text{kg}^{-1}\,\text{s}^{-2}$ (Gravitationskonstante)
    \item $c = 299,\!792,\!458\,\text{m/s}$ (Lichtgeschwindigkeit)
    \item $M$: Masse des Zentralkörpers [kg]
    \item $a$: Große Halbachse [m]
    \item $e$: Exzentrizität ($0 \leq e < 1$)
    \item $\phi$: Wahre Anomalie [rad]
    \item $h = \sqrt{G M a(1-e^2)}$ (Spezifischer Drehimpuls)
    \item $\kappa = \sqrt{1 - \frac{6GM}{c^2 a(1-e^2)}}$ (Relativistischer Korrekturfaktor)
\end{itemize}

\subsection{Exakte Bahngleichung}
\begin{equation}
r(\phi) = \frac{a(1-e^2)}{1 + e \cos(\kappa \phi)}
\end{equation}

\subsection{Geschwindigkeitskomponenten}
\subsubsection{Radialkomponente}
\begin{equation}
v_r(\phi) = \frac{h e \kappa \sin(\kappa \phi)}{a(1-e^2)}
\end{equation}

\subsubsection{Azimutalkomponente}
\begin{equation}
v_\phi(\phi) = \frac{h}{r(\phi)} = \sqrt{\frac{GM}{a(1-e^2)}} \left(1 + e \cos(\kappa \phi)\right)
\end{equation}

\subsection{Vektorielle Geschwindigkeit}
\begin{equation}
\vec{v}(\phi) = \sqrt{\frac{GM}{a(1-e^2)}} \left(
    \frac{e \kappa \sin(\kappa \phi)}{1 + e \cos(\kappa \phi)} \, \hat{r}
    + \left[1 + e \cos(\kappa \phi)\right] \hat{\phi}
\right)
\end{equation}
\section{N-Körper-Integration mit Velocity-Verlet}

\subsection{Physikalische Grundgleichungen}
\begin{equation}
\vec{F}_{ij} = -G \frac{m_i m_j (\vec{x}_i - \vec{x}_j)}{|\vec{x}_i - \vec{x}_j|^3}
\end{equation}

\subsection{Velocity-Verlet Algorithmus}
\subsubsection{Initialisierung (t = 0)}
\begin{itemize}
    \item Startpositionen $\vec{x}_i(0)$ und Geschwindigkeiten $\vec{v}_i(0)$
    \item Anfangsbeschleunigungen:
    \begin{equation}
    \vec{a}_i(0) = \frac{1}{m_i} \sum_{j \neq i} \vec{F}_{ij}(0)
    \end{equation}
\end{itemize}

\subsubsection{Zeitschritt $t \to t + \Delta t$}
\begin{enumerate}
    \item Halber Geschwindigkeitsschritt:
    \begin{equation}
    \vec{v}_i\left(t + \frac{\Delta t}{2}\right) = \vec{v}_i(t) + \frac{1}{2} \vec{a}_i(t) \Delta t
    \end{equation}
    
    \item Positionsupdate:
    \begin{equation}
    \vec{x}_i(t + \Delta t) = \vec{x}_i(t) + \vec{v}_i\left(t + \frac{\Delta t}{2}\right) \Delta t
    \end{equation}
    
    \item Neue Beschleunigungen berechnen:
    \begin{equation}
    \vec{a}_i(t + \Delta t) = \frac{1}{m_i} \sum_{j \neq i} \vec{F}_{ij}(t + \Delta t)
    \end{equation}
    
    \item Vollständiger Geschwindigkeitsschritt:
    \begin{equation}
    \vec{v}_i(t + \Delta t) = \vec{v}_i\left(t + \frac{\Delta t}{2}\right) + \frac{1}{2} \vec{a}_i(t + \Delta t) \Delta t
    \end{equation}
\end{enumerate}

\subsection{Energieerhaltung}
\begin{equation}
E_{\text{ges}} = \sum_i \frac{1}{2} m_i |\vec{v}_i|^2 - G \sum_{i < j} \frac{m_i m_j}{|\vec{x}_i - \vec{x}_j|}
\end{equation}

\subsection{Zeitschrittkontrolle}
\begin{equation}
\Delta t \approx \frac{T}{10^4} \quad \text{(mit $T$ = typische Umlaufzeit)}
\end{equation}
\section{Universelles Zeitformat für Himmelskörper}

\subsection{Standardisiertes Format}
\begin{equation}
\tau = \text{floor}\left(\frac{t}{T}\right) + \frac{\phi(t)}{2\pi}
\end{equation}
wobei:
\begin{itemize}
    \item $t$ = Zeit in Sekunden seit Referenzpunkt
    \item $T$ = Umlaufperiode des Referenzkörpers
    \item $\phi(t)$ = Wahre Anomalie zum Zeitpunkt $t$
\end{itemize}

\subsection{Anwendungsbeispiele}
\begin{itemize}
    \item \textbf{Erde-Mond System:} 2030.5000000
    \begin{itemize}
        \item 2030 = Erdumläufe seit Referenz
        \item 0.5000000 = Mondposition $\phi = \pi$ (180°)
    \end{itemize}
    
    \item \textbf{Mars Mission:} 15.7843210
    \begin{itemize}
        \item 15 = Marsjahre seit Referenz
        \item 0.7843210 = Position $\phi \approx 4.93$ rad (282°)
    \end{itemize}
\end{itemize}

\subsection{Technische Umsetzung}
\begin{verbatim}
typedef struct {
    uint32_t base_cycles;  // Ganzzahlige Umläufe
    double phase;          // Bahnphase [0,1)
} CelestialTime;
\end{verbatim}

\subsection{Vorteile}
\begin{itemize}
    \item Universell anwendbar auf alle Himmelskörper
    \item Präzision: 7 Dezimalstellen ($\pm 0.03$s für Erdumlauf)
    \item Menschenlesbare Darstellung
    \item Keine Schaltsekunden nötig
\end{itemize}

\subsection{Vergleich mit anderen Systemen}
\begin{tabular}{lllll}
    \hline
    System & Präzision & Astronomisch & Mehrkörper & Menschlich \\
    \hline
    UTC & $\pm 1$s & Nein & Nein & Ja \\
    Julianisches Datum & Mikrosekunden & Ja & Nein & Nein \\
    \textbf{YYYY.ZZZZZZZ} & 0.03s (Erde) & Ja & Ja & Ja \\
    \hline
\end{tabular}

\subsection{Mars Rover Beispiel}
\begin{equation}
5.3274510
\end{equation}
\begin{itemize}
    \item 5 = Fünftes Marsjahr seit Landung
    \item 0.3274510 = Position $\phi \approx 2.057$ rad (118°)
\end{itemize}
\section{Vorteile des himmelsmechanischen Zeitsystems}

\subsection{Physikalisch konsistente Zeitmessung}
\begin{equation}
\tau(t) = \frac{1}{2\pi} \int_0^t \dot{\phi}(t') dt'
\end{equation}
\begin{itemize}
    \item Keine willkürlichen Korrekturen wie Schaltsekunden
    \item Automatische Berücksichtigung von Bahnstörungen
    \item Direkte Kopplung an die tatsächliche Position im Orbit
\end{itemize}

\subsection{Universelle Anwendbarkeit}
\begin{tabular}{lll}
    \hline
    Körper & Zeitdefinition & Zykluslänge \\
    \hline
    Erde & $\tau_E = N_E + \frac{\phi_E}{2\pi}$ & 365.25 Tage \\
    Mond & $\tau_M = N_M + \frac{\phi_M}{2\pi}$ & 27.3 Tage \\
    Mars & $\tau_{Mars} = N_{Mars} + \frac{\phi_{Mars}}{2\pi}$ & 687 Tage \\
    \hline
\end{tabular}

\subsection{Präzisionsgewinn}
\subsubsection{Astronomische Beobachtungen}
\begin{equation}
t_{obs} \rightarrow \phi(t_{obs}) \rightarrow r(\phi)
\end{equation}

\subsubsection{Raumfahrtmissionen}
\begin{equation}
\Delta\tau = \tau_1 - \tau_2 = \frac{\Delta\phi}{2\pi} T
\end{equation}

\subsection{Praktische Anwendungen}
\subsubsection{Für Mondkolonien}
\begin{itemize}
    \item Natürliche Tageseinteilung nach Sonnenstand ($\phi$-Wert)
    \item Automatische Synchronisation mit Erde ohne Zeitzonen
    \item Energieplanung basierend auf Solarwinkel
\end{itemize}

\subsection{Langfristige Stabilität}
\begin{tabular}{lll}
    \hline
    Aspekt & UTC-System & Winkelzeit-System \\
    \hline
    Genauigkeit & $\pm0.9$s (UT1-UTC) & $10^{-12}$s \\
    Korrekturen & 27 Schaltsekunden & Automatisch \\
    Anwendungsbereich & Nur Erde & Beliebige Himmelskörper \\
    \hline
\end{tabular}

\subsection{Implementierungsbeispiel}
\begin{verbatim}
function earthToLunarTime(earthTime) {
    const a = 384748e3;  // Große Halbachse [m]
    const e = 0.0549;    // Exzentrizität
    const T = 27.321661 * 86400;  // Umlaufperiode [s]
    
    const M = 2 * Math.PI * earthTime / T;
    let E = M;
    for(let i = 0; i < 10; i++) {
        E = M + e * Math.sin(E);
    }
    const phi = 2 * Math.atan(Math.sqrt((1+e)/(1-e)) * Math.tan(E/2));
    
    return {
        cycles: Math.floor(earthTime / T),
        angle: phi % (2 * Math.PI)
    };
}
\end{verbatim}
\section{Natürliche Zeitdefinition für Himmelskörper}

\subsection{Grundprinzip der Winkelzeit}
\begin{equation}
\tau = N + \frac{\phi}{2\pi}
\end{equation}
\begin{itemize}
    \item $N$ = Anzahl vollendeter Umläufe (ganzzahlig)
    \item $\phi$ = wahre Anomalie ($0 \leq \phi < 2\pi$)
\end{itemize}

\subsection{Erde-Mond-Zeitsystem}
\subsubsection{Erdzeit (ET)}
\begin{equation}
\tau_{\text{Erde}} = N_E + \frac{\phi_E}{2\pi}
\end{equation}
\begin{itemize}
    \item 1 ET-Jahr = 1 Erdumlauf (365.25 Tage)
    \item 1 ET-Tag = $2\pi$ Rotation (24 Stunden)
\end{itemize}

\subsubsection{Mondzeit (LT)}
\begin{equation}
\tau_{\text{Mond}} = N_M + \frac{\phi_M}{2\pi}
\end{equation}
\begin{itemize}
    \item 1 LT-Jahr = 1 Mondumlauf (27.3 Tage)
    \item 1 LT-Tag = $2\pi$ Rotation (29.5 ET-Tage)
\end{itemize}

\subsection{Zeitumrechnung}
\subsubsection{Kepler-Gleichung für den Mond}
\begin{equation}
E - e\sin E = M(t) = \sqrt{\frac{GM}{a^3}} \cdot t
\end{equation}
\begin{equation}
\phi_M = 2 \arctan\left(\sqrt{\frac{1+e}{1-e}} \tan\frac{E}{2}\right)
\end{equation}

\subsection{Kalendersystem}
\begin{tabular}{lll}
    \hline
    Element & Erde & Mond \\
    \hline
    Grundzyklus & Sonnenumlauf (Jahr) & Erdumlauf (Monat) \\
    Untereinheit & Eigenrotation (Tag) & Eigenrotation (Lunation) \\
    Natürliche Zeit & $\tau_E = N_E + \frac{\phi_E}{2\pi}$ & $\tau_M = N_M + \frac{\phi_M}{2\pi}$ \\
    \hline
\end{tabular}

\subsection{Implementierung}
\begin{itemize}
    \item Natürliche Synchronisation mit Himmelskörpern
    \item Keine willkürlichen Zeitzonen
    \item Direkte Korrelation mit Sonnen-/Erdposition
    \item Universelle Anwendbarkeit auf alle Himmelskörper
\end{itemize}

\begin{verbatim}
LOCAL TIME SYSTEM: LUNA-STATION-1
MOON TIME: CYCLES=683.214 [PHI=1.34rad]
EARTH TIME: CYCLES=1969.552 [PHI=4.71rad]
SUN POSITION: 47° ABOVE HORIZON
EARTH POSITION: 23° ABOVE HORIZON
\end{verbatim}

\end{document}
