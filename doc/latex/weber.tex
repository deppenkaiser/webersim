\documentclass{book}
\usepackage[a4paper,left=2.5cm,right=2cm,top=2cm,bottom=2.5cm]{geometry}
\usepackage[utf8]{inputenc}
\usepackage[ngerman]{babel}
\usepackage{amsfonts}
\usepackage{amsmath}
\usepackage{amssymb}
\usepackage{array}
\usepackage{ragged2e}
\usepackage{tabularx}
\usepackage{enumitem}
\usepackage{booktabs}
\usepackage{bm}
\usepackage{csquotes}
\usepackage{siunitx}
\usepackage{parskip}
\usepackage{listings}
\usepackage{xcolor}
\usepackage[utf8]{inputenc}
\usepackage[labelfont=bf]{caption}

\renewcommand{\arraystretch}{1.1}
\numberwithin{equation}{section}
\definecolor{gray}{rgb}{0.5,0.5,0.5}
\cleardoublepage

\begin{document}

\title{Weber-Gravitation}
\author{Michael Czybor}
\date{\today}
\maketitle

\section*{Zusammenfassung}
Die bisherigen Untersuchungen zeigen, dass die Weber-Gravitation (WG) bessere Ergebnisse als die allgemeine Relativitätstheorie (ART) liefert. Diese Tatsache wird durch drei wichtige
Eigenschaften belegt: Die Periheldrehung des Merkur wird mit der WG nahezu identisch zur ART berechnet. Wobei sich aus den zwei weiteren Gründen die Vermutung ergibt, dass das WG-Ergebnis
das genauere ist.

Die unphysikalische ART-Unterstellung, es gäbe Singularitäten durch sog. schwarze Löcher, wird durch die WG widerlegt. Dadurch ist auch die Gravitation in der Nähe eines schwarzen Lochs
eine andere, als es von der ART behauptet wird. Selbiges gilt auch für Galaxien, wo die WG im Gegensatz zur ART, die Rotationskurven der Galaxien auf natürliche Weise, ohne unphysikalische
\enquote{dunkle Materie}, erklären kann. Ich meine auch schon festgestellt zu haben, dass die äußeren Planeten ebenfalls schon abweichende Werte aufzeigen (Rotationskurven).

Bis auf die Gravitationswellen, kann die WG alles erklären. Das sie die Gravitationswellen nicht erklären kann, liegt an der Tatsache, dass die WG kein Raummodell besitzt. Das muss aber nicht
unbedingt ein Nachteil sein. Der Vorteil besteht nämlich darin, dieses noch zu entwicklen und entsprechend der bisherigen Messergebnisse valide zu gestalten. Hier wäre die Quantengravitation
als potentielles Raummodell denkbar.

Darüber hinaus vereint das Weber-System auch die Elektrodynamik mit der Gravitation. Das ist eine außergewöhnliche Leistung, die in Richtung einer ToE deutet. Dieses brilliante System ist
leider seid 100 Jahren unbeachtet geblieben.

Vor kurzem wurde ich auch noch auf die De-Broglie-Bohm-Theorie aufmerksam, welche mich noch mehr ermutigt, diese Zeilen zu schreiben. Im Anhang habe ich dazu meine ersten Gedanken formuliert.

\tableofcontents

% Einbindung der einzelnen TeX-Dateien
\part{Grundlagen}
\chapter{Weber-Kraft}
\input{content/07_weber_kraft_em}
\input{content/09_grundgleichungen}
\newpage
\section{Bahngleichungen}
\subsection{Bahngleichung 1. Ordnung}
Die Bahngleichung \(r(\phi)\) in der Weber-Gravitation bis zur Ordnung \(\mathcal{O}(c^{-2})\) lautet:

\begin{equation}
\label{eq:bahngleichung_1_ordnung}
r(\phi) = \frac{a(1 - e^2)}{1 + e \cos\left(\kappa\phi\right)}
\end{equation}

\noindent mit der Definition:
\begin{equation}
\label{eq:kappa_1_ordnung}
\kappa = \sqrt{1 - \frac{6GM}{c^2a(1 - e^2)}}
\end{equation}

\subsection*{Mathematische Herleitung}
Die Gleichung folgt aus der Lösung der Bewegungsgleichung:
\begin{equation}
\frac{d^2u}{d\phi^2} + u = \frac{GM}{h^2} + \frac{6GM}{c^2} u^2 \quad \left(u = \frac{1}{r}\right),
\end{equation}

wobei der Term \(\frac{6GM}{c^2} u^2\) die Weber-spezifische Korrektur 1. Ordnung darstellt. Der Ansatz \(u(\phi) = \frac{1 + e \cos(\kappa\phi)}{a(1 - e^2)}\) führt auf die angegebene Lösung.

Mit $u=1/r$ und Drehimpuls $h$ (\ref{eq:spezifischer_drehimpuls_h}):
\begin{equation}
\frac{d^2u}{d\phi^2} + u = \frac{GM}{h^2} + \frac{6GM}{c^2}u^2 + \frac{GM}{2c^2}\left(u\frac{d^2u}{d\phi^2} + \left(\frac{du}{d\phi}\right)^2\right)
\end{equation}

\subsection{Bahngleichung 2. Ordnung}
Bahngleichung:
\begin{equation}
\label{eq:bahngleichung_2_ordnung}
    \boxed
    {
        r(\phi) = \frac{a(1-e^2)}{1 + e\cos\left(\kappa\phi + \alpha\phi^2\right)}
    }
\end{equation}

mit:
$h$ aus Gleichung (\ref{eq:spezifischer_drehimpuls_h})
\begin{equation}
\label{eq:kappa_2_ordnung}
\kappa = \sqrt{1 - \frac{6GM}{c^2a(1-e^2)} + \frac{27G^2M^2}{2c^4a^2(1-e^2)^2}}
\end{equation}
\begin{equation}
\label{eq:alpha}
\alpha = \frac{3G^2M^2e}{8h^4c^4}
\end{equation}

\section{Periheldrehung}
\subsection{Periheldrehung 1. Ordnung}
Die Periheldrehung $\Delta\phi$ in der Weber-Gravitation ergibt sich aus der modifizierten Bahngleichung und lässt sich wie folgt herleiten:

\subsection*{Perihelbedingung}
Das Perihel (sonnennächster Punkt) tritt auf, wenn der Nenner maximal wird, d.h. wenn:\\
\[\cos(\kappa\phi) = 1\]
Die Lösungen dieser Bedingung sind: $\kappa\phi = 2\pi n \quad \text{(für $n \in \mathbb{Z}$)}$.\\

Somit ergeben sich die Winkel für aufeinanderfolgende Periheldurchgänge zu:
\[
    \phi_n = \frac{2\pi n}{\kappa}.
\]

\subsection*{Periheldrehung pro Umlauf}
Die Periheldrehung $\Delta\phi$ ist die Differenz zwischen dem Winkel für einen vollständigen Umlauf ($n = 1$) und dem Newton'schen Fall ($\kappa = 1$):
\[
    \Delta\phi = \phi_1 - 2\pi = \frac{2\pi}{\kappa} - 2\pi.
\]
Daraus folgt die gesuchte Gleichung:
\begin{equation}
\boxed
{
    \Delta\phi = 2\pi\left(\frac{1}{\kappa} - 1\right)
}.
\end{equation}

\subsection*{Interpretation}
\begin{itemize}
\item Im Newton'schen Grenzfall ($\kappa = 1$) verschwindet die Periheldrehung ($\Delta\phi = 0$).
\item Für $\kappa < 1$ (Weber-Gravitation) ergibt sich eine positive Periheldrehung, die mit Beobachtungen (z.B. Merkurperihel) übereinstimmt.
\end{itemize}

\subsection{Periheldrehung in 2. Ordnung}
\subsection*{Entwicklung von $\kappa$}
Eine Taylor-Entwicklung von $\kappa$ bis zur 2. Ordnung liefert:
\[
    \kappa \approx 1 - \frac{3GM}{c^2 a(1 - e^2)} + \frac{27G^2 M^2}{4c^4 a^2 (1 - e^2)^2} + \mathcal{O}(c^{-6}).
\]

\subsection*{Perihelbedingung}
Das Perihel tritt auf bei:
\[
    \cos\left(\kappa\phi + \alpha\phi^2\right) = 1 \quad \Rightarrow \quad \kappa\phi + \alpha\phi^2 = 2\pi n.
\]

\subsection*{Lösung für $\Delta\phi$}
Für $n=1$ (ein Umlauf) ergibt sich die quadratische Gleichung:
\[
\alpha\phi^2 + \kappa\phi - 2\pi = 0.
\]
Die Lösung lautet:
\begin{equation}
\phi = \frac{-\kappa + \sqrt{\kappa^2 + 8\pi\alpha}}{2\alpha}.
\end{equation}

\subsection*{Näherung für kleine Korrekturen}
Da $\alpha \sim c^{-4}$ klein ist, entwickeln wir die Wurzel:
\[
    \phi \approx \frac{2\pi}{\kappa} - \frac{4\pi^2\alpha}{\kappa^3} + \mathcal{O}(\alpha^2).
\]
Die Periheldrehung pro Umlauf wird damit:
\begin{equation}
\Delta\phi = \phi - 2\pi \approx 2\pi\left(\frac{1}{\kappa} - 1\right) - \frac{4\pi^2\alpha}{\kappa^3}.    
\end{equation}

\subsection*{Endgültige Formel}
Einsetzen von $\kappa \approx 1$ im Korrekturterm liefert:
\begin{equation}
\boxed
{
    \Delta\phi \approx 2\pi\left(\frac{1}{\kappa} - 1\right) - 4\pi^2\alpha
},
\end{equation}

\subsection*{Vollständige Koeffizienten}
Explizit ausgedrückt, mit Bezug auf die Ordnungen:
\begin{align*}
\Delta\phi^{(2)} &= \frac{6\pi GM}{c^2 a(1 - e^2)} \left[1 + \frac{9GM}{4c^2 a(1 - e^2)}\right] - \frac{3\pi^2 G^2 M^2 e}{2c^4 h^4} \\
&= \Delta\phi^{(1)} + \frac{27\pi G^2 M^2}{2c^4 a^2 (1 - e^2)^2} - \frac{3\pi^2 G^2 M^2 e}{2c^4 [GMa(1 - e^2)]^2}
\end{align*}

\newpage
\section{Winkelgeschwindigkeit 1. Ordnung}
Die Winkelgeschwindigkeit \(\omega(\phi)\) in der Weber-Gravitation bis zur Ordnung \(\mathcal{O}(c^{-2})\) lautet:

\begin{equation}
\omega(\phi) = \frac{h}{a^2(1 - e^2)^2} \left[1 + e \cos\left(\kappa\phi\right)\right]^2
\end{equation}

wobei:
$h$ aus Gleichung (\ref{eq:spezifischer_drehimpuls_h}), $\kappa$ aus Gleichung (\ref{eq:kappa_1_ordnung}) stammt.

\subsection*{Bedeutung der Terme}
\begin{itemize}
    \item \(\kappa\) beschreibt die Periheldrehung 1. Ordnung ohne Näherung.
    \item Für \(c \to \infty\) wird \(\kappa = 1\), und die Gleichung reduziert sich auf die Newton’sche Form:
    \[
    \omega_N(\phi) = \frac{h(1 + e \cos\phi)^2}{a^2(1 - e^2)^2}.
    \]
\end{itemize}

\section{Winkelgeschwindigkeit 2. Ordnung}

\subsection{Winkelgeschwindigkeit}
Mit $h$ aus Gleichung (\ref{eq:spezifischer_drehimpuls_h}), $\kappa$ aus Gleichung (\ref{eq:kappa_2_ordnung}) und $\alpha$ aus Gleichung (\ref{eq:alpha}):
\begin{equation}
\boxed
{
    \omega(\phi) = \frac{h[1 + e\cos(\kappa\phi + \alpha\phi^2)]^2}{a^2(1-e^2)^2}
}
\end{equation}

\newpage
\section{Bahngeschwindigkeit in 1. Ordnung}
Die Bahngeschwindigkeit \(v(\phi)\) in der Weber-Gravitation bis zur Ordnung \(\mathcal{O}(c^{-2})\) lautet:

\begin{equation}
v(\phi) = \frac{h}{a(1 - e^2)} \left(1 + e \cos\left(\kappa\phi\right)\right)
\end{equation}

\noindent mit den Definitionen:
$h$ aus Gleichung (\ref{eq:spezifischer_drehimpuls_h}), $\kappa$ aus Gleichung (\ref{eq:kappa_1_ordnung})

\subsection*{Physikalische Interpretation}
\begin{itemize}
    \item \textbf{Struktur}: Die Geschwindigkeit folgt aus \(v(\phi) = h/r(\phi)\) mit der Bahngleichung \(r(\phi) = \frac{a(1 - e^2)}{1 + e \cos(\kappa\phi)}\).
    \item \textbf{Relativistische Korrektur}: \(\kappa\) modifiziert die Periheldrehung gegenüber Newton (\(\kappa = 1\)).
    \item \textbf{Grenzfälle}:
        \begin{itemize}
            \item Perihel (\(\phi = 0\)): \(v(0) = \frac{h(1 + e)}{a(1 - e^2)}\),
            \item Aphel (\(\phi = \pi\)): \(v(\pi) = \frac{h(1 - e)}{a(1 - e^2)}\),
            \item Newton (\(c \to \infty\)): \(v_N(\phi) = \frac{h(1 + e \cos\phi)}{a(1 - e^2)}\).
        \end{itemize}
\end{itemize}

\section{Bahngeschwindigkeit in 2. Ordnung}
Die Bahngeschwindigkeit $v(\phi)$ ergibt sich aus Winkelgeschwindigkeit $\omega(\phi)$ und Radialabstand $r(\phi)$:
\begin{equation}
v(\phi) = \omega(\phi) \cdot r(\phi) = \frac{h}{r(\phi)}
\end{equation}

Mit der Bahngleichung und Winkelgeschwindigkeit:
\begin{align}
r(\phi) &= \frac{a(1-e^2)}{1 + e\cos\left(\kappa\phi + \alpha\phi^2\right)}\\
\omega(\phi) &= \frac{h[1 + e\cos(\kappa\phi + \alpha\phi^2)]^2}{a^2(1-e^2)^2}
\end{align}

ergibt sich:
\begin{equation}
v(\phi) = \frac{h \left(1 + e\cos(\kappa\phi + \alpha\phi^2)\right)}{a(1 - e^2)}.
\end{equation}

\chapter{Sonnensystem}
\input{content/15_perihel}
\input{content/16_lichtablenkung}
\newpage
\subsection{Umlaufperiode $T$ in 1. Ordnung}
Die Umlaufperiode $T$ eines Planeten in der Weber-Gravitation (WG) ergibt sich aus der modifizierten Bahngleichung (\ref{eq:bahngleichung_1_ordnung}) und dem spezifischen Drehimpuls $h$ (\ref{eq:spezifischer_drehimpuls_h}).

\subsubsection*{Ausgangsgleichungen}
\begin{enumerate}
    \item $\kappa$ aus Gl. (\ref{eq:kappa_1_ordnung})
    \item Winkelgeschwindigkeit Gl. (\ref{eq:winkelgeschwindigkeit_1_ordnung})
\end{enumerate}

\subsubsection*{Schritt 2: Integration über einen Umlauf}
Die Periode $T$ ist die Zeit für $\phi = 0 \to 2\pi/\kappa$ (WG-Korrektur durch $\kappa$):
\begin{align}
    T &= \int_0^{2\pi/\kappa} \frac{d\phi}{\dot{\phi}} 
       = \frac{a^2(1-e^2)^2}{h} \int_0^{2\pi/\kappa} \frac{d\phi}{(1 + e \cos(\kappa \phi))^2}.
\end{align}

\subsubsection*{Schritt 3: Lösung des Integrals}
Mit der Substitution $\psi = \kappa \phi$ und $\cos^2$-Identität:
\begin{align}
\label{eq:integral_t}
    T &= \frac{a^2(1-e^2)^2}{h \kappa} \int_0^{2\pi} \frac{d\psi}{(1 + e \cos \psi)^2} 
       = \frac{2\pi a^2(1-e^2)^2}{h \kappa (1-e^2)^{3/2}} 
       = \frac{2\pi a^{3/2}}{\sqrt{GM} \kappa}.
\end{align}

\subsubsection*{Schritt 4: Entwicklung von $\kappa$}
Für kleine relativistische Korrekturen ($c^{-2}$-Ordnung) gilt:
\begin{equation}
    \kappa \approx 1 - \frac{3GM}{c^2 a(1-e^2)} + \mathcal{O}(c^{-4}).
\end{equation}
Einsetzen in Gl.~\eqref{eq:integral_t} liefert die Periode in 1. Ordnung:
\begin{equation}
    \boxed{
    T \approx \frac{2\pi a^{3/2}}{\sqrt{GM}} \left(1 + \frac{3GM}{c^2 a(1-e^2)}\right).
    }
\end{equation}

\newpage
\section{Umlaufperiode \( T \) 2. Ordnung}

\subsubsection*{Ausgangsgleichungen}
\begin{enumerate}
    \item Bahngleichung in Polarkoordinaten (\ref{eq:bahngleichung_2_ordnung})
    \item $\kappa$ aus Gl. (\ref{eq:kappa_2_ordnung})
    \item $\alpha$ aus Gl. (\ref{eq:alpha})
    \item $h$ aus Gl. (\ref{eq:spezifischer_drehimpuls_h})
\end{enumerate}

\subsection*{Schritt 1: Entwicklung von \(\kappa\)}
\begin{equation}
\kappa \approx 1 - \frac{3GM}{c^2a(1-e^2)} + \frac{27G^2M^2}{4c^4a^2(1-e^2)^2} - \frac{81G^3M^3}{8c^6a^3(1-e^2)^3} + \mathcal{O}(c^{-8}) 
\end{equation}

\subsection*{Schritt 2: Vollständige Integration}
Die Umlaufperiode \( T \) ist:
\begin{equation}
T = \frac{1}{h} \int_0^{2\pi} r^2(\phi) \, d\phi = \frac{a^2(1-e^2)^2}{h} \int_0^{2\pi} \frac{d\phi}{\left[1 + e\cos\left(\kappa\phi + \alpha\phi^2\right)\right]^2}
\end{equation}

\subsection*{Schritt 3: Behandlung des Integrals}
Mit Substitution \(\psi = \kappa\phi + \alpha\phi^2\) und Entwicklung bis \(\mathcal{O}(c^{-4})\):
\begin{align}
T &= \frac{a^2(1-e^2)^2}{h} \left[ \int_0^{2\pi} \frac{d\phi}{(1 + e\cos\psi)^2} + \mathcal{O}(c^{-6}) \right] \\
  &= \frac{2\pi a^{3/2}}{\sqrt{GM}} \left[1 + \frac{3GM}{2c^2a(1-e^2)} + \frac{45G^2M^2}{8c^4a^2(1-e^2)^2}\left(1 - \frac{e^2}{3}\right)\right]
\end{align}

\part{Kosmologie}
\input{content/21_DM.tex}
\part{Anhang}
\chapter{Diskussionen}
\input{content/26_welle}
\chapter{Ergänzende Informationen}
\input{content/27_beta}

\end{document}
