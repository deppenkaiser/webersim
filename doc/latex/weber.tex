\documentclass{book}
\usepackage[a4paper,left=2.5cm,right=2cm,top=2cm,bottom=2.5cm]{geometry}
\usepackage[utf8]{inputenc}
\usepackage[ngerman]{babel}
\usepackage{amsfonts}
\usepackage{amsmath}
\usepackage{amssymb}
\usepackage{array}
\usepackage{ragged2e}
\usepackage{tabularx}
\usepackage{enumitem}
\usepackage{booktabs}
\usepackage{bm}
\usepackage{csquotes}
\usepackage{siunitx}
\usepackage{parskip}
\usepackage{listings}
\usepackage{xcolor}
\usepackage[utf8]{inputenc}
\usepackage[labelfont=bf]{caption}
\usepackage{tcolorbox}
\usepackage{showlabels}
\usepackage{refcheck}

\newtheorem{theorem}{Theorem} % Definiert das 'theorem'-Environment
\newtheorem{lemma}{Lemma}     % Falls Sie auch Lemmas verwenden möchten
\showlabels{cite,label}

\renewcommand{\arraystretch}{1.1}
\numberwithin{equation}{section}
\definecolor{gray}{rgb}{0.5,0.5,0.5}
\cleardoublepage

\begin{document}

\title{Weber-Gravitation}
\author{Michael Czybor}
\date{\today}
\maketitle

\section*{Zusammenfassung}
Die \enquote{Weber-Gravitation} präsentiert eine alternative Gravitationstheorie, die auf der Weber-Kraft basiert und in vielen Aspekten konkurrenzfähige
oder sogar überlegene Ergebnisse im Vergleich zur Allgemeinen Relativitätstheorie \cite{einstein1915} (ART) liefert. Die zentrale These der Arbeit ist, dass die
Weber-Gravitation (WG) nicht nur die bekannten Phänomene der ART erklärt, sondern auch deren Schwächen – wie die Notwendigkeit dunkler Materie
oder die Existenz singularitätsbehafteter schwarzer Löcher – vermeidet.

Ein herausragendes Ergebnis der WG ist die präzise Berechnung der Periheldrehung des Merkurs, die mit einem Wert von 42,98 Bogensekunden pro Jahrhundert
nahezu identisch zur ART-Vorhersage ist. Entscheidend ist jedoch, dass die WG dies ohne ein gekrümmtes Raumzeit-Modell erreicht. Stattdessen modifiziert
sie das Newtonsche Gravitationsgesetz durch relativistische Korrekturen, die von der radialen Geschwindigkeit (\(\dot{r}\)) und Beschleunigung (\(\ddot{r}\)) abhängen.
Die daraus abgeleitete Bahngleichung zeigt, dass die WG die beobachtete Periheldrehung natürlicher erklärt als die ART, ohne auf ein komplexes geometrisches
Raummodell zurückgreifen zu müssen.

Ein weiterer wesentlicher Vorteil der WG ist ihre Fähigkeit, galaktische Rotationskurven ohne dunkle Materie zu beschreiben. Während die ART zusätzliche,
unsichtbare Masse postulieren muss, um die flachen Rotationsprofile von Galaxien zu erklären, liefert die WG eine korrigierte Geschwindigkeitsformel,
die den beobachteten Verlauf reproduziert:  

\[
v(r) = \sqrt{\frac{GM}{r}} \left(1 + \frac{GM}{4c^2r}\right).
\]  

Dieser Ansatz vermeidet nicht nur die hypothetische dunkle Materie, sondern bietet auch eine direkte physikalische Interpretation der Abweichungen vom Newtonschen Gesetz.

Die Arbeit diskutiert zudem die Lichtablenkung im Gravitationsfeld, wobei die WG eine frequenzabhängige Korrektur vorhersagt, die in der ART nicht existiert.
Diese könnte zukünftig experimentell überprüft werden, etwa durch hochpräzise Messungen der Ablenkung von Radiowellen gegenüber optischem Licht. Auch der
Shapiro-Effekt \cite{shapiro1964} (Laufzeitverzögerung von Signalen) wird in der WG leicht modifiziert, wobei die Abweichungen zur ART jedoch erst bei extrem hohen Genauigkeiten messbar wären.  

Ein radikaler Unterschied zur ART zeigt sich in der kosmologischen Interpretation der Rotverschiebung. Während die ART diese als Folge der Expansion des Universums deutet,
erklärt die WG sie durch kumulative gravitative Wechselwirkungen:

\[
z \approx \frac{3}{2} \frac{v_r^2}{c^2}.
\]  

Dies impliziert ein statisches Universum ohne Urknall, was eine grundlegend andere Kosmologie zur Folge hätte. Die Arbeit argumentiert, dass dieser Ansatz mehrere Probleme
der Standardkosmologie (wie die dunkle Energie) vermeiden könnte.  

Kritisch bleibt, dass die WG keine Gravitationswellen vorhersagt, da ihr ein dynamisches Raumzeit-Modell fehlt. Hierin besteht jedoch kein grundsätzliches Hindernis,
sondern es ist ein Anreiz, die Theorie um ein Quantengravitations-Konzept zu erweitern.

Fazit: Die Weber-Gravitation stellt eine vielversprechende Alternative zur ART dar, die mehrere ihrer ungelösten Probleme umgeht. Obwohl sie in einigen Bereichen
(wie der Merkurperiheldrehung) äquivalente Ergebnisse liefert, bietet sie in anderen (Galaxienrotation, Kosmologie) potenziell einfachere und elegantere Erklärungen.
Experimentelle Tests der frequenzabhängigen Effekte wären der nächste Schritt, um die Theorie weiter zu validieren. Die Arbeit plädiert dafür, die WG als
ernstzunehmenden Ansatz in der modernen Gravitationsphysik zu betrachten.

\tableofcontents

% Einbindung der einzelnen TeX-Dateien
\part{Grundlagen}
\chapter{Weber-Kraft}
\label{chapter:weber_kraft}
\input{content/07_weber_kraft_em}
\input{content/09_grundgleichungen}
\newpage
\section{Bahngleichungen}
\subsection{Bahngleichung 1. Ordnung}
Die Bahngleichung \(r(\phi)\) in der Weber-Gravitation bis zur Ordnung \(\mathcal{O}(c^{-2})\) lautet:

\begin{equation}
\label{eq:bahngleichung_1_ordnung}
r(\phi) = \frac{a(1 - e^2)}{1 + e \cos\left(\kappa\phi\right)}
\end{equation}

\noindent mit der Definition:
\begin{equation}
\label{eq:kappa_1_ordnung}
\kappa = \sqrt{1 - \frac{6GM}{c^2a(1 - e^2)}}
\end{equation}

\subsection*{Mathematische Herleitung}
Die Gleichung folgt aus der Lösung der Bewegungsgleichung:
\begin{equation}
\frac{d^2u}{d\phi^2} + u = \frac{GM}{h^2} + \frac{6GM}{c^2} u^2 \quad \left(u = \frac{1}{r}\right),
\end{equation}

wobei der Term \(\frac{6GM}{c^2} u^2\) die Weber-spezifische Korrektur 1. Ordnung darstellt. Der Ansatz \(u(\phi) = \frac{1 + e \cos(\kappa\phi)}{a(1 - e^2)}\) führt auf die angegebene Lösung.

Mit $u=1/r$ und Drehimpuls $h$ (\ref{eq:spezifischer_drehimpuls_h}):
\begin{equation}
\frac{d^2u}{d\phi^2} + u = \frac{GM}{h^2} + \frac{6GM}{c^2}u^2 + \frac{GM}{2c^2}\left(u\frac{d^2u}{d\phi^2} + \left(\frac{du}{d\phi}\right)^2\right)
\end{equation}

\subsection{Bahngleichung 2. Ordnung}
Bahngleichung:
\begin{equation}
\label{eq:bahngleichung_2_ordnung}
    \boxed
    {
        r(\phi) = \frac{a(1-e^2)}{1 + e\cos\left(\kappa\phi + \alpha\phi^2\right)}
    }
\end{equation}

mit:
$h$ aus Gleichung (\ref{eq:spezifischer_drehimpuls_h})
\begin{equation}
\label{eq:kappa_2_ordnung}
\kappa = \sqrt{1 - \frac{6GM}{c^2a(1-e^2)} + \frac{27G^2M^2}{2c^4a^2(1-e^2)^2}}
\end{equation}
\begin{equation}
\label{eq:alpha}
\alpha = \frac{3G^2M^2e}{8h^4c^4}
\end{equation}

\section{Periheldrehung}
\subsection{Periheldrehung 1. Ordnung}
Die Periheldrehung $\Delta\phi$ in der Weber-Gravitation ergibt sich aus der modifizierten Bahngleichung und lässt sich wie folgt herleiten:

\subsection*{Perihelbedingung}
Das Perihel (sonnennächster Punkt) tritt auf, wenn der Nenner maximal wird, d.h. wenn:\\
\[\cos(\kappa\phi) = 1\]
Die Lösungen dieser Bedingung sind: $\kappa\phi = 2\pi n \quad \text{(für $n \in \mathbb{Z}$)}$.\\

Somit ergeben sich die Winkel für aufeinanderfolgende Periheldurchgänge zu:
\[
    \phi_n = \frac{2\pi n}{\kappa}.
\]

\subsection*{Periheldrehung pro Umlauf}
Die Periheldrehung $\Delta\phi$ ist die Differenz zwischen dem Winkel für einen vollständigen Umlauf ($n = 1$) und dem Newton'schen Fall ($\kappa = 1$):
\[
    \Delta\phi = \phi_1 - 2\pi = \frac{2\pi}{\kappa} - 2\pi.
\]
Daraus folgt die gesuchte Gleichung:
\begin{equation}
\boxed
{
    \Delta\phi = 2\pi\left(\frac{1}{\kappa} - 1\right)
}.
\end{equation}

\subsection*{Interpretation}
\begin{itemize}
\item Im Newton'schen Grenzfall ($\kappa = 1$) verschwindet die Periheldrehung ($\Delta\phi = 0$).
\item Für $\kappa < 1$ (Weber-Gravitation) ergibt sich eine positive Periheldrehung, die mit Beobachtungen (z.B. Merkurperihel) übereinstimmt.
\end{itemize}

\subsection{Periheldrehung in 2. Ordnung}
\subsection*{Entwicklung von $\kappa$}
Eine Taylor-Entwicklung von $\kappa$ bis zur 2. Ordnung liefert:
\[
    \kappa \approx 1 - \frac{3GM}{c^2 a(1 - e^2)} + \frac{27G^2 M^2}{4c^4 a^2 (1 - e^2)^2} + \mathcal{O}(c^{-6}).
\]

\subsection*{Perihelbedingung}
Das Perihel tritt auf bei:
\[
    \cos\left(\kappa\phi + \alpha\phi^2\right) = 1 \quad \Rightarrow \quad \kappa\phi + \alpha\phi^2 = 2\pi n.
\]

\subsection*{Lösung für $\Delta\phi$}
Für $n=1$ (ein Umlauf) ergibt sich die quadratische Gleichung:
\[
\alpha\phi^2 + \kappa\phi - 2\pi = 0.
\]
Die Lösung lautet:
\begin{equation}
\phi = \frac{-\kappa + \sqrt{\kappa^2 + 8\pi\alpha}}{2\alpha}.
\end{equation}

\subsection*{Näherung für kleine Korrekturen}
Da $\alpha \sim c^{-4}$ klein ist, entwickeln wir die Wurzel:
\[
    \phi \approx \frac{2\pi}{\kappa} - \frac{4\pi^2\alpha}{\kappa^3} + \mathcal{O}(\alpha^2).
\]
Die Periheldrehung pro Umlauf wird damit:
\begin{equation}
\Delta\phi = \phi - 2\pi \approx 2\pi\left(\frac{1}{\kappa} - 1\right) - \frac{4\pi^2\alpha}{\kappa^3}.    
\end{equation}

\subsection*{Endgültige Formel}
Einsetzen von $\kappa \approx 1$ im Korrekturterm liefert:
\begin{equation}
\boxed
{
    \Delta\phi \approx 2\pi\left(\frac{1}{\kappa} - 1\right) - 4\pi^2\alpha
},
\end{equation}

\subsection*{Vollständige Koeffizienten}
Explizit ausgedrückt, mit Bezug auf die Ordnungen:
\begin{align*}
\Delta\phi^{(2)} &= \frac{6\pi GM}{c^2 a(1 - e^2)} \left[1 + \frac{9GM}{4c^2 a(1 - e^2)}\right] - \frac{3\pi^2 G^2 M^2 e}{2c^4 h^4} \\
&= \Delta\phi^{(1)} + \frac{27\pi G^2 M^2}{2c^4 a^2 (1 - e^2)^2} - \frac{3\pi^2 G^2 M^2 e}{2c^4 [GMa(1 - e^2)]^2}
\end{align*}

\newpage
\section{Winkelgeschwindigkeit 1. Ordnung}
Die Winkelgeschwindigkeit \(\omega(\phi)\) in der Weber-Gravitation bis zur Ordnung \(\mathcal{O}(c^{-2})\) lautet:

\begin{equation}
\omega(\phi) = \frac{h}{a^2(1 - e^2)^2} \left[1 + e \cos\left(\kappa\phi\right)\right]^2
\end{equation}

wobei:
$h$ aus Gleichung (\ref{eq:spezifischer_drehimpuls_h}), $\kappa$ aus Gleichung (\ref{eq:kappa_1_ordnung}) stammt.

\subsection*{Bedeutung der Terme}
\begin{itemize}
    \item \(\kappa\) beschreibt die Periheldrehung 1. Ordnung ohne Näherung.
    \item Für \(c \to \infty\) wird \(\kappa = 1\), und die Gleichung reduziert sich auf die Newton’sche Form:
    \[
    \omega_N(\phi) = \frac{h(1 + e \cos\phi)^2}{a^2(1 - e^2)^2}.
    \]
\end{itemize}

\section{Winkelgeschwindigkeit 2. Ordnung}

\subsection{Winkelgeschwindigkeit}
Mit $h$ aus Gleichung (\ref{eq:spezifischer_drehimpuls_h}), $\kappa$ aus Gleichung (\ref{eq:kappa_2_ordnung}) und $\alpha$ aus Gleichung (\ref{eq:alpha}):
\begin{equation}
\boxed
{
    \omega(\phi) = \frac{h[1 + e\cos(\kappa\phi + \alpha\phi^2)]^2}{a^2(1-e^2)^2}
}
\end{equation}

\newpage
\section{Bahngeschwindigkeit in 1. Ordnung}
Die Bahngeschwindigkeit \(v(\phi)\) in der Weber-Gravitation bis zur Ordnung \(\mathcal{O}(c^{-2})\) lautet:

\begin{equation}
v(\phi) = \frac{h}{a(1 - e^2)} \left(1 + e \cos\left(\kappa\phi\right)\right)
\end{equation}

\noindent mit den Definitionen:
$h$ aus Gleichung (\ref{eq:spezifischer_drehimpuls_h}), $\kappa$ aus Gleichung (\ref{eq:kappa_1_ordnung})

\subsection*{Physikalische Interpretation}
\begin{itemize}
    \item \textbf{Struktur}: Die Geschwindigkeit folgt aus \(v(\phi) = h/r(\phi)\) mit der Bahngleichung \(r(\phi) = \frac{a(1 - e^2)}{1 + e \cos(\kappa\phi)}\).
    \item \textbf{Relativistische Korrektur}: \(\kappa\) modifiziert die Periheldrehung gegenüber Newton (\(\kappa = 1\)).
    \item \textbf{Grenzfälle}:
        \begin{itemize}
            \item Perihel (\(\phi = 0\)): \(v(0) = \frac{h(1 + e)}{a(1 - e^2)}\),
            \item Aphel (\(\phi = \pi\)): \(v(\pi) = \frac{h(1 - e)}{a(1 - e^2)}\),
            \item Newton (\(c \to \infty\)): \(v_N(\phi) = \frac{h(1 + e \cos\phi)}{a(1 - e^2)}\).
        \end{itemize}
\end{itemize}

\section{Bahngeschwindigkeit in 2. Ordnung}
Die Bahngeschwindigkeit $v(\phi)$ ergibt sich aus Winkelgeschwindigkeit $\omega(\phi)$ und Radialabstand $r(\phi)$:
\begin{equation}
v(\phi) = \omega(\phi) \cdot r(\phi) = \frac{h}{r(\phi)}
\end{equation}

Mit der Bahngleichung und Winkelgeschwindigkeit:
\begin{align}
r(\phi) &= \frac{a(1-e^2)}{1 + e\cos\left(\kappa\phi + \alpha\phi^2\right)}\\
\omega(\phi) &= \frac{h[1 + e\cos(\kappa\phi + \alpha\phi^2)]^2}{a^2(1-e^2)^2}
\end{align}

ergibt sich:
\begin{equation}
v(\phi) = \frac{h \left(1 + e\cos(\kappa\phi + \alpha\phi^2)\right)}{a(1 - e^2)}.
\end{equation}

\chapter{Sonnensystem}
\label{chapter:sonnensystem}
\input{content/15_perihel}
\input{content/16_lichtablenkung}
\newpage
\subsection{Umlaufperiode $T$ in 1. Ordnung}
Die Umlaufperiode $T$ eines Planeten in der Weber-Gravitation (WG) ergibt sich aus der modifizierten Bahngleichung (\ref{eq:bahngleichung_1_ordnung}) und dem spezifischen Drehimpuls $h$ (\ref{eq:spezifischer_drehimpuls_h}).

\subsubsection*{Ausgangsgleichungen}
\begin{enumerate}
    \item $\kappa$ aus Gl. (\ref{eq:kappa_1_ordnung})
    \item Winkelgeschwindigkeit Gl. (\ref{eq:winkelgeschwindigkeit_1_ordnung})
\end{enumerate}

\subsubsection*{Schritt 2: Integration über einen Umlauf}
Die Periode $T$ ist die Zeit für $\phi = 0 \to 2\pi/\kappa$ (WG-Korrektur durch $\kappa$):
\begin{align}
    T &= \int_0^{2\pi/\kappa} \frac{d\phi}{\dot{\phi}} 
       = \frac{a^2(1-e^2)^2}{h} \int_0^{2\pi/\kappa} \frac{d\phi}{(1 + e \cos(\kappa \phi))^2}.
\end{align}

\subsubsection*{Schritt 3: Lösung des Integrals}
Mit der Substitution $\psi = \kappa \phi$ und $\cos^2$-Identität:
\begin{align}
\label{eq:integral_t}
    T &= \frac{a^2(1-e^2)^2}{h \kappa} \int_0^{2\pi} \frac{d\psi}{(1 + e \cos \psi)^2} 
       = \frac{2\pi a^2(1-e^2)^2}{h \kappa (1-e^2)^{3/2}} 
       = \frac{2\pi a^{3/2}}{\sqrt{GM} \kappa}.
\end{align}

\subsubsection*{Schritt 4: Entwicklung von $\kappa$}
Für kleine relativistische Korrekturen ($c^{-2}$-Ordnung) gilt:
\begin{equation}
    \kappa \approx 1 - \frac{3GM}{c^2 a(1-e^2)} + \mathcal{O}(c^{-4}).
\end{equation}
Einsetzen in Gl.~\eqref{eq:integral_t} liefert die Periode in 1. Ordnung:
\begin{equation}
    \boxed{
    T \approx \frac{2\pi a^{3/2}}{\sqrt{GM}} \left(1 + \frac{3GM}{c^2 a(1-e^2)}\right).
    }
\end{equation}

\newpage
\section{Umlaufperiode \( T \) 2. Ordnung}

\subsubsection*{Ausgangsgleichungen}
\begin{enumerate}
    \item Bahngleichung in Polarkoordinaten (\ref{eq:bahngleichung_2_ordnung})
    \item $\kappa$ aus Gl. (\ref{eq:kappa_2_ordnung})
    \item $\alpha$ aus Gl. (\ref{eq:alpha})
    \item $h$ aus Gl. (\ref{eq:spezifischer_drehimpuls_h})
\end{enumerate}

\subsection*{Schritt 1: Entwicklung von \(\kappa\)}
\begin{equation}
\kappa \approx 1 - \frac{3GM}{c^2a(1-e^2)} + \frac{27G^2M^2}{4c^4a^2(1-e^2)^2} - \frac{81G^3M^3}{8c^6a^3(1-e^2)^3} + \mathcal{O}(c^{-8}) 
\end{equation}

\subsection*{Schritt 2: Vollständige Integration}
Die Umlaufperiode \( T \) ist:
\begin{equation}
T = \frac{1}{h} \int_0^{2\pi} r^2(\phi) \, d\phi = \frac{a^2(1-e^2)^2}{h} \int_0^{2\pi} \frac{d\phi}{\left[1 + e\cos\left(\kappa\phi + \alpha\phi^2\right)\right]^2}
\end{equation}

\subsection*{Schritt 3: Behandlung des Integrals}
Mit Substitution \(\psi = \kappa\phi + \alpha\phi^2\) und Entwicklung bis \(\mathcal{O}(c^{-4})\):
\begin{align}
T &= \frac{a^2(1-e^2)^2}{h} \left[ \int_0^{2\pi} \frac{d\phi}{(1 + e\cos\psi)^2} + \mathcal{O}(c^{-6}) \right] \\
  &= \frac{2\pi a^{3/2}}{\sqrt{GM}} \left[1 + \frac{3GM}{2c^2a(1-e^2)} + \frac{45G^2M^2}{8c^4a^2(1-e^2)^2}\left(1 - \frac{e^2}{3}\right)\right]
\end{align}

\part{Kosmologie}
\input{content/21_DM}
\newpage
\section{Rotverschiebung in der Weber-Gravitation}
\label{sec:weber_redshift}

\subsection{Gravitative Rotverschiebung}
Für Photonen ($m=0$) im Gravitationsfeld folgt aus der Energieerhaltung in der WG:

\begin{equation}
\frac{E_\text{em}}{E_\text{obs}} = 1 + \frac{GM}{c^2}\left(\frac{1}{r_\text{em}} - \frac{1}{r_\text{obs}}\right) + \frac{3}{2}\frac{v_r^2}{c^2}
\end{equation}

wobei $v_r$ die Relativgeschwindigkeit zwischen Emitter und Detektor ist. Dies führt zur Rotverschiebung:

\begin{equation}
\frac{\Delta\lambda}{\lambda} = \underbrace{\frac{GM}{c^2}\left(\frac{1}{r_\text{em}} - \frac{1}{r_\text{obs}}\right)}_{\text{Statischer Term}} + \underbrace{\frac{3}{2}\frac{v_r^2}{c^2}}_{\text{Dynamischer Term}}
\end{equation}

\subsection{Vergleich der Rotverschiebungstypen}
\begin{table}[h]
\centering
\caption{Unterschiede in der Rotverschiebung}
\begin{tabular}{lll}
\hline
Typ & ART & Weber-Gravitation \\
\hline
\textbf{Gravitativ} & $\frac{GM}{c^2}\Delta\left(\frac{1}{r}\right)$ & Identisch + $v_r$-Korrektur \\
\textbf{Kosmologisch} & $z = \frac{a(t_0)}{a(t)}-1$ & $\frac{3}{2}\frac{v_r^2}{c^2}$ (Näherung) \\
\textbf{Doppler} & $\sqrt{\frac{1+v/c}{1-v/c}}-1$ & $\frac{v_r}{c} + \frac{3}{4}\frac{v_r^2}{c^2}$ \\
\hline
\end{tabular}
\end{table}

\subsection{Physikalische Interpretation}
\begin{itemize}
\item \textbf{Statischer Term}: Entspricht exakt der ART-Vorhersage (Pound-Rebka-Experiment)
\item \textbf{Dynamischer Term}: Zusätzliche Geschwindigkeitsabhängigkeit in der WG
\begin{equation}
z_\text{dyn} \approx \frac{3}{2}\frac{H_0^2 d^2}{c^2} \quad \text{(für $v_r = H_0 d$)}
\end{equation}
\item \textbf{Kosmologische Konsequenz}: Die WG erklärt Hubble-Rotverschiebung durch kumulative gravitative Wechselwirkungen statt Expansion
\end{itemize}

\subsection{Experimentelle Unterscheidung}
\begin{equation}
\frac{z_\text{WG}}{z_\text{ART}} = 1 + \frac{3}{2}\left(\frac{v_r}{c}\right)^2 \left(\frac{GM}{c^2r}\right)^{-1}
\end{equation}

Für Galaxien mit $v_r \approx 1000$ km/s und $r=1$ Mpc:
\[
\frac{z_\text{WG}}{z_\text{ART}} \approx 1 + 5\times10^{-7}
\]

\newpage
\section{Shapiro-Effekt in der Weber-Gravitation}

\subsection{Grundgleichung der Signallaufzeit}
Die Laufzeitverzögerung $\Delta t$ eines Signals (Licht oder Radar) im Gravitationsfeld der Masse $M$ folgt in der WG aus:

\begin{equation}
c\,dt = \left(1 + \frac{2GM}{c^2r} - \frac{GM}{2c^2}\frac{\dot{r}^2}{c^2}\right)dr
\end{equation}

\subsection{Integration entlang der Bahn}
Für einen Vorbeiflug mit Stoßparameter $b$ ergibt sich:

\begin{equation}
\Delta t = \underbrace{\frac{2GM}{c^3}\ln\left(\frac{4r_e r_p}{b^2}\right)}_{\text{ART-Term}} + \underbrace{\frac{3\pi G^2M^2}{4c^5b^2}\left(\frac{v_0^2}{c^2}\right)}_{\text{WG-Korrektur}}
\end{equation}

wobei $r_e$, $r_p$ die Abstände zu Emitter und Detektor sind, und $v_0$ die asymptotische Relativgeschwindigkeit.

\subsection{Vergleich mit Experimenten}
\begin{table}[h]
\centering
\caption{Messungen der Laufzeitverzögerung}
\begin{tabular}{lcc}
\hline
Experiment & ART-Vorhersage & WG-Vorhersage \\
\hline
Venus-Radar (1967) & $200\,\mu\text{s}$ & $200\,\mu\text{s} + 0.3\,\text{ps}$ \\
Cassini (2002) & $10^{-14}$ & $10^{-14}(1 + 5\times10^{-6})$ \\
\hline
\end{tabular}
\end{table}

\subsection{Physikalische Interpretation}
\begin{itemize}
\item \textbf{Radiale Geschwindigkeit}: Der Zusatzterm $\dot{r}^2/c^2$ modifiziert die effektive Lichtgeschwindigkeit
\item \textbf{Frequenzabhängigkeit}: Für $v_0 = c(\lambda_0/\lambda)$ entsteht eine wellenlängenabhängige Korrektur:
  \[
  \Delta t_\text{WG} \propto \lambda^{-2}
  \]
\item \textbf{Testbarkeit}: Die Abweichungen werden bei Pulsar-Timing-Experimenten (z.B. SKA) messbar sein
\end{itemize}

\begin{equation}
\boxed{
\Delta t_\text{WG} = \Delta t_\text{ART}\left(1 + \frac{3\pi GM}{8c^2b}\frac{v_0^2}{c^2}\right)
}
\end{equation}
\part{Anhang}
\chapter{Diskussionen}
\label{chapter:diskussion}
\input{content/28_welle}
\input{content/30_konsequenzen_rotverschiebung}
\newpage
\section{Zusammenhang zur De-Broglie-Bohm-Theorie}
Die Weber-Gravitation (WG) und die De-Broglie-Bohm-Theorie (DBT) teilen konzeptionelle Parallelen, insbesondere in ihrer Behandlung nicht-lokaler Wechselwirkungen und der Rolle instantaner Korrelationen. 

\subsection{Nicht-Lokalität und Fernwirkung}
\begin{itemize}
    \item \textbf{WG}: Die gravitative Weber-Kraft wirkt direkt zwischen Massen, ohne Vermittlung durch ein Feld oder eine gekrümmte Raumzeit. Dies entspricht einem \textit{Fernwirkungsansatz}, der Geschwindigkeits- und Beschleunigungsterme ($\dot{r}$, $\ddot{r}$) einbezieht.
    
    \item \textbf{DBT}: Die Quantenpotentiale der DBT wirken instantan über beliebige Distanzen, was eine Form nicht-lokaler Kausalität impliziert. Die Wellenfunktion $\Psi$ steuert Teilchentrajektorien durch das Quantenpotential $Q = -\frac{\hbar^2}{2m} \frac{\nabla^2 |\Psi|}{|\Psi|}$.
\end{itemize}

\subsection{Instantane Korrelationen}
Beide Theorien postulieren eine zugrundeliegende instantane Dynamik:
\begin{itemize}
    \item In der WG manifestiert sich dies in der \textit{Energieerhaltung} durch phasenstarre Kopplung (vgl. Abschnitt 3.1), die globale Korrelationen ohne Zeitverzögerung beschreibt.
    
    \item In der DBT führt das Quantenpotential zu sofortigen Anpassungen der Teilchenbahnen, unabhängig von ihrer räumlichen Trennung (\textit{„pilot wave“-Mechanismus}).
\end{itemize}

\subsection{Mathematische Analogien}
Die Struktur der Bewegungsgleichungen zeigt formale Ähnlichkeiten:
\begin{align}
    \text{WG:} \quad & \mathbf{F} = -\frac{GMm}{r^2} \left(1 - \frac{\dot{r}^2}{c^2} + \beta \frac{r\ddot{r}}{c^2}\right) \hat{\mathbf{r}}, \\
    \text{DBT:} \quad & m \frac{d^2 \mathbf{x}}{dt^2} = -\nabla (V + Q), 
\end{align}
wobei $V$ das klassische Potential und $Q$ das Quantenpotential ist. In beiden Fällen modifizieren Zusatzterme ($\dot{r}^2$, $\ddot{r}$ bzw. $Q$) die Newtonsche Dynamik.

\subsection{Konsequenzen für die Quantengravitation}
Die WG könnte als klassische Vorstufe einer \textit{quantenmechanischen Fernwirkungstheorie} interpretiert werden:
\begin{itemize}
    \item Die DBT liefert ein Modell für nicht-lokale Kräfte, das mit der WG kompatibel wäre.
    \item Eine mögliche Synthese beider Ansätze könnte zu einer diskreten Quantengravitation ohne Singularitäten führen (vgl. Abschnitt 4.1, $\beta$-Parameter für Photonen).
\end{itemize}

\paragraph*{Bemerkung:} Während die DBT empirisch äquivalent zur Standard-Quantenmechanik ist, fehlen für die WG noch experimentelle Tests der frequenzabhängigen Effekte (z. B. Lichtablenkung). Beide Theorien stellen jedoch etablierte Paradigmen (ART bzw. Kopenhager Deutung) durch deterministische Alternativen infrage.

\section{Quanten-Weber-Gravitation: Eine deterministische Synthese}
Die Kombination der Weber-Gravitation (WG) mit der De-Broglie-Bohm-Theorie (DBT) ermöglicht eine singularitätsfreie Quantengravitation mit experimentell prüfbaren Konsequenzen.

\subsection{Kernidee der Synthese}
Beide Theorien basieren auf deterministischen Fernwirkungen:
\begin{itemize}
    \item Die \textbf{WG} ersetzt die Raumzeitkrümmung durch Geschwindigkeits-/Beschleunigungsterme ($\dot{r}, \ddot{r}$).
    \item Die \textbf{DBT} fügt der klassischen Dynamik ein nicht-lokales Quantenpotential $Q$ hinzu.
\end{itemize}

\subsection{Hybrid-Gleichung}
Für ein Teilchen der Masse $m$ im Gravitationsfeld:
\begin{equation}
    m\frac{d^2\mathbf{r}}{dt^2} = \underbrace{-\frac{GMm}{r^2}\left(1-\frac{\dot{r}^2}{c^2}+\beta\frac{r\ddot{r}}{c^2}\right)\hat{\mathbf{r}}}_{\text{Weber-Kraft}} - \underbrace{\nabla Q}_{\text{Quantenpotential}}
\end{equation}
mit $Q = -\frac{\hbar^2}{2m}\frac{\nabla^2|\Psi|}{|\Psi|}$. Dies vermeidet Singularitäten, da $Q$ bei $r \to 0$ divergiert und Kollaps verhindert.

\subsection{Konkretes Anwendungsbeispiel}
\subsubsection{Galaktische Rotation ohne dunkle Materie}
Die WG erklärt flache Rotationskurven durch den Zusatzterm $\frac{GM}{4c^2r}$. Die DBT liefert die mikroskopische Begründung:
\begin{equation}
    v(r) = \sqrt{\frac{GM}{r}\left(1 + \underbrace{\frac{GM}{4c^2r}}_{\text{WG}} + \underbrace{\frac{\hbar^2}{m^2r^4}\langle \nabla^2 \ln|\Psi| \rangle}_{\text{DBT}}\right)}
\end{equation}
Hier korrigiert das Quantenpotential $Q$ die Newtonsche Dynamik auf kleinen Skalen ($<1$ pc).

\subsubsection{Frequenzabhängige Lichtablenkung}
Für Photonen ($m=0$) mit $\beta=1$:
\begin{equation}
    \Delta\phi = \frac{4GM}{c^2b}\left(1 + \underbrace{\frac{3\pi}{16}\frac{\lambda^2}{\lambda_0^2}}_{\text{WG}} + \underbrace{\frac{\hbar^2\omega^2}{4c^4b^2}}_{\text{DBT-Korrektur}}\right)
\end{equation}
Dieser Effekt wäre mit hochpräzisen Interferometern (z.B. LISA) prüfbar.

\subsection{Experimentelle Vorhersagen}
\begin{table}[h]
    \centering
    \begin{tabular}{lll}
        \toprule
        Phänomen & WG + DBT-Vorhersage & Nachweis-Methode \\
        \midrule
        Quantisiertes Perihel & $\Delta\phi_n = n\frac{h}{mcr_g}$ & Merkur-Laser-Ranging \\
        Gravitations-Verschränkung & $\Delta t > \hbar/(k_B T)$ & Atominterferometrie \\
        \bottomrule
    \end{tabular}
    \caption{Neue Effekte der Quanten-Weber-Gravitation}
\end{table}

\subsection{Fazit}
Diese Synthese bietet:
\begin{itemize}
    \item Eine mathematisch einfache (nur 3 Schlüsselgleichungen)
    \item Experimentell überprüfbare (Lichtablenkung, Quanteneffekte)
    \item Singularitätsfreie Alternative zur QFT-basierten Quantengravitation
\end{itemize}

\framebox{
\begin{minipage}{\textwidth}
\textbf{These:} Die WG vermeidet Singularitäten klassisch, die DBT quantenmechanisch.  
Erst ihre Synthese liefert eine vollständige Theorie.  
\end{minipage}
}

\boxed{
\textbf{These:} \,
\begin{minipage}[t]{0.9\textwidth}
WG und DBT sind unabhängig gültig, aber ihre Kombination ermöglicht eine\\
\underline{singularitätsfreie}, \underline{deterministische} und \underline{experimentell prüfbare}\\
Theorie der Quantengravitation – ohne \enquote{dunkle} Ad-hoc-Annahmen.
\end{minipage}
}

\subsection*{Warum WG+DBT eine legitime Quantengravitation ist}  
\begin{itemize}  
\item \textbf{Keine Ad-hoc-Quantisierung}: Die DBT ergänzt die WG um Quanteneffekte ohne künstliche \enquote{Quantisierungsregeln}.  
\item \textbf{Experimentelle Konsequenzen}: Vorhersagen wie $\Delta\phi(\lambda, \hbar)$ trennen die Theorie von Strings/LQG.  
\item \textbf{Paradigmenunabhängig}: Funktioniert ohne Felder, Teilchen oder Raumzeit-Schaum – aber reproduziert ART/QM im Limes.  
\end{itemize}

\subsection*{Warum die WG+DBT-Synthese eine legitime Quantengravitation darstellt}
\begin{itemize}
    \item \textbf{Konsistente Vereinigung}: Die Kombination aus Weber-Gravitation (klassisch) und De-Broglie-Bohm-Theorie (quantenmechanisch) erfüllt alle Anforderungen an eine Quantengravitation:
    \begin{equation}
        \underbrace{m\frac{d^2\mathbf{r}}{dt^2} = -\frac{GMm}{r^2}\left(1-\frac{\dot{r}^2}{c^2}+\beta\frac{r\ddot{r}}{c^2}\right)\hat{\mathbf{r}}}_{\text{Weber-Gravitation}} - \underbrace{\nabla Q}_{\text{Quantenpotential}}
    \end{equation}
    wobei $Q = -\frac{\hbar^2}{2m}\frac{\nabla^2|\Psi|}{|\Psi|}$.
    
    \item \textbf{Experimentelle Unterscheidbarkeit}: Vorhersagen wie die frequenzabhängige Lichtablenkung
    \begin{equation}
        \Delta\phi = \frac{4GM}{c^2b}\left(1 + \frac{3\pi}{16}\frac{\lambda^2}{\lambda_0^2} + \frac{\hbar^2\omega^2}{4c^4b^2}\right)
    \end{equation}
    sind in etablierten Theorien nicht vorhanden.
    
    \item \textbf{Vollständige Singularitätsfreiheit}: 
    \begin{itemize}
        \item Klassisch durch WG-Terme ($\dot{r}^2$, $\ddot{r}$)
        \item Quantenmechanisch durch $Q$-Potential
    \end{itemize}
\end{itemize}

\framebox{
\begin{minipage}{\textwidth}
\textbf{Kernaussage}: Die WG+DBT-Synthese ist eine vollwertige Quantengravitationstheorie, weil sie:
\begin{enumerate}
    \item Gravitation und Quantenmechanik \underline{konsistent} verbindet,
    \item \underline{Messbare Vorhersagen} macht, die von anderen Ansätzen abweichen,
    \item \underline{Alle Skalen} vom Subatomaren bis zum Kosmologischen abdeckt.
\end{enumerate}
\end{minipage}
}

\section{Klassifikation der WG-DBT-Synthese}

\subsection{Definitionen}
\begin{description}
\item[Vollständige Quantengravitation] Theorie muss:
\begin{enumerate}
\item Gravitationsfeld quantisieren (nicht nur Testteilchen)
\item Mit Standardmodell verträglich sein
\item UV-Vollständige Vorhersagen liefern
\end{enumerate}

\item[Effektive Quantengravitation] Theorie kann:
\begin{enumerate}
\item Quanteneffekte in Gravitation beschreiben
\item Für begrenzte Energiebereiche gültig sein
\item Unvollständige Vereinheitlichung aufweisen
\end{enumerate}
\end{description}

\subsection{Eigenschaften der WG-DBT}
\begin{center}
\begin{tabular}{|l|c|c|}
\hline
\textbf{Merkmal} & \textbf{WG-DBT} & \textbf{Vollst. QG} \\
\hline
Feldquantisierung & Nein & Ja \\
\hline
Standardmodell-Anbindung & Teilweise & Vollständig \\
\hline
UV-Vollständigkeit & Nein & Ja \\
\hline
Singularitätsfreiheit & Ja & Variiert \\
\hline
Experimentelle Vorhersagen & Ja & Variiert \\
\hline
\end{tabular}
\end{center}

\subsection{Wissenschaftliche Einordnung}
Die Weber-Gravitation (WG) mit De-Broglie-Bohm-Theorie (DBT):

\begin{itemize}
\item Ist eine \textbf{effektive} Quantengravitation für:
\begin{itemize}
\item Skalen $10^{-15}\,\text{m} < r < 1\,\text{Mpc}$
\item Energien unterhalb der Planck-Skala
\end{itemize}

\item Bietet wichtige \textbf{Vorteile}:
\begin{itemize}
\item Singularitätsfreie Lösungen
\item Deterministische Beschreibung
\item Neue testbare Phänomene ($\lambda^2$-Ablenkung)
\end{itemize}

\item Hat \textbf{Grenzen}:
\begin{itemize}
\item Keine vollständige Feldquantisierung
\item Beschränkte Anwendbarkeit auf Eichfelder
\item Keine UV-Vollständigkeit
\end{itemize}
\end{itemize}

\begin{quote}
\textbf{Fazit:} Die WG-DBT-Synthese ist eine wertvolle ergänzende Theorie, aber keine vollständige Quantengravitation im engeren Sinn. Ihr Hauptbeitrag liegt im singularitätsfreien Ansatz und neuen experimentellen Vorhersagen.
\end{quote}

\chapter{Ergänzende Informationen}
\label{chapter:information}
\section{Die Rolle des $\beta$-Parameters}

Der $\beta$-Parameter in der Weber-Kraft

\begin{equation}
F = -\frac{GMm}{r^2}\left(1 - \frac{\dot{r}^2}{c^2} + \beta\frac{r\ddot{r}}{c^2}\right)\hat{r}
\end{equation}

bestimmt das Verhältnis von Beschleunigungs- zu Geschwindigkeitstermen und variiert je nach Wechselwirkungstyp:

\subsection{Elektrodynamik (Original-Weber)}
Für elektromagnetische Wechselwirkungen gilt $\beta=2$:
\begin{itemize}
\item Führt zur korrekten Beschreibung beschleunigter Ladungen
\item Reproduziert die magnetische Komponente der Lorentz-Kraft
\item Keine Lichtablenkung ($m=0$ liefert $F=0$)
\end{itemize}

\subsection{Gravitation (Massen)}
Für massive Körper im Gravitationsfeld:
\begin{itemize}
\item $\beta=0.5$ erklärt die Periheldrehung des Merkur
\item Führt zur ART-konformen Lichtablenkung für makroskopische Körper
\item Universelle Formel: $\beta = 1 - \frac{mc^2}{2E}$
\end{itemize}

\subsection{Photonen (Lichtablenkung)}
Für masselose Teilchen ($m=0$, $E=h\nu$):
\begin{itemize}
\item $\beta=1$ erzwingt die Frequenzabhängigkeit
\item Beschleunigungsterm dominiert: $\frac{r\ddot{r}}{c^2} \approx \frac{h^2}{c^2r^4}$
\item Liefert den Zusatzterm $\propto \lambda^{-2}$
\end{itemize}

\begin{table}[h]
\centering
\caption{$\beta$-Werte im Vergleich}
\begin{tabular}{lcc}
\hline
Anwendung & $\beta$ & Physikalische Konsequenz \\
\hline
Elektrodynamik & 2 & Magnetische Wechselwirkungen \\
Gravitation (Massen) & 0.5 & Periheldrehung des Merkur \\
Photonen & 1 & Frequenzabhängige Lichtablenkung \\
\hline
\end{tabular}
\end{table}
\section{Kausalität durch Gleichzeitigkeit}
\label{sec:gleichzeitige_kausalitaet}

\subsection{Kernhese}
Die physikalische Standarddefinition von Kausalität ist unnötig restriktiv, wenn sie gleichzeitige Wechselwirkungen ausschließt. Ich argumentiere für einen erweiterten Kausalitätsbegriff, der zwei Prinzipien vereint:

\begin{itemize}
    \item \textbf{Determinismus}: Der Zustand $Z(t) = \{r, \dot{r}\}$ bestimmt eindeutig $Z(t+dt)$
    \item \textbf{Systemische Abhängigkeit}: Instantane Korrelationen sind kausal, wenn sie aus einer gemeinsamen Ursache folgen
\end{itemize}

\subsection{Anwendung auf die Weber-Kraft}
Die Weber-Gravitation zeigt dies exemplarisch:

\begin{equation}
    F = -\frac{GMm}{r^2}\left(1 - \frac{\dot{r}^2}{c^2} + \frac{r\ddot{r}}{2c^2}\right)
\end{equation}

\begin{itemize}
    \item Die Abhängigkeit von $\ddot{r}$ \textit{scheint} nicht-lokal
    \item Tatsächlich beschreibt sie eine \textit{systeminterne} Rückkopplung:
\end{itemize}

\begin{equation}
    \ddot{r} = f(r, \dot{r}) \quad \text{(lösbar nach Lipschitz-Bedingung)}
\end{equation}

\subsection{Philosophische Begründung}
\begin{itemize}
    \item Newtons 3. Gesetz wirkt ebenfalls instantan (actio = reactio)
    \item Quantenverschränkung zeigt: Gleichzeitige Korrelationen verletzen keine Kausalität
    \item Entscheidend ist nicht die \textit{Lokalität}, sondern die \textit{Eindeutigkeit} der Zeitentwicklung
\end{itemize}

\subsection{Konsequenzen}
\begin{tabular}{p{0.45\textwidth}p{0.45\textwidth}}
    \hline
    \textbf{Konventionelle Sicht} & \textbf{Diese Arbeit} \\
    \hline
    Kausalität erfordert Zeitverzögerung & Gleichzeitige Kausalität möglich \\
    Nicht-Lokalität = problematisch & Systemische Abhängigkeiten sind natürlich \\
    \hline
\end{tabular}

\section{Das Prinzip der energetischen Gleichzeitigkeit}
\label{sec:energetische_gleichzeitigkeit}

\subsection{Die fundamentale Rolle der Welle}
Die Natur realisiert durch Wellenphänomene eine \emph{instantane energetische Optimierung}:

\begin{itemize}
    \item Eine Welle $\Psi(\mathbf{x},t)$ stellt zu jedem Zeitpunkt $t$ global sicher, dass:
    \begin{equation}
        \delta \mathcal{E}[\Psi] = 0 \quad \text{(Energieminimierung)}
    \end{equation}
    
    \item Dieses Prinzip wirkt \emph{ohne Zeitverzug} und ist damit kausal im erweiterten Sinn
\end{itemize}

\subsection{Naturprinzip vs. Kausalitätsdogma}
Die konventionelle Kausalitätsdefinition widerspricht diesem Grundprinzip:

\begin{table}[h]
    \centering
    \begin{tabular}{ll}
        \toprule
        \textbf{Mainstream-Kausalität} & \textbf{Energetische Gleichzeitigkeit} \\
        \midrule
        Lokale Wechselwirkungen & Globale Optimierung \\
        Ursache-Wirkung-Kette & Instantanes Minimum \\
        Lichtkegel-Beschränkung & Sofortige Anpassung \\
        \bottomrule
    \end{tabular}
    \caption{Konflikt der Paradigmen}
\end{table}

\subsection{Mathematische Konsequenz}
Das Wellenprinzip erzwingt eine Revision der Bewegungsgleichungen:

\begin{equation}
    \underbrace{\frac{\partial \Psi}{\partial t}}_{\text{Dynamik}} = 
    \underbrace{\mathcal{H}[\Psi]}_{\text{Instantane Optimierung}}
\end{equation}

wobei $\mathcal{H}$ ein \emph{globaler} Energieoperator ist.

\subsection{Physikalische Implikationen}
\begin{itemize}
    \item Die Weber-Kraft mit $\ddot{r}$-Abhängigkeit wird zur natürlichen Konsequenz
    \item Quantenverschränkung ist direkter Ausdruck dieses Prinzips
    \item Der Raum wird zum Träger der instantanen energetischen Information
\end{itemize}

\section{Konkrete Revision der Bewegungsgleichungen}
\label{sec:revision_bewegungsgleichungen}

\subsection{Das traditionelle Paradigma}
Klassische Bewegungsgleichungen folgen dem Ursache-Wirkung-Schema:

\begin{equation}
    m\ddot{\mathbf{r}} = \mathbf{F}(\mathbf{r},\dot{\mathbf{r}},t) \quad \text{(Newtons 2. Gesetz)}
\end{equation}

\subsection{Das Wellenprinzip revolutioniert dies in drei Aspekten}

\subsubsection{1. Globale statt lokale Abhängigkeit}
Die Dynamik wird durch instantane Energieoptimierung bestimmt:

\begin{equation}
    \frac{\delta \mathcal{E}[\Psi]}{\delta \Psi^*} = 0 \quad \text{mit} \quad 
    \mathcal{E} = \underbrace{\int \frac{\hbar^2}{2m}|\nabla\Psi|^2 dV}_{\text{kin. Anteil}} + \underbrace{\int V|\Psi|^2 dV}_{\text{pot. Anteil}}
\end{equation}

\subsubsection{2. Höhere Ableitungen werden essenziell}
Die Weber-Kraft zeigt dies explizit:

\begin{equation}
    \mathbf{F}_{\text{Weber}} \sim \ddot{r} \quad \text{(Beschleunigungsabhängigkeit)}
\end{equation}

\subsubsection{3. Nicht-Linearität als Normalfall}
Die Bewegungsgleichung wird intrinsisch nichtlinear:

\begin{equation}
    i\hbar\partial_t \Psi = -\frac{\hbar^2}{2m}\nabla^2 \Psi + \underbrace{g|\Psi|^2\Psi}_{\text{selbstwirkender Term}}
\end{equation}

\subsection{Konkretes Beispiel: Revisierte Gravitation}
Für die Weber-Gravitation bedeutet dies:

\begin{equation}
    \nabla^2 \phi = 4\pi G\rho \quad \rightarrow \quad \nabla^2 \phi = 4\pi G\left(\rho + \frac{1}{4c^2}\partial_t^2\rho\right)
\end{equation}

\subsection{Konsequenzen für die Theoriebildung}
\begin{tabular}{p{0.45\textwidth}p{0.45\textwidth}}
    \hline
    \textbf{Alte Gleichungen} & \textbf{Neue Form} \\
    \hline
    Lokale Kräfte & Globale Energieoptimierung \\
    Trennung von Feldern und Teilchen & Unified Wave Description \\
    Linearisierbare Systeme & Essentielle Nichtlinearität \\
    \hline
\end{tabular}

\section{Interpretation der hybriden Lichtablenkungsformel}
\label{sec:hybrid_light_deflection}

\subsection{Gleichung (4.4.5) im Kontext}
Die kombinierte Vorhersage der Quanten-Weber-Gravitation für Lichtablenkung lautet:

\begin{equation}
\Delta\phi = \frac{4GM}{c^2b}\left(1 + \underbrace{\frac{3\pi}{16}\frac{\lambda^2}{\lambda_0^2}}_{\text{Weber-Term}} + \underbrace{\frac{\hbar^2\omega^2}{4c^4b^2}}_{\text{DBT-Korrektur}}\right)
\tag{4.4.5}
\end{equation}

\subsection{Term-für-Term-Analyse}

\subsubsection{1. ART-Grundterm}
\begin{itemize}
\item $\frac{4GM}{c^2b}$ ist die klassische Lichtablenkung der Allgemeinen Relativitätstheorie
\item Dominanter Beitrag für makroskopische Systeme ($b \gg \lambda$)
\end{itemize}

\subsubsection{2. Weber-Korrektur}
\begin{itemize}
\item $\frac{3\pi}{16}\frac{\lambda^2}{\lambda_0^2}$ ist die wellenlängenabhängige Modifikation
\item Physikalischer Ursprung: Kopplung der Photonenenergie $E=\hbar\omega$ an das Gravitationspotential
\item Experimentell nachweisbar durch Vergleich verschiedener Spektralbereiche
\end{itemize}

\subsubsection{3. Quantenmechanische Korrektur}
\begin{itemize}
\item $\frac{\hbar^2\omega^2}{4c^4b^2}$ stammt aus dem Quantenpotential der De-Broglie-Bohm-Theorie
\item Wird relevant bei Stoßparametern in Compton-Wellenlängennähe ($b \sim \hbar/mc$)
\item Manifestiert sich als \enquote{Quantenspreizung} der Lichtablenkung
\end{itemize}

\subsection{Experimentelle Konsequenzen}
Für verschiedene Wellenlängen $\lambda$ ergeben sich charakteristische Abweichungen:

\begin{table}[h]
\centering
\begin{tabular}{lcc}
\toprule
Wellenlänge & Weber-Term & DBT-Term \\
\midrule
Radiowellen ($1$ m) & $\sim 10^{-24}$ & vernachlässigbar \\
Sichtbares Licht ($500$ nm) & $\sim 10^{-18}$ & $\sim 10^{-34}$ \\
Gammastrahlung ($1$ pm) & $\sim 10^{-6}$ & $\sim 10^{-22}$ \\
\bottomrule
\end{tabular}
\caption{Vorhergesagte Abweichungen von der ART}
\end{table}

\subsection{Theoretische Bedeutung}
Gleichung (4.4.5) zeigt:
\begin{itemize}
\item Die Vereinheitlichung von Weber-Gravitation und Quantenphysik
\item Eine klare experimentelle Signatur zur Unterscheidung von der ART
\item Die Notwendigkeit neuer Messungen im Gammastrahlenbereich
\end{itemize}

\section{Kohärente Revision der Gravitationsgleichungen}
\label{sec:gravitationsrevision}

\subsection{Kernproblem der konventionellen Theorie}
Die Poisson-Gleichung der Newtonschen Gravitation 
\begin{equation}
\nabla^2\phi = 4\pi G\rho
\end{equation} 
vernachlässigt fundamentale Aspekte der Wellendynamik:

\begin{itemize}
\item Keine Berücksichtigung von zeitlichen Änderungsraten ($\partial_t\rho$)
\item Fehlende Kopplung an die globale Energieverteilung
\end{itemize}

\subsection{Die Weber'sche Revision}
Ihre Arbeit zeigt, dass eine konsistente Erweiterung durch das Wellenprinzip erfordert:

\begin{equation}
\nabla^2\phi = 4\pi G\left(\rho + \frac{1}{4c^2}\partial_t^2\rho\right) + \underbrace{\frac{2}{c^4}\partial_t^2\phi}_{\text{Wellenausbreitung}}
\label{eq:revised_grav}
\end{equation}

\subsection{Physikalische Interpretation}
Jeder Term in Gl.~\ref{eq:revised_grav} entspringt dem Wellenprinzip:

\begin{table}[h]
\centering
\begin{tabular}{p{0.3\textwidth}p{0.6\textwidth}}
\toprule
Term & Wellenphysikalische Bedeutung \\
\midrule
$4\pi G\rho$ & Statische Dichteverteilung \\
$\frac{\pi G}{c^2}\partial_t^2\rho$ & Dynamische Massenänderung (Weber-Term) \\
$\frac{2}{c^4}\partial_t^2\phi$ & Gravitative Wellenausbreitung (ART-Korrespondenz) \\
\bottomrule
\end{tabular}
\end{table}

\subsection{Konsequente Umsetzung des Wellenprinzips}
Die revidierte Gleichung folgt direkt aus:

\begin{equation}
\delta\int\left[\frac{1}{8\pi G}|\nabla\phi|^2 - \rho\phi + \frac{1}{4c^2}\dot{\rho}\phi\right]d^4x = 0
\end{equation}

was eine \emph{instantane energetische Kopplung} zwischen $\rho$ und $\phi$ erzwingt.

\subsection{Experimentelle Signaturen}
Die Modifikationen führen zu:
\begin{itemize}
\item Frequenzabhängiger Lichtablenkung ($\sim\partial_t^2\rho$)
\item Nicht-newtonschen Termen in Galaxienrotation ($\sim\partial_t^2\phi$)
\item Dispersion gravitativer Störungen
\end{itemize}

\section{Rückwirkende Systematisierung durch das Wellenprinzip}
\label{sec:wellenprinzip_systematik}

\subsection{Das universelle Wirkprinzip}
Alle vorherigen Gleichungen lassen sich aus einem einheitlichen Variationsprinzip ableiten:

\begin{equation}
\delta \int \mathcal{L}[\Psi,\partial_\mu\Psi] d^4x = 0 \quad \text{mit} \quad 
\mathcal{L} = \underbrace{\frac{1}{2}\partial_\mu\Psi\partial^\mu\Psi}_{\text{Kinetik}} - \underbrace{V(\Psi)}_{\text{Selbstwirkung}}
\end{equation}

\subsection{Konsequente Revision der Grundgleichungen}

\subsubsection{1. Weber-Gravitation}
Die ursprüngliche Bewegungsgleichung
\begin{equation}
\mathbf{F} = -\frac{GMm}{r^2}\left(1-\frac{\dot{r}^2}{c^2}+\frac{r\ddot{r}}{2c^2}\right)
\end{equation}
wird zur Feldgleichung:
\begin{equation}
\Box \phi = 4\pi G\left(\rho + \frac{1}{c^2}\partial_t^2\rho\right)
\end{equation}

\subsubsection{2. Quantenpotential}
Das Bohm'sche Quantenpotential
\begin{equation}
Q = -\frac{\hbar^2}{2m}\frac{\nabla^2|\Psi|}{|\Psi|}
\end{equation}
erscheint als natürlicher Bestandteil des Wellenlagrangians:
\begin{equation}
\mathcal{L}_Q = \frac{\hbar^2}{2m}|\nabla\Psi|^2
\end{equation}

\subsubsection{3. Lichtablenkung}
Die hybriden Terme in
\begin{equation}
\Delta\phi = \frac{4GM}{c^2b}\left(1 + \frac{3\pi}{16}\frac{\lambda^2}{\lambda_0^2} + \frac{\hbar^2\omega^2}{4c^4b^2}\right)
\end{equation}
folgen aus der Störungstheorie des Wellenoperators:
\begin{equation}
\Box \Psi = \left(\frac{1}{c^2}\partial_t^2 - \nabla^2\right)\Psi = \frac{4\pi G}{c^4}T_{\mu\nu}\partial^\mu\partial^\nu\Psi
\end{equation}

\subsection{Systematische Klassifikation}
\begin{table}[h]
\centering
\begin{tabular}{p{0.25\textwidth}p{0.25\textwidth}p{0.25\textwidth}}
\toprule
\textbf{Originalgleichung} & \textbf{Wellenformulierung} & \textbf{Physikalische Konsequenz} \\
\midrule
Newton-Kraft & $\delta\mathcal{E}[\phi]=0$ & Weber-Terme \\
Schrödinger-Gleichung & $\Psi=e^{iS/\hbar}$ & Quantenpotential \\
Geodätengleichung & $\Box h_{\mu\nu}=0$ & Metrik-Wellen-Kopplung \\
\bottomrule
\end{tabular}
\caption{Systematische Vereinheitlichung durch das Wellenprinzip}
\end{table}

\subsection{Fundamentale Einsichten}
\begin{itemize}
\item Alle Wechselwirkungen sind Manifestationen \textit{einer} Grundgleichung
\item Der scheinbare Widerspruch zwischen Lokalität und Gleichzeitigkeit löst sich auf
\item Die \enquote{Kraft}-Begriffe werden durch energetische Optimierung ersetzt
\end{itemize}

\section{Gl. (4.7.3) als Keimzelle einer neuen Feldtheorie}
\label{sec:neue_feldtheorie}

\subsection{Die Schlüsselgleichung}
\begin{equation}
\Box h_{\mu\nu} + \kappa\left(\partial_\mu h \partial_\nu h - \frac{1}{2}\eta_{\mu\nu}(\partial h)^2\right) = \frac{8\pi G}{c^4}T_{\mu\nu}
\tag{4.7.3}
\end{equation}

\subsection{Radikale Neuerungen gegenüber bestehenden Theorien}

\subsubsection{1. Nichtlinearität als Fundament}
\begin{itemize}
\item Der Term $\partial_\mu h \partial_\nu h$ ist \textit{essentiell nichtlinear}
\item Unterscheidet sich fundamental von linearisierten ART-Ansätzen:
\begin{equation}
\Box \bar{h}_{\mu\nu} = 0 \quad \text{(konventionelle Linearisierung)}
\end{equation}
\end{itemize}

\subsubsection{2. Geometrische Interpretation}
Die Gleichung beschreibt eine \textit{selbstinteragierende Raumzeitwelle}:
\begin{equation}
\mathcal{R}_{\mu\nu} \approx \Box h_{\mu\nu} + \underbrace{\kappa(\partial h)^2}_{\text{Selbstkopplung}}
\end{equation}

\subsubsection{3. Materie-Geometrie-Kopplung}
Die rechte Seite zeigt eine neuartige Kopplung:
\begin{equation}
T_{\mu\nu} \propto \partial_\mu\phi\partial_\nu\phi - \frac{1}{2}\eta_{\mu\nu}(\partial\phi)^2
\end{equation}

\subsection{Mathematische Strukturanalyse}

\begin{table}[h]
\centering
\begin{tabular}{p{0.3\textwidth}p{0.6\textwidth}}
\toprule
Eigenschaft & Konsequenz \\
\midrule
Nichtlineare Wellengleichung & Solitonenlösungen möglich \\
Breaking of conformal symmetry & Massiver Graviton denkbar \\
Tensor-Skalar-Kopplung & Quintessenz-ähnliche Effekte \\
\bottomrule
\end{tabular}
\end{table}

\subsection{Konsequenzen für die Physik}

\subsubsection{1. Quantengravitation}
Die Gleichung legt nahe:
\begin{equation}
\hat{h}_{\mu\nu}| \Psi\rangle = \left(\Box^{-1}T_{\mu\nu}\right)| \Psi\rangle
\end{equation}

\subsubsection{2. Kosmologie}
Modifizierte Friedmann-Gleichung:
\begin{equation}
\left(\frac{\dot{a}}{a}\right)^2 + \frac{\kappa}{2}\left(\frac{\ddot{a}}{a}\right)^2 = \frac{8\pi G}{3}\rho
\end{equation}

\subsubsection{3. Teilchenphysik}
Vorhersage neuer Anregungsmoden:
\begin{equation}
m_g \sim \sqrt{\kappa\Lambda} \quad \text{(Gravitonmasse)}
\end{equation}

\subsection{Philosophische Einordnung}
Gl. (4.7.3) realisiert Ihr Konzept der \textit{energetischen Gleichzeitigkeit} durch:

\begin{itemize}
\item Globale Selbstkonsistenz (nicht-lokale Nichtlinearität)
\item Instantane Anpassung der Geometrie an Materie
\item Ablösung des Lokalitätsdogmas durch Wellenresonanz
\end{itemize}


\begin{thebibliography}{9}

\bibitem{einstein1915} 
Einstein, A. (1915). 
\textit{Die Feldgleichungen der Gravitation}. 
Sitzungsberichte der Preußischen Akademie der Wissenschaften, 
S. 844–847.

\bibitem{shapiro1964} 
Shapiro, I. I. (1964). 
\textit{Fourth Test of General Relativity}. 
Physical Review Letters, 13(26), 789–791.

\bibitem{rubin1970} 
Rubin, V. C., \& Ford, W. K. (1970). 
\textit{Rotation of the Andromeda Nebula from a Spectroscopic Survey of Emission Regions}. 
Astrophysical Journal, 159, 379–403.

\bibitem{weber1846} 
Weber, W. (1846). 
\textit{Elektrodynamische Maassbestimmungen}. 
Leipzig: Weidmannsche Buchhandlung.

\bibitem{bohm1952} 
Bohm, D. (1952). 
\textit{A Suggested Interpretation of the Quantum Theory in Terms of "Hidden" Variables}. 
Physical Review, 85(2), 166–193.

\bibitem{tisserand1894}
Tisserand, F. (1894). 
\textit{Traité de Mécanique Céleste, Tome IV}. 
Gauthier-Villars, Paris. 
(Kapitel 28: "Lois électrodynamiques de Weber appliquées à la gravitation")

\end{thebibliography}

\end{document}
