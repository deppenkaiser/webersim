\documentclass{book}
\usepackage[utf8]{inputenc}
\usepackage[ngerman]{babel}  % Falls du Deutsch schreibst
\usepackage{amsmath}
\usepackage{amssymb}

\begin{document}

\title{Mein Dokument}
\author{Dein Name}
\date{\today}
\maketitle

% Einbindung der einzelnen TeX-Dateien
\chapter{Grundlagen}
\section{Grundgleichungen der Weber-Kraft}
\[ F = -\frac{GMm}{r^2}\left(1 - \frac{\dot{r}^2}{c^2} + \frac{r\ddot{r}}{2c^2}\right) \]

Daraus folgt die Bewegungsgleichung:
\[ \ddot{r} - r\dot{\varphi}^2 = -\frac{GM}{r^2}\left(1 - \frac{\dot{r}^2}{c^2} + \frac{r\ddot{r}}{2c^2}\right) \]

\section{Klassische Lösung (0. Ordnung)}
Für \( c \to \infty \) ergibt sich die Kepler-Bahn:
\[ r_0(\varphi) = \frac{a(1-e^2)}{1 + e \cos\varphi} \]
\[ a_0(\varphi) = -\frac{GM}{r_0^2(\varphi)} \]

\section{Relativistische Korrektur (1. Ordnung)}
Störungsansatz für die Beschleunigung:
\[ a(\varphi) = a_0(\varphi) + \frac{GM}{c^2} a_1(\varphi) + \mathcal{O}(1/c^4) \]

Einsetzen in die Bewegungsgleichung liefert den Korrekturterm:
\[ a_1(\varphi) = \frac{GM}{r_0^2(\varphi)} \left( \frac{3h^2}{r_0^2(\varphi)} - \frac{h^2}{2GM r_0(\varphi)} \left(\frac{dr_0}{d\varphi}\right)^2 \right) \]

\section{Beschleunigung bis zur 1. Ordnung}
\[ a(\varphi) = -\frac{GM}{r_0^2(\varphi)} \left[ 
1 - \frac{1}{c^2} \left( 
\frac{3h^2}{r_0^2(\varphi)} - \frac{h^2}{2GM r_0(\varphi)} \left(\frac{dr_0}{d\varphi}\right)^2 
\right) 
\right] \]

\textbf{Hinweis:} \( r_0(\varphi) \) ist die klassische Kepler-Lösung, \( h \) der spezifische Drehimpuls.

\section{Explizite Form mit Bahnelementen}
Einsetzen von \( r_0(\varphi) = \frac{a(1-e^2)}{1 + e \cos\varphi} \):
\[ a(\varphi) = -\frac{GM(1 + e \cos\varphi)^2}{a^2(1-e^2)^2} \left[ 
1 - \frac{3h^2(1 + e \cos\varphi)^2}{c^2 a^2(1-e^2)^2} 
+ \frac{h^2 e^2 \sin^2\varphi}{2c^2 GM a^3(1-e^2)^3} (1 + e \cos\varphi)^3 
\right] \]

\section{Theoretische Grundlage}
\[
r(\phi) = r_{\text{ART}}(\phi) + \delta r(\phi)
\]
Hier ist \( r_{\text{ART}}(\phi) \) die analytische Näherung (ART-genau) und \( \delta r(\phi) \) die numerisch berechnete Korrektur.

\section{Schrittweitensteuerung}
Die Schrittweite \( \Delta \phi \) wird dynamisch aus den analytischen Ableitungen bestimmt:
\[
\Delta \phi = \min \left( \Delta \phi_{\text{max}}, \frac{\epsilon}{|w(\phi)| + |v(\phi)|} \right)
\]
mit \( v(\phi) = \frac{dr}{d\phi} \) und \( w(\phi) = \frac{d^2r}{d\phi^2} \) aus der ART-Näherung.

\section{Numerische Korrektur}
In jedem Schritt wird nur die Abweichung von der ART-Näherung numerisch integriert:
\[
\delta r(\phi + \Delta \phi) = \delta r(\phi) + \text{Numerische Integration von } \left( \text{DGL} - \text{ART-Ableitung} \right)
\]

\section{Gesamtlösung}
Die finale Lösung kombiniert beide Anteile:
\[
r(\phi + \Delta \phi) = r_{\text{ART}}(\phi + \Delta \phi) + \delta r(\phi + \Delta \phi)
\]

\section{Kartesische Koordinaten}
\[ \vec{r}(\phi) = \begin{pmatrix} x(\phi) \\ y(\phi) \end{pmatrix} \]
\[ r(\phi) = \sqrt{x(\phi)^2 + y(\phi)^2} \]
\[ \omega(\phi) = \frac{d\phi}{dt} = \frac{h}{r(\phi)^2} \]
\section{Weber-Kraft in kartesischer Form}
\[ \vec{F} = -\frac{GMm}{r^3}\vec{r}\left(1 - \frac{|\dot{\vec{r}}|^2}{c^2} + \frac{\vec{r}\cdot\ddot{\vec{r}}}{2c^2}\right) \]
\section{Zeitliche Ableitungen}
\[ \dot{\vec{r}} = \omega\frac{d\vec{r}}{d\phi} = \omega\vec{r}' \]
\[ \ddot{\vec{r}} = \omega^2\vec{r}'' + \omega\frac{d\omega}{d\phi}\vec{r}' \]
\section{Skalarprodukte}
\[ |\dot{\vec{r}}|^2 = \omega^2(x'^2 + y'^2) \]
\[ \vec{r}\cdot\ddot{\vec{r}} = \omega^2(xx'' + yy'') + \omega\frac{d\omega}{d\phi}(xx' + yy') \]
\section{Differentialgleichung für \(x(\phi)\)}
\[ x'' = \frac{1}{1 + \frac{GM}{2c^2r}} \left[ \frac{2(x'^2 + y'^2)}{r^2}x - \frac{GM}{\omega^2 r^3}x\left(1 - \frac{\omega^2(x'^2 + y'^2)}{c^2}\right) \right] \]
\section{Differentialgleichung für \(y(\phi)\)}
\[ y'' = \frac{1}{1 + \frac{GM}{2c^2r}} \left[ \frac{2(x'^2 + y'^2)}{r^2}y - \frac{GM}{\omega^2 r^3}y\left(1 - \frac{\omega^2(x'^2 + y'^2)}{c^2}\right) \right] \]
\section{Differentialgleichung für \(\omega(\phi)\)}
\[ \frac{d\omega}{d\phi} = -\frac{2h}{r^3}(xx' + yy') \]
\section{Zusammenfassung des DGL-Systems}
\[ \vec{Y} = \begin{pmatrix} x \\ y \\ x' \\ y' \\ \omega \end{pmatrix} \]
\[ \frac{d\vec{Y}}{d\phi} = \begin{pmatrix} x' \\ y' \\ x'' \\ y'' \\ \omega' \end{pmatrix} \]
\section{Koordinatensystem und Basisvektoren}
\[
\begin{aligned}
\hat{e}_r &= \cos\phi\,\hat{i} + \sin\phi\,\hat{j} \\
\hat{e}_\phi &= -\sin\phi\,\hat{i} + \cos\phi\,\hat{j}
\end{aligned}
\]
\[
\vec{r} = r\hat{e}_r, \quad \dot{\vec{r}} = \dot{r}\hat{e}_r + r\dot{\phi}\hat{e}_\phi
\]
\section{Post-Newtonische Kraft\\in vektorieller Form}
\[
\vec{F} = -\frac{GMm}{r^2}\left(1 - \frac{|\dot{\vec{r}}|^2}{c^2} + \frac{(\vec{r} \cdot \ddot{\vec{r}})}{2c^2}\right)\hat{e}_r
\]
\section{Geschwindigkeitsquadrat}
\[
|\dot{\vec{r}}|^2 = \dot{r}^2 + r^2\dot{\phi}^2
\]
\section{Beschleunigungsskalarprodukt}
\[
\vec{r} \cdot \ddot{\vec{r}} = r\ddot{r} - r^2\dot{\phi}^2
\]
\section{Bewegungsgleichung in vektorieller Form}
\[
m\ddot{\vec{r}} = -\frac{GMm}{r^2}\left(1 - \frac{\dot{r}^2 + r^2\dot{\phi}^2}{c^2} + \frac{r\ddot{r} - r^2\dot{\phi}^2}{2c^2}\right)\hat{e}_r
\]
\section{Differentialgleichungssystem}
\[
\begin{cases}
\frac{d^2x}{d\phi^2} = f_x\left(x,y,\frac{dx}{d\phi},\frac{dy}{d\phi}\right) \\
\frac{d^2y}{d\phi^2} = f_y\left(x,y,\frac{dx}{d\phi},\frac{dy}{d\phi}\right)
\end{cases}
\]
\section{Explizite DGL für x-Komponente}
\[
\frac{d^2x}{d\phi^2} = \frac{ \frac{GMm^2}{L^2}\frac{x}{r^3} - \frac{x}{r^2} - \frac{GM}{c^2}\left[ \frac{1}{r^2}\left(\frac{dx}{d\phi}\frac{dy}{d\phi}(y\frac{dx}{d\phi}-x\frac{dy}{d\phi}) + \frac{x}{2r^4}\left((\frac{dx}{d\phi})^2 + (\frac{dy}{d\phi})^2\right)\right) \right] }{ 1 - \frac{GM}{2c^2r} }
\]
\section{Explizite DGL für y-Komponente}
\[
\frac{d^2y}{d\phi^2} = \frac{ \frac{GMm^2}{L^2}\frac{y}{r^3} - \frac{y}{r^2} - \frac{GM}{c^2}\left[ \frac{1}{r^2}\left(\frac{dx}{d\phi}\frac{dy}{d\phi}(x\frac{dy}{d\phi}-y\frac{dx}{d\phi}) + \frac{y}{2r^4}\left((\frac{dx}{d\phi})^2 + (\frac{dy}{d\phi})^2\right)\right) \right] }{ 1 - \frac{GM}{2c^2r} }
\]
\section{Transformiertes System 1. Ordnung}
\[
\begin{cases}
\frac{dx}{d\phi} = v_x \\
\frac{dy}{d\phi} = v_y \\
\frac{dv_x}{d\phi} = f_x(x,y,v_x,v_y) \\
\frac{dv_y}{d\phi} = f_y(x,y,v_x,v_y)
\end{cases}
\]
\section{Klassische Weber-Kraft (Elektrodynamik)}
\[ F_{Weber}^{EM} = \frac{Qq}{4\pi\epsilon_0 r^2}\left(1 - \frac{\dot{r}^2}{c^2} + \frac{2r\ddot{r}}{c^2}\right)\hat{r} \]

\subsection*{Beschreibung der Symbole:}
\begin{itemize}
    \item $F_{Weber}^{EM}$: Weber-Elektrodynamische Kraft zwischen zwei Ladungen
    \item $Q, q$: Elektrische Ladungen der beiden wechselwirkenden Teilchen
    \item $\epsilon_0$: Elektrische Feldkonstante (Permittivität des Vakuums)
    \item $r$: Abstand zwischen den Ladungen
    \item $\dot{r} = \frac{dr}{dt}$: Relative Radialgeschwindigkeit der Ladungen
    \item $\ddot{r} = \frac{d^2r}{dt^2}$: Relative Radialbeschleunigung der Ladungen
    \item $c$: Lichtgeschwindigkeit im Vakuum
    \item $\hat{r}$: Einheitsvektor in radialer Richtung
\end{itemize}

\subsection*{Zusammenhang zur Coulomb-Kraft:}
Die Weber-Kraft verallgemeinert das Coulomb-Gesetz für bewegte Ladungen:
\begin{itemize}
    \item Der erste Term $\frac{Qq}{4\pi\epsilon_0 r^2}$ entspricht genau der klassischen Coulomb-Kraft zwischen statischen Ladungen.
    \item Die zusätzlichen Terme $\left(-\frac{\dot{r}^2}{c^2} + \frac{2r\ddot{r}}{c^2}\right)$ beschreiben Geschwindigkeits- und Beschleunigungsabhängige Korrekturen zur Coulomb-Wechselwirkung.
    \item Für $\dot{r} = 0$ und $\ddot{r} = 0$ (statischer Fall) reduziert sich die Weber-Kraft auf die Coulomb-Kraft.
\end{itemize}

\subsection*{Bedeutung der Weber-Kraft im Vergleich zu Maxwell:}
\begin{itemize}
    \item Die Weber-Elektrodynamik bietet eine alternative Beschreibung elektromagnetischer Phänomene zur Maxwell-Theorie.
    \item Im Gegensatz zu Maxwells Feldtheorie beschreibt Webers Ansatz die elektrodynamische Wechselwirkung direkt zwischen Ladungen (Fernwirkungskonzept).
    \item Die Weber-Kraft enthält implizit retardierte Effekte (durch die Geschwindigkeits- und Beschleunigungsterme), während Maxwell diese explizit durch retardierte Potentiale beschreibt.
    \item Die Weber-Theorie sagt für viele Phänomene (wie die Ampere-Kraft zwischen Stromleitern) dieselben Ergebnisse voraus wie Maxwell, unterscheidet sich aber in einigen Spezialfällen.
    \item Ein wesentlicher Unterschied ist, dass die Weber-Theorie keine elektromagnetischen Wellen im Vakuum vorhersagt, was ein zentrales Element der Maxwell-Theorie ist.
\end{itemize}

\section{Zusammenhang zwischen Maxwell-Theorie und ART: Wellenausbreitung und Raummodelle}

\subsection{Maxwells elektromagnetische Wellen im flachen Raum}
Die Maxwell-Gleichungen in ihrer klassischen Form,
\[
\nabla \cdot \mathbf{E} = \frac{\rho}{\epsilon_0}, \quad
\nabla \times \mathbf{B} = \mu_0\mathbf{J} + \mu_0\epsilon_0\frac{\partial\mathbf{E}}{\partial t},
\]
implizieren die Existenz elektromagnetischer Wellen im Vakuum ($\rho=0$, $\mathbf{J}=0$), beschrieben durch die Wellengleichung:
\[
\left(\nabla^2 - \frac{1}{c^2}\frac{\partial^2}{\partial t^2}\right)\mathbf{E} = 0, \quad \left(\nabla^2 - \frac{1}{c^2}\frac{\partial^2}{\partial t^2}\right)\mathbf{B} = 0
\]
\begin{itemize}
    \item \textbf{Raummodell}: Flacher Minkowski-Raum $\mathbb{R}^{3,1}$ mit konstanter Metrik $\eta_{\mu\nu} = \mathrm{diag}(-1,1,1,1)$
    \item \textbf{Lichtausbreitung}: Geradlinige Ausbreitung mit $c = \frac{1}{\sqrt{\mu_0\epsilon_0}}$ als universelle Konstante
    \item \textbf{Voraussetzung}: Isotropie und Homogenität des Raumes für Wellenausbreitung
\end{itemize}

\subsection{Allgemeine Relativitätstheorie und gekrümmte Raumzeit}
In der ART wird die Metrik $g_{\mu\nu}$ dynamisch durch die Einstein-Gleichungen bestimmt:
\[
G_{\mu\nu} = \frac{8\pi G}{c^4}T_{\mu\nu}
\]
\begin{itemize}
    \item \textbf{Wellenausbreitung}: Licht folgt nullgeodätischen Bahnen mit $ds^2 = g_{\mu\nu}dx^\mu dx^\nu = 0$
    \item \textbf{Konsequenzen}:
    \begin{enumerate}
        \item Gravitative Lichtablenkung durch Raumzeitkrümmung
        \item Zeitverzögerung (Shapiro-Verzögerung)
        \item Frequenzverschiebung (gravitativer Rot-/Blauverschiebung)
    \end{enumerate}
    \item \textbf{Kontinuum}: Existenz einer differenzierbaren Mannigfaltigkeit als fundamentale Voraussetzung
\end{itemize}

\subsection{Konzeptioneller Brückenschlag}
\begin{tabular}{|l|l|l|}
\hline
\textbf{Aspekt} & \textbf{Maxwell (flache Raumzeit)} & \textbf{ART (gekrümmte Raumzeit)} \\
\hline
Wellengleichung & Lineare DGL in $\eta_{\mu\nu}$ & Geodätengleichung $\frac{d^2x^\mu}{d\lambda^2} + \Gamma^\mu_{\alpha\beta}\frac{dx^\alpha}{d\lambda}\frac{dx^\beta}{d\lambda} = 0$ \\
\hline
Ausbreitungsmedium & Kein Äther, aber absoluter Raum & Dynamische Geometrie $g_{\mu\nu}(x)$ \\
\hline
Invarianzen & Lorentz-Transformationen & Allgemeine Kovarianz \\
\hline
\end{tabular}

\subsection*{Fundamentale Erkenntnis}
Die ART verallgemeinert das Maxwellsche Konzept der Wellenausbreitung:
\begin{itemize}
    \item Maxwells $c$ wird zur lokalen Größe in gekrümmter Raumzeit
    \item Die konstante Metrik $\eta_{\mu\nu}$ wird durch das dynamische Feld $g_{\mu\nu}$ ersetzt
    \item Die ART benötigt dabei zwingend ein Kontinuumsmodell des Raumes, während Maxwell dies nur implizit voraussetzt
\end{itemize}

\include{quantisierte_weber_kraft}
\section{Elektrisches Feld als Deformationsgradient}
\[ \vec{E} = \frac{\Delta (\text{Zellvolumen})}{L_p^3} \cdot \hat{r} \]
\section{Universelle Weber-Kraft}
\[ F_{universal} = \frac{K \cdot V_1(t) V_2(t)}{(nL_p)^2} \left(1 - \frac{v_{eff}^2}{c^2} + \frac{\beta L_p a_{eff}}{c^2}\right)\hat{r} \]
\section{Energie-Impuls-Beziehung für Photonen}
\[ E = \hbar \nu = \frac{h c}{\lambda} \]
\section{Webers Gravitationskraft}
\[ F = \frac{G \cdot M \cdot m}{r^2} \cdot \left[1 - \frac{v^2}{c^2} + \frac{r \cdot a}{c^2}\right] \]
\section{Theorievergleich: ART vs. Weber}
\begin{tabular}{|l|l|l|}
\hline
\textbf{Aspekt} & \textbf{ART} & \textbf{Weber} \\
\hline
Raummodell & Raumzeitkrümmung & Direkte Teilchenwechselwirkung \\
\hline
Gravitationswellen & Vorhanden & Nicht existent \\
\hline
Schwarze Löcher & Singularitäten & Keine Singularitäten \\
\hline
Galaxienrotation & Dunkle Materie benötigt & Natürliche Erklärung \\
\hline
Quantenkompatibilität & Problemhaft & Einfacher quantisierbar \\
\hline
\end{tabular}
\section{Vorteile der Weber-Theorie}
\begin{itemize}
\item Erklärt Galaxienrotation ohne Dunkle Materie
\item Vermeidet Singularitäten
\item Leichter mit Quantenphysik vereinbar
\item Direkte Kräfte zwischen Teilchen (keine Raumkrümmung)
\end{itemize}
\section{Historische Dominanz der ART}
\begin{itemize}
\item Frühe experimentelle Bestätigung (1919)
\item Einsteins Bekanntheit
\item Forschungsinfrastruktur auf ART ausgerichtet
\item Weber-Theorie als "altmodisch" abgetan
\end{itemize}
\section{Quantengravitation mit Weber}
\begin{itemize}
\item Keine Hawking-Strahlung vorhergesagt
\item Neue Gravitationssignal-Typen möglich
\item Direkte Quantisierung der Kraftgleichung
\item Kompatibel mit Quantenfeldtheorien
\end{itemize}
\include{weber_gravitationskraft}
\section{Periheldrehung des Merkur}
\[ \Delta\theta = \frac{6\pi GM}{a c^2 (1-e^2)} \]
\section{Allgemeine $\beta$-Formel}
\[ \beta = 2 \cdot \left( \frac{1}{2} \right)^{\delta} \cdot \left(1 - \frac{m c^2}{E}\right) \]
\include{universelle_kraft}
\section{Gravitationswellengleichung}
\[ \Box h_{\mu\nu} = -\frac{16\pi G}{c^4} \left( T_{\mu\nu} - \frac{1}{2} \beta \cdot \partial_t^2 Q_{\mu\nu} \right) \]
\section{Quantisierte Weber-Kraft (QED)}
\[ F_{Weber}^{QED} = \frac{V_1(t) V_2(t)}{4\pi\epsilon_0 (nL_p)^2} \left(1 - \frac{(\Delta L_p / \Delta t_p)^2}{c^2} + \frac{2 L_p \Delta^2 L_p}{c^2 \Delta t_p^2}\right)\hat{r} \]
\section{Frequenzabhängige Lichtablenkung}
\[ \Delta \phi \sim \frac{4GM}{c^2b}\left(1 + \frac{\lambda_0^2}{\lambda^2}\right) \]
\section{Hamiltonian des Dodekaeder-Gitters}
\[ \mathcal{H} = \sum_{\text{Kanten}} \epsilon (V_i(t) - V_j(t))^2 \]

\end{document}
