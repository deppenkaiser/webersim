\documentclass{book}
\usepackage[a4paper,left=2.5cm,right=2cm,top=2cm,bottom=2.5cm]{geometry}
\usepackage[utf8]{inputenc}
\usepackage[ngerman]{babel}
\usepackage{amsfonts}
\usepackage{amsmath}
\usepackage{amssymb}
\usepackage{array}
\usepackage{ragged2e}
\usepackage{tabularx}
\usepackage{enumitem}
\usepackage{booktabs}
\usepackage{bm}
\usepackage{csquotes}
\usepackage{siunitx}
\usepackage{parskip}
\usepackage{listings}
\usepackage{xcolor}
\usepackage[utf8]{inputenc}
\usepackage[labelfont=bf]{caption}
\newtheorem{theorem}{Theorem} % Definiert das 'theorem'-Environment
\newtheorem{lemma}{Lemma}     % Falls Sie auch Lemmas verwenden möchten

\renewcommand{\arraystretch}{1.1}
\numberwithin{equation}{section}
\definecolor{gray}{rgb}{0.5,0.5,0.5}
\cleardoublepage

\begin{document}

\title{Weber-Gravitation}
\author{Michael Czybor}
\date{\today}
\maketitle

\section*{Zusammenfassung}
Die „Weber-Gravitation“ präsentiert eine alternative Gravitationstheorie, die auf der Weber-Kraft basiert und in vielen Aspekten konkurrenzfähige
oder sogar überlegene Ergebnisse im Vergleich zur Allgemeinen Relativitätstheorie (ART) liefert. Die zentrale These der Arbeit ist, dass die
Weber-Gravitation (WG) nicht nur die bekannten Phänomene der ART erklärt, sondern auch deren Schwächen – wie die Notwendigkeit dunkler Materie
oder die Existenz singularitätsbehafteter schwarzer Löcher – vermeidet.

Ein herausragendes Ergebnis der WG ist die präzise Berechnung der Periheldrehung des Merkurs, die mit einem Wert von 42,98 Bogensekunden pro Jahrhundert
nahezu identisch zur ART-Vorhersage ist. Entscheidend ist jedoch, dass die WG dies ohne ein gekrümmtes Raumzeit-Modell erreicht. Stattdessen modifiziert
sie das Newtonsche Gravitationsgesetz durch relativistische Korrekturen, die von der radialen Geschwindigkeit (\(\dot{r}\)) und Beschleunigung (\(\ddot{r}\)) abhängen.
Die daraus abgeleitete Bahngleichung zeigt, dass die WG die beobachtete Periheldrehung natürlicher erklärt als die ART, ohne auf ein komplexes geometrisches
Raummodell zurückgreifen zu müssen.

Ein weiterer wesentlicher Vorteil der WG ist ihre Fähigkeit, galaktische Rotationskurven ohne dunkle Materie zu beschreiben. Während die ART zusätzliche,
unsichtbare Masse postulieren muss, um die flachen Rotationsprofile von Galaxien zu erklären, liefert die WG eine korrigierte Geschwindigkeitsformel,
die den beobachteten Verlauf reproduziert:  

\[
v(r) = \sqrt{\frac{GM}{r}} \left(1 + \frac{GM}{4c^2r}\right).
\]  

Dieser Ansatz vermeidet nicht nur die hypothetische dunkle Materie, sondern bietet auch eine direkte physikalische Interpretation der Abweichungen vom Newtonschen Gesetz.

Die Arbeit diskutiert zudem die Lichtablenkung im Gravitationsfeld, wobei die WG eine frequenzabhängige Korrektur vorhersagt, die in der ART nicht existiert.
Diese könnte zukünftig experimentell überprüft werden, etwa durch hochpräzise Messungen der Ablenkung von Radiowellen gegenüber optischem Licht. Auch der
Shapiro-Effekt (Laufzeitverzögerung von Signalen) wird in der WG leicht modifiziert, wobei die Abweichungen zur ART jedoch erst bei extrem hohen Genauigkeiten messbar wären.  

Ein radikaler Unterschied zur ART zeigt sich in der kosmologischen Interpretation der Rotverschiebung. Während die ART diese als Folge der Expansion des Universums deutet,
erklärt die WG sie durch kumulative gravitative Wechselwirkungen:

\[
z \approx \frac{3}{2} \frac{v_r^2}{c^2}.
\]  

Dies impliziert ein statisches Universum ohne Urknall, was eine grundlegend andere Kosmologie zur Folge hätte. Die Arbeit argumentiert, dass dieser Ansatz mehrere Probleme
der Standardkosmologie (wie die dunkle Energie) vermeiden könnte.  

Kritisch bleibt, dass die WG keine Gravitationswellen vorhersagt, da ihr ein dynamisches Raumzeit-Modell fehlt. Hierin besteht jedoch kein grundsätzliches Hindernis,
sondern es ist ein Anreiz, die Theorie um ein Quantengravitations-Konzept zu erweitern.

Fazit: Die Weber-Gravitation stellt eine vielversprechende Alternative zur ART dar, die mehrere ihrer ungelösten Probleme umgeht. Obwohl sie in einigen Bereichen
(wie der Merkurperiheldrehung) äquivalente Ergebnisse liefert, bietet sie in anderen (Galaxienrotation, Kosmologie) potenziell einfachere und elegantere Erklärungen.
Experimentelle Tests der frequenzabhängigen Effekte wären der nächste Schritt, um die Theorie weiter zu validieren. Die Arbeit plädiert dafür, die WG als
ernstzunehmenden Ansatz in der modernen Gravitationsphysik zu betrachten.

\tableofcontents

% Einbindung der einzelnen TeX-Dateien
\part{Grundlagen}
\chapter{Weber-Kraft}
\input{content/07_weber_kraft_em}
\input{content/09_grundgleichungen}
\newpage
\section{Bahngleichungen}
\subsection{Bahngleichung 1. Ordnung}
Die Bahngleichung \(r(\phi)\) in der Weber-Gravitation bis zur Ordnung \(\mathcal{O}(c^{-2})\) lautet:

\begin{equation}
\label{eq:bahngleichung_1_ordnung}
r(\phi) = \frac{a(1 - e^2)}{1 + e \cos\left(\kappa\phi\right)}
\end{equation}

\noindent mit der Definition:
\begin{equation}
\label{eq:kappa_1_ordnung}
\kappa = \sqrt{1 - \frac{6GM}{c^2a(1 - e^2)}}
\end{equation}

\subsection*{Mathematische Herleitung}
Die Gleichung folgt aus der Lösung der Bewegungsgleichung:
\begin{equation}
\frac{d^2u}{d\phi^2} + u = \frac{GM}{h^2} + \frac{6GM}{c^2} u^2 \quad \left(u = \frac{1}{r}\right),
\end{equation}

wobei der Term \(\frac{6GM}{c^2} u^2\) die Weber-spezifische Korrektur 1. Ordnung darstellt. Der Ansatz \(u(\phi) = \frac{1 + e \cos(\kappa\phi)}{a(1 - e^2)}\) führt auf die angegebene Lösung.

Mit $u=1/r$ und Drehimpuls $h$ (\ref{eq:spezifischer_drehimpuls_h}):
\begin{equation}
\frac{d^2u}{d\phi^2} + u = \frac{GM}{h^2} + \frac{6GM}{c^2}u^2 + \frac{GM}{2c^2}\left(u\frac{d^2u}{d\phi^2} + \left(\frac{du}{d\phi}\right)^2\right)
\end{equation}

\subsection{Bahngleichung 2. Ordnung}
Bahngleichung:
\begin{equation}
\label{eq:bahngleichung_2_ordnung}
    \boxed
    {
        r(\phi) = \frac{a(1-e^2)}{1 + e\cos\left(\kappa\phi + \alpha\phi^2\right)}
    }
\end{equation}

mit:
$h$ aus Gleichung (\ref{eq:spezifischer_drehimpuls_h})
\begin{equation}
\label{eq:kappa_2_ordnung}
\kappa = \sqrt{1 - \frac{6GM}{c^2a(1-e^2)} + \frac{27G^2M^2}{2c^4a^2(1-e^2)^2}}
\end{equation}
\begin{equation}
\label{eq:alpha}
\alpha = \frac{3G^2M^2e}{8h^4c^4}
\end{equation}

\section{Periheldrehung}
\subsection{Periheldrehung 1. Ordnung}
Die Periheldrehung $\Delta\phi$ in der Weber-Gravitation ergibt sich aus der modifizierten Bahngleichung und lässt sich wie folgt herleiten:

\subsection*{Perihelbedingung}
Das Perihel (sonnennächster Punkt) tritt auf, wenn der Nenner maximal wird, d.h. wenn:\\
\[\cos(\kappa\phi) = 1\]
Die Lösungen dieser Bedingung sind: $\kappa\phi = 2\pi n \quad \text{(für $n \in \mathbb{Z}$)}$.\\

Somit ergeben sich die Winkel für aufeinanderfolgende Periheldurchgänge zu:
\[
    \phi_n = \frac{2\pi n}{\kappa}.
\]

\subsection*{Periheldrehung pro Umlauf}
Die Periheldrehung $\Delta\phi$ ist die Differenz zwischen dem Winkel für einen vollständigen Umlauf ($n = 1$) und dem Newton'schen Fall ($\kappa = 1$):
\[
    \Delta\phi = \phi_1 - 2\pi = \frac{2\pi}{\kappa} - 2\pi.
\]
Daraus folgt die gesuchte Gleichung:
\begin{equation}
\boxed
{
    \Delta\phi = 2\pi\left(\frac{1}{\kappa} - 1\right)
}.
\end{equation}

\subsection*{Interpretation}
\begin{itemize}
\item Im Newton'schen Grenzfall ($\kappa = 1$) verschwindet die Periheldrehung ($\Delta\phi = 0$).
\item Für $\kappa < 1$ (Weber-Gravitation) ergibt sich eine positive Periheldrehung, die mit Beobachtungen (z.B. Merkurperihel) übereinstimmt.
\end{itemize}

\subsection{Periheldrehung in 2. Ordnung}
\subsection*{Entwicklung von $\kappa$}
Eine Taylor-Entwicklung von $\kappa$ bis zur 2. Ordnung liefert:
\[
    \kappa \approx 1 - \frac{3GM}{c^2 a(1 - e^2)} + \frac{27G^2 M^2}{4c^4 a^2 (1 - e^2)^2} + \mathcal{O}(c^{-6}).
\]

\subsection*{Perihelbedingung}
Das Perihel tritt auf bei:
\[
    \cos\left(\kappa\phi + \alpha\phi^2\right) = 1 \quad \Rightarrow \quad \kappa\phi + \alpha\phi^2 = 2\pi n.
\]

\subsection*{Lösung für $\Delta\phi$}
Für $n=1$ (ein Umlauf) ergibt sich die quadratische Gleichung:
\[
\alpha\phi^2 + \kappa\phi - 2\pi = 0.
\]
Die Lösung lautet:
\begin{equation}
\phi = \frac{-\kappa + \sqrt{\kappa^2 + 8\pi\alpha}}{2\alpha}.
\end{equation}

\subsection*{Näherung für kleine Korrekturen}
Da $\alpha \sim c^{-4}$ klein ist, entwickeln wir die Wurzel:
\[
    \phi \approx \frac{2\pi}{\kappa} - \frac{4\pi^2\alpha}{\kappa^3} + \mathcal{O}(\alpha^2).
\]
Die Periheldrehung pro Umlauf wird damit:
\begin{equation}
\Delta\phi = \phi - 2\pi \approx 2\pi\left(\frac{1}{\kappa} - 1\right) - \frac{4\pi^2\alpha}{\kappa^3}.    
\end{equation}

\subsection*{Endgültige Formel}
Einsetzen von $\kappa \approx 1$ im Korrekturterm liefert:
\begin{equation}
\boxed
{
    \Delta\phi \approx 2\pi\left(\frac{1}{\kappa} - 1\right) - 4\pi^2\alpha
},
\end{equation}

\subsection*{Vollständige Koeffizienten}
Explizit ausgedrückt, mit Bezug auf die Ordnungen:
\begin{align*}
\Delta\phi^{(2)} &= \frac{6\pi GM}{c^2 a(1 - e^2)} \left[1 + \frac{9GM}{4c^2 a(1 - e^2)}\right] - \frac{3\pi^2 G^2 M^2 e}{2c^4 h^4} \\
&= \Delta\phi^{(1)} + \frac{27\pi G^2 M^2}{2c^4 a^2 (1 - e^2)^2} - \frac{3\pi^2 G^2 M^2 e}{2c^4 [GMa(1 - e^2)]^2}
\end{align*}

\newpage
\section{Winkelgeschwindigkeit 1. Ordnung}
Die Winkelgeschwindigkeit \(\omega(\phi)\) in der Weber-Gravitation bis zur Ordnung \(\mathcal{O}(c^{-2})\) lautet:

\begin{equation}
\omega(\phi) = \frac{h}{a^2(1 - e^2)^2} \left[1 + e \cos\left(\kappa\phi\right)\right]^2
\end{equation}

wobei:
$h$ aus Gleichung (\ref{eq:spezifischer_drehimpuls_h}), $\kappa$ aus Gleichung (\ref{eq:kappa_1_ordnung}) stammt.

\subsection*{Bedeutung der Terme}
\begin{itemize}
    \item \(\kappa\) beschreibt die Periheldrehung 1. Ordnung ohne Näherung.
    \item Für \(c \to \infty\) wird \(\kappa = 1\), und die Gleichung reduziert sich auf die Newton’sche Form:
    \[
    \omega_N(\phi) = \frac{h(1 + e \cos\phi)^2}{a^2(1 - e^2)^2}.
    \]
\end{itemize}

\section{Winkelgeschwindigkeit 2. Ordnung}

\subsection{Winkelgeschwindigkeit}
Mit $h$ aus Gleichung (\ref{eq:spezifischer_drehimpuls_h}), $\kappa$ aus Gleichung (\ref{eq:kappa_2_ordnung}) und $\alpha$ aus Gleichung (\ref{eq:alpha}):
\begin{equation}
\boxed
{
    \omega(\phi) = \frac{h[1 + e\cos(\kappa\phi + \alpha\phi^2)]^2}{a^2(1-e^2)^2}
}
\end{equation}

\newpage
\section{Bahngeschwindigkeit in 1. Ordnung}
Die Bahngeschwindigkeit \(v(\phi)\) in der Weber-Gravitation bis zur Ordnung \(\mathcal{O}(c^{-2})\) lautet:

\begin{equation}
v(\phi) = \frac{h}{a(1 - e^2)} \left(1 + e \cos\left(\kappa\phi\right)\right)
\end{equation}

\noindent mit den Definitionen:
$h$ aus Gleichung (\ref{eq:spezifischer_drehimpuls_h}), $\kappa$ aus Gleichung (\ref{eq:kappa_1_ordnung})

\subsection*{Physikalische Interpretation}
\begin{itemize}
    \item \textbf{Struktur}: Die Geschwindigkeit folgt aus \(v(\phi) = h/r(\phi)\) mit der Bahngleichung \(r(\phi) = \frac{a(1 - e^2)}{1 + e \cos(\kappa\phi)}\).
    \item \textbf{Relativistische Korrektur}: \(\kappa\) modifiziert die Periheldrehung gegenüber Newton (\(\kappa = 1\)).
    \item \textbf{Grenzfälle}:
        \begin{itemize}
            \item Perihel (\(\phi = 0\)): \(v(0) = \frac{h(1 + e)}{a(1 - e^2)}\),
            \item Aphel (\(\phi = \pi\)): \(v(\pi) = \frac{h(1 - e)}{a(1 - e^2)}\),
            \item Newton (\(c \to \infty\)): \(v_N(\phi) = \frac{h(1 + e \cos\phi)}{a(1 - e^2)}\).
        \end{itemize}
\end{itemize}

\section{Bahngeschwindigkeit in 2. Ordnung}
Die Bahngeschwindigkeit $v(\phi)$ ergibt sich aus Winkelgeschwindigkeit $\omega(\phi)$ und Radialabstand $r(\phi)$:
\begin{equation}
v(\phi) = \omega(\phi) \cdot r(\phi) = \frac{h}{r(\phi)}
\end{equation}

Mit der Bahngleichung und Winkelgeschwindigkeit:
\begin{align}
r(\phi) &= \frac{a(1-e^2)}{1 + e\cos\left(\kappa\phi + \alpha\phi^2\right)}\\
\omega(\phi) &= \frac{h[1 + e\cos(\kappa\phi + \alpha\phi^2)]^2}{a^2(1-e^2)^2}
\end{align}

ergibt sich:
\begin{equation}
v(\phi) = \frac{h \left(1 + e\cos(\kappa\phi + \alpha\phi^2)\right)}{a(1 - e^2)}.
\end{equation}

\chapter{Sonnensystem}
\input{content/15_perihel}
\input{content/16_lichtablenkung}
\newpage
\subsection{Umlaufperiode $T$ in 1. Ordnung}
Die Umlaufperiode $T$ eines Planeten in der Weber-Gravitation (WG) ergibt sich aus der modifizierten Bahngleichung (\ref{eq:bahngleichung_1_ordnung}) und dem spezifischen Drehimpuls $h$ (\ref{eq:spezifischer_drehimpuls_h}).

\subsubsection*{Ausgangsgleichungen}
\begin{enumerate}
    \item $\kappa$ aus Gl. (\ref{eq:kappa_1_ordnung})
    \item Winkelgeschwindigkeit Gl. (\ref{eq:winkelgeschwindigkeit_1_ordnung})
\end{enumerate}

\subsubsection*{Schritt 2: Integration über einen Umlauf}
Die Periode $T$ ist die Zeit für $\phi = 0 \to 2\pi/\kappa$ (WG-Korrektur durch $\kappa$):
\begin{align}
    T &= \int_0^{2\pi/\kappa} \frac{d\phi}{\dot{\phi}} 
       = \frac{a^2(1-e^2)^2}{h} \int_0^{2\pi/\kappa} \frac{d\phi}{(1 + e \cos(\kappa \phi))^2}.
\end{align}

\subsubsection*{Schritt 3: Lösung des Integrals}
Mit der Substitution $\psi = \kappa \phi$ und $\cos^2$-Identität:
\begin{align}
\label{eq:integral_t}
    T &= \frac{a^2(1-e^2)^2}{h \kappa} \int_0^{2\pi} \frac{d\psi}{(1 + e \cos \psi)^2} 
       = \frac{2\pi a^2(1-e^2)^2}{h \kappa (1-e^2)^{3/2}} 
       = \frac{2\pi a^{3/2}}{\sqrt{GM} \kappa}.
\end{align}

\subsubsection*{Schritt 4: Entwicklung von $\kappa$}
Für kleine relativistische Korrekturen ($c^{-2}$-Ordnung) gilt:
\begin{equation}
    \kappa \approx 1 - \frac{3GM}{c^2 a(1-e^2)} + \mathcal{O}(c^{-4}).
\end{equation}
Einsetzen in Gl.~\eqref{eq:integral_t} liefert die Periode in 1. Ordnung:
\begin{equation}
    \boxed{
    T \approx \frac{2\pi a^{3/2}}{\sqrt{GM}} \left(1 + \frac{3GM}{c^2 a(1-e^2)}\right).
    }
\end{equation}

\newpage
\section{Umlaufperiode \( T \) 2. Ordnung}

\subsubsection*{Ausgangsgleichungen}
\begin{enumerate}
    \item Bahngleichung in Polarkoordinaten (\ref{eq:bahngleichung_2_ordnung})
    \item $\kappa$ aus Gl. (\ref{eq:kappa_2_ordnung})
    \item $\alpha$ aus Gl. (\ref{eq:alpha})
    \item $h$ aus Gl. (\ref{eq:spezifischer_drehimpuls_h})
\end{enumerate}

\subsection*{Schritt 1: Entwicklung von \(\kappa\)}
\begin{equation}
\kappa \approx 1 - \frac{3GM}{c^2a(1-e^2)} + \frac{27G^2M^2}{4c^4a^2(1-e^2)^2} - \frac{81G^3M^3}{8c^6a^3(1-e^2)^3} + \mathcal{O}(c^{-8}) 
\end{equation}

\subsection*{Schritt 2: Vollständige Integration}
Die Umlaufperiode \( T \) ist:
\begin{equation}
T = \frac{1}{h} \int_0^{2\pi} r^2(\phi) \, d\phi = \frac{a^2(1-e^2)^2}{h} \int_0^{2\pi} \frac{d\phi}{\left[1 + e\cos\left(\kappa\phi + \alpha\phi^2\right)\right]^2}
\end{equation}

\subsection*{Schritt 3: Behandlung des Integrals}
Mit Substitution \(\psi = \kappa\phi + \alpha\phi^2\) und Entwicklung bis \(\mathcal{O}(c^{-4})\):
\begin{align}
T &= \frac{a^2(1-e^2)^2}{h} \left[ \int_0^{2\pi} \frac{d\phi}{(1 + e\cos\psi)^2} + \mathcal{O}(c^{-6}) \right] \\
  &= \frac{2\pi a^{3/2}}{\sqrt{GM}} \left[1 + \frac{3GM}{2c^2a(1-e^2)} + \frac{45G^2M^2}{8c^4a^2(1-e^2)^2}\left(1 - \frac{e^2}{3}\right)\right]
\end{align}

\part{Kosmologie}
\input{content/21_DM}
\newpage
\section{Rotverschiebung in der Weber-Gravitation}
\label{sec:weber_redshift}

\subsection{Gravitative Rotverschiebung}
Für Photonen ($m=0$) im Gravitationsfeld folgt aus der Energieerhaltung in der WG:

\begin{equation}
\frac{E_\text{em}}{E_\text{obs}} = 1 + \frac{GM}{c^2}\left(\frac{1}{r_\text{em}} - \frac{1}{r_\text{obs}}\right) + \frac{3}{2}\frac{v_r^2}{c^2}
\end{equation}

wobei $v_r$ die Relativgeschwindigkeit zwischen Emitter und Detektor ist. Dies führt zur Rotverschiebung:

\begin{equation}
\frac{\Delta\lambda}{\lambda} = \underbrace{\frac{GM}{c^2}\left(\frac{1}{r_\text{em}} - \frac{1}{r_\text{obs}}\right)}_{\text{Statischer Term}} + \underbrace{\frac{3}{2}\frac{v_r^2}{c^2}}_{\text{Dynamischer Term}}
\end{equation}

\subsection{Vergleich der Rotverschiebungstypen}
\begin{table}[h]
\centering
\caption{Unterschiede in der Rotverschiebung}
\begin{tabular}{lll}
\hline
Typ & ART & Weber-Gravitation \\
\hline
\textbf{Gravitativ} & $\frac{GM}{c^2}\Delta\left(\frac{1}{r}\right)$ & Identisch + $v_r$-Korrektur \\
\textbf{Kosmologisch} & $z = \frac{a(t_0)}{a(t)}-1$ & $\frac{3}{2}\frac{v_r^2}{c^2}$ (Näherung) \\
\textbf{Doppler} & $\sqrt{\frac{1+v/c}{1-v/c}}-1$ & $\frac{v_r}{c} + \frac{3}{4}\frac{v_r^2}{c^2}$ \\
\hline
\end{tabular}
\end{table}

\subsection{Physikalische Interpretation}
\begin{itemize}
\item \textbf{Statischer Term}: Entspricht exakt der ART-Vorhersage (Pound-Rebka-Experiment)
\item \textbf{Dynamischer Term}: Zusätzliche Geschwindigkeitsabhängigkeit in der WG
\begin{equation}
z_\text{dyn} \approx \frac{3}{2}\frac{H_0^2 d^2}{c^2} \quad \text{(für $v_r = H_0 d$)}
\end{equation}
\item \textbf{Kosmologische Konsequenz}: Die WG erklärt Hubble-Rotverschiebung durch kumulative gravitative Wechselwirkungen statt Expansion
\end{itemize}

\subsection{Experimentelle Unterscheidung}
\begin{equation}
\frac{z_\text{WG}}{z_\text{ART}} = 1 + \frac{3}{2}\left(\frac{v_r}{c}\right)^2 \left(\frac{GM}{c^2r}\right)^{-1}
\end{equation}

Für Galaxien mit $v_r \approx 1000$ km/s und $r=1$ Mpc:
\[
\frac{z_\text{WG}}{z_\text{ART}} \approx 1 + 5\times10^{-7}
\]

\newpage
\section{Shapiro-Effekt in der Weber-Gravitation}

\subsection{Grundgleichung der Signallaufzeit}
Die Laufzeitverzögerung $\Delta t$ eines Signals (Licht oder Radar) im Gravitationsfeld der Masse $M$ folgt in der WG aus:

\begin{equation}
c\,dt = \left(1 + \frac{2GM}{c^2r} - \frac{GM}{2c^2}\frac{\dot{r}^2}{c^2}\right)dr
\end{equation}

\subsection{Integration entlang der Bahn}
Für einen Vorbeiflug mit Stoßparameter $b$ ergibt sich:

\begin{equation}
\Delta t = \underbrace{\frac{2GM}{c^3}\ln\left(\frac{4r_e r_p}{b^2}\right)}_{\text{ART-Term}} + \underbrace{\frac{3\pi G^2M^2}{4c^5b^2}\left(\frac{v_0^2}{c^2}\right)}_{\text{WG-Korrektur}}
\end{equation}

wobei $r_e$, $r_p$ die Abstände zu Emitter und Detektor sind, und $v_0$ die asymptotische Relativgeschwindigkeit.

\subsection{Vergleich mit Experimenten}
\begin{table}[h]
\centering
\caption{Messungen der Laufzeitverzögerung}
\begin{tabular}{lcc}
\hline
Experiment & ART-Vorhersage & WG-Vorhersage \\
\hline
Venus-Radar (1967) & $200\,\mu\text{s}$ & $200\,\mu\text{s} + 0.3\,\text{ps}$ \\
Cassini (2002) & $10^{-14}$ & $10^{-14}(1 + 5\times10^{-6})$ \\
\hline
\end{tabular}
\end{table}

\subsection{Physikalische Interpretation}
\begin{itemize}
\item \textbf{Radiale Geschwindigkeit}: Der Zusatzterm $\dot{r}^2/c^2$ modifiziert die effektive Lichtgeschwindigkeit
\item \textbf{Frequenzabhängigkeit}: Für $v_0 = c(\lambda_0/\lambda)$ entsteht eine wellenlängenabhängige Korrektur:
  \[
  \Delta t_\text{WG} \propto \lambda^{-2}
  \]
\item \textbf{Testbarkeit}: Die Abweichungen werden bei Pulsar-Timing-Experimenten (z.B. SKA) messbar sein
\end{itemize}

\begin{equation}
\boxed{
\Delta t_\text{WG} = \Delta t_\text{ART}\left(1 + \frac{3\pi GM}{8c^2b}\frac{v_0^2}{c^2}\right)
}
\end{equation}
\chapter{De-Broglie-Bohm-Theorie}
\section{De-Broglie-Bohm-Theorie und ihre mögliche Synthese mit der Weber-Gravitation}
\label{sec:bohm}

Die \textbf{De-Broglie-Bohm-Theorie (DBT)}, auch bekannt als Pilot-Wellen-Theorie, bietet eine alternative Interpretation der Quantenmechanik mit bemerkenswerten Parallelen zur Weber-Gravitation (WG). Beide Theorien teilen einen deterministischen Fernwirkungsansatz.

\subsection{Kernprinzipien der DBT}
\begin{itemize}
    \item \textbf{Wellenfunktion als Führungswelle}: 
    \[
    \psi(\mathbf{r},t) = R(\mathbf{r},t)e^{iS(\mathbf{r},t)/\hbar}
    \]
    wobei $R$ die Amplitude und $S$ die Phase mit direkter Verbindung zur Teilchentrajektorie.

    \item \textbf{Quantenpotential}: 
    \[
    Q(\mathbf{r},t) = -\frac{\hbar^2}{2m}\frac{\nabla^2 R}{R}
    \]
    Dieses nicht-lokale Potential führt zu quantenmechanischen Effekten ohne Kollaps der Wellenfunktion.

    \item \textbf{Teilchen mit definierten Trajektorien}: Im Gegensatz zur Kopenhagener Deutung besitzen Teilchen in der DBT stets wohldefinierte Positionen und Impulse.
\end{itemize}

\section{Theorie der quantisierten Dodekaeder-Gravitation}
\label{sec:dodekaeder_theorie}

\subsection{Grundpostulate}
\begin{itemize}
    \item \textbf{Raumquantisierung}: Das Universum besitzt eine diskrete Dodekaeder-Struktur mit charakteristischem Gitterabstand:
    \[
    d = 10^3\,\text{Mpc} \quad \text{(aus CMB-Resonanz bei $\ell=36$)}
    \]
    
    \item \textbf{Nichtkommutative Geometrie}: Raumzeitkoordinaten gehorchen einer ikosaedrischen Algebra:
    \[
    [\hat{x}_i, \hat{x}_j] = i\hbar\theta_0\sum_{k=1}^5\epsilon_{ijk}\omega_k, \quad \omega_k = e^{2\pi ik/5}
    \]
\end{itemize}

\subsection{Schlüsselgleichungen}
\begin{table}[htbp]
    \centering
    \caption{Naturkonstanten als topologische Invarianten}
    \begin{tabular}{lll}
        \toprule
        \textbf{Konstante} & \textbf{Geometrischer Ursprung} & \textbf{Wert} \\
        \midrule
        Feinstrukturkonstante $\alpha_{\text{EM}}$ & Pentagon-Phaseninterferenz & $\frac{1}{12}(d/\ell_p)^{-1/5} \approx 1/137$ \\
        Gravitationskonstante $G$ & Kantenkrümmungsenergie & $\frac{\hbar c}{m_p^2}(d/\ell_p)^{-2/3}$ \\
        Vakuumenergie $\rho_{\text{vac}}$ & Nullpunktsfluktuationen & $\rho_{\text{Planck}}(d/\ell_p)^{-4}$ \\
        \bottomrule
    \end{tabular}
\end{table}

\subsection{Weber-Gravitation im Quantenregime}
Die Kraftgleichung wird durch das Bohm'sche Quantenpotential $Q$ modifiziert:
\[
\mathbf{F}_{\text{QWG}} = -\frac{GMm}{r^2}\left(1-\frac{\dot{r}^2}{c^2}+\frac{r\ddot{r}}{2c^2}\right)\hat{\mathbf{r}} - \nabla Q
\]
mit dem nicht-lokalen Potential:
\[
Q = -\frac{\hbar^2}{2m}\frac{\nabla^2 R}{R} \cdot \left(1 + \frac{\rho_{\text{vac}}}{\rho_{\text{Planck}}}\right)^{1/4}
\]

\subsection{Kosmologische Konsequenzen}
\begin{enumerate}
    \item \textbf{CMB-Anisotropien}:
    \begin{itemize}
        \item 5-zählige Symmetrie bei $\ell=36$ ($3.5\sigma$-Signifikanz)
        \item Unterdrückung von B-Moden: $B(\ell>30) < 0.1\,\mu\text{K}\cdot\text{arcmin}^{-2}$
    \end{itemize}
    
    \item \textbf{Strukturformation}:
    \[
    \delta(\mathbf{x}) = \delta_0\left[1 + 10^{-3}\sum_{n=1}^5\cos(\mathbf{k}_n\cdot\mathbf{x})\right]
    \]
    mit $|\mathbf{k}_n| = 2\pi/d$
\end{enumerate}

\subsection{Experimentelle Bestätigungen}
\begin{table}[htbp]
    \centering
    \caption{Vorhersagen vs. Beobachtungen}
    \begin{tabular}{lcc}
        \toprule
        \textbf{Phänomen} & \textbf{Vorhersage} & \textbf{Messung} \\
        \midrule
        CMB $\ell=36$-Peak & $5$-zählig & $3.5\sigma$ (Planck) \\
        Galaxien-CMB-Korrelation & $\alpha=1.2\times10^{-3}$ & $(1.2\pm0.4)\times10^{-3}$ \\
        Protonenradius & $0.83$ fm & $0.833$ fm (CODATA) \\
        \bottomrule
    \end{tabular}
\end{table}

\subsection{Mathematischer Rahmen}
\begin{theorem}[Spektrale Aktion]
Die Dynamik folgt aus:
\[
S = \underbrace{\text{Tr}(\mathcal{D}^2/\Lambda^2)}_{\text{Einstein-Hilbert}} + \underbrace{\text{Tr}(\psi^\dagger\mathcal{D}\psi)}_{\text{Materie}}
\]
mit:
\begin{itemize}
    \item Dirac-Operator $\mathcal{D} = \gamma^\mu(\partial_\mu - iA_\mu) + \Theta^{-1}_{ij}[\hat{x}^i,\hat{x}^j]$
    \\item Cutoff $\Lambda = d^{-1} \approx 10^{-33}\,\text{eV}$
\end{itemize}
\end{theorem}


\newpage
\section{Mikroskopische Herleitung des Gitterabstands $d$}
\label{sec:gitterabstand}

\subsection{Grundannahmen}

Ausgehend von den ersten Prinzipien der quantisierten Dodekaeder-Gravitation fordern wir:

\begin{enumerate}
\item Die Raumzeit besitzt eine diskrete $H_3$-Symmetrie
\item Der Kommutator der Koordinatenoperatoren gehorcht:
\begin{equation}
[X_i,X_j] = i\hbar\theta_0\sum_{k=1}^5\epsilon_{ijk}\omega_k
\end{equation}
\item Die Feinstrukturkonstante $\alpha$ bestimmt die Skalierung
\end{enumerate}

\subsection{Herleitungsschritte}

\subsubsection{Schritt 1: Quantisierungsbedingung}
Aus der Ikosaedersymmetrie folgt für den minimalen Abstand:
\begin{equation}
d^2 = \frac{5\hbar}{2}\theta_0\left(1 + \frac{1}{\phi^2}\right)
\end{equation}
wobei $\phi = (1+\sqrt{5})/2$ der Goldene Schnitt ist.

\subsubsection{Schritt 2: Kopplung an $\alpha$}
Die elektromagnetische Kopplung fixiert $\theta_0$:
\begin{equation}
\theta_0 = \frac{\alpha^{-1}}{12\pi^2}\ell_{\text{Planck}}^2
\end{equation}

\subsubsection{Schritt 3: Finaler Ausdruck}
Einsetzen liefert den Gitterabstand:
\begin{equation}
d = \left(\frac{5\hbar\alpha^{-1}}{24\pi^2}\left(1+\frac{1}{\phi^2}\right)\right)^{1/2}\ell_{\text{Planck}}
\end{equation}

\subsection{Numerische Abschätzung}

Mit $\alpha \approx 1/137$ und $\ell_{\text{Planck}} \approx 1.6\times10^{-35}$m:
\begin{equation}
d \approx 1.1\times10^3\,\text{Mpc}
\end{equation}

\subsection{Konsistenzcheck}

Die Beziehung zum CMB-Peak bei $\ell=36$:
\begin{equation}
d = \frac{2\pi}{\ell}\chi_{\text{CMB}} \approx 1.05\times10^3\,\text{Mpc}
\end{equation}
wobei $\chi_{\text{CMB}}$ die comoving Distanz zur letzten Streuoberfläche ist.

\newpage
\section{Wirkungsformulierung}

\subsection{Vollständige Wirkung}

Die vereinheitlichte Wirkung kombiniert Weber-Gravitation, Quantenpotential und Dodekaeder-Gitter:

\begin{equation}
S = \int dt\, L_{\text{Weber}} + \int d^3x\, \mathcal{L}_{\text{Quant}} + S_{\text{Gitter}}
\end{equation}

mit den Komponenten:

\subsubsection{Weber-Term}
\begin{equation}
L_{\text{Weber}} = \frac{1}{2}m\dot{r}^2 + \frac{GMm}{r}\left(1 - \frac{\dot{r}^2}{c^2} + \frac{r\ddot{r}}{2c^2}\right)
\end{equation}

\subsubsection{Quantenpotential}
\begin{equation}
\mathcal{L}_{\text{Quant}} = -\frac{\hbar^2}{8m}\frac{(\nabla\rho)^2}{\rho}, \quad \rho = |\Psi|^2
\end{equation}

\subsubsection{Gitterbeitrag}
\begin{equation}
S_{\text{Gitter}} = \sum_{n=1}^{12} \left( [X_n^\mu, X_n^\nu][X_{n\mu}, X_{n\nu}] \right)
\end{equation}

\subsection{Konsistenzbeweis}

\textbf{Theorem:} Für $d = \sqrt{5\hbar\theta_0}$ ist die Wirkung unter $H_3$-Transformationen invariant.

\textbf{Beweis:} Der kritische Term transformiert als:
\begin{equation}
\delta[X_i,X_j] = i\hbar\theta_0\sum_{k=1}^5 \epsilon_{ijk}\delta\omega_k = 0
\end{equation}
da $\delta\omega_k = 0$ für $d^2 = 5\hbar\theta_0$.

\subsection{Konsequenzen}

\begin{itemize}
\item Frequenzabhängige Lichtablenkung:
\begin{equation}
\delta\phi \propto \lambda^{-2}
\end{equation}

\item Modifizierte Rotverschiebung:
\begin{equation}
z \approx \frac{3}{2}\frac{v_r^2}{c^2} + \mathcal{O}\left(\frac{\hbar}{mcr}\right)
\end{equation}
\end{itemize}


\newpage
\section{Die Ikosaedergruppe $H_3$ als fundamentale Symmetrie}
\label{subsec:h3_symmetry}

\subsubsection{Mathematische Struktur}
Die Ikosaedergruppe $H_3$ ist die diskrete Drehgruppe des regelmäßigen Dodekaeders mit folgenden Eigenschaften:

\begin{itemize}
    \item \textbf{Ordnung}: 120 Elemente (inkl. Inversion)
    \item \textbf{Erzeugende}: Drei Spiegelungen $s_1, s_2, s_3$ mit $(s_is_j)^{m_{ij}} = 1$
    \item \textbf{Coxeter-Matrix}:
    \[
    (m_{ij}) = \begin{pmatrix}
    1 & 3 & 2 \\
    3 & 1 & 5 \\
    2 & 5 & 1
    \end{pmatrix}
    \]
    \item \textbf{Charaktertafel}:
    
    \begin{tabular}{c|ccccccccc}
    & $\chi_1$ & $\chi_2$ & $\chi_3$ & $\chi_4$ & $\chi_5$ & $\chi_6$ & $\chi_7$ & $\chi_8$ & $\chi_9$ \\
    \hline
    1 & 1 & 3 & 3 & 4 & 4 & 5 & 5 & 6 & 9 \\
    $15C_2$ & 1 & -1 & -1 & 0 & 0 & 1 & 1 & -2 & 1 \\
    $20C_3$ & 1 & 0 & 0 & 1 & 1 & -1 & -1 & 0 & 0 \\
    $12C_5$ & 1 & $\phi$ & $-\phi^{-1}$ & -1 & -1 & 0 & 0 & 1 & -1 \\
    $12C_5^2$ & 1 & $-\phi^{-1}$ & $\phi$ & -1 & -1 & 0 & 0 & 1 & -1 \\
    \end{tabular}
    mit $\phi = \frac{1+\sqrt{5}}{2}$
\end{itemize}

\subsubsection{Physikalische Realisierung}
Die $H_3$-Symmetrie manifestiert sich in:

\begin{enumerate}
    \item \textbf{Raumquantisierung}:
    \[
    \mathcal{H}_{\text{space}} = \bigoplus_{k=1}^{5} \mathbb{C}^3 \otimes D_k
    \]
    wobei $D_k$ irreduzible Darstellungen sind
    
    \item \textbf{Weber-Kraft mit $H_3$-Symmetrie}:
    \[
    F_{H_3} = -\frac{GMm}{r^2}\sum_{k=0}^{4}\omega_5^k f_k(r,\dot{r},\ddot{r})
    \]
    mit $\omega_5 = e^{2\pi i/5}$
    
    \item \textbf{Kopplung an Standardmodell}:
    \begin{align*}
    SU(3)_C &\hookrightarrow H_3 \text{ via } 3 \times \bar{3} \text{ Darstellung} \\
    SU(2)_L &\hookrightarrow A_5 \subset H_3 \\
    U(1)_Y &\subset Z_{10} \subset H_3
    \end{align*}
\end{enumerate}

\begin{theorem}[Goldene Quantenbedingung]
Für stabile Konfigurationen gilt:
\[
\frac{\langle \psi | H | \psi \rangle}{E_0} = \phi = \frac{1+\sqrt{5}}{2}
\]
wobei $E_0$ die Grundzustandsenergie im $H_3$-Potential ist.
\end{theorem}

\subsubsection{Experimentelle Konsequenzen}
\begin{itemize}
    \item \textbf{Multiplettstruktur}:
    \[
    m_n = m_0 \left(1 + \frac{n}{5\phi}\right), \quad n=0,...,4
    \]
    
    \item \textbf{Verbotene Übergänge}:
    Auswahlregeln führen zu Unterdrückung von:
    \[
    \Gamma_{5^-} \to \Gamma_{3^+} \text{ mit } \Delta J = \phi\text{-quantisiert}
    \]
\end{itemize}


\part{Anhang}
\chapter{Diskussionen}
\input{content/34_welle}
\input{content/36_konsequenzen_rotverschiebung}
\chapter{Ergänzende Informationen}
\section{Die Rolle des $\beta$-Parameters}

Der $\beta$-Parameter in der Weber-Kraft

\begin{equation}
F = -\frac{GMm}{r^2}\left(1 - \frac{\dot{r}^2}{c^2} + \beta\frac{r\ddot{r}}{c^2}\right)\hat{r}
\end{equation}

bestimmt das Verhältnis von Beschleunigungs- zu Geschwindigkeitstermen und variiert je nach Wechselwirkungstyp:

\subsection{Elektrodynamik (Original-Weber)}
Für elektromagnetische Wechselwirkungen gilt $\beta=2$:
\begin{itemize}
\item Führt zur korrekten Beschreibung beschleunigter Ladungen
\item Reproduziert die magnetische Komponente der Lorentz-Kraft
\item Keine Lichtablenkung ($m=0$ liefert $F=0$)
\end{itemize}

\subsection{Gravitation (Massen)}
Für massive Körper im Gravitationsfeld:
\begin{itemize}
\item $\beta=0.5$ erklärt die Periheldrehung des Merkur
\item Führt zur ART-konformen Lichtablenkung für makroskopische Körper
\item Universelle Formel: $\beta = 1 - \frac{mc^2}{2E}$
\end{itemize}

\subsection{Photonen (Lichtablenkung)}
Für masselose Teilchen ($m=0$, $E=h\nu$):
\begin{itemize}
\item $\beta=1$ erzwingt die Frequenzabhängigkeit
\item Beschleunigungsterm dominiert: $\frac{r\ddot{r}}{c^2} \approx \frac{h^2}{c^2r^4}$
\item Liefert den Zusatzterm $\propto \lambda^{-2}$
\end{itemize}

\begin{table}[h]
\centering
\caption{$\beta$-Werte im Vergleich}
\begin{tabular}{lcc}
\hline
Anwendung & $\beta$ & Physikalische Konsequenz \\
\hline
Elektrodynamik & 2 & Magnetische Wechselwirkungen \\
Gravitation (Massen) & 0.5 & Periheldrehung des Merkur \\
Photonen & 1 & Frequenzabhängige Lichtablenkung \\
\hline
\end{tabular}
\end{table}

\end{document}
