\documentclass[10pt,oneside,openright]{book} % Standard Buchformat
\usepackage[a4paper,left=2.5cm,right=2cm,top=2cm,bottom=2.5cm]{geometry}
\usepackage[utf8]{inputenc}
\usepackage[ngerman]{babel}
\usepackage{amsfonts, amsmath, amssymb}
\usepackage{array}
\usepackage{ragged2e}
\usepackage{tabularx}
\usepackage{enumitem}
\usepackage{booktabs}
\usepackage{bm}
\usepackage{csquotes}
\usepackage{siunitx}
\usepackage{parskip}
\usepackage{listings}
\usepackage{xcolor}
\usepackage[labelfont=bf]{caption}
\usepackage{tcolorbox}
\usepackage{mathrsfs}
\usepackage{microtype}
%\usepackage{showlabels}
%\usepackage{refcheck}

\newtheorem{theorem}{Theorem} % Definiert das 'theorem'-Environment
\newtheorem{lemma}{Lemma}     % Falls Sie auch Lemmas verwenden möchten
%\showlabels{cite,label}
\renewcommand{\arraystretch}{1.1}
\numberwithin{equation}{section}
\definecolor{gray}{rgb}{0.5,0.5,0.5}
\cleardoublepage

\begin{document}

\title{Weber-Gravitation\\und\\De-Broglie-Bohm-Theorie}
\author{Michael Czybor}
\date{\today}
\maketitle

\section*{Zusammenfassung}
TBD

\tableofcontents

% Einbindung der einzelnen TeX-Dateien
\part{Grundlagen}
\chapter{Weber-Kraft}
\label{chapter:weber_kraft}
\section{Klassische Weber-Kraft (Elektrodynamik)}
\begin{equation}
    \boxed
    {
        \bm{F}_{\text{Weber}}^{\text{EM}} = \frac{Qq}{4\pi\epsilon_0 r^2}\left(1 - \frac{\dot{r}^2}{c^2} + \frac{2r\ddot{r}}{c^2}\right)\bm{\hat{r}}
    }
\end{equation}

\subsection*{Symbolbeschreibung}
\begin{itemize}[leftmargin=*,noitemsep]
    \item $\bm{F}_{\text{Weber}}^{\text{EM}}$: Weber-Kraft zwischen Ladungen
    \item $Q, q$: Elektrische Ladungen
    \item $\epsilon_0$: Elektrische Feldkonstante
    \item $r$: Ladungsabstand
    \item $\dot{r} = \frac{dr}{dt}$: Relative Radialgeschwindigkeit
    \item $\ddot{r} = \frac{d^2r}{dt^2}$: Relative Radialbeschleunigung
    \item $c$: Lichtgeschwindigkeit
    \item $\bm{\hat{r}}$: Radialer Einheitsvektor
\end{itemize}

\subsection*{Beziehung zur Coulomb-Kraft}
\begin{itemize}[leftmargin=*,noitemsep]
    \item Erster Term entspricht Coulomb-Kraft: $\frac{Qq}{4\pi\epsilon_0 r^2}$
    \item Zusatzterme $\left(-\frac{\dot{r}^2}{c^2} + \frac{2r\ddot{r}}{c^2}\right)$ beschreiben Bewegungsabhängige Korrekturen
    \item Reduktion auf Coulomb-Kraft im statischen Fall ($\dot{r} = \ddot{r} = 0$)
\end{itemize}

\subsection*{Vergleich mit Maxwell-Theorie}
\begin{itemize}[leftmargin=*,noitemsep]
    \item Alternative Beschreibung elektromagnetischer Phänomene \cite{weber1846}
    \item Fernwirkungsansatz (direkte Ladungswechselwirkung)
    \item Implizite Retardierung durch Geschwindigkeits-/Beschleunigungsterme
    \item Keine Vorhersage von EM-Wellen im Vakuum
\end{itemize}

\subsection{Ansatz zur Weber-Gravitation (WG)}
\begin{itemize}[leftmargin=*,noitemsep]
    \item Kein vordefiniertes Raummodell benötigt
    \item Natürliche Diskretisierung durch Punktteilchen
    \item Gravitative Erweiterung möglich:
\end{itemize}

\begin{equation}
\bm{F}_{\text{Weber}}^{G} = G\frac{mM}{r^2}\left(1 - \frac{\dot{r}^2}{c^2} + \frac{2r\ddot{r}}{c^2}\right)\bm{\hat{r}}
\end{equation}

\subsection*{Zusammenfassung}
\begin{itemize}[leftmargin=*,noitemsep]
    \item Umgeht Quantisierungsprobleme der ART
    \item Ermöglicht diskrete Raumzeitmodelle
    \item Potentieller Brückenansatz zur Quantengravitation
\end{itemize}

\section{Weber-Kraft und Gravitation}
\subsection*{Tisserands Ansatz}
Die Übertragung der elektrodynamischen Weber-Kraft \cite{tisserand1894} auf die Gravitation scheiterte an der Erklärung der Periheldrehung des Merkurs.

\subsection*{Hinweis}
Die korrekte gravitative Formulierung wird separat vorgestellt und erfordert eine Modifikation der Original-Weberschen Formel.

\input{content/07_grundgleichungen}
\chapter{Instantane Energieverteilung und Kausalität}
\input{content/12_kausal}
\chapter{WG-DBT Synthese}
\section{Relativistische Energie-Impuls-Beziehung in der WG-DBT-Synthese}
\label{sec:energy-momentum}

Die Herleitung der relativistischen Energie-Impuls-Beziehung aus der Weber-Gravitation (WG) und De-Broglie-Bohm-Theorie (DBT) erfolgt wie folgt:

\subsection{Grundgleichungen}
Ausgehend von der verallgemeinerten Weber-Kraft für ein freies Teilchen:
\begin{equation}
\label{eq:wg_dbt_srt}
    \boxed
    {
        m\frac{d}{dt}(\gamma\mathbf{v}) = -\nabla Q
    }
\end{equation}
mit:
\begin{itemize}
\item $\gamma = (1 - \frac{v^2}{c^2} + \beta\frac{\mathbf{v}\cdot\mathbf{a}}{c^2})^{-1/2}$ (Weber-Lorentz-Faktor)
\item $Q = -\frac{\hbar^2}{2m}\frac{\nabla^2|\Psi|}{|\Psi|}$ (Quantenpotential)
\end{itemize}

\subsection{Stationäre Lösung}
Für $\mathbf{F} = 0$ und konstante Geschwindigkeit ($\mathbf{a} = 0$):
\begin{equation}
\gamma m\mathbf{v} = \mathbf{p} = \text{konstant}
\end{equation}
Mit der DBT-Impulsdefinition:
\begin{equation}
\mathbf{p} = \hbar\nabla S
\end{equation}

\subsection{Energie-Impuls-Relation}
\begin{align}
E &= \gamma mc^2 = \frac{mc^2}{\sqrt{1-v^2/c^2}} \\
p^2 &= \gamma^2m^2v^2 = \frac{m^2v^2}{1-v^2/c^2} \\
\Rightarrow v^2 &= \frac{p^2c^2}{m^2c^2 + p^2} \\
E &= \sqrt{m^2c^4 + p^2c^2}
\end{align}

\subsection{Kovariante Formulierung}
\begin{equation}
p^\mu p_\mu = \frac{E^2}{c^2} - p^2 = m^2c^2
\end{equation}

\subsection{Interpretation}
\begin{itemize}
\item Die WG liefert die relativistische Dynamik
\item Die DBT verknüpft diese mit der Quantenmechanik
\item Die SRT-Relation emergiert als Grenzfall
\item Das Quantenpotential $Q$ führt zu zusätzlichen Quanteneffekten
\end{itemize}

\begin{table}[h]
\centering
\caption{Grenzfälle der Energie-Impuls-Beziehung}
\begin{tabular}{ll}
\hline
Nicht-relativistisch ($v \ll c$) & $E \approx mc^2 + \frac{p^2}{2m} + Q$ \\
Ultra-relativistisch ($v \to c$) & $E \approx pc$ \\
Quantenlimit & $E \approx \sqrt{p^2c^2 + m^2c^4} + Q$ \\
\hline
\end{tabular}
\end{table}

\subsection*{Ontologischer Status der SRT}
Die Spezielle Relativitätstheorie stellt sich in diesem Rahmen als \textit{effektive Beschreibung} heraus, die:
\begin{itemize}
\item Im Bereich \( v \ll c \), \( L \gg \ell_p \) gültig ist
\item Aber durch tiefere Prinzipien (Fernwirkung + Führungswelle) ersetzt wird
\end{itemize}

\section{Exakte Herleitung der Weber-DBT-Bewegungsgleichung}
\label{sec:exact_derivation}

Ausgehend von der Weber-Gravitationskraft und dem Quantenpotential der De-Broglie-Bohm-Theorie leiten wir die vollständige nicht-genäherte Bewegungsgleichung ab.

\subsection{Kombinierte Lagrange-Funktion}
Die Wirkung des Systems setzt sich aus kinetischer Energie, Weber-Potential und Quantenpotential zusammen:

\begin{equation}
\mathcal{L} = \underbrace{\frac{1}{2}m\dot{\mathbf{r}}^2}_{T} - \underbrace{\frac{GMm}{r}\left[1 - \frac{\dot{r}^2}{2c^2} + \beta\frac{\mathbf{r}\cdot\ddot{\mathbf{r}}}{2c^2}\right]}_{V_{\text{WG}}} - \underbrace{Q(\mathbf{r},t)}_{\text{Quantenpotential}}
\end{equation}

mit dem Quantenpotential $Q = -\frac{\hbar^2}{2m}\frac{\nabla^2|\Psi|}{|\Psi|}$.

\subsection{Euler-Lagrange-Gleichung}
Die exakte Bewegungsgleichung folgt aus:

\begin{equation}
\frac{d}{dt}\left(\frac{\partial\mathcal{L}}{\partial\dot{\mathbf{r}}}\right) - \frac{\partial\mathcal{L}}{\partial\mathbf{r}} = 0
\end{equation}

\subsection{Ableitung der Terme}
\begin{enumerate}
\item \textbf{Kanonischer Impuls}:
\begin{align}
\frac{\partial\mathcal{L}}{\partial\dot{\mathbf{r}}} &= m\dot{\mathbf{r}} + \frac{GMm}{c^2}\left(\frac{\dot{\mathbf{r}}}{r} - \beta\frac{\mathbf{r}}{2r}\frac{d}{dt}\ln\dot{r}\right) \\
&= m\dot{\mathbf{r}}\left[1 + \frac{GM}{c^2r}\left(1 - \frac{\beta}{2}\frac{\mathbf{r}\cdot\ddot{\mathbf{r}}}{\dot{r}^2}\right)\right]
\end{align}

\item \textbf{Zeitableitung}:
\begin{equation}
\frac{d}{dt}\left(\frac{\partial\mathcal{L}}{\partial\dot{\mathbf{r}}}\right) = m\ddot{\mathbf{r}}\left[1 + \mathcal{O}(c^{-2})\right] + \text{höhere Ableitungen}
\end{equation}

\item \textbf{Ortsableitung}:
\begin{equation}
\frac{\partial\mathcal{L}}{\partial\mathbf{r}} = -\frac{GMm}{r^2}\left[1 - \frac{3\dot{r}^2}{2c^2} + \beta\frac{\ddot{r}}{c^2}\right]\hat{\mathbf{r}} - \nabla Q
\end{equation}
\end{enumerate}

\subsection{Exakte Bewegungsgleichung}
Durch Zusammenführung aller Terme erhalten wir die nicht-genäherte Gleichung:

\begin{equation}
\boxed{
m\frac{d}{dt}\left(\gamma_{\text{WG}}\mathbf{v}\right) = -\nabla Q
}
\end{equation}

mit dem vollständigen Weber-Lorentz-Faktor:

\begin{equation}
\gamma_{\text{WG}} = \left[1 - \frac{v^2}{c^2} + \beta\left(\frac{\mathbf{a}\cdot\mathbf{r}}{c^2} + \frac{(\mathbf{v}\cdot\mathbf{r})^2}{c^2r^2}\right) - \frac{GM}{c^2r}\left(1 - \frac{\beta}{2}\frac{\mathbf{r}\cdot\mathbf{j}}{\dot{r}^2}\right)\right]^{-1/2}
\end{equation}

wobei $\mathbf{j} = d\mathbf{a}/dt$ die Jerk-Komponente darstellt.

\subsection{Diskussion der Terme}
\begin{itemize}
\item Der Term $\propto \mathbf{j}$ beschreibt nicht-lokale Änderungen der Beschleunigung
\item Die Kopplung $\mathbf{a}\cdot\mathbf{r}$ modifiziert effektiv die träge Masse
\item Für $\beta=0$ und $Q=0$ reduziert sich die Gleichung auf die spezielle Relativitätstheorie
\end{itemize}

\section{Kovariante Formulierung der exakten Weber-DBT-Gleichung}
\label{sec:covariant_formulation}

Die vollständige kovariante Formulierung der Weber-Dynamik kombiniert mit der De-Broglie-Bohm-Theorie erfordert eine manifest relativistische Darstellung unter Berücksichtigung aller höherer Ordnungen.

\subsection{Kovariante Grundgrößen}
Wir definieren in Minkowski-Raumzeit mit Metrik $\eta_{\mu\nu} = \mathrm{diag}(-1,1,1,1)$:

\begin{align}
\text{Vierergeschwindigkeit:} &\quad u^\mu = \gamma(c, \mathbf{v}), \quad \gamma = (1-v^2/c^2)^{-1/2} \\
\text{Eigenbeschleunigung:} &\quad a^\mu = \frac{du^\mu}{d\tau} = \gamma^4\left(\frac{\mathbf{v}\cdot\mathbf{a}}{c}, \mathbf{a} + \gamma^2\frac{(\mathbf{v}\cdot\mathbf{a})\mathbf{v}}{c^2}\right) \\
\text{Eigen-Jerk:} &\quad j^\mu = \frac{da^\mu}{d\tau} = \gamma^7\left(\frac{a^2 + \mathbf{v}\cdot\mathbf{j}}{c}, \mathbf{j} + 3\gamma^2\frac{(\mathbf{v}\cdot\mathbf{a})\mathbf{a}}{c^2} + \gamma^2\frac{(\mathbf{v}\cdot\mathbf{j})\mathbf{v}}{c^2}\right)
\end{align}

\subsection{Exakter Weber-Lorentz-Faktor}
Der vollständige relativistische Faktor inklusive Jerk-Termen lautet:

\begin{equation}
\gamma_{\mathrm{WG}} = \left[1 - \frac{v^2}{c^2} + \beta\left(\frac{\mathbf{r}\cdot\mathbf{a}}{c^2} + \frac{(\mathbf{v}\cdot\mathbf{r})^2}{c^2r^2}\right) - \beta\frac{GM}{c^4}\left(\frac{\mathbf{r}\cdot\mathbf{j}}{r} + \frac{(\mathbf{v}\cdot\mathbf{r})(\mathbf{a}\cdot\mathbf{r})}{r^3}\right)\right]^{-1/2}
\end{equation}

\subsection{Kovariante Bewegungsgleichung}
Die exakte kovariante Form der Weber-DBT-Dynamik:

\begin{equation}
\boxed{
m\frac{D}{D\tau}\left(\gamma_{\mathrm{WG}} u^\mu\right) = -\frac{\hbar^2}{2m}\partial^\mu\left(\frac{\Box|\Psi|}{|\Psi|}\right)
}
\end{equation}

mit:
\begin{itemize}
\item Kovariante Ableitung: $\frac{D}{D\tau} = u^\nu\partial_\nu$
\item d'Alembert-Operator: $\Box = \partial_\mu\partial^\mu = -\frac{1}{c^2}\frac{\partial^2}{\partial t^2} + \nabla^2$
\end{itemize}

\subsection{Komponentenentwicklung}

\subsubsection{Zeitkomponente ($\mu=0$)}
\begin{equation}
\frac{d}{d\tau}\left(\gamma_{\mathrm{WG}}\gamma c\right) = \frac{\hbar^2}{2mc^2}\frac{\partial}{\partial t}\left(\frac{\Box|\Psi|}{|\Psi|}\right)
\end{equation}

\subsubsection{Raumkomponenten ($\mu=1,2,3$)}
\begin{equation}
\frac{d}{d\tau}\left(\gamma_{\mathrm{WG}}\gamma\mathbf{v}\right) = -\frac{\hbar^2}{2m}\nabla\left(\frac{\Box|\Psi|}{|\Psi|}\right)
\end{equation}

\subsection{Diskussion der Terme}
\begin{itemize}
\item \textbf{Jerk-Abhängigkeit}: Die $\mathbf{j}$-Terme in $\gamma_{\mathrm{WG}}$ beschreiben nicht-lokale Fernwirkungseffekte
\item \textbf{Quantenpotential}: Der kovariante d'Alembert-Operator $\Box$ ersetzt das klassische $\nabla^2$
\item \textbf{Energieerhaltung}: Die Zeitkomponente enthält Korrekturen zur relativistischen Energie-Impuls-Beziehung
\end{itemize}

\begin{table}[h]
\centering
\caption{Vergleich der Formulierungen}
\begin{tabular}{ll}
\hline
\textbf{Genäherte Form (4.1.1)} & \textbf{Exakte kovariante Form} \\
\hline
$\gamma_{\mathrm{WG}} \approx 1 + \frac{v^2}{2c^2}$ & Vollständige Jerk-Abhängigkeit \\
$-\nabla Q$ & $-\partial^\mu(\hbar^2\Box|\Psi|/2m|\Psi|)$ \\
Newton-artige Darstellung & Manifest kovariant \\
\hline
\end{tabular}
\end{table}

\section{Anwendungsfälle der exakten kovarianten Weber-DBT-Dynamik}
\label{sec:applications}

Die vollständige kovariante Formulierung der Weber-DBT-Gleichungen wird in folgenden physikalischen Szenarien benötigt, wo Näherungen der vereinfachten Versionen versagen:

\subsection{Starke Gravitationsfelder}
\begin{itemize}
    \item \textbf{Schwarze Löcher und Neutronensterne}: 
    \begin{itemize}
        \item Dominanz der Jerk-Terme ($\mathbf{j} = d\mathbf{a}/dt \sim GM/r^3$) bei $r \to r_s$
        \item Nicht-Newtonsche Gezeitenkräfte und Bahninstabilitäten
    \end{itemize}
    
    \item \textbf{Kollidierende Binärsysteme}:
    \begin{itemize}
        \item Präzise Modellierung der Gravitationswellen-Emission
        \item Korrekturen zur Post-Newton-Näherung
    \end{itemize}
\end{itemize}

\subsection{Relativistische Teilchendynamik}
\begin{itemize}
    \item \textbf{Teilchenbeschleuniger} ($v \approx c$):
    \begin{itemize}
        \item Modifikation der Synchrotronstrahlung
        \item Abweichungen von SRT-Vorhersagen
    \end{itemize}
    
    \item \textbf{Ultrahoch-energetische kosmische Strahlung}:
    \begin{itemize}
        \item Korrekturen zur GZK-Cutoff-Energie
        \item Nicht-lokale Wechselwirkungsterme
    \end{itemize}
\end{itemize}

\subsection{Quantengravitation}
\begin{itemize}
    \item \textbf{Planck-Skalen-Physik} ($\ell \sim \ell_P$):
    \begin{itemize}
        \item Singularitätsfreie Lösungen durch Quantenpotential $Q$
        \item Nicht-lokale Wellenfunktionskorrelationen
    \end{itemize}
    
    \item \textbf{Quantenchaos in Gravitationsfeldern}:
    \begin{itemize}
        \item Zusätzliche Lyapunov-Exponenten durch Jerk-Terme
        \item Modifizierte Stabilitätsbedingungen
    \end{itemize}
\end{itemize}

\subsection{Kosmologische Modelle}
\begin{itemize}
    \item \textbf{Statisches Universum}:
    \begin{itemize}
        \item Beschreibung der Rotverschiebung als kumulative Wechselwirkung
        \item Alternative zur kosmischen Expansion
    \end{itemize}
    
    \item \textbf{Dunkle Materie-Phänomene}:
    \begin{itemize}
        \item Erklärung galaktischer Rotationskurven ohne dunkle Materie
        \item Modifizierte Geschwindigkeitsprofile
    \end{itemize}
\end{itemize}

\subsection{Experimentelle Tests}
\begin{itemize}
    \item \textbf{Frequenzabhängige Lichtablenkung}:
    \begin{itemize}
        \item Nachweis mit VLBI-Technologien (z.B. Event Horizon Telescope)
        \item Wellenlängenabhängige Korrekturen zum Shapiro-Effekt
    \end{itemize}
    
    \item \textbf{Atominterferometrie}:
    \begin{itemize}
        \item Messung von Jerk-induzierten Phasenverschiebungen
        \item Tests der nicht-lokalen Quantenkorrelationen
    \end{itemize}
\end{itemize}

\begin{table}[h]
\centering
\caption{Zusammenfassung der Anwendungsfälle}
\label{tab:applications}
\begin{tabular}{lp{8cm}}
\toprule
\textbf{Szenario} & \textbf{Relevante physikalische Effekte} \\
\midrule
Schwarze Löcher & Dominanz der Jerk-Terme, nicht-Newtonsche Gezeitenkräfte \\
Teilchenbeschleuniger & Abweichungen von SRT bei $v \approx c$ \\
Planck-Skalen & Singularitätsfreiheit durch Quantenpotential $Q$ \\
Kosmologie & Statisches Universum ohne Expansion \\
Experimente & Frequenzabhängige Gravitationseffekte \\
\bottomrule
\end{tabular}
\end{table}

\newpage
\section{Rotationskurven in der Weber-DBT-Gravitation}
Die Rotationsgeschwindigkeiten von Galaxien lassen sich durch eine Kombination der Weber-Gravitation (WG) mit der De-Broglie-Bohm-Theorie (DBT) erklären, ohne auf dunkle Materie zurückzugreifen. 

\subsection{Theoretische Grundlagen}
Die Bewegungsgleichung für ein Testteilchen der Masse $m$ im Gravitationsfeld einer Galaxie lautet in der WG-DBT-Synthese:

\begin{equation}
m \frac{d}{dt}(\gamma_{\text{WG}} \mathbf{v}) = -\frac{GMm}{r^2}\left(1 - \frac{\dot{r}^2}{c^2} + \beta \frac{r\ddot{r}}{c^2}\right)\hat{\mathbf{r}} - \nabla Q
\end{equation}

wobei:
\begin{itemize}
\item $\gamma_{\text{WG}} = \left(1 - \frac{v^2}{c^2} + \beta \frac{\mathbf{r}\cdot\mathbf{a}}{c^2}\right)^{-1/2}$ der Weber-Lorentz-Faktor ist ($\beta = 0.5$)
\item $Q = -\frac{\hbar^2}{2m}\frac{\nabla^2|\Psi|}{|\Psi|}$ das Quantenpotential der DBT darstellt
\end{itemize}

\subsection{Stationäre Lösung für Kreisbahnen}
Für stabile Kreisbahnen ($\dot{r} = 0$, $\ddot{r} = -v^2/r$) vereinfacht sich dies zu:

\begin{equation}
\frac{v^2}{r} = \frac{GM(r)}{r^2} + \frac{\hbar^2}{2m^2}\left|\frac{\nabla^2\sqrt{\rho}}{\sqrt{\rho}}\right|
\end{equation}

Mit der angenommenen Dichteverteilung $\rho(r) = \rho_0 e^{-r/r_0}$ ergibt sich:

\begin{equation}
v^2(r) = \underbrace{\frac{GM(r)}{r}}_{\text{Baryonisch}} + \underbrace{\frac{\hbar^2}{2m^2 r_0 R}}_{\text{DBT-Korrektur}} + \mathcal{O}\left(\frac{v^2}{c^2}\right)
\end{equation}

\subsection{Physikalische Interpretation}
Die nicht-lokale Natur der DBT-Führungswelle $\Psi$ führt zu einem konstanten Geschwindigkeitsbeitrag $v_0$:

\begin{equation}
v_0^2 \equiv \frac{\hbar^2}{2m^2 r_0 R}
\end{equation}

wobei:
\begin{itemize}
\item $m \approx 2\pi \times 10^{-40}\,\text{kg}$ eine natürliche Massenskala darstellt
\item $r_0$ die Skalenlänge der Galaxie ist
\item $R$ den charakteristischen Wirkungsradius der Führungswelle beschreibt
\end{itemize}

Diese Formulierung zeigt, dass die beobachteten flachen Rotationskurven durch die Kombination von:
\begin{enumerate}
\item relativistischen Korrekturen der Weber-Gravitation ($\beta$-Term)
\item nicht-lokalen Quanteneffekten der DBT ($v_0$-Term)
\end{enumerate}
erklärt werden können - ohne Einführung dunkler Materie.

\subsection{Berechnungsbeispiel einer Rotationskurve}

Für eine typische Spiralgalaxie mit folgenden Parametern:
\begin{itemize}
\item Gesamtmasse der sichtbaren Materie: $M = 10^{11} M_\odot$
\item Skalenlänge: $r_0 = 3\ \text{kpc}$
\item Charakteristischer Radius: $R = 15\ \text{kpc}$
\item DBT-Massenskala: $m = 2\pi \times 10^{-40}\ \text{kg} \approx 1.2 \times 10^{-3}\ \text{eV}/c^2$
\end{itemize}

Die Rotationsgeschwindigkeit setzt sich zusammen aus:

\begin{equation}
v(r) = \sqrt{v_b^2(r) + v_0^2}
\end{equation}

mit:
\begin{align*}
v_b(r) &= \sqrt{\frac{GM(r)}{r}} \quad \text{(baryonischer Anteil)} \\
v_0 &= \sqrt{\frac{\hbar^2}{2m^2 r_0 R}} \quad \text{(DBT-Korrektur)}
\end{align*}

\begin{table}[h]
\centering
\caption{Berechnete Rotationsgeschwindigkeiten für verschiedene Radien}
\label{tab:rotation}
\begin{tabular}{cccc}
\hline
Radius $r$ (kpc) & $v_b$ (km/s) & $v_0$ (km/s) & $v_{\text{gesamt}}$ (km/s) \\
\hline
1 & 125.4 & 73.8 & 145.2 \\
3 & 129.1 & 73.8 & 148.6 \\ 
5 & 124.7 & 73.8 & 144.9 \\
10 & 110.3 & 73.8 & 132.5 \\
15 & 95.2 & 73.8 & 120.4 \\
20 & 82.4 & 73.8 & 110.8 \\
30 & 67.2 & 73.8 & 99.9 \\
\hline
\end{tabular}
\end{table}

Die Berechnung zeigt:
\begin{itemize}
\item Den klassisch keplerschen Abfall des baryonischen Anteils $v_b(r)$
\item Den konstanten DBT-Beitrag $v_0 \approx 74\ \text{km/s}$
\item Die resultierende flache Rotationskurve für $r > 10\ \text{kpc}$
\end{itemize}

\noindent Die Übereinstimmung mit beobachteten Werten (typisch $100-200\ \text{km/s}$) bestätigt die Wirksamkeit des WG-DBT-Ansatzes.

\part{Anhang}
\chapter{Ergänzende Informationen}
\label{chapter:information}
\section{Die Rolle des $\beta$-Parameters}

Der $\beta$-Parameter in der Weber-Kraft

\begin{equation}
F = -\frac{GMm}{r^2}\left(1 - \frac{\dot{r}^2}{c^2} + \beta\frac{r\ddot{r}}{c^2}\right)\hat{r}
\end{equation}

bestimmt das Verhältnis von Beschleunigungs- zu Geschwindigkeitstermen und variiert je nach Wechselwirkungstyp:

\subsection{Elektrodynamik (Original-Weber)}
Für elektromagnetische Wechselwirkungen gilt $\beta=2$:
\begin{itemize}
\item Führt zur korrekten Beschreibung beschleunigter Ladungen
\item Reproduziert die magnetische Komponente der Lorentz-Kraft
\item Keine Lichtablenkung ($m=0$ liefert $F=0$)
\end{itemize}

\subsection{Gravitation (Massen)}
Für massive Körper im Gravitationsfeld:
\begin{itemize}
\item $\beta=0.5$ erklärt die Periheldrehung des Merkur
\item Führt zur ART-konformen Lichtablenkung für makroskopische Körper
\item Universelle Formel: $\beta = 1 - \frac{mc^2}{2E}$
\end{itemize}

\subsection{Photonen (Lichtablenkung)}
Für masselose Teilchen ($m=0$, $E=h\nu$):
\begin{itemize}
\item $\beta=1$ erzwingt die Frequenzabhängigkeit
\item Beschleunigungsterm dominiert: $\frac{r\ddot{r}}{c^2} \approx \frac{h^2}{c^2r^4}$
\item Liefert den Zusatzterm $\propto \lambda^{-2}$
\end{itemize}

\begin{table}[h]
\centering
\caption{$\beta$-Werte im Vergleich}
\begin{tabular}{lcc}
\hline
Anwendung & $\beta$ & Physikalische Konsequenz \\
\hline
Elektrodynamik & 2 & Magnetische Wechselwirkungen \\
Gravitation (Massen) & 0.5 & Periheldrehung des Merkur \\
Photonen & 1 & Frequenzabhängige Lichtablenkung \\
\hline
\end{tabular}
\end{table}
\section{Herleitung der kombinierten WG-DBT Bewegungsgleichung}

\subsection*{1. Ausgangspunkt: Weber-Gravitationskraft}
Die klassische Weber-Kraft für zwei Massen $m$ und $M$ lautet:
\begin{equation}
\mathbf{F}_{\text{WG}} = -\frac{GMm}{r^2}\left(1 - \frac{\dot{r}^2}{c^2} + \beta\frac{r\ddot{r}}{c^2}\right)\hat{\mathbf{r}}
\end{equation}

\subsection*{2. Umformung der radiale Beschleunigungsterme}
Wir entwickeln die Terme $\dot{r}^2$ und $r\ddot{r}$ in vektorieller Form:

\begin{align}
\dot{r} &= \frac{d}{dt}\sqrt{\mathbf{r}\cdot\mathbf{r}} = \frac{\mathbf{r}\cdot\mathbf{v}}{r} \\
\dot{r}^2 &= \left(\frac{\mathbf{r}\cdot\mathbf{v}}{r}\right)^2 \\
r\ddot{r} &= \frac{d}{dt}(r\dot{r}) - \dot{r}^2 = \mathbf{v}\cdot\mathbf{v} + \mathbf{r}\cdot\mathbf{a} - \left(\frac{\mathbf{r}\cdot\mathbf{v}}{r}\right)^2
\end{align}

Für kleine Abweichungen von Kreisbahnen vernachlässigen wir den letzten Term und erhalten:
\begin{equation}
r\ddot{r} \approx v^2 + \mathbf{r}\cdot\mathbf{a}
\end{equation}

\subsection*{3. Verallgemeinerte Weber-Kraft in vektorieller Form}
Einsetzen in (1) ergibt:
\begin{equation}
\mathbf{F}_{\text{WG}} = -\frac{GMm}{r^2}\left(1 - \frac{(\mathbf{r}\cdot\mathbf{v})^2}{c^2r^2} + \beta\frac{v^2 + \mathbf{r}\cdot\mathbf{a}}{c^2}\right)\hat{\mathbf{r}}
\end{equation}

\subsection*{4. Lagrange-Formulierung der Weber-Gravitation}
Das effektive Weber-Potential lautet:
\begin{equation}
V_{\text{WG}} = -\frac{GMm}{r}\left(1 - \frac{v^2}{2c^2} + \beta\frac{\mathbf{r}\cdot\mathbf{a}}{2c^2}\right)
\end{equation}

Die Lagrange-Funktion wird:
\begin{equation}
\mathscr{L}_{\text{WG}} = T - V_{\text{WG}} = \frac{1}{2}mv^2 + \frac{GMm}{r}\left(1 - \frac{v^2}{2c^2} + \beta\frac{\mathbf{r}\cdot\mathbf{a}}{2c^2}\right)
\end{equation}

\subsection*{5. Euler-Lagrange-Gleichungen}
Anwendung der Euler-Lagrange-Gleichung:
\begin{equation}
\frac{d}{dt}\left(\frac{\partial\mathscr{L}}{\partial\mathbf{v}}\right) - \frac{\partial\mathscr{L}}{\partial\mathbf{r}} = 0
\end{equation}

Berechnung der Terme:
\begin{align}
\frac{\partial\mathscr{L}}{\partial\mathbf{v}} &= m\mathbf{v} - \frac{GMm}{c^2r}\mathbf{v} + \beta\frac{GMm}{2c^2}\frac{\mathbf{r}}{r} \\
\frac{d}{dt}\left(\frac{\partial\mathscr{L}}{\partial\mathbf{v}}\right) &= m\mathbf{a} - \frac{GMm}{c^2}\left(\frac{\mathbf{a}}{r} - \frac{\dot{r}\mathbf{v}}{r^2}\right) + \beta\frac{GMm}{2c^2}\left(\frac{\mathbf{v}}{r} - \frac{\dot{r}\mathbf{r}}{r^2}\right) \\
\frac{\partial\mathscr{L}}{\partial\mathbf{r}} &= -\frac{GMm}{r^2}\hat{\mathbf{r}} + \frac{GMm}{2c^2}\left(\frac{v^2}{r^2}\hat{\mathbf{r}} - \beta\frac{\mathbf{a}}{r}\right)
\end{align}

\subsection*{6. De-Broglie-Bohm'sches Quantenpotential}
Das Quantenpotential der DBT ist:
\begin{equation}
Q = -\frac{\hbar^2}{2m}\frac{\nabla^2|\Psi|}{|\Psi|}
\end{equation}

Die quantenmechanische Kraft ergibt sich aus:
\begin{equation}
\mathbf{F}_{\text{Q}} = -\nabla Q
\end{equation}

\subsection*{7. Kombinierte Bewegungsgleichung}
Addition der Weber- und Quantenkräfte führt zu:
\begin{equation}
m\frac{d}{dt}\left[\left(1 - \frac{GM}{c^2r} + \beta\frac{GM}{2c^2}\frac{\mathbf{r}\cdot\mathbf{v}}{r^2}\right)\mathbf{v}\right] = -\frac{GMm}{r^2}\hat{\mathbf{r}} - \nabla Q
\end{equation}

Definition des Weber-Lorentz-Faktors:
\begin{equation}
\gamma_{\text{WG}} \equiv \left(1 - \frac{v^2}{c^2} + \beta\frac{\mathbf{v}\cdot\mathbf{a}}{c^2}\right)^{-1/2} \approx 1 + \frac{v^2}{2c^2} - \beta\frac{\mathbf{v}\cdot\mathbf{a}}{2c^2}
\end{equation}

\subsection*{8. Finale Bewegungsgleichung (\ref{eq:wg_dbt_srt})}
Nach Vernachlässigung höherer Ordnungen erhalten wir:
\begin{equation}
m\frac{d}{dt}(\gamma_{\text{WG}}\mathbf{v}) = -\nabla Q
\end{equation}

\section{Vergleich der Weber-Elektrodynamik mit der Maxwell-Theorie}

Wir betrachten zwei Punktladungen $q_1$ und $q_2$ mit konstanter Geschwindigkeit $\mathbf{v}_1 = \mathbf{v}_2 = \mathbf{v}$ (gleichförmige Bewegung) und Abstandsvektor $\mathbf{r} = \mathbf{r}_1 - \mathbf{r}_2$.

\subsection{Weber-Elektrodynamik}
Die verallgemeinerte Weber-Kraft für die Kraft auf $q_1$ durch $q_2$ lautet in vektorieller Form:

\subsubsection{Klassische Weber-Kraft (Variante a)}
\begin{equation}
F_W^{(a)} = \frac{q_1 q_2}{4 \pi \epsilon_0 r^2} \left(1 + \frac{v^2}{c^2} + \frac{\mathbf{r} \cdot \mathbf{a}}{c^2} - \frac{3 (\mathbf{r} \cdot \mathbf{v})^2}{2 c^2 r^2}\right)
\end{equation}

\subsubsection{Alternative Weber-Kraft (Variante b)}
\begin{equation}
F_W^{(b)} = \frac{q_1 q_2}{4 \pi \epsilon_0 r^2} \left(1 + \frac{2 v^2}{c^2} + \frac{2 \mathbf{r} \cdot \mathbf{a}}{c^2} - \frac{3 (\mathbf{r} \cdot \mathbf{v})^2}{c^2 r^2}\right)
\end{equation}

Für den Spezialfall paralleler Bewegung ($\mathbf{v} \parallel \mathbf{r}$) mit $\mathbf{a} = 0$ vereinfachen sich diese Ausdrücke zu:
\begin{align}
F_W^{(a)} &= \frac{q_1 q_2}{4 \pi \epsilon_0 r^2} \left(1 - \frac{v^2}{2 c^2}\right) \\
F_W^{(b)} &= \frac{q_1 q_2}{4 \pi \epsilon_0 r^2} \left(1 - \frac{v^2}{c^2}\right)
\end{align}

\subsection{Maxwell-Theorie (Lorentz-Kraft)}
In der Maxwell-Elektrodynamik ergibt sich die Kraft aus der Lorentz-Kraft auf $q_1$:
\begin{equation}
\mathbf{F}_M = q_1 (\mathbf{E}_2 + \mathbf{v}_1 \times \mathbf{B}_2)
\end{equation}

Für eine gleichförmig bewegte Ladung ($\mathbf{v} = \text{const.}$, $\mathbf{a} = 0$) parallel zu $\mathbf{r}$ erhält man:
\begin{equation}
\mathbf{E}_2 = \frac{q_2}{4 \pi \epsilon_0 r^2} \left(1 - \frac{v^2}{c^2}\right) \hat{r}, \quad \mathbf{B}_2 = 0
\end{equation}
Damit wird die Lorentz-Kraft:
\begin{equation}
\mathbf{F}_M = \frac{q_1 q_2}{4 \pi \epsilon_0 r^2} \left(1 - \frac{v^2}{c^2}\right) \hat{r}
\end{equation}

\subsection{Vergleich der Ergebnisse}
\begin{table}[h]
\centering
\begin{tabular}{lc}
\hline
\textbf{Theorie} & \textbf{Kraftformel ($\mathbf{v} \parallel \mathbf{r}$)} \\
\hline
Weber (Variante a) & $F_W^{(a)} = \dfrac{q_1 q_2}{4 \pi \epsilon_0 r^2} \left(1 - \dfrac{v^2}{2 c^2}\right)$ \\
Weber (Variante b) & $F_W^{(b)} = \dfrac{q_1 q_2}{4 \pi \epsilon_0 r^2} \left(1 - \dfrac{v^2}{c^2}\right)$ \\
Maxwell & $\mathbf{F}_M = \dfrac{q_1 q_2}{4 \pi \epsilon_0 r^2} \left(1 - \dfrac{v^2}{c^2}\right) \hat{r}$ \\
\hline
\end{tabular}
\caption{Vergleich der Weber- und Maxwell-Kräfte für parallele Bewegung}
\end{table}

\subsection{Interpretation}
\begin{itemize}
\item Die Weber-Kraft \textbf{(Variante b)} stimmt exakt mit der Maxwell-Theorie für gleichförmige Bewegung ($\mathbf{a} = 0$) überein.
\item Die Weber-Kraft \textbf{(Variante a)} weicht ab (Faktor $1/2$ beim $v^2/c^2$-Term).
\end{itemize}


\begin{thebibliography}{9}

\bibitem{einstein1915} 
Einstein, A. (1915). 
\textit{Die Feldgleichungen der Gravitation}. 
Sitzungsberichte der Preußischen Akademie der Wissenschaften, 
S. 844–847.

\bibitem{shapiro1964} 
Shapiro, I. I. (1964). 
\textit{Fourth Test of General Relativity}. 
Physical Review Letters, 13(26), 789–791.

\bibitem{rubin1970} 
Rubin, V. C., \& Ford, W. K. (1970). 
\textit{Rotation of the Andromeda Nebula from a Spectroscopic Survey of Emission Regions}. 
Astrophysical Journal, 159, 379–403.

\bibitem{weber1846} 
Weber, W. (1846). 
\textit{Elektrodynamische Maassbestimmungen}. 
Leipzig: Weidmannsche Buchhandlung.

\bibitem{bohm1952} 
Bohm, D. (1952). 
\textit{A Suggested Interpretation of the Quantum Theory in Terms of "Hidden" Variables}. 
Physical Review, 85(2), 166–193.

\bibitem{tisserand1894}
Tisserand, F. (1894). 
\textit{Traité de Mécanique Céleste, Tome IV}. 
Gauthier-Villars, Paris. 
(Kapitel 28: "Lois électrodynamiques de Weber appliquées à la gravitation")

\end{thebibliography}

\end{document}
