\documentclass{book}
\usepackage[a4paper,left=2.5cm,right=2cm,top=2cm,bottom=2.5cm]{geometry}
\usepackage[utf8]{inputenc}
\usepackage[ngerman]{babel}
\usepackage{amsfonts}
\usepackage{amsmath}
\usepackage{amssymb}
\usepackage{array}
\usepackage{ragged2e}
\usepackage{tabularx}
\usepackage{enumitem}
\usepackage{booktabs}
\usepackage{bm}
\usepackage{csquotes}
\usepackage{siunitx}
\usepackage{parskip}

\renewcommand{\arraystretch}{1.1}
\numberwithin{equation}{section}

\begin{document}

\title{Weber-Gravitation}
\author{Michael Czybor}
\date{\today}
\maketitle

\tableofcontents

% Einbindung der einzelnen TeX-Dateien
\part{Grundlagen}
\chapter{Weber-Kraft}
\section{Klassische Weber-Kraft (Elektrodynamik)}
\begin{equation}\label{eq:weber_em}
\bm{F}_{\text{Weber}}^{\text{EM}} = \frac{Qq}{4\pi\epsilon_0 r^2}\left(1 - \frac{\dot{r}^2}{c^2} + \frac{2r\ddot{r}}{c^2}\right)\bm{\hat{r}}
\end{equation}

\subsection*{Symbolbeschreibung}
\begin{itemize}[leftmargin=*,noitemsep]
    \item $\bm{F}_{\text{Weber}}^{\text{EM}}$: Weber-Kraft zwischen Ladungen
    \item $Q, q$: Elektrische Ladungen
    \item $\epsilon_0$: Elektrische Feldkonstante
    \item $r$: Ladungsabstand
    \item $\dot{r} = \frac{dr}{dt}$: Relative Radialgeschwindigkeit
    \item $\ddot{r} = \frac{d^2r}{dt^2}$: Relative Radialbeschleunigung
    \item $c$: Lichtgeschwindigkeit
    \item $\bm{\hat{r}}$: Radialer Einheitsvektor
\end{itemize}

\subsection*{Beziehung zur Coulomb-Kraft}
\begin{itemize}[leftmargin=*,noitemsep]
    \item Erster Term entspricht Coulomb-Kraft: $\frac{Qq}{4\pi\epsilon_0 r^2}$
    \item Zusatzterme $\left(-\frac{\dot{r}^2}{c^2} + \frac{2r\ddot{r}}{c^2}\right)$ beschreiben Bewegungsabhängige Korrekturen
    \item Reduktion auf Coulomb-Kraft im statischen Fall ($\dot{r} = \ddot{r} = 0$)
\end{itemize}

\subsection*{Vergleich mit Maxwell-Theorie}
\begin{itemize}[leftmargin=*,noitemsep]
    \item Alternative Beschreibung elektromagnetischer Phänomene
    \item Fernwirkungsansatz (direkte Ladungswechselwirkung)
    \item Implizite Retardierung durch Geschwindigkeits-/Beschleunigungsterme
    \item Keine Vorhersage von EM-Wellen im Vakuum
\end{itemize}

\subsection{Ansatz zur Weber-Gravitation (WG)}
\begin{itemize}[leftmargin=*,noitemsep]
    \item Kein vordefiniertes Raummodell benötigt
    \item Natürliche Diskretisierung durch Punktteilchen
    \item Gravitative Erweiterung möglich:
    \begin{equation}\label{eq:weber_g}
    \bm{F}_{\text{Weber}}^{G} = G\frac{mM}{r^2}\left(1 - \frac{\dot{r}^2}{c^2} + \frac{2r\ddot{r}}{c^2}\right)\bm{\hat{r}}
    \end{equation}
    Die Gleichung \textbf{\ref{eq:weber_g}} entspricht der Gleichung \textbf{\ref{eq:weber_em}} als hypothetische Annahme über die Gravitationskraft.
\end{itemize}

\subsection*{Zusammenfassung}
\begin{itemize}[leftmargin=*,noitemsep]
    \item Umgeht Quantisierungsprobleme der ART
    \item Ermöglicht diskrete Raumzeitmodelle
    \item Potentieller Brückenansatz zur Quantengravitation
\end{itemize}

\section{Weber-Kraft und Gravitation}

\subsection*{Tisserands Ansatz}
Die Übertragung der elektrodynamischen Weber-Kraft \textbf{\ref{eq:weber_em}} auf die Gravitation \textbf{\ref{eq:weber_g}} scheiterte
an der Erklärung der Periheldrehung des Merkur.

\subsection*{Hinweis}
Die korrekte gravitative Formulierung wird separat vorgestellt und erfordert eine Modifikation der Original-Weberschen Formel.

\section{Grundgleichungen der Weber-Kraft}
\subsection*{Weber-Kraft Gleichung}
\begin{equation}\label{eq:weber_gravitationskraft}
\mathbf{F} = -\frac{GMm}{r^2}\left(1 - \frac{\dot{r}^2}{c^2} + \frac{r\ddot{r}}{2c^2}\right)\mathbf{\hat{r}}
\end{equation}

\subsection*{Bewegungsgleichung in Polarkoordinaten}
\begin{equation}\label{eq:weber_bewegungsgleichung}
\mathbf{a} = \left(\ddot{r} - r\dot{\varphi}^2\right)\mathbf{\hat{r}} + \left(r\ddot{\varphi} + 2\dot{r}\dot{\varphi}\right)\mathbf{\hat{\varphi}} = -\frac{GM}{r^2}\left(1 - \frac{\dot{r}^2}{c^2} + \frac{r\ddot{r}}{2c^2}\right)\mathbf{\hat{r}}
\end{equation}

\subsection*{Variablenbeschreibung}
\begin{itemize}[leftmargin=*,noitemsep]
    \item $\mathbf{F}$: Gravitationskraftvektor (Weber-Kraft) [N]
    \item $\mathbf{a}$: Beschleunigungsvektor [m/s²]
    \item $G$: Gravitationskonstante [m³/kg/s²]
    \item $M$, $m$: Massen der wechselwirkenden Körper [kg]
    \item $r$: Abstand zwischen den Massenschwerpunkten [m]
    \item $\dot{r} = \frac{dr}{dt}$: Radiale Relativgeschwindigkeit [m/s]
    \item $\ddot{r} = \frac{d^2r}{dt^2}$: Radiale Relativbeschleunigung [m/s²]
    \item $c$: Lichtgeschwindigkeit [m/s]
    \item $\varphi$: Azimutwinkel [rad]
    \item $\dot{\varphi} = \frac{d\varphi}{dt}$: Winkelgeschwindigkeit [rad/s]
    \item $\ddot{\varphi} = \frac{d^2\varphi}{dt^2}$: Winkelbeschleunigung [rad/s²]
    \item $\mathbf{\hat{r}}$: Radialer Einheitsvektor (zeigt von $M$ zu $m$)
    \item $\mathbf{\hat{\varphi}}$: Azimutaler Einheitsvektor (senkrecht zu $\mathbf{\hat{r}}$)
\end{itemize}

\subsection*{Physikalische Interpretation}
\begin{itemize}[leftmargin=*,noitemsep]
    \item Der Term $-\frac{GMm}{r^2}$ entspricht der klassischen Newton'schen Gravitation
    \item $\frac{\dot{r}^2}{c^2}$: Relativistische Korrektur für radiale Bewegung
    \item $\frac{r\ddot{r}}{2c^2}$: Korrektur für radiale Beschleunigung
    \item $r\dot{\varphi}^2$: Zentripetalbeschleunigung
    \item $2\dot{r}\dot{\varphi}$: Coriolis-Term
\end{itemize}

\section{Bewegungsgleichungen}
Die Weber-Kraft in Polarkoordinaten:
\begin{equation}
\mathbf{F} = -\frac{GMm}{r^2}\left(1 - \frac{\dot{r}^2}{c^2} + \frac{r\ddot{r}}{2c^2}\right)\mathbf{\hat{r}}
\end{equation}

Mit $u=1/r$ und Drehimpuls $h=r^2\dot{\phi}$:
\begin{equation}
\frac{d^2u}{d\phi^2} + u = \frac{GM}{h^2} + \frac{6GM}{c^2}u^2 + \frac{GM}{2c^2}\left(u\frac{d^2u}{d\phi^2} + \left(\frac{du}{d\phi}\right)^2\right)
\end{equation}

\section{Störungsrechnung}
Ansatz:
\begin{equation}
u(\phi) = \sum_{k=0}^2 \frac{u_k(\phi)}{c^{2k}} + \mathcal{O}(c^{-6})
\end{equation}

\subsection{Ordnungen}
\begin{itemize}
\item[0.] Kepler-Lösung:
\begin{equation}
u_0 = \frac{GM}{h^2}(1 + e\cos\phi)
\end{equation}

\item[1.] ART-Äquivalent:
\begin{equation}
u_1 = \frac{3G^2M^2e}{h^4}\phi\sin\phi,\quad \Delta\phi_1 = \frac{6\pi GM}{c^2a(1-e^2)}
\end{equation}

\item[2.] Korrekturterm:
\begin{equation}
u_2 = \frac{G^3M^3e}{h^6}\left(\frac{27}{4}\phi\sin\phi + \frac{3e}{8}\phi^2\cos\phi\right)
\end{equation}
\end{itemize}

\section{Resultate}
Bahngleichung:
\begin{equation}
r(\phi) = \frac{a(1-e^2)}{1 + e\cos\left(\kappa\phi + \frac{\alpha\phi^2}{c^4}\right)}
\end{equation}
mit:
\begin{align}
\kappa &= \sqrt{1 - \frac{6GM}{c^2a(1-e^2)} + \frac{27G^2M^2}{2c^4a^2(1-e^2)^2}}\\
\alpha &= \frac{3G^2M^2e}{8h^4}
\end{align}

Periheldrehung:
\begin{equation}
\Delta\phi = \frac{6\pi GM}{c^2a(1-e^2)}\left(1 \underbrace{- \frac{3GM}{4c^2a(1-e^2)}}_{\text{Korrektur}}\right)
\end{equation}

\section{Interpretation}
\begin{itemize}
\item ART-Wert ($43''$/Jh.) wird systematisch überschätzt
\item WG zeigt konsistente Reduktion durch $c^{-4}$-Terme
\item Physikalische Ursache: Nichtlineare Rückkopplung der Beschleunigungsterme
\end{itemize}

\newpage
\section{Winkelgeschwindigkeit in 1. Ordnung}
Die Winkelgeschwindigkeit \(\omega(\phi)\) in der Weber-Gravitation bis zur Ordnung \(\mathcal{O}(c^{-2})\) lautet:

\begin{equation}
\omega(\phi) = \frac{h}{a^2(1 - e^2)^2} \left[1 + e \cos\left(\kappa\phi\right)\right]^2
\end{equation}

wobei:
\begin{itemize}
    \item \(h = \sqrt{GMa(1 - e^2)}\) der spezifische Drehimpuls ist,
    \item \(\kappa = \sqrt{1 - \frac{6GM}{c^2a(1 - e^2)}}\),
    \item Terme der Ordnung \(\mathcal{O}(c^{-4})\) (z. B. \(\alpha\phi^2\)) werden vernachlässigt.
\end{itemize}

\subsection*{Bedeutung der Terme}
\begin{itemize}
    \item Die Wurzel \(\kappa\) beschreibt die Periheldrehung 1. Ordnung ohne Näherung.
    \item Für \(c \to \infty\) wird \(\kappa = 1\), und die Gleichung reduziert sich auf die Newton’sche Form:
    \[
    \omega_N(\phi) = \frac{h(1 + e \cos\phi)^2}{a^2(1 - e^2)^2}.
    \]
\end{itemize}

\section{Winkelgeschwindigkeit in 2. Ordnung}

\subsection{Korrekte Entwicklung von \(\kappa\)}
Die Konstante \(\kappa\) muss bis zur 2. Ordnung präzise sein:
\begin{equation}
\kappa = \sqrt{1 - \frac{6GM}{c^2a(1-e^2)} + \frac{27G^2M^2}{2c^4a^2(1-e^2)^2}}
\end{equation}

\subsection{Winkelgeschwindigkeit}
Mit dem exakten \(\kappa\) und \(\alpha = \frac{3G^2M^2e}{8h^4c^4}\):
\begin{equation}
\boxed
{
    \omega(\phi) = \frac{h[1 + e\cos(\kappa\phi + \alpha\phi^2)]^2}{a^2(1-e^2)^2}
}
\end{equation}

\newpage
\section{Bahngeschwindigkeit in Weber-Gravitation}
\subsection*{Definition}
Die Bahngeschwindigkeit $v(\phi)$ ergibt sich aus Winkelgeschwindigkeit $\omega(\phi)$ und Radialabstand $r(\phi)$:
\begin{equation}
v(\phi) = \omega(\phi) \cdot r(\phi) = \frac{h}{r(\phi)}
\end{equation}

\subsection{Explizite Formel}
Mit der Bahngleichung:
\begin{equation}
r(\phi) = \frac{a(1-e^2)}{1 + e\cos\left(\kappa\phi + \alpha\phi^2/c^4\right)}
\end{equation}
folgt:
\begin{equation}\boxed{
v(\phi) = \frac{h}{a(1-e^2)} \left(1 + e\cos\left[\left(1 - \frac{3GM}{c^2a(1-e^2)} + \frac{9G^2M^2}{8c^4a^2(1-e^2)^2}\right)\phi + \frac{3G^2M^2e}{8c^4h^4}\phi^2\right]\right)
}\end{equation}

\subsection*{Physikalische Terme}
\begin{itemize}[leftmargin=*,noitemsep]
    \item $h$: Drehimpuls pro Masse ($h = r^2\dot{\phi}$)
    \item $\kappa\phi$: Periheldrehung (1. Ordnung in $c^{-2}$)
    \item $\alpha\phi^2/c^4$: Nichtlineare Bahnstörung (2. Ordnung)
\end{itemize}

\subsection*{Grenzfälle}
\begin{align*}
    \text{Newton: } & c \to \infty \Rightarrow v_N = \frac{h(1+e\cos\phi)}{a(1-e^2)} \\
    \text{Perihel: } & \phi=0 \Rightarrow v(0) = \frac{h(1+e)}{a(1-e^2)} \\
    \text{Aphel: } & \phi=\pi \Rightarrow v(\pi) = \frac{h(1-e)}{a(1-e^2)}
\end{align*}

\chapter{Sonnensystem}
\newpage
\section{Exakte Berechnung der Umlaufperiode \( T \)}

\subsection*{Gegebene Gleichungen}
\begin{align}
r(\phi) &= \frac{a(1-e^2)}{1 + e\cos\left(\kappa\phi + \alpha\phi^2\right)} \label{eq:orbit} \\
\kappa &= \sqrt{1 - \frac{6GM}{c^2a(1-e^2)} + \frac{27G^2M^2}{2c^4a^2(1-e^2)^2}} \label{eq:kappa} \\
\alpha &= \frac{3G^2M^2e}{8c^4h^4}, \quad h = \sqrt{GMa(1-e^2)} \label{eq:alpha}
\end{align}

\subsection*{Schritt 1: Exakte Entwicklung von \(\kappa\)}
\begin{equation}
\kappa \approx 1 - \frac{3GM}{c^2a(1-e^2)} + \frac{27G^2M^2}{4c^4a^2(1-e^2)^2} - \frac{81G^3M^3}{8c^6a^3(1-e^2)^3} + \mathcal{O}(c^{-8}) \label{eq:kappa_expanded}
\end{equation}
*Begründung:* Taylor-Entwicklung der Wurzel in Gl. \eqref{eq:kappa} um \(c^{-2} = 0\).

\subsection*{Schritt 2: Vollständige Integration}
Die Umlaufperiode \( T \) ist:
\begin{equation}
T = \frac{1}{h} \int_0^{2\pi} r^2(\phi) \, d\phi = \frac{a^2(1-e^2)^2}{h} \int_0^{2\pi} \frac{d\phi}{\left[1 + e\cos\left(\kappa\phi + \alpha\phi^2\right)\right]^2} \label{eq:T_integral}
\end{equation}

\subsection*{Schritt 3: Behandlung des Integrals}
Mit Substitution \(\psi = \kappa\phi + \alpha\phi^2\) und Entwicklung bis \(\mathcal{O}(c^{-4})\):
\begin{align}
T &= \frac{a^2(1-e^2)^2}{h} \left[ \int_0^{2\pi} \frac{d\phi}{(1 + e\cos\psi)^2} + \mathcal{O}(c^{-6}) \right] \\
  &= \frac{2\pi a^{3/2}}{\sqrt{GM}} \left[1 + \frac{3GM}{2c^2a(1-e^2)} + \frac{45G^2M^2}{8c^4a^2(1-e^2)^2}\left(1 - \frac{e^2}{3}\right)\right] \label{eq:T_final}
\end{align}
*Kritische Schritte:*
\begin{itemize}
\item Keine Vernachlässigung von \(\alpha\phi^2\) – trägt zu \(\mathcal{O}(c^{-4})\)-Termen bei.
\item Exakte Koeffizienten aus Gl. \eqref{eq:kappa_expanded} werden verwendet.
\end{itemize}

\subsection*{Numerisches Beispiel (Merkur)}
\begin{align*}
T_{\text{Newton}} &= \SI{7.6005e6}{\second} \\
\Delta T^{(1)} &= \frac{3GM}{2c^2a(1-e^2)} T_{\text{Newton}} \approx \SI{0.002}{\second} \\
\Delta T^{(2)} &= \frac{45G^2M^2}{8c^4a^2(1-e^2)^2}\left(1 - \frac{e^2}{3}\right) T_{\text{Newton}} \approx \SI{8.5e-12}{\second}
\end{align*}

\subsection*{Zusammenfassung}
\begin{itemize}
\item \textbf{1. Ordnung} (\(\propto c^{-2}\)): Weber-Korrektur ist halb so groß wie in der ART.
\item \textbf{2. Ordnung} (\(\propto c^{-4}\)): Weber-spezifischer Term mit \(e^2\)-Abhängigkeit.
\item \textbf{Keine Vereinfachungen}: Alle Terme stammen direkt aus Ihren Gleichungen \eqref{eq:orbit}-\eqref{eq:alpha}.
\end{itemize}

\part{Kosmologie}
\section{Kernaussage zur dunklen Materie}
Die Weber-Gravitation erklärt galaktische Rotationskurven \textbf{ohne dunkle Materie}\\durch ihre nicht-newtonschen Terme:
\begin{equation}
\mathbf{F}_{\text{Weber}}^G = -\frac{GMm}{r^2}\left(1 \underbrace{-\frac{\dot{r}^2}{c^2} + \frac{r\ddot{r}}{2c^2}}_{\text{relativistische Korrekturen}}\right)\mathbf{\hat{r}}
\end{equation}

\section*{Mathematischer Beweis}

\subsection*{Rotationskurven von Galaxien}
Für eine Kreisbahn (\(\dot{r}=0\), \(\ddot{r} = -r\dot{\varphi}^2\)) reduziert sich die Weber-Kraft zu:
\begin{equation}
F_{\text{Weber}} = -\frac{GMm}{r^2}\left(1 - \frac{v^2}{2c^2}\right), \quad v = r\dot{\varphi}
\end{equation}
Die Zentripetalkraft \(F = mv^2/r\) führt zur modifizierten Geschwindigkeit:
\begin{equation}
v(r) = \sqrt{\frac{GM}{r}} \left(1 + \frac{GM}{4c^2r}\right)
\end{equation}

\subsection*{Vergleich mit Beobachtungen}
\begin{itemize}
\item \textbf{Newton}: \(v \propto r^{-1/2}\) (Abfall nicht beobachtet)
\item \textbf{Weber}: Zusatzterm \(\propto r^{-3/2}\) kompensiert den Abfall bei großen \(r\)
\item \textbf{ART}: Erfordert dunkle Materie für flache Rotationskurven
\end{itemize}

\section*{Numerisches Beispiel (Milchstraße)}
\begin{align*}
\text{Bereich} &\quad r = \SI{10}{kpc} \\
\text{Weber-Korrektur} &\quad \frac{GM}{4c^2r} \approx 0.12 \quad (\text{12\% Erhöhung}) \\
\text{Beobachtung} &\quad v \approx \SI{220}{km/s} \ (\text{konstant über } r)
\end{align*}

\section*{Konsequenzen}
\begin{itemize}
\item \textbf{Keine dunkle Materie}: Die Weber-Korrektur wirkt wie eine effektive Massenerhöhung \(\Delta M \approx \frac{GM(r)}{4c^2r}M\).
\item \textbf{Quantitativ}: Für \(r \to \infty\) wird \(v(r)\) konstant – genau wie beobachtet.
\item \textbf{Unterschied zu MOND}: Die Korrektur folgt natürlicherweise aus der Weber-Formel, ohne ad-hoc-Anpassungen.
\end{itemize}


\end{document}
