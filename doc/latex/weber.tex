\documentclass{book}
\usepackage[a4paper,left=2.5cm,right=2cm,top=2cm,bottom=2.5cm]{geometry}
\usepackage[utf8]{inputenc}
\usepackage[ngerman]{babel}
\usepackage{amsfonts}
\usepackage{amsmath}
\usepackage{amssymb}
\usepackage{array}
\usepackage{ragged2e}
\usepackage{tabularx}
\usepackage{enumitem}
\usepackage{booktabs}
\usepackage{bm}
\usepackage{csquotes}
\usepackage{siunitx}

\renewcommand{\arraystretch}{1.1}
\numberwithin{equation}{section}

\begin{document}

\title{Mein Dokument}
\author{Dein Name}
\date{\today}
\maketitle

\tableofcontents

% Einbindung der einzelnen TeX-Dateien
\part{Grundlagen}
\chapter{Weber-Kraft}
\section{Klassische Weber-Kraft (Elektrodynamik)}
\begin{equation}\label{eq:weber_em}
\bm{F}_{\text{Weber}}^{\text{EM}} = \frac{Qq}{4\pi\epsilon_0 r^2}\left(1 - \frac{\dot{r}^2}{c^2} + \frac{2r\ddot{r}}{c^2}\right)\bm{\hat{r}}
\end{equation}

\subsection*{Symbolbeschreibung}
\begin{itemize}[leftmargin=*,noitemsep]
    \item $\bm{F}_{\text{Weber}}^{\text{EM}}$: Weber-Kraft zwischen Ladungen
    \item $Q, q$: Elektrische Ladungen
    \item $\epsilon_0$: Elektrische Feldkonstante
    \item $r$: Ladungsabstand
    \item $\dot{r} = \frac{dr}{dt}$: Relative Radialgeschwindigkeit
    \item $\ddot{r} = \frac{d^2r}{dt^2}$: Relative Radialbeschleunigung
    \item $c$: Lichtgeschwindigkeit
    \item $\bm{\hat{r}}$: Radialer Einheitsvektor
\end{itemize}

\subsection*{Beziehung zur Coulomb-Kraft}
\begin{itemize}[leftmargin=*,noitemsep]
    \item Erster Term entspricht Coulomb-Kraft: $\frac{Qq}{4\pi\epsilon_0 r^2}$
    \item Zusatzterme $\left(-\frac{\dot{r}^2}{c^2} + \frac{2r\ddot{r}}{c^2}\right)$ beschreiben Bewegungsabhängige Korrekturen
    \item Reduktion auf Coulomb-Kraft im statischen Fall ($\dot{r} = \ddot{r} = 0$)
\end{itemize}

\subsection*{Vergleich mit Maxwell-Theorie}
\begin{itemize}[leftmargin=*,noitemsep]
    \item Alternative Beschreibung elektromagnetischer Phänomene
    \item Fernwirkungsansatz (direkte Ladungswechselwirkung)
    \item Implizite Retardierung durch Geschwindigkeits-/Beschleunigungsterme
    \item Keine Vorhersage von EM-Wellen im Vakuum
\end{itemize}

\subsection{Ansatz zur Weber-Gravitation (WG)}
\begin{itemize}[leftmargin=*,noitemsep]
    \item Kein vordefiniertes Raummodell benötigt
    \item Natürliche Diskretisierung durch Punktteilchen
    \item Gravitative Erweiterung möglich:
    \begin{equation}\label{eq:weber_g}
    \bm{F}_{\text{Weber}}^{G} = G\frac{mM}{r^2}\left(1 - \frac{\dot{r}^2}{c^2} + \frac{2r\ddot{r}}{c^2}\right)\bm{\hat{r}}
    \end{equation}
    Die Gleichung \textbf{\ref{eq:weber_g}} entspricht der Gleichung \textbf{\ref{eq:weber_em}} als hypothetische Annahme über die Gravitationskraft.
\end{itemize}

\subsection*{Zusammenfassung}
\begin{itemize}[leftmargin=*,noitemsep]
    \item Umgeht Quantisierungsprobleme der ART
    \item Ermöglicht diskrete Raumzeitmodelle
    \item Potentieller Brückenansatz zur Quantengravitation
\end{itemize}

\section{Weber-Kraft und Gravitation}

\subsection*{Tisserands Ansatz}
Die Übertragung der elektrodynamischen Weber-Kraft \textbf{\ref{eq:weber_em}} auf die Gravitation \textbf{\ref{eq:weber_g}} scheiterte
an der Erklärung der Periheldrehung des Merkur.

\subsection*{Hinweis}
Die korrekte gravitative Formulierung wird separat vorgestellt und erfordert eine Modifikation der Original-Weberschen Formel.

\section{Grundgleichungen der Weber-Kraft}
\subsection*{Weber-Kraft Gleichung}
\begin{equation}\label{eq:weber_gravitationskraft}
\mathbf{F} = -\frac{GMm}{r^2}\left(1 - \frac{\dot{r}^2}{c^2} + \frac{r\ddot{r}}{2c^2}\right)\mathbf{\hat{r}}
\end{equation}

\subsection*{Bewegungsgleichung in Polarkoordinaten}
\begin{equation}\label{eq:weber_bewegungsgleichung}
\mathbf{a} = \left(\ddot{r} - r\dot{\varphi}^2\right)\mathbf{\hat{r}} + \left(r\ddot{\varphi} + 2\dot{r}\dot{\varphi}\right)\mathbf{\hat{\varphi}} = -\frac{GM}{r^2}\left(1 - \frac{\dot{r}^2}{c^2} + \frac{r\ddot{r}}{2c^2}\right)\mathbf{\hat{r}}
\end{equation}

\subsection*{Variablenbeschreibung}
\begin{itemize}[leftmargin=*,noitemsep]
    \item $\mathbf{F}$: Gravitationskraftvektor (Weber-Kraft) [N]
    \item $\mathbf{a}$: Beschleunigungsvektor [m/s²]
    \item $G$: Gravitationskonstante [m³/kg/s²]
    \item $M$, $m$: Massen der wechselwirkenden Körper [kg]
    \item $r$: Abstand zwischen den Massenschwerpunkten [m]
    \item $\dot{r} = \frac{dr}{dt}$: Radiale Relativgeschwindigkeit [m/s]
    \item $\ddot{r} = \frac{d^2r}{dt^2}$: Radiale Relativbeschleunigung [m/s²]
    \item $c$: Lichtgeschwindigkeit [m/s]
    \item $\varphi$: Azimutwinkel [rad]
    \item $\dot{\varphi} = \frac{d\varphi}{dt}$: Winkelgeschwindigkeit [rad/s]
    \item $\ddot{\varphi} = \frac{d^2\varphi}{dt^2}$: Winkelbeschleunigung [rad/s²]
    \item $\mathbf{\hat{r}}$: Radialer Einheitsvektor (zeigt von $M$ zu $m$)
    \item $\mathbf{\hat{\varphi}}$: Azimutaler Einheitsvektor (senkrecht zu $\mathbf{\hat{r}}$)
\end{itemize}

\subsection*{Physikalische Interpretation}
\begin{itemize}[leftmargin=*,noitemsep]
    \item Der Term $-\frac{GMm}{r^2}$ entspricht der klassischen Newton'schen Gravitation
    \item $\frac{\dot{r}^2}{c^2}$: Relativistische Korrektur für radiale Bewegung
    \item $\frac{r\ddot{r}}{2c^2}$: Korrektur für radiale Beschleunigung
    \item $r\dot{\varphi}^2$: Zentripetalbeschleunigung
    \item $2\dot{r}\dot{\varphi}$: Coriolis-Term
\end{itemize}

\section{Bewegungsgleichungen}
Die Weber-Kraft in Polarkoordinaten:
\begin{equation}
\mathbf{F} = -\frac{GMm}{r^2}\left(1 - \frac{\dot{r}^2}{c^2} + \frac{r\ddot{r}}{2c^2}\right)\mathbf{\hat{r}}
\end{equation}

Mit $u=1/r$ und Drehimpuls $h=r^2\dot{\phi}$:
\begin{equation}
\frac{d^2u}{d\phi^2} + u = \frac{GM}{h^2} + \frac{6GM}{c^2}u^2 + \frac{GM}{2c^2}\left(u\frac{d^2u}{d\phi^2} + \left(\frac{du}{d\phi}\right)^2\right)
\end{equation}

\section{Störungsrechnung}
Ansatz:
\begin{equation}
u(\phi) = \sum_{k=0}^2 \frac{u_k(\phi)}{c^{2k}} + \mathcal{O}(c^{-6})
\end{equation}

\subsection{Ordnungen}
\begin{itemize}
\item[0.] Kepler-Lösung:
\begin{equation}
u_0 = \frac{GM}{h^2}(1 + e\cos\phi)
\end{equation}

\item[1.] ART-Äquivalent:
\begin{equation}
u_1 = \frac{3G^2M^2e}{h^4}\phi\sin\phi,\quad \Delta\phi_1 = \frac{6\pi GM}{c^2a(1-e^2)}
\end{equation}

\item[2.] Korrekturterm:
\begin{equation}
u_2 = \frac{G^3M^3e}{h^6}\left(\frac{27}{4}\phi\sin\phi + \frac{3e}{8}\phi^2\cos\phi\right)
\end{equation}
\end{itemize}

\section{Resultate}
Bahngleichung:
\begin{equation}
r(\phi) = \frac{a(1-e^2)}{1 + e\cos\left(\kappa\phi + \frac{\alpha\phi^2}{c^4}\right)}
\end{equation}
mit:
\begin{align}
\kappa &= \sqrt{1 - \frac{6GM}{c^2a(1-e^2)} + \frac{27G^2M^2}{2c^4a^2(1-e^2)^2}}\\
\alpha &= \frac{3G^2M^2e}{8h^4}
\end{align}

Periheldrehung:
\begin{equation}
\Delta\phi = \frac{6\pi GM}{c^2a(1-e^2)}\left(1 \underbrace{- \frac{3GM}{4c^2a(1-e^2)}}_{\text{Korrektur}}\right)
\end{equation}

\section{Interpretation}
\begin{itemize}
\item ART-Wert ($43''$/Jh.) wird systematisch überschätzt
\item WG zeigt konsistente Reduktion durch $c^{-4}$-Terme
\item Physikalische Ursache: Nichtlineare Rückkopplung der Beschleunigungsterme
\end{itemize}


\end{document}
