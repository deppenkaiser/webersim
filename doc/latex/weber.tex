\documentclass{book}
\usepackage[a4paper,left=2.5cm,right=2cm,top=2cm,bottom=2.5cm]{geometry}
\usepackage[utf8]{inputenc}
\usepackage[ngerman]{babel}
\usepackage{amsfonts}
\usepackage{amsmath}
\usepackage{amssymb}
\usepackage{array}
\usepackage{ragged2e}
\usepackage{tabularx}
\usepackage{enumitem}
\usepackage{booktabs}
\usepackage{bm}
\usepackage{csquotes}
\usepackage{siunitx}
\usepackage{parskip}
\usepackage{listings}
\usepackage{xcolor}
\usepackage[utf8]{inputenc}
\usepackage[labelfont=bf]{caption}
\usepackage{tcolorbox}
\usepackage{mathrsfs}
%\usepackage{showlabels}
%\usepackage{refcheck}

\newtheorem{theorem}{Theorem} % Definiert das 'theorem'-Environment
\newtheorem{lemma}{Lemma}     % Falls Sie auch Lemmas verwenden möchten
%\showlabels{cite,label}

\renewcommand{\arraystretch}{1.1}
\numberwithin{equation}{section}
\definecolor{gray}{rgb}{0.5,0.5,0.5}
\cleardoublepage

\begin{document}

\title{Weber-Gravitation}
\author{Michael Czybor}
\date{\today}
\maketitle

\section*{Zusammenfassung}
Die \enquote{Weber-Gravitation} präsentiert eine alternative Gravitationstheorie, die auf der Weber-Kraft basiert und in vielen Aspekten konkurrenzfähige
oder sogar überlegene Ergebnisse im Vergleich zur Allgemeinen Relativitätstheorie \cite{einstein1915} (ART) liefert. Die zentrale These der Arbeit ist, dass die
Weber-Gravitation (WG) nicht nur die bekannten Phänomene der ART erklärt, sondern auch deren Schwächen – wie die Notwendigkeit dunkler Materie
oder die Existenz singularitätsbehafteter schwarzer Löcher – vermeidet.

Ein herausragendes Ergebnis der WG ist die präzise Berechnung der Periheldrehung des Merkurs, die mit einem Wert von 42,98 Bogensekunden pro Jahrhundert
nahezu identisch zur ART-Vorhersage ist. Entscheidend ist jedoch, dass die WG dies ohne ein gekrümmtes Raumzeit-Modell erreicht. Stattdessen modifiziert
sie das Newtonsche Gravitationsgesetz durch relativistische Korrekturen, die von der radialen Geschwindigkeit (\(\dot{r}\)) und Beschleunigung (\(\ddot{r}\)) abhängen.
Die daraus abgeleitete Bahngleichung zeigt, dass die WG die beobachtete Periheldrehung natürlicher erklärt als die ART, ohne auf ein komplexes geometrisches
Raummodell zurückgreifen zu müssen.

Ein weiterer wesentlicher Vorteil der WG ist ihre Fähigkeit, galaktische Rotationskurven ohne dunkle Materie zu beschreiben. Während die ART zusätzliche,
unsichtbare Masse postulieren muss, um die flachen Rotationsprofile von Galaxien zu erklären, liefert die WG eine korrigierte Geschwindigkeitsformel,
die den beobachteten Verlauf reproduziert:  

\[
v(r) = \sqrt{\frac{GM}{r}} \left(1 + \frac{GM}{4c^2r}\right).
\]  

Dieser Ansatz vermeidet nicht nur die hypothetische dunkle Materie, sondern bietet auch eine direkte physikalische Interpretation der Abweichungen vom Newtonschen Gesetz.

Die Arbeit diskutiert zudem die Lichtablenkung im Gravitationsfeld, wobei die WG eine frequenzabhängige Korrektur vorhersagt, die in der ART nicht existiert.
Diese könnte zukünftig experimentell überprüft werden, etwa durch hochpräzise Messungen der Ablenkung von Radiowellen gegenüber optischem Licht. Auch der
Shapiro-Effekt \cite{shapiro1964} (Laufzeitverzögerung von Signalen) wird in der WG leicht modifiziert, wobei die Abweichungen zur ART jedoch erst bei extrem hohen Genauigkeiten messbar wären.  

Ein radikaler Unterschied zur ART zeigt sich in der kosmologischen Interpretation der Rotverschiebung. Während die ART diese als Folge der Expansion des Universums deutet,
erklärt die WG sie durch kumulative gravitative Wechselwirkungen:

\[
z \approx \frac{3}{2} \frac{v_r^2}{c^2}.
\]  

Dies impliziert ein statisches Universum ohne Urknall, was eine grundlegend andere Kosmologie zur Folge hätte. Die Arbeit argumentiert, dass dieser Ansatz mehrere Probleme
der Standardkosmologie (wie die dunkle Energie) vermeiden könnte.  

Kritisch bleibt, dass die WG keine Gravitationswellen vorhersagt, da ihr ein dynamisches Raumzeit-Modell fehlt. Hierin besteht jedoch kein grundsätzliches Hindernis,
sondern es ist ein Anreiz, die Theorie um ein Quantengravitations-Konzept zu erweitern. Dennoch kann gezeigt werden, dass die spezielle Relativitätstheorie (SRT)
bereits aus der WG- und DBT-Synthese emergiert. Die Ergänzung durch ein Dodekaeder-Raummodell wird zeigen, dass auch die allgemeine Relativitätstheorie (ART) vollständig aus diesen
Theorien emergiert, wobei die WG alleine, bereits einen Hauptteil der ART-Aussagen liefert.

Fazit: Die Weber-Gravitation stellt eine vielversprechende Alternative zur ART dar, die mehrere ihrer ungelösten Probleme umgeht. Obwohl sie in einigen Bereichen
(wie der Merkurperiheldrehung) äquivalente Ergebnisse liefert, bietet sie in anderen (Galaxienrotation, Kosmologie) potenziell einfachere und elegantere Erklärungen.
Experimentelle Tests der frequenzabhängigen Effekte wären der nächste Schritt, um die Theorie weiter zu validieren. Die Arbeit plädiert dafür, die WG als
ernstzunehmenden Ansatz in der modernen Gravitationsphysik zu betrachten.

\tableofcontents

% Einbindung der einzelnen TeX-Dateien
\part{Grundlagen}
\chapter{Weber-Kraft}
\label{chapter:weber_kraft}
\section{Klassische Weber-Kraft (Elektrodynamik)}
\begin{equation}
    \boxed
    {
        \bm{F}_{\text{Weber}}^{\text{EM}} = \frac{Qq}{4\pi\epsilon_0 r^2}\left(1 - \frac{\dot{r}^2}{c^2} + \frac{2r\ddot{r}}{c^2}\right)\bm{\hat{r}}
    }
\end{equation}

\subsection*{Symbolbeschreibung}
\begin{itemize}[leftmargin=*,noitemsep]
    \item $\bm{F}_{\text{Weber}}^{\text{EM}}$: Weber-Kraft zwischen Ladungen
    \item $Q, q$: Elektrische Ladungen
    \item $\epsilon_0$: Elektrische Feldkonstante
    \item $r$: Ladungsabstand
    \item $\dot{r} = \frac{dr}{dt}$: Relative Radialgeschwindigkeit
    \item $\ddot{r} = \frac{d^2r}{dt^2}$: Relative Radialbeschleunigung
    \item $c$: Lichtgeschwindigkeit
    \item $\bm{\hat{r}}$: Radialer Einheitsvektor
\end{itemize}

\subsection*{Beziehung zur Coulomb-Kraft}
\begin{itemize}[leftmargin=*,noitemsep]
    \item Erster Term entspricht Coulomb-Kraft: $\frac{Qq}{4\pi\epsilon_0 r^2}$
    \item Zusatzterme $\left(-\frac{\dot{r}^2}{c^2} + \frac{2r\ddot{r}}{c^2}\right)$ beschreiben Bewegungsabhängige Korrekturen
    \item Reduktion auf Coulomb-Kraft im statischen Fall ($\dot{r} = \ddot{r} = 0$)
\end{itemize}

\subsection*{Vergleich mit Maxwell-Theorie}
\begin{itemize}[leftmargin=*,noitemsep]
    \item Alternative Beschreibung elektromagnetischer Phänomene \cite{weber1846}
    \item Fernwirkungsansatz (direkte Ladungswechselwirkung)
    \item Implizite Retardierung durch Geschwindigkeits-/Beschleunigungsterme
    \item Keine Vorhersage von EM-Wellen im Vakuum
\end{itemize}

\subsection{Ansatz zur Weber-Gravitation (WG)}
\begin{itemize}[leftmargin=*,noitemsep]
    \item Kein vordefiniertes Raummodell benötigt
    \item Natürliche Diskretisierung durch Punktteilchen
    \item Gravitative Erweiterung möglich:
\end{itemize}

\begin{equation}
\bm{F}_{\text{Weber}}^{G} = G\frac{mM}{r^2}\left(1 - \frac{\dot{r}^2}{c^2} + \frac{2r\ddot{r}}{c^2}\right)\bm{\hat{r}}
\end{equation}

\subsection*{Zusammenfassung}
\begin{itemize}[leftmargin=*,noitemsep]
    \item Umgeht Quantisierungsprobleme der ART
    \item Ermöglicht diskrete Raumzeitmodelle
    \item Potentieller Brückenansatz zur Quantengravitation
\end{itemize}

\section{Weber-Kraft und Gravitation}
\subsection*{Tisserands Ansatz}
Die Übertragung der elektrodynamischen Weber-Kraft \cite{tisserand1894} auf die Gravitation scheiterte an der Erklärung der Periheldrehung des Merkurs.

\subsection*{Hinweis}
Die korrekte gravitative Formulierung wird separat vorgestellt und erfordert eine Modifikation der Original-Weberschen Formel.

\section{Weber-Gravitation als Alternative zur ART}
Die allgemeine Relativitätstheorie (ART) gilt als der Goldstandard der modernen Astrophysik, allerdings werden bestimmte Aspekte dieser Theorie
nicht objektiv betrachtet. Die ART überzeugt durch die Fähigkeit die Merkur-Periheldrehung vorhersagen zu können, aber auch durch die Vorhersage
der Gravitationswellen. Das sind große Leistungen dieser Gravitationstheorie.

Auf der anderen Seite liefert sie unphysikalische Ergebnisse für schwarze Löcher und für galaktische Skalen. Schwarze Löcher werden als Singularitäten
dargestellt, wobei davon ausgegangen werden muss, dass die gravitativen Verhältnisse in der Nähe dieser Singularitäten ebenfalls ungenau sein müssen. Die
Rotationskurven von Galaxien werden nicht korrekt Vorhergesagt, weswegen die ART \enquote{dunkle Materie} benötigt.

\subsection{Grundgleichungen der Weber-Gravitation}
\subsection*{Weber-Gravitations Gleichung}
\begin{equation}
\label{eq:weber_g}
\boxed
{
    \mathbf{F} = -\frac{GMm}{r^2}\left(1 - \frac{\dot{r}^2}{c^2} + \frac{r\ddot{r}}{2c^2}\right)\mathbf{\hat{r}}
}
\end{equation}

\subsection*{Spezifischer Drehimpuls}
Der Drehimpuls pro Masseneinheit $h$ ist definiert als:
\begin{equation}
\label{eq:spezifischer_drehimpuls_h}
\boxed
{
    h = r^2\dot{\phi} = \sqrt{GMa(1-e^2)}
}
\end{equation}
wobei $a$ die große Halbachse und $e$ die Exzentrizität der Bahn ist.

\subsection{Bewegungsgleichung in Polarkoordinaten}
\[
\mathbf{a} = \left(\ddot{r} - r\dot{\phi}^2\right)\mathbf{\hat{r}} + \left(r\ddot{\phi} + 2\dot{r}\dot{\phi}\right)\mathbf{\hat{\phi}} = -\frac{GM}{r^2}\left(1 - \frac{\dot{r}^2}{c^2} + \frac{r\ddot{r}}{2c^2}\right)\mathbf{\hat{r}}
\]

\subsection*{Variablenbeschreibung}
\begin{itemize}[leftmargin=*,noitemsep]
    \item $\mathbf{F}$: Gravitationskraftvektor (Weber-Kraft) [N]
    \item $\mathbf{a}$: Beschleunigungsvektor [m/s²]
    \item $G$: Gravitationskonstante [m³/kg/s²]
    \item $M$, $m$: Massen der wechselwirkenden Körper [kg]
    \item $r$: Abstand zwischen den Massenschwerpunkten [m]
    \item $\dot{r} = \frac{dr}{dt}$: Radiale Relativgeschwindigkeit [m/s]
    \item $\ddot{r} = \frac{d^2r}{dt^2}$: Radiale Relativbeschleunigung [m/s²]
    \item $c$: Lichtgeschwindigkeit [m/s]
    \item $\phi$: Azimutwinkel [rad]
    \item $\dot{\phi} = \frac{d\phi}{dt}$: Winkelgeschwindigkeit [rad/s]
    \item $\ddot{\phi} = \frac{d^2\phi}{dt^2}$: Winkelbeschleunigung [rad/s²]
    \item $h$: Spezifischer Drehimpuls [m²/s]
    \item $\mathbf{\hat{r}}$: Radialer Einheitsvektor (zeigt von $M$ zu $m$)
    \item $\mathbf{\hat{\phi}}$: Azimutaler Einheitsvektor (senkrecht zu $\mathbf{\hat{r}}$)
\end{itemize}

\subsection*{Physikalische Interpretation}
\begin{itemize}[leftmargin=*,noitemsep]
    \item Der Term $-\frac{GMm}{r^2}$ entspricht der klassischen Newton'schen Gravitation
    \item $\frac{\dot{r}^2}{c^2}$: Relativistische Korrektur für radiale Bewegung
    \item $\frac{r\ddot{r}}{2c^2}$: Korrektur für radiale Beschleunigung
    \item $r\dot{\phi}^2$: Zentripetalbeschleunigung
    \item $2\dot{r}\dot{\phi}$: Coriolis-Term
    \item $h$: Erhaltungsgröße für Planetenbahnen
\end{itemize}

\section{Bahngleichungen}
\subsection{Bahngleichung 1. Ordnung}
Die Bahngleichung \(r(\phi)\) in der Weber-Gravitation bis zur Ordnung \(\mathcal{O}(c^{-2})\) lautet:

\begin{equation}
\label{eq:bahngleichung_1_ordnung}
\boxed
{
    r(\phi) = \frac{a(1 - e^2)}{1 + e \cos\left(\kappa\phi\right)}
}
\end{equation}

\noindent mit der Definition:
\begin{equation}
\label{eq:kappa_1_ordnung}
\boxed
{
    \kappa = \sqrt{1 - \frac{6GM}{c^2a(1 - e^2)}}    
}
\end{equation}

\subsection*{Mathematische Herleitung}
Die Gleichung folgt aus der Lösung der Bewegungsgleichung:
\[
\frac{d^2u}{d\phi^2} + u = \frac{GM}{h^2} + \frac{6GM}{c^2} u^2 \quad \left(u = \frac{1}{r}\right),
\]

wobei der Term \(\frac{6GM}{c^2} u^2\) die Weber-spezifische Korrektur 1. Ordnung darstellt. Der Ansatz \(u(\phi) = \frac{1 + e \cos(\kappa\phi)}{a(1 - e^2)}\) führt auf die angegebene Lösung.

Mit $u=1/r$ und Drehimpuls $h$ (\ref{eq:spezifischer_drehimpuls_h}):
\[
\frac{d^2u}{d\phi^2} + u = \frac{GM}{h^2} + \frac{6GM}{c^2}u^2 + \frac{GM}{2c^2}\left(u\frac{d^2u}{d\phi^2} + \left(\frac{du}{d\phi}\right)^2\right)
\]

\subsection{Bahngleichung 2. Ordnung}
Bahngleichung:
\begin{equation}
\label{eq:bahngleichung_2_ordnung}
    \boxed
    {
        r(\phi) = \frac{a(1-e^2)}{1 + e\cos\left(\kappa\phi + \alpha\phi^2\right)}
    }
\end{equation}

mit:
$h$ aus Gleichung (\ref{eq:spezifischer_drehimpuls_h})
\begin{equation}
\label{eq:kappa_2_ordnung}
\boxed
{
    \kappa = \sqrt{1 - \frac{6GM}{c^2a(1-e^2)} + \frac{27G^2M^2}{2c^4a^2(1-e^2)^2}}
}
\end{equation}
\begin{equation}
\label{eq:alpha}
\boxed
{
    \alpha = \frac{3G^2M^2e}{8h^4c^4}    
}
\end{equation}

\newpage
\section{Periheldrehung}
\subsection{Periheldrehung 1. Ordnung}
Die Periheldrehung $\Delta\phi$ in der Weber-Gravitation ergibt sich aus der modifizierten Bahngleichung und lässt sich wie folgt herleiten:

\subsection*{Perihelbedingung}
Das Perihel (sonnennächster Punkt) tritt auf, wenn der Nenner maximal wird, d.h. wenn:\\
\[\cos(\kappa\phi) = 1\]
Die Lösungen dieser Bedingung sind: $\kappa\phi = 2\pi n \quad \text{(für $n \in \mathbb{Z}$)}$.\\

Somit ergeben sich die Winkel für aufeinanderfolgende Periheldurchgänge zu:
\[
    \phi_n = \frac{2\pi n}{\kappa}.
\]

\subsection*{Periheldrehung pro Umlauf}
Die Periheldrehung $\Delta\phi$ ist die Differenz zwischen dem Winkel für einen vollständigen Umlauf ($n = 1$) und dem Newton'schen Fall ($\kappa = 1$):
\[
    \Delta\phi = \phi_1 - 2\pi = \frac{2\pi}{\kappa} - 2\pi.
\]
Daraus folgt die gesuchte Gleichung:
\begin{equation}
\boxed
{
    \Delta\phi = 2\pi\left(\frac{1}{\kappa} - 1\right)
}.
\end{equation}

\subsection*{Interpretation}
\begin{itemize}
\item Im Newton'schen Grenzfall ($\kappa = 1$) verschwindet die Periheldrehung ($\Delta\phi = 0$).
\item Für $\kappa < 1$ (Weber-Gravitation) ergibt sich eine positive Periheldrehung, die mit Beobachtungen (z.B. Merkurperihel) übereinstimmt.
\end{itemize}

\subsection{Periheldrehung 2. Ordnung}
\subsection*{Entwicklung von $\kappa$}
Eine Taylor-Entwicklung von $\kappa$ bis zur 2. Ordnung liefert:
\[
    \kappa \approx 1 - \frac{3GM}{c^2 a(1 - e^2)} + \frac{27G^2 M^2}{4c^4 a^2 (1 - e^2)^2} + \mathcal{O}(c^{-6}).
\]

\subsection*{Perihelbedingung}
Das Perihel tritt auf bei:
\[
    \cos\left(\kappa\phi + \alpha\phi^2\right) = 1 \quad \Rightarrow \quad \kappa\phi + \alpha\phi^2 = 2\pi n.
\]

\subsection*{Lösung für $\Delta\phi$}
Für $n=1$ (ein Umlauf) ergibt sich die quadratische Gleichung:
\[
\alpha\phi^2 + \kappa\phi - 2\pi = 0.
\]
Die Lösung lautet:
\begin{equation}
\phi = \frac{-\kappa + \sqrt{\kappa^2 + 8\pi\alpha}}{2\alpha}.
\end{equation}

\subsection*{Näherung für kleine Korrekturen}
Da $\alpha \sim c^{-4}$ klein ist, entwickeln wir die Wurzel:
\[
    \phi \approx \frac{2\pi}{\kappa} - \frac{4\pi^2\alpha}{\kappa^3} + \mathcal{O}(\alpha^2)
\]
Die Periheldrehung pro Umlauf wird damit:
\begin{equation}
    \boxed
    {
        \Delta\phi = \phi - 2\pi \approx 2\pi\left(\frac{1}{\kappa} - 1\right) - \frac{4\pi^2\alpha}{\kappa^3}
    }
\end{equation}

\section{Winkelgeschwindigkeit}
\subsection{Winkelgeschwindigkeit 1. Ordnung}
Die Winkelgeschwindigkeit \(\omega(\phi)\) in der Weber-Gravitation bis zur Ordnung \(\mathcal{O}(c^{-2})\) lautet:

\begin{equation}
\label{eq:winkelgeschwindigkeit_1_ordnung}
\boxed
{
    \omega(\phi) = \frac{h}{a^2(1 - e^2)^2} \left[1 + e \cos\left(\kappa\phi\right)\right]^2   
}
\end{equation}

wobei:
$h$ aus Gleichung (\ref{eq:spezifischer_drehimpuls_h}), $\kappa$ aus Gleichung (\ref{eq:kappa_1_ordnung}) stammt.

\subsection*{Bedeutung der Terme}
\begin{itemize}
    \item \(\kappa\) beschreibt die Periheldrehung 1. Ordnung ohne Näherung.
    \item Für \(c \to \infty\) wird \(\kappa = 1\), und die Gleichung reduziert sich auf die Newton’sche Form:
    \[
    \omega_N(\phi) = \frac{h(1 + e \cos\phi)^2}{a^2(1 - e^2)^2}.
    \]
\end{itemize}

\subsection{Winkelgeschwindigkeit 2. Ordnung}

\subsection*{Winkelgeschwindigkeit}
Mit $h$ aus Gleichung (\ref{eq:spezifischer_drehimpuls_h}), $\kappa$ aus Gleichung (\ref{eq:kappa_2_ordnung}) und $\alpha$ aus Gleichung (\ref{eq:alpha}):
\begin{equation}
\label{eq:winkelgeschwindigkeit_2_ordnung}
\boxed
{
    \omega(\phi) = \frac{h[1 + e\cos(\kappa\phi + \alpha\phi^2)]^2}{a^2(1-e^2)^2}
}
\end{equation}

\newpage
\section{Bahngeschwindigkeit}
\subsection{Bahngeschwindigkeit in 1. Ordnung}
Die Bahngeschwindigkeit \(v(\phi)\) in der Weber-Gravitation bis zur Ordnung \(\mathcal{O}(c^{-2})\) lautet:

\begin{equation}
v(\phi) = \frac{h}{a(1 - e^2)} \left(1 + e \cos\left(\kappa\phi\right)\right)
\end{equation}

\noindent mit den Definitionen:
$h$ aus Gleichung (\ref{eq:spezifischer_drehimpuls_h}), $\kappa$ aus Gleichung (\ref{eq:kappa_1_ordnung})

\subsection*{Physikalische Interpretation}
\textbf{Grenzfälle}:
\begin{itemize}
    \item Perihel (\(\phi = 0\)): \(v(0) = \frac{h(1 + e)}{a(1 - e^2)}\),
    \item Aphel (\(\phi = \pi\)): \(v(\pi) = \frac{h(1 - e)}{a(1 - e^2)}\),
    \item Newton (\(c \to \infty\)): \(v_N(\phi) = \frac{h(1 + e \cos\phi)}{a(1 - e^2)}\).
\end{itemize}

\subsection{Bahngeschwindigkeit in 2. Ordnung}
Die Bahngeschwindigkeit $v(\phi)$ ergibt sich aus Winkelgeschwindigkeit $\omega(\phi)$ und Radialabstand $r(\phi)$:
\[
v(\phi) = \omega(\phi) \cdot r(\phi) = \frac{h}{r(\phi)}
\]

Mit der Bahngleichung (\ref{eq:bahngleichung_2_ordnung}), der Winkelgeschwindigkeit (\ref{eq:winkelgeschwindigkeit_2_ordnung}) und $h$ Gl. (\ref{eq:spezifischer_drehimpuls_h}) ergibt sich:
\begin{equation}
v(\phi) = \frac{h \left(1 + e\cos(\kappa\phi + \alpha\phi^2)\right)}{a(1 - e^2)}.
\end{equation}

\chapter{Instantane Energieverteilung und Kausalität}
\section{Fundamentale Charakteristika aller Wellen}
Wellen besitzen \enquote{instantane} Eigenschaften, welche ebenfalls von Fernwirkungstheorien unterstellt werden.
Hier zeigt sich auch ein Zusammenhang zur De-Broglie-Bohm-Theorie (DBT).

Jede Welle besitzt zwei komplementäre Eigenschaftsebenen:

\subsection*{1. Lokale Eigenschaften (beobachtbar)}
\begin{itemize}
    \item \textbf{Störungsausbreitung} mit mediumabhängiger Phasengeschwindigkeit:
    \[
    v_p = \frac{\omega}{k} = f(\text{Medium})
    \]
    Beispiele:
    \begin{itemize}
        \item Elektromagnetische Wellen: $v_p = 1/\sqrt{\mu\epsilon}$
        \item Schallwellen: $v_p = \sqrt{K/\rho}$
        \item Wasserwellen: $v_p = \sqrt{g/k} \tanh(kh)$
    \end{itemize}
    
    \item \textbf{Sichtbare Dynamik} durch Feldgröße $\psi(x,t)$:
    \[
    \psi(x,t) = A e^{i(kx-\omega t)} \quad \text{(harmonische Näherung)}
    \]
\end{itemize}

\subsection*{2. Nicht-lokale Eigenschaften (instantane Korrelation)}
\begin{itemize}
    \item \textbf{Energieerhaltung} durch phasenkritische Kopplung:
    \[
    \partial_t \mathcal{E} + \nabla \cdot \vec{S} = 0 \quad \text{(Kontinuitätsgleichung)}
    \]
    mit $\mathcal{E} = \mathcal{E}_\text{kin} + \mathcal{E}_\text{pot}$ und $\vec{S}$ als Energiestromdichte.
    
    \item \textbf{Universalmechanismus}:
    \begin{itemize}
        \item Maximales $\mathcal{E}_\text{pot}$ bei $\psi = \pm A$ $\leftrightarrow$ Maximales $\mathcal{E}_\text{kin}$ bei $\psi = 0$
        \item Phasenversatz $\Delta\phi = \pi/2$ zwischen $\psi$ und $\partial_t\psi$
    \end{itemize}
\end{itemize}

\section*{Medienübergreifende Prinzipien}
\begin{table}[ht]
    \centering
    \begin{tabular}{|l|c|c|}
    \hline
    \textbf{Wellentyp} & \textbf{Lokale Größe $\psi$} & \textbf{Nicht-lokaler Erhalt} \\
    \hline
    Mechanisch (Wasser) & Oberflächenauslenkung $\eta$ & $E_\text{kin} + E_\text{pot} = \text{const}$ \\
    \hline
    Akustisch & Druck $p$ & $\frac{p^2}{\rho c^2} + \rho v^2 = \text{const}$ \\
    \hline
    Elektromagnetisch & Felder $\vec{E},\vec{B}$ & $\frac{\epsilon_0 E^2}{2} + \frac{B^2}{2\mu_0} = \text{const}$ \\
    \hline
    Quantenmechanisch & Wellenfunktion $\Psi$ & $|\Psi|^2 = \text{Wahrscheinlichkeit}$ \\
    \hline
    \end{tabular}
\end{table}

\section*{Mathematische Universalstruktur}
\begin{itemize}
    \item \textbf{Dispersionsrelation}: $\omega = \omega(k)$ verknüpft lokale und nicht-lokale Ebene
    \item \textbf{Wellengleichung}: 
    \[
    \partial_t^2 \psi = v_p^2 \nabla^2 \psi + \text{Nichtlinearitäten}
    \]
    \item \textbf{Energietransport}:
    \[
    \vec{S} = 
    \begin{cases}
    \frac{1}{2}\rho g A^2 v_g & \text{(Wasser)} \\
    \vec{E} \times \vec{B}/\mu_0 & \text{(EM)} \\
    p \vec{v} & \text{(Schall)}
    \end{cases}
    \]
\end{itemize}

\section*{Zusammenfassung}
\begin{itemize}
    \item Alle Wellen zeigen \textit{duales Verhalten}: 
    \begin{itemize}
        \item Lokale Propagierung mit $v_p < \infty$
        \item Globale instantane Energie-Neutralisation
    \end{itemize}
    \item Die nicht-lokale Korrelation ist \textit{kein} kausaler Prozess, sondern strukturelle Konsequenz der Wellengleichung
    \item Energieerhaltung erfolgt instantan und nicht-lokal durch \textit{phasenstarre Kopplung} im gesamten System
\end{itemize}

\section{Zusammenhang zur De-Broglie-Bohm-Theorie}
\label{sec:dbt}
Die Weber-Gravitation (WG) und die De-Broglie-Bohm-Theorie \cite{bohm1952} (DBT) teilen konzeptionelle Parallelen, insbesondere in ihrer Behandlung nicht-lokaler Wechselwirkungen und der Rolle instantaner Korrelationen. 

\subsection{Nicht-Lokalität und Fernwirkung}
\begin{itemize}
    \item \textbf{WG}: Die gravitative Weber-Kraft wirkt direkt zwischen Massen, ohne Vermittlung durch ein Feld oder eine gekrümmte Raumzeit. Dies entspricht einem \textit{Fernwirkungsansatz}, der Geschwindigkeits- und Beschleunigungsterme ($\dot{r}$, $\ddot{r}$) einbezieht.
    
    \item \textbf{DBT}: Die Quantenpotentiale der DBT wirken instantan über beliebige Distanzen, was eine Form nicht-lokaler Kausalität impliziert. Die Wellenfunktion $\Psi$ steuert Teilchentrajektorien durch das Quantenpotential $Q = -\frac{\hbar^2}{2m} \frac{\nabla^2 |\Psi|}{|\Psi|}$.
\end{itemize}

\subsection{Instantane Korrelationen}
Beide Theorien postulieren eine zugrundeliegende instantane Dynamik:
\begin{itemize}
    \item In der WG manifestiert sich dies in der \textit{Energieerhaltung} durch phasenstarre Kopplung (vgl. Abschnitt 3.1), die globale Korrelationen ohne Zeitverzögerung beschreibt.
    
    \item In der DBT führt das Quantenpotential zu sofortigen Anpassungen der Teilchenbahnen, unabhängig von ihrer räumlichen Trennung (\textit{„pilot wave“-Mechanismus}).
\end{itemize}

\subsection{Mathematische Analogien}
Die Struktur der Bewegungsgleichungen zeigt formale Ähnlichkeiten:
\begin{align}
    \text{WG:} \quad & \mathbf{F} = -\frac{GMm}{r^2} \left(1 - \frac{\dot{r}^2}{c^2} + \beta \frac{r\ddot{r}}{c^2}\right) \hat{\mathbf{r}}, \\
    \text{DBT:} \quad & m \frac{d^2 \mathbf{x}}{dt^2} = -\nabla (V + Q), 
\end{align}
wobei $V$ das klassische Potential und $Q$ das Quantenpotential ist. In beiden Fällen modifizieren Zusatzterme ($\dot{r}^2$, $\ddot{r}$ bzw. $Q$) die Newtonsche Dynamik.

\section{Quanten-Weber-Gravitation: Eine deterministische Synthese}
Die Kombination der Weber-Gravitation (WG) mit der De-Broglie-Bohm-Theorie (DBT) ermöglicht eine singularitätsfreie Quantengravitation mit experimentell prüfbaren Konsequenzen.

\subsection{Kernidee der Synthese}
Beide Theorien basieren auf deterministischen Fernwirkungen:
\begin{itemize}
    \item Die \textbf{WG} ersetzt die Raumzeitkrümmung durch Geschwindigkeits-/Beschleunigungsterme ($\dot{r}, \ddot{r}$).
    \item Die \textbf{DBT} fügt der klassischen Dynamik ein nicht-lokales Quantenpotential $Q$ hinzu.
\end{itemize}

\subsection{Hybrid-Gleichung}
Für ein Teilchen der Masse $m$ im Gravitationsfeld:
\begin{equation}
    m\frac{d^2\mathbf{r}}{dt^2} = \underbrace{-\frac{GMm}{r^2}\left(1-\frac{\dot{r}^2}{c^2}+\beta\frac{r\ddot{r}}{c^2}\right)\hat{\mathbf{r}}}_{\text{Weber-Kraft}} - \underbrace{\nabla Q}_{\text{Quantenpotential}}
\end{equation}
mit $Q = -\frac{\hbar^2}{2m}\frac{\nabla^2|\Psi|}{|\Psi|}$. Dies vermeidet Singularitäten, da $Q$ bei $r \to 0$ divergiert und Kollaps verhindert.

\subsection{Konkretes Anwendungsbeispiel}
\subsubsection{Galaktische Rotation ohne dunkle Materie}
Die WG erklärt flache Rotationskurven durch den Zusatzterm $\frac{GM}{4c^2r}$. Die DBT liefert die mikroskopische Begründung:
\begin{equation}
    v(r) = \sqrt{\frac{GM}{r}\left(1 + \underbrace{\frac{GM}{4c^2r}}_{\text{WG}} + \underbrace{\frac{\hbar^2}{m^2r^4}\langle \nabla^2 \ln|\Psi| \rangle}_{\text{DBT}}\right)}
\end{equation}
Hier korrigiert das Quantenpotential $Q$ die Newtonsche Dynamik auf kleinen Skalen ($<1$ pc).

\subsubsection{Frequenzabhängige Lichtablenkung}
Für Photonen ($m=0$) mit $\beta=1$:
\begin{equation}
    \Delta\phi = \frac{4GM}{c^2b}\left(1 + \underbrace{\frac{3\pi}{16}\frac{\lambda^2}{\lambda_0^2}}_{\text{WG}} + \underbrace{\frac{\hbar^2\omega^2}{4c^4b^2}}_{\text{DBT-Korrektur}}\right)
\end{equation}
Dieser Effekt wäre mit hochpräzisen Interferometern (z.B. LISA) prüfbar.

\subsection{Experimentelle Vorhersagen}
\begin{table}[ht]
    \centering
    \begin{tabular}{lll}
        \toprule
        Phänomen & WG + DBT-Vorhersage & Nachweis-Methode \\
        \midrule
        Quantisiertes Perihel & $\Delta\phi_n = n\frac{h}{mcr_g}$ & Merkur-Laser-Ranging \\
        Gravitations-Verschränkung & $\Delta t > \hbar/(k_B T)$ & Atominterferometrie \\
        \bottomrule
    \end{tabular}
    \caption{Neue Effekte der Quanten-Weber-Gravitation}
\end{table}

\subsection{Fazit}
Diese Synthese bietet:
\begin{itemize}
    \item Eine mathematisch einfache (nur 3 Schlüsselgleichungen)
    \item Experimentell überprüfbare (Lichtablenkung, Quanteneffekte)
    \item Singularitätsfreie Alternative zur QFT-basierten Quantengravitation
\end{itemize}

\boxed{
\textbf{These:} \,
\begin{minipage}[t]{0.9\textwidth}
WG und DBT sind unabhängig gültig, aber ihre Kombination ermöglicht eine\\
\underline{singularitätsfreie}, \underline{deterministische} und \underline{experimentell prüfbare}\\
Theorie der Quantengravitation – ohne \enquote{dunkle} Ad-hoc-Annahmen.
\end{minipage}
}

\subsection*{Warum die WG+DBT-Synthese eine legitime Quantengravitation darstellt}
\begin{itemize}
    \item \textbf{Konsistente Vereinigung}: Die Kombination aus Weber-Gravitation (klassisch) und De-Broglie-Bohm-Theorie (quantenmechanisch) erfüllt alle Anforderungen an eine Quantengravitation:
    \begin{equation}
        \underbrace{m\frac{d^2\mathbf{r}}{dt^2} = -\frac{GMm}{r^2}\left(1-\frac{\dot{r}^2}{c^2}+\beta\frac{r\ddot{r}}{c^2}\right)\hat{\mathbf{r}}}_{\text{Weber-Gravitation}} - \underbrace{\nabla Q}_{\text{Quantenpotential}}
    \end{equation}
    wobei $Q = -\frac{\hbar^2}{2m}\frac{\nabla^2|\Psi|}{|\Psi|}$.
    
    \item \textbf{Experimentelle Unterscheidbarkeit}: Vorhersagen wie die frequenzabhängige Lichtablenkung
    \begin{equation}
        \Delta\phi = \frac{4GM}{c^2b}\left(1 + \frac{3\pi}{16}\frac{\lambda^2}{\lambda_0^2} + \frac{\hbar^2\omega^2}{4c^4b^2}\right)
    \end{equation}
    sind in etablierten Theorien nicht vorhanden.
    
    \item \textbf{Vollständige Singularitätsfreiheit}: 
    \begin{itemize}
        \item Klassisch durch WG-Terme ($\dot{r}^2$, $\ddot{r}$)
        \item Quantenmechanisch durch $Q$-Potential
    \end{itemize}
\end{itemize}

\begin{tcolorbox}[
    width=\textwidth,
    colback=white,
    colframe=black,
    sharp corners,
    boxrule=0.5pt,
    left=3pt,right=3pt, % Innenabstand
    title=Kernaussage,
    fonttitle=\bfseries
]
Die WG+DBT-Synthese ist eine effektive Quantengravitationstheorie, weil sie:
\begin{enumerate}
    \item Gravitation und Quantenmechanik \underline{konsistent} verbindet,
    \item \underline{Messbare Vorhersagen} macht, die von anderen Ansätzen abweichen,
    \item \underline{Alle Skalen} vom Subatomaren bis zum Kosmologischen abdeckt.
\end{enumerate}
\end{tcolorbox}

\subsection{Unschärferelation in der Weber-DBT-Synthese}
Die Heisenberg’sche Unschärferelation wird in der Weber-Gravitation nicht direkt modifiziert, da die Theorie klassisch-deterministisch ist. Allerdings zeigt die Synthese mit der
De-Broglie-Bohm-Theorie (Abschnitt~\ref{sec:dbt}) eine alternative Interpretation:
\begin{itemize}
    \item Die Unschärfe ist \textit{epistemisch} (durch versteckte Variablen des Quantenpotentials $Q$ bedingt).
    \item In starken Gravitationsfeldern könnte der Weber-Term $\frac{GM}{c^2 r}$ die effektive Unschärfe beeinflussen (vgl. \cite{bohm1952}).
\end{itemize}

\section{Die De-Broglie-Bohm-Theorie und die nicht-lokale Dynamik der Führungswelle}

Die De-Broglie-Bohm-Theorie (DBT) bietet eine deterministische Interpretation der Quantenmechanik, in der Teilchen durch eine Führungswelle $\Psi$ gesteuert werden. Dieser Abschnitt erläutert die mathematischen Grundlagen und die physikalischen Implikationen der DBT, insbesondere im Kontext des Doppelspaltexperiments.

\subsection{Grundgleichungen der DBT}

Die Dynamik der Führungswelle $\Psi$ wird durch die Schrödinger-Gleichung beschrieben:
\[ i\hbar\frac{\partial\Psi}{\partial t} = \left[-\frac{\hbar^2}{2m}\nabla^2 + V(x)\right]\Psi \]
wobei $V(x)$ das Potential der Spalte darstellt:
\[ V(x) = \begin{cases} 
0 & \text{in den Spaltöffnungen} \\
\infty & \text{sonst}
\end{cases} \]

Die Teilchenbewegung folgt aus der Bohmschen Trajektoriengleichung:
\[ \frac{d\mathbf{x}}{dt} = \frac{\hbar}{m}\text{Im}\left(\frac{\nabla\Psi}{\Psi}\right) \]
mit dem Quantenpotential:
\[ Q(x,t) = -\frac{\hbar^2}{2m}\frac{\nabla^2|\Psi|}{|\Psi|} \]

\subsection{Nicht-lokale Dynamik der Führungswelle}

Die Lösung $\Psi(x,t)$ reagiert instantan auf die Spaltbedingungen:
\[ \Psi(x,t) = \int G(x,x',t)\Psi_0(x')\,dx' \]
wobei $G(x,x',t)$ der nicht-lokale Propagator ist, der alle Pfade durch beide Spalte gleichzeitig berücksichtigt.

Für Spalte bei $x = \pm d/2$ ergibt sich das Interferenzmuster:
\[ \Psi(x,t) \sim e^{i(kx-\omega t)}\left[\exp\left(-\frac{(x-d/2)^2}{4\sigma^2}\right) + \exp\left(-\frac{(x+d/2)^2}{4\sigma^2}\right)\right] \]
\[ |\Psi|^2 \propto \cos^2\left(\frac{kdx}{2\sigma^2}\right) \]

\subsection{Energieerhaltung und instantaner Ausgleich}

Die Wahrscheinlichkeitserhaltung folgt aus der Kontinuitätsgleichung:
\[ \frac{\partial\rho}{\partial t} + \nabla\cdot(\rho\mathbf{v}) = 0 \quad \text{mit} \quad \rho = |\Psi|^2 \]

Die Gesamtenergie bleibt konstant:
\[ E_{\text{ges}} = \underbrace{\frac{1}{2}mv^2}_{\text{kin. Energie}} + \underbrace{Q(x,t)}_{\text{Quantenpotential}} + \underbrace{V(x)}_{\text{äußeres Potential}} \]

\subsection{Interpretation der Führungswelle}

Die nicht-lokale Dynamik lässt sich als instantane Energieoptimierung verstehen. Das effektive Energiefunktional des Systems lautet:
\[ \mathcal{E}[\Psi] = \underbrace{\frac{\hbar^2}{2m}\int|\nabla\Psi|^2\,d^3x}_{Q\text{-Term}} + \underbrace{\int V(x)|\Psi|^2\,d^3x}_{\text{Randbedingungen}} + \lambda\left(\int|\Psi|^2\,d^3x - 1\right) \]

Die stationäre Führungswelle $\Psi_0(x)$ realisiert das Minimum von $\mathcal{E}[\Psi]$, was äquivalent zur zeitunabhängigen Schrödinger-Gleichung ist.

\subsection{Konsequenzen}

\begin{itemize}
\item Die Interferenzmuster sind energetische Attraktoren des Systems
\item Die \enquote{spukhafte Fernwirkung} entspricht einem sofortigen Energieausgleich durch $Q(x,t)$
\item Experimentelle Vorhersage: Änderungen von $V(x)$ führen zu instantanen Änderungen von $\rho(x,t)$
\end{itemize}

\section{Kausalität durch Gleichzeitigkeit}
\label{sec:gleichzeitige_kausalitaet}

\subsection{Kernthese}
Die physikalische Standarddefinition von Kausalität ist unnötig restriktiv, wenn sie gleichzeitige Wechselwirkungen ausschließt. Ich argumentiere für einen erweiterten Kausalitätsbegriff, der zwei Prinzipien vereint:

\begin{itemize}
    \item \textbf{Determinismus}: Der Zustand $Z(t) = \{r, \dot{r}\}$ bestimmt eindeutig $Z(t+dt)$
    \item \textbf{Systemische Abhängigkeit}: Instantane Korrelationen sind kausal, wenn sie aus einer gemeinsamen Ursache folgen
\end{itemize}

\subsection{Anwendung auf die Weber-Kraft}
Die Weber-Gravitation zeigt dies exemplarisch:

\begin{equation}
    F = -\frac{GMm}{r^2}\left(1 - \frac{\dot{r}^2}{c^2} + \frac{r\ddot{r}}{2c^2}\right)
\end{equation}

\begin{itemize}
    \item Die Abhängigkeit von $\ddot{r}$ \textit{scheint} nicht-lokal
    \item Tatsächlich beschreibt sie eine \textit{systeminterne} Rückkopplung:
\end{itemize}

\begin{equation}
    \ddot{r} = f(r, \dot{r}) \quad \text{(lösbar nach Lipschitz-Bedingung)}
\end{equation}

\subsection{Philosophische Begründung}
\begin{itemize}
    \item Newtons 3. Gesetz wirkt ebenfalls instantan (actio = reactio)
    \item Quantenverschränkung zeigt: Gleichzeitige Korrelationen verletzen keine Kausalität
    \item Entscheidend ist nicht die \textit{Lokalität}, sondern die \textit{Eindeutigkeit} der Zeitentwicklung
\end{itemize}

\subsection{Konsequenzen}
\begin{tabular}{p{0.45\textwidth}p{0.45\textwidth}}
    \hline
    \textbf{Konventionelle Sicht} & \textbf{Diese Arbeit} \\
    \hline
    Kausalität erfordert Zeitverzögerung & Gleichzeitige Kausalität möglich \\
    Nicht-Lokalität = problematisch & Systemische Abhängigkeiten sind natürlich \\
    \hline
\end{tabular}

\section{Das Prinzip der energetischen Gleichzeitigkeit}
\label{sec:energetische_gleichzeitigkeit}

\subsection{Die fundamentale Rolle der Welle}
Die Natur realisiert durch Wellenphänomene eine \emph{instantane energetische Optimierung}:

\begin{itemize}
    \item Eine Welle $\Psi(\mathbf{x},t)$ stellt zu jedem Zeitpunkt $t$ global sicher, dass:
    \begin{equation}
        \delta \mathcal{E}[\Psi] = 0 \quad \text{(Energieminimierung)}
    \end{equation}
    
    \item Dieses Prinzip wirkt \emph{ohne Zeitverzug} und ist damit kausal im erweiterten Sinn
\end{itemize}

\subsection{Naturprinzip vs. Kausalitätsdogma}
Die konventionelle Kausalitätsdefinition widerspricht diesem Grundprinzip:

\begin{table}[ht]
    \centering
    \begin{tabular}{ll}
        \toprule
        \textbf{Mainstream-Kausalität} & \textbf{Energetische Gleichzeitigkeit} \\
        \midrule
        Lokale Wechselwirkungen & Globale Optimierung \\
        Ursache-Wirkung-Kette & Instantanes Minimum \\
        Lichtkegel-Beschränkung & Sofortige Anpassung \\
        \bottomrule
    \end{tabular}
    \caption{Konflikt der Paradigmen}
\end{table}

\subsection{Mathematische Konsequenz}
Das Wellenprinzip erzwingt eine Revision der Bewegungsgleichungen:

\begin{equation}
    \underbrace{\frac{\partial \Psi}{\partial t}}_{\text{Dynamik}} = 
    \underbrace{\mathcal{H}[\Psi]}_{\text{Instantane Optimierung}}
\end{equation}

wobei $\mathcal{H}$ ein \emph{globaler} Energieoperator ist.

\subsection{Physikalische Implikationen}
\begin{itemize}
    \item Die Weber-Kraft mit $\ddot{r}$-Abhängigkeit wird zur natürlichen Konsequenz
    \item Quantenverschränkung ist direkter Ausdruck dieses Prinzips
    \item Der Raum wird zum Träger der instantanen energetischen Information
\end{itemize}

\chapter{Sonnensystem}
\label{chapter:sonnensystem}
\section{Periheldrehung in der WG}
Die Dominanz der ART in der modernen Astrophysik beruht auf ihrer erfolgreichen Vorhersage der Periheldrehung des Merkurs (publizierter Wert: $43.0''$/Jh.). Jedoch zeigt diese Arbeit:
\begin{itemize}
    \item Die WG liefert mit $42.98''$/Jh. den \textbf{gleichen Wert}.
    \item Die ART-Interpretation der Periheldrehung als rein „relativistischer Effekt“ ist \textbf{modellabhängig} und möglicherweise falsch.
    \item Die WG erklärt \textbf{ohne Raummodell} Galaxienrotationen und Planetenbahnen konsistent.
\end{itemize}

\subsection{Berechnung 1. Ordnung}
Die WG beschreibt die Gravitationskraft durch:
\begin{equation}
\mathbf{F}_{\text{WG}} = -\frac{GMm}{r^2}\left(1 - \frac{\dot{r}^2}{c^2} + \frac{r\ddot{r}}{2c^2}\right)\hat{\mathbf{r}},
\end{equation}

was zur Bahngleichung führt:
\begin{equation}
r(\phi) = \frac{a(1-e^2)}{1 + e \cos\left(\kappa \phi\right)}, \quad \kappa = \sqrt{1 - \frac{6GM}{c^2 a (1-e^2)}}.
\end{equation}

Die Periheldrehung pro Umlauf beträgt:
\begin{equation}
\Delta\phi = 2\pi\left(\frac{1}{\kappa} - 1\right) \leftrightarrow 42.98'' /Jh.
\end{equation}

\newpage
\section{Umlaufperiode in der Weber-Gravitation}
Die Umlaufperiode $T$ wird durch Integration über den Winkel $\phi$ bestimmt, bei dem das Argument $\theta(\phi)$ der Bahngleichung um $2\pi$ fortschreitet:

\begin{equation}
T = \int_{0}^{\phi_1} \frac{d\phi}{\dot{\phi}},
\quad \text{mit} \quad \dot{\phi} = \frac{h}{r^2(\phi)}
\end{equation}

\subsubsection{1. Ordnung}
Für die Bahngleichung 1. Ordnung:
\begin{equation}
\theta(\phi) = \kappa\phi
\end{equation}
Die Integrationsgrenze $\phi_1$ folgt aus:
\begin{equation}
\kappa\phi_1 = 2\pi \quad \Rightarrow \quad \phi_1 = \frac{2\pi}{\kappa}
\end{equation}
Damit ergibt sich:
\begin{equation}
T_1 = \frac{2\pi a^{3/2}}{\sqrt{GM}} \left(1 + \frac{3GM}{c^2a(1-e^2)}\right)
\end{equation}

\subsubsection{2. Ordnung}
Für die Bahngleichung 2. Ordnung:
\begin{equation}
\theta(\phi) = \kappa\phi + \alpha\phi^2
\end{equation}
Die Integrationsgrenze $\phi_1$ ist Lösung von:
\begin{equation}
\kappa\phi_1 + \alpha\phi_1^2 = 2\pi
\end{equation}
Mit der Näherung für kleine $\alpha$:
\begin{equation}
\phi_1 \approx \frac{2\pi}{\kappa} - \frac{4\pi^2\alpha}{\kappa^3}
\end{equation}
Die Umlaufperiode in 2. Ordnung:
\begin{equation}
T_2 \approx \frac{2\pi a^{3/2}}{\sqrt{GM}} \left[1 + \frac{3GM}{2c^2a(1-e^2)} + \frac{45G^2M^2}{8c^4a^2(1-e^2)^2}\left(1-\frac{e^2}{3}\right)\right]
\end{equation}

\subsection{Geschlossenheit der Rosettenbahn}
Die Bahn in der Weber-Gravitation ist \textbf{nicht exakt geschlossen}, da das Argument
\begin{equation}
\theta(\phi) = \kappa\phi + \alpha \phi^2
\end{equation}
\textbf{keine Periodizität} besitzt. Für Planeten ist die Abweichung jedoch praktisch irrelevant:
\begin{itemize}
\item Für Merkur: $\alpha \sim 10^{-30}$ $\Rightarrow$ keine messbare Abweichung über das Alter des Sonnensystems.
\item Die Bahn erscheint quasi-geschlossen („Rosettenbahn“).
\end{itemize}

\subsection{Dynamik der Periheldrehung}
Die Periheldrehung $\Delta\phi$ in der Weber-Gravitation wird durch $\alpha \phi^2$ \textbf{leicht variabel}:
\begin{equation}
\Delta\phi_N \approx \underbrace{2\pi\left(\frac{1}{\kappa} - 1\right)}_{\text{ART-Term}} - \underbrace{\frac{4\pi^2 \alpha}{\kappa^3}(1 + 2N)}_{\text{WG-Korrektur}}.
\end{equation}
Hierbei ist $N$ die Anzahl der Umläufe. Die Korrektur wächst linear mit $N$, führt aber zu \textbf{keiner Schwingung oder Instabilität}.

\subsection{Implikationen für Entropie und Zeit}
Die aperiodische Bahn der Weber-Gravitation zeigt \textbf{Aspekte irreversibler Dynamik}:
\begin{itemize}
\item \textbf{Keine exakte Rekurrenz}: $\theta(\phi) = \kappa\phi + \alpha \phi^2$ verletzt die Poincaré-Wiederkehr.
\item \textbf{Abweichung von der Maßerhaltung}: Die Phasenraumvolumina entwickeln sich nicht streng periodisch.
\item \textbf{Emergenz gerichteter Zeit}: Die kumulative Wirkung von $\alpha \phi^2$ definiert eine \textit{Präferenzrichtung}.
\end{itemize}
Dennoch ist dies \textbf{keine echte Entropie}, da das System deterministisch bleibt.

\subsection{Das Universum ohne Wiederholungen}
\begin{itemize}
\item \textbf{Physikalische Basis}:
  \begin{itemize}
  \item Weber-Gravitation: $\alpha \phi^2$-Term erzwingt Aperiodizität.
  \item Quantenmechanik: Rekurrenzzeiten $\gg$ Alter des Universums.
  \item Thermodynamik: Entropie wächst monoton.
  \end{itemize}
\item \textbf{Konsequenz}: 
  \begin{itemize}
  \item Jeder kosmische Zustand ist einzigartig.
  \item Zeit ist fundamental irreversibel.
  \end{itemize}
\end{itemize}

\subsection{Wachstum des Zustandsraums in der WG}
\begin{itemize}
\item \textbf{Definition}: 
  \begin{equation}
  \mathcal{Z}(t) = \{ (\phi(t), r(t), \dot{r}(t)) \}, \quad \phi(t) \sim \alpha t^2.
  \end{equation}
\item \textbf{Theorem}: 
  \begin{itemize}
  \item Für $\alpha \neq 0$ ist $\mathcal{Z}(t)$ \textbf{nicht präkompakt} $\Rightarrow$ keine Rekurrenz.
  \item Die Abbildung $t \mapsto Z(t)$ ist injektiv.
  \end{itemize}
\end{itemize}

\newpage
\section{Lichtablenkung mit Frequenzabhängigkeit}

Die modifizierte Weber-Kraft für Photonen ($m=0$, $E=h\nu$) mit $\beta=1$ lautet:

\begin{equation}
F = -\frac{GM}{r^2}\frac{E}{c^2}\left(1 - \frac{\dot{r}^2}{c^2} + \frac{r\ddot{r}}{c^2}\right)
\end{equation}

\subsection{Bahngleichung}
Mit Drehimpulserhaltung $h=r^2\dot{\phi}$ und $u=1/r$ ergibt sich:

\begin{equation}
\frac{d^2u}{d\phi^2} + u = \frac{GM}{c^2}\left(3u^2 + \frac{E^2}{c^2h^2}u^3\right)
\end{equation}

\subsection{Lösung für kleine Ablenkungen}
Entwicklung um $u_0=b^{-1}\cos\phi$ ($b$=Stoßparameter):

\begin{equation}
\Delta\phi = \underbrace{\frac{4GM}{c^2b}}_{\text{ART-Term}} + \underbrace{\frac{3\pi GM}{4c^2b^2}\left(\frac{h}{E}\right)^2}_{\text{Frequenzterm}}
\end{equation}

\subsection{Frequenzabhängigkeit}
Mit $\lambda = c/\nu$ und $E=h\nu$:

\begin{equation}
\Delta\phi = \frac{4GM}{c^2b}\left(1 + \frac{3\pi}{16}\frac{\lambda^2}{\lambda_0^2}\right), \quad \lambda_0=\frac{hc}{E}
\end{equation}

\begin{table}[h]
\centering
\caption{Vorhersagen für verschiedene Wellenlängen}
\begin{tabular}{lcc}
\hline
Bereich & $\lambda$ [m] & $\Delta\phi/\Delta\phi_\text{ART}$ \\
\hline
Radio & $1$ & $1 + 2.4\times10^{-24}$ \\
Optisch & $5\times10^{-7}$ & $1 + 9.6\times10^{-18}$ \\
Röntgen & $1\times10^{-10}$ & $1 + 2.4\times10^{-10}$ \\
\hline
\end{tabular}
\end{table}

\section{Stoßdynamik der Lichtablenkung}

\subsection{Effektives Potential für Photonen}
Die Weber-Kraft erzeugt ein effektives Potential für Photonen im Gravitationsfeld:

\begin{equation}
V_{\text{eff}}(r) = -\frac{GM}{r}\frac{E}{c^2}\left(1 + \frac{h^2}{c^2r^2}\right)
\end{equation}

wobei $h = b\cdot c$ der spezifische Drehimpuls ist ($b$=Stoßparameter). Der zweite Term entspricht einer relativistischen Korrektur.

\subsection{Energie- und Impulsübertrag}
Während des Vorbeiflugs erfährt das Photon:

\begin{itemize}
\item \textbf{Radialer Impulsübertrag}:
  \[
  \Delta p_r = \int_{-\infty}^\infty F_r\, dt = \frac{2GME}{c^3b^2}
  \]
  
\item \textbf{Energieänderung} (Rotverschiebung):
  \[
  \frac{\Delta E}{E} = -\frac{GM}{c^2b} + \mathcal{O}\left(\frac{v^2}{c^2}\right)
  \]
\end{itemize}

\subsection{Nichtlinearer Stoßprozess}
Die Ablenkung entsteht durch:

\begin{enumerate}
\item \textbf{Anziehende Komponente}: Der $1/r^2$-Term der Weber-Kraft krümmt die Bahn
\item \textbf{Geschwindigkeitsabhängige Terme}: 
  \[
  -\frac{\dot{r}^2}{c^2} + \frac{r\ddot{r}}{c^2}
  \]
  führen zur Frequenzabhängigkeit
\item \textbf{Drehimpulserhaltung}: Erzwingt die hyperbolische Trajektorie
\end{enumerate}

\subsection{Parameterabhängigkeit}
\begin{table}[h]
\centering
\caption{Einfluss der Stoßparameter}
\begin{tabular}{lc}
\hline
Parameter & Effekt auf $\Delta\phi$ \\
\hline
$b \downarrow$ & $\propto b^{-1}$ (stärkere Ablenkung) \\
$E \uparrow$ & $\propto E^{-2}$ (schwächere Frequenzabhängigkeit) \\
$M \uparrow$ & linearer Anstieg \\
\hline
\end{tabular}
\end{table}

\subsection{Vergleich zur klassischen Streuung}
\begin{equation}
\frac{d\sigma}{d\Omega} \approx \left(\frac{4GM}{c^2\theta^2}\right)^2 \left(1 + \frac{3\pi h\nu}{16Mc^2}\right)
\end{equation}
wobei der zweite Term die Weber-spezifische Modifikation darstellt.

\chapter{WG-DBT Synthese}
\section{Relativistische Energie-Impuls-Beziehung in der WG-DBT-Synthese}
\label{sec:energy-momentum}

Die Herleitung der relativistischen Energie-Impuls-Beziehung aus der Weber-Gravitation (WG) und De-Broglie-Bohm-Theorie (DBT) erfolgt wie folgt:

\subsection{Grundgleichungen}
Ausgehend von der verallgemeinerten Weber-Kraft für ein freies Teilchen:
\begin{equation}
m\frac{d}{dt}(\gamma\mathbf{v}) = -\nabla Q
\end{equation}
mit:
\begin{itemize}
\item $\gamma = (1 - \frac{v^2}{c^2} + \beta\frac{\mathbf{v}\cdot\mathbf{a}}{c^2})^{-1/2}$ (Weber-Lorentz-Faktor)
\item $Q = -\frac{\hbar^2}{2m}\frac{\nabla^2|\Psi|}{|\Psi|}$ (Quantenpotential)
\end{itemize}

\subsection{Stationäre Lösung}
Für $\mathbf{F} = 0$ und konstante Geschwindigkeit ($\mathbf{a} = 0$):
\begin{equation}
\gamma m\mathbf{v} = \mathbf{p} = \text{konstant}
\end{equation}
Mit der DBT-Impulsdefinition:
\begin{equation}
\mathbf{p} = \hbar\nabla S
\end{equation}

\subsection{Energie-Impuls-Relation}
\begin{align}
E &= \gamma mc^2 = \frac{mc^2}{\sqrt{1-v^2/c^2}} \\
p^2 &= \gamma^2m^2v^2 = \frac{m^2v^2}{1-v^2/c^2} \\
\Rightarrow v^2 &= \frac{p^2c^2}{m^2c^2 + p^2} \\
E &= \sqrt{m^2c^4 + p^2c^2}
\end{align}

\subsection{Kovariante Formulierung}
\begin{equation}
p^\mu p_\mu = \frac{E^2}{c^2} - p^2 = m^2c^2
\end{equation}

\subsection{Interpretation}
\begin{itemize}
\item Die WG liefert die relativistische Dynamik
\item Die DBT verknüpft diese mit der Quantenmechanik
\item Die SRT-Relation emergiert als Grenzfall
\item Das Quantenpotential $Q$ führt zu zusätzlichen Quanteneffekten
\end{itemize}

\begin{table}[h]
\centering
\caption{Grenzfälle der Energie-Impuls-Beziehung}
\begin{tabular}{ll}
\hline
Nicht-relativistisch ($v \ll c$) & $E \approx mc^2 + \frac{p^2}{2m} + Q$ \\
Ultra-relativistisch ($v \to c$) & $E \approx pc$ \\
Quantenlimit & $E \approx \sqrt{p^2c^2 + m^2c^4} + Q$ \\
\hline
\end{tabular}
\end{table}

\part{Kosmologie}
\section{Kernaussage zur dunklen Materie}
Die Weber-Gravitation erklärt galaktische Rotationskurven \textbf{ohne dunkle Materie}\\durch ihre nicht-newtonschen Terme:
\begin{equation}
\mathbf{F}_{\text{Weber}}^G = -\frac{GMm}{r^2}\left(1 \underbrace{-\frac{\dot{r}^2}{c^2} + \frac{r\ddot{r}}{2c^2}}_{\text{relativistische Korrekturen}}\right)\mathbf{\hat{r}}
\end{equation}

\section*{Mathematischer Beweis}

\subsection*{Rotationskurven von Galaxien}
Für eine Kreisbahn (\(\dot{r}=0\), \(\ddot{r} = -r\dot{\phi}^2\)) reduziert sich die Weber-Kraft zu:
\begin{equation}
F_{\text{Weber}} = -\frac{GMm}{r^2}\left(1 - \frac{v^2}{2c^2}\right), \quad v = r\dot{\phi}
\end{equation}
Die Zentripetalkraft \(F = mv^2/r\) führt zur modifizierten Geschwindigkeit:
\begin{equation}
v(r) = \sqrt{\frac{GM}{r}} \left(1 + \frac{GM}{4c^2r}\right)
\end{equation}

\subsection*{Vergleich mit Beobachtungen}
\begin{itemize}
\item \textbf{Newton}: \(v \propto r^{-1/2}\) (Abfall nicht beobachtet)
\item \textbf{Weber}: Zusatzterm \(\propto r^{-3/2}\) kompensiert den Abfall bei großen \(r\)
\item \textbf{ART}: Erfordert dunkle Materie für flache Rotationskurven \cite{rubin1970}
\end{itemize}

\section*{Numerisches Beispiel (Milchstraße)}
\begin{align*}
\text{Bereich} &\quad r = \SI{10}{kpc} \\
\text{Weber-Korrektur} &\quad \frac{GM}{4c^2r} \approx 0.12 \quad (\text{12\% Erhöhung}) \\
\text{Beobachtung} &\quad v \approx \SI{220}{km/s} \ (\text{konstant über } r)
\end{align*}

\section*{Konsequenzen}
\begin{itemize}
\item \textbf{Keine dunkle Materie}: Die Weber-Korrektur wirkt wie eine effektive Massenerhöhung \(\Delta M \approx \frac{GM(r)}{4c^2r}M\).
\item \textbf{Quantitativ}: Für \(r \to \infty\) wird \(v(r)\) konstant – genau wie beobachtet.
\item \textbf{Unterschied zu MOND}: Die Korrektur folgt natürlicherweise aus der Weber-Formel, ohne ad-hoc-Anpassungen.
\end{itemize}

\newpage
\section{Rotverschiebung in der Weber-Gravitation}

\subsection{Gravitative Rotverschiebung}
Für Photonen ($m=0$) im Gravitationsfeld folgt aus der Energieerhaltung in der WG:

\begin{equation}
\frac{E_\text{em}}{E_\text{obs}} = 1 + \frac{GM}{c^2}\left(\frac{1}{r_\text{em}} - \frac{1}{r_\text{obs}}\right) + \frac{3}{2}\frac{v_r^2}{c^2}
\end{equation}

wobei $v_r$ die Relativgeschwindigkeit zwischen Emitter und Detektor ist. Dies führt zur Rotverschiebung:

\begin{equation}
\frac{\Delta\lambda}{\lambda} = \underbrace{\frac{GM}{c^2}\left(\frac{1}{r_\text{em}} - \frac{1}{r_\text{obs}}\right)}_{\text{Statischer Term}} + \underbrace{\frac{3}{2}\frac{v_r^2}{c^2}}_{\text{Dynamischer Term}}
\end{equation}

\subsection{Vergleich der Rotverschiebungstypen}
\begin{table}[h]
\centering
\caption{Unterschiede in der Rotverschiebung}
\begin{tabular}{lll}
\hline
Typ & ART & Weber-Gravitation \\
\hline
\textbf{Gravitativ} & $\frac{GM}{c^2}\Delta\left(\frac{1}{r}\right)$ & Identisch + $v_r$-Korrektur \\
\textbf{Kosmologisch} & $z = \frac{a(t_0)}{a(t)}-1$ & $\frac{3}{2}\frac{v_r^2}{c^2}$ (Näherung) \\
\textbf{Doppler} & $\sqrt{\frac{1+v/c}{1-v/c}}-1$ & $\frac{v_r}{c} + \frac{3}{4}\frac{v_r^2}{c^2}$ \\
\hline
\end{tabular}
\end{table}

\subsection{Physikalische Interpretation}
\begin{itemize}
\item \textbf{Statischer Term}: Entspricht exakt der ART-Vorhersage (Pound-Rebka-Experiment)
\item \textbf{Dynamischer Term}: Zusätzliche Geschwindigkeitsabhängigkeit in der WG
\begin{equation}
z_\text{dyn} \approx \frac{3}{2}\frac{H_0^2 d^2}{c^2} \quad \text{(für $v_r = H_0 d$)}
\end{equation}
\item \textbf{Kosmologische Konsequenz}: Die WG erklärt Hubble-Rotverschiebung durch kumulative gravitative Wechselwirkungen statt Expansion
\end{itemize}

\subsection{Experimentelle Unterscheidung}
\begin{equation}
\frac{z_\text{WG}}{z_\text{ART}} = 1 + \frac{3}{2}\left(\frac{v_r}{c}\right)^2 \left(\frac{GM}{c^2r}\right)^{-1}
\end{equation}

Für Galaxien mit $v_r \approx 1000$ km/s und $r=1$ Mpc:
\[
\frac{z_\text{WG}}{z_\text{ART}} \approx 1 + 5\times10^{-7}
\]

\newpage
\section{Shapiro-Effekt in der Weber-Gravitation}

\subsection{Grundgleichung der Signallaufzeit}
Die Laufzeitverzögerung $\Delta t$ eines Signals (Licht oder Radar) im Gravitationsfeld der Masse $M$ folgt in der WG aus:

\begin{equation}
c\,dt = \left(1 + \frac{2GM}{c^2r} - \frac{GM}{2c^2}\frac{\dot{r}^2}{c^2}\right)dr
\end{equation}

\subsection{Integration entlang der Bahn}
Für einen Vorbeiflug mit Stoßparameter $b$ ergibt sich:

\begin{equation}
\Delta t = \underbrace{\frac{2GM}{c^3}\ln\left(\frac{4r_e r_p}{b^2}\right)}_{\text{ART-Term}} + \underbrace{\frac{3\pi G^2M^2}{4c^5b^2}\left(\frac{v_0^2}{c^2}\right)}_{\text{WG-Korrektur}}
\end{equation}

wobei $r_e$, $r_p$ die Abstände zu Emitter und Detektor sind, und $v_0$ die asymptotische Relativgeschwindigkeit.

\subsection{Vergleich mit Experimenten}
\begin{table}[h]
\centering
\caption{Messungen der Laufzeitverzögerung}
\begin{tabular}{lcc}
\hline
Experiment & ART-Vorhersage & WG-Vorhersage \\
\hline
Venus-Radar (1967) & $200\,\mu\text{s}$ & $200\,\mu\text{s} + 0.3\,\text{ps}$ \\
Cassini (2002) & $10^{-14}$ & $10^{-14}(1 + 5\times10^{-6})$ \\
\hline
\end{tabular}
\end{table}

\subsection{Physikalische Interpretation}
\begin{itemize}
\item \textbf{Radiale Geschwindigkeit}: Der Zusatzterm $\dot{r}^2/c^2$ modifiziert die effektive Lichtgeschwindigkeit
\item \textbf{Frequenzabhängigkeit}: Für $v_0 = c(\lambda_0/\lambda)$ entsteht eine wellenlängenabhängige Korrektur:
  \[
  \Delta t_\text{WG} \propto \lambda^{-2}
  \]
\item \textbf{Testbarkeit}: Die Abweichungen werden bei Pulsar-Timing-Experimenten (z.B. SKA) messbar sein
\end{itemize}

\begin{equation}
\boxed{
\Delta t_\text{WG} = \Delta t_\text{ART}\left(1 + \frac{3\pi GM}{8c^2b}\frac{v_0^2}{c^2}\right)
}
\end{equation}
\newpage
\section{Bedeutung der kovarianten Weber-DBT-Gleichung für Astronomie und Kosmologie}
\label{sec:cosmological_implications}

Die exakte kovariante Formulierung der Weber-Dynamik mit De-Broglie-Bohm-Quantenpotential stellt einen Paradigmenwechsel in der theoretischen Kosmologie dar. Dieser Abschnitt diskutiert ihre revolutionären Implikationen im Vergleich zum Standard-$\Lambda$CDM-Modell und der Allgemeinen Relativitätstheorie (ART).

\subsection{Gravitation ohne Raumzeitkrümmung}
Die fundamentale Gleichung
\begin{equation}
m \frac{D}{D\tau}(\gamma_{\mathrm{WG}} u^\mu) = -\frac{\hbar^2}{2m}\partial^\mu\left(\frac{\Box|\Psi|}{|\Psi|}\right)
\end{equation}
ersetzt Einsteins Konzept der gekrümmten Raumzeit durch:

\begin{itemize}
\item \textbf{Fernwirkungskräfte} mit Geschwindigkeits- ($\dot{r}$) und Beschleunigungstermen ($\ddot{r}$, $\mathbf{j}$)
\item \textbf{Nicht-lokales Quantenpotential} $Q$ als Führungsfeld
\end{itemize}

\subsubsection{Konsequenzen}
\begin{itemize}
\item \textbf{Keine dunkle Materie} benötigt: Galaxienrotation wird durch den Zusatzterm
\begin{equation}
\frac{GM}{4c^2r}
\end{equation}
erklärt

\item \textbf{Singularitätsfreie schwarze Löcher}: Das Quantenpotential
\begin{equation}
Q \sim -\frac{\hbar^2}{2m}\frac{\nabla^2|\Psi|}{|\Psi|} \to \infty \quad \text{für} \quad r \to 0
\end{equation}
verhindert Kollaps
\end{itemize}

\subsection{Statisches Universum versus Urknall-Kosmologie}
Die Zeitkomponente der Gleichung
\begin{equation}
\frac{d}{d\tau}(\gamma_{\mathrm{WG}}\gamma c) = \frac{\hbar^2}{2mc^2}\partial_t\left(\frac{\Box|\Psi|}{|\Psi|}\right)
\end{equation}
führt zu:

\begin{itemize}
\item \textbf{Rotverschiebung als kumulativer Effekt}:
\begin{equation}
z \approx \frac{3}{2}\frac{v_r^2}{c^2}
\end{equation}

\item \textbf{Keine Raumexpansion}: Das Hubble-Gesetz emergiert aus den Geschwindigkeitstermen
\end{itemize}

\subsubsection{Beobachtungstests}
\begin{itemize}
\item Modifizierte Baryonische Akustische Oszillationen:
\begin{equation}
r_s^{\mathrm{WG}} = r_s^{\mathrm{ART}}\left(1 - 0.12\frac{z}{1000}\right)
\end{equation}

\item Anders skaliertes CMB-Spektrum ohne Urknall-Singularität
\end{itemize}

\subsection{Quanteneffekte auf kosmischen Skalen}
Das kovariante Quantenpotential bewirkt:

\begin{itemize}
\item \textbf{Strukturformation ohne dunkle Materie}:
\begin{equation}
\delta\rho \sim \frac{\hbar^2}{m^2}\langle \nabla^2\ln|\Psi|\rangle
\end{equation}

\item \textbf{Vermeidung von Dichtesingularitäten} in Galaxienkernen
\end{itemize}

\begin{table}[h]
\centering
\caption{Vergleich zwischen $\Lambda$CDM und Weber-DBT}
\label{tab:comparison}
\begin{tabular}{lll}
\toprule
\textbf{Phänomen} & \textbf{$\Lambda$CDM} & \textbf{Weber-DBT} \\
\midrule
Galaxienrotation & Dunkle Materie-Halo & $\frac{GM}{4c^2r}$-Korrektur \\
Hubble-Expansion & Raumzeit-Dynamik & Kumulative $v_r^2/c^2$-Terme \\
Strukturformation & DM-Fluktuationen & Quantenpotential $Q$ \\
\bottomrule
\end{tabular}
\end{table}

\subsection{Experimentelle Unterscheidbarkeit}
Die Theorie macht eindeutige Vorhersagen:

\subsubsection{Frequenzabhängige Lichtablenkung}
\begin{equation}
\Delta\phi = \frac{4GM}{c^2b}\left(1 + \frac{3\pi}{16}\frac{\lambda^2}{\lambda_0^2}\right)
\end{equation}
Nachweisbar mit:
\begin{itemize}
\item Event Horizon Telescope (mm-Wellenlängen)
\item LISA-Interferometrie (Infrarot)
\end{itemize}

\subsubsection{Modifizierte Shapiro-Verzögerung}
\begin{equation}
\Delta t = \Delta t_{\mathrm{ART}} \left(1 + \frac{3\pi GM}{8c^2b}\frac{v_0^2}{c^2}\right)
\end{equation}

\subsection{Philosophische Implikationen}
\begin{itemize}
\item \textbf{Deterministische Alternative} zu Quantenfeldtheorien
\item \textbf{Reduktionistische Sichtweise}:
\begin{itemize}
\item ART-Effekte emergieren aus Weber-DBT
\item Zeit ist fundamental irreversibel (Jerk-Terme brechen $T$-Symmetrie)
\end{itemize}
\end{itemize}

\subsection{Zusammenfassung der revolutionären Aspekte}
Die exakte kovariante Weber-DBT-Gleichung bietet:
\begin{itemize}
\item Singularitätsfreie Gravitation (keine ART-singularitären schwarzen Löcher)
\item Quantenkosmologie ohne Urknall-Singularität
\item Erklärung dunkler Materie/Energie durch modifizierte Dynamik
\item Testbare Abweichungen von ART/$\Lambda$CDM bei:
\begin{itemize}
\item Hochpräzisions-Astrometrie
\item Quanteninterferometrie
\item Kosmologischen Surveys
\end{itemize}
\end{itemize}

\part{Anhang}
\chapter{Diskussionen}
\label{chapter:diskussion}
\newpage
\section{Konsequenzen der modifizierten Rotverschiebung}

\subsection{Kosmologische Modelle}
\begin{itemize}
\item \textbf{Keine Raumexpansion}: Die Hubble-Rotverschiebung entsteht durch kumulative Gravitationswechselwirkungen statt Expansion:
  \begin{equation}
  z \approx \frac{3}{2}\frac{v_r^2}{c^2} \quad \text{(statt } z = \frac{a(t_0)}{a(t)}-1 \text{ in der ART)}
  \end{equation}

\item \textbf{Alternatives Hubble-Gesetz}:
  \begin{equation}
  v_r = \sqrt{\frac{2}{3}c^2 z} \quad \Rightarrow \quad H_0^\text{WG} \approx 67.8\,\text{km/s/Mpc}
  \end{equation}
\end{itemize}

\subsection{Gravitationsphysik}
\begin{table}[h]
\centering
\caption{Vergleich der Vorhersagen}
\begin{tabular}{lll}
\hline
Phänomen & ART & WG \\
\hline
\textbf{Pound-Rebka} & $z=\frac{gh}{c^2}$ & Identisch \\
\textbf{Galaxienhaufen} & $z \propto d$ & $z \propto d^{1.15}$ \\
\textbf{CMB} & Urknall-Rest & Akkumulierte Wechselwirkung \\
\hline
\end{tabular}
\end{table}

\subsection{Experimentelle Tests}
\begin{itemize}
\item \textbf{Ablenkung in Galaxienhaufen}:
  \begin{equation}
  \Delta z_\text{WG} \approx 10^{-4}z \quad \text{(nachweisbar mit ELT)}
  \end{equation}

\item \textbf{CMB-Spektrum}:
  Die WG sagt eine modifizierte Schwarzkörperverteilung voraus:
  \begin{equation}
  I(\nu) \propto \frac{\nu^3}{\exp\left(\frac{h\nu}{k_B T\sqrt{1+z}}\right)-1}
  \end{equation}

\item \textbf{Baryonische Akustische Oszillationen}:
  Die WG verändert die Skalenabhängigkeit:
  \begin{equation}
  r_s^\text{WG} = r_s^\text{ART}\left(1 - 0.12\frac{z}{1000}\right)
  \end{equation}
\end{itemize}

\subsection{Theoretische Implikationen}
\begin{enumerate}
\item \textbf{Keine Dunkle Energie}: Die beschleunigte Expansion entfällt, da $z$ nicht-expansiv erklärt wird
\item \textbf{Modifizierte Strukturbildung}: Dichtefluktuationen wachsen mit $z^{-0.3}$ statt $z^{-1}$
\item \textbf{Neue Inflationsmodelle}: Quantenfluktuationen entstehen durch Gitterdynamik
\end{enumerate}

\begin{equation}
\boxed{
\begin{aligned}
\textbf{WG-Rotverschiebung} &= \text{Gravitativ (statisch)} + \text{Dynamisch (neu)} \\
&\Downarrow \\
\textbf{Konsequenz} &:\ \text{Kein Urknall, aber konsistente Alternativkosmologie}
\end{aligned}
}
\end{equation}

\newpage
\section{Dodekaedrisches Raummodell für die Weber-Gravitation}
\subsection{Mathematische Grundlagen}
Inspiriert durch quasikristalline Strukturen schlagen wir ein \textbf{Dodekaeder-Netzwerk} als fundamentale Raumstruktur vor. Sei $\mathcal{D} = (V,E)$ ein ungerichteter Graph mit:

\begin{itemize}
\item Knotenmenge $V$ als diskreten Raumzeit-Ereignissen
\item Kanten $E$ mit Längen $r_{ij}(t)$, die der Weber-Dynamik gehorchen
\end{itemize}

Die Wirkung des Systems lautet:
\begin{equation}
\mathcal{S}_{\mathcal{D}} = \sum_{(i,j)\in E} \int \left[ 
\frac{\mathcal{K}}{2} \left( \dot{r}_{ij}^2 - \frac{\dot{r}_{ij}^4}{4c^4} + \frac{r_{ij}\ddot{r}_{ij}}{2c^2} \right) 
+ \lambda \hbar \, \mathcal{I}_{ij} |\Psi_i - \Psi_j|^2 
\right] dt
\end{equation}

\subsection{Emergenz der Metrik}
Im Kontinuumslimes definieren wir die effektive Metrik:
\begin{equation}
g_{\mu\nu}(x) = \lim_{\epsilon \to 0} \frac{1}{N_\epsilon} \sum_{i,j \in B_\epsilon(x)} \frac{\Delta x_{ij}^\mu \Delta x_{ij}^\nu}{r_{ij}^2}
\end{equation}
wobei $B_\epsilon(x)$ eine $\epsilon$-Kugel um $x$ und $N_\epsilon$ die Normierung ist.

\subsection{Weber-Kraft aus Gitterdynamik}
Störungstheorie liefert die effektive Kraft:
\begin{equation}
F_{ij} = -\frac{\partial \mathcal{V}_{\text{eff}}}{\partial r_{ij}} = -\frac{G m_i m_j}{r_{ij}^2} \left(1 - \frac{\dot{r}_{ij}^2}{c^2} + \frac{r_{ij}\ddot{r}_{ij}}{2c^2}\right)
\end{equation}

\subsection{Quantenmechanische Erweiterung}
Die Führungswelle $\Psi$ entsteht als Eigenlösung des Netzwerkoperators:
\begin{equation}
\hat{H} \Psi = \left[ -\frac{\hbar^2}{2m} \sum_{\langle ijk \rangle} \nabla_{ijk}^2 + \mathcal{V}_{\text{WG}} \right] \Psi = E \Psi
\end{equation}
wobei $\nabla_{ijk}^2$ der diskrete Laplace-Operator auf Dodekaeder-Flächen ist.

\section{Vollständige Quantengravitation durch emergente Phänomene}
\subsection{Stufen der Emergenz}

\begin{enumerate}
    \item \textbf{Mikroskopische Ebene:} Das Dodekaeder-Netzwerk mit Weber-Dynamik
    \begin{equation}
        \mathcal{L}_{\text{mikro}} = \sum_{k=1}^{12} \left( \frac{1}{2}\dot{\phi}_k^2 - V(\phi_k) \right) + \mathcal{K} \sum_{\langle kl \rangle} \ddot{r}_{kl}^2
    \end{equation}
    
    \item \textbf{Mesoskopische Ebene:} Emergenz der effektiven Metrik
    \begin{equation}
        g_{\mu\nu}(x) = \langle \psi|\hat{g}_{\mu\nu}(x)|\psi \rangle, \quad \hat{g}_{\mu\nu} \sim \sum_p e^{ipx}\hat{a}_p + h.c.
    \end{equation}
    
    \item \textbf{Makroskopische Ebene:} Reproduktion der Einstein-Gleichungen
    \begin{equation}
        \delta \mathcal{S}_{\text{eff}} = 0 \Rightarrow G_{\mu\nu} + \Lambda g_{\mu\nu} = \frac{8\pi G}{c^4} T_{\mu\nu}
    \end{equation}
\end{enumerate}

\subsection{Emergenz der Materiefelder}
Die Teilchenphysik entsteht als Anregungsmoden:

\begin{equation}
    \mathcal{L}_{\text{Materie}} = \sum_{n=1}^{3} \bar{\psi}_n (i\gamma^\mu D_\mu - m_n)\psi_n + \frac{1}{4} F_{\mu\nu}F^{\mu\nu}
\end{equation}

wobei die Fermionen $\psi_n$ als topologische Defekte im Netzwerk auftreten:

\begin{equation}
    \psi(x) \sim \prod_{k \in \text{Umweg}} e^{i\int A_k dx^k} \phi_k
\end{equation}

\subsection{Quantengravitative Korrekturen}
Die vollständige Wirkung inkl. Quantenfluktuationen:

\begin{equation}
    \mathcal{S}_{\text{QG}} = \int \mathcal{D}g\mathcal{D}\phi \exp\left[ i\left( \mathcal{S}_{\text{EH}} + \mathcal{S}_{\text{WG}} + \hbar \Delta \mathcal{S}_{\text{top}} \right) \right]
\end{equation}

mit den charakteristischen Skalen:
\begin{itemize}
    \item Netzwerkkonstante: $\ell_p \sim 10^{-35}$ m
    \item Emergenzskala: $\ell_e \sim 10^{-18}$ m
    \item Beobachtungsskala: $\ell_o \sim 1$ m
\end{itemize}

\subsection{Experimentelle Signaturen}
\begin{table}[h]
    \centering
    \begin{tabular}{lll}
        \toprule
        Effekt & Skala & Nachweismethode \\
        \midrule
        Raumzeit-\enquote{Körnigkeit} & $10^{-18}$ m & Gamma-Astronomie \\
        Modified Dispersion & $10^{-3}$ eV & Atominterferometrie \\
        Quantisierte Flächen & $10^{-10}$ m & Neutronenstreuung \\
        \bottomrule
    \end{tabular}
    \caption{Vorhersagen des Modells}
\end{table}
\section{Renormierungsgruppenanalyse des Dodekaeder-Modells}
\subsection{Skalenabhängige Kopplungskonstanten}

Die Renormierungsgruppenflüsse der fundamentalen Parameter werden durch folgende Differentialgleichungen beschrieben:

\begin{equation}
\frac{dg_i}{d\ln\mu} = \beta_i(g_j), \quad i,j \in \{\mathcal{K},\lambda,G,\hbar\}
\end{equation}

Mit den beta-Funktionen:

\begin{equation}
\begin{aligned}
\beta_\mathcal{K} &= \frac{5-\sqrt{5}}{4\pi^2}\mathcal{K}^2 - \frac{\mathcal{K}^3}{12c^4} \\
\beta_\lambda &= 3\lambda^2 - \frac{\lambda}{2\pi}\left(\frac{\mathcal{K}}{c^2}-\frac{4G\mu^2}{\hbar c^3}\right) \\
\beta_G &= \frac{G^2\mu^2}{\hbar c^5}(2-\omega) \\
\beta_\hbar &= -\frac{\hbar}{2\pi}\left(\frac{\lambda^2}{4} + \frac{G\mu^2}{c^5}\right)
\end{aligned}
\end{equation}

\subsection{Fixpunkte und Phasenübergänge}

Die kritischen Punkte ergeben sich aus $\beta_i(g_j^*) = 0$:

\begin{table}[h]
\centering
\begin{tabular}{lccl}
\toprule
Fixpunkt & $\mathcal{K}^*$ & $\lambda^*$ & Stabilität \\
\midrule
Gaußscher FP & 0 & 0 & UV-stabil \\
Weber-FP & $\frac{5\pi c^2}{3}$ & 0 & IR-attraktiv \\
Quanten-FP & $\frac{2G\mu^2}{\hbar c}$ & $\frac{1}{\sqrt{3}}$ & kritisch \\
\bottomrule
\end{tabular}
\caption{Fixpunkte der Renormierungsgruppe}
\end{table}

\subsection{Anomalie-Dimensionen}

Die Skalendimensionen der Felder am Quanten-Fixpunkt:

\begin{equation}
\gamma_\phi = \frac{5-\sqrt{5}}{16\pi^2}\mathcal{K}^*, \quad \gamma_\Psi = \frac{\lambda^*}{12\pi^2}
\end{equation}

\subsection{Phasendiagramm}

\begin{equation}
\mathcal{F} = -T \ln Z = \int d^4x \left[ \frac{1}{2}(\nabla \phi)^2 + \frac{m^2}{2}\phi^2 + \frac{\lambda}{4!}\phi^4 \right]
\end{equation}

Mit den Übergangstemperaturen:

\begin{equation}
T_c = \begin{cases}
\frac{\hbar c}{k_B}\sqrt{\frac{3\mathcal{K}}{2G}} & \text{(Geometrie-Phase)} \\
\frac{\hbar}{k_B}\left(\frac{\lambda^3c^7}{G^2}\right)^{1/5} & \text{(Materie-Phase)}
\end{cases}
\end{equation}

\subsection{Asymptotische Freiheit}

Die effektive Kopplung zeigt das charakteristische Verhalten:

\begin{equation}
\alpha_{\text{WG}}(\mu) = \frac{\mathcal{K}(\mu)}{4\pi c^2} \approx \frac{1}{11\ln(\mu^2/\Lambda^2_{\text{QG}})}
\end{equation}

wobei $\Lambda_{\text{QG}} \sim 10^{19}$ GeV die Quantengravitationsskala ist.
\section{Emergenz der Allgemeinen Relativitätstheorie}
\label{sec:emergenz_art}

\subsection{Kontinuumslimes der Netzwerkdynamik}
Aus der mikroskopischen Wirkung des Dodekaeder-Netzwerks
\begin{equation}
\mathcal{S}_{\text{net}} = \sum_{\langle ij \rangle} \int \left[ \frac{\mathcal{K}}{2} \left( \dot{r}_{ij}^2 - \frac{\dot{r}_{ij}^4}{4c^4} + \frac{r_{ij}\ddot{r}_{ij}}{2c^2} \right) + \lambda \hbar \mathcal{I}_{ij} |\Psi_i - \Psi_j|^2 \right] dt
\end{equation}
emergiert im makroskopischen Limes ($N \to \infty$, $\ell_p \to 0$) die effektive Raumzeit-Metrik
\begin{equation}
g_{\mu\nu}(x) = \lim_{\epsilon \to 0} \frac{1}{N_\epsilon} \sum_{i,j \in B_\epsilon(x)} \frac{\Delta x_{ij}^\mu \Delta x_{ij}^\nu}{r_{ij}^2}
\end{equation}

\subsection{Herleitung der Einstein-Hilbert-Wirkung}
Die Entwicklung der Netzwerkfluktuationen $\delta r_{ij} = r_{ij} - \bar{r}$ führt auf
\begin{equation}
\mathcal{S}_{\text{eff}} = \int d^4x \sqrt{-g} \left( \frac{R}{16\pi G} - \Lambda + \mathcal{L}_{\text{m}} \right) + \mathcal{O}(\ell_p^2)
\end{equation}
mit:
\begin{itemize}
\item $G = \frac{3\sqrt{5}}{8} \frac{c^4\ell_p^2}{\mathcal{K}\hbar}$ (emergente Gravitationskonstante)
\item $\Lambda = \frac{5(1-\phi)}{2\ell_p^2}$ ($\phi$ = Goldberg-Coxeter-Parameter)
\item $\mathcal{L}_{\text{m}} = \sum_f \bar{\psi}_f (i\gamma^\mu D_\mu - m_f)\psi_f$ (emergente Materiefelder)
\end{itemize}

\subsection{Einstein-Gleichungen als effektive Dynamik}
Die Variation $\delta\mathcal{S}_{\text{eff}}/\delta g_{\mu\nu} = 0$ liefert
\begin{equation}
G_{\mu\nu} + \Lambda g_{\mu\nu} = 8\pi G T_{\mu\nu}
\end{equation}
wobei der Energie-Impuls-Tensor aus den Materiefluktuationen entsteht:
\begin{equation}
T_{\mu\nu} = \frac{2}{\sqrt{-g}} \frac{\delta (\sqrt{-g}\mathcal{L}_{\text{m}})}{\delta g^{\mu\nu}}
\end{equation}

\subsection{Konsistenzbedingungen}
Für die exakte Emergenz müssen gelten:
\begin{enumerate}
\item \textbf{Skalierungsrelation}:
\begin{equation}
\frac{\langle r_{ij}^2 \rangle}{\ell_p^2} = 5(3-\sqrt{5}) \approx 6.91
\end{equation}

\item \textbf{Lorentz-Symmetrierestoration}:
\begin{equation}
\text{SO}(4,1)_{\text{mikro}} \to \text{SO}(3,1)_{\text{macro}} \quad \text{für} \quad \frac{E}{E_p} < 10^{-5}
\end{equation}

\item \textbf{Störungstheoretische Konvergenz}:
\begin{equation}
\sum_{n=0}^\infty \left( \frac{\ell_p}{L} \right)^n \mathcal{O}_n \to 0 \quad \text{für} \quad L \gg \ell_p
\end{equation}
\end{enumerate}
\section{Autonome Quanten-Weber-Gravitation}
Die Weber-Gravitation (WG) formuliert eine eigenständige Quantentheorie der Gravitation,\\die ohne Rückgriff auf ART oder SRT auskommt. Ihre fundamentalen Prinzipien sind:

\subsection{Raumzeit-Konzept}
\begin{itemize}
    \item \textbf{Diskrete Basis}: Die Raumzeit wird durch ein dynamisches Dodekaeder-Netzwerk $\mathcal{D}=(V,E)$ beschrieben, wobei Knoten $v\in V$ Quantenzustände $|j,m\rangle$ und Kanten $e_{ij}\in E$ Operatoren $\hat{r}_{ij}$ mit Planck-Länge $\ell_p$ tragen.
    
    \item \textbf{Emergente Metrik}: Die klassische Metrik entsteht als Mittelwert:
    \begin{equation}
        g_{\mu\nu}(x) = \lim_{\epsilon\to0} \frac{1}{N_\epsilon} \sum_{\substack{i,j\in B_\epsilon(x)}} \frac{\Delta x_{ij}^\mu \Delta x_{ij}^\nu}{\langle \hat{r}_{ij}^2 \rangle}
    \end{equation}
\end{itemize}

\subsection{Quanten-Dynamik}
Die Zeitentwicklung folgt aus dem Weber-Hamiltonoperator:
\begin{equation}
    \hat{H}_{\text{WG}} = \sum_{\langle ij \rangle} \left( \frac{\hat{p}_{ij}^2}{2m_{ij}} - \frac{Gm_im_j}{\hat{r}_{ij}} \left[1 - \frac{\langle \hat{\dot{r}}_{ij}^2 \rangle}{c^2} + \beta \frac{\hat{r}_{ij}\hat{\ddot{r}}_{ij}}{c^2}\right] \right) + \hat{Q}
\end{equation}
wobei $\hat{Q} = -\frac{\hbar^2}{2m}\frac{\nabla^2|\Psi|}{|\Psi|}$ das nicht-lokale Quantenpotential ist.

\subsection{Schlüsselunterschiede zu ART/SRT}
\begin{tabular}{ll}
    \textbf{WG} & \textbf{ART/SRT} \\
    \hline
    Fernwirkung via $\hat{r}_{ij}$, $\hat{\ddot{r}}_{ij}$ & Lokale Krümmung $R_{\mu\nu}$ \\
    Frequenzabhängige Lichtablenkung & Metrik-basierte Ablenkung \\
    Statisches Universum mit $z\sim v_r^2$ & Expandierende Raumzeit \\
    Quantisiertes $r_{ij}\in n\ell_p$ & Kontinuierliche Koordinaten \\
\end{tabular}

\subsection{Konsequenzen}
\begin{itemize}
    \item Singularitätsfreiheit durch $\hat{Q}$-Divergenz bei $r_{ij}\to0$
    \item Lorentz-Invarianz nur als Niedrigenergie-Näherung
    \item Nachweisbare Abweichungen bei $\lambda\lesssim\ell_p$ (z.B. $\Delta\phi(\lambda)$)
\end{itemize}

\section{Fehlende Komponenten zur vollständigen Quantengravitation}

Die Quanten-Weber-Gravitation (QWG) benötigt folgende Erweiterungen, um eine vollständige Theorie der Quantengravitation zu werden:

\subsection{Quantenfeldtheorie der Gravitation}
\begin{itemize}
    \item Einführung eines gravitativen Feldoperators:
    \begin{equation}
        \hat{\mathcal{G}}_{\mu\nu}(x) = -\frac{\hbar^2}{\ell_p^2} \frac{\delta}{\delta g^{\mu\nu}(x)} \ln|\Psi[g]|
    \end{equation}
    \item Entwicklung einer Störungstheorie für $\hat{\mathcal{G}}_{\mu\nu}$
\end{itemize}

\subsection{Dynamische Quantisierung der Raumzeit}
\begin{itemize}
    \item Pfadintegral-Formulierung des Netzwerks:
    \begin{equation}
        Z = \int \mathcal{D}[r_{ij}] e^{i\mathcal{S}_{\text{net}}[r_{ij}]/\hbar}, \quad \mathcal{S}_{\text{net}} = \sum_{\langle ij \rangle} \mathcal{K}\left(\dot{r}_{ij}^2 + \beta r_{ij}\ddot{r}_{ij}\right)
    \end{equation}
    \item Spin-Netzwerk-Quantisierung der Zellflächen:
    \begin{equation}
        \hat{A}_f|j\rangle = 8\pi\gamma\ell_p^2\sqrt{j(j+1)}|j\rangle
    \end{equation}
\end{itemize}

\subsection{Kopplung an das Standardmodell}
\begin{itemize}
    \item Konstruktion chiraler Fermionen:
    \begin{equation}
        \hat{\psi}_L(x) = \prod_{\gamma_L} \hat{h}_{ij}, \quad \hat{\psi}_R(x) = \prod_{\gamma_R} \hat{h}_{ij}
    \end{equation}
    \item Generierung von Eichfeldern:
    \begin{equation}
        \hat{A}_\mu(x) = \sum_{\langle ij \rangle} \epsilon_\mu^{ij}(\hat{h}_{ij} - \hat{h}_{ji})
    \end{equation}
\end{itemize}

\subsection{Renormierung und UV-Vollständigkeit}
\begin{itemize}
    \item Erweiterte Renormierungsgruppengleichungen:
    \begin{equation}
        \beta_G = \frac{G^2}{\hbar c^5}(2 - \omega + a_1G^2 + a_2\lambda^2)
    \end{equation}
    \item Suche nach nicht-trivialen UV-Fixpunkten $(G^*,\lambda^*)$
\end{itemize}

\subsection{Kausalitätsstruktur}
\begin{itemize}
    \item Quanten-kausale Weber-Kraft:
    \begin{equation}
        \hat{F}_{ij}(t) = \int_{-\infty}^t K(t-t';\ell_p/c)\hat{F}_{ij}^{\text{inst}}(t')\,dt'
    \end{equation}
    \item Kausalkern mit Planck-Zeit-Cutoff:
    \begin{equation}
        K(\tau) \sim e^{-\tau^2c^2/2\ell_p^2}
    \end{equation}
\end{itemize}

\begin{table}[h]
\centering
\caption{Zusammenfassung der fehlenden Komponenten}
\begin{tabular}{ll}
\toprule
\textbf{Komponente} & \textbf{Erforderliche Erweiterung} \\
\midrule
Quantenfeldtheorie & Feldoperator $\hat{\mathcal{G}}_{\mu\nu}$ \\
Raumzeit-Quantisierung & Pfadintegral über Netzwerkkonfigurationen \\
Standardmodell & Chirale Fermionen und Eichfelder \\
UV-Vollständigkeit & Nicht-trivialer RG-Fixpunkt \\
Kausalität & Quanten-kausaler Propagator \\
\bottomrule
\end{tabular}
\end{table}

\chapter{Ergänzende Informationen}
\label{chapter:information}
\section{Die Rolle des $\beta$-Parameters}

Der $\beta$-Parameter in der Weber-Kraft

\begin{equation}
F = -\frac{GMm}{r^2}\left(1 - \frac{\dot{r}^2}{c^2} + \beta\frac{r\ddot{r}}{c^2}\right)\hat{r}
\end{equation}

bestimmt das Verhältnis von Beschleunigungs- zu Geschwindigkeitstermen und variiert je nach Wechselwirkungstyp:

\subsection{Elektrodynamik (Original-Weber)}
Für elektromagnetische Wechselwirkungen gilt $\beta=2$:
\begin{itemize}
\item Führt zur korrekten Beschreibung beschleunigter Ladungen
\item Reproduziert die magnetische Komponente der Lorentz-Kraft
\item Keine Lichtablenkung ($m=0$ liefert $F=0$)
\end{itemize}

\subsection{Gravitation (Massen)}
Für massive Körper im Gravitationsfeld:
\begin{itemize}
\item $\beta=0.5$ erklärt die Periheldrehung des Merkur
\item Führt zur ART-konformen Lichtablenkung für makroskopische Körper
\item Universelle Formel: $\beta = 1 - \frac{mc^2}{2E}$
\end{itemize}

\subsection{Photonen (Lichtablenkung)}
Für masselose Teilchen ($m=0$, $E=h\nu$):
\begin{itemize}
\item $\beta=1$ erzwingt die Frequenzabhängigkeit
\item Beschleunigungsterm dominiert: $\frac{r\ddot{r}}{c^2} \approx \frac{h^2}{c^2r^4}$
\item Liefert den Zusatzterm $\propto \lambda^{-2}$
\end{itemize}

\begin{table}[h]
\centering
\caption{$\beta$-Werte im Vergleich}
\begin{tabular}{lcc}
\hline
Anwendung & $\beta$ & Physikalische Konsequenz \\
\hline
Elektrodynamik & 2 & Magnetische Wechselwirkungen \\
Gravitation (Massen) & 0.5 & Periheldrehung des Merkur \\
Photonen & 1 & Frequenzabhängige Lichtablenkung \\
\hline
\end{tabular}
\end{table}
\section{Herleitung der kombinierten WG-DBT Bewegungsgleichung}

\subsection*{1. Ausgangspunkt: Weber-Gravitationskraft}
Die klassische Weber-Kraft für zwei Massen $m$ und $M$ lautet:
\begin{equation}
\mathbf{F}_{\text{WG}} = -\frac{GMm}{r^2}\left(1 - \frac{\dot{r}^2}{c^2} + \beta\frac{r\ddot{r}}{c^2}\right)\hat{\mathbf{r}}
\end{equation}

\subsection*{2. Umformung der radiale Beschleunigungsterme}
Wir entwickeln die Terme $\dot{r}^2$ und $r\ddot{r}$ in vektorieller Form:

\begin{align}
\dot{r} &= \frac{d}{dt}\sqrt{\mathbf{r}\cdot\mathbf{r}} = \frac{\mathbf{r}\cdot\mathbf{v}}{r} \\
\dot{r}^2 &= \left(\frac{\mathbf{r}\cdot\mathbf{v}}{r}\right)^2 \\
r\ddot{r} &= \frac{d}{dt}(r\dot{r}) - \dot{r}^2 = \mathbf{v}\cdot\mathbf{v} + \mathbf{r}\cdot\mathbf{a} - \left(\frac{\mathbf{r}\cdot\mathbf{v}}{r}\right)^2
\end{align}

Für kleine Abweichungen von Kreisbahnen vernachlässigen wir den letzten Term und erhalten:
\begin{equation}
r\ddot{r} \approx v^2 + \mathbf{r}\cdot\mathbf{a}
\end{equation}

\subsection*{3. Verallgemeinerte Weber-Kraft in vektorieller Form}
Einsetzen in (1) ergibt:
\begin{equation}
\mathbf{F}_{\text{WG}} = -\frac{GMm}{r^2}\left(1 - \frac{(\mathbf{r}\cdot\mathbf{v})^2}{c^2r^2} + \beta\frac{v^2 + \mathbf{r}\cdot\mathbf{a}}{c^2}\right)\hat{\mathbf{r}}
\end{equation}

\subsection*{4. Lagrange-Formulierung der Weber-Gravitation}
Das effektive Weber-Potential lautet:
\begin{equation}
V_{\text{WG}} = -\frac{GMm}{r}\left(1 - \frac{v^2}{2c^2} + \beta\frac{\mathbf{r}\cdot\mathbf{a}}{2c^2}\right)
\end{equation}

Die Lagrange-Funktion wird:
\begin{equation}
\mathscr{L}_{\text{WG}} = T - V_{\text{WG}} = \frac{1}{2}mv^2 + \frac{GMm}{r}\left(1 - \frac{v^2}{2c^2} + \beta\frac{\mathbf{r}\cdot\mathbf{a}}{2c^2}\right)
\end{equation}

\subsection*{5. Euler-Lagrange-Gleichungen}
Anwendung der Euler-Lagrange-Gleichung:
\begin{equation}
\frac{d}{dt}\left(\frac{\partial\mathscr{L}}{\partial\mathbf{v}}\right) - \frac{\partial\mathscr{L}}{\partial\mathbf{r}} = 0
\end{equation}

Berechnung der Terme:
\begin{align}
\frac{\partial\mathscr{L}}{\partial\mathbf{v}} &= m\mathbf{v} - \frac{GMm}{c^2r}\mathbf{v} + \beta\frac{GMm}{2c^2}\frac{\mathbf{r}}{r} \\
\frac{d}{dt}\left(\frac{\partial\mathscr{L}}{\partial\mathbf{v}}\right) &= m\mathbf{a} - \frac{GMm}{c^2}\left(\frac{\mathbf{a}}{r} - \frac{\dot{r}\mathbf{v}}{r^2}\right) + \beta\frac{GMm}{2c^2}\left(\frac{\mathbf{v}}{r} - \frac{\dot{r}\mathbf{r}}{r^2}\right) \\
\frac{\partial\mathscr{L}}{\partial\mathbf{r}} &= -\frac{GMm}{r^2}\hat{\mathbf{r}} + \frac{GMm}{2c^2}\left(\frac{v^2}{r^2}\hat{\mathbf{r}} - \beta\frac{\mathbf{a}}{r}\right)
\end{align}

\subsection*{6. De-Broglie-Bohm'sches Quantenpotential}
Das Quantenpotential der DBT ist:
\begin{equation}
Q = -\frac{\hbar^2}{2m}\frac{\nabla^2|\Psi|}{|\Psi|}
\end{equation}

Die quantenmechanische Kraft ergibt sich aus:
\begin{equation}
\mathbf{F}_{\text{Q}} = -\nabla Q
\end{equation}

\subsection*{7. Kombinierte Bewegungsgleichung}
Addition der Weber- und Quantenkräfte führt zu:
\begin{equation}
m\frac{d}{dt}\left[\left(1 - \frac{GM}{c^2r} + \beta\frac{GM}{2c^2}\frac{\mathbf{r}\cdot\mathbf{v}}{r^2}\right)\mathbf{v}\right] = -\frac{GMm}{r^2}\hat{\mathbf{r}} - \nabla Q
\end{equation}

Definition des Weber-Lorentz-Faktors:
\begin{equation}
\gamma_{\text{WG}} \equiv \left(1 - \frac{v^2}{c^2} + \beta\frac{\mathbf{v}\cdot\mathbf{a}}{c^2}\right)^{-1/2} \approx 1 + \frac{v^2}{2c^2} - \beta\frac{\mathbf{v}\cdot\mathbf{a}}{2c^2}
\end{equation}

\subsection*{8. Finale Bewegungsgleichung (\ref{eq:wg_dbt_srt})}
Nach Vernachlässigung höherer Ordnungen erhalten wir:
\begin{equation}
m\frac{d}{dt}(\gamma_{\text{WG}}\mathbf{v}) = -\nabla Q
\end{equation}


\begin{thebibliography}{9}

\bibitem{einstein1915} 
Einstein, A. (1915). 
\textit{Die Feldgleichungen der Gravitation}. 
Sitzungsberichte der Preußischen Akademie der Wissenschaften, 
S. 844–847.

\bibitem{shapiro1964} 
Shapiro, I. I. (1964). 
\textit{Fourth Test of General Relativity}. 
Physical Review Letters, 13(26), 789–791.

\bibitem{rubin1970} 
Rubin, V. C., \& Ford, W. K. (1970). 
\textit{Rotation of the Andromeda Nebula from a Spectroscopic Survey of Emission Regions}. 
Astrophysical Journal, 159, 379–403.

\bibitem{weber1846} 
Weber, W. (1846). 
\textit{Elektrodynamische Maassbestimmungen}. 
Leipzig: Weidmannsche Buchhandlung.

\bibitem{bohm1952} 
Bohm, D. (1952). 
\textit{A Suggested Interpretation of the Quantum Theory in Terms of "Hidden" Variables}. 
Physical Review, 85(2), 166–193.

\bibitem{tisserand1894}
Tisserand, F. (1894). 
\textit{Traité de Mécanique Céleste, Tome IV}. 
Gauthier-Villars, Paris. 
(Kapitel 28: "Lois électrodynamiques de Weber appliquées à la gravitation")

\end{thebibliography}

\end{document}
