\documentclass[11pt, a4paper, twoside, openright]{book}
\usepackage[ngerman]{babel}
\usepackage[T1]{fontenc}
\usepackage[utf8]{inputenc}
\usepackage{lmodern}
\usepackage{microtype}
\usepackage{csquotes}
\usepackage{verbatim}  % Im Kopf des Dokuments einfügen
\usepackage{geometry}
\usepackage{fancyhdr}
\usepackage{amsmath, amssymb, amsthm}  % Mathe
\usepackage{mathtools}                  % \coloneqq, \xrightarrow
\usepackage{bm}                        % Fette Symbole (\bm{B} für Magnetfeld)
\usepackage{siunitx}                   % \SI{1.23}{\meter\per\second}
\usepackage{graphicx}                  % \includegraphics
\usepackage{subcaption}                % Unterabbildungen
\usepackage{booktabs}                  % Professionelle Tabellen
\usepackage[
    backend=biber,
    style=phys,         % APS-Zitierstil (für Physik)
    sorting=nyt,        % Sortierung: Name, Jahr, Titel
]{biblatex}
\usepackage[acronym, toc]{glossaries}
\usepackage{hyperref}
\geometry{
    a4paper,
    top=25mm,
    inner=30mm,    % Bundsteg (größerer Rand für Buchbindung)
    outer=25mm,
    bottom=30mm,
    headheight=15pt,
}

\pagestyle{fancy}
\fancyhf{}
\fancyhead[LE,RO]{\thepage}
\fancyhead[RE]{\leftmark}    % Kapitelname (gerade Seiten)
\fancyhead[LO]{\rightmark}   % Abschnittname (ungerade Seiten)
\renewcommand{\headrulewidth}{0.4pt}

\theoremstyle{definition}
\newtheorem{definition}{Definition}[chapter]
\newtheorem{law}{Physikalisches Gesetz}[chapter]
\theoremstyle{plain}
\newtheorem{theorem}{Theorem}[chapter]
\newtheorem{lemma}[theorem]{Lemma}
\theoremstyle{remark}
\newtheorem{remark}{Bemerkung}[chapter]

\hypersetup{
    colorlinks=true,
    linkcolor=blue,
    citecolor=black,
    urlcolor=black,
    pdftitle={Emergenz der Kosmologie: Die WDBT als Ur-Theorie},
    pdfauthor={Dipl.-Ing. (FH) Michael Czybor},
}

\addbibresource{literatur.bib}  % Ihre .bib-Datei
\makeglossaries

\setlength{\headheight}{26.76852pt}
\definecolor{quantenblau}{RGB}{0, 100, 200}
\definecolor{weberrot}{RGB}{180, 20, 60}
\definecolor{hintergrund}{RGB}{20, 20, 40}
\usetikzlibrary{shapes, calc, 3d}
\pgfplotsset{compat=1.18} % Aktuelle Version verwenden

\newacronym{qm}{QM}{Quantum Mechanics}
\newacronym{art}{ART}{General Theory of Relativity}
\newacronym{srt}{SRT}{Special Theory of Relativity}
\newacronym{cmb}{CMB}{Cosmic Microwave Background}
\newacronym{qed}{QED}{Quantum Electrodynamics}
\newacronym{epr}{EPR Paradox}{Einstein-Podolsky-Rosen Paradox}
\newacronym{wg}{WG}{Weber Gravitation}
\newacronym{dbt}{DBT}{De Broglie-Bohm Theory}
\newacronym{wdbt}{WDBT}{Weber-De Broglie-Bohm Theory}
\newacronym{mt}{MT}{Maxwell Theory}
\newacronym{mhd}{MHD}{Magnetohydrodynamics}
\newacronym{wed}{WED}{Weber Electrodynamics}
\newacronym{eu}{EU}{Electric Universe}

\newglossaryentry{gls:quantenmechanik}
{
    name={Quantum Mechanics},
    description={Theory of matter and radiation at the atomic and subatomic level}
}
\newglossaryentry{gls:hamiltonian}
{
    name={\ensuremath{\mathcal{H}}},
    description={Hamiltonian operator, describes the total energy of a system},
    sort={hamiltonian}
}

\begin{document}

\frontmatter
\title{Weber-Gravitation\\und\\De-Broglie-Bohm-Theorie}
\author{Dipl.-Ing. (FH) Michael Czybor}
\date{\today}
\maketitle

\tableofcontents

\mainmatter
\chapter{Einleitung}
\section{Motivation}
Viele Schüler und Studierende erleben den Physikunterricht als frustrierend und unverständlich. Besonders die moderne Physik – mit der Allgemeinen Relativitätstheorie (ART)
und der Speziellen Relativitätstheorie (SRT) – wirkt oft unphysikalisch und voller logischer Widersprüche. Energie scheint unter bestimmten Bedingungen unendlich zu werden,
Überlichtgeschwindigkeit wird in manchen Fällen postuliert, obwohl sie eigentlich unmöglich sein soll, und Begriffe wie \enquote{dunkle Energie} oder \enquote{dunkle Materie} wirken wie
Platzhalter für unser Unverständnis.

Ein grundlegendes Problem liegt in den Widersprüchen zwischen ART und SRT. Die SRT baut auf Inertialsystemen auf, also Bezugssystemen, die sich gleichförmig und unbeschleunigt
bewegen. Doch laut ART gibt es keine perfekten Inertialsysteme, da jede Masse die Raumzeit krümmt und damit Beschleunigungen erzeugt. Schon allein dieser Widerspruch wirft
Fragen auf: Wenn Inertialsysteme streng genommen punktförmig sein müssten, um frei von jeder Krümmung zu sein, bräuchte man unendlich viele davon – und damit auch unendlich
viele verschiedene Lichtgeschwindigkeiten, da diese vom Bezugssystem abhängt.

Hinzu kommt, dass viele Konzepte der modernen Physik unserer Intuition widersprechen. Die Quantenmechanik verlangt, dass Teilchen gleichzeitig Wellen sind und erst durch
Beobachtung einen definierten Zustand annehmen. Die ART beschreibt eine gekrümmte Raumzeit, die sich kaum jemand wirklich vorstellen kann, und die SRT führt zu scheinbar
paradoxen Zeitdehnungen und Längenkontraktionen. Selbst der Urknall als Anfangspunkt des Universums wirft Fragen auf: Wie kann etwas aus dem Nichts entstehen? Warum gibt es
überhaupt eine Singularität, wenn doch unsere physikalischen Gesetze dort versagen?

All diese Punkte zeigen, dass die moderne Physik noch lange nicht abgeschlossen ist. Statt blind akzeptierte Theorien als absolute Wahrheit zu betrachten, sollten wir die
Widersprüche hinterfragen und nach konsistenteren Erklärungen suchen.


\backmatter
\printbibliography[title=Literaturverzeichnis]
\printglossary[title=Glossar]
\printglossary[type=acronym, title=Abkürzungen]

\end{document}












\documentclass[10pt,twoside,openright]{book} % Standard Buchformat
\usepackage[a4paper,left=2.5cm,right=2cm,top=2cm,bottom=2.5cm]{geometry}
\usepackage[utf8]{inputenc}
\usepackage[ngerman]{babel}
\usepackage{amsfonts, amsmath, amssymb}
\usepackage{array}
\usepackage{ragged2e}
\usepackage{tabularx}
\usepackage{enumitem}
\usepackage{booktabs}
\usepackage{bm}
\usepackage{csquotes}
\usepackage{siunitx}
\usepackage{parskip}
\usepackage{listings}
\usepackage{xcolor}
\usepackage[labelfont=bf]{caption}
\usepackage{tcolorbox}
\usepackage{mathrsfs}
\usepackage{microtype}
%\usepackage{showlabels}
%\usepackage{refcheck}

\newtheorem{theorem}{Theorem} % Definiert das 'theorem'-Environment
\newtheorem{lemma}{Lemma}     % Falls Sie auch Lemmas verwenden möchten
%\showlabels{cite,label}
\renewcommand{\arraystretch}{1.1}
\numberwithin{equation}{section}
\definecolor{gray}{rgb}{0.5,0.5,0.5}

\begin{document}

\date{\today}
\maketitle

\section*{Zusammenfassung}
Diese Arbeit untersucht die Synthese der Weber-Gravitation (WG) mit der De-Broglie-Bohm-Theorie (DBT) als alternative Herangehensweise zu den etablierten Theorien der
Allgemeinen Relativitätstheorie (ART) und Quantenmechanik. Die Weber-Gravitation, ursprünglich in der Elektrodynamik formuliert, wird auf gravitative Wechselwirkungen
übertragen und durch die Einbeziehung des Quantenpotentials der DBT erweitert. 

Zentrales Element ist die verallgemeinerte Weber-Kraft, die neben dem klassischen Newtonschen Term zusätzliche Geschwindigkeits- und Beschleunigungsabhängigkeiten enthält.
Diese wird durch das nicht-lokale Quantenpotential der DBT ergänzt, wodurch eine deterministische Beschreibung quantenmechanischer Phänomene ermöglicht wird. Die kombinierte Theorie
zeigt bemerkenswerte Parallelen zwischen instantanen Korrelationen in Wellenphänomenen und nicht-lokalen Wechselwirkungen in der Quantenmechanik.

Anwendungen der WG-DBT-Synthese werden für verschiedene astrophysikalische Phänomene untersucht: 
\begin{itemize}
    \item Die Periheldrehung des Merkurs ergibt sich natürlicherweise aus dem Weber-Formalismus mit $\beta=0.5$
    \item Galaktische Rotationskurven werden ohne dunkle Materie durch den DBT-Beitrag erklärt
    \item Lichtablenkung und Shapiro-Effekt zeigen charakteristische Abweichungen von der ART
\end{itemize}

Die Arbeit argumentiert für einen erweiterten Kausalitätsbegriff, der instantane Wechselwirkungen als systeminterne Rückkopplungen interpretiert. Mathematisch manifestiert
sich dies in der kovarianten Formulierung der Bewegungsgleichungen, die Jerk-Terme und Quantenpotentiale vereint. Die Ergebnisse legen nahe, dass die WG-DBT-Synthese eine
vielversprechende Grundlage für eine singularitätsfreie Quantengravitation bieten könnte.

\tableofcontents

% Einbindung der einzelnen TeX-Dateien
\part{Grundlagen}
\chapter{Weber-Kraft}
\label{chapter:weber_kraft}
\section{Klassische Weber-Kraft (Elektrodynamik)}
\begin{equation}
    \boxed
    {
        \bm{F}_{\text{Weber}}^{\text{EM}} = \frac{Qq}{4\pi\epsilon_0 r^2}\left(1 - \frac{\dot{r}^2}{c^2} + \frac{2r\ddot{r}}{c^2}\right)\bm{\hat{r}}
    }
\end{equation}

\subsection*{Symbolbeschreibung}
\begin{itemize}[leftmargin=*,noitemsep]
    \item $\bm{F}_{\text{Weber}}^{\text{EM}}$: Weber-Kraft zwischen Ladungen
    \item $Q, q$: Elektrische Ladungen
    \item $\epsilon_0$: Elektrische Feldkonstante
    \item $r$: Ladungsabstand
    \item $\dot{r} = \frac{dr}{dt}$: Relative Radialgeschwindigkeit
    \item $\ddot{r} = \frac{d^2r}{dt^2}$: Relative Radialbeschleunigung
    \item $c$: Lichtgeschwindigkeit
    \item $\bm{\hat{r}}$: Radialer Einheitsvektor
\end{itemize}

\subsection*{Beziehung zur Coulomb-Kraft}
\begin{itemize}[leftmargin=*,noitemsep]
    \item Erster Term entspricht Coulomb-Kraft: $\frac{Qq}{4\pi\epsilon_0 r^2}$
    \item Zusatzterme $\left(-\frac{\dot{r}^2}{c^2} + \frac{2r\ddot{r}}{c^2}\right)$ beschreiben Bewegungsabhängige Korrekturen
    \item Reduktion auf Coulomb-Kraft im statischen Fall ($\dot{r} = \ddot{r} = 0$)
\end{itemize}

\subsection*{Vergleich mit Maxwell-Theorie}
\begin{itemize}[leftmargin=*,noitemsep]
    \item Alternative Beschreibung elektromagnetischer Phänomene \cite{weber1846}
    \item Fernwirkungsansatz (direkte Ladungswechselwirkung)
    \item Implizite Retardierung durch Geschwindigkeits-/Beschleunigungsterme
    \item Keine Vorhersage von EM-Wellen im Vakuum
\end{itemize}

\subsection{Ansatz zur Weber-Gravitation (WG)}
\begin{itemize}[leftmargin=*,noitemsep]
    \item Kein vordefiniertes Raummodell benötigt
    \item Natürliche Diskretisierung durch Punktteilchen
    \item Gravitative Erweiterung möglich:
\end{itemize}

\begin{equation}
\bm{F}_{\text{Weber}}^{G} = G\frac{mM}{r^2}\left(1 - \frac{\dot{r}^2}{c^2} + \frac{2r\ddot{r}}{c^2}\right)\bm{\hat{r}}
\end{equation}

\subsection*{Zusammenfassung}
\begin{itemize}[leftmargin=*,noitemsep]
    \item Umgeht Quantisierungsprobleme der ART
    \item Ermöglicht diskrete Raumzeitmodelle
    \item Potentieller Brückenansatz zur Quantengravitation
\end{itemize}

\section{Weber-Kraft und Gravitation}
\subsection*{Tisserands Ansatz}
Die Übertragung der elektrodynamischen Weber-Kraft \cite{tisserand1894} auf die Gravitation scheiterte an der Erklärung der Periheldrehung des Merkurs.

\subsection*{Hinweis}
Die korrekte gravitative Formulierung wird separat vorgestellt und erfordert eine Modifikation der Original-Weberschen Formel.

\section{Weber-Gravitation als Alternative zur ART}
Die allgemeine Relativitätstheorie (ART) gilt als der Goldstandard der modernen Astrophysik, allerdings werden bestimmte Aspekte dieser Theorie
nicht objektiv betrachtet. Die ART überzeugt durch die Fähigkeit die Merkur-Periheldrehung vorhersagen zu können, aber auch durch die Vorhersage
der Gravitationswellen. Das sind große Leistungen dieser Gravitationstheorie.

Auf der anderen Seite liefert sie unphysikalische Ergebnisse für schwarze Löcher und für galaktische Skalen. Schwarze Löcher werden als Singularitäten
dargestellt, wobei davon ausgegangen werden muss, dass die gravitativen Verhältnisse in der Nähe dieser Singularitäten ebenfalls ungenau sein müssen. Die
Rotationskurven von Galaxien werden nicht korrekt Vorhergesagt, weswegen die ART \enquote{dunkle Materie} benötigt.

\subsection{Grundgleichungen der Weber-Gravitation}
\subsection*{Weber-Gravitations Gleichung}
\begin{equation}
\label{eq:weber_g}
\boxed
{
    \mathbf{F} = -\frac{GMm}{r^2}\left(1 - \frac{\dot{r}^2}{c^2} + \frac{r\ddot{r}}{2c^2}\right)\mathbf{\hat{r}}
}
\end{equation}

\subsection*{Spezifischer Drehimpuls}
Der Drehimpuls pro Masseneinheit $h$ ist definiert als:
\begin{equation}
\label{eq:spezifischer_drehimpuls_h}
\boxed
{
    h = r^2\dot{\phi} = \sqrt{GMa(1-e^2)}
}
\end{equation}
wobei $a$ die große Halbachse und $e$ die Exzentrizität der Bahn ist.

\subsection{Bewegungsgleichung in Polarkoordinaten}
\[
\mathbf{a} = \left(\ddot{r} - r\dot{\phi}^2\right)\mathbf{\hat{r}} + \left(r\ddot{\phi} + 2\dot{r}\dot{\phi}\right)\mathbf{\hat{\phi}} = -\frac{GM}{r^2}\left(1 - \frac{\dot{r}^2}{c^2} + \frac{r\ddot{r}}{2c^2}\right)\mathbf{\hat{r}}
\]

\subsection*{Variablenbeschreibung}
\begin{itemize}[leftmargin=*,noitemsep]
    \item $\mathbf{F}$: Gravitationskraftvektor (Weber-Kraft) [N]
    \item $\mathbf{a}$: Beschleunigungsvektor [m/s²]
    \item $G$: Gravitationskonstante [m³/kg/s²]
    \item $M$, $m$: Massen der wechselwirkenden Körper [kg]
    \item $r$: Abstand zwischen den Massenschwerpunkten [m]
    \item $\dot{r} = \frac{dr}{dt}$: Radiale Relativgeschwindigkeit [m/s]
    \item $\ddot{r} = \frac{d^2r}{dt^2}$: Radiale Relativbeschleunigung [m/s²]
    \item $c$: Lichtgeschwindigkeit [m/s]
    \item $\phi$: Azimutwinkel [rad]
    \item $\dot{\phi} = \frac{d\phi}{dt}$: Winkelgeschwindigkeit [rad/s]
    \item $\ddot{\phi} = \frac{d^2\phi}{dt^2}$: Winkelbeschleunigung [rad/s²]
    \item $h$: Spezifischer Drehimpuls [m²/s]
    \item $\mathbf{\hat{r}}$: Radialer Einheitsvektor (zeigt von $M$ zu $m$)
    \item $\mathbf{\hat{\phi}}$: Azimutaler Einheitsvektor (senkrecht zu $\mathbf{\hat{r}}$)
\end{itemize}

\subsection*{Physikalische Interpretation}
\begin{itemize}[leftmargin=*,noitemsep]
    \item Der Term $-\frac{GMm}{r^2}$ entspricht der klassischen Newton'schen Gravitation
    \item $\frac{\dot{r}^2}{c^2}$: Relativistische Korrektur für radiale Bewegung
    \item $\frac{r\ddot{r}}{2c^2}$: Korrektur für radiale Beschleunigung
    \item $r\dot{\phi}^2$: Zentripetalbeschleunigung
    \item $2\dot{r}\dot{\phi}$: Coriolis-Term
    \item $h$: Erhaltungsgröße für Planetenbahnen
\end{itemize}

\chapter{Instantane Energieverteilung und Kausalität}
\section{Fundamentale Charakteristika aller Wellen}
\label{sec:wellen}
Wellen besitzen \enquote{instantane} Eigenschaften, welche ebenfalls von Fernwirkungstheorien unterstellt werden.
Hier zeigt sich auch ein Zusammenhang zur De-Broglie-Bohm-Theorie (DBT).

Jede Welle besitzt zwei komplementäre Eigenschaftsebenen:

\subsection*{1. Lokale Eigenschaften (beobachtbar)}
\begin{itemize}
    \item \textbf{Störungsausbreitung} mit mediumabhängiger Phasengeschwindigkeit:
    \[
    v_p = \frac{\omega}{k} = f(\text{Medium})
    \]
    Beispiele:
    \begin{itemize}
        \item Elektromagnetische Wellen: $v_p = 1/\sqrt{\mu\epsilon}$
        \item Schallwellen: $v_p = \sqrt{K/\rho}$
        \item Wasserwellen: $v_p = \sqrt{g/k} \tanh(kh)$
    \end{itemize}
    
    \item \textbf{Sichtbare Dynamik} durch Feldgröße $\psi(x,t)$:
    \[
    \psi(x,t) = A e^{i(kx-\omega t)} \quad \text{(harmonische Näherung)}
    \]
\end{itemize}

\subsection*{2. Nicht-lokale Eigenschaften (instantane Korrelation)}
\begin{itemize}
    \item \textbf{Energieerhaltung} durch phasenkritische Kopplung:
    \[
    \partial_t \mathcal{E} + \nabla \cdot \vec{S} = 0 \quad \text{(Kontinuitätsgleichung)}
    \]
    mit $\mathcal{E} = \mathcal{E}_\text{kin} + \mathcal{E}_\text{pot}$ und $\vec{S}$ als Energiestromdichte.
    
    \item \textbf{Universalmechanismus}:
    \begin{itemize}
        \item Maximales $\mathcal{E}_\text{pot}$ bei $\psi = \pm A$ $\leftrightarrow$ Maximales $\mathcal{E}_\text{kin}$ bei $\psi = 0$
        \item Phasenversatz $\Delta\phi = \pi/2$ zwischen $\psi$ und $\partial_t\psi$
    \end{itemize}
\end{itemize}

\section*{Medienübergreifende Prinzipien}
\begin{table}[ht]
    \centering
    \begin{tabular}{|l|c|c|}
    \hline
    \textbf{Wellentyp} & \textbf{Lokale Größe $\psi$} & \textbf{Nicht-lokaler Erhalt} \\
    \hline
    Mechanisch (Wasser) & Oberflächenauslenkung $\eta$ & $E_\text{kin} + E_\text{pot} = \text{const}$ \\
    \hline
    Akustisch & Druck $p$ & $\frac{p^2}{\rho c^2} + \rho v^2 = \text{const}$ \\
    \hline
    Elektromagnetisch & Felder $\vec{E},\vec{B}$ & $\frac{\epsilon_0 E^2}{2} + \frac{B^2}{2\mu_0} = \text{const}$ \\
    \hline
    Quantenmechanisch & Wellenfunktion $\Psi$ & $|\Psi|^2 = \text{Wahrscheinlichkeit}$ \\
    \hline
    \end{tabular}
\end{table}

\section*{Mathematische Universalstruktur}
\begin{itemize}
    \item \textbf{Dispersionsrelation}: $\omega = \omega(k)$ verknüpft lokale und nicht-lokale Ebene
    \item \textbf{Wellengleichung}: 
    \[
    \partial_t^2 \psi = v_p^2 \nabla^2 \psi + \text{Nichtlinearitäten}
    \]
    \item \textbf{Energietransport}:
    \[
    \vec{S} = 
    \begin{cases}
    \frac{1}{2}\rho g A^2 v_g & \text{(Wasser)} \\
    \vec{E} \times \vec{B}/\mu_0 & \text{(EM)} \\
    p \vec{v} & \text{(Schall)}
    \end{cases}
    \]
\end{itemize}

\section*{Zusammenfassung}
\begin{itemize}
    \item Alle Wellen zeigen \textit{duales Verhalten}: 
    \begin{itemize}
        \item Lokale Propagierung mit $v_p < \infty$
        \item Globale instantane Energie-Neutralisation
    \end{itemize}
    \item Die nicht-lokale Korrelation ist \textit{kein} kausaler Prozess, sondern strukturelle Konsequenz der Wellengleichung
    \item Energieerhaltung erfolgt instantan und nicht-lokal durch \textit{phasenstarre Kopplung} im gesamten System
\end{itemize}

\section{Zusammenhang zur De-Broglie-Bohm-Theorie}
\label{sec:dbt}
Die Weber-Gravitation (WG) und die De-Broglie-Bohm-Theorie \cite{bohm1952} (DBT) teilen konzeptionelle Parallelen, insbesondere in ihrer Behandlung nicht-lokaler Wechselwirkungen und der Rolle instantaner Korrelationen. 

\subsection{Nicht-Lokalität und Fernwirkung}
\begin{itemize}
    \item \textbf{WG}: Die gravitative Weber-Kraft wirkt direkt zwischen Massen, ohne Vermittlung durch ein Feld oder eine gekrümmte Raumzeit. Dies entspricht einem \textit{Fernwirkungsansatz}, der Geschwindigkeits- und Beschleunigungsterme ($\dot{r}$, $\ddot{r}$) einbezieht.
    
    \item \textbf{DBT}: Die Quantenpotentiale der DBT wirken instantan über beliebige Distanzen, was eine Form nicht-lokaler Kausalität impliziert. Die Wellenfunktion $\Psi$ steuert Teilchentrajektorien durch das Quantenpotential $Q = -\frac{\hbar^2}{2m} \frac{\nabla^2 |\Psi|}{|\Psi|}$.
\end{itemize}

\subsection{Instantane Korrelationen}
Beide Theorien postulieren eine zugrundeliegende instantane Dynamik:
\begin{itemize}
    \item In der WG manifestiert sich dies in der \textit{Energieerhaltung} durch phasenstarre Kopplung (vgl. Abschnitt \ref{sec:wellen}), die globale Korrelationen ohne Zeitverzögerung beschreibt.
    
    \item In der DBT führt das Quantenpotential zu sofortigen Anpassungen der Teilchenbahnen, unabhängig von ihrer räumlichen Trennung (\textit{„pilot wave“-Mechanismus}).
\end{itemize}

\subsection{Mathematische Analogien}
Die Struktur der Bewegungsgleichungen zeigt formale Ähnlichkeiten:
\begin{align}
    \text{WG:} \quad & \mathbf{F} = -\frac{GMm}{r^2} \left(1 - \frac{\dot{r}^2}{c^2} + \beta \frac{r\ddot{r}}{c^2}\right) \hat{\mathbf{r}}, \\
    \text{DBT:} \quad & m \frac{d^2 \mathbf{x}}{dt^2} = -\nabla (V + Q), 
\end{align}
wobei $V$ das klassische Potential und $Q$ das Quantenpotential ist. In beiden Fällen modifizieren Zusatzterme ($\dot{r}^2$, $\ddot{r}$ bzw. $Q$) die Newtonsche Dynamik.

\section{Quanten-Weber-Gravitation: Eine deterministische Synthese}
Die Kombination der Weber-Gravitation (WG) mit der De-Broglie-Bohm-Theorie (DBT) ermöglicht eine singularitätsfreie Quantengravitation mit experimentell prüfbaren Konsequenzen.

\subsection{Kernidee der Synthese}
Beide Theorien basieren auf deterministischen Fernwirkungen:
\begin{itemize}
    \item Die \textbf{WG} ersetzt die Raumzeitkrümmung durch Geschwindigkeits-/Beschleunigungsterme ($\dot{r}, \ddot{r}$).
    \item Die \textbf{DBT} fügt der klassischen Dynamik ein nicht-lokales Quantenpotential $Q$ hinzu.
\end{itemize}

\subsection{Hybrid-Gleichung}
Für ein Teilchen der Masse $m$ im Gravitationsfeld:
\begin{equation}
    m\frac{d^2\mathbf{r}}{dt^2} = \underbrace{-\frac{GMm}{r^2}\left(1-\frac{\dot{r}^2}{c^2}+\beta\frac{r\ddot{r}}{c^2}\right)\hat{\mathbf{r}}}_{\text{Weber-Kraft}} - \underbrace{\nabla Q}_{\text{Quantenpotential}}
\end{equation}
mit $Q = -\frac{\hbar^2}{2m}\frac{\nabla^2|\Psi|}{|\Psi|}$. Dies vermeidet Singularitäten, da $Q$ bei $r \to 0$ divergiert und Kollaps verhindert.

\subsection{Unschärferelation in der Weber-DBT-Synthese}
Die Heisenberg’sche Unschärferelation wird in der Weber-Gravitation nicht direkt modifiziert, da die Theorie klassisch-deterministisch ist. Allerdings zeigt die Synthese mit der
De-Broglie-Bohm-Theorie (Abschnitt~\ref{sec:dbt}) eine alternative Interpretation:
\begin{itemize}
    \item Die Unschärfe ist \textit{epistemisch} (durch versteckte Variablen des Quantenpotentials $Q$ bedingt).
    \item In starken Gravitationsfeldern könnte der Weber-Term $\frac{GM}{c^2 r}$ die effektive Unschärfe beeinflussen (vgl. \cite{bohm1952}).
\end{itemize}

\section{Die De-Broglie-Bohm-Theorie und die nicht-lokale Dynamik der Führungswelle}

Die De-Broglie-Bohm-Theorie (DBT) bietet eine deterministische Interpretation der Quantenmechanik, in der Teilchen durch eine Führungswelle $\Psi$ gesteuert werden. Dieser Abschnitt erläutert die mathematischen Grundlagen und die physikalischen Implikationen der DBT, insbesondere im Kontext des Doppelspaltexperiments.

\subsection{Grundgleichungen der DBT}

Die Dynamik der Führungswelle $\Psi$ wird durch die Schrödinger-Gleichung beschrieben:
\[ i\hbar\frac{\partial\Psi}{\partial t} = \left[-\frac{\hbar^2}{2m}\nabla^2 + V(x)\right]\Psi \]
wobei $V(x)$ das Potential der Spalte darstellt:
\[ V(x) = \begin{cases} 
0 & \text{in den Spaltöffnungen} \\
\infty & \text{sonst}
\end{cases} \]

Die Teilchenbewegung folgt aus der Bohmschen Trajektoriengleichung:
\[ \frac{d\mathbf{x}}{dt} = \frac{\hbar}{m}\text{Im}\left(\frac{\nabla\Psi}{\Psi}\right) \]
mit dem Quantenpotential:
\[ Q(x,t) = -\frac{\hbar^2}{2m}\frac{\nabla^2|\Psi|}{|\Psi|} \]

\subsection{Nicht-lokale Dynamik der Führungswelle}

Die Lösung $\Psi(x,t)$ reagiert instantan auf die Spaltbedingungen:
\[ \Psi(x,t) = \int G(x,x',t)\Psi_0(x')\,dx' \]
wobei $G(x,x',t)$ der nicht-lokale Propagator ist, der alle Pfade durch beide Spalte gleichzeitig berücksichtigt.

Für Spalte bei $x = \pm d/2$ ergibt sich das Interferenzmuster:
\[ \Psi(x,t) \sim e^{i(kx-\omega t)}\left[\exp\left(-\frac{(x-d/2)^2}{4\sigma^2}\right) + \exp\left(-\frac{(x+d/2)^2}{4\sigma^2}\right)\right] \]
\[ |\Psi|^2 \propto \cos^2\left(\frac{kdx}{2\sigma^2}\right) \]

\subsection{Energieerhaltung und instantaner Ausgleich}

Die Wahrscheinlichkeitserhaltung folgt aus der Kontinuitätsgleichung:
\[ \frac{\partial\rho}{\partial t} + \nabla\cdot(\rho\mathbf{v}) = 0 \quad \text{mit} \quad \rho = |\Psi|^2 \]

Die Gesamtenergie bleibt konstant:
\[ E_{\text{ges}} = \underbrace{\frac{1}{2}mv^2}_{\text{kin. Energie}} + \underbrace{Q(x,t)}_{\text{Quantenpotential}} + \underbrace{V(x)}_{\text{äußeres Potential}} \]

\subsection{Interpretation der Führungswelle}

Die nicht-lokale Dynamik lässt sich als instantane Energieoptimierung verstehen. Das effektive Energiefunktional des Systems lautet:
\[ \mathcal{E}[\Psi] = \underbrace{\frac{\hbar^2}{2m}\int|\nabla\Psi|^2\,d^3x}_{Q\text{-Term}} + \underbrace{\int V(x)|\Psi|^2\,d^3x}_{\text{Randbedingungen}} + \lambda\left(\int|\Psi|^2\,d^3x - 1\right) \]

Die stationäre Führungswelle $\Psi_0(x)$ realisiert das Minimum von $\mathcal{E}[\Psi]$, was äquivalent zur zeitunabhängigen Schrödinger-Gleichung ist.

\subsection{Konsequenzen}

\begin{itemize}
\item Die Interferenzmuster sind energetische Attraktoren des Systems
\item Die \enquote{spukhafte Fernwirkung} entspricht einem sofortigen Energieausgleich durch $Q(x,t)$
\item Experimentelle Vorhersage: Änderungen von $V(x)$ führen zu instantanen Änderungen von $\rho(x,t)$
\end{itemize}

\section{Kausalität durch Gleichzeitigkeit}
\label{sec:gleichzeitige_kausalitaet}

\subsection{Kernthese}
Die physikalische Standarddefinition von Kausalität ist unnötig restriktiv, wenn sie gleichzeitige Wechselwirkungen ausschließt. Ich argumentiere für einen erweiterten Kausalitätsbegriff, der zwei Prinzipien vereint:

\begin{itemize}
    \item \textbf{Determinismus}: Der Zustand $Z(t) = \{r, \dot{r}\}$ bestimmt eindeutig $Z(t+dt)$
    \item \textbf{Systemische Abhängigkeit}: Instantane Korrelationen sind kausal, wenn sie aus einer gemeinsamen Ursache folgen
\end{itemize}

\subsection{Anwendung auf die Weber-Kraft}
Die Weber-Gravitation zeigt dies exemplarisch:

\begin{equation}
    F = -\frac{GMm}{r^2}\left(1 - \frac{\dot{r}^2}{c^2} + \frac{r\ddot{r}}{2c^2}\right)
\end{equation}

\begin{itemize}
    \item Die Abhängigkeit von $\ddot{r}$ \textit{scheint} nicht-lokal
    \item Tatsächlich beschreibt sie eine \textit{systeminterne} Rückkopplung:
\end{itemize}

\begin{equation}
    \ddot{r} = f(r, \dot{r}) \quad \text{(lösbar nach Lipschitz-Bedingung)}
\end{equation}

\subsection{Philosophische Begründung}
\begin{itemize}
    \item Newtons 3. Gesetz wirkt ebenfalls instantan (actio = reactio)
    \item Quantenverschränkung zeigt: Gleichzeitige Korrelationen verletzen keine Kausalität
    \item Entscheidend ist nicht die \textit{Lokalität}, sondern die \textit{Eindeutigkeit} der Zeitentwicklung
\end{itemize}

\subsection{Konsequenzen}
\begin{tabular}{p{0.45\textwidth}p{0.45\textwidth}}
    \hline
    \textbf{Konventionelle Sicht} & \textbf{Diese Arbeit} \\
    \hline
    Kausalität erfordert Zeitverzögerung & Gleichzeitige Kausalität möglich \\
    Nicht-Lokalität = problematisch & Systemische Abhängigkeiten sind natürlich \\
    \hline
\end{tabular}

\section{Das Prinzip der energetischen Gleichzeitigkeit}
\label{sec:energetische_gleichzeitigkeit}

\subsection{Die fundamentale Rolle der Welle}
Die Natur realisiert durch Wellenphänomene eine \emph{instantane energetische Optimierung}:

\begin{itemize}
    \item Eine Welle $\Psi(\mathbf{x},t)$ stellt zu jedem Zeitpunkt $t$ global sicher, dass:
    \begin{equation}
        \delta \mathcal{E}[\Psi] = 0 \quad \text{(Energieminimierung)}
    \end{equation}
    
    \item Dieses Prinzip wirkt \emph{ohne Zeitverzug} und ist damit kausal im erweiterten Sinn
\end{itemize}

\subsection{Naturprinzip vs. Kausalitätsdogma}
Die konventionelle Kausalitätsdefinition widerspricht diesem Grundprinzip:

\begin{table}[ht]
    \centering
    \begin{tabular}{ll}
        \toprule
        \textbf{Mainstream-Kausalität} & \textbf{Energetische Gleichzeitigkeit} \\
        \midrule
        Lokale Wechselwirkungen & Globale Optimierung \\
        Ursache-Wirkung-Kette & Instantanes Minimum \\
        Lichtkegel-Beschränkung & Sofortige Anpassung \\
        \bottomrule
    \end{tabular}
    \caption{Konflikt der Paradigmen}
\end{table}

\subsection{Mathematische Konsequenz}
Das Wellenprinzip erzwingt eine Revision der Bewegungsgleichungen:

\begin{equation}
    \underbrace{\frac{\partial \Psi}{\partial t}}_{\text{Dynamik}} = 
    \underbrace{\mathcal{H}[\Psi]}_{\text{Instantane Optimierung}}
\end{equation}

wobei $\mathcal{H}$ ein \emph{globaler} Energieoperator ist.

\subsection{Physikalische Implikationen}
\begin{itemize}
    \item Die Weber-Kraft mit $\ddot{r}$-Abhängigkeit wird zur natürlichen Konsequenz
    \item Quantenverschränkung ist direkter Ausdruck dieses Prinzips
    \item Der Raum wird zum Träger der instantanen energetischen Information
\end{itemize}

\chapter{WG-DBT Synthese}
\section{Relativistische Energie-Impuls-Beziehung in der WG-DBT-Synthese}
\label{sec:energy-momentum}

Die Herleitung der relativistischen Energie-Impuls-Beziehung aus der Weber-Gravitation (WG) und De-Broglie-Bohm-Theorie (DBT) erfolgt wie folgt:

\subsection{Grundgleichungen}
Ausgehend von der verallgemeinerten Weber-Kraft für ein freies Teilchen:
\begin{equation}
\label{eq:wg_dbt_srt}
    \boxed
    {
        m\frac{d}{dt}(\gamma\mathbf{v}) = -\nabla Q
    }
\end{equation}
mit:
\begin{itemize}
\item $\gamma = (1 - \frac{v^2}{c^2} + \beta\frac{\mathbf{v}\cdot\mathbf{a}}{c^2})^{-1/2}$ (Weber-Lorentz-Faktor)
\item $Q = -\frac{\hbar^2}{2m}\frac{\nabla^2|\Psi|}{|\Psi|}$ (Quantenpotential)
\end{itemize}

\subsection{Stationäre Lösung}
Für $\mathbf{F} = 0$ und konstante Geschwindigkeit ($\mathbf{a} = 0$):
\begin{equation}
\gamma m\mathbf{v} = \mathbf{p} = \text{konstant}
\end{equation}
Mit der DBT-Impulsdefinition:
\begin{equation}
\mathbf{p} = \hbar\nabla S
\end{equation}

\subsection{Energie-Impuls-Relation}
\begin{align}
E &= \gamma mc^2 = \frac{mc^2}{\sqrt{1-v^2/c^2}} \\
p^2 &= \gamma^2m^2v^2 = \frac{m^2v^2}{1-v^2/c^2} \\
\Rightarrow v^2 &= \frac{p^2c^2}{m^2c^2 + p^2} \\
E &= \sqrt{m^2c^4 + p^2c^2}
\end{align}

\subsection{Kovariante Formulierung}
\begin{equation}
p^\mu p_\mu = \frac{E^2}{c^2} - p^2 = m^2c^2
\end{equation}

\subsection{Interpretation}
\begin{itemize}
\item Die WG liefert die relativistische Dynamik
\item Die DBT verknüpft diese mit der Quantenmechanik
\item Die SRT-Relation emergiert als Grenzfall
\item Das Quantenpotential $Q$ führt zu zusätzlichen Quanteneffekten
\end{itemize}

\begin{table}[h]
\centering
\caption{Grenzfälle der Energie-Impuls-Beziehung}
\begin{tabular}{ll}
\hline
Nicht-relativistisch ($v \ll c$) & $E \approx mc^2 + \frac{p^2}{2m} + Q$ \\
Ultra-relativistisch ($v \to c$) & $E \approx pc$ \\
Quantenlimit & $E \approx \sqrt{p^2c^2 + m^2c^4} + Q$ \\
\hline
\end{tabular}
\end{table}

\subsection*{Ontologischer Status der SRT}
Die Spezielle Relativitätstheorie stellt sich in diesem Rahmen als \textit{effektive Beschreibung} heraus, die:
\begin{itemize}
\item Im Bereich \( v \ll c \), \( L \gg \ell_p \) gültig ist
\item Aber durch tiefere Prinzipien (Fernwirkung + Führungswelle) ersetzt wird
\end{itemize}

\section{Exakte Herleitung der Weber-DBT-Bewegungsgleichung}
\label{sec:exact_derivation}

Ausgehend von der Weber-Gravitationskraft und dem Quantenpotential der De-Broglie-Bohm-Theorie leiten wir die vollständige nicht-genäherte Bewegungsgleichung ab.

\subsection{Kombinierte Lagrange-Funktion}
Die Wirkung des Systems setzt sich aus kinetischer Energie, Weber-Potential und Quantenpotential zusammen:

\begin{equation}
\mathcal{L} = \underbrace{\frac{1}{2}m\dot{\mathbf{r}}^2}_{T} - \underbrace{\frac{GMm}{r}\left[1 - \frac{\dot{r}^2}{2c^2} + \beta\frac{\mathbf{r}\cdot\ddot{\mathbf{r}}}{2c^2}\right]}_{V_{\text{WG}}} - \underbrace{Q(\mathbf{r},t)}_{\text{Quantenpotential}}
\end{equation}

mit dem Quantenpotential $Q = -\frac{\hbar^2}{2m}\frac{\nabla^2|\Psi|}{|\Psi|}$.

\subsection{Euler-Lagrange-Gleichung}
Die exakte Bewegungsgleichung folgt aus:

\begin{equation}
\frac{d}{dt}\left(\frac{\partial\mathcal{L}}{\partial\dot{\mathbf{r}}}\right) - \frac{\partial\mathcal{L}}{\partial\mathbf{r}} = 0
\end{equation}

\subsection{Ableitung der Terme}
\begin{enumerate}
\item \textbf{Kanonischer Impuls}:
\begin{align}
\frac{\partial\mathcal{L}}{\partial\dot{\mathbf{r}}} &= m\dot{\mathbf{r}} + \frac{GMm}{c^2}\left(\frac{\dot{\mathbf{r}}}{r} - \beta\frac{\mathbf{r}}{2r}\frac{d}{dt}\ln\dot{r}\right) \\
&= m\dot{\mathbf{r}}\left[1 + \frac{GM}{c^2r}\left(1 - \frac{\beta}{2}\frac{\mathbf{r}\cdot\ddot{\mathbf{r}}}{\dot{r}^2}\right)\right]
\end{align}

\item \textbf{Zeitableitung}:
\begin{equation}
\frac{d}{dt}\left(\frac{\partial\mathcal{L}}{\partial\dot{\mathbf{r}}}\right) = m\ddot{\mathbf{r}}\left[1 + \mathcal{O}(c^{-2})\right] + \text{höhere Ableitungen}
\end{equation}

\item \textbf{Ortsableitung}:
\begin{equation}
\frac{\partial\mathcal{L}}{\partial\mathbf{r}} = -\frac{GMm}{r^2}\left[1 - \frac{3\dot{r}^2}{2c^2} + \beta\frac{\ddot{r}}{c^2}\right]\hat{\mathbf{r}} - \nabla Q
\end{equation}
\end{enumerate}

\subsection{Exakte Bewegungsgleichung}
Durch Zusammenführung aller Terme erhalten wir die nicht-genäherte Gleichung:

\begin{equation}
\boxed{
m\frac{d}{dt}\left(\gamma_{\text{WG}}\mathbf{v}\right) = -\nabla Q
}
\end{equation}

mit dem vollständigen Weber-Lorentz-Faktor:

\begin{equation}
\gamma_{\text{WG}} = \left[1 - \frac{v^2}{c^2} + \beta\left(\frac{\mathbf{a}\cdot\mathbf{r}}{c^2} + \frac{(\mathbf{v}\cdot\mathbf{r})^2}{c^2r^2}\right) - \frac{GM}{c^2r}\left(1 - \frac{\beta}{2}\frac{\mathbf{r}\cdot\mathbf{j}}{\dot{r}^2}\right)\right]^{-1/2}
\end{equation}

wobei $\mathbf{j} = d\mathbf{a}/dt$ die Jerk-Komponente darstellt.

\subsection{Diskussion der Terme}
\begin{itemize}
\item Der Term $\propto \mathbf{j}$ beschreibt nicht-lokale Änderungen der Beschleunigung
\item Die Kopplung $\mathbf{a}\cdot\mathbf{r}$ modifiziert effektiv die träge Masse
\item Für $\beta=0$ und $Q=0$ reduziert sich die Gleichung auf die spezielle Relativitätstheorie
\end{itemize}

\section{Kovariante Formulierung der exakten Weber-DBT-Gleichung}
\label{sec:covariant_formulation}

Die vollständige kovariante Formulierung der Weber-Dynamik kombiniert mit der De-Broglie-Bohm-Theorie erfordert eine manifest relativistische Darstellung unter Berücksichtigung aller höherer Ordnungen.

\subsection{Kovariante Grundgrößen}
Wir definieren in Minkowski-Raumzeit mit Metrik $\eta_{\mu\nu} = \mathrm{diag}(-1,1,1,1)$:

\begin{align}
\text{Vierergeschwindigkeit:} &\quad u^\mu = \gamma(c, \mathbf{v}), \quad \gamma = (1-v^2/c^2)^{-1/2} \\
\text{Eigenbeschleunigung:} &\quad a^\mu = \frac{du^\mu}{d\tau} = \gamma^4\left(\frac{\mathbf{v}\cdot\mathbf{a}}{c}, \mathbf{a} + \gamma^2\frac{(\mathbf{v}\cdot\mathbf{a})\mathbf{v}}{c^2}\right) \\
\text{Eigen-Jerk:} &\quad j^\mu = \frac{da^\mu}{d\tau} = \gamma^7\left(\frac{a^2 + \mathbf{v}\cdot\mathbf{j}}{c}, \mathbf{j} + 3\gamma^2\frac{(\mathbf{v}\cdot\mathbf{a})\mathbf{a}}{c^2} + \gamma^2\frac{(\mathbf{v}\cdot\mathbf{j})\mathbf{v}}{c^2}\right)
\end{align}

\subsection{Exakter Weber-Lorentz-Faktor}
Der vollständige relativistische Faktor inklusive Jerk-Termen lautet:

\begin{equation}
\gamma_{\mathrm{WG}} = \left[1 - \frac{v^2}{c^2} + \beta\left(\frac{\mathbf{r}\cdot\mathbf{a}}{c^2} + \frac{(\mathbf{v}\cdot\mathbf{r})^2}{c^2r^2}\right) - \beta\frac{GM}{c^4}\left(\frac{\mathbf{r}\cdot\mathbf{j}}{r} + \frac{(\mathbf{v}\cdot\mathbf{r})(\mathbf{a}\cdot\mathbf{r})}{r^3}\right)\right]^{-1/2}
\end{equation}

\subsection{Kovariante Bewegungsgleichung}
Die exakte kovariante Form der Weber-DBT-Dynamik:

\begin{equation}
\boxed{
m\frac{D}{D\tau}\left(\gamma_{\mathrm{WG}} u^\mu\right) = -\frac{\hbar^2}{2m}\partial^\mu\left(\frac{\Box|\Psi|}{|\Psi|}\right)
}
\end{equation}

mit:
\begin{itemize}
\item Kovariante Ableitung: $\frac{D}{D\tau} = u^\nu\partial_\nu$
\item d'Alembert-Operator: $\Box = \partial_\mu\partial^\mu = -\frac{1}{c^2}\frac{\partial^2}{\partial t^2} + \nabla^2$
\end{itemize}

\subsection{Komponentenentwicklung}

\subsubsection{Zeitkomponente ($\mu=0$)}
\begin{equation}
\frac{d}{d\tau}\left(\gamma_{\mathrm{WG}}\gamma c\right) = \frac{\hbar^2}{2mc^2}\frac{\partial}{\partial t}\left(\frac{\Box|\Psi|}{|\Psi|}\right)
\end{equation}

\subsubsection{Raumkomponenten ($\mu=1,2,3$)}
\begin{equation}
\frac{d}{d\tau}\left(\gamma_{\mathrm{WG}}\gamma\mathbf{v}\right) = -\frac{\hbar^2}{2m}\nabla\left(\frac{\Box|\Psi|}{|\Psi|}\right)
\end{equation}

\subsection{Diskussion der Terme}
\begin{itemize}
\item \textbf{Jerk-Abhängigkeit}: Die $\mathbf{j}$-Terme in $\gamma_{\mathrm{WG}}$ beschreiben nicht-lokale Fernwirkungseffekte
\item \textbf{Quantenpotential}: Der kovariante d'Alembert-Operator $\Box$ ersetzt das klassische $\nabla^2$
\item \textbf{Energieerhaltung}: Die Zeitkomponente enthält Korrekturen zur relativistischen Energie-Impuls-Beziehung
\end{itemize}

\begin{table}[h]
\centering
\caption{Vergleich der Formulierungen}
\begin{tabular}{ll}
\hline
\textbf{Genäherte Form (4.1.1)} & \textbf{Exakte kovariante Form} \\
\hline
$\gamma_{\mathrm{WG}} \approx 1 + \frac{v^2}{2c^2}$ & Vollständige Jerk-Abhängigkeit \\
$-\nabla Q$ & $-\partial^\mu(\hbar^2\Box|\Psi|/2m|\Psi|)$ \\
Newton-artige Darstellung & Manifest kovariant \\
\hline
\end{tabular}
\end{table}

\newpage
\section{Rotationskurven in der Weber-DBT-Gravitation}
Die Rotationsgeschwindigkeiten von Galaxien lassen sich durch eine Kombination der Weber-Gravitation (WG) mit der De-Broglie-Bohm-Theorie (DBT) erklären, ohne auf dunkle Materie zurückzugreifen. 

\subsection{Theoretische Grundlagen}
Die Bewegungsgleichung für ein Testteilchen der Masse $m$ im Gravitationsfeld einer Galaxie lautet in der WG-DBT-Synthese:

\begin{equation}
m \frac{d}{dt}(\gamma_{\text{WG}} \mathbf{v}) = -\frac{GMm}{r^2}\left(1 - \frac{\dot{r}^2}{c^2} + \beta \frac{r\ddot{r}}{c^2}\right)\hat{\mathbf{r}} - \nabla Q
\end{equation}

wobei:
\begin{itemize}
\item $\gamma_{\text{WG}} = \left(1 - \frac{v^2}{c^2} + \beta \frac{\mathbf{r}\cdot\mathbf{a}}{c^2}\right)^{-1/2}$ der Weber-Lorentz-Faktor ist ($\beta = 0.5$)
\item $Q = -\frac{\hbar^2}{2m}\frac{\nabla^2|\Psi|}{|\Psi|}$ das Quantenpotential der DBT darstellt
\end{itemize}

\subsection{Stationäre Lösung für Kreisbahnen}
Für stabile Kreisbahnen ($\dot{r} = 0$, $\ddot{r} = -v^2/r$) vereinfacht sich dies zu:

\begin{equation}
\frac{v^2}{r} = \frac{GM(r)}{r^2} + \frac{\hbar^2}{2m^2}\left|\frac{\nabla^2\sqrt{\rho}}{\sqrt{\rho}}\right|
\end{equation}

Mit der angenommenen Dichteverteilung $\rho(r) = \rho_0 e^{-r/r_0}$ ergibt sich:

\begin{equation}
v^2(r) = \underbrace{\frac{GM(r)}{r}}_{\text{Baryonisch}} + \underbrace{\frac{\hbar^2}{2m^2 r_0 R}}_{\text{DBT-Korrektur}} + \mathcal{O}\left(\frac{v^2}{c^2}\right)
\end{equation}

\subsection{Physikalische Interpretation}
Die nicht-lokale Natur der DBT-Führungswelle $\Psi$ führt zu einem konstanten Geschwindigkeitsbeitrag $v_0$:

\begin{equation}
v_0^2 \equiv \frac{\hbar^2}{2m^2 r_0 R}
\end{equation}

wobei:
\begin{itemize}
\item $m \approx 2\pi \times 10^{-40}\,\text{kg}$ eine natürliche Massenskala darstellt
\item $r_0$ die Skalenlänge der Galaxie ist
\item $R$ den charakteristischen Wirkungsradius der Führungswelle beschreibt
\end{itemize}

Diese Formulierung zeigt, dass die beobachteten flachen Rotationskurven durch die Kombination von:
\begin{enumerate}
\item relativistischen Korrekturen der Weber-Gravitation ($\beta$-Term)
\item nicht-lokalen Quanteneffekten der DBT ($v_0$-Term)
\end{enumerate}
erklärt werden können - ohne Einführung dunkler Materie.

\subsection{Berechnungsbeispiel einer Rotationskurve}

Für eine typische Spiralgalaxie mit folgenden Parametern:
\begin{itemize}
\item Gesamtmasse der sichtbaren Materie: $M = 10^{11} M_\odot$
\item Skalenlänge: $r_0 = 3\ \text{kpc}$
\item Charakteristischer Radius: $R = 15\ \text{kpc}$
\item DBT-Massenskala: $m = 2\pi \times 10^{-40}\ \text{kg} \approx 1.2 \times 10^{-3}\ \text{eV}/c^2$
\end{itemize}

Die Rotationsgeschwindigkeit setzt sich zusammen aus:

\begin{equation}
v(r) = \sqrt{v_b^2(r) + v_0^2}
\end{equation}

mit:
\begin{align*}
v_b(r) &= \sqrt{\frac{GM(r)}{r}} \quad \text{(baryonischer Anteil)} \\
v_0 &= \sqrt{\frac{\hbar^2}{2m^2 r_0 R}} \quad \text{(DBT-Korrektur)}
\end{align*}

\begin{table}[h]
\centering
\caption{Berechnete Rotationsgeschwindigkeiten für verschiedene Radien}
\label{tab:rotation}
\begin{tabular}{cccc}
\hline
Radius $r$ (kpc) & $v_b$ (km/s) & $v_0$ (km/s) & $v_{\text{gesamt}}$ (km/s) \\
\hline
1 & 125.4 & 73.8 & 145.2 \\
3 & 129.1 & 73.8 & 148.6 \\ 
5 & 124.7 & 73.8 & 144.9 \\
10 & 110.3 & 73.8 & 132.5 \\
15 & 95.2 & 73.8 & 120.4 \\
20 & 82.4 & 73.8 & 110.8 \\
30 & 67.2 & 73.8 & 99.9 \\
\hline
\end{tabular}
\end{table}

Die Berechnung zeigt:
\begin{itemize}
\item Den klassisch keplerschen Abfall des baryonischen Anteils $v_b(r)$
\item Den konstanten DBT-Beitrag $v_0 \approx 74\ \text{km/s}$
\item Die resultierende flache Rotationskurve für $r > 10\ \text{kpc}$
\end{itemize}

\noindent Die Übereinstimmung mit beobachteten Werten (typisch $100-200\ \text{km/s}$) bestätigt die Wirksamkeit des WG-DBT-Ansatzes.

\newpage
\section{Lichtablenkung in der Weber-DBT-Gravitation}
\label{sec:light_deflection}

Die Ablenkung von Licht im Gravitationsfeld lässt sich in der Weber-DBT-Theorie durch eine Modifikation der geodätischen Gleichung beschreiben. Wir leiten den Ablenkwinkel $\alpha$ für einen Lichtstrahl mit Stoßparameter $b$ her.

\subsection{Bewegungsgleichung für Photonen}
Aus der WG-DBT-Gleichung folgt für masselose Teilchen ($m \to 0$, aber $E = h\nu \neq 0$):

\begin{equation}
\frac{d}{d\lambda}\left(\frac{dx^\mu}{d\lambda}\right) = -\Gamma^\mu_{\nu\sigma}\frac{dx^\nu}{d\lambda}\frac{dx^\sigma}{d\lambda} - \frac{1}{E}\nabla^\mu Q
\end{equation}

wobei:
\begin{itemize}
\item $\lambda$ ein affiner Parameter ist
\item $Q = -\frac{\hbar^2}{2E}\frac{\Box|\Psi|}{|\Psi|}$ das quantenmechanische Potential für Photonen
\item $\Gamma^\mu_{\nu\sigma}$ die Weber-Korrekturen zu den Christoffel-Symbolen enthält
\end{itemize}

\subsection{Lösung für kleine Ablenkungen}
Für einen Lichtstrahl in $z$-Richtung mit Stoßparameter $b$ lautet die transversale Beschleunigung:

\begin{equation}
\frac{d^2x}{dz^2} \approx -\frac{GM}{c^2}\left(\frac{1}{b^2} + \beta\frac{\partial^2_x \Phi}{c^2}\right)x - \frac{\hbar^2}{2E^2}\partial_x\left(\frac{\Box|\Psi|}{|\Psi|}\right)
\end{equation}

mit $\Phi = -GM/r$ dem Newton-Potential. 

\subsection{Quantenpotential für Licht}
Für eine ebene Welle $|\Psi| \propto e^{-r^2/2\sigma^2}$ ergibt sich:

\begin{equation}
Q \approx -\frac{\hbar^2}{2E\sigma^2}\left(1 - \frac{r^2}{\sigma^2}\right)
\end{equation}

Die typische Wirkungsskala ist $\sigma \sim b$, sodass:

\begin{equation}
\frac{1}{E}\nabla_x Q \approx \frac{\hbar^2}{E^2b^3}x
\end{equation}

\subsection{Integrierter Ablenkwinkel}
Der Gesamtablenkwinkel $\alpha$ ergibt sich durch Integration entlang der Trajektorie:

\begin{align}
\alpha &= \frac{2GM}{c^2b}\left(1 + \beta\frac{2GM}{c^2b}\right) + \frac{\pi\hbar^2}{4E^2b^2} \\
&= \underbrace{\frac{4GM}{c^2b}}_{\text{Einstein (ART)}} + \underbrace{\frac{2GM}{c^2b}\left(\beta - 2\right)}_{\text{Weber-Korrektur}} + \underbrace{\frac{\pi\hbar^2}{4E^2b^2}}_{\text{DBT-Term}}
\end{align}

Für $\beta = 1$ (Lichtablenkung) und $E = h\nu$:

\begin{equation}
\boxed{
\alpha = \frac{4GM}{c^2b} - \frac{2GM}{c^2b} + \frac{\pi h^2}{4(h\nu)^2b^2}
}
\end{equation}

\section{Shapiro-Effekt in der Weber-DBT-Gravitation}
\label{sec:shapiro_effect}

Der Shapiro-Effekt beschreibt die gravitative Zeitverzögerung von Lichtsignalen. In der Weber-DBT-Theorie ergibt sich eine modifizierte Version dieses Effekts durch die Kombination aus Weber-Gravitation und Quantenpotential.

\subsection{Laufzeitverzögerung}
Für ein Lichtsignal, das an einer Masse $M$ mit minimalem Abstand $b$ vorbeiläuft, beträgt die zusätzliche Laufzeit:

\begin{equation}
\Delta t = \frac{2GM}{c^3}\left[\ln\left(\frac{4r_e r_p}{b^2}\right) + \frac{\beta GM}{c^2b}\right] + \frac{\hbar^2}{4E^2c^3b^2}(r_e + r_p)
\end{equation}

wobei:
\begin{itemize}
\item $r_e$ und $r_p$ die Abstände von Masse zu Emitter bzw. Detektor sind
\item $\beta = 0.5$ für die Weber-Gravitation
\item $E = h\nu$ die Photonenenergie
\end{itemize}

\subsection{Herleitung}
Aus der WG-DBT-Metrik für schwache Felder:

\begin{equation}
ds^2 = -\left(1 - \frac{2GM}{c^2r} + \frac{Q}{E}\right)c^2dt^2 + \left(1 + \frac{2GM}{c^2r}\right)dr^2
\end{equation}

Die Lichtlaufzeit folgt aus:

\begin{equation}
\Delta t = 2\int_{b}^{r_e} \frac{1}{c}\left[\left(1 + \frac{2GM}{c^2r} - \frac{Q}{E}\right)^{-1} - 1\right] dr
\end{equation}

Mit dem Quantenpotential $Q \approx -\hbar^2/(2Eb^2)$ für $r \approx b$:

\begin{equation}
\Delta t \approx \frac{2GM}{c^3}\ln\left(\frac{4r_e r_p}{b^2}\right) + \frac{\beta G^2M^2}{c^5b} + \frac{\hbar^2(r_e + r_p)}{4E^2c^3b^2}
\end{equation}

\subsection{Physikalische Interpretation}
\begin{itemize}
\item Der erste Term entspricht der klassischen ART-Vorhersage
\item Der Weber-Term ($\beta$) führt zu einer zusätzlichen $1/b$-Abhängigkeit
\item Der DBT-Term zeigt charakteristische Frequenzabhängigkeit ($\propto \nu^{-2}$)
\end{itemize}

Für Radarsignale ($\nu \sim 10^{10}$ Hz) im Sonnensystem:
\begin{equation}
\Delta t_{\text{WG-DBT}} \approx 240\,\mu\text{s} - 10^{-36}\,\mu\text{s} + 10^{-72}\,\mu\text{s}
\end{equation}

Die Quantenkorrektur ist vernachlässigbar, aber prinzipiell vorhanden.

\chapter{WG-DBT-Kinetik}
\section{Bahngleichung in der Weber-DBT-Gravitation}
\label{sec:bahngleichung}

Die Kombination der Weber-Gravitation (WG) mit der De-Broglie-Bohm-Theorie (DBT) führt zu einer modifizierten Bahndynamik, die durch eine nichtlineare Differentialgleichung beschrieben wird. Im Folgenden leiten wir die exakte Bahngleichung $r(\phi)$ her.

\subsection{Kraftgleichung und Potentiale}
Ausgehend von der verallgemeinerten Bewegungsgleichung (Gl.~3.2.7 der Arbeit):

\begin{equation}
m \frac{d}{dt}(\gamma_{\mathrm{WG}}\mathbf{v}) = \mathbf{F}_{\mathrm{WG}} + \mathbf{F}_Q
\end{equation}

mit den Komponenten:
\begin{itemize}
\item Weber-Gravitationskraft:
\begin{equation}
\mathbf{F}_{\mathrm{WG}} = -\frac{GMm}{r^2}\left(1-\frac{\dot{r}^2}{c^2}+\beta\frac{r\ddot{r}}{c^2}\right)\hat{\mathbf{r}}
\end{equation}

\item Quantenkraft:
\begin{equation}
\mathbf{F}_Q = -\nabla Q = \frac{\hbar^2}{2m}\nabla\left(\frac{\nabla^2|\Psi|}{|\Psi|}\right)
\end{equation}

\item Weber-Lorentz-Faktor:
\begin{equation}
\gamma_{\mathrm{WG}} = \left[1-\frac{v^2}{c^2}+\beta\left(\frac{\mathbf{a}\cdot\mathbf{r}}{c^2}+\frac{(\mathbf{v}\cdot\mathbf{r})^2}{c^2r^2}\right)\right]^{-1/2}
\end{equation}
\end{itemize}

\subsection{Transformation auf Polarkoordinaten}
Mit den Polarkoordinaten $(r,\phi)$ und dem spezifischen Drehimpuls $h = r^2\dot{\phi} = \mathrm{const.}$ ergibt sich:

\begin{align}
\mathbf{v} &= \dot{r}\hat{\mathbf{r}} + r\dot{\phi}\hat{\boldsymbol{\phi}} \\
\mathbf{a} &= (\ddot{r}-r\dot{\phi}^2)\hat{\mathbf{r}} + (r\ddot{\phi}+2\dot{r}\dot{\phi})\hat{\boldsymbol{\phi}}
\end{align}

\subsection{Radiale Komponente der Bewegungsgleichung}
Die radiale Komponente lautet:

\begin{equation}
\frac{d}{dt}(\gamma_{\mathrm{WG}}\dot{r}) - \gamma_{\mathrm{WG}}r\dot{\phi}^2 = -\frac{GM}{r^2}\left(1-\frac{\dot{r}^2}{c^2}+\beta\frac{r\ddot{r}}{c^2}\right) + \frac{\hbar^2}{2m^2}\frac{\partial}{\partial r}\left(\frac{\nabla^2|\Psi|}{|\Psi|}\right)
\end{equation}

\subsection{Substitution und exakte Differentialgleichung}
Mit der Variablentransformation $u = 1/r$ und den Ableitungen:

\begin{align}
\dot{r} &= -h\frac{du}{d\phi} \\
\ddot{r} &= -h^2u^2\frac{d^2u}{d\phi^2}
\end{align}

erhalten wir die nichtlineare Bahngleichung:

\begin{equation}
\boxed{
\frac{d^2u}{d\phi^2}\left(1-\beta\frac{GM}{c^2}u\right) + u = \frac{GM}{h^2}\left(1-\frac{h^2}{c^2}\left(\frac{du}{d\phi}\right)^2\right) - \frac{\hbar^2}{2m^2h^2u^2}\frac{d}{du}\left(\frac{\nabla^2|\Psi|}{|\Psi|}\right)
}
\label{eq:master}
\end{equation}

\subsection{Diskussion der Terme}
\begin{itemize}
\item Der Term $\propto \beta$ modifiziert die effektive Masse
\item Der $(\frac{du}{d\phi})^2$-Term entspricht der relativistischen Korrektur
\item Das Quantenpotential $\propto \hbar^2$ führt zu nicht-lokalen Effekten
\end{itemize}

\subsection{Grenzfälle}
\begin{enumerate}
\item \textbf{Newton'scher Grenzfall} ($c\to\infty$, $\hbar\to0$):
\begin{equation}
\frac{d^2u}{d\phi^2} + u = \frac{GM}{h^2}
\end{equation}

\item \textbf{Reine Weber-Gravitation} ($\hbar\to0$):
\begin{equation}
\frac{d^2u}{d\phi^2}\left(1-\beta\frac{GM}{c^2}u\right) + u = \frac{GM}{h^2}\left(1-\frac{h^2}{c^2}\left(\frac{du}{d\phi}\right)^2\right)
\end{equation}
\end{enumerate}

\begin{table}[h]
\centering
\caption{Parameter der Bahngleichung}
\begin{tabular}{ll}
\hline
Symbol & Physikalische Bedeutung \\ \hline
$\beta$ & Weber-Beschleunigungsparameter ($\beta=0.5$ für Gravitation) \\
$h$ & Spezifischer Drehimpuls \\
$Q$ & Quantenpotential \\
\hline
\end{tabular}
\end{table}

\section{Periheldrechnung in der Weber-DBT-Theorie}
Die Bewegungsgleichung der Weber-DBT-Synthese (Gl. 4.1.10) lautet vollständig:

\begin{equation}
\frac{d^2 u}{d\phi^2} \left(1 - \beta \frac{GM}{c^2} u \right) + u = \frac{GM}{h^2} \left(1 - \frac{h^2}{c^2} \left(\frac{du}{d\phi}\right)^2 \right) - \frac{\hbar^2}{2m^2 h^2 u^2} \frac{d}{du} \left(\frac{\nabla^2 |\Psi|}{|\Psi|} \right)
\end{equation}

\subsection{Quantenpotential-Explizierung}
Für das Quantenpotential wird die Wellenfunktion eines kohärenten makroskopischen Zustands angesetzt:
\begin{align}
|\Psi| &\propto e^{-(r - r_0)^2/(2\sigma^2)}, \quad \sigma \sim \text{Planetenradius} \\
\frac{\nabla^2 |\Psi|}{|\Psi|} &= \frac{1}{\sigma^2}\left(1 - \frac{(r - r_0)^2}{\sigma^2}\right) \\
\frac{d}{du} \left(\frac{\nabla^2 |\Psi|}{|\Psi|}\right) &= \frac{2r^3}{\sigma^4}(r - r_0)
\end{align}

\subsection{Vollständige Differentialgleichung}
Einsetzen aller Terme ergibt:
\begin{equation}
\frac{d^2 u}{d\phi^2} \left(1 - \frac{GM}{2c^2} u \right) + u = \frac{GM}{h^2} \left(1 - \frac{h^2}{c^2} \left(\frac{du}{d\phi}\right)^2 \right) - \frac{\hbar^2 r^3 (r - r_0)}{m^2 h^2 \sigma^4 u^2}
\end{equation}

\subsection{Störungstheorie um Newtonsche Lösung}
\begin{itemize}
\item Newtonsche Bahn: $u_0(\phi) = \frac{GM}{h^2}(1 + e \cos\phi)$
\item Ansatz: $u = u_0 + \delta u$ mit Störung $\delta u$
\item Exakte Störungsgleichung:
\begin{equation}
\frac{d^2 \delta u}{d\phi^2} + \delta u = \underbrace{\frac{GM}{2c^2} u_0 \frac{d^2 u_0}{d\phi^2}}_{\text{Weber-Term}} - \underbrace{\frac{h^2}{c^2} \left(\frac{du_0}{d\phi}\right)^2}_{\text{Relativistisch}} - \underbrace{\frac{\hbar^2 r^3 (r - r_0)}{m^2 h^2 \sigma^4 u_0^2}}_{\text{Quantenterm}}
\end{equation}
\end{itemize}

\subsection{Beitragsanalyse}
\begin{itemize}
\item Weber-Term: $-\frac{G^2 M^2 e}{2c^2 h^4} \cos\phi$
\item Relativistischer Term: $-\frac{G^2 M^2 e^2}{c^2 h^4} \sin^2\phi$
\item Quantenterm: $\mathcal{O}\left(\frac{\hbar^2}{m^2 \sigma^4}\left(\frac{h^2}{GM}\right)^5\right) \approx 10^{-80} \text{ (formal erhalten)}$
\end{itemize}

\subsection{Resultat}
Die Periheldrehung pro Umlauf ergibt sich aus der säkularen Drift:
\begin{equation}
\Delta \phi = \frac{6\pi GM}{c^2 a(1 - e^2)} + \mathcal{O}\left(\frac{\hbar^2}{m^2 \sigma^4}\right)
\end{equation}


\part{Anhang}
\chapter{Ergänzende Informationen}
\label{chapter:information}
\section{Die Rolle des $\beta$-Parameters}

Der $\beta$-Parameter in der Weber-Kraft

\begin{equation}
F = -\frac{GMm}{r^2}\left(1 - \frac{\dot{r}^2}{c^2} + \beta\frac{r\ddot{r}}{c^2}\right)\hat{r}
\end{equation}

bestimmt das Verhältnis von Beschleunigungs- zu Geschwindigkeitstermen und variiert je nach Wechselwirkungstyp:

\subsection{Elektrodynamik (Original-Weber)}
Für elektromagnetische Wechselwirkungen gilt $\beta=2$:
\begin{itemize}
\item Führt zur korrekten Beschreibung beschleunigter Ladungen
\item Reproduziert die magnetische Komponente der Lorentz-Kraft
\item Keine Lichtablenkung ($m=0$ liefert $F=0$)
\end{itemize}

\subsection{Gravitation (Massen)}
Für massive Körper im Gravitationsfeld:
\begin{itemize}
\item $\beta=0.5$ erklärt die Periheldrehung des Merkur
\item Führt zur ART-konformen Lichtablenkung für makroskopische Körper
\item Universelle Formel: $\beta = 1 - \frac{mc^2}{2E}$
\end{itemize}

\subsection{Photonen (Lichtablenkung)}
Für masselose Teilchen ($m=0$, $E=h\nu$):
\begin{itemize}
\item $\beta=1$ erzwingt die Frequenzabhängigkeit
\item Beschleunigungsterm dominiert: $\frac{r\ddot{r}}{c^2} \approx \frac{h^2}{c^2r^4}$
\item Liefert den Zusatzterm $\propto \lambda^{-2}$
\end{itemize}

\begin{table}[h]
\centering
\caption{$\beta$-Werte im Vergleich}
\begin{tabular}{lcc}
\hline
Anwendung & $\beta$ & Physikalische Konsequenz \\
\hline
Elektrodynamik & 2 & Magnetische Wechselwirkungen \\
Gravitation (Massen) & 0.5 & Periheldrehung des Merkur \\
Photonen & 1 & Frequenzabhängige Lichtablenkung \\
\hline
\end{tabular}
\end{table}
\section{Herleitung der kombinierten WG-DBT Bewegungsgleichung}

\subsection*{1. Ausgangspunkt: Weber-Gravitationskraft}
Die klassische Weber-Kraft für zwei Massen $m$ und $M$ lautet:
\begin{equation}
\mathbf{F}_{\text{WG}} = -\frac{GMm}{r^2}\left(1 - \frac{\dot{r}^2}{c^2} + \beta\frac{r\ddot{r}}{c^2}\right)\hat{\mathbf{r}}
\end{equation}

\subsection*{2. Umformung der radiale Beschleunigungsterme}
Wir entwickeln die Terme $\dot{r}^2$ und $r\ddot{r}$ in vektorieller Form:

\begin{align}
\dot{r} &= \frac{d}{dt}\sqrt{\mathbf{r}\cdot\mathbf{r}} = \frac{\mathbf{r}\cdot\mathbf{v}}{r} \\
\dot{r}^2 &= \left(\frac{\mathbf{r}\cdot\mathbf{v}}{r}\right)^2 \\
r\ddot{r} &= \frac{d}{dt}(r\dot{r}) - \dot{r}^2 = \mathbf{v}\cdot\mathbf{v} + \mathbf{r}\cdot\mathbf{a} - \left(\frac{\mathbf{r}\cdot\mathbf{v}}{r}\right)^2
\end{align}

Für kleine Abweichungen von Kreisbahnen vernachlässigen wir den letzten Term und erhalten:
\begin{equation}
r\ddot{r} \approx v^2 + \mathbf{r}\cdot\mathbf{a}
\end{equation}

\subsection*{3. Verallgemeinerte Weber-Kraft in vektorieller Form}
Einsetzen in (1) ergibt:
\begin{equation}
\mathbf{F}_{\text{WG}} = -\frac{GMm}{r^2}\left(1 - \frac{(\mathbf{r}\cdot\mathbf{v})^2}{c^2r^2} + \beta\frac{v^2 + \mathbf{r}\cdot\mathbf{a}}{c^2}\right)\hat{\mathbf{r}}
\end{equation}

\subsection*{4. Lagrange-Formulierung der Weber-Gravitation}
Das effektive Weber-Potential lautet:
\begin{equation}
V_{\text{WG}} = -\frac{GMm}{r}\left(1 - \frac{v^2}{2c^2} + \beta\frac{\mathbf{r}\cdot\mathbf{a}}{2c^2}\right)
\end{equation}

Die Lagrange-Funktion wird:
\begin{equation}
\mathscr{L}_{\text{WG}} = T - V_{\text{WG}} = \frac{1}{2}mv^2 + \frac{GMm}{r}\left(1 - \frac{v^2}{2c^2} + \beta\frac{\mathbf{r}\cdot\mathbf{a}}{2c^2}\right)
\end{equation}

\subsection*{5. Euler-Lagrange-Gleichungen}
Anwendung der Euler-Lagrange-Gleichung:
\begin{equation}
\frac{d}{dt}\left(\frac{\partial\mathscr{L}}{\partial\mathbf{v}}\right) - \frac{\partial\mathscr{L}}{\partial\mathbf{r}} = 0
\end{equation}

Berechnung der Terme:
\begin{align}
\frac{\partial\mathscr{L}}{\partial\mathbf{v}} &= m\mathbf{v} - \frac{GMm}{c^2r}\mathbf{v} + \beta\frac{GMm}{2c^2}\frac{\mathbf{r}}{r} \\
\frac{d}{dt}\left(\frac{\partial\mathscr{L}}{\partial\mathbf{v}}\right) &= m\mathbf{a} - \frac{GMm}{c^2}\left(\frac{\mathbf{a}}{r} - \frac{\dot{r}\mathbf{v}}{r^2}\right) + \beta\frac{GMm}{2c^2}\left(\frac{\mathbf{v}}{r} - \frac{\dot{r}\mathbf{r}}{r^2}\right) \\
\frac{\partial\mathscr{L}}{\partial\mathbf{r}} &= -\frac{GMm}{r^2}\hat{\mathbf{r}} + \frac{GMm}{2c^2}\left(\frac{v^2}{r^2}\hat{\mathbf{r}} - \beta\frac{\mathbf{a}}{r}\right)
\end{align}

\subsection*{6. De-Broglie-Bohm'sches Quantenpotential}
Das Quantenpotential der DBT ist:
\begin{equation}
Q = -\frac{\hbar^2}{2m}\frac{\nabla^2|\Psi|}{|\Psi|}
\end{equation}

Die quantenmechanische Kraft ergibt sich aus:
\begin{equation}
\mathbf{F}_{\text{Q}} = -\nabla Q
\end{equation}

\subsection*{7. Kombinierte Bewegungsgleichung}
Addition der Weber- und Quantenkräfte führt zu:
\begin{equation}
m\frac{d}{dt}\left[\left(1 - \frac{GM}{c^2r} + \beta\frac{GM}{2c^2}\frac{\mathbf{r}\cdot\mathbf{v}}{r^2}\right)\mathbf{v}\right] = -\frac{GMm}{r^2}\hat{\mathbf{r}} - \nabla Q
\end{equation}

Definition des Weber-Lorentz-Faktors:
\begin{equation}
\gamma_{\text{WG}} \equiv \left(1 - \frac{v^2}{c^2} + \beta\frac{\mathbf{v}\cdot\mathbf{a}}{c^2}\right)^{-1/2} \approx 1 + \frac{v^2}{2c^2} - \beta\frac{\mathbf{v}\cdot\mathbf{a}}{2c^2}
\end{equation}

\subsection*{8. Finale Bewegungsgleichung (\ref{eq:wg_dbt_srt})}
Nach Vernachlässigung höherer Ordnungen erhalten wir:
\begin{equation}
m\frac{d}{dt}(\gamma_{\text{WG}}\mathbf{v}) = -\nabla Q
\end{equation}

\newpage
\section{Vergleich der Weber-Elektrodynamik mit der Maxwell-Theorie}
Es gibt in der Literatur mindestens zwei Weber-Kraft Varianten, hier soll gezeigt werden, weshalb ich mich für diese Variante entschieden habe.

Wir betrachten zwei Punktladungen $q_1$ und $q_2$ mit konstanter Geschwindigkeit $\mathbf{v}_1 = \mathbf{v}_2 = \mathbf{v}$ (gleichförmige Bewegung) und Abstandsvektor $\mathbf{r} = \mathbf{r}_1 - \mathbf{r}_2$.

\subsection{Weber-Elektrodynamik}
Die verallgemeinerte Weber-Kraft für die Kraft auf $q_1$ durch $q_2$ lautet in vektorieller Form:

\subsubsection{Klassische Weber-Kraft (Variante a)}
\begin{equation}
F_W^{(a)} = \frac{q_1 q_2}{4 \pi \epsilon_0 r^2} \left(1 + \frac{v^2}{c^2} + \frac{\mathbf{r} \cdot \mathbf{a}}{c^2} - \frac{3 (\mathbf{r} \cdot \mathbf{v})^2}{2 c^2 r^2}\right)
\end{equation}

\subsubsection{Alternative Weber-Kraft (Variante b)}
\begin{equation}
F_W^{(b)} = \frac{q_1 q_2}{4 \pi \epsilon_0 r^2} \left(1 + \frac{2 v^2}{c^2} + \frac{2 \mathbf{r} \cdot \mathbf{a}}{c^2} - \frac{3 (\mathbf{r} \cdot \mathbf{v})^2}{c^2 r^2}\right)
\end{equation}

Für den Spezialfall paralleler Bewegung ($\mathbf{v} \parallel \mathbf{r}$) mit $\mathbf{a} = 0$ vereinfachen sich diese Ausdrücke zu:
\begin{align}
F_W^{(a)} &= \frac{q_1 q_2}{4 \pi \epsilon_0 r^2} \left(1 - \frac{v^2}{2 c^2}\right) \\
F_W^{(b)} &= \frac{q_1 q_2}{4 \pi \epsilon_0 r^2} \left(1 - \frac{v^2}{c^2}\right)
\end{align}

\subsection{Maxwell-Theorie (Lorentz-Kraft)}
In der Maxwell-Elektrodynamik ergibt sich die Kraft aus der Lorentz-Kraft auf $q_1$:
\begin{equation}
\mathbf{F}_M = q_1 (\mathbf{E}_2 + \mathbf{v}_1 \times \mathbf{B}_2)
\end{equation}

Für eine gleichförmig bewegte Ladung ($\mathbf{v} = \text{const.}$, $\mathbf{a} = 0$) parallel zu $\mathbf{r}$ erhält man:
\begin{equation}
\mathbf{E}_2 = \frac{q_2}{4 \pi \epsilon_0 r^2} \left(1 - \frac{v^2}{c^2}\right) \hat{r}, \quad \mathbf{B}_2 = 0
\end{equation}
Damit wird die Lorentz-Kraft:
\begin{equation}
\mathbf{F}_M = \frac{q_1 q_2}{4 \pi \epsilon_0 r^2} \left(1 - \frac{v^2}{c^2}\right) \hat{r}
\end{equation}

\subsection{Vergleich der Ergebnisse}
\begin{table}[h]
\centering
\begin{tabular}{lc}
\hline
\textbf{Theorie} & \textbf{Kraftformel ($\mathbf{v} \parallel \mathbf{r}$)} \\
\hline
Weber (Variante a) & $F_W^{(a)} = \dfrac{q_1 q_2}{4 \pi \epsilon_0 r^2} \left(1 - \dfrac{v^2}{2 c^2}\right)$ \\
Weber (Variante b) & $F_W^{(b)} = \dfrac{q_1 q_2}{4 \pi \epsilon_0 r^2} \left(1 - \dfrac{v^2}{c^2}\right)$ \\
Maxwell & $\mathbf{F}_M = \dfrac{q_1 q_2}{4 \pi \epsilon_0 r^2} \left(1 - \dfrac{v^2}{c^2}\right) \hat{r}$ \\
\hline
\end{tabular}
\caption{Vergleich der Weber- und Maxwell-Kräfte für parallele Bewegung}
\end{table}

\subsection{Interpretation}
\begin{itemize}
\item Die Weber-Kraft \textbf{(Variante b)} stimmt exakt mit der Maxwell-Theorie für gleichförmige Bewegung ($\mathbf{a} = 0$) überein.
\item Die Weber-Kraft \textbf{(Variante a)} weicht ab (Faktor $1/2$ beim $v^2/c^2$-Term).
\end{itemize}


\begin{thebibliography}{9}
\bibitem{einstein1915} 
Einstein, A. (1915). 
\textit{Die Feldgleichungen der Gravitation}. 
Sitzungsberichte der Preußischen Akademie der Wissenschaften, 
S. 844–847.

\bibitem{shapiro1964} 
Shapiro, I. I. (1964). 
\textit{Fourth Test of General Relativity}. 
Physical Review Letters, 13(26), 789–791.

\bibitem{rubin1970} 
Rubin, V. C., \& Ford, W. K. (1970). 
\textit{Rotation of the Andromeda Nebula from a Spectroscopic Survey of Emission Regions}. 
Astrophysical Journal, 159, 379–403.

\bibitem{weber1846} 
Weber, W. (1846). 
\textit{Elektrodynamische Maassbestimmungen}. 
Leipzig: Weidmannsche Buchhandlung.

\bibitem{bohm1952} 
Bohm, D. (1952). 
\textit{A Suggested Interpretation of the Quantum Theory in Terms of "Hidden" Variables}. 
Physical Review, 85(2), 166–193.

\bibitem{tisserand1894}
Tisserand, F. (1894). 
\textit{Traité de Mécanique Céleste, Tome IV}. 
Gauthier-Villars, Paris. 
(Kapitel 28: "Lois électrodynamiques de Weber appliquées à la gravitation")
\end{thebibliography}

\end{document}
