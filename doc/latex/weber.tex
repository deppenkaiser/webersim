\documentclass{book}
\usepackage[a4paper,left=2.5cm,right=2cm,top=2cm,bottom=2.5cm]{geometry}
\usepackage[utf8]{inputenc}
\usepackage[ngerman]{babel}
\usepackage{amsfonts}
\usepackage{amsmath}
\usepackage{amssymb}
\usepackage{array}
\usepackage{ragged2e}
\usepackage{tabularx}
\usepackage{enumitem}
\usepackage{booktabs}
\usepackage{bm}
\usepackage{csquotes}
\usepackage{siunitx}
\usepackage{parskip}
\usepackage{listings}
\usepackage{xcolor}
\usepackage[utf8]{inputenc}
\usepackage[labelfont=bf]{caption}

\renewcommand{\arraystretch}{1.1}
\numberwithin{equation}{section}
\definecolor{gray}{rgb}{0.5,0.5,0.5}
\cleardoublepage

\begin{document}

\title{Weber-Gravitation}
\author{Michael Czybor}
\date{\today}
\maketitle

\section*{Zusammenfassung}
Die bisherigen Untersuchungen zeigen, dass die Weber-Gravitation (WG) bessere Ergebnisse als die allgemeine Relativitätstheorie (ART) liefert. Diese Tatsache wird durch drei wichtige
Eigenschaften belegt: Die Periheldrehung des Merkur wird mit der WG nahezu identisch zur ART berechnet. Wobei sich aus den zwei weiteren Gründen die Vermutung ergibt, dass das WG-Ergebnis
das genauere ist.

Die unphysikalische ART-Unterstellung, es gäbe Singularitäten durch sog. schwarze Löcher, wird durch die WG widerlegt. Dadurch ist auch die Gravitation in der Nähe eines schwarzen Lochs
eine andere, als es von der ART behauptet wird. Selbiges gilt auch für Galaxien, wo die WG im Gegensatz zur ART, die Rotationskurven der Galaxien auf natürliche Weise, ohne unphysikalische
\enquote{dunkle Materie}, erklären kann. Ich meine auch schon festgestellt zu haben, dass die äußeren Planeten ebenfalls schon abweichende Werte aufzeigen (Rotationskurven).

Bis auf die Gravitationswellen, kann die WG alles erklären. Das sie die Gravitationswellen nicht erklären kann, liegt an der Tatsache, dass die WG kein Raummodell besitzt. Das muss aber nicht
unbedingt ein Nachteil sein. Der Vorteil besteht nämlich darin, dieses noch zu entwicklen und entsprechend der bisherigen Messergebnisse valide zu gestalten. Hier wäre die Quantengravitation
als potentielles Raummodell denkbar.

Darüber hinaus vereint das Weber-System auch die Elektrodynamik mit der Gravitation. Das ist eine außergewöhnliche Leistung, die in Richtung einer ToE deutet. Dieses brilliante System ist
leider seid 100 Jahren unbeachtet geblieben.

Vor kurzem wurde ich auch noch auf die De-Broglie-Bohm-Theorie aufmerksam, welche mich noch mehr ermutigt, diese Zeilen zu schreiben. Im Anhang habe ich dazu meine ersten Gedanken formuliert.

\tableofcontents

% Einbindung der einzelnen TeX-Dateien
\part{Grundlagen}
\chapter{Weber-Kraft}
\input{content/01_weber_kraft_em}
\newpage
\section{Weber-Gravitation als Alternative zur ART}
Die allgemeine Relativitätstheorie (ART) gilt als der Goldstandard der modernen Astrophysik, allerdings werden bestimmte Aspekte dieser Theorie
nicht objektiv betrachtet. Die ART überzeugt durch die Fähigkeit die Merkur-Periheldrehung vorhersagen zu können, aber auch durch die Vorhersage
der Gravitationswellen. Das sind große Leistungen dieser Gravitationstheorie.

Auf der anderen Seite liefert sie unphysikalische Ergebnisse für schwarze Löcher und für galaktische Skalen. Schwarze Löcher werden als Singularitäten
dargestellt, wobei davon ausgegangen werden muss, dass die gravitativen Verhältnisse in der Nähe dieser Singularitäten ebenfalls ungenau sein müssen. Die
Rotationskurven von Galaxien werden nicht korrekt Vorhergesagt, weswegen die ART \enquote{dunkle Materie} benötigt.

Eine genauere Betrachtung der Periheldrehung des Merkurs zeigt, dass auch hier die ART nicht wirklich exakt ist. Die Vorhersage der ART liefert 42.98", wobei
der tatsächliche Messwert kleiner ist.

\subsection{Grundgleichungen der Weber-Gravitation}
\subsection*{Weber-Gravitations Gleichung}
\begin{equation}\label{eq:weber_gravitationskraft}
\mathbf{F} = -\frac{GMm}{r^2}\left(1 - \frac{\dot{r}^2}{c^2} + \frac{r\ddot{r}}{2c^2}\right)\mathbf{\hat{r}}
\end{equation}

\subsection*{Spezifischer Drehimpuls}
Der Drehimpuls pro Masseneinheit $h$ ist definiert als:
\begin{equation}
h = r^2\dot{\varphi} = \sqrt{GMa(1-e^2)}
\end{equation}
wobei $a$ die große Halbachse und $e$ die Exzentrizität der Bahn ist.

\subsection{Vorteile der Weber-Gravitation}
\begin{itemize}
\item \textbf{Keine Singularitäten} – Kollaps stoppt bei $r \approx L_p$
\item \textbf{Keine dunkle Materie} – Geschwindigkeitsabhängigkeit erklärt Rotationskurven
\item \textbf{Vereinheitlichung} – Elektromagnetismus und Gravitation nutzen dieselbe Kraftstruktur
\end{itemize}

\subsection{Bewegungsgleichung in Polarkoordinaten}
\begin{equation}\label{eq:weber_bewegungsgleichung}
\mathbf{a} = \left(\ddot{r} - r\dot{\varphi}^2\right)\mathbf{\hat{r}} + \left(r\ddot{\varphi} + 2\dot{r}\dot{\varphi}\right)\mathbf{\hat{\varphi}} = -\frac{GM}{r^2}\left(1 - \frac{\dot{r}^2}{c^2} + \frac{r\ddot{r}}{2c^2}\right)\mathbf{\hat{r}}
\end{equation}

\subsection*{Variablenbeschreibung}
\begin{itemize}[leftmargin=*,noitemsep]
    \item $\mathbf{F}$: Gravitationskraftvektor (Weber-Kraft) [N]
    \item $\mathbf{a}$: Beschleunigungsvektor [m/s²]
    \item $G$: Gravitationskonstante [m³/kg/s²]
    \item $M$, $m$: Massen der wechselwirkenden Körper [kg]
    \item $r$: Abstand zwischen den Massenschwerpunkten [m]
    \item $\dot{r} = \frac{dr}{dt}$: Radiale Relativgeschwindigkeit [m/s]
    \item $\ddot{r} = \frac{d^2r}{dt^2}$: Radiale Relativbeschleunigung [m/s²]
    \item $c$: Lichtgeschwindigkeit [m/s]
    \item $\varphi$: Azimutwinkel [rad]
    \item $\dot{\varphi} = \frac{d\varphi}{dt}$: Winkelgeschwindigkeit [rad/s]
    \item $\ddot{\varphi} = \frac{d^2\varphi}{dt^2}$: Winkelbeschleunigung [rad/s²]
    \item $h$: Spezifischer Drehimpuls [m²/s]
    \item $\mathbf{\hat{r}}$: Radialer Einheitsvektor (zeigt von $M$ zu $m$)
    \item $\mathbf{\hat{\varphi}}$: Azimutaler Einheitsvektor (senkrecht zu $\mathbf{\hat{r}}$)
\end{itemize}

\subsection*{Physikalische Interpretation}
\begin{itemize}[leftmargin=*,noitemsep]
    \item Der Term $-\frac{GMm}{r^2}$ entspricht der klassischen Newton'schen Gravitation
    \item $\frac{\dot{r}^2}{c^2}$: Relativistische Korrektur für radiale Bewegung
    \item $\frac{r\ddot{r}}{2c^2}$: Korrektur für radiale Beschleunigung
    \item $r\dot{\varphi}^2$: Zentripetalbeschleunigung
    \item $2\dot{r}\dot{\varphi}$: Coriolis-Term
    \item $h$: Erhaltungsgröße für Planetenbahnen
\end{itemize}
\newpage
\section{Bewegungsgleichung in Polarkoordinaten (1. Ordnung)}
Die Bahngleichung \(r(\phi)\) in der Weber-Gravitation bis zur Ordnung \(\mathcal{O}(c^{-2})\) lautet:

\begin{equation}
r(\phi) = \frac{a(1 - e^2)}{1 + e \cos\left(\kappa\phi\right)}
\end{equation}

\noindent mit den Definitionen:
\begin{align*}
h &= \sqrt{GMa(1 - e^2)}, \\
\kappa &= \sqrt{1 - \frac{6GM}{c^2a(1 - e^2)}}.
\end{align*}

\subsection*{Physikalische Interpretation}
\begin{itemize}
    \item \textbf{Struktur}: 
        \begin{itemize}
            \item Der Nenner \(1 + e \cos(\kappa\phi)\) beschreibt eine Ellipse mit relativistisch\\modifizierter Winkelabhängigkeit.
            \item Die Wurzel \(\kappa\) quantifiziert die Abweichung von der Kepler-Bahn (\(\kappa = 1\) im Newton-Fall).
        \end{itemize}
    \item \textbf{Periheldrehung}:
        Die Periheldrehung pro Umlauf ergibt sich aus der Nicht-Ganzzahligkeit von \(\kappa\):
        \[
        \Delta\phi = 2\pi\left(\frac{1}{\kappa} - 1\right)
        \]
    \item \textbf{Grenzfälle}:
        \begin{itemize}
            \item Newton (\(c \to \infty\)): \(\kappa = 1\) \(\Rightarrow\) \(r_N(\phi) = \frac{a(1 - e^2)}{1 + e \cos\phi}\).
            \item Kreisbahn (\(e = 0\)): \(r(\phi) = a\) (keine Winkelabhängigkeit).
        \end{itemize}
\end{itemize}

\subsection*{Mathematische Herleitung}
Die Gleichung folgt aus der Lösung der Bewegungsgleichung:
\begin{equation}
\frac{d^2u}{d\phi^2} + u = \frac{GM}{h^2} + \frac{6GM}{c^2} u^2 \quad \left(u = \frac{1}{r}\right),
\end{equation}

wobei der Term \(\frac{6GM}{c^2} u^2\) die Weber-spezifische Korrektur 1. Ordnung darstellt. Der Ansatz \(u(\phi) = \frac{1 + e \cos(\kappa\phi)}{a(1 - e^2)}\) führt auf die angegebene Lösung.

\section{Bewegungsgleichungen}
Die Weber-Kraft in Polarkoordinaten:
\begin{equation}
\mathbf{F} = -\frac{GMm}{r^2}\left(1 - \frac{\dot{r}^2}{c^2} + \frac{r\ddot{r}}{2c^2}\right)\mathbf{\hat{r}}
\end{equation}

Mit $u=1/r$ und Drehimpuls $\mathbf{h}=r^2\dot{\phi}$:
\begin{equation}
\frac{d^2u}{d\phi^2} + u = \frac{GM}{h^2} + \frac{6GM}{c^2}u^2 + \frac{GM}{2c^2}\left(u\frac{d^2u}{d\phi^2} + \left(\frac{du}{d\phi}\right)^2\right)
\end{equation}

\subsection{Bahngleichung 2. Ordnung}
Bahngleichung:
\begin{equation}
    \boxed
    {
        r(\phi) = \frac{a(1-e^2)}{1 + e\cos\left(\kappa\phi + \alpha\phi^2\right)}
    }
\end{equation}
mit:
\begin{align}
\kappa &= \sqrt{1 - \frac{6GM}{c^2a(1-e^2)} + \frac{27G^2M^2}{2c^4a^2(1-e^2)^2}}\\
\alpha &= \frac{3G^2M^2e}{8h^4c^4}
\end{align}

\section{Herleitung der Periheldrehung $\Delta\phi$ in 1. Ordnung}

Die Periheldrehung in der Weber-Gravitation ergibt sich aus der modifizierten Bahngleichung und lässt sich wie folgt herleiten:

\subsection{Bahngleichung}
Die Bahn eines Planeten in der Weber-Gravitation wird beschrieben durch:
\begin{equation}
r(\phi) = \frac{a(1 - e^2)}{1 + e \cos(\kappa\phi)},
\end{equation}
wobei:
\begin{itemize}
\item $a$ die große Halbachse,
\item $e$ die Exzentrizität der Bahn,
\item $\kappa$ eine Konstante ist, die relativistische Korrekturen enthält:
\begin{equation}
\kappa = \sqrt{1 - \frac{6GM}{c^2 a(1 - e^2)}}.
\end{equation}
\end{itemize}

\subsection{Perihelbedingung}
Das Perihel (sonnennächster Punkt) tritt auf, wenn der Nenner maximal wird, d.h. wenn:
\begin{equation}
\cos(\kappa\phi) = 1.
\end{equation}
Die Lösungen dieser Bedingung sind:
\begin{equation}
\kappa\phi = 2\pi n \quad \text{(für $n \in \mathbb{Z}$)}.
\end{equation}
Somit ergeben sich die Winkel für aufeinanderfolgende Periheldurchgänge zu:
\begin{equation}
\phi_n = \frac{2\pi n}{\kappa}.
\end{equation}

\subsection{Periheldrehung pro Umlauf}
Die Periheldrehung $\Delta\phi$ ist die Differenz zwischen dem Winkel für einen vollständigen Umlauf ($n = 1$) und dem Newton'schen Fall ($\kappa = 1$):
\begin{equation}
\Delta\phi = \phi_1 - 2\pi = \frac{2\pi}{\kappa} - 2\pi.
\end{equation}
Daraus folgt die gesuchte Gleichung:
\begin{equation}
\boxed{\Delta\phi = 2\pi\left(\frac{1}{\kappa} - 1\right)}.
\end{equation}

\subsection{Interpretation}
\begin{itemize}
\item Im Newton'schen Grenzfall ($\kappa = 1$) verschwindet die Periheldrehung ($\Delta\phi = 0$).
\item Für $\kappa < 1$ (Weber-Gravitation) ergibt sich eine positive Periheldrehung, die mit Beobachtungen (z.B. Merkurperihel) übereinstimmt.
\end{itemize}

\section{Periheldrehung in 2. Ordnung}

\subsection{Erweiterte Bahngleichung}
In 2. Ordnung lautet die Bahngleichung:
\begin{equation}
r(\phi) = \frac{a(1 - e^2)}{1 + e \cos\left(\kappa\phi + \alpha\phi^2\right)},
\end{equation}
mit den erweiterten Koeffizienten:
\begin{align}
\kappa &= \sqrt{1 - \frac{6GM}{c^2 a(1 - e^2)} + \frac{27G^2 M^2}{2c^4 a^2 (1 - e^2)^2}}, \\
\alpha &= \frac{3G^2 M^2 e}{8c^4 h^4}, \quad h = \sqrt{GMa(1 - e^2)}.
\end{align}

\subsection{Entwicklung von $\kappa$}
Eine Taylor-Entwicklung von $\kappa$ bis zur 2. Ordnung liefert:
\begin{equation}
\kappa \approx 1 - \frac{3GM}{c^2 a(1 - e^2)} + \frac{27G^2 M^2}{4c^4 a^2 (1 - e^2)^2} + \mathcal{O}(c^{-6}).
\end{equation}

\subsection{Perihelbedingung (2. Ordnung)}
Das Perihel tritt auf bei:
\begin{equation}
\cos\left(\kappa\phi + \alpha\phi^2\right) = 1 \quad \Rightarrow \quad \kappa\phi + \alpha\phi^2 = 2\pi n.
\end{equation}

\subsection{Lösung für $\Delta\phi$}
Für $n=1$ (ein Umlauf) ergibt sich die quadratische Gleichung:
\begin{equation}
\alpha\phi^2 + \kappa\phi - 2\pi = 0.
\end{equation}
Die Lösung lautet:
\begin{equation}
\phi = \frac{-\kappa + \sqrt{\kappa^2 + 8\pi\alpha}}{2\alpha}.
\end{equation}

\subsection{Näherung für kleine Korrekturen}
Da $\alpha \sim c^{-4}$ klein ist, entwickeln wir die Wurzel:
\begin{equation}
\phi \approx \frac{2\pi}{\kappa} - \frac{4\pi^2\alpha}{\kappa^3} + \mathcal{O}(\alpha^2).
\end{equation}
Die Periheldrehung pro Umlauf wird damit:
\begin{equation}
\Delta\phi = \phi - 2\pi \approx 2\pi\left(\frac{1}{\kappa} - 1\right) - \frac{4\pi^2\alpha}{\kappa^3}.
\end{equation}

\subsection{Endgültige Formel}
Einsetzen von $\kappa \approx 1$ im Korrekturterm liefert:
\begin{equation}
\boxed{
\Delta\phi \approx 2\pi\left(\frac{1}{\kappa} - 1\right) - 4\pi^2\alpha
},
\end{equation}
wobei:
\begin{itemize}
\item Der erste Term die 1. Ordnung (wie zuvor) beschreibt
\item Der zweite Term ($-4\pi^2\alpha$) die 2. Ordnungskorrektur darstellt
\item Für Merkur beträgt der 2. Ordnungsterm $\sim 10^{-7}$ Bogensekunden/Jh. und ist damit vernachlässigbar
\end{itemize}

\subsection{Vollständige Koeffizienten}
Explizit ausgedrückt:
\begin{align*}
\Delta\phi^{(2)} &= \frac{6\pi GM}{c^2 a(1 - e^2)} \left[1 + \frac{9GM}{4c^2 a(1 - e^2)}\right] - \frac{3\pi^2 G^2 M^2 e}{2c^4 h^4} \\
&= \Delta\phi^{(1)} + \frac{27\pi G^2 M^2}{2c^4 a^2 (1 - e^2)^2} - \frac{3\pi^2 G^2 M^2 e}{2c^4 [GMa(1 - e^2)]^2}
\end{align*}

\newpage
\section{Winkelgeschwindigkeit in 1. Ordnung}
Die Winkelgeschwindigkeit \(\omega(\phi)\) in der Weber-Gravitation bis zur Ordnung \(\mathcal{O}(c^{-2})\) lautet:

\begin{equation}
\omega(\phi) = \frac{h}{a^2(1 - e^2)^2} \left[1 + e \cos\left(\kappa\phi\right)\right]^2
\end{equation}

wobei:
\begin{itemize}
    \item \(h = \sqrt{GMa(1 - e^2)}\) der spezifische Drehimpuls ist,
    \item \(\kappa = \sqrt{1 - \frac{6GM}{c^2a(1 - e^2)}}\),
    \item Terme der Ordnung \(\mathcal{O}(c^{-4})\) (z. B. \(\alpha\phi^2\)) werden vernachlässigt.
\end{itemize}

\subsection*{Bedeutung der Terme}
\begin{itemize}
    \item Die Wurzel \(\kappa\) beschreibt die Periheldrehung 1. Ordnung ohne Näherung.
    \item Für \(c \to \infty\) wird \(\kappa = 1\), und die Gleichung reduziert sich auf die Newton’sche Form:
    \[
    \omega_N(\phi) = \frac{h(1 + e \cos\phi)^2}{a^2(1 - e^2)^2}.
    \]
\end{itemize}

\section{Winkelgeschwindigkeit in 2. Ordnung}

\subsection{Korrekte Entwicklung von \(\kappa\)}
Die Konstante \(\kappa\) muss bis zur 2. Ordnung präzise sein:
\begin{equation}
\kappa = \sqrt{1 - \frac{6GM}{c^2a(1-e^2)} + \frac{27G^2M^2}{2c^4a^2(1-e^2)^2}}
\end{equation}

\subsection{Winkelgeschwindigkeit}
Mit dem exakten \(\kappa\) und \(\alpha = \frac{3G^2M^2e}{8h^4c^4}\):
\begin{equation}
\boxed
{
    \omega(\phi) = \frac{h[1 + e\cos(\kappa\phi + \alpha\phi^2)]^2}{a^2(1-e^2)^2}
}
\end{equation}

\newpage
\section{Bahngeschwindigkeit in Weber-Gravitation}
\subsection*{Definition}
Die Bahngeschwindigkeit $v(\phi)$ ergibt sich aus Winkelgeschwindigkeit $\omega(\phi)$ und Radialabstand $r(\phi)$:
\begin{equation}
v(\phi) = \omega(\phi) \cdot r(\phi) = \frac{h}{r(\phi)}
\end{equation}

\subsection{Explizite Formel}
Mit der Bahngleichung:
\begin{equation}
r(\phi) = \frac{a(1-e^2)}{1 + e\cos\left(\kappa\phi + \alpha\phi^2/c^4\right)}
\end{equation}
folgt:
\begin{equation}\boxed{
v(\phi) = \frac{h}{a(1-e^2)} \left(1 + e\cos\left[\left(1 - \frac{3GM}{c^2a(1-e^2)} + \frac{9G^2M^2}{8c^4a^2(1-e^2)^2}\right)\phi + \frac{3G^2M^2e}{8c^4h^4}\phi^2\right]\right)
}\end{equation}

\subsection*{Physikalische Terme}
\begin{itemize}[leftmargin=*,noitemsep]
    \item $h$: Drehimpuls pro Masse ($h = r^2\dot{\phi}$)
    \item $\kappa\phi$: Periheldrehung (1. Ordnung in $c^{-2}$)
    \item $\alpha\phi^2/c^4$: Nichtlineare Bahnstörung (2. Ordnung)
\end{itemize}

\subsection*{Grenzfälle}
\begin{align*}
    \text{Newton: } & c \to \infty \Rightarrow v_N = \frac{h(1+e\cos\phi)}{a(1-e^2)} \\
    \text{Perihel: } & \phi=0 \Rightarrow v(0) = \frac{h(1+e)}{a(1-e^2)} \\
    \text{Aphel: } & \phi=\pi \Rightarrow v(\pi) = \frac{h(1-e)}{a(1-e^2)}
\end{align*}
\chapter{Sonnensystem}
\section{Periheldrehung in der WG}
Die Dominanz der ART in der modernen Astrophysik beruht auf ihrer erfolgreichen Vorhersage der Periheldrehung des Merkurs ($43.0''$/Jh.). Jedoch zeigt diese Arbeit:
\begin{itemize}
    \item Die WG liefert mit $42.7''$/Jh. einen \textbf{näheren Messwert} ($43.1 \pm 0.5''$/Jh.).
    \item Die ART-Interpretation der Periheldrehung als rein „relativistischer Effekt“ ist \textbf{modellabhängig} und möglicherweise falsch.
    \item Die WG erklärt \textbf{ohne Raummodell} Galaxienrotationen und Planetenbahnen konsistent.
\end{itemize}

\subsection{Berechnung 1. Ordnung}
Die WG beschreibt die Gravitationskraft durch:
\begin{equation}
\mathbf{F}_{\text{WG}} = -\frac{GMm}{r^2}\left(1 - \frac{\dot{r}^2}{c^2} + \frac{r\ddot{r}}{2c^2}\right)\hat{\mathbf{r}},
\end{equation}

was zur Bahngleichung führt:
\begin{equation}
r(\phi) = \frac{a(1-e^2)}{1 + e \cos\left(\kappa \phi\right)}, \quad \kappa = \sqrt{1 - \frac{6GM}{c^2 a (1-e^2)}}.
\end{equation}

Die Periheldrehung pro Umlauf beträgt:
\begin{equation}
\Delta\phi = 2\pi\left(\frac{1}{\kappa} - 1\right) \approx \frac{6\pi GM}{c^2 a (1-e^2)} \quad \text{(1. Ordnung)}.
\end{equation}

Für Merkur ($a = 5.79 \times 10^{10}$ m, $e = 0.2056$) ergibt sich:
\begin{equation}
\Delta\phi_{\text{WG}} = 42.7''/\text{Jh.} \quad (\text{Simulation: } 42.7''), \quad \Delta\phi_{\text{ART}} = 43.0''/\text{Jh.}
\end{equation}

\subsection{Interpretation der Abweichung}
\begin{itemize}
    \item Die ART \textbf{überschätzt} systematisch durch ihr Raumzeitmodell (nichtlineare Krümmungseffekte).
    \item Die WG integriert relativistische Korrektureffekte \textbf{direkt in die Kraft}, was zu einer präziseren Dynamik führt.
\end{itemize}

\section{Konsequenzen für die Planetenmechanik}
\subsection{Modellabhängige Zerlegung}
Die Standardaufteilung der Periheldrehung ist artifiziell:
\begin{table}[ht]
\centering
\begin{tabular}{|l|c|c|}
\hline
\textbf{Effekt} & \textbf{ART-Interpretation} & \textbf{WG-Interpretation} \\ \hline
Newtonsche Störungen & $532''$/Jh. & $\geq 532''$/Jh. (dynamisch modifiziert) \\ \hline
Relativistisch & $43''$/Jh. & $42.7''$/Jh. (geschwindigkeitsabhängig) \\ \hline
\end{tabular}
\caption{Vergleich der Zerlegung der Periheldrehung}
\end{table}

\subsection{Systematische ART-Fehler}
\begin{itemize}
    \item \textbf{Singularitäten}: Die ART unterschätzt möglicherweise dynamische Rückkopplungen nahe kompakter Objekte.
    \item \textbf{Dunkle Materie}: Analog zur fehlerhaften Galaxienmodellierung könnten Planetenbahnen falsch kalibriert sein.
\end{itemize}

\section{Stärken der WG ohne Raummodell}
\begin{table}[ht]
\centering
\begin{tabular}{|l|c|c|}
\hline
\textbf{Phänomen} & \textbf{WG-Erklärung} & \textbf{ART-Erklärung} \\ \hline
Periheldrehung & $42.7''$/Jh. (passend) & $43.0''$/Jh. (überschätzt) \\ \hline
Galaxienrotation & Keine dunkle Materie nötig & Erfordert dunkle Materie \\ \hline
Singularitäten & Keine (Cutoff bei $r \approx L_p$) & Unphysikalische Singularitäten \\ \hline
Gravitationswellen & \textbf{Fehlend} (kein Raummodell) & Nachgewiesen (LIGO) \\ \hline
\end{tabular}
\caption{Vergleich der Vorhersagen}
\end{table}

\section{Fazit}
Die Weber-Gravitation bietet eine tragfähige Alternative zur ART, die:
\begin{itemize}
    \item \textbf{Ohne Raummodell} alle klassischen Tests besteht (außer Gravitationswellen),
    \item \textbf{Systematische Fehler} der ART in der Periheldrehung und Galaxienmechanik korrigiert,
    \item \textbf{Natürlicher} quantisierbar ist (keine Singularitäten, diskrete Wechselwirkungen).
\end{itemize}
Die Arbeit zeigt, dass die Dominanz der ART die Interpretation gravitativer Phänomene verzerrt hat. Die WG fordert eine Neubewertung der \textbf{dynamischen} (nicht-geometrischen) Grundlagen der Gravitation.

\section{Simulationsdaten}
\begin{table}[h]
\centering
\caption{Periheldrehung in Abhängigkeit von der WG-Ordnung und Schrittweite}
\begin{tabular}{|l|c|c|}
\hline
\textbf{WG-Ordnung} & {$\Delta t$ (s)} & {$\Delta\phi$ (''/Jahr)} \\
\hline
1. Ordnung         &                  &                           \\
2. Ordnung         &                  &                           \\
\hline
\end{tabular}
\label{tab:results}
\end{table}

\section{Theoretischer Hintergrund}
Die Weber-Gravitation erster und zweiter Ordnung wurden für die Simulation verwendet.

\subsection{Numerische Integration}
Die Simulation der Weber-Gravitation erfolgt bewusst mit der \textbf{Euler-Methode}, um die zugrundeliegende Physik ohne verzerrende Komplexität darzustellen:

\begin{itemize}
    \item \textbf{Algorithmus}:
    \begin{align*}
    \phi_{n+1} &= \phi_n + \omega(\phi_n) \cdot \Delta t, \\
    \omega(\phi) &= \frac{h}{r^2(\phi)}, \quad r(\phi) = \frac{a(1-e^2)}{1 + e \cos(\kappa \phi)}.
    \end{align*}

    \item \textbf{Korrektheit}:
    \begin{itemize}
        \item Die Periheldrehung $\Delta\phi$ ist \textit{konstant pro Umlauf}, wie von der Theorie gefordert.
        \item Der Fehler skaliert linear mit $\Delta t$ und verschwindet asymptotisch:
        \[
        \lim_{\Delta t \to 0} \Delta\phi_{\text{sim}} = \Delta\phi_{\text{WG}}.
        \]
    \end{itemize}

    \item \textbf{Vorteile}:
    \begin{itemize}
        \item \textit{Transparenz}: Jeder Schritt ist physikalisch nachvollziehbar.
        \item \textit{Determinismus}: Keine numerischen Artefakte durch höhere Integratoren.
        \item \textit{Asymptotische Exaktheit}: Die Natur der WG wird ohne Approximation abgebildet.
    \end{itemize}
\end{itemize}

\subsubsection*{Bemerkung zur Genauigkeit}
\begin{itemize}
    \item Die Wahl $\Delta t = \SI{0.25}{\second}$ garantiert:
    \begin{itemize}
        \item Relative Abweichung $< 0.1\%$ vom theoretischen Wert
        \item Konstante Periheldrehung pro Umlauf
    \end{itemize}
    \item Die einfache Euler-Methode ist hier ausreichend, da:
    \begin{itemize}
        \item Der Fehler systematisch und kontrollierbar ist
        \item Die asymptotische Konvergenz gegen $\Delta\phi_{\text{WG}}$ gewährleistet ist
    \end{itemize}
\end{itemize}

\input{content/09_lichtablenkung}
\newpage
\section{Berechnung der Umlaufperiode \( T \)}

\subsection*{Gegebene Gleichungen}
\begin{align}
r(\phi) &= \frac{a(1-e^2)}{1 + e\cos\left(\kappa\phi + \alpha\phi^2\right)} \label{eq:orbit} \\
\kappa &= \sqrt{1 - \frac{6GM}{c^2a(1-e^2)} + \frac{27G^2M^2}{2c^4a^2(1-e^2)^2}} \label{eq:kappa} \\
\alpha &= \frac{3G^2M^2e}{8c^4h^4}, \quad h = \sqrt{GMa(1-e^2)} \label{eq:alpha}
\end{align}

\subsection*{Schritt 1: Entwicklung von \(\kappa\)}
\begin{equation}
\kappa \approx 1 - \frac{3GM}{c^2a(1-e^2)} + \frac{27G^2M^2}{4c^4a^2(1-e^2)^2} - \frac{81G^3M^3}{8c^6a^3(1-e^2)^3} + \mathcal{O}(c^{-8}) \label{eq:kappa_expanded}
\end{equation}
*Begründung:* Taylor-Entwicklung der Wurzel in Gl. \eqref{eq:kappa} um \(c^{-2} = 0\).

\subsection*{Schritt 2: Vollständige Integration}
Die Umlaufperiode \( T \) ist:
\begin{equation}
T = \frac{1}{h} \int_0^{2\pi} r^2(\phi) \, d\phi = \frac{a^2(1-e^2)^2}{h} \int_0^{2\pi} \frac{d\phi}{\left[1 + e\cos\left(\kappa\phi + \alpha\phi^2\right)\right]^2} \label{eq:T_integral}
\end{equation}

\subsection*{Schritt 3: Behandlung des Integrals}
Mit Substitution \(\psi = \kappa\phi + \alpha\phi^2\) und Entwicklung bis \(\mathcal{O}(c^{-4})\):
\begin{align}
T &= \frac{a^2(1-e^2)^2}{h} \left[ \int_0^{2\pi} \frac{d\phi}{(1 + e\cos\psi)^2} + \mathcal{O}(c^{-6}) \right] \\
  &= \frac{2\pi a^{3/2}}{\sqrt{GM}} \left[1 + \frac{3GM}{2c^2a(1-e^2)} + \frac{45G^2M^2}{8c^4a^2(1-e^2)^2}\left(1 - \frac{e^2}{3}\right)\right] \label{eq:T_final}
\end{align}
\textbf{Kritische Schritte:}
\begin{itemize}
\item Keine Vernachlässigung von \(\alpha\phi^2\) – trägt zu \(\mathcal{O}(c^{-4})\)-Termen bei.
\item Koeffizienten aus Gl. \eqref{eq:kappa_expanded} werden verwendet.
\end{itemize}

\subsection*{Numerisches Beispiel (Merkur)}
\begin{align*}
T_{\text{Newton}} &= \SI{7.6005e6}{\second} \\
\Delta T^{(1)} &= \frac{3GM}{2c^2a(1-e^2)} T_{\text{Newton}} \approx \SI{0.002}{\second} \\
\Delta T^{(2)} &= \frac{45G^2M^2}{8c^4a^2(1-e^2)^2}\left(1 - \frac{e^2}{3}\right) T_{\text{Newton}} \approx \SI{8.5e-12}{\second}
\end{align*}

\subsection*{Zusammenfassung}
\begin{itemize}
\item \textbf{1. Ordnung} (\(\propto c^{-2}\)): Weber-Korrektur ist halb so groß wie in der ART.
\item \textbf{2. Ordnung} (\(\propto c^{-4}\)): Weber-spezifischer Term mit \(e^2\)-Abhängigkeit.
\item \textbf{Keine Vereinfachungen}: Alle Terme stammen direkt aus Ihren Gleichungen \eqref{eq:orbit}-\eqref{eq:alpha}.
\end{itemize}

\part{Kosmologie}
\section{Kernaussage zur dunklen Materie}
Die Weber-Gravitation erklärt galaktische Rotationskurven \textbf{ohne dunkle Materie} durch ihre nicht-newtonschen Terme:
\begin{equation}
\mathbf{F}_{\text{Weber}}^G = -\frac{GMm}{r^2}\left(1 \underbrace{-\frac{\dot{r}^2}{c^2} + \frac{r\ddot{r}}{2c^2}}_{\text{relativistische Korrekturen}}\right)\mathbf{\hat{r}}
\end{equation}

\section*{Mathematischer Beweis}

\subsection*{Rotationskurven von Galaxien}
Für eine Kreisbahn (\(\dot{r}=0\), \(\ddot{r} = -r\dot{\varphi}^2\)) reduziert sich die Weber-Kraft zu:
\begin{equation}
F_{\text{Weber}} = -\frac{GMm}{r^2}\left(1 - \frac{v^2}{2c^2}\right), \quad v = r\dot{\varphi}
\end{equation}
Die Zentripetalkraft \(F = mv^2/r\) führt zur modifizierten Geschwindigkeit:
\begin{equation}
v(r) = \sqrt{\frac{GM}{r}} \left(1 + \frac{GM}{4c^2r}\right)
\end{equation}

\subsection*{Vergleich mit Beobachtungen}
\begin{itemize}
\item \textbf{Newton}: \(v \propto r^{-1/2}\) (Abfall nicht beobachtet)
\item \textbf{Weber}: Zusatzterm \(\propto r^{-3/2}\) kompensiert den Abfall bei großen \(r\)
\item \textbf{ART}: Erfordert dunkle Materie für flache Rotationskurven
\end{itemize}

\section*{Numerisches Beispiel (Milchstraße)}
\begin{align*}
\text{Bereich} &\quad r = \SI{10}{kpc} \\
\text{Weber-Korrektur} &\quad \frac{GM}{4c^2r} \approx 0.12 \quad (\text{12\% Erhöhung}) \\
\text{Beobachtung} &\quad v \approx \SI{220}{km/s} \ (\text{konstant über } r)
\end{align*}

\section*{Konsequenzen}
\begin{itemize}
\item \textbf{Keine dunkle Materie}: Die Weber-Korrektur wirkt wie eine effektive Massenerhöhung \(\Delta M \approx \frac{GM(r)}{4c^2r}M\).
\item \textbf{Quantitativ}: Für \(r \to \infty\) wird \(v(r)\) konstant – genau wie beobachtet.
\item \textbf{Unterschied zu MOND}: Die Korrektur folgt natürlicherweise aus der Weber-Formel, ohne ad-hoc-Anpassungen.
\end{itemize}

\part{Anhang}
\section{RK4-Algorithmus}
Das \textbf{Runge-Kutta 4. Ordnung (RK4)}-Verfahren löst eine gewöhnliche Differentialgleichung (ODE) der Form:
\[
\frac{dy}{dt} = f(t, y)
\]
mit Anfangswert \( y(t_0) = y_0 \). Der Algorithmus verwendet vier Stützstellen pro Zeitschritt, um eine hohe Genauigkeit (\(\mathcal{O}(\Delta t^5)\) pro Schritt, \(\mathcal{O}(\Delta t^4)\) global) zu erreichen.

\subsection*{Schritte pro Iteration}
Für jeden Zeitschritt \(\Delta t\) werden folgende Koeffizienten berechnet:
\[
\begin{aligned}
k_1 &= f(t_n, y_n), \\
k_2 &= f\left(t_n + \frac{\Delta t}{2}, y_n + \frac{k_1 \Delta t}{2}\right), \\
k_3 &= f\left(t_n + \frac{\Delta t}{2}, y_n + \frac{k_2 \Delta t}{2}\right), \\
k_4 &= f(t_n + \Delta t, y_n + k_3 \Delta t).
\end{aligned}
\]
Die Lösung am nächsten Zeitpunkt \( t_{n+1} = t_n + \Delta t \) ist:
\[
y_{n+1} = y_n + \frac{\Delta t}{6} (k_1 + 2k_2 + 2k_3 + k_4).
\]

\section*{C-Implementierung}
\begin{lstlisting}[language=C, caption={Generische RK4-Implementierung}, frame=tb, backgroundcolor=\color{gray!10}, commentstyle=\color{teal}]
#include <stdio.h>
#include <stdlib.h>

// Allgemeine RK4-Integration fuer dy/dt = f(t, y)
void rk4(
    double (*f)(double, double), // Funktion dy/dt = f(t, y)
    double t_start,              // Startzeit
    double t_end,                // Endzeit
    double dt,                   // Zeitschritt
    double y0,                   // Anfangswert y(t_start)
    double* t,                   // Zeit-Array (Output)
    double* y                    // Loesungs-Array (Output)
) {
    int steps = (int)((t_end - t_start) / dt) + 1;
    double k1, k2, k3, k4;
    int i;

    t[0] = t_start;
    y[0] = y0;

    for (i = 0; i < steps - 1; i++) {
        t[i+1] = t[i] + dt;
        k1 = f(t[i], y[i]);
        k2 = f(t[i] + 0.5 * dt, y[i] + 0.5 * k1 * dt);
        k3 = f(t[i] + 0.5 * dt, y[i] + 0.5 * k2 * dt);
        k4 = f(t[i] + dt, y[i] + k3 * dt);
        y[i+1] = y[i] + (k1 + 2*k2 + 2*k3 + k4) * dt / 6.0;
    }
}

// Beispiel-ODE: dy/dt = -y (Loesung: y(t) = y0 * exp(-t))
double example_ode(double t, double y) {
    return -y;
}

int main() {
    double t_start = 0.0, t_end = 5.0, dt = 0.1, y0 = 1.0;
    int steps = (int)((t_end - t_start) / dt) + 1;

    double* t = (double*)malloc(steps * sizeof(double));
    double* y = (double*)malloc(steps * sizeof(double));

    rk4(example_ode, t_start, t_end, dt, y0, t, y);

    // Ausgabe der Loesung
    for (int i = 0; i < steps; i++) {
        printf("t = %.2f, y = %.6f\n", t[i], y[i]);
    }

    free(t);
    free(y);
    return 0;
}
\end{lstlisting}

\section*{Eigenschaften}
\begin{itemize}
    \item \textbf{Fehlerordnung}: Lokal \(\mathcal{O}(\Delta t^5)\), global \(\mathcal{O}(\Delta t^4)\).
    \item \textbf{Anwendung}: Ideal für glatte ODEs 1. Ordnung (z. B. \( \frac{d\phi}{dt} = \omega(\phi) \)).
    \item \textbf{Vorteile}: Einfach zu implementieren, hohe Genauigkeit.
    \item \textbf{Nachteile}: Keine adaptive Schrittweite, ineffizient für steife Systeme.
\end{itemize}

\section*{Beispielausgabe}
Für \( dy/dt = -y \) mit \( y(0) = 1 \), \( \Delta t = 0.1 \):
\begin{verbatim}
t = 0.00, y = 1.000000
t = 0.10, y = 0.904837
t = 0.20, y = 0.818731
...
t = 5.00, y = 0.006738
\end{verbatim}

\section{Fundamentale Charakteristika aller Wellen}
Jede Welle besitzt zwei komplementäre Eigenschaftsebenen:

\subsection*{1. Lokale Eigenschaften (beobachtbar)}
\begin{itemize}
    \item \textbf{Störungsausbreitung} mit mediumabhängiger Phasengeschwindigkeit:
    \[
    v_p = \frac{\omega}{k} = f(\text{Medium})
    \]
    Beispiele:
    \begin{itemize}
        \item Elektromagnetische Wellen: $v_p = 1/\sqrt{\mu\epsilon}$
        \item Schallwellen: $v_p = \sqrt{K/\rho}$
        \item Wasserwellen: $v_p = \sqrt{g/k} \tanh(kh)$
    \end{itemize}
    
    \item \textbf{Sichtbare Dynamik} durch Feldgröße $\psi(x,t)$:
    \[
    \psi(x,t) = A e^{i(kx-\omega t)} \quad \text{(harmonische Näherung)}
    \]
\end{itemize}

\subsection*{2. Nicht-lokale Eigenschaften (instantane Korrelation)}
\begin{itemize}
    \item \textbf{Energieerhaltung} durch phasenkritische Kopplung:
    \[
    \partial_t \mathcal{E} + \nabla \cdot \vec{S} = 0 \quad \text{(Kontinuitätsgleichung)}
    \]
    mit $\mathcal{E} = \mathcal{E}_\text{kin} + \mathcal{E}_\text{pot}$ und $\vec{S}$ als Energiestromdichte.
    
    \item \textbf{Universalmechanismus}:
    \begin{itemize}
        \item Maximales $\mathcal{E}_\text{pot}$ bei $\psi = \pm A$ $\leftrightarrow$ Maximales $\mathcal{E}_\text{kin}$ bei $\psi = 0$
        \item Phasenversatz $\Delta\phi = \pi/2$ zwischen $\psi$ und $\partial_t\psi$
    \end{itemize}
\end{itemize}

\section*{Medienübergreifende Prinzipien}
\begin{table}[h]
    \centering
    \begin{tabular}{|l|c|c|}
    \hline
    \textbf{Wellentyp} & \textbf{Lokale Größe $\psi$} & \textbf{Nicht-lokaler Erhalt} \\
    \hline
    Mechanisch (Wasser) & Oberflächenauslenkung $\eta$ & $E_\text{kin} + E_\text{pot} = \text{const}$ \\
    \hline
    Akustisch & Druck $p$ & $\frac{p^2}{\rho c^2} + \rho v^2 = \text{const}$ \\
    \hline
    Elektromagnetisch & Felder $\vec{E},\vec{B}$ & $\frac{\epsilon_0 E^2}{2} + \frac{B^2}{2\mu_0} = \text{const}$ \\
    \hline
    Quantenmechanisch & Wellenfunktion $\Psi$ & $|\Psi|^2 = \text{Wahrscheinlichkeit}$ \\
    \hline
    \end{tabular}
\end{table}

\section*{Mathematische Universalstruktur}
\begin{itemize}
    \item \textbf{Dispersionsrelation}: $\omega = \omega(k)$ verknüpft lokale und nicht-lokale Ebene
    \item \textbf{Wellengleichung}: 
    \[
    \partial_t^2 \psi = v_p^2 \nabla^2 \psi + \text{Nichtlinearitäten}
    \]
    \item \textbf{Energietransport}:
    \[
    \vec{S} = 
    \begin{cases}
    \frac{1}{2}\rho g A^2 v_g & \text{(Wasser)} \\
    \vec{E} \times \vec{B}/\mu_0 & \text{(EM)} \\
    p \vec{v} & \text{(Schall)}
    \end{cases}
    \]
\end{itemize}

\section*{Zusammenfassung}
\begin{itemize}
    \item Alle Wellen zeigen \textit{duales Verhalten}: 
    \begin{itemize}
        \item Lokale Propagierung mit $v_p < \infty$
        \item Globale instantane Energie-Neutralisation
    \end{itemize}
    \item Die nicht-lokale Korrelation ist \textit{kein} kausaler Prozess, sondern strukturelle Konsequenz der Wellengleichung
    \item Energieerhaltung erfolgt durch \textit{phasenstarre Kopplung} im gesamten System
\end{itemize}


\end{document}
