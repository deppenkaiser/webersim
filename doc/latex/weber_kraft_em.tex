\section{Klassische Weber-Kraft (Elektrodynamik)}
\[ \bm{F}_{\text{Weber}}^{\text{EM}} = \frac{Qq}{4\pi\epsilon_0 r^2}\left(1 - \frac{\dot{r}^2}{c^2} + \frac{2r\ddot{r}}{c^2}\right)\bm{\hat{r}} \]

\subsection*{Symbolbeschreibung}
\begin{itemize}[leftmargin=*,noitemsep]
    \item $\bm{F}_{\text{Weber}}^{\text{EM}}$: Weber-Kraft zwischen Ladungen
    \item $Q, q$: Elektrische Ladungen
    \item $\epsilon_0$: Elektrische Feldkonstante
    \item $r$: Ladungsabstand
    \item $\dot{r} = \frac{dr}{dt}$: Relative Radialgeschwindigkeit
    \item $\ddot{r} = \frac{d^2r}{dt^2}$: Relative Radialbeschleunigung
    \item $c$: Lichtgeschwindigkeit
    \item $\bm{\hat{r}}$: Radialer Einheitsvektor
\end{itemize}

\subsection*{Beziehung zur Coulomb-Kraft}
\begin{itemize}[leftmargin=*,noitemsep]
    \item Erster Term entspricht Coulomb-Kraft: $\frac{Qq}{4\pi\epsilon_0 r^2}$
    \item Zusatzterme $\left(-\frac{\dot{r}^2}{c^2} + \frac{2r\ddot{r}}{c^2}\right)$ beschreiben Bewegungsabhängige Korrekturen
    \item Reduktion auf Coulomb-Kraft im statischen Fall ($\dot{r} = \ddot{r} = 0$)
\end{itemize}

\subsection*{Vergleich mit Maxwell-Theorie}
\begin{itemize}[leftmargin=*,noitemsep]
    \item Alternative Beschreibung elektromagnetischer Phänomene
    \item Fernwirkungsansatz (direkte Ladungswechselwirkung)
    \item Implizite Retardierung durch Geschwindigkeits-/Beschleunigungsterme
    \item Keine Vorhersage von EM-Wellen im Vakuum
\end{itemize}

\section{Maxwell und ART: Wellen und Raummodelle}

\subsection{Maxwell im flachen Raum}
\begin{itemize}[leftmargin=*,noitemsep]
    \item Wellengleichung im Vakuum:
    \[ \left(\nabla^2 - \frac{1}{c^2}\partial_t^2\right)\bm{E} = 0 \]
    \item Raummodell: Minkowski-Raum $\mathbb{R}^{3,1}$ mit $\eta_{\mu\nu}$
    \item Lichtausbreitung: Geradlinig mit $c = 1/\sqrt{\mu_0\epsilon_0}$
\end{itemize}

\subsection{ART in gekrümmter Raumzeit}
\begin{itemize}[leftmargin=*,noitemsep]
    \item Einstein-Gleichungen:
    \[ G_{\mu\nu} = \frac{8\pi G}{c^4}T_{\mu\nu} \]
    \item Lichtausbreitung: Nullgeodäten ($ds^2 = 0$)
    \item Konsequenzen:
    \begin{enumerate}[noitemsep]
        \item Gravitative Lichtablenkung
        \item Shapiro-Verzögerung
        \item Gravitative Rot-/Blauverschiebung
    \end{enumerate}
\end{itemize}

\begin{table}[h]
\centering
\caption{Vergleich Maxwell und ART}
\begin{tabular}{ll}
\toprule
\textbf{Maxwell} & \textbf{ART} \\
\midrule
Lineare Wellengleichung & Geodätengleichung \\
Flache Metrik $\eta_{\mu\nu}$ & Dynamische Metrik $g_{\mu\nu}$ \\
Lorentz-Invarianz & Allgemeine Kovarianz \\
\bottomrule
\end{tabular}
\end{table}

\section{Weber-Kraft und Quantengravitation}

\subsection{Konzeptionelle Vorteile}
\begin{itemize}[leftmargin=*,noitemsep]
    \item Kein vordefiniertes Raummodell benötigt
    \item Natürliche Diskretisierung durch Punktteilchen
    \item Gravitative Erweiterung möglich:
    \[ \bm{F}_{\text{Weber}}^{G} = G\frac{mM}{r^2}\left(1 - \frac{\dot{r}^2}{c^2} + \frac{2r\ddot{r}}{c^2}\right)\bm{\hat{r}} \]
\end{itemize}

\begin{table}[h]
\centering
\caption{Quantisierungsprobleme und Alternativen}
\begin{tabularx}{\linewidth}{lX}
\toprule
\textbf{ART-Problem} & \textbf{Weber-Lösungsansatz} \\
\midrule
Nichtrenormierbarkeit & Keine Geometriequantisierung nötig \\
Singularitäten & Punktteilchen ohne Metrik \\
Zeitproblem & Explizite Zeitabhängigkeit in $\dot{r}$, $\ddot{r}$ \\
\bottomrule
\end{tabularx}
\end{table}

\subsection*{Zusammenfassung}
\begin{itemize}[leftmargin=*,noitemsep]
    \item Umgeht Quantisierungsprobleme der ART
    \item Ermöglicht diskrete Raumzeitmodelle
    \item Offene Fragen:
    \begin{itemize}[noitemsep]
        \item Verallgemeinerung auf nicht-abelsche Theorien
        \item Quantenfeldtheoretische Formulierung
        \item Experimentelle Tests
    \end{itemize}
    \item Potentieller Brückenansatz zur Quantengravitation
\end{itemize}

\section{Weber-Kraft und Gravitation}

\subsection*{Tisserands Ansatz}
\begin{itemize}[leftmargin=*,noitemsep]
    \item Übertragung der elektrodynamischen Weber-Kraft (Originalformel) auf Gravitation:
    \[ \bm{F} = G\frac{mM}{r^2}\left(1 - \frac{\dot{r}^2}{c^2} + \frac{2r\ddot{r}}{c^2}\right)\bm{\hat{r}} \]
    \item Scheiterte an der Erklärung der Periheldrehung des Merkur
\end{itemize}

\subsection*{Hinweis}
\begin{itemize}[leftmargin=*,noitemsep]
    \item Die korrekte gravitative Formulierung wird separat vorgestellt
    \item Erfordert Modifikation der Original-Weberschen Formel
\end{itemize}
