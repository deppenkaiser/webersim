\documentclass{book}
\usepackage[utf8]{inputenc}
\usepackage[german]{babel}
\usepackage{amsmath}
\usepackage{amssymb}
\usepackage{graphicx}
\usepackage{hyperref}
\usepackage{xcolor}
\usepackage{booktabs}
\usepackage{tabularx}
\usepackage{enumitem}
\usepackage{geometry}
\usepackage{float}
\usepackage{subfiles}
\usepackage{textcomp}
\usepackage{pifont}
\usepackage{listings}

\geometry{a4paper, margin=2cm}

\title{Weber-Kraft als fundamentale Theorie der Quantengravitation}
\author{}
\date{}

% Farbdefinitionen
\definecolor{resultbg}{RGB}{232, 244, 248}
\definecolor{derivbg}{RGB}{245, 245, 245}
\definecolor{defboxcolor}{RGB}{220,220,220}
\definecolor{examplecolor}{rgb}{0.95,0.95,0.85}
\definecolor{summarycolor}{RGB}{240,240,200}
\definecolor{warningbg}{RGB}{255,255,204}

% Box-Definitionen
\newcommand{\eqbox}[1]{
  \fcolorbox{blue}{white}{
    \parbox{\linewidth}{#1}
  }
}

\newcommand{\resultbox}[1]{
  \fcolorbox{blue}{resultbg}{
    \parbox{\linewidth}{#1}
  }
}

\newcommand{\derivbox}[1]{
  \fcolorbox{gray}{derivbg}{
    \parbox{\linewidth}{#1}
  }
}

\newcommand{\importantbox}[1]{
  \fcolorbox{red}{white}{
    \parbox{\linewidth}{\textbf{\textcolor{red}{#1}}}
  }
}

\newcommand{\defbox}[1]{%
  \noindent
  \fcolorbox{blue}{defboxcolor}{%
    \parbox{\dimexpr\linewidth-2\fboxsep-2\fboxrule\relax}{%
      \raggedright
      \setlength{\emergencystretch}{2em}%
      #1
    }%
  }%
}

\newcommand{\examplebox}[1]{%
  \noindent
  \fcolorbox{green}{examplecolor}{%
    \parbox{\dimexpr\linewidth-2\fboxsep-2\fboxrule\relax}{%
      \raggedright
      \setlength{\emergencystretch}{2em}%
      #1
    }%
  }%
}

\newcommand{\summarybox}[1]{%
  \noindent
  \fcolorbox{blue}{summarycolor}{%
    \parbox{\dimexpr\linewidth-2\fboxsep-2\fboxrule\relax}{%
      \raggedright
      \setlength{\emergencystretch}{2em}%
      #1
    }%
  }%
}

\newcommand{\warningbox}[1]{
  \fcolorbox{red}{warningbg}{
    \begin{minipage}{\dimexpr\linewidth-2\fboxsep-2\fboxrule\relax}
    #1
    \end{minipage}
  }
}

\begin{document}
\maketitle
\chapter{Einleitung}
\subfile{index/sections/00_zusammenfassung}
\subfile{index/sections/01_einfuehrung}
\subfile{index/sections/02_periheldrehung}
\subfile{index/sections/03_interpretation}
\subfile{index/sections/04_grenzen}
\subfile{index/sections/05_beta_formel}
\subfile{index/sections/06_universal_formel}
\subfile{index/sections/07_rotverschiebung}
\subfile{index/sections/08_gravitationswellen}
\subfile{index/sections/09_quantized_space}
\subfile{index/sections/10_knoten_theorie}
\subfile{index/sections/11_qed}
\subfile{index/sections/12_vorhersagekraft}
\subfile{index/sections/13_history}
\subfile{index/sections/14_roadmap}
\subfile{index/sections/15_vergleich}
\subfile{index/sections/16_literatur}
\chapter{Bahndynamik}
\subfile{vektorgleichungen/sections/01_vektordefinitionen}
\subfile{vektorgleichungen/sections/02_weber_kraft}
\subfile{vektorgleichungen/sections/03_loesungen}
\subfile{vektorgleichungen/sections/04_nkoerper}
\subfile{vektorgleichungen/sections/05_tensor}
\subfile{merkur_bahn/sections/01_grundgleichungen}
\subfile{merkur_bahn/sections/02_herleitung}
\subfile{merkur_bahn/sections/03_loesung}
\subfile{merkur_bahn/sections/04_anwendung}
\subfile{merkur_bahn/sections/05_zusammenfassung}
\chapter{Störungsrechnung}
\subfile{stoerungsrechnung/sections/01_einfuehrung}
\subfile{stoerungsrechnung/sections/02_systemdefinition}
\subfile{stoerungsrechnung/sections/03_stoerterme}
\subfile{stoerungsrechnung/sections/04_gesamtstoerungen}
\subfile{stoerungsrechnung/sections/05_beispiel}
\subfile{stoerungsrechnung/sections/06_zusammenfassung}
\chapter{Baryzentrisches Koordinatensystem}
\subfile{baryzentrisch/sections/01_transformation}
\subfile{baryzentrisch/sections/02_validierung}
\subfile{baryzentrisch/sections/03_beispiel}
\subfile{baryzentrisch/sections/04_implementierung}
\end{document}
