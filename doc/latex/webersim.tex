\documentclass{book}
\usepackage[utf8]{inputenc}
\usepackage[german]{babel}
\usepackage{amsmath}
\usepackage{amssymb}
\usepackage{graphicx}
\usepackage{hyperref}
\usepackage{xcolor}
\usepackage{booktabs}
\usepackage{tabularx}
\usepackage{enumitem}
\usepackage{geometry}
\usepackage{float}
\usepackage{subfiles}
\usepackage{textcomp}
\usepackage{pifont}
\usepackage{listings}
\usepackage{csquotes}

\geometry{a4paper, margin=2cm}
\setlength{\parindent}{0pt} % Keine Einrückung
\setlength{\parskip}{5pt}   % 6pt Abstand zwischen Absätzen

\title{Weber-Kraft als fundamentale Theorie der Quantengravitation}
\author{Michael Czybor}
\date{20. Juni 2025}

% Farbdefinitionen
\definecolor{resultbg}{RGB}{232, 244, 248}
\definecolor{derivbg}{RGB}{245, 245, 245}
\definecolor{defboxcolor}{RGB}{220,220,220}
\definecolor{examplecolor}{rgb}{0.95,0.95,0.85}
\definecolor{summarycolor}{RGB}{240,240,200}
\definecolor{warningbg}{RGB}{255,255,204}

% Box-Definitionen
\newcommand{\eqbox}[1]{
  \fcolorbox{blue}{white}{
    \parbox{\linewidth}{#1}
  }
}

\newcommand{\resultbox}[1]{
  \fcolorbox{blue}{resultbg}{
    \parbox{\linewidth}{#1}
  }
}

\newcommand{\derivbox}[1]{
  \fcolorbox{gray}{derivbg}{
    \parbox{\linewidth}{#1}
  }
}

\newcommand{\importantbox}[1]{
  \fcolorbox{red}{white}{
    \parbox{\linewidth}{\textbf{\textcolor{red}{#1}}}
  }
}

\newcommand{\defbox}[1]{%
  \noindent
  \fcolorbox{blue}{defboxcolor}{%
    \parbox{\dimexpr\linewidth-2\fboxsep-2\fboxrule\relax}{%
      \raggedright
      \setlength{\emergencystretch}{2em}%
      #1
    }%
  }%
}

\newcommand{\examplebox}[1]{%
  \noindent
  \fcolorbox{green}{examplecolor}{%
    \parbox{\dimexpr\linewidth-2\fboxsep-2\fboxrule\relax}{%
      \raggedright
      \setlength{\emergencystretch}{2em}%
      #1
    }%
  }%
}

\newcommand{\summarybox}[1]{%
  \noindent
  \fcolorbox{blue}{summarycolor}{%
    \parbox{\dimexpr\linewidth-2\fboxsep-2\fboxrule\relax}{%
      \raggedright
      \setlength{\emergencystretch}{2em}%
      #1
    }%
  }%
}

\newcommand{\warningbox}[1]{
  \fcolorbox{red}{warningbg}{
    \begin{minipage}{\dimexpr\linewidth-2\fboxsep-2\fboxrule\relax}
    #1
    \end{minipage}
  }
}

\begin{document}
\cleardoublepage
\pagenumbering{Roman}
\maketitle
\tableofcontents
\cleardoublepage
\setcounter{page}{1}
\pagenumbering{arabic}
\chapter{Einleitung}
\subfile{index/sections/00_zusammenfassung}
\subfile{index/sections/01_einfuehrung}
\subfile{index/sections/02_periheldrehung}
\subfile{index/sections/03_interpretation}
\subfile{index/sections/04_grenzen}
\subfile{index/sections/05_beta_formel}
\subfile{index/sections/06_universal_formel}
\subfile{index/sections/07_rotverschiebung}
\subfile{index/sections/08_gravitationswellen}
\subfile{index/sections/09_quantized_space}
\subfile{index/sections/10_knoten_theorie}
\subfile{index/sections/11_qed}
\subfile{index/sections/12_vorhersagekraft}
\subfile{index/sections/13_history}
\subfile{index/sections/14_roadmap}
\subfile{index/sections/15_vergleich}
\subfile{index/sections/16_literatur}
\chapter{Bahndynamik}
\subfile{vektorgleichungen/sections/01_vektordefinitionen}
\subfile{vektorgleichungen/sections/02_weber_kraft}
\subfile{vektorgleichungen/sections/03_loesungen}
\subfile{vektorgleichungen/sections/04_nkoerper}
\subfile{vektorgleichungen/sections/05_tensor}
\subfile{merkur_bahn/sections/01_grundgleichungen}
\subfile{merkur_bahn/sections/02_herleitung}
\subfile{merkur_bahn/sections/03_loesung}
\subfile{merkur_bahn/sections/04_anwendung}
\subfile{merkur_bahn/sections/05_zusammenfassung}
\chapter{Störungsrechnung}
\subfile{stoerungsrechnung/sections/01_einfuehrung}
\subfile{stoerungsrechnung/sections/02_systemdefinition}
\subfile{stoerungsrechnung/sections/03_stoerterme}
\subfile{stoerungsrechnung/sections/04_gesamtstoerungen}
\subfile{stoerungsrechnung/sections/05_beispiel}
\subfile{stoerungsrechnung/sections/06_zusammenfassung}
\chapter{Baryzentrisches Koordinatensystem}
\subfile{baryzentrisch/sections/01_transformation}
\subfile{baryzentrisch/sections/02_validierung}
\subfile{baryzentrisch/sections/03_beispiel}
\subfile{baryzentrisch/sections/04_implementierung}
\chapter{Simulation}
\section{Funktionsweise}
Die Simulation der Dynamik sämtlicher Objekte (Planeten, Sonne, etc.) wird mit Hilfe eines Zustandsautomaten
realisiert. Jeder Zyklus entspricht einem Zeitschritt $T$. $T$ ist im Moment konstant, könnte aber auch variabel sein.

Mit Hilfe der aus der Weber-Gravitationskraft hergeleiteten Gleichungen für $\vec{r}(\phi)$, $\vec{v}(\phi)$, $\vec{\omega}(\phi)$
lassen sich die Anfangsbedingungen (Startwerte der Simulation) bestimmen. Obwohl diese Gleichungen nur Näherungen sind,
erreichen sie dennoch die selbe Genauigkeit wie die ART (Periheldrehung des Planeten Merkur). Im Anschluss werden die Störungen
$\delta\vec{r}$, $\delta\vec{v}$, $\delta\vec{\omega}$ berechnet, welche sich aus den Positionen und Geschwindigkeiten ergeben;
diese werden den jeweiligen Startwerten hinzugefügt.

Die Weber-Gravitation besitzt einen beschleunigungsabhängigen Term, welcher wiederum von der Weber-Gravitationskraft abhängig ist.
Eine exakte analytische Berechnung ist daher nicht möglich; theoretisch ließen sich aber die Startwerte auch numerisch berechnen.

Eine weitere Ungenauigkeit der Simulation besteht durch die unvollständige Störungsberechnung. Die Störungen der Himmelskörper werden
von ungestörten Positionen aus berechnet. Richtig wäre es, gestörte Ausgangswerte zur Berechnung der Störung eines bestimmten
Himmelskörpers zu verwenden. Das Problem hierbei wäre eine Rekursion, die erst dann endet, wenn die Störungen gegen einen bestimmten
Wert konvergieren.

Aus praktischer Sicht wäre dieser Aufwand ungerechtfertigt, da die \enquote{Störung der Störung} im numerischen Rauschen untergeht.
Auch die numerischen Ungenauigkeiten spielen eine Rolle, wobei sie durch das verwendete Simulationsprinzip (Integration des Winkels)
stark reduziert werden.

Bis zu diesem Zeitpunkt wurden alle Berechnungen im heliozentrischen Koordinatensystem ducrhgeführt, wobei die Sonne die
Geschwindigkeit $\vec{v}_\text{Sonne} = \vec{0}$ hat und sich im Koordinatenursprung $\vec{r}_\text{Sonne} = \vec{0}$ befindet.
Für die Berechnung der Planeten wurden deren Monde (falls vorhanden), als zusätzliche Massen der jeweiligen Planetenmasse hinzugefügt.
Eine zukünftige Erweiterung der Simulation könnte auch die Monde simulieren.

Im nächsten Berechnungszustand werden alle Planeten ins baryzentrische Koordinatensystem verschoben. Hierdurch kann auch die
Position und Bewegung der Sonne bestimmt werden.

Durch die Verwendung des baryzentrischen Systems entsteht auch die Möglichkeit einer einfachen Überprüfung der Plausibilität der
kinetischen Berechnungen. Die Summe über alle Impulse muss Null ergeben. Das gleiche gilt auch für die Summe des Systemschwerpunkts.
Alle Himmelskörper inkl. der Sonne, müssen sich um einen virtuellen Nullpunkt bewegen; dem Baryzentrum des Sonnensystems. Sollten
Raumfahrzeuge mitsimuliert werden (Dynamik eines Raumschiffs im Sonnensystem), wären diese hier auszuschließen.

Als letzter Simulationszustand kann nun der zukünftige Winkel $\phi_\text{n+1} = \omega(\phi_n) * T$ für jeden Himmelskörper
berechnet werden. Die Integration findet also nicht über die Ortsvektoren und den Kräften statt, sondern über den Winkel; hierdurch
ensteht eine höhere numerische Stabilität. Je kleiner der Zeitschritt gewählt wird, desto präziser ist die Simulation.

Im nächsten Simulationsschritt wiederholt sich der ganze Vorgang. Es gibt somit keine Fehler in der Berechnung der Positionen
oder Geschwindigkeiten, da diese immer von einem klar definierten Winkel abhängig sind. Es kann lediglich ein Fehler in der
Integration des Winkels entstehen, da sich die Winkelgeschwindigkeit kontinuierlich ändert, aber für die Zeitdauer eines Schritts
eine konstante Winkelgeschwindigkeit angenommen wird.

\section{Ausblick}
Im Moment werden noch nicht alle Bahnelemete verwendet. So könnte z. B. die Bahnneigung genutzt werden, um die Lage der Himmelksörper
im Raum besser zu beschreiben. Hierdurch würde sich die Störungsrechnung verändern. Alle Himmelskörper werden zur Zeit durch einen
Zufallswinkel im Sonnensystem positioniert. Die genaue Berechnung des Startwinkels abhängig vom Datum und der Weltzeit (UTC), wäre ebenfalls
eine sinnvolle Erweiterung, welche die Simulation noch realistischer machen würde.

Die weiterführende Entwicklung der Weber-Gravitationstheorie, hin zu einer Quantengravitation, erfordert ebenfalls eine schrittweise
Berechnung (diskretes Raumgitter). Somit ist diese Simulation, bezüglich ihrer physikalischen Aussagekraft, erweiterbar und prinzipiell
auf dem richtigen Weg.

Eine besondere Eigenschaft der Weber-Kraft ist, dass sie drei Kräfte (Elektromagnetismus, Gravitation) einheitlich beschreiben - und
somit auch zusammenfassen kann. Sie ist deshalb für eine \enquote{Theorie von allem} predestiniert.

\chapter{Anhang}
\section{Quellcode und Dokumentation}
https://github.com/deppenkaiser/webersim
\end{document}
